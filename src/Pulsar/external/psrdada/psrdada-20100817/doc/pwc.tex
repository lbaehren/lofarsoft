\section{Primary Write Client Command Interface}
\label{sec:pwc}

This section describes the behaviour of the Write Client software that
will run on each of the Primary nodes, known as the Primary Write
Client (PWC) software.  In the case of PuMa-II, the PWC is 
{\tt puma2\_dmadb}.

\subsection{Operational States}

The PWC has four main states of operation:
\begin{itemize}
\item {\bf idle} waiting for configuration parameters
\item {\bf prepared} configuration parameters received; waiting for start
\item {\bf clocking} clocking data but not recording
\item {\bf recording} recording data
\end{itemize}

\subsubsection{Idle State}

In the idle state, the PWC sleeps until configuration parameters are
sent from the control software.  All configuration parameters are sent
in a single ASCII header, and at any time before entering the {\bf
recording}, the duration of the recording may be specified
\begin{itemize} 
\item {\tt\bf HEADER text} copy text to the Header Block and enter 
the {\bf prepared} state.
\item {\tt\bf DURATION time} record for the time specified as
{\it HH:MM:SS} or number of samples.
\end{itemize}

\subsubsection{Prepared State}

In the prepared state, the PWC sleeps until a start command is sent
from the control software.  There are two different start commands
that can be received in this state:
\begin{itemize}
\item {\tt\bf CLOCK} enter the {\bf clocking} state
\item {\tt\bf START [YYYY-MM-DD-hh:mm:ss]} enter the {\bf recording} state
\end{itemize}
If the optional UTC time argument is specified, the PWC will enter the
{\bf recording} state at the specified time.  Otherwise, the PWC will
enter the requested state at the next available opportunity (for
PuMa-II, on the next {\tt SYSTICK}).

\subsubsection{Clocking State}

In this state, the PWC software clocks data into the Data Block but
does not flag the data as valid.  The PWC will overwrite the data in
each sub-block, and will remain in this state until one of the
following commands is received:
\begin{itemize}
\item {\tt\bf STOP} enter the {\bf idle} state immediately
\item {\tt\bf REC\_START YYYY-MM-DD-hh:mm:ss} raise the valid data flag
	at the specified UTC time in the data stream and enter the {\bf
	recording} state
\end{itemize}
Note that the UTC time specified in the first argument to {\tt
REC\_START} may be any time in the future.  If it is in the past, then
the difference between the specified UTC and the present cannot be
greater than the amount of time corresponding to the length of the
Data Block.

\subsubsection{Recording State}

In this state, the PWC software clocks data into the Data Block, flags
the data as valid, and will not overwrite a sub-block until it has
been flagged as cleared.  The PWC will remain in this state until one
of the following commands is received:
\begin{itemize}
\item {\tt\bf STOP [YYYY-MM-DD-hh:mm:ss]} enter the {\bf idle} state
\item {\tt\bf REC\_STOP YYYY-MM-DD-hh:mm:ss} raise the end of data flag
	at the specified UTC time in the data stream and enter the {\bf
	clocking} state
\end{itemize}

Note that with the exception of {\tt REC\_START} all UTC times {\bf
must} be in the future.  If this is not true, the command will take
effect immediately.  If {\tt DURATION} has been specified, it will not
be corrected to reflect the difference between requested and actual
start times.

The command interface described in this section is implemented by the
{\tt dada\_pwc} software, as documented in Appendix~\ref{app:dada_pwc}.

\section{Primary Write Client Main Loop}
\label{sec:pwc_main}

In addition to providing the command interface described in the
previous section, the Primary Write Client must implement the transfer
of data to the Data Block.  This section gives an outline of how this
will be done.

\subsection{Top Down Description}

The following describes the behaviour of the Primary Write Client

\begin{itemize}
\item Initialization
\vspace{-3mm}
	\begin{itemize}
	\item read configuration file
	\item parse command line options
	\item initialize DMA and PiC cards
	\item connect to Data and Header Blocks
	\item open a port and listen for command connection
	\end{itemize}
\item Main Loop
\vspace{-3mm}
	\begin{itemize}
	\item Idle State
	\vspace{-2mm}
		\begin{itemize}
		\item wait for configuration
		\item set configuration
		\item enter {\bf prepared} state
		\end{itemize}
	\item Prepared State
	\vspace{-2mm}
		\begin{itemize}
		\item wait for a command
		\item if command={\tt CLOCK}, enter {\bf clocking} state
		\item if command={\tt START}, enter {\bf recording} state
		\item if command={\tt STOP}, return to {\bf idle} state
		\end{itemize}
	\item Clocking State (loop)
	\vspace{-2mm}
		\begin{itemize}
		\item check for a command
		\item if command={\tt STOP}, return to {\bf idle} state
		\item if command={\tt REC\_START}, enter {\bf recording} state
		\item copy buffer from DMA to Data Block
		\end{itemize}
	\item Recording State (loop)
	\vspace{-2mm}
		\begin{itemize}
		\item check for a command
		\item if command={\tt STOP}, flag end of data (EOD) and return to {\bf idle} state
		\item if command={\tt REC\_STOP}, flag EOD and enter {\bf clocking} state
		\item wait for next free Data Block sub-block
		\item copy buffer from DMA to Data Block and flag as valid
		\end{itemize}
	\end{itemize}
\item Shutdown
\vspace{-3mm}
	\begin{itemize}
	\item close command connection
	\item disconnect from Data and Header Blocks
	\end{itemize}
\end{itemize}

\begin{figure}
\centerline{\psfig{figure=pwc_state.eps,height=3in,angle=0}}
\caption [\sffamily DADA Control Flow]
{
Schematic overview of DADA control flow.  Note that the state 
can be toggled between clocking and recording.
}
\label{fig:pwc_state}
\end{figure}

The behaviour described in this section is implemented by the {\tt
dada\_pwc\_main} software, as documented in
Appendix~\ref{app:dada_pwc_main}.
