\documentclass[12pt]{article}

\begin{document}
\title{User manual for smosaic}
\author{Alexander Usov}
\maketitle

\section{Intro}

Smosaic is a simple and straightforward mosaicking tool developed 
for LOFAR project. The primary reasons to develop new tool were the
lack of support of multi-dimensional data cubes in the most packages
targetted for the optical data and  lack of the direct support of 
the HDF5 data cubes produced by the LOFAR pipeline.

\section{Current status}

In it's current state smosaic does not support all the intended 
functionality completely. It's capable to perform basic mosaicking 
of multi dimensional cubes, however non-spatial dimensions of the
input and output cubes should be compatible. File format support
currently implemented only for FITS files, however is should be trivial
to update it for HDF5 cubes once their support is implemented in 
casacore package.

\section{Usage}

\subsection{Building smosaic}

In order to compile smosaic one need to have working installation of the
recent casacore package. CMake autoconfiguration is not implemented yet,
so manual editing of the Makefile is needed to put in there correct path 
for casacore installation. Once it's done, simply run make to compile it.
As a result you should get statically linked executable \verb/smosaic/.

\subsection{Running mosaicking}

In order to make a mosaic a number of preparational steps should be 
performed. First put all input files into separate directory, as smosaic 
uses directories to group input files.

Second step is to prepare target header template. In the current 
implementation template files use slightly simplified FITS header 
syntax. To simplify creation of template files there is no need to 
pad all lines to 80 characters, and newline symbols can be used freely. 
An easy way to obtain valid template file is to run a command like 
following in the terminal: 
``\verb/fold file.fits | head -n 100 > img.hdr/''
and cut the resulting header file (\verb/img.hdr/) right after 
FITS END statement. Then you can edit projection related parameters 
to suit your needs.

The general syntax of the \verb/smosaic/ tool is the following: \\
``\verb/smosaic <parameter>=value <parameter>=value ..../'', where the list of 
possible parameters and their values is given below. An up-to-date list of parameters
and their default values can be obtained by running \verb/smosaic/ with option
 ``\verb/-h/''.

Currently implemented are folowing parameters:
\begin{itemize}
\item[\bf indir] : Path to the directory with input images. 
No reasonable default value can be provided.

\item[\bf template] : Path to the template file. 
No reasonable default value can be provided.

\item[\bf out] : Filename (and path of course) for the FITS file containing 
resulting mosaic. Defaults to ``\verb/out.fits/''.

\item[\bf interp] : Pixel interpolation method. Possible values are 0, 1 or 
2 corresponding to nearest pixel, linear and cubic interpolaiton methods 
respectively. To my knowledge this changes interpolation scheme used to 
calculate pixel values after reprojection of the image. Our tests show that 
cubic interpolation is most accurate even for reasonably noisy images. 
Defaults to cubic interpolation.

\item[\bf autoBB] : Whether \verb/smosaic/ should automatically adjust size
of the target mosaic to include all input images. Possible values are 
True (for automatical image size adjustement) and False. This switch is 
implemented to simplify the usage of \verb/smosaic/. When \verb/autoBB/ 
option switched on, \verb/smosaic/ will automatically adjust size and 
reference point of the target mosaic in such a way, that all input images
are fully present on the mosaic. If you want to manually define size of the
target mosaic to get just some specific area of it you should switch it off.
Defaults to True.
\end{itemize}

\subsection{Test case}

A simple test case is provided so that you can check whether your 
\verb/smosaic/ and \verb/casacore/ are working properly. To check it simply
run \\
``\verb/smosaic indir=test_input header=test_template.hdr/''. This
should produce file \verb/out.fits/ in the current directory. Compare it with
\verb/test_result.fits/.

\emph{ It's a bit strange, but my test\_result.fits still has problems with rugged
image bounday. I believe it should have been fixed a time ago in casacore 
regridder. Might be that my local version of casacore is outdate, need to check it
later.}

\end{document}
