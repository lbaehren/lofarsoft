% Complete documentation on the extended LaTeX markup used for Python
% documentation is available in ``Documenting Python'', which is part
% of the standard documentation for Python.  It may be found online
% at:
%
%     http://www.python.org/doc/current/doc/doc.html

\documentclass[hyperref]{manual}
\pagestyle{plain}

% latex2html doesn't know [T1]{fontenc}, so we cannot use that:(

\usepackage{amsmath}
\usepackage[latin1]{inputenc}
\usepackage{textcomp}


% The commands of this document do not reset module names at section level
% (nor at chapter level).
% --> You have to do that manually when a new module starts!
%     (use \py@reset)
%begin{latexonly}
\makeatletter
\renewcommand{\section}{\@startsection{section}{1}{\z@}%
   {-3.5ex \@plus -1ex \@minus -.2ex}%
   {2.3ex \@plus.2ex}%
   {\reset@font\Large\py@HeaderFamily}}
\makeatother
%end{latexonly}


% additional mathematical functions
\DeclareMathOperator{\abs}{abs}

% provide a cross-linking command for the index
%begin{latexonly}
\newcommand*\see[2]{\protect\seename #1}
\newcommand*{\seename}{$\to$}
%end{latexonly}


% some convenience declarations
\newcommand{\numarray}{numarray}
\newcommand{\Numarray}{Numarray}  % Only beginning of sentence, otherwise use \numarray
\newcommand{\NUMARRAY}{NumArray}
\newcommand{\numpy}{Numeric}
\newcommand{\NUMPY}{Numerical Python}
\newcommand{\python}{Python}


% mark internal comments
% for any published version switch to the second (empty) definition of the macro!
% \newcommand{\remark}[1]{(\textbf{Note to authors: #1})}
\newcommand{\remark}[1]{}


\title{numarray\\User's Manual}

\author{Perry Greenfield \\
   Todd Miller \\
   Rick White \\
   J.C. Hsu \\
   Paul Barrett \\
   Jochen K�pper \\
   Peter J. Verveer \\[1ex]
   Previously authored by: \\
   David Ascher \\
   Paul F. Dubois \\
   Konrad Hinsen \\
   Jim Hugunin \\
   Travis Oliphant \\[1ex]
   with contributions from the Numerical Python community}

\authoraddress{Space Telescope Science Institute, 3700 San Martin Dr,
   Baltimore, MD 21218 \\ UCRL-MA-128569}

% I use date to indicate the manual-updates,
% release below gives the matching software version.
\date{November 2, 2005}        % update before release!
                                % Use an explicit date so that reformatting
                                % doesn't cause a new date to be used.  Setting
                                % the date to \today can be used during draft
                                % stages to make it easier to handle versions.

\release{1.5}                 % (software) release version;
\setshortversion{1.5}         % this is used to define the \version macro

\makeindex                      % tell \index to actually write the .idx file



\begin{document}

\maketitle

% This makes the contents more accessible from the front page of the HTML.
\ifhtml
\part*{General}
\chapter*{Front Matter}
\label{front}
\fi

\section*{Legal Notice}
\label{sec:legal-notice}

Please see file LICENSE.txt in the source distribution.  

This open source project has been contributed to by many people, including
personnel of the Lawrence Livermore National Laboratory, Livermore, CA, USA.
The following notice covers those contributions, including contributions to
this this manual.

Copyright (c) 1999, 2000, 2001.  The Regents of the University of California.
All rights reserved.

Permission to use, copy, modify, and distribute this software for any purpose
without fee is hereby granted, provided that this entire notice is included in
all copies of any software which is or includes a copy or modification of this
software and in all copies of the supporting documentation for such software.

This work was produced at the University of California, Lawrence Livermore
National Laboratory under contract no. W-7405-ENG-48 between the U.S.
Department of Energy and The Regents of the University of California for the
operation of UC LLNL.



\subsection*{Special license for package numarray.ma}
\label{sec:license-numarray.ma}


The package \module{numarray.ma} was written by Paul Dubois, Lawrence Livermore
National Laboratory, Livermore, CA, USA.

Copyright (c) 1999, 2000. The Regents of the University of California. All
rights reserved.

Permission to use, copy, modify, and distribute this software for any purpose
without fee is hereby granted, provided that this entire notice is included in
all copies of any software which is or includes a copy or modification of this
software and in all copies of the supporting documentation for such software.

This work was produced at the University of California, Lawrence Livermore
National Laboratory under contract no. W-7405-ENG-48 between the U.S.
Department of Energy and The Regents of the University of California for the
operation of UC LLNL.



\subsection*{Disclaimer}

This software was prepared as an account of work sponsored by an agency of the
United States Government. Neither the United States Government nor the
University of California nor any of their employees, makes any warranty,
express or implied, or assumes any liability or responsibility for the
accuracy, completeness, or usefulness of any information, apparatus, product,
or process disclosed, or represents that its use would not infringe
privately-owned rights. Reference herein to any specific commercial products,
process, or service by trade name, trademark, manufacturer, or otherwise, does
not necessarily constitute or imply its endorsement, recommendation, or
favoring by the United States Government or the University of California. The
views and opinions of authors expressed herein do not necessarily state or
reflect those of the United States Government or the University of California,
and shall not be used for advertising or product endorsement purposes.




%% Local Variables:
%% mode: LaTeX
%% mode: auto-fill
%% fill-column: 79
%% indent-tabs-mode: nil
%% ispell-dictionary: "american"
%% reftex-fref-is-default: nil
%% TeX-auto-save: t
%% TeX-command-default: "pdfeLaTeX"
%% TeX-master: "numarray"
%% TeX-parse-self: t
%% End:
  \cleardoublepage


\tableofcontents


\part{Numerical Python}

\NUMARRAY{} (``\numarray{}'') adds a fast multidimensional array facility to
Python.  This part contains all you need to know about ``\numarray{}'' arrays
and the functions that operate upon them.

\label{part:numerical-python}

\declaremodule{extension}{numarray}
\moduleauthor{The numarray team}{numpy-discussion@lists.sourceforge.net}
\modulesynopsis{Numerics}

\chapter{Introduction}
\label{cha:introduction}

\begin{quote}
   This chapter introduces the numarray Python extension and outlines the rest
   of the document.
\end{quote}

Numarray is a set of extensions to the Python programming language which allows
Python programmers to efficiently manipulate large sets of objects organized in
grid-like fashion. These sets of objects are called arrays, and they can have
any number of dimensions. One-dimensional arrays are similar to standard Python
sequences, and two-dimensional arrays are similar to matrices from linear
algebra. Note that one-dimensional arrays are also different from any other
Python sequence, and that two-dimensional matrices are also different from the
matrices of linear algebra. One significant difference is that numarray objects
must contain elements of homogeneous type, while standard Python sequences can
contain elements of mixed type. Two-dimensional arrays differ from matrices
primarily in the way multiplication is performed; 2-D arrays are multiplied
element-by-element.

This is a reimplementation of the earlier Numeric module (aka numpy). For the
most part, the syntax of numarray is identical to that of Numeric, although
there are significant differences. The differences are primarily in new
features. For Python 2.2 and later, the syntax is completely backwards
compatible. See the High-Level Overview (chapter \ref{cha:high-level-overview})
for incompatibilities for earlier versions of Python. The reasons for rewriting
Numeric and a comparison between Numeric and numarray are also described in
chapter \ref{cha:high-level-overview}. Portions of the present document are
almost word-for-word identical to the Numeric manual. It has been updated to
reflect the syntax and behavior of numarray, and there is a new section
(~\ref{sec:diff-numarray-numpy}) on differences between Numeric and numarray.

Why are these extensions needed? The core reason is a very prosaic one:
manipulating a set of a million numbers in Python with the
standard data structures such as lists, tuples or classes is much too slow and
uses too much space. A more subtle
reason for these extensions, however, is that the kinds of operations that
programmers typically want to do on arrays, while sometimes very complex, can
often be decomposed into a set of fairly standard operations. This
decomposition has been similarly developed in many array languages. In some
ways, numarray is simply the application of this experience to the Python
language.  Thus many of the operations described in numarray work the way they
do because experience has shown that way to be a good one, in a variety of
contexts. The languages which were used to guide the development of numarray
include the infamous APL family of languages, Basis, MATLAB, FORTRAN, S and S+,
and others.  This heritage will be obvious to users of numarray who already
have experience with these other languages.  This manual, however, does not
assume any such background, and all that is expected of the reader is a
reasonable working knowledge of the standard Python language.

This document is the ``official'' documentation for numarray. It is both a
tutorial and the most authoritative source of information about numarray with
the exception of the source code. The tutorial material will walk you through a
set of manipulations of simple, small arrays of numbers. This choice was made
because:
\begin{itemize}
\item A concrete data set makes explaining the behavior of some functions much
   easier to motivate than simply talking about abstract operations on abstract
   data sets.
\item Every reader will have at least an intuition as to the meaning of the
   data and organization of image files.  \remark{These ``image files'' are not
      mentioned anywhere before, and not really used later...?}
\end{itemize}
All users of numarray, whether interested in image processing or not, are
encouraged to follow the tutorial with a working numarray installed,
testing the examples, and more importantly, transferring the
understanding gained by working on arrays to their specific domain. The best
way to learn is by doing --- the aim of this tutorial is to guide you along
this "doing."

This manual contains:
\begin{description}
\item[Installing numarray] Chapter \ref{cha:installation} provides information
   on testing Python, numarray, and compiling and installing numarray if
   necessary.
\item[High-Level Overview] Chapter \ref{cha:high-level-overview} gives a
   high-level overview of the components of the numarray system as a whole.
\item[Array Basics] Chapter \ref{cha:array-basics} provides a detailed
   step-by-step introduction to the most important aspect of numarray, the
   multidimensional array objects.
\item[Ufuncs] Chapter \ref{cha:ufuncs} provides information on universal
   functions, the mathematical functions which operate on arrays and other
   sequences elementwise.
\item[Pseudo Indices] Chapter \ref{cha:pseudo-indices} covers syntax for some
   special indexing operators.
\item[Array Functions] Chapter \ref{cha:array-functions} is a catalog of each
   of the utility functions which allow easy algorithmic processing of arrays.
\item[Array Methods] Chapter \ref{cha:array-methods} discusses the methods of
   array objects.
\item[Array Attributes] Chapter \ref{cha:array-attributes} presents the
   attributes of array objects.
\item[Character Array] Chapter \ref{cha:character-array} describes the
  \code{numarray.strings} module that provides support for arrays of fixed
  length strings.
\item[Record Array] Chapter \ref{cha:record-array} describes the
   \code{numarray.records} module that supports arrays of fixed length records
   of string or numerical data.
\item[Object Array] Chapter \ref{cha:object-array} describes the
   \code{numarray.objects} module that supports arrays of Python objects.
\item[C extension API] Chapter \ref{cha:C-API} describes the C-APIs provided
   for \module{numarray} based extension modules.
\item[Convolution] Chapter \ref{cha:convolve} describes the
   \module{numarray.convolve} module for computing one-D and two-D convolutions
   and correlations of \class{numarray} objects.
\item[Fast-Fourier-Transform] Chapter \ref{cha:fft} describes the
   \module{numarray.fft} module for computing Fast-Fourier-Transforms
   (FFT) and Inverse FFTs over \class{numarray} objects in one- or
   two-dimensional manner.  Ported from Numeric.
\item[Linear Algebra] Chapter \ref{cha:linear-algebra} describes the
   \module{numarray.linear_algebra} module which provides a simple
   interface to some commonly used linear algebra routines; 
   \program{LAPACK}.   Ported from Numeric.
\item[Masked Arrays] Chapter \ref{cha:masked-arrays} describes the
   \module{numarray.ma} module which supports Masked Arrays: arrays which
   potentially have missing or invalid elements.  Ported from Numeric.
\item[Random Numbers] Chapter \ref{cha:random-array} describes the
   \module{numarray.random_array} module which supports generation of arrays of
   random numbers.  Ported from Numeric.
\item[Multidimentional image analysis functions] Chapter \ref{cha:ndimage}
   describes the \module{numarray.ndimage} module which provides
   functions for multidimensional image analysis such as filtering,
   morphology or interpolation.
\item[Glossary] Appendix \ref{cha:glossary} gives a glossary of terms.
\end{description}


\section{Where to get information and code}

Numarray and its documentation are available at SourceForge
(\ulink{sourceforge.net}{http://sourceforge.net}; SourceForge addresses can
also be abbreviated as \ulink{sf.net}{http://sf.net}). The main web site is:
\url{http://numpy.sourceforge.net}. Downloads, bug reports, a patch facility,
and releases are at the main project page, reachable from the above site or
directly at: \url{http://sourceforge.net/projects/numpy} (see Numarray under
"Latest File Releases").  The Python web site is \url{http://www.python.org}.
For up-to-date status on compatible modules available for numarray, please 
check \url{http://www.stsci.edu/resources/software_hardware/numarray/}.

NOTE: because numarray shares the numpy Source Forge project with Numeric and
Numeric3, there are dedicated Source Forge ``Trackers'' for numarray, .e.g.
``Numarray Bugs'' rather than just ``Bugs''.  When submitting bug reports,
patches, or requests, please look for the numarray version of the tracker under
the top level menu item ``Tracker'', nominally here:
\url{http://sourceforge.net/tracker/?group_id=1369}.

\section{Acknowledgments}

Numerical Python was the outgrowth of a long collaborative design process
carried out by the Matrix SIG of the Python Software Activity (PSA). Jim
Hugunin, while a graduate student at MIT, wrote most of the code and initial
documentation. When Jim joined CNRI and began working on JPython, he didn't
have the time to maintain Numerical Python so Paul Dubois at LLNL agreed to
become the maintainer of Numerical Python. David Ascher, working as a
consultant to LLNL, wrote most of the Numerical Python version of this
document, incorporating contributions from Konrad Hinsen and Travis Oliphant,
both of whom are major contributors to Numerical Python.  The reimplementation
of Numeric as numarray was done primarily by Perry Greenfield, Todd Miller, and
Rick White, with some assistance from J.C. Hsu and Paul Barrett. Although
numarray is almost a completely new implementation, it owes a great deal to the
ideas, interface and behavior expressed in the Numeric implementation. It is
not an overstatement to say that the existence of Numeric made the
implementation of numarray far, far easier that it would otherwise have been.
Since the source for the original Numeric module was moved to SourceForge, the
numarray user community has become a significant part of the process.
Numeric/numarray illustrates the power of the open source software concept.
Please send comments and corrections to this manual to
\ulink{perry@stsci.edu}{mailto:perry@stsci.edu}, or to Perry Greenfield, 3700
San Martin Dr, Baltimore, MD 21218, U.S.A.




%% Local Variables:
%% mode: LaTeX
%% mode: auto-fill
%% fill-column: 79
%% indent-tabs-mode: nil
%% ispell-dictionary: "american"
%% reftex-fref-is-default: nil
%% TeX-auto-save: t
%% TeX-command-default: "pdfeLaTeX"
%% TeX-master: "numarray"
%% TeX-parse-self: t
%% End:

\chapter{Installing numarray}
\label{cha:installation}

\begin{quote}
   This chapter explains how to install and test numarray, from either the
   source distribution or from the binary distribution.
\end{quote}

Before we start with the actual tutorial, we will describe the steps needed for
you to be able to follow along the examples step by step. These steps include
installing and testing Python, the numarray extensions, and some tools and
sample files used in the examples of this tutorial.


\section{Testing the Python installation}

The first step is to install Python if you haven't already. Python is available
from the Python project page at \url{http://sourceforge.net/projects/python}.
Click on the link corresponding to your platform, and follow the instructions
described there. Unlike earlier versions of numarray, version 0.7 and later
require Python version 2.2.2 at a minimum.  When installed, starting Python by
typing python at the shell or double-clicking on the Python interpreter should
give a prompt such as:
\begin{verbatim}
Python 2.3 (#2, Aug 22 2003, 13:47:10) [C] on sunos5
Type "help", "copyright", "credits" or "license" for more information.
\end{verbatim}
If you have problems getting this part to work, consider contacting a local
support person or emailing
\ulink{python-help@python.org}{mailto:python-help@python.org} for help. If
neither solution works, consider posting on the
\ulink{comp.lang.python}{news:comp.lang.python} newsgroup (details on the
newsgroup/mailing list are available at
\url{http://www.python.org/psa/MailingLists.html\#clp}).


\section{Testing the Numarray Python Extension Installation}

The standard Python distribution does not come, as of this writing, with the
numarray Python extensions installed, but your system administrator may have
installed them already. To find out if your Python interpreter has numarray
installed, type \samp{import numarray} at the Python prompt. You'll see one of
two behaviors (throughout this document user input and python interpreter
output will be emphasized as shown in the block below):
\begin{verbatim}
>>> import numarray
Traceback (innermost last):
File "<stdin>", line 1, in ?
ImportError: No module named numarray
\end{verbatim}
indicating that you don't have numarray installed, or:
\begin{verbatim}
>>> import numarray
>>> numarray.__version__
'0.6'
\end{verbatim}
indicating that you do. If you do, you can skip the next section
and go ahead to section \ref{sec:at-sourceforge}.  If you don't, you have to
get and install the numarray extensions as described in section
\ref{sec:installing-numarray}.

\section{Installing numarray}
\label{sec:installing-numarray}

The release facility at SourceForge is accessed through the project page,
\url{http://sourceforge.net/projects/numpy}.  Click on the "Numarray" release
and you will be presented with a list of the available files. The files whose
names end in ".tar.gz" are source code releases. The other files are binaries
for a given platform (if any are available).

It is possible to get the latest sources directly from our CVS repository using
the facilities described at SourceForge. Note that while every effort is made
to ensure that the repository is always ``good'', direct use of the repository
is subject to more errors than using a standard release.


\subsection{Installing on Unix, Linux, and Mac OSX}
\label{sec:installing-unix}

The source distribution should be uncompressed and unpacked as follows (for
example):
\begin{verbatim}
gunzip numarray-0.6.tar.gz
tar xf numarray-0.6.tar
\end{verbatim}
Follow the instructions in the top-level directory for compilation and
installation. Note that there are options you must consider before beginning.
Installation is usually as simple as:
\begin{verbatim}
python setup.py install
\end{verbatim}
or:
\begin{verbatim}
python setupall.py install
\end{verbatim}
if you want to install all additional packages, which include
\module{\mbox{numarray.convolve}}, \module{\mbox{numarray.fft}},
\module{\mbox{numarray.linear_algebra}}, and
\module{\mbox{numarray.random_array}}.

See numarray-X.XX/Doc/INSTALL.txt for the latest details (X.XX is the version 
number).

\paragraph*{Important Tip} \label{sec:tip:from-numarray-import} Just like all
Python modules and packages, the numarray module can be invoked using either
the \samp{import numarray} form, or the \samp{from numarray import ...} form.
Because most of the functions we'll talk about are in the numarray module, in
this document, all of the code samples will assume that they have been preceded
by a statement:
\begin{verbatim}
>>> from numarray import *
\end{verbatim}
Note the lowercase name in \module{\numarray} as opposed to \module{\numpy}.


\subsection{Installing on Windows}
\label{sec:installing-windows}

To install numarray, you need to be in an account with Administrator
privileges.  As a general rule, always remove (or hide) any old version of
numarray before installing the next version.

We have tested Numarrray on several Win-32 platforms including:
   
\begin{itemize}
\item Windows-XP-Pro-x86 ( MSVC-6.0) 
\item Windows-NT-x86 (MSVC-6.0) 
\item Windows-98-x86 (MSVC-6.0)
\end{itemize}

\subsubsection{Installation from source}

\begin{enumerate}
\item Unpack the distribution: (NOTE: You may have to download an "unzipping"
   utility)
\begin{verbatim}
C:\> unzip numarray.zip 
C:\> cd numarray
\end{verbatim}
\item Build it using the distutils defaults:
\begin{verbatim}
C:\numarray> python setup.py install
\end{verbatim}
   This installs numarray in \texttt{C:\textbackslash{}pythonXX} where XX is the version
   number of your python installation, e.g. 20, 21, etc.
\end{enumerate}


\subsubsection{Installation from self-installing executable}

\begin{enumerate}
\item Click on the executable's icon to run the installer.
\item Click "next" several times.  I have not experimented with customizing the
   installation directory and don't recommend changing any of the installation
   defaults.  If you do and have problems, let us know.
\item Assuming everything else goes smoothly, click "finish".
\end{enumerate}


\subsubsection{Testing your Installation}

Once you have installed numarray, test it with:
\begin{verbatim}
C:\numarray> python
Python 2.2.2 (#18, Dec 30 2002, 02:26:03) [MSC 32 bit (Intel)] on win32
Type "copyright", "credits" or "license" for more information.
>>> import numarray.testall as testall
>>> testall.test()
numeric:  (0, 1115)
records:  (0, 48)
strings:  (0, 166)
objects:  (0, 72)
memmap:  (0, 75)
\end{verbatim}
Each line in the above output indicates that 0 of X tests failed.  X grows
steadily with each release, so the numbers shown above may not be current.


\subsubsection{Installation on Cygwin}

For an installation of numarray for python running on Cygwin, see section
\ref{sec:installing-unix}.



\section{At the SourceForge...}
\label{sec:at-sourceforge}

The SourceForge project page for numarray is at
\url{http://sourceforge.net/projects/numpy}. On this project page you will find
links to:
\begin{description}
\item[The Numpy Discussion List] You can subscribe to a discussion list about
   numarray using the project page at SourceForge. The list is a good place to
   ask questions and get help. Send mail to
   numpy-discussion@lists.sourceforge.net.  Note that there is no
   numarray-discussion group, we share the list created by the numeric community.

\item[The Web Site] Click on "home page" to get to the Numarray Home Page,
   which has links to documentation and other resources, including tools for
   connecting numarray to Fortran.
\item[Bugs and Patches] Bug tracking and patch-management facilities is
   provided on the SourceForge project page.
\item[CVS Repository] You can get the latest and greatest (albeit less tested
   and trustworthy) version of numarray directly from our CVS repository.
\item[FTP Site] The FTP Site contains this documentation in several formats,
   plus maybe some other goodies we have lying around.
\end{description}



%% Local Variables:
%% mode: LaTeX
%% mode: auto-fill
%% fill-column: 79
%% indent-tabs-mode: nil
%% ispell-dictionary: "american"
%% reftex-fref-is-default: nil
%% TeX-auto-save: t
%% TeX-command-default: "pdfeLaTeX"
%% TeX-master: "numarray"
%% TeX-parse-self: t
%% End:

\chapter{High-Level Overview}
\label{cha:high-level-overview}

\begin{quote} 
   In this chapter, a high-level overview of the extensions is provided, giving
   the reader the definitions of the key components of the system. This section
   defines the concepts used by the remaining sections.
\end{quote}

Numarray makes available a set of universal functions (technically ufunc
objects), used in the same way they were used in Numeric. These are discussed
in some detail in chapter \ref{cha:ufuncs}.


\section{Numarray Objects}
\label{sec:numarray-objects}

The array objects are generally homogeneous collections of potentially large
numbers of numbers. All numbers in a numarray are the same kind (i.e. number
representation, such as double-precision floating point). Array objects must be
full (no empty cells are allowed), and their size is immutable. The specific
numbers within them can change throughout the life of the array, however.
There is a "mask array" package ("MA") for Numeric, which has been ported
to numarray as ``numarray.ma''.

Mathematical operations on arrays return new arrays containing the results of
these operations performed element-wise on the arguments of the operation.

The size of an array is the total number of elements therein (it can be 0 or
more). It does not change throughout the life of the array, unless the array
is explicitly resized using the resize function.

The shape of an array is the number of dimensions of the array and its extent
in each of these dimensions (it can be 0, 1 or more). It can change throughout
the life of the array. In Python terms, the shape of an array is a tuple of
integers, one integer for each dimension that represents the extent in that
dimension.  The rank of an array is the number of dimensions along which it is
defined. It can change throughout the life of the array. Thus, the rank is the
length of the shape (except for rank 0). \note{This is not the same meaning of
rank as in linear algebra.}

Use more familiar mathematicial examples: A vector is a rank-1 array
(it has only one dimension along which it can be indexed). A matrix as used in
linear algebra is a rank-2 array (it has two dimensions along which it can be
indexed). It is possible to create a rank-0 array which is just a scalar of 
one single value --- it has no dimension along which it can be indexed.

The type of an array is a description of the kind of element it contains. It
determines the itemsize of the array.  In contrast to Numeric, an array type in
numarray is an instance of a NumericType class, rather than a single character
code. However, it has been implemented in such a way that one may use aliases,
such as `\constant{u1}', `\constant{i1}', `\constant{i2}', `\constant{i4}',
`\constant{f4}', `\constant{f8}', etc., as well as the original character
codes, to set array types.  The itemsize of an array is the number of 8-bit
bytes used to store a single element in the array. The total memory used by an
array tends to be its size times its itemsize, when the size is large (there
is a fixed overhead per array, as well as a fixed overhead per dimension).

Here is an example of Python code using the array objects:
\begin{verbatim}
>>> vector1 = array([1,2,3,4,5])
>>> print vector1
[1 2 3 4 5]
>>> matrix1 = array([[0,1],[1,3]])
>>> print matrix1
[[0 1]
 [1 3]]
>>> print vector1.shape, matrix1.shape
(5,) (2,2)
>>> print vector1 + vector1
[ 2  4  6  8  10]
>>> print matrix1 * matrix1
[[0 1]                                  # note that this is not the matrix
 [1 9]]                                 # multiplication of linear algebra
\end{verbatim}
If this example complains of an unknown name "array", you forgot to begin
your session with
\begin{verbatim}
>>> from numarray import *
\end{verbatim}
See section \ref{sec:tip:from-numarray-import}.


\section{Universal Functions}
\label{sec:universal-functions}

Universal functions (ufuncs) are functions which operate on arrays and other
sequences. Most ufuncs perform mathematical operations on their arguments, also
elementwise.

Here is an example of Python code using the ufunc objects:
\begin{verbatim}
>>> print sin([pi/2., pi/4., pi/6.])
[ 1. 0.70710678 0.5       ]
>>> print greater([1,2,4,5], [5,4,3,2])
[0 0 1 1]
>>> print add([1,2,4,5], [5,4,3,2])
[6 6 7 7]
>>> print add.reduce([1,2,4,5])
12                                      # 1 + 2 + 4 + 5
\end{verbatim}
Ufuncs are covered in detail in "Ufuncs" on page~\pageref{cha:ufuncs}.


\section{Convenience Functions}
\label{sec:conv-funct}

The numarray module provides, in addition to the functions which are needed to
create the objects above, a set of powerful functions to manipulate arrays,
select subsets of arrays based on the contents of other arrays, and other
array-processing operations.
\begin{verbatim}
>>> data = arange(10)                   # analogous to builtin range()
>>> print data
[0 1 2 3 4 5 6 7 8 9]
>>> print where(greater(data, 5), -1, data)
[ 0  1  2  3  4  5 -1 -1 -1 -1]         # selection facility
>>> data = resize(array([0,1]), (9, 9)) # or just: data=resize([0,1], (9,9))
>>> print data
[[0 1 0 1 0 1 0 1 0]
 [1 0 1 0 1 0 1 0 1]
 [0 1 0 1 0 1 0 1 0]
 [1 0 1 0 1 0 1 0 1]
 [0 1 0 1 0 1 0 1 0]
 [1 0 1 0 1 0 1 0 1]
 [0 1 0 1 0 1 0 1 0]
 [1 0 1 0 1 0 1 0 1]
 [0 1 0 1 0 1 0 1 0]]
\end{verbatim}
All of the functions which operate on numarray arrays are described in chapter
\ref{cha:array-functions}.  See page \pageref{func:where} for more information
about \function{where} and page \pageref{func:resize} for
information on \function{resize}.

\section{Differences between numarray and Numeric.}
\label{sec:diff-numarray-numpy}

This new module numarray was developed for a number of reasons. To 
summarize, we regularly deal with large datasets and numarray gives us the
capabilities that we feel are necessary for working with such datasets. In
particular:
\begin{enumerate}
\item Avoid promotion of array types in expressions involving Python scalars
   (e.g., \code{2.*<Float32 array>} should not result in a \code{Float64}
   array).
\item Ability to use memory mapped files.
\item Ability to access fields in arrays of records as numeric arrays without
   copying the data to a new array.
\item Ability to reference byteswapped data or non-aligned data (as might be
   found in record arrays) without producing new temporary arrays.
\item Reuse temporary arrays in expressions when possible.
\item Provide more convenient use of index arrays (put and take).
\end{enumerate}
We decided to implement a new module since many of the existing Numeric
developers agree that the existing Numeric implementation is not suitable 
for massive changes and enhancements.

This version has nearly the full functionality of the basic Numeric.
\emph{Numarray is not fully compatible with Numeric}.
(But it is very similar in most respects).

The incompatibilities are listed below. 
\begin{enumerate}
\item Coercion rules are different. Expressions involving scalars may not
   produce the same type of arrays.  
\item Types are represented by Type Objects rather than character codes (though
   the old character codes may still be used as arguments to the functions).
\item For versions of Python prior to 2.2, arrays have no public attributes.
   Accessor functions must be used instead (e.g., to get shape for array x, one
   must use x.getshape() instead of x.shape). When using Python 2.2 or later,
   however, the attributes of Numarray are in fact available.
\end{enumerate}
A further comment on type is appropriate here. In numarray, types are
represented by type objects and not character codes. As with Numeric there is a
module variable Float32, but now it represents an instance of a FloatingType
class. For example, if x is a Float32 array, x.type() will return a
FloatingType instance associated with 32-bit floats (instead of using
x.typecode() as is done in Numeric). The following will still work in
numarray, to be backward compatible:
\begin{verbatim}
>>> if x.typecode() == 'f':
\end{verbatim}
or use:
\begin{verbatim}
>>> if x.type() == Float32:
\end{verbatim}
(All examples presume ``\code{from numarray import *}'' has been used instead
of ``\code{import numarray}'', see section \ref{sec:tip:from-numarray-import}.)
The advantage of the new scheme is that other kinds of tests become simpler.
The type classes are hierarchical so one can easily test to see if the array is
an integer array. For example:
\begin{verbatim}
>>> if isinstance(x.type(), IntegralType): 
\end{verbatim}
or:
\begin{verbatim}
>>> if isinstance(x.type(), UnsignedIntegralType):
\end{verbatim}



%% Local Variables:
%% mode: LaTeX
%% mode: auto-fill
%% fill-column: 79
%% indent-tabs-mode: nil
%% ispell-dictionary: "american"
%% reftex-fref-is-default: nil
%% TeX-auto-save: t
%% TeX-command-default: "pdfeLaTeX"
%% TeX-master: "numarray"
%% TeX-parse-self: t
%% End:

\chapter{Array Basics}
\label{cha:array-basics}

\begin{quote} 
   This chapter introduces some of the basic functions which will be used
   throughout the text.
\end{quote}

\section{Basics}
\label{sec:arraybasics:basics}

Before we explore the world of image manipulation as a case-study in array
manipulation, we should first define a few terms which we'll use over and
over again. Discussions of arrays and matrices and vectors can get confusing
due to differences in nomenclature. Here is a brief definition of the terms
used in this tutorial, and more or less consistently in the error messages of
\numarray{}.

The Python objects under discussion are formally called ``\NUMARRAY{}'' (or
even more correctly ``\numarray{}'') objects (N-dimensional arrays), but
informally we'll just call them ``array objects'' or just ``arrays''. These are
different from the array objects defined in the standard Python \module{array}
module (which is an older module designed for processing one-dimensional data
such as sound files).

These array objects hold their data in a fixed length, homogeneous (but not
necessarily contiguous) block of elements, i.e.\ their elements all have the
same C type (such as a 64-bit floating-point number). This is quite different
from most Python container objects, which are variable length heterogeneous
collections.

Any given array object has a \index{rank}rank, which is the number of
``dimensions'' or ``axes'' it has. For example, a point in 3D space \code{[1,
   2, 1]} is an array of rank 1 --- it has one dimension. That dimension has a
length of 3.  As another example, the array
\begin{verbatim}
1.0 0.0 0.0
0.0 1.0 2.0
\end{verbatim}
is an array of rank 2 (it is 2-dimensional). The first dimension has a length
of 2, the second dimension has a length of 3. Because the word ``dimension''
has many different meanings to different folks, in general the word ``axis''
will be used instead. Axes are numbered just like Python list indices: they
start at 0, and can also be counted from the end, so that \code{axis=-1} is the
last axis of an array, \code{axis=-2} is the penultimate axis, etc.  

There are there important and potentially unintuitive behaviors of
\module{numarray} arrays which take some getting used to. The first is that by
default, operations on arrays are performed elementwise.\footnote{This is
common to IDL behavior but contrary to Matlab behavior.}  This means that when
adding two arrays, the resulting array has as elements the pairwise sums of the
two operand arrays.  This is true for all operations, including multiplication.
Thus, array multiplication using the * operator will default to elementwise
multiplication, not matrix multiplication as used in linear algebra. Many
people will want to use arrays as linear algebra-type matrices (including their
\index{rank}rank-1 versions, vectors). For those users, the matrixmultiply
function will be useful.

The second behavior which will catch many users by surprise is that
certain operations, such as slicing, return arrays which are simply different
views of the same data; that is, they will in fact share their data. This will
be discussed at length in examples later.  Now that these definitions and 
warnings are laid out, let's see what we can do with these arrays.

The third behavior which may catch Matlab or Fortran users unaware is the use
of row-major data storage as is done in C.  So while a Fortran array might 
be indexed a[x,y],  numarray is indexed a[y,x].

\newpage
\section{Creating arrays from scratch}
\label{sec:creating-arrays-from}


\subsection{array() and types}
\label{sec:array-types}

\begin{funcdesc}{array}{sequence=None, typecode=None, copy=1, savespace=0,
    type=None, shape=None}
   There are many ways to create arrays. The most basic one is the use of the
   \function{array} function:
\begin{verbatim}
>>> a = array([1.2, 3.5, -1])
\end{verbatim}
   to make sure this worked, do:
\begin{verbatim}
>>> print a
[ 1.2  3.5 -1. ]
\end{verbatim}
   The \function{array} function takes several arguments --- the first one is
   the values, which can be a Python sequence object (such as a list or a
   tuple).  If the optional argument \code{type} is omitted, numarray tries to
   find the best data type which can represent all the elements. 
   
   Since the elements we gave our example were two floats and one integer, it
   chose \class{Float64} as the type of the resulting array. One can specify
   unequivocally the \code{type} of the elements --- this is especially 
   useful when, for example, one wants an array contains floats even
   though all of its input elements are integers:
\begin{verbatim}
>>> x,y,z = 1,2,3
>>> a = array([x,y,z])                  # integers are enough for 1, 2 and 3
>>> print a
[1 2 3]
>>> a = array([x,y,z], type=Float32)    # not the default type
>>> print a
[ 1.  2.  3.]
\end{verbatim}
    Another optional argument is the \code{shape} to use for the array.  When
    passed a \class{NumArray} instance, by default \function{array} will make
    an independent, aligned, contiguous, non-byteswapped copy.  If also passed
    a shape or different type, the resulting ``copy'' will be reshaped or
    cast as the new type.
\end{funcdesc}

\begin{funcdesc}{asarray}{seq, type=None, typecode=None}
   This function converts scalars, lists and tuples to a \class{numarray}, when
   possible. It passes \class{numarray}s through, making copies only to
   convert types.  In any other case a \class{TypeError} is raised.
\end{funcdesc}

\begin{funcdesc}{inputarray}{seq, type=None, typecode=None}
  This is an obosolete alias for \function{asarray}.
\end{funcdesc}


\paragraph*{Important Tip} \label{sec:important-tips} 
Pop Quiz: What will be the type of the array below:
\begin{verbatim}
>>> mystery = array([1, 2.0, -3j])
\end{verbatim}
Hint: -3j is an imaginary number. \\
Answer: Complex64
         
A very common mistake is to call \function{array} with a set of numbers as
arguments, as in \code{array(1, 2, 3, 4, 5)}. This doesn't produce the expected
result if at least two numbers are used, because the first argument to
\function{array} must be the entire data for the array --- thus, in most cases,
a sequence of numbers. The correct way to write the preceding invocation is
most likely \code{array([1, 2, 3, 4, 5])}.

Possible values for the type \index{type argument}argument to the
\function{array} creator function (and indeed to any function which accepts a
so-called type for arrays) are:
\begin{enumerate}
\item Elements that can have values true or false: \index{Bool}\class{Bool}.
\item Unsigned numeric types: \index{UInt8}\class{UInt8},
  \index{UInt16}\class{UInt16}, \index{UInt32}\class{UInt32}, and
  \index{UInt64}\class{UInt64}\footnote[1]{UInt64 is unsupported on Windows}.
\item Signed numeric types: 
   \begin{itemize}
   \item Signed integer choices: \index{Int8}\class{Int8},
      \index{Int16}\class{Int16}, \index{Int32}\class{Int32}, \index{Int64}\class{Int64}.
   \item Floating point choices: \index{Float32}\class{Float32},
      \index{Float64}\class{Float64}.
   \end{itemize}
\item Complex number types: \index{Complex32}\class{Complex32},
   \index{Complex64}\class{Complex64}.
\end{enumerate}

To specify a type, e.g. \class{UInt8}, etc, the easiest method is just to
specify it as a string:
\begin{verbatim}
a = array([10], type = 'UInt8')
\end{verbatim}

The various means for specifying types are defined in table
\ref{tab:type-specifiers}, with each item in a row being equivalent.  The
\emph{preferred} methods are in the first 3 columns: numarray type object, type
string, or type code.  The last two columns were added for backwards
compatabililty with Numeric and are not recommended for new code.  Numarray
type object and string names denote the size of the type in bits.  The numarray
type code names denote the size of the type in bytes.  The type objects must be
imported from or referenced via the numerictypes module.  All type strings and
type codes are specified using ordinary Python strings, and hence don't require
an import.  Complex type names denote the size of one component, real or
imaginary, in bits/bytes, but the letter code is the total size of the 
whole number ('c8' and 'c16').

\begin{table}[h]
  \centering
  \caption{Type specifiers}
  \label{tab:type-specifiers}
  \begin{tabular}{|l|l|l|l|l|}
    \hline
    Numarray Type&Numarray String&Numarray Code&Numeric String&Numeric Code\\
    \hline
    Int8&'Int8'&'i1'&'Byte'&'1'\\
    \hline
    UInt8&'UInt8'&'u1'&'UByte'& \\
    \hline
    Int16&'Int16'&'i2'&'Short'&'s'\\
    \hline
    UInt16&'UInt16'&'u2'&'UShort'& \\
    \hline
    Int32&'Int32'&'i4'&'Int'&'i'\\
    \hline
    UInt32&'UInt32'&'u4'&'UInt'&'u'\\
    \hline
    Int64&'Int64'&'i8'& & \\
    \hline
    UInt64\footnotemark[1]&'UInt64'&'u8'& & \\
    \hline
    Float32&'Float32'&'f4'&'Float'&'f'\\
    \hline
    Float64&'Float64'&'f8'&'Double'&'d'\\
    \hline
    Complex32&'Complex32'&'c8'& &'F'\\
    \hline
    Complex64&'Complex64'&'c16'&'Complex'&'D'\\
    \hline
    Bool&'Bool'& & & \\
    \hline
  \end{tabular}
\end{table}

\subsection{Multidimensional Arrays}
\label{sec:multi-dim-arrays}

The following example shows one way of creating \index{multidimensional
   arrays}multidimensional arrays:
\begin{verbatim}
>>> ma = array([[1,2,3],[4,5,6]])
>>> print ma
[[1 2 3]
 [4 5 6]]
\end{verbatim}
The first argument to \function{array} in the code above is a single
\class{list} containing two lists, each containing three elements. If we wanted
floats instead, we could specify, as discussed in the previous section, the
optional type we wished:
\begin{verbatim}
>>> ma_floats = array([[1,2,3],[4,5,6]], type=Float32)
>>> print ma_floats
[[ 1.  2.  3.]
 [ 4.  5.  6.]]
\end{verbatim}
This array allows us to introduce the notion of \index{shape}``shape''. The
shape of an array is the set of numbers which define its dimensions. The shape
of the array \var{ma} defined above is 2 by 3. More precisely, all arrays have
an attribute which is a tuple of integers giving the shape. The
\index{getshape}\method{getshape} method returns this tuple.  In general, one
can directly use the \member{shape} attribute (but only for Python 2.2 and
later) to get or set its value. Since it isn't supported for earlier versions
of Python, subsequent examples will use \method{getshape} and
\index{setshape}\method{setshape} only. So, in this case:
\begin{verbatim}
>>> print ma.shape                      # works only with Python 2.2 or later
>>> print ma.getshape()                 # works with all Python versions
(2, 3)
\end{verbatim}
Using the earlier definitions, this is a shape of \index{rank}rank 2, where the
first axis has length 2, and the second axis has length 3. The rank of an array
\code{A} is always equal to \code{len(A.getshape())}.  Note that shape is an
attribute and \method{getshape} is a method of array objects. They are the
first of several that we will see throughout this tutorial. If you're not used
to object-oriented programming, you can think of attributes as ``features'' or
``qualities'' of individual arrays, and methods are functions that operate on
individual arrays.  The relation between an array and its shape is similar to
the relation between a person and their hair color. In Python, it's called an
object/attribute relation.

\begin{funcdesc}{reshape}{a, shape}
   What if one wants to change the dimensions of an array? For now, let us
   consider changing the shape of an array without making it ``grow''. Say, for
   example, we want to make the 2x3 array defined above (\var{ma}) an array of
   rank 1:
\begin{verbatim}
>>> flattened_ma = reshape(ma, (6,))
>>> print flattened_ma
[1 2 3 4 5 6]
\end{verbatim}
   One can change the shape of arrays to any shape as long as the product of
   all the lengths of all the axes is kept constant (in other words, as long as
   the number of elements in the array doesn't change):
\begin{verbatim}
>>> a = array([1,2,3,4,5,6,7,8])
>>> print a
[1 2 3 4 5 6 7 8]
>>> b = reshape(a, (2,4))               # 2*4 == 8
>>> print b
[[1 2 3 4]
 [5 6 7 8]]
>>> c = reshape(b, (4,2))               # 4*2 == 8
>>> print c
[[1 2]
 [3 4]
 [5 6]
 [7 8]]
\end{verbatim}
   The function \function{reshape} expects an array/sequence as its 
   first argument, and a shape as its second argument.
   The shape has to be a sequence of integers (a \class{list} or a
   \class{tuple}).  There is also a \method{setshape}
   method, which changes the shape of an array in-place (see below).
   
   One nice feature of shape tuples is that one entry in the shape tuple is
   allowed to be -1. The -1 will be automatically replaced by whatever number
   is needed to build a shape which does not change the size of the array.
   Thus:
\begin{verbatim}
>>> a = reshape(array(range(25)), (5,-1))
>>> print a, a.getshape()
[[ 0  1  2  3  4]
 [ 5  6  7  8  9]
 [10 11 12 13 14]
 [15 16 17 18 19]
 [20 21 22 23 24]] (5, 5)
\end{verbatim}
   The \member{shape} of an array is a modifiable attribute of the array, but
   it is an internal attribute. You can change the shape of an array by calling
   the \method{setshape} method (or by assigning a \class{tuple} to the shape
   attribute, in Python 2.2 and later), which assigns a new shape to the array:
\begin{verbatim}
>>> a = array([1,2,3,4,5,6,7,8,9,10])
>>> a.getshape()
(10,)
>>> a.setshape((2,5))
>>> a.shape = (2,5)                     # for Python 2.2 and later
>>> print a
[[ 1  2  3  4  5]
 [ 6  7  8  9 10]]
>>> a.setshape((10,1))                  # second axis has length 1
>>> print a
[[ 1]
 [ 2]
 [ 3]
 [ 4]
 [ 5]
 [ 6]
 [ 7]
 [ 8]
 [ 9]
 [10]]
>>> a.setshape((5,-1))                  # note the -1 trick described above
>>> print a
[[ 1  2]
 [ 3  4]
 [ 5  6]
 [ 7  8]
 [ 9 10]]
\end{verbatim}
   As in the rest of Python, violating rules (such as the one about which
   shapes are allowed) results in exceptions:
\begin{verbatim}
>>> a.setshape((6,-1))
Traceback (innermost last):
  File "<stdin>", line 1, in ?
ValueError: New shape is not consistent with the old shape
\end{verbatim}
\end{funcdesc}


\paragraph*{For Advanced Users: Printing arrays}

\begin{quote}
   Sections denoted ``For Advanced Users'' indicates 
   function aspects which may not be needed for a first introduction of
   \numarray{}, but is mentioned for the sake of completeness.
\end{quote}

The default \index{printing arrays}printing routine provided by the
\module{\numarray} module prints arrays as follows:
\begin{enumerate}
\item The last axis is always printed left to right.
\item The next-to-last axis is printed top to bottom.
\end{enumerate}
The remaining axes are printed top to bottom with increasing numbers of
separators.

This explains why rank-1 arrays are printed from left to right, rank-2 arrays
have the first dimension going down the screen and the second dimension going
from left to right, etc.

If you want to change the shape of an array so that it has more elements than
it started with (i.e. grow it), then you have several options: One solution is
to use the \index{concatenate}\function{concatenate} function discussed later.
\begin{verbatim}
>>> print a
[0 1 2 3 4 5 6 6 7]
>>> print concatenate([[a],[a]])
>>> print b
[[0 1 2 3 4 5 6 7]
 [0 1 2 3 4 5 6 7]]
>>> print b.getshape()
(2, 8)
\end{verbatim}


\begin{funcdesc}{resize}{array, shape}
   A final possibility is the \function{resize} function, which takes a
   \var{base} array as its first argument and the desired \var{shape} as the
   second argument.  Unlike \function{reshape}, the shape argument to
   \function{resize} can be a smaller or larger shape than the input
   array. Smaller shapes will result in arrays with the data at the
   ``beginning'' of the input array, and larger shapes result in arrays with
   data containing as many replications of the input array as are needed to
   fill the shape. For example, starting with a simple array
\begin{verbatim}
>>> base = array([0,1])
\end{verbatim}
   one can quickly build a large array with replicated data:
\begin{verbatim}
>>> big = resize(base, (9,9))
>>> print big
[[0 1 0 1 0 1 0 1 0]
 [1 0 1 0 1 0 1 0 1]
 [0 1 0 1 0 1 0 1 0]
 [1 0 1 0 1 0 1 0 1]
 [0 1 0 1 0 1 0 1 0]
 [1 0 1 0 1 0 1 0 1]
 [0 1 0 1 0 1 0 1 0]
 [1 0 1 0 1 0 1 0 1]
 [0 1 0 1 0 1 0 1 0]]
\end{verbatim}
\end{funcdesc}

\newpage
\section{Creating arrays with values specified ``on-the-fly''}
\label{sec:creating-arrays-on-the-fly}

\begin{funcdesc}{zeros}{shape, type}
\end{funcdesc}
\begin{funcdesc}{ones}{shape, type}
   Often, one needs to manipulate arrays filled with numbers which aren't
   available beforehand. The \module{\numarray} module provides a few functions
   which create arrays from scratch: \function{zeros} and \function{ones}
   simply create arrays of a given \var{shape} filled with zeros and ones
   respectively:
\begin{verbatim}
>>> z = zeros((3,3))
>>> print z
[[0 0 0]
 [0 0 0]
 [0 0 0]]
>>> o = ones([2,3])
>>> print o
[[1 1 1]
 [1 1 1]]
\end{verbatim}
   Note that the first argument is a shape --- it needs to be a \class{tuple} or
   a \class{list} of integers. Also note that the default type for the
   returned arrays is \class{Int}, which you can override, e. g.: 
\begin{verbatim}
>>> o = ones((2,3), Float32)
>>> print o
[[ 1.  1.  1.]
 [ 1.  1.  1.]]
\end{verbatim}
\end{funcdesc}


\begin{funcdesc}{arrayrange}{a1, a2=None, stride=1, type=None, shape=None}
\end{funcdesc}
\begin{funcdesc}{arange}{a1, a2=None, stride=1, type=None, shape=None}
   The \function{arange} function is similar to the \function{range} function
   in Python, except that it returns an \class{array} as opposed to a
   \class{list}.
   \function{arange} and \function{arrayrange} are equivalent.
\begin{verbatim}
>>> r = arange(10)
>>> print r
[0 1 2 3 4 5 6 7 8 9]
\end{verbatim}
   Combining the \function{arange} with the \function{reshape} function, we can
   get:
\begin{verbatim}
>>> big = reshape(arange(100),(10,10))
>>> print big
[[ 0  1  2  3  4  5  6  7  8  9]
 [10 11 12 13 14 15 16 17 18 19]
 [20 21 22 23 24 25 26 27 28 29]
 [30 31 32 33 34 35 36 37 38 39]
 [40 41 42 43 44 45 46 47 48 49]
 [50 51 52 53 54 55 56 57 58 59]
 [60 61 62 63 64 65 66 67 68 69]
 [70 71 72 73 74 75 76 77 78 79]
 [80 81 82 83 84 85 86 87 88 89]
 [90 91 92 93 94 95 96 97 98 99]]
\end{verbatim}
   One can set the \code{a1}, \code{a2}, and \code{stride} arguments, which 
   allows for more varied ranges:
\begin{verbatim}
>>> print arange(10,-10,-2)
[10  8  6  4  2  0  -2  -4  -6  -8]
\end{verbatim}
   An important feature of arange is that it can be used with non-integer
   starting points and strides:
\begin{verbatim}
>>> print arange(5.0)
[ 0. 1. 2. 3. 4.]
>>> print arange(0, 1, .2)
[ 0.   0.2  0.4  0.6  0.8]
\end{verbatim}
   If you want to create an array with just one value, repeated over and over,
   you can use the * operator applied to lists
\begin{verbatim}
>>> a = array([[3]*5]*5)
>>> print a
[[3 3 3 3 3]
 [3 3 3 3 3]
 [3 3 3 3 3]
 [3 3 3 3 3]
 [3 3 3 3 3]]
\end{verbatim}
   but that is relatively slow, since the duplication is done on Python lists.
   A quicker way would be to start with 0's and add 3:
\begin{verbatim}
         >>> a = zeros([5,5]) + 3
         >>> print a
         [[3 3 3 3 3]
          [3 3 3 3 3]
          [3 3 3 3 3]
          [3 3 3 3 3]
          [3 3 3 3 3]]
\end{verbatim}
   The optional \code{type} argument forces the type of the resulting array,
   which is otherwise the ``highest'' of the \code{a1}, \code{a2}, and 
   \code{stride} arguments.  The \code{a1} argument defaults to 0 if not 
   specified. Note that if the specified \code{type} is
   is ``lower'' than what \function{arange} would
   normally use, the array is the result of a precision-losing cast (a
   round-down, as that used in the \method{astype} method for arrays.)
\end{funcdesc}


\subsection{Creating an array from a function}
\label{sec:creating-array-from-function}

\begin{funcdesc}{fromfunction}{object, shape} 
   Finally, one may want to create an array whose elements are the result
   of a function evaluation. This is done using the \function{fromfunction}
   function, which takes two arguments, a \var{shape} and a callable
   \var{object} (usually a function).  For example:
\begin{verbatim}
>>> def dist(x,y):
...   return (x-5)**2+(y-5)**2          # distance from (5,5) squared
...
>>> m = fromfunction(dist, (10,10))
>>> print m
[[50 41 34 29 26 25 26 29 34 41]
 [41 32 25 20 17 16 17 20 25 32]
 [34 25 18 13 10  9 10 13 18 25]
 [29 20 13  8  5  4  5  8 13 20]
 [26 17 10  5  2  1  2  5 10 17]
 [25 16  9  4  1  0  1  4  9 16]
 [26 17 10  5  2  1  2  5 10 17]
 [29 20 13  8  5  4  5  8 13 20]
 [34 25 18 13 10  9 10 13 18 25]
 [41 32 25 20 17 16 17 20 25 32]]
>>> m = fromfunction(lambda i,j,k: 100*(i+1)+10*(j+1)+(k+1), (4,2,3))
>>> print m
[[[111 112 113]
  [121 122 123]]
 [[211 212 213]
  [221 222 223]]
 [[311 312 313]
  [321 322 323]]
 [[411 412 413]
  [421 422 423]]]
\end{verbatim}
   These examples show that \function{fromfunction}
   creates an array of the shape specified by its second argument, and with the
   contents corresponding to the value of the function argument (the first
   argument) evaluated at the indices of the array. Thus the value of
   \code{m[3, 4]} in the first example above is the value of dist when
   \code{x=3} and \code{y=4}.  Similarly for the lambda function in the second
   example, but with a rank-3 array.  The implementation of
   \function{fromfunction} consists of:
\begin{verbatim}
def fromfunction(function, dimensions):
    return apply(function, tuple(indices(dimensions)))
\end{verbatim}
   which means that the function \function{function} is called with arguments
   given by the sequence \code{indices(dimensions)}. As described in the
   definition of indices, this consists of arrays of indices which will be of
   rank one less than that specified by dimensions. This means that the
   function argument must accept the same number of arguments as there are
   dimensions in \var{dimensions}, and that each argument will be an array of
   the same shape as that specified by dimensions.  Furthermore, the array
   which is passed as the first argument corresponds to the indices of each
   element in the resulting array along the first axis, that which is passed as
   the second argument corresponds to the indices of each element in the
   resulting array along the second axis, etc. A consequence of this is that
   the function which is used with \function{fromfunction} will work as
   expected only if it performs a separable computation on its arguments, and
   expects its arguments to be indices along each axis. Thus, no logical
   operation on the arguments can be performed, or any non-shape preserving
   operation. Thus, the following will not work as expected:
\begin{verbatim}
>>> def buggy(test):
...     if test > 4: return 1
...     else: return 0
...
>>> print fromfunction(buggy,(10,))
Traceback (most recent call last):
...
RuntimeError: An array doesn't make sense as a truth value.  Use any(a) or
all(a).
\end{verbatim}
The reason \function{buggy()} failed is that indices((10,)) results in an array
passed as \var{test}.  The result of comparing \var{test} with 4 is also an
array which has no unambiguous meaning as a truth value.

Here is how to do it properly. We add a print statement to the
   function for clarity:
\begin{verbatim}
>>> def notbuggy(test):                 # only works in Python 2.1 & later
...     print test
...     return where(test>4,1,0)
...
>>> fromfunction(notbuggy,(10,))
[0 1 2 3 4 5 6 7 8 9]
array([0, 0, 0, 0, 0, 1, 1, 1, 1, 1])
\end{verbatim}
   We leave it as an excercise for the reader to figure out why the ``buggy''
   example gave the result 1.
\end{funcdesc}


\begin{funcdesc}{identity}{size}
   The \function{identity} function takes a single integer argument and returns
   a square identity array (in the ``matrix'' sense) of that \var{size} of
   integers:
\begin{verbatim}
      >>> print identity(5)
      [[1 0 0 0 0]
       [0 1 0 0 0]
       [0 0 1 0 0]
       [0 0 0 1 0]
       [0 0 0 0 1]]
\end{verbatim}
\end{funcdesc}



\newpage
\section{Coercion and Casting}
\label{sec:coercion-casting}

We've mentioned the types of arrays, and how to create arrays with the right
type.  But what happens when arrays with different types interact?  For 
some operations, the behavior of \numarray{} is significantly different 
from Numeric.

\subsection{Automatic Coercions and Binary Operations}
\label{sec:automatic-coercion-binary-casting}

In \numarray{} (in contrast to Numeric), there is now a distinction between how
coercion is treated in two basic cases: array/scalar operations and array/array
operations. In the array/array case, the coercion rules are nearly identical to
those of Numeric, the only difference being combining signed and unsigned
integers of the same size.  The array/array result types are enumerated in
table \ref{tab:array-array-result-types}.
\begin{table}[h]
\footnotesize
\centering
\caption{Array/Array Result Types}
\label{tab:array-array-result-types}
\begin{tabular}{|l|l|l|l|l|l|l|l|l|l|l|l|l|l|}
\hline
 &Bool&Int8&UInt8&Int16&UInt16&Int32&UInt32&Int64&UInt64&Float32&Float64&Complex32&Complex64\\
\hline
Bool&Int8&Int8&UInt8&Int16&UInt16&Int32&UInt32&Int64&UInt64&Float32&Float64&Complex32&Complex64\\
\hline
Int8& &Int8&Int16&Int16&Int32&Int32&Int64&Int64&Int64&Float32&Float64&Complex32&Complex64\\
\hline
UInt8& & &UInt8&Int16&UInt16&Int32&UInt32&Int64&UInt64&Float32&Float64&Complex32&Complex64\\
\hline
Int16& & & &Int16&Int32&Int32&Int64&Int64&Int64&Float32&Float64&Complex32&Complex64\\
\hline
UInt16& & & & &UInt16&Int32&UInt32&Int64&UInt64&Float32&Float64&Complex32&Complex64\\
\hline
Int32& & & & & &Int32&Int64&Int64&Int64&Float32&Float64&Complex32&Complex64\\
\hline
UInt32& & & & & & &UInt32&Int64&UInt64&Float32&Float64&Complex32&Complex64\\
\hline
Int64& & & & & & & &Int64&Int64&Float64&Float64&Complex64&Complex64\\
\hline
UInt64& & & & & & & & &UInt64&Float64&Float64&Complex64&Complex64\\
\hline
Float32& & & & & & & & & &Float32&Float64&Complex32&Complex64\\
\hline
Float64& & & & & & & & & & &Float64&Complex64&Complex64\\
\hline
Complex32& & & & & & & & & & & &Complex32&Complex64\\
\hline
Complex64& & & & & & & & & & & & &Complex64\\
\hline
\end{tabular}
\end{table}

Scalars, however, are treated differently. If the scalar is of the same
``kind'' as the array (for example, the array and scalar are both integer
types) then the output is the type of the array, even if it is of a normally
``lower'' type than the scalar.  Adding an \class{Int16} array with an integer
scalar results in an \class{Int16} array, not an \class{Int32} array as is the
case in Numeric.  Likewise adding a \class{Float32} array to a float scalar
results in a \class{Float32} array rather than a \class{Float64} array as is
the case with Numeric.  Adding an \class{Int16} array and a float scalar will
result in a \class{Float64} array, however, since the scalar is of a higher
kind than the array.  Finally, when scalars and arrays are operated on
together, the scalar is converted to a rank-0 array first.  Thus, adding a
``small'' integer to a ``large'' floating point array is equivalent to first
casting the integer ``up'' to the type of the array.
\begin{verbatim}
>>> print (array ((1, 2, 3), type=Int16) * 2).type()
numarray type: Int16
>>> arange(0, 1.0, .1) + 12
array([ 12. , 12.1, 12.2, 12.3, 12.4, 12.5, 12.6, 12.7, 12.8, 12.9]
\end{verbatim}

The results of array/scalar operations are enumerated in table
\ref{tab:Array-Scalar-Result-Types}.  Entries marked with " are identical to
their neighbors on the same row.
\begin{table}[h]
\footnotesize
\centering
\caption{Array/Scalar Result Types}
\label{tab:Array-Scalar-Result-Types}
\begin{tabular}{|l|l|l|l|l|l|l|l|l|l|l|l|l|l|}
\hline
 &Bool&Int8&UInt8&Int16&UInt16&Int32&UInt32&Int64&UInt64&Float32&Float64&Complex32&Complex64\\
\hline
int&Int32&Int8&UInt8&Int16&UInt16&Int32&UInt32&Int64&UInt64&Float32&Float64&Complex32&Complex64\\
\hline
long&Int32&Int8&UInt8&Int16&UInt16&Int32&UInt32&Int64&UInt64&Float32&Float64&Complex32&Complex64\\
\hline
float&Float64&"&"&"&"&"&"&"&Float64&Float32&Float64&Complex32&Complex64\\
\hline
complex&Complex64&"&"&"&"&"&"&"&"&"&"&"&Complex64\\
\hline
\end{tabular}
\end{table}

\footnotetext[10]{Float64}
\footnotetext[20]{Complex64}

\subsection{The type value table}
\label{sec:type-value-table}

The type identifiers (\class{Float32}, etc.) are \class{NumericType} instances.
The mapping between type and the equivalent C variable is machine dependent.
The correspondences between types and C variables for 32-bit architectures are
shown in Table \ref{tab:type-identifiers}.

\begin{table}[h]
   \centering
   \caption{Type identifier table on a x86 computer.}
   \label{tab:type-identifiers}
   \begin{tabular}{ccl}
      \# of bytes & \# of bits      & Identifier \\
           1      &       8         &   Bool \\
           1      &       8         &   Int8 \\
           1      &       8         &   UInt8 \\
           2      &       16        &   Int16 \\
           2      &       16        &   UInt16 \\
           4      &       32        &   Int32 \\
           4      &       32        &   UInt32 \\
           8      &       64        &   Int64 \\
           8      &       64        &   UInt64 \\
           4      &       32        &   Float32 \\
           8      &       64        &   Float64 \\
           8      &       64        &   Complex32 \\
           16     &      128        &   Complex64 
   \end{tabular}
\end{table}

\subsection{Long: the platform relative type}
The type identifier \class{Long} is aliased to either \class{Int32} or
\class{Int64}, depending on the machine architecture where numarray is
installed.  On 32-bit platforms, \class{Long} is defined as \class{Int32}.  On
64-bit (LP64) platforms, \class{Long} is defined as \class{Int64}. \class{Long}
is used as the default integer type for arrays and for index values, such as
those returned by \function{nonzero}.  

\subsection{Deliberate casts (potentially down)}
\label{sec:deliberate-casts}

\begin{methoddesc}{astype}{type}
   You may also force \module{numarray} to cast any number array to another
   number array.  For example, to take an array of any numeric type
   (\class{IntX} or \class{FloatX} or \class{ComplexX} or \class{UIntX}) and
   convert it to a 64-bit float, one can do:
\begin{verbatim}
>>> floatarray = otherarray.astype(Float64)
\end{verbatim}
   The \var{type} can be any of the number types, ``larger'' or ``smaller''. If
   it is larger, this is a cast-up. If it is smaller, the standard casting
   rules of the underlying language (C) are used, which means that truncation
   or integer wrap-around can occur:
\begin{verbatim}
>>> print x
[   0.     0.4    0.8    1.2  300.6]
>>> print x.astype(Int32)
[  0   0   0   1 300]
>>> print x.astype(Int8)      # wrap-around
[ 0  0  0  1 44]
\end{verbatim}
   If the \var{type} used with \method{astype} is the original array's type,
   then a copy of the original array is returned.
\end{methoddesc}


\newpage
\section{Operating on Arrays}
\label{sec:operating-arrays}

\subsection{Simple operations}
\label{sec:simple-operations}

If you have a keen eye, you have noticed that some of the previous examples did
something new: they added a number to an array. Indeed, most Python operations
applicable to numbers are directly applicable to arrays:
\begin{verbatim}
>>> print a
[1 2 3]
>>> print a * 3
[3 6 9]
>>> print a + 3
[4 5 6]
\end{verbatim}
Note that the mathematical operators behave differently depending on the types
of their operands. When one of the operands is an array and the other a
number, the number is added to all the elements of the array, and the resulting
array is returned. This is called \index{broadcasting}\var{broadcasting}. 
This also occurs for unary mathematical operations such as sine and the 
negative sign:
\begin{verbatim}
>>> print sin(a)
[ 0.84147096 0.90929741 0.14112 ]
>>> print -a
[-1 -2 -3]
\end{verbatim}
When both elements are arrays of the same shape, then a new array is created,
where each element is the operation result of the corresponding elements in 
the original arrays:
\begin{verbatim}
>>> print a + a
[2 4 6]
\end{verbatim}
If the operands of operations such as addition, are arrays having the same
rank but different dimensions, then an exception is generated:
\begin{verbatim}
>>> a = array([1,2,3])
>>> b = array([4,5,6,7])                # note this has four elements
>>> print a + b
Traceback (innermost last):
  File "<stdin>", line 1, in ?
ValueError: Arrays have incompatible shapes
\end{verbatim}
This is because there is no reasonable way for numarray to interpret addition
of a \code{(3,)} shaped array and a \code{(4,)} shaped array.

Note what happens when adding arrays with different rank:
\begin{verbatim}
>>> print a
[1 2 3]
>>> print b
[[ 4  8 12]
 [ 5  9 13]
 [ 6 10 14]
 [ 7 11 15]]
>>> print a + b
[[ 5 10 15]
 [ 6 11 16]
 [ 7 12 17]
 [ 8 13 18]]
\end{verbatim}
This is another form of \index{broadcasting}broadcasting. To understand this,
one needs to look carefully at the shapes of \code{a} and \code{b}:
\begin{verbatim}
>>> a.getshape()
(3,)
>>> b.getshape()
(4,3)
\end{verbatim}
Note that the last axis of \code{a} is the same length as that of \code{b}
(i.e.\ compare the last elements in their shape tuples).  Because \code{a}'s
and \code{b}'s last dimensions both have length 3, those two dimensions were
``matched'', and a new dimension was created and automatically ``assumed'' for
array \code{a}. The data already in \code{a} were ``replicated'' as many 
times as needed (4, in this case) to make the shapes of the two operand 
arrays conform. This
replication (\index{broadcasting}broadcasting) occurs when arrays are operands
to binary operations and their shapes differ, based on the following algorithm:
\begin{itemize}
\item starting from the last axis, the axis lengths (dimensions) of the
   operands are compared,
\item if both arrays have axis lengths greater than 1, but the lengths differ,
   an exception is raised,
\item if one array has an axis length greater than 1, then the other array's
   axis is ``stretched'' to match the length of the first axis; if the other
   array's axis is not present (i.e., if the other array has smaller rank),
   then a new axis of the same length is created.
\end{itemize}

Operands with the following shapes will work:
\begin{verbatim}
(3, 2, 4) and (3, 2, 4)
(3, 2, 4) and (2, 4)
(3, 2, 4) and (4,)
(2, 1, 2) and (2, 2)
\end{verbatim}

But not these:
\begin{verbatim}
(3, 2, 4) and (2, 3, 4)
(3, 2, 4) and (3, 4)
(4,) and (0,)
(2, 1, 2) and (0, 2)
\end{verbatim}

This algorithm is complex to describe, but intuitive in practice.


\subsection{In-place operations}
\label{sec:inplace-operations}

Beginning with Python 2.0, Python supports the in-place operators
\index{+=}\code{+=}, \index{+=}\code{-=}, \index{*=}\code{*=}, and
\index{/=}\code{/=}. \Numarray{} supports these operations, but you need to be
careful. The right-hand side should be of the same type. Some violation of this
is possible, but in general contortions may be necessary for using the smaller
``kinds'' of types.
\begin{verbatim}
>>> x = array ([1, 2, 3], type=Int16)
>>> x += 3.5
>>> print x
[4 5 6]
\end{verbatim}


%% Local Variables:
%% mode: LaTeX
%% mode: auto-fill
%% fill-column: 79
%% indent-tabs-mode: nil
%% ispell-dictionary: "american"
%% reftex-fref-is-default: nil
%% TeX-auto-save: t
%% TeX-command-default: "pdfeLaTeX"
%% TeX-master: "numarray"
%% TeX-parse-self: t
%% End:

\chapter{Array Indexing}
\label{cha:array-indexing}

This chapter discusses the rich and varied ways of indexing numarray
objects to specify individual elements, sub-arrays, sub-samplings, and
even random collections of elements.

\section{Getting and Setting array values}
\label{sec:get-set-array-values}

Just like other Python sequences, array contents are manipulated with the
\code{[]} notation. For rank-1 arrays, there are no differences between list
and array notations:
\begin{verbatim}
>>> a = arange(10)
>>> print a[0]                          # get first element
0
>>> print a[1:5]                        # get second through fifth elements
[1 2 3 4]
>>> print a[-1]                         # get last element
9
>>> print a[:-1]                        # get all but last element
[0 1 2 3 4 5 6 7 8]
\end{verbatim}
If an array is multidimensional (of rank > 1), then specifying a single 
integer index will return an array of
dimension one less than the original array.

\begin{verbatim}
>>> a = arange(9, shape=(3,3))
>>> print a
[[0 1 2]
 [3 4 5]
 [6 7 8]]
>>> print a[0]                          # get first row, not first element!
[0 1 2]
>>> print a[1]                          # get second row
[3 4 5]
\end{verbatim}
To get to individual elements in a rank-2 array, one specifies both indices
separated by commas:
\begin{verbatim}
>>> print a[0,0]                        # get element at first row, first column
0
>>> print a[0,1]                        # get element at first row, second column
1
>>> print a[1,0]                        # get element at second row, first column
3
>>> print a[2,-1]                       # get element at third row, last column
8
\end{verbatim}
Of course, the \code{[]} notation can be used to set values as well:
\begin{verbatim}
>>> a[0,0] = 123
>>> print a
[[123   1   2]
 [  3   4   5]
 [  6   7   8]]
\end{verbatim}
Note that when referring to rows, the right hand side of the equal sign needs
to be a sequence which ``fits'' in the referred array subset, as described 
by the broadcast rule (in the code sample below, a 3-element row):
\begin{verbatim}
>>> a[1] = [10,11,12] ; print a
[[123   1   2]
 [ 10  11  12]
 [  6   7   8]]
>>> a[2] = 99 ; print a
[[123   1   2]
 [ 10  11  12]
 [ 99  99  99]]
\end{verbatim}

Note also that when assigning floating point values to integer arrays that
the values are silently truncated:
\begin{verbatim}
>>> a[1] = 93.999432
[[123   1   2]
 [ 93  93  93]
 [ 99  99  99]]
\end{verbatim}

\newpage
\section{Slicing Arrays}
\label{sec:slicing-arrays}

The standard rules of Python slicing apply to arrays, on a per-dimension basis.
Assuming a 3x3 array:
\begin{verbatim}
>>> a = reshape(arange(9),(3,3))
>>> print a
[[0 1 2]
 [3 4 5]
 [6 7 8]]
\end{verbatim}
The plain \code{[:]} operator slices from beginning to end:
\begin{verbatim}
>>> print a[:,:]
[[0 1 2]
 [3 4 5]
 [6 7 8]]
\end{verbatim}
In other words, \code{[:]} with no arguments is the same as \code{[:]} for
lists --- it can be read ``all indices along this axis''.  (Actually, there is
an important distinction; see below.) So, to get the second row along the
second dimension:
\begin{verbatim}
>>> print a[:,1]
[1 4 7]
\end{verbatim}
Note that what was a ``column'' vector is now a ``row'' vector.  Any ``integer
slice'' (as in the 1 in the example above) results in a returned array with
rank one less than the input array.  

There is one important distinction between
slicing arrays and slicing standard Python sequence objects. A slice of a
\class{list} is a new copy of that subset of the \class{list}; a slice of an
array is just a view into the data of the first array.  To force a copy, you
can use the \function{copy} method. For example:
\begin{verbatim}
>>> a = arange (20)
>>> b = a[3:8]
>>> c = a[3:8].copy()
>>> a[5] = -99
>>> print b
[  3   4 -99   6   7]
>>> print c
[3 4 5 6 7]
\end{verbatim}
If one does not specify as many slices as there are dimensions in an array,
then the remaining slices are assumed to be ``all''. If \var{A} is a rank-3
array, then
\begin{verbatim}
A[1] == A[1,:] == A[1,:,:]
\end{verbatim}
An additional slice notation for arrays which does not exist for Python
lists (before Python 2.3), i. e. the optional third argument, meaning 
the ``step size'', also called \index{stride}stride or increment. Its 
default value is 1, meaning return every element in the specified range.  
Alternate values allow one to skip some of the elements in the slice:
\begin{verbatim}
>>> a = arange(12)
>>> print a
[ 0  1  2  3  4  5  6  7  8  9 10 11]
>>> print a[::2]                        # return every *other* element
[ 0  2  4  6  8 10]
\end{verbatim}
\index{stride!Negative}Negative strides are allowed as long as the starting
index is greater than the stopping index:
\begin{verbatim}
>>> a = reshape(arange(9),(3,3))                                                                                          Array Basics
>>> print a
[[0 1 2]
 [3 4 5]
 [6 7 8]]
>>> print a[:, 0]
[0 3 6]
>>> print a[0:3, 0]
[0 3 6]
>>> print a[2::-1, 0]
[6 3 0]
\end{verbatim}
If a negative stride is specified and the starting or stopping indices are
omitted, they default to ``end of axis'' and ``beginning of axis''
respectively.  Thus, the following two statements are equivalent for the array
given:
\begin{verbatim}
>>> print a[2::-1, 0]
[6 3 0]
>>> print a[::-1, 0]
[6 3 0]
>>> print a[::-1]                       # this reverses only the first axis
[[6 7 8]
 [3 4 5]
 [0 1 2]]
>>> print a[::-1,::-1]                  # this reverses both axes
[[8 7 6]
 [5 4 3]
 [2 1 0]]
\end{verbatim}
One final way of slicing arrays is with the keyword \samp{...} This keyword is
somewhat complicated. It stands for ``however many `:' I need depending on the
rank of the object I'm indexing, so that the indices I \emph{do} specify are at
the end of the index list as opposed to the usual beginning''.

So, if one has a rank-3 array \var{A}, then \code{A[...,0]} is the same thing
as \code{A[:,:,0]}, but if \var{B} is rank-4, then \code{B[...,0]} is the same
thing as: \code{B[:,:,:,0]}. Only one \samp{...} is expanded in an index
expression, so if one has a rank-5 array \var{C}, then \code{C[...,0,...]} is
the same thing as \code{C[:,:,:,0,:]}.

When assigment source and destination locations overlap, i.e. when an array is
assigned onto itself at overlapping indices, it may produce something
"surprising":

\begin{verbatim}
>>> n=numarray.arange(36)
>>> n[11:18]=n[7:14]
>>> n
array([ 0,  1,  2,  3,  4,  5,  6,  7,  8,  9, 10,  7,  8,  9, 10,  7,
        8,  9, 18, 19, 20, 21, 22, 23, 24, 25, 26, 27, 28, 29, 30, 31,
       32, 33, 34, 35])
\end{verbatim}

If the slice on the right hand side (RHS) is AFTER that on the left hand side
(LHS) for 1-D array, then it works fine:

\begin{verbatim}
>>> n=numarray.arange(36)
>>> n[1:8]=n[7:14]       
>>> n
array([ 0,  7,  8,  9, 10, 11, 12, 13,  8,  9, 10, 11, 12, 13, 14, 15,
       16, 17, 18, 19, 20, 21, 22, 23, 24, 25, 26, 27, 28, 29, 30, 31,
       32, 33, 34, 35])
\end{verbatim}

Actually, this behavior can be undedrstood if we follow the pixel by pixel
copying logic.  Parts of the slice start to get the "updated" values when the
RHS is before the LHS.

An easy solution which is guaranteed to work is to use the copy() method on the
righ hand side:

\begin{verbatim}
>>> n=numarray.arange(36)
>>> n[11:18]=n[7:14].copy()
>>> n
array([ 0,  1,  2,  3,  4,  5,  6,  7,  8,  9, 10,  7,  8,  9, 10, 11,
       12, 13, 18, 19, 20, 21, 22, 23, 24, 25, 26, 27, 28, 29, 30, 31,
       32, 33, 34, 35])
\end{verbatim}

\newpage
\section{Pseudo Indices}
This section discusses pseudo-indices, which allow arrays to have their shapes
modified by adding axes, sometimes only for the duration of the evaluation of a
Python expression.

Consider multiplication of a rank-1 array by a scalar:
\begin{verbatim}
>>> a = array([1,2,3])
>>> print a * 2
[2 4 6]
\end{verbatim}
This should be trivial by now; we've just multiplied a rank-1 array by a
scalar . The scalar was converted to a rank-0 array which was then broadcast to
the next rank. This works for adding some two rank-1 arrays as well:
\begin{verbatim}
>>> print a
[1 2 3]
>>> a + array([4])
[5 6 7]
\end{verbatim}
but it won't work if either of the two rank-1 arrays have non-matching
dimensions which aren't 1.  In other words, broadcast only works for
dimensions which are either missing (e.g. a lower-rank array) or for dimensions
of 1.

With this in mind, consider a classic task, matrix multiplication. Suppose we
want to multiply the row vector [10,20] by the column vector [1,2,3].
\begin{verbatim}
>>> a = array([10,20])
>>> b = array([1,2,3])
>>> a * b
ValueError: Arrays have incompatible shapes
\end{verbatim}
% In "This makes sense - we're ..." the hyphen disappears in the PDF.
This makes sense: we're trying to multiply a rank-1 array of shape (2,) with a
rank-1 array of shape (3,). This violates the laws of broadcast. What we really
want to do is make the second vector a vector of shape (3,1), so that the first
vector can be broadcast across the second axis of the second vector. One way to
do this is to use the reshape function:
\begin{verbatim}
>>> a.getshape()
(2,)
>>> b.getshape()
(3,)
>>> b2 = reshape(b, (3,1))
>>> print b2
[[1]
 [2]
 [3]]
>>> b2.getshape()
(3, 1)
>>> print a * b2    # Note: b2 * a gives the same result
[[10 20]
 [20 40]
 [30 60]]
\end{verbatim}
This is such a common operation that a special feature was added (it turns out
to be useful in many other places as well) --� the NewAxis "pseudo-index",
originally developed in the Yorick language. NewAxis is an index, just like
integers, so it is used inside of the slice brackets []. It can be thought of
as meaning "add a new axis here," in much the same ways as adding a 1 to an
array's shape adds an axis. Again, examples help clarify the situation:
\begin{verbatim}
>>> print b
[1 2 3]
>>> b.getshape()
(3,)
>>> c = b[:, NewAxis]
>>> print c
[[1]
 [2]
 [3]]
>>> c.getshape()
(3,1)
\end{verbatim}
Why use such a pseudo-index over the reshape function or setshape assignments?
Often one doesn't really want a new array with a new axis, one just wants it
for an intermediate computation. Witness the array multiplication mentioned
above, without and with pseudo-indices:
\begin{verbatim}
>>> without = a * reshape(b, (3,1))
>>> with = a * b[:,NewAxis]
\end{verbatim}
The second is much more readable (once you understand how NewAxis works), and
it's much closer to the intended meaning. Also, it's independent of the
dimensions of the array b. You might counter that using something like
reshape(b, (-1,1)) is also dimension-independent, but 
it's less readable and impossible with rank-3 or higher arrays? The
NewAxis-based idiom also works nicely with higher rank arrays, and with the ...
"rubber index" mentioned earlier. Adding an axis before the last axis in an
array can be done simply with:
\begin{verbatim}
>>> a[...,NewAxis,:]
\end{verbatim}
Note that \code{NewAxis} is a \code{numarray} object, so if you used 
\code{import numarray} instead of \code{from numarray import *}, you'll 
need \code{numarray.NewAxis}.

\newpage
\section{Index Arrays}
\label{sec:index-arrays}

Arrays used as subscripts have special meanings which implicitly invoke the
functions \function{put} (page \pageref{func:put}), \function{take} (page
\pageref{func:take}), or \function{compress} (page \pageref{func:compress}). If
the array is of \class{Bool} type, then the indexing will be treated as the
equivalent of the compress function. If the array is of an integer type, then a
\function{take} or \function{put} operation is implied. We will generalize the
existing take and put as follows: If \var{ind1}, \var{ind2}, ...  \var{indN}
are index arrays (arrays of integers whose values indicate the index into
another array), then \code{x[ind1, ind2]} forms a new array with the same shape
as \var{ind1}, \var{ind2} (they all must be broadcastable to the same shape)
and values such: \samp{result[i,j,k] = x[ind1[i,j,k], ind2[i,j,k]]} In this
example, \var{ind1}, \var{ind2} are index arrays with 3 dimensions (but they
could have an arbitrary number of dimensions).  To illustrate with some
specific examples:
\begin{verbatim}
>>> x=2*arange(10)
>>> ind1=[0,4,3,7]
>>> x[ind1]
array([ 0,  8,  6, 14])
>>> ind1=[[0,4],[3,7]]
>>> x[ind1]
array([[ 0,  8],
       [ 6, 14]])
\end{verbatim}
This shows that the same elements in the same order are extracted from x by
both forms of ind1, but the result shares the shape of ind1 Something similar
happens in two dimensions:
\begin{verbatim}
>>> x=reshape(arange(12),(3,4))
>>> x
array([[ 0,  1,  2,  3],
       [ 4,  5,  6,  7],
       [ 8,  9, 10, 11]])
>>> ind1=array([2,1])
>>> ind2=array([0,3])
>>> x[ind1,ind2]
array([8, 7])
\end{verbatim}
Notice this pulls out x[2,0] and x[1,3] as a one-dimensional array.
\begin{verbatim}
>>> ind1=array([[2,2],[1,0]])
>>> ind2=array([[0,1],[3,2]])
>>> x[ind1,ind2]
array([[8, 9],
       [7, 2]])
\end{verbatim}
This pulls out x[2,0], x[2,1], x[1,3], and x[0,2], reading the ind1 and ind2
arrays left to right, and then reshapes the result to the same (2,2) shape as
ind1 and ind2 have.
\begin{verbatim}
>>> ind1.shape=(4,)
>>> ind2.shape=(4,)
>>> x[ind1,ind2]
array([8, 9, 7, 2])
\end{verbatim}

\newpage
Notice this is the same values in the same order, but now as a one-d array.
One index array does a broadcast:
\begin{verbatim}
>>> x[ind1]
array([[ 8,  9, 10, 11],
       [ 8,  9, 10, 11],
       [ 4,  5,  6,  7],
       [ 0,  1,  2,  3]])
>>> ind1.shape=(2,2)
>>> x[ind1]
array([[[ 8,  9, 10, 11],
        [ 8,  9, 10, 11]],

       [[ 4,  5,  6,  7],
        [ 0,  1,  2,  3]]])
\end{verbatim}

Again, note that the same 'elements', in this case rows of x, are returned in
both cases.  But in the second case, ind1 had two dimensions, and so using it
to index only one dimension of a two-d array results in a three-d output of
shape (2,2,4);  i.e., a 2 by 2 'array' of 4-element rows.

When using constants for some of the index positions, then the result uses that
constant for all values. Slices and strides (at least initially) will not be
permitted in the same subscript as index arrays. So
\begin{verbatim}
>>> x[ind1, 2]
array([[10, 10],
  [ 6,  2]])
\end{verbatim}
would be legal, but
\begin{verbatim}
>>> x[ind1, 1:3]
Traceback (most recent call last):
...
IndexError: Cannot mix arrays and slices as indices
\end{verbatim}
would not be.  Similarly for assignment:
\begin{verbatim}
x[ind1, ind2, ind3] = values
\end{verbatim}
will form a new array such that:
\begin{verbatim}
x[ind1[i,j,k], ind2[i,j,k], ind3[i,j,k]] = values[i,j,k]
\end{verbatim}

The index arrays and the value array must be broadcast consistently. (As an
example: \code{ind1.setshape((5,4))}, \code{ind2.setshape((5,))},
\code{ind3.setshape((1,4))}, and \code{values.setshape((1,))}.)
\begin{verbatim}
>>> x=zeros((10,10))
>>> x[[2,5,6],array([0,1,9,3])[:,NewAxis]]=array([1,2,3,4])[:,NewAxis]
>>> x
array([[0, 0, 0, 0, 0, 0, 0, 0, 0, 0],
       [0, 0, 0, 0, 0, 0, 0, 0, 0, 0],
       [1, 2, 0, 4, 0, 0, 0, 0, 0, 3],
       [0, 0, 0, 0, 0, 0, 0, 0, 0, 0],
       [0, 0, 0, 0, 0, 0, 0, 0, 0, 0],
       [1, 2, 0, 4, 0, 0, 0, 0, 0, 3],
       [1, 2, 0, 4, 0, 0, 0, 0, 0, 3],
       [0, 0, 0, 0, 0, 0, 0, 0, 0, 0],
       [0, 0, 0, 0, 0, 0, 0, 0, 0, 0],
       [0, 0, 0, 0, 0, 0, 0, 0, 0, 0]])
\end{verbatim}
If indices are repeated, the last value encountered will be stored.  When an
index is too large, Numarray raises an IndexError exception. When an index is
negative, Numarray will interpret it in the usual Python style, counting
backwards from the end.  Use of the equivalent \index{take}\function{take} and
\index{put}\function{put} functions will allow other interpretations of the
indices (clip out of bounds indices, allow negative indices to work backwards
as they do when used individually, or for indices to wrap around). The same
behavior applies for functions such as choose and where.

%% Local Variables:
%% mode: LaTeX
%% mode: auto-fill
%% fill-column: 79
%% indent-tabs-mode: nil
%% ispell-dictionary: "american"
%% reftex-fref-is-default: nil
%% TeX-auto-save: t
%% TeX-command-default: "pdfeLaTeX"
%% TeX-master: "numarray"
%% TeX-parse-self: t
%% End:

\chapter{Intermediate Topics}
\label{cha:intermediate-topics}

This chapter discusses a few of the more esoteric features of numarray which
are certainly useful but probably not a top priority for new users.

\section{Rank-0 Arrays}
\label{sec:rank-0-arrays}
numarray provides limited support for dimensionless arrays which represent
single values, also known as rank-0 arrays.  Rank-0 arrays are the array
representation of a scalar value.  They have the advantage over scalars that
they include array specific type information, e.g. \var{Int16}.  Rank-0 arrays
can be created as follows:
\begin{verbatim}
>>> a = array(1); a
array(1)
\end{verbatim}
A rank-0 array has a 0-length or empty shape:
\begin{verbatim}
>>> a.shape
()
\end{verbatim}
numarray's rank-0 array handling differs from that of Numeric in two ways.
First, numarray's rank-0 arrays cannot be indexed by 0:
\begin{verbatim}
>>> array(1)[0]
Traceback (most recent call last):
...
IndexError: Too many indices
\end{verbatim}
Second, numarray's rank-0 arrays do not have a length.
\begin{verbatim}
>>> len(array(1))
Traceback (most recent call last):
...
ValueError: Rank-0 array has no length.
\end{verbatim}
Finally, numarray's rank-0 arrays can be converted to a Python scalar by
subscripting with an empty tuple as follows:
\begin{verbatim}
>>> a = array(1)
>>> a[()]
1
\end{verbatim}

\newpage
\section{Exception Handling}
\label{sec:exception-handling}

We desired better control over exception handling than currently exists in
Numeric. This has traditionally been a problem area (see the numerous posts in
\ulink{comp.lang.python}{news:comp.lang.python} regarding floating point
exceptions, especially those by Tim Peters). Numeric raises an exception for
integer computations that result in a divide by zero or multiplications that
result in overflows. The exception is raised after that operation has completed
on all the array elements. No exceptions are raised for floating point errors
(divide by zero, overflow, underflow, and invalid results), the compiler and
processor are left to their default behavior (which is usually to return Infs
and NaNs as values).

The approach for numarray is to provide customizable error handling behavior.
It should be possible to specify three different behaviors for each of the four
error types independently. These are:
\begin{itemize}
\item Ignore the error.
\item Print a warning.
\item Raise a Python exception.
\end{itemize}
The current implementation does that and has been tested successfully on
Windows, Solaris, Redhat and Tru64.  The implementation uses the floating point
processor ``sticky status flags'' to detect errors. One can set the error mode
by calling the error object's setMode method. For example:
\begin{verbatim}
>>> Error.setMode(all="warn") # the default mode
>>> Error.setMode(dividebyzero="raise", underflow="ignore", invalid="warn")
\end{verbatim}

The Error object can also be used in a stacking manner, by using the \function{pushMode}
and \function{popMode} methods rather than \function{setMode}.  For example:
\begin{verbatim}
>>> Error.getMode()
_NumErrorMode(overflow='warn', underflow='warn', dividebyzero='warn', invalid='warn')
>>> Error.pushMode(all="raise") # get really picky...
>>> Error.getMode()
_NumErrorMode(overflow='raise', underflow='raise', dividebyzero='raise', invalid='raise')
>>> Error.popMode()  # pop and return the ``new'' mode
_NumErrorMode(overflow='raise', underflow='raise', dividebyzero='raise', invalid='raise')
>>> Error.getMode()  # verify the original mode is back
_NumErrorMode(overflow='warn', underflow='warn', dividebyzero='warn', invalid='warn')
\end{verbatim}
Integer exception modes work the same way. Although integer computations do not
affect the floating point status flag directly, our code checks the denominator
of 0 in divisions (in much the same way Numeric does) and then performs a
floating point divide by zero to set the status flag (overflows are handled
similarly). So even integer exceptions use the floating point status flags
indirectly.

\newpage
\section{IEEE-754 Not a Number (NAN) and Infinity}
\label{sec:ieee-special-values}
\module{numarray.ieeespecial} has support for manipulating IEEE-754 floating
point special values NaN (Not a Number), Inf (infinity), etc.  The special
values are denoted using lower case as follows:
\begin{verbatim}
>>> import numarray.ieeespecial as ieee
>>> ieee.inf
inf
>>> ieee.plus_inf
inf
>>> ieee.minus_inf
-inf
>>> ieee.nan
nan
>>> ieee.plus_zero
0.0
>>> ieee.minus_zero
-0.0
\end{verbatim}
Note that the representation of IEEE special values is platform dependent so
your Python might for instance say \var{Infinity} rather than \var{inf}.
Below, \var{inf} is seen to arise as the result of floating point division by 0
and \var{nan} is seen to arise from 0 divided by 0:
\begin{verbatim}
>>> a = array([0.0, 1.0])
>>> b = a/0.0
Warning: Encountered invalid numeric result(s)  in divide
Warning: Encountered divide by zero(s)  in divide
>>> b
array([              nan,               inf])
\end{verbatim}
A curious property of \var{nan} is that it does not compare to \emph{itself} as
equal:
\begin{verbatim}
>>> b == ieee.nan
array([0, 0], type=Bool)
\end{verbatim}
The \function{isnan}, \function{isinf}, and \function{isfinite} functions
return boolean arrays which have the value True where the corresponding
predicate holds.  These functions detect bit ranges and are therefore more
robust than simple equality checks.
\begin{verbatim}
>>> ieee.isnan(b)
array([1, 0], type=Bool)
>>> ieee.isinf(b)
array([0, 1], type=Bool)
>>> ieee.isfinite(b)
array([0, 0], type=Bool)
\end{verbatim}
Array based indexing provides a convenient way to replace special values:
\begin{verbatim}
>>> b[ieee.isnan(b)] = 999
>>> b[ieee.isinf(b)] = 5
>>> b
array([ 999.,    5.])
\end{verbatim}

Here's an easy approach for compressing your data arrays to remove
NaNs:
\begin{verbatim}
>>> x, y = arange(10.), arange(10.)
>>> x[5] = ieee.nan
>>> y[6] = ieee.nan
>>> keep = ~ieee.isnan(x) & ~ieee.isnan(y)
>>> x[keep]
array([ 0.,  1.,  2.,  3.,  4.,  7.,  8.,  9.])
>>> y[keep]
array([ 0.,  1.,  2.,  3.,  4.,  7.,  8.,  9.])
\end{verbatim}

%% Local Variables:
%% mode: LaTeX
%% mode: auto-fill
%% fill-column: 79
%% indent-tabs-mode: nil
%% ispell-dictionary: "american"
%% reftex-fref-is-default: nil
%% TeX-auto-save: t
%% TeX-command-default: "pdfeLaTeX"
%% TeX-master: "numarray"
%% TeX-parse-self: t
%% End:

\chapter{Ufuncs}
\label{cha:ufuncs}

\section{What are Ufuncs?}
\label{sec:what-are-ufuncs}

The operations on arrays that were mentioned in the previous section
(element-wise addition, multiplication, etc.) all share some features --- they
all follow similar rules for broadcasting, coercion and ``element-wise
operation''. Just as standard addition is available in Python through the add
function in the operator module, array operations are available through
callable objects as well. Thus, the following objects are available in the
numarray module:

\begin{table}[htbp]
   \centering
   \caption{Universal Functions, or ufuncs. The operators which invoke them
   when applied to arrays are indicated in parentheses. The entries in slanted
   typeface refer to unary ufuncs, while the others refer to binary ufuncs.} 
   \label{tab:ufuncs}
   \begin{tabular}{llll}
      add ($+$)         & subtract ($-$)   & multiply (*)       & divide ($/$) \\
      remainder (\%)    & power (**)       & \textsl{arccos}    & \textsl{arccosh} \\
      \textsl{arcsin}   & \textsl{arcsinh} & \textsl{arctan}    & \textsl{arctanh} \\
      \textsl{cos}      & \textsl{cosh}    & \textsl{tan}       & \textsl{tanh} \\
      \textsl{log10}    & \textsl{sin}     & \textsl{sinh}      & \textsl{sqrt} \\
      \textsl{absolute (abs)} & \textsl{fabs}    & \textsl{floor}     & \textsl{ceil} \\
      fmod              & \textsl{exp}     & \textsl{log}       & \textsl{conjugate} \\
      maximum           & minimum \\
      greater ($>$)     & greater\_equal ($>=$) & equal ($==$)  \\
      less ($<$)        & less\_equal ($<=$)  & not\_equal ($!=$) \\
      logical\_or       & logical\_xor     & logical\_not  & logical\_and \\
      bitwise\_or ($|$) & bitwise\_xor (\^{}) 
                        & bitwise\_not (\textasciitilde)  & bitwise\_and (\&)
      \\
      rshift ($>>$)       & lshift ($<<$)
   \end{tabular}
\end{table}

\remark{Add a remark here on how numarray does (or will) handle 'true'
and 'floor' division, which can be activated in Python 2.2 with
\samp{from __future__ import division}?.
Note: with 'true' division, \samp{1/2 == 0.5} and not \samp{0}.}

All of these ufuncs can be used as functions. For example, to use
\function{add}, which is a binary ufunc (i.e.\ it takes two arguments), one can
do either of:
\begin{verbatim}
>>> a = arange(10)
>>> print add(a,a)
[ 0  2  4  6  8 10 12 14 16 18]
>>> print a + a
[ 0  2  4  6  8 10 12 14 16 18]
\end{verbatim}
In other words, the \code{+} operator on arrays performs exactly the same thing
as the \function{add} ufunc when operated on arrays.  For a unary ufunc such as
\function{sin}, one can do, e.g.:
\begin{verbatim}
>>> a = arange(10)
>>> print sin(a)
[ 0.          0.84147096  0.90929741  0.14112    -0.7568025
      -0.95892429 -0.27941549  0.65698659  0.98935825  0.41211849]
\end{verbatim}
A unary ufunc returns an array with the same shape as its argument array, but
with each element replaced by the application of the function to that element
(\code{sin(0)=0}, \code{sin(1)=0.84147098}, etc.).

There are three additional features of ufuncs which make them different from
standard Python functions. They can operate on any Python sequence in addition
to arrays; they can take an ``output'' argument; they have methods which are
themselves callable with arrays and sequences. Each of these will be described
in turn.

Ufuncs can operate on any Python sequence. Ufuncs have so far been described as
callable objects which take either one or two arrays as arguments (depending on
whether they are unary or binary). In fact, any Python sequence which can be
the input to the \function{array} constructor can be used.  The return value
from ufuncs is always an array.  Thus:
\begin{verbatim}
>>> add([1,2,3,4], (1,2,3,4))
array([2, 4, 6, 8])
\end{verbatim}


\subsection{Ufuncs can take output arguments}
\label{sec:ufuncs-can-take}

In many computations with large sets of numbers, arrays are often used only
once. For example, a computation on a large set of numbers could involve the
following step
\begin{verbatim}
dataset = dataset * 1.20 
\end{verbatim}
This can also be written as the following using the Ufunc form:
\begin{verbatim}
dataset = multiply(dataset, 1.20)
\end{verbatim}
In both cases, a temporary array is created to store the results of the
computation before it is finally copied into \var{dataset}. It is
more efficient, both in terms of memory and computation time, to do an
``in-place'' operation.  This can be done by specifying an existing array as
the place to store the result of the ufunc. In this example, one can 
write:\footnote[1]{for Python-2.2.2 or later: `dataset *= 1.20' also works}
\begin{verbatim}
multiply(dataset, 1.20, dataset) 
\end{verbatim}
This is not a step to take lightly, however. For example, the ``big and slow''
version (\code{dataset = dataset * 1.20}) and the ``small and fast'' version
above will yield different results in at least one case:
\begin{itemize}
\item If the type of the target array is not that which would normally be
   computed, the operation will not coerce the array to the expected data type.
   (The result is done in the expected data type, but coerced back to the
   original array type.)
\item Example:
\begin{verbatim}
\>>> a=arange(5,type=Int32)
>>> print a[::-1]*1.7
[ 6.8  5.1  3.4  1.7  0. ]
>>> multiply(a[::-1],1.7,a)
>>> print a
[6 5 3 1 0]
>>> a *= 1.7
>>> print a
[0 1 3 5 6]
\end{verbatim}
\end{itemize}

The output array does not need to be the same variable as the input array. In
numarray, in contrast to Numeric, the output array may have any type (automatic
conversion is performed on the output).

\subsection{Ufuncs have special methods}
\label{sec:ufuncs-have-special-methods}


\begin{methoddesc}{reduce}{a, axis=0}
   If you don't know about the \function{reduce} command in Python, review
   section 5.1.3 of the Python Tutorial
   (\url{http://www.python.org/doc/current/tut/}). Briefly,
   \function{reduce} is most often used with two arguments, a callable object
   (such as a function), and a sequence. It calls the callable object with the
   first two elements of the sequence, then with the result of that operation
   and the third element, and so on, returning at the end the successive
   ``reduction'' of the specified callable object over the sequence elements.
   Similarly, the \method{reduce} method of ufuncs is called with a sequence as
   an argument, and performs the reduction of that ufunc on the sequence. As an
   example, adding all of the elements in a rank-1 array can be done with:
\begin{verbatim}
>>> a = array([1,2,3,4])
>>> print add.reduce(a)   # with Python's reduce, same as reduce(add, a)
10
\end{verbatim}
   When applied to arrays which are of rank greater than one, the reduction
   proceeds by default along the first axis:
\begin{verbatim}
>>> b = array([[1,2,3,4],[6,7,8,9]])
>>> print b
[[1 2 3 4]
 [6 7 8 9]]
>>> print add.reduce(b)
[ 7  9 11 13]
\end{verbatim}
   A different axis of reduction can be specified with a second integer
   argument:
\begin{verbatim}
>>> print b
[[1 2 3 4]
 [6 7 8 9]]
>>> print add.reduce(b, axis=1)
[10 30]
\end{verbatim}
\end{methoddesc}


\begin{methoddesc}{accumulate}{a}
   The \method{accumulate} ufunc method is simular to \method{reduce}, except
   that it returns an array containing the intermediate results of the
   reduction:
\begin{verbatim}
>>> a = arange(10)
>>> print a
[0 1 2 3 4 5 6 7 8 9]
>>> print add.accumulate(a)
[ 0  1  3  6 10 15 21 28 36 45] # 0, 0+1, 0+1+2, 0+1+2+3, ... 0+...+9
>>> print add.reduce(a) # same as add.accumulate(a)[-1] w/o side effects on a
45                                      
\end{verbatim}
\end{methoddesc}


\begin{methoddesc}{outer}{a, b}
   The third ufunc method is \method{outer}, which takes two arrays as
   arguments and returns the ``outer ufunc'' of the two arguments. Thus the
   \method{outer} method of the \function{multiply} ufunc, results in the outer
   product. The \method{outer} method is only supported for binary methods.
\begin{verbatim}
>>> print a
[0 1 2 3 4]
>>> print b
[0 1 2 3]
>>> print add.outer(a,b)
[[0 1 2 3]
 [1 2 3 4]
 [2 3 4 5]
 [3 4 5 6]
 [4 5 6 7]]
>>> print multiply.outer(b,a)
[[ 0  0  0  0  0]
 [ 0  1  2  3  4]
 [ 0  2  4  6  8]
 [ 0  3  6  9 12]]
>>> print power.outer(a,b)
[[ 1  0  0  0]
 [ 1  1  1  1]
 [ 1  2  4  8]
 [ 1  3  9 27]
 [ 1  4 16 64]]
\end{verbatim}
\end{methoddesc}


\begin{methoddesc}{reduceat}{}
   The reduceat method of Numeric has not been implemented in numarray.
\end{methoddesc}

\subsection{Ufuncs always return new arrays}
\label{sec:ufuncs-always-return}

Except when the output argument is used as described above, ufuncs always
return new arrays which do not share any data with the input arrays.


\section{Which are the Ufuncs?}
\label{sec:which-are-ufuncs}

Table \ref{tab:ufuncs} lists all the ufuncs. We will first discuss the
mathematical ufuncs, which perform operations very similar to the functions in
the \module{math} and \module{cmath} modules, albeit elementwise, on arrays.
Originally,  numarray ufuncs came in two forms, unary and binary.  More
recently (1.3) numarray has added support for ufuncs with up to 16 total
input or output parameters.  These newer ufuncs are called N-ary ufuncs.

\subsection{Unary Mathematical Ufuncs}
\label{sec:unary-math-ufuncs}

Unary ufuncs take only one argument.  The following ufuncs apply the
predictable functions on their single array arguments, one element at a time:
\function{arccos}, \function{arccosh}, \function{arcsin}, \function{arcsinh},
\function{arctan}, \function{arctanh}, \function{cos}, \function{cosh},
\function{exp}, \function{log}, \function{log10}, \function{sin},
\function{sinh}, \function{sqrt}, \function{tan}, \function{tanh},
\function{abs}, \function{fabs}, \function{floor}, \function{ceil},
\function{conjugate}.  As an example:
\begin{verbatim}
>>> print x
[0 1 2 3 4]
>>> print cos(x)
[ 1.          0.54030231 -0.41614684 -0.9899925  -0.65364362]
>>> print arccos(cos(x))
[ 0.          1.          2.          3.          2.28318531]
# not a bug, but wraparound: 2*pi%4 is 2.28318531
\end{verbatim}


\subsection{Binary Mathematical Ufuncs}
\label{sec:binary-math-ufuncs}

These ufuncs take two arrays as arguments, and perform the specified
mathematical operation on them, one pair of elements at a time: \function{add},
\function{subtract}, \function{multiply}, \function{divide},
\function{remainder}, \function{power}, \function{fmod}.


\subsection{Logical and bitwise ufuncs}
\label{sec:logical-ufuncs}

The ``logical'' ufuncs also perform their operations on arrays or numbers 
in elementwise fashion, just like the "mathematical" ones.  Two are special
(\function{maximum} and \function{miminum}) in that they return arrays with
entries taken from their input arrays:
\begin{verbatim}
>>> print x
[0 1 2 3 4]
>>> print y
[ 2.   2.5  3.   3.5  4. ]
>>> print maximum(x, y)
[ 2.   2.5  3.   3.5  4. ]
>>> print minimum(x, y)
[ 0.  1.  2.  3.  4.]
\end{verbatim}
The others logical ufuncs return arrays of 0's or 1's and of type Bool:
\function{logical_and}, \function{logical_or}, \function{logical_xor},
\function{logical_not}, 
These are fairly
self-explanatory, especially with the associated symbols from the standard
Python version of the same operations in Table \ref{tab:ufuncs} above. 
The bitwise ufuncs,
\function{bitwise_and}, \function{bitwise_or},
\function{bitwise_xor}, \function{bitwise_not},  
\function{lshift}, \function{rshift},  
on the other hand, only work with integer arrays (of any word size), and
will return integer arrays of the larger bit size of the two input arrays:
\begin{verbatim}
>>> x
array([7, 7, 0], type=Int8)
>>> y
array([4, 5, 6])
>>> x & y          # bitwise_and(x,y)
array([4, 5, 0])
>>> x | y          # bitwise_or(x,y)
array([7, 7, 6])   
>>> x ^ y          # bitwise_xor(x,y)
array([3, 2, 6]) 
>>> ~ x            # bitwise_not(x)
array([-8, -8, -1], type=Int8)

\end{verbatim}
We discussed finding contents of arrays based on arrays' indices by using slice.
Often, especially when dealing with the result of computations or data
analysis, one needs to ``pick out'' parts of matrices based on the content of
those matrices. For example, it might be useful to find out which elements of
an array are negative, and which are positive. The comparison ufuncs are
designed for such operation. Assume an array with various positive
and negative numbers in it (for the sake of the example we'll generate it from
scratch):
\begin{verbatim}
>>> print a
[[ 0  1  2  3  4]
 [ 5  6  7  8  9]
 [10 11 12 13 14]
 [15 16 17 18 19]
 [20 21 22 23 24]]
>>> b = sin(a)
>>> print b
[[ 0.          0.84147098  0.90929743  0.14112001 -0.7568025 ]
 [-0.95892427 -0.2794155   0.6569866   0.98935825  0.41211849]
 [-0.54402111 -0.99999021 -0.53657292  0.42016704  0.99060736]
 [ 0.65028784 -0.28790332 -0.96139749 -0.75098725  0.14987721]
 [ 0.91294525  0.83665564 -0.00885131 -0.8462204  -0.90557836]]
>>> print greater(b, .3)
[[0 1 1 0 0]
 [0 0 1 1 1]
 [0 0 0 1 1]
 [1 0 0 0 0]
 [1 1 0 0 0]]
\end{verbatim}


\subsection{Comparisons}
\label{sec:comparisons}

The comparison functions \function{equal}, \function{not_equal},
\function{greater}, \function{greater_equal}, \function{less}, and
\function{less_equal} are invoked by the operators \code{==}, \code{!=},
\code{>}, \code{>=}, \code{<}, and \code{<=} respectively, but they can also be
called directly as functions. Continuing with the preceding example,
\begin{verbatim}
>>> print less_equal(b, 0)
[[1 0 0 0 1]
 [1 1 0 0 0]
 [1 1 1 0 0]
 [0 1 1 1 0]
 [0 0 1 1 1]]
\end{verbatim}
This last example has 1's where the corresponding elements are less than or
equal to 0, and 0's everywhere else.

The operators and the comparison functions are not exactly equivalent.  To
compare an array a with an object b, if b can be converted to an array, the
result of the comparison is returned. Otherwise, zero is returned. This means
that comparing a list and comparing an array can return quite different
answers. Since the functional forms such as equal will try to make arrays from
their arguments, using equal can result in a different result than using
\code{==}.
\begin{verbatim}
>>> a = array([1, 2, 3])
>>> b = [1, 2, 3]
>>> print a == 2
[0 1 0]
>>> print b == 2  
0          # (False since 2.3)
>>> print equal(a, 2)
[0 1 0]
>>> print equal(b, 2)
[0 1 0]
\end{verbatim}

\subsection{Ufunc shorthands}
\label{sec:ufunc-shorthands}

Numarray defines a few functions which correspond to popular ufunc methods:
for example, \function{add.reduce} is synonymous with the \function{sum}
utility function:
\begin{verbatim}
>>> a = arange(5)                       # [0 1 2 3 4]
>>> print sum(a)                        # 0 + 1 + 2 + 3 + 4
10
\end{verbatim}
Similarly, \function{cumsum} is equivalent to \function{add.accumulate} (for
``cumulative sum''), \function{product} to \function{multiply.reduce}, and
\function{cumproduct} to \function{multiply.accumulate}.  Additional useful
``utility'' functions are \function{all} and \function{any}:
\begin{verbatim}
>>> a = array([0,1,2,3,4])
>>> print greater(a,0)
[0 1 1 1 1]
>>> all(greater(a,0))
0
>>> any(greater(a,0))
1
\end{verbatim}

\section{Writing your own ufuncs!}

This section describes a new process for defining your own universal functions.
It explains a new interface that enables the description of N-ary ufuncs, those
that use semi-arbitrary numbers \((<= 16)\) of inputs and outputs.

\subsection{Runtime components of a ufunc}

A numarray universal function maps from a Python function name to a set of C
functions.  Ufuncs are polymorphic and figure out what to do in C when passed a
particular set of input parameter types.  C functions, on the other hand, can
only be run on parameters which match their type signatures.  The task of
defining a universal function is one of describing how different parameter
sequences are mapped from Python array types to C function signatures and back.

At runtime, there are three principle kinds of things used to define a
universal function.

\begin {enumerate}
\item Ufunc 

The universal function is itself a callable Python object.  Ufuncs organize a
collection of Cfuncs to be called based on the actual parameter types seen at
runtime.  The same Ufunc is typically associated with several Cfuncs each of
which handles a unique Ufunc type signature.  Because a Ufunc typically has
more than one C func, it can also be implemented using more than one library
function.

\item Library function

A pre-existing function written in C or Fortran which will ultimately be called
for each element of the ufunc parameter arrays.  

\item Cfunc

Cfuncs are binding objects that map C library functions safely into Python.
It's the job of a Cfunc to interpret typeless pointers corresponding to the
parameter arrays as particular C data types being passed down from the ufunc.
Further, the Cfunc casts array elements from the input type to the Libraray
function parameter type.  This process lets the ufunc implementer describe the
ufunc type signatures which will be processed most efficiently by the
underlying Library function by enabling per-call element-by-element type casts.
Ufunc calling signatures which are not represented directly by a Cfunc result
in blockwise coercion to the closest matching Cfunc, which is slower.

\end {enumerate}

\subsection{Source components of a ufunc}
There are 4 source components required to define numarray ufuncs, one of which
is hand written, two are generated, and the last is assumed to be pre-existing:

\begin {enumerate}
\item Code generation script

The primary source component for defining new universal functions is a Python
script used to generate the other components.  For a standalone set of
functions, putting the code generation directives in setup.py can be done as in
the example numarray/Examples/ufunc/setup_airy.py.  The code generation script
only executes at install time.

\item Extension module

A private extension module is generated which contains a collection of Cfuncs
for the package being created.  The extension module contains a dictionary
mapping from ufuncs/types to Cfuncs.

\item Ufunc init file 

A Python script used at ufunc import time is required to construct Ufunc
objects from Cfuncs.  This code is boilerplate created with the code generation
directive \function{make_stub()}.

\item Library functions

The C functions which are ultimately called by a Ufunc need to be defined
somewhere, typically in a third party C or Fortran library which is linked
to the Extension module.
\end{enumerate}

\subsection{Ufunc code generation}
There are several code generation directives provided by package
numarray.codegenerator which are called at installation time to generate the
Cfunc extension module and Ufunc init file.

\begin{funcdesc}{UfuncModule}{module_name}
The \class{UfuncModule} constructor creates a module object which collects code
which is later output to form the Cfunc extension module.  The name passed to
the constructor defines the name of the Python extension module, not the source
code.
\begin{verbatim}
m = UfuncModule("_na_special")
\end{verbatim}
\end{funcdesc}

\begin{methoddesc}{add_code}{code_string}
The \method{add_code()} method of a \class{UfuncModule} object is used to add
arbitrary code to the module at the point that \method{add_code()} is
called. Here it includes a header file used to define prototypes for the C
library functions which this extension will ultimately call.
\begin{verbatim}
m.add_code('#include "airy.h"')
\end{verbatim}
\end{methoddesc}

\begin{methoddesc}{add_nary_ufunc}{ufunc_name, c_name,
    ufunc_signatures, c_signature, forms=None} 
The \method{add_nary_ufunc()} method declares a Ufunc and relates it to one
library function and a collection of Cfunc bindings for it.  The
\var{signatures} parameter defines which ufunc type signatures receive Cfunc
bindings. Input types which don't match those signature are blockwise coerced
to the best matching signature.  \method{add_nary_ufunc()} can be called for
the same Ufunc name more than once and can thus be used to associate multiple
library functions with the same Ufunc.
\begin{verbatim}
m.add_nary_ufunc(ufunc_name = "airy",
                 c_function  = "airy",    
                 signatures  =["dxdddd",
                               "fxffff"],
                 c_signature = "dxdddd")
\end{verbatim}
\end{methoddesc}

\begin{methoddesc}{generate}{source_filename}
The \method{generate()} method asks the \class{UfuncModule} object to emit the
code for an extension module to the specified \var{source_filename}.
\begin{verbatim}
m.generate("Src/_na_specialmodule.c")
\end{verbatim}
\end{methoddesc}

\begin{funcdesc}{make_stub}{stub_filename, cfunc_extension, add_code=None}
The \function{make_stub()} function is used to generate the boilerplate Python
code which constructs universal functions from a Cfunc extension module at
import time.  \function{make_stub()} accepts a \var{add_code} parameter which
should be a string containing any additional Python code to be injected into
the stub module.  Here \function{make_stub()} creates the init file
``Lib/__init__.py'' associated with the Cfunc extension ``_na_special'' and
includes some extra Python code to define the \function{plot_airy()} function.
\begin{verbatim}
extra_stub_code = '''

import matplotlib.pylab as mpl

def plot_airy(start=-10,stop=10,step=0.1,which=1):
    a = mpl.arange(start, stop, step)
    mpl.plot(a, airy(a)[which])

    b = 1.j*a + a
    ba = airy(b)[which]

    h = mpl.figure(2)
    mpl.plot(b.real, ba.real)

    i = mpl.figure(3)
    mpl.plot(b.imag, ba.imag)
    
    mpl.show()
'''

make_stub("Lib/__init__", "_na_special", add_code=extra_stub_code)
\end{verbatim}

\end{funcdesc}

\subsection{Type signatures and signature ordering}

Type signatures are described using the single character typecodes from
Numeric.  Since the type signature and form of a Cfunc need to be encoded in
its name for later identification, it must be brief.  

\begin{verbatim}
typesignature ::= <inputtypes> + ``x'' + <outputtypes>
inputtypes ::= [<typecode>]+
outputtypes ::= [<typecode>]+
typecode ::= "B" | "1" | "b" | "s" | "w" | "i" | "u" |
             "N" | "U" | "f" | "d" | "F" | "D"
\end{verbatim}

For example,  the type signature corresponding to one Int32 input and one Int16
output is "ixs".

A type signature is a sequence of ordered types.  One signature can be compared
to another by comparing corresponding elements, in left to right order.
Individual elements are ranked using the order from the previous section.  A
ufunc maintains its associated Cfuncs as a sorted sequence and selects the
first Cfunc which is \(>=\) the input type signature;  this defines the notion
of ``best matching''.

\subsection{Forms}

The \method{add_nary_ufunc()} method has a parameter \var{forms} which enables
generation of code with some extra properties.  It specifies the list of
function forms for which dedicated code will be generated.  If you don't
specify \var{forms}, it defaults to a (list of a) single form which specifies
that all inputs and outputs corresponding to the type signature are vectors.
Input vectors are passed by value, output vectors are passed by reference.  The
default form implies that the library function return value, if there is one,
is ignored.  The following Python code shows the default form:

\begin{verbatim}
["v"*n_inputs + "x" + "v"*n_outputs] 
\end{verbatim}

Forms are denoted using a syntax very similar to, and typically symmetric with,
type signatures.

\begin{verbatim}

form ::=  <inputs> "x" <outputs>
inputs ::= ["v"|"s"]*
outputs ::= ["f"]?["v"]* | "A" | "R"

The form character values have different meanings than for type
signatures:

'v'  :   vector,  an array of input or output values
's'  :   scalar,  a non-array input value
'f'  :   function,  the c_function returns a value
'R'  :   reduce,    this binary ufunc needs a reduction method
'A'  :   accumulate this binary ufunc needs an accumulate method
'x'  :   separator  delineates inputs from outputs

\end{verbatim}

So, a form consists of some input codes followed by a lower case "x" followed
by some output codes.  

The form for a C function which takes 4 input values, the last of which is
assumed to be a scalar, returns one value, and fills in 2 additional output
values is:  "vvvsxfvv".

Using "s" to designate scalar parameters is a useful performance
optimization for cases where it is known that only a single value is
passed in from Python to be used in all calls to the c function.  This
prevents the blockwise expansion of the scalar value into a vector.

Use "f" to specify that the C function return value should be kept; it must
always be the first output and will therefore appear as the first element of
the result tuple.

For ufuncs of two input parameters (binary ufuncs), two additional form
characters are possible: A (accumulate) and R (reduce).  Each of these
characters constitutes the *entire* ufunc form, so the form is denoted "R" or
"A".  For these kinds of cfuncs, the type signature always reads \code{<t>x<t>}
where \code{<t>} is one of the type characters.  

One reason for all these codes is so that the many Cfuncs generated for Ufuncs
can be easily named.  The name for the Cfunc which implements \function{add()}
for two Int32 inputs and one Int32 output and where all parameters are arrays
is: "add_iixi_vvxv".  The cfunc name for \method{add.reduce()} with two integer
parameters would be written as "add_ixi_R" and for \method{add.accumulate()}
as "add_ixi_A".

The set of Cfuncs generated is based on the signatures \emph{crossed} with the
forms.  Multiple calls to \method{add_nary_ufunc()} can be used the reduce the
effects of signature/form crossing.

\newpage
\subsection{Ufunc Generation Example}

This section includes code from Examples/ufunc/setup_airy.py in the numarray
source distribution to illustrate how to create a package which defines your
own universal functions.  

This script eventually generates two files: _na_airymodule.c and
__init__.py.  The former defines an extension module which creates
numarray cfuncs, c helpers for the numarray airy() ufunc.  The latter
file includes Python code which automatically constructs numarray
universal functions (ufuncs) from the cfuncs in _na_airymodule.c.

\begin{verbatim}

import distutils, os, sys
from distutils.core import setup
from numarray.codegenerator import UfuncModule, make_stub
from numarray.numarrayext import NumarrayExtension

m = UfuncModule("_na_special")

m.add_code('#include "airy.h"')

m.add_nary_ufunc(ufunc_name = "airy",
                 c_function  = "airy",    
                 signatures  =["dxdddd",
                               "fxffff"],
                 c_signature = "dxdddd")

m.add_nary_ufunc(ufunc_name = "airy",
                 c_function  ="cairy_fake",
                 signatures  =["DxDDDD",
                               "FxFFFF"],
                 c_signature = "DxDDDD")

m.generate("Src/_na_specialmodule.c")

\end{verbatim}

\begin{verbatim}

extra_stub_code = '''
def plot_airy(start=-10,stop=10,step=0.1,which=1):
    import matplotlib.pylab as mpl;

    a = mpl.arange(start, stop, step);
    mpl.plot(a, airy(a)[which]);

    b = 1.j*a + a
    ba = airy(b)[which]

    h = mpl.figure(2)
    mpl.plot(b.real, ba.real)

    i = mpl.figure(3)
    mpl.plot(b.imag, ba.imag)
    
    mpl.show()
'''

make_stub("Lib/__init__", "_na_special", 
          add_code=extra_stub_code)

dist = setup(name = "na_special",
      version = "0.1",
      maintainer = "Todd Miller",
      maintainer_email = "jmiller@stsci.edu",
      description = "airy() universal function for numarray",
      url = "http://www.scipy.org/",
      packages = ["numarray.special"],
      package_dir = { "numarray.special":"Lib" },
      ext_modules = [ NumarrayExtension( 'numarray.special._na_special',
                                         ['Src/_na_specialmodule.c',
                                          'Src/airy.c',
                                          'Src/const.c',
                                          'Src/polevl.c']
                                        )
                     ]
     )

\end{verbatim}

Additional explanatory text is available in
numarray/Examples/ufunc/setup_airy.py.  Scripts used to extract
numarray ufunc specs from the existing Numeric ufunc definitions
in scipy.special are in numarray/Examples/ufunc/RipNumeric as an
example of how to convert existing Numeric code to numarray.



%% Local Variables:
%% mode: LaTeX
%% mode: auto-fill
%% fill-column: 79
%% indent-tabs-mode: nil
%% ispell-dictionary: "american"
%% reftex-fref-is-default: nil
%% TeX-auto-save: t
%% TeX-command-default: "pdfeLaTeX"
%% TeX-master: "numarray"
%% TeX-parse-self: t
%% End:

\chapter{Array Functions}
\label{cha:array-functions}

Most of the useful manipulations on arrays are done with functions. This might
be surprising given Python's object-oriented framework, and that many of these
functions could have been implemented using methods instead. Choosing functions
means that the same procedures can be applied to arbitrary python sequences,
not just to arrays. For example, while \code{transpose([[1,2],[3,4]])} works
just fine, \code{[[1,2],[3,4]].transpose()} does not. This approach also allows
uniformity in interface between functions defined in the numarray Python
system, whether implemented in C or in Python, and functions defined in
extension modules. We've already covered two functions which operate on arrays:
\code{reshape} and \code{resize}.

\begin{funcdesc}{take}{array, indices, axis=0, clipmode=CLIP}
   \label{sec:array-functions:take}
   \label{func:take}
   The function \code{take} is a generalized indexing/slicing of the array.  In 
   its simplest form, it is equivalent to indexing:
\begin{verbatim}
>>> a1 = array([10,20,30,40])
>>> print a1[[1,3]]
[20 40]
>>> print take(a1,[1,3])
[20 40]
\end{verbatim}
   See the description of index
   arrays in the Array Basics section for an alternative to \code{take} 
   and \code{put}. \code{take}
   selects the elements of the array (the first argument) based on the
   indices (the second argument). Unlike slicing, however, the array
   returned by \code{take} has the same rank as the input array.  
   Some 2-D examples:
\begin{verbatim}
>>> print a2
[[ 0  1  2  3  4]
 [ 5  6  7  8  9]
 [10 11 12 13 14]
 [15 16 17 18 19]]
>>> print take(a2, (0,))                 # first row
[[0 1 2 3 4]]
>>> print take(a2, (0,1))                # first and second row
[[0 1 2 3 4]
 [5 6 7 8 9]]
>>> print take (a2, (0, -1))             # index relative to dimension end
[[ 0 1 2 3 4]
[15 16 17 18 19]]
\end{verbatim}
   The optional third argument specifies the axis along which the selection
   occurs, and the default value (as in examples above) is 0, the first
   axis.  If you want a different axis, then you need to specify it:
\begin{verbatim}
>>> print take(a2, (0,), axis=1)         # first column
[[ 0]
 [ 5]
 [10]
 [15]]
>>> print take(a2, (0,1), axis=1)        # first and second column
[[ 0  1]
 [ 5  6]
 [10 11]
 [15 16]]
>>> print take(a2, (0,4), axis=1)        # first and last column
[[ 0  4]
 [ 5  9]
 [10 14]
 [15 19]]
\end{verbatim}
   
   This is considered to be a \var{structural} operation, because its 
   result does
   not depend on the content of the arrays or the result of a computation on
   those contents but uniquely on the structure of the array. Like all such
   structural operations, the default axis is 0 (the first rank). 
   Later in this tutorial, we will see other functions with a default axis 
   of -1.
   
   \function{take} is often used to create multidimensional arrays with the
   indices from a rank-1 array. As in the earlier examples, the shape of the
   array returned by \function{take} is a combination of the shape of its first
   argument and the shape of the array that elements are "taken" from �-- when
   that array is rank-1, the shape of the returned array has the same shape as
   the index sequence. This, as with many other facets of numarray, is best
   understood by experiment.
\begin{verbatim}
>>> x = arange(10) * 100
>>> print x
[  0 100 200 300 400 500 600 700 800 900]
>>> print take(x, [[2,4],[1,2]])
[[200 400]
 [100 200]]
\end{verbatim}
   A typical example of using \function{take} is to replace the grey values in
   an image according to a "translation table."  For example, suppose \code{m51}
   is a 2-D array of type \code{UInt8} containing a greyscale image. We can
   create a table mapping the integers 0--255 to integers 0--255 using a
   "compressive nonlinearity":
\begin{verbatim}
>>> table = (255 - arange(256)**2 / 256).astype(UInt8)
\end{verbatim}
   Then we can perform the take() operation
\begin{verbatim}
>>> m51b = take(table, m51)
\end{verbatim}
The numarray version of \function{take} can also take a sequence as its 
axis value:
\begin{verbatim}
>>> print take(a2, [0,1], [0,1])
1
>>> print take(a2, [0,1], [1,0])
5
\end{verbatim}
In this case, it functions like indexing: a2[0,1] and a2[1,0] respectively.
\end{funcdesc}


\begin{funcdesc}{put}{array, indices, values, axis=0, clipmode=CLIP}
  \label{func:put}
   \function{put} is the opposite of \function{take}. The values of \var{array}
   at the locations specified in \var{indices} are set to the corresponding
   \var{values}.  The \var{array} must be a contiguous array. The \var{indices}
   can be any integer sequence object with values suitable for indexing into
   the flat form of \var{array}.  The \var{values} must be any sequence of
   values that can be converted to the type of \var{a}.
\begin{verbatim}
>>> x = arange(6)
>>> put(x, [2,4], [20,40])
>>> print x
[ 0  1 20  3 40  5]
\end{verbatim}
   Note that the target \var{array} is not required to be one-dimensional.
   Since \var{array} is contiguous and stored in row-major order, the
   \var{indices} can be treated as indexing \var{array}'s elements in storage
   order.  The routine \function{put} is thus equivalent to the following
   (although the loop is in C for speed):
\begin{verbatim}
ind = array(indices, copy=0)
v = array(values, copy=0).astype(a.type())
for i in range(len(ind)): a.flat[i] = v[i]
\end{verbatim}
\end{funcdesc}


\begin{funcdesc}{putmask}{array, mask, values}
   \function{putmask} sets those elements of \var{array} for which 
   \var{mask} is true to the corresponding value in \var{values}. 
   The array \var{array} must be contiguous. The argument \var{mask} 
   must be an integer sequence of the same size (but not necessarily the 
   same shape) as \var{array}. The argument \var{values} will be 
   repeated as necessary; in particular it can be a
   scalar. The array values must be convertible to the type of \var{array}.
\begin{verbatim}
>>> x=arange(5)
>>> putmask(x, [1,0,1,0,1], [10,20,30,40,50])
>>> print x
[10  1 30  3 50]
>>> putmask(x, [1,0,1,0,1], [-1,-2])
>>> print x
[-1  1 -1  3 -1]
\end{verbatim}
   Note how in the last example, the third argument was treated as if it were
   \code{[-1, -2, -1, -2, -1]}.
\end{funcdesc}


\begin{funcdesc}{transpose}{array, axes=None}
   \function{transpose} takes an array \var{array} and returns a new 
   array which corresponds to \var{a} with the order of axes specified 
   by the second argument \var{axes} which is a sequence of dimension 
   indices.  The default is to reverse the order of all axes, i.e. 
   \code{axes=arange(a.rank)[::-1]}.
\begin{verbatim}
>>> a2=arange(6,shape=(2,3)); print a2
[[0 1 2]
 [3 4 5]]
>>> print transpose(a2)  # same as transpose(a2, axes=(1,0))
[[0 3]
 [1 4]
 [2 5]]
>>> a3=arange(24,shape=(2,3,4)); print a3
[[[ 0  1  2  3]
  [ 4  5  6  7]
  [ 8  9 10 11]]

 [[12 13 14 15]
  [16 17 18 19]
  [20 21 22 23]]]
>>> print transpose(a3)  # same as transpose(a3, axes=(2,1,0))
[[[ 0 12]
  [ 4 16]
  [ 8 20]]

 [[ 1 13]
  [ 5 17]
  [ 9 21]]

 [[ 2 14]
  [ 6 18]
  [10 22]]

 [[ 3 15]
  [ 7 19]
  [11 23]]]
>>> print transpose(a3, axes=(1,0,2))
[[[ 0  1  2  3]
  [12 13 14 15]]

 [[ 4  5  6  7]
  [16 17 18 19]]

 [[ 8  9 10 11]
  [20 21 22 23]]]
\end{verbatim}
\end{funcdesc}


\begin{funcdesc}{repeat}{array, repeats, axis=0}
   \function{repeat} takes an array \var{array} and returns an array 
   with each element in the input array repeated as often as indicated by the
   corresponding elements in the second array. It operates along the specified
   axis. So, to stretch an array evenly, one needs the repeats array to contain
   as many instances of the integer scaling factor as the size of the specified
   axis:
\begin{verbatim}
>>> print a
[[0 1 2]
 [3 4 5]]
>>> print repeat(a, 2*ones(a.shape[0]))   # i.e. repeat(a, (2,2)), broadcast 
                   # rules apply, so this is also equivalent to repeat(a, 2)
[[0 1 2]
 [0 1 2]
 [3 4 5]
 [3 4 5]]
>>> print repeat(a, 2*ones(a.shape[1]), 1)  # i.e. repeat(a, (2,2,2), 1), or
                                            # repeat(a, 2, 1)
[[0 0 1 1 2 2]
 [3 3 4 4 5 5]]
>>> print repeat(a, (1,2))
[[0 1 2]
 [3 4 5]
 [3 4 5]]
\end{verbatim}
\end{funcdesc}


\begin{funcdesc}{where}{condition, x, y, out=None}
  \label{func:where}
   The \function{where} function creates an array whose values are those of
   \var{x} at those indices where \var{condition} is true, and those of \var{y}
   otherwise.  The shape of the result is the shape of \var{condition}. The
   type of the result is determined by the types of \var{x} and \var{y}. Either
   \var{x} or \var{y} (or both) can be a scalar, which is then used for all
   appropriate elements of condition.  \var{out} can be used to specify an
   output array.
\begin{verbatim}
>>> where(arange(10) >= 5, 1, 2)
array([2, 2, 2, 2, 2, 1, 1, 1, 1, 1])
\end{verbatim}

   Starting with numarray-0.6, \function{where} supports a one parameter form
   that is equivalent to the \var{nonzero} function but reads better:

\begin{verbatim}
>>> where(arange(10) % 2)
(array([1, 3, 5, 7, 9]),)   # indices where expression is true 
\end{verbatim}
   Note that in this case, the output is a tuple.

   Like \function{nonzero}, the one parameter form of \function{where} can be
   used to do array indexing:

\begin{verbatim}
>>> a = arange(10,20)
>>> a[ where( a % 2 ) ]
array([11, 13, 15, 17, 19])
\end{verbatim}

   Note that for array indices which are boolean arrays, using \function{where}
   is not necessary but is still OK:

\begin{verbatim}
>>> a[(a % 2) != 0]
array([11, 13, 15, 17, 19])
>>> a[where((a%2) != 0)]
array([11, 13, 15, 17, 19])
\end{verbatim}
\end{funcdesc}

\begin{funcdesc}{choose}{selector, population, outarr=None, clipmode=RAISE}
   The function \function{choose} provides a more general mechanism for
   selecting members of a \var{population} based on a \var{selector} array.
   Unlike \function{where}, \function{choose} can select values from more than
   two arrays.  \var{selector} is an array of integers between \constant{0} and
   \constant{n}. The resulting array will have the same shape as
   \var{selector}, with element selected from \code{population=(b0, ..., bn)}
   as indicated by the value of the corresponding element in \var{selector}.
   Assume \var{selector} is an array that you want to "clip" so that no values
   are greater than \constant{100.0}.
\begin{verbatim}
>>> choose(greater(a, 100.0), (a, 100.0))
\end{verbatim}
   Everywhere that \code{greater(a, 100.0)} is false (i.e.\ \constant{0}) this
   will ``choose'' the corresponding value in \var{a}. Everywhere else 
   it will ``choose'' \constant{100.0}.  This works as well with arrays. 
   Try to figure out what the following does:
\begin{verbatim}
>>> ret = choose(greater(a,b), (c,d))
\end{verbatim}
\end{funcdesc}

\begin{funcdesc}{ravel}{array}
   Returns the argument \var{array} as a 1-D array. It is 
   equivalent to \code{reshape(a, (-1,))}. There is a \method{ravel} 
   method which reshapes the array in-place. Unlike \code{a.ravel()}, 
   however, the \function{ravel} function works with non-contiguous arrays.
\begin{verbatim}
>>> a=arange(25)
>>> a.setshape(5,5)
>>> a.transpose()
>>> a.iscontiguous()
0
>>> a
array([[ 0,  5, 10, 15, 20],
  [ 1,  6, 11, 16, 21],
  [ 2,  7, 12, 17, 22],
  [ 3,  8, 13, 18, 23],
  [ 4,  9, 14, 19, 24]])
>>> a.ravel()
Traceback (most recent call last):
...
TypeError: Can't reshape non-contiguous arrays
>>> ravel(a)
array([ 0,  5, 10, 15, 20,  1,  6, 11, 16, 21,  2,  7, 12, 17, 22,  3,
        8, 13, 18, 23,  4,  9, 14, 19, 24])
\end{verbatim}
\end{funcdesc}


\begin{funcdesc}{nonzero}{a}
   \function{nonzero} returns a tuple of arrays containing the indices of the
   elements in \var{a} that are nonzero.

\begin{verbatim}
>>> a = array([-1, 0, 1, 2])
>>> nonzero(a)
(array([0, 2, 3]),)
>>> print a2
[[-1  0  1  2]
 [ 9  0  4  0]]
>>> print nonzero(a2)
(array([0, 0, 0, 1, 1]), array([0, 2, 3, 0, 2]))
\end{verbatim}
\end{funcdesc}

\begin{funcdesc}{compress}{condition, a, axis=0}
  \label{func:compress}
   Returns those elements of a corresponding to those elements of condition
   that are nonzero. \var{condition} must be the same size as the given axis of
   \var{a}.  Alternately, \var{condition} may match \var{a} in shape; in this
   case the result is a 1D array and \var{axis} should not be specified.
\begin{verbatim}
>>> print x
[1 0 6 2 3 4]
>>> print greater(x, 2)
[0 0 1 0 1 1]
>>> print compress(greater(x, 2), x)
[6 3 4]
>>> print a2
[[-1  0  1  2]
 [ 9  0  4  0]]
>>> print compress(a2>1, a2)
[2 9 4]
>>> a = array([[1,2],[3,4]])
>>> print compress([1,0], a, axis = 1)
[[1]
[3]]
>>> print compress([[1,0],[0,1]], a)
[1, 4]
\end{verbatim}
\end{funcdesc}


\begin{funcdesc}{diagonal}{a, offset=0, axis1=0, axis2=1}
   Returns the entries along the diagonal of \var{a} specified by \var{offset}.
   The \var{offset} is relative to the \var{axis2} axis.  This is designed for
   2-D arrays. For arrays of higher ranks, it will return the diagonal of each
   2-D sub-array.  The 2-D array does not have to be square.

   Warning:  in Numeric (and numarray 0.7 or before), there is a bug in 
   the \function{diagonal} function which will give erronous result for 
   arrays of 3-D or higher.
\begin{verbatim}
>>> print x
[[ 0  1  2  3  4]
 [ 5  6  7  8  9]
 [10 11 12 13 14]
 [15 16 17 18 19]
 [20 21 22 23 24]]
>>> print diagonal(x)
[ 0  6 12 18 24]
>>> print diagonal(x, 1)
[ 1  7 13 19]
>>> print diagonal(x, -1)
[ 5 11 17 23]
\end{verbatim}
\end{funcdesc}


\begin{funcdesc}{trace}{a, offset=0, axis1=0, axis2=1}
   Returns the sum of the elements in a along the diagonal specified by offset.

   Warning:  in Numeric (and numarray 0.7 or before), there is a bug in 
   the \function{trace} function which will give erronous result for 
   arrays of 3-D or higher.
\begin{verbatim}
>>> print x
[[ 0  1  2  3  4]
 [ 5  6  7  8  9]
 [10 11 12 13 14]
 [15 16 17 18 19]
 [20 21 22 23 24]]
>>> print trace(x)                      # 0 + 6 + 12 + 18 + 24
60
>>> print trace(x, -1)                  # 5 + 11 + 17 + 23
56
>>> print trace(x, 1)                   # 1 + 7 + 13 + 19
40
\end{verbatim}
\end{funcdesc}


\begin{funcdesc}{searchsorted}{bin, values}
   Called with a rank-1 array sorted in ascending order,
   \function{searchsorted} will return the indices of the positions in 
   \var{bin} where the corresponding \var{values} would fit.
\begin{verbatim}
>>> print bin_boundaries
[ 0.   0.1  0.2  0.3  0.4  0.5  0.6  0.7  0.8  0.9  1. ]
>>> print data
[ 0.31  0.79  0.82  5.  -2.  -0.1 ]
>>> print searchsorted(bin_boundaries, data)
[4 8 9 11 0 0]
\end{verbatim}
   This can be used for example to write a simple histogramming function:
\begin{verbatim}
>>> def histogram(a, bins):
...         # Note that the argument names below are reverse of the 
...         # searchsorted argument names
...         n = searchsorted(sort(a), bins)
...         n = concatenate([n, [len(a)]])
...         return n[1:]-n[:-1]
...
>>> print histogram([0,0,0,0,0,0,0,.33,.33,.33], arange(0,1.0,.1))
[7 0 0 3 0 0 0 0 0 0]
>>> print histogram(sin(arange(0,10,.2)), arange(-1.2, 1.2, .1))
[0 0 4 2 2 2 0 2 1 2 1 3 1 3 1 3 2 3 2 3 4 9 0 0]
\end{verbatim}
\end{funcdesc}


\begin{funcdesc}{sort}{array, axis=-1}
   This function returns an array containing a copy of the data in 
   \var{array}, with the same shape as \var{array}, but with the 
   order of the elements along the specified \var{axis} sorted. The shape 
   of the returned array is the same as \var{array}'s.  Thus, 
   \code{sort(a, 3)} will be an array of the same shape as \var{array}, 
   where the elements of \var{array} have been sorted along the fourth
   axis.
\begin{verbatim}
>>> print data
[[5 0 1 9 8]
 [2 5 8 3 2]
 [8 0 3 7 0]
 [9 6 9 5 0]
 [9 0 9 7 7]]
>>> print sort(data)                    # Axis -1 by default
[[0 1 5 8 9]
 [2 2 3 5 8]
 [0 0 3 7 8]
 [0 5 6 9 9]
 [0 7 7 9 9]]
>>> print sort(data, 0)
[[2 0 1 3 0]
 [5 0 3 5 0]
 [8 0 8 7 2]
 [9 5 9 7 7]
 [9 6 9 9 8]]
\end{verbatim}
\end{funcdesc}


\begin{funcdesc}{argsort}{array, axis=-1}
   \function{argsort} will return the indices of the elements of the array
   needed to produce \code{sort(array)}. In other words, for a 1-D array,
   \code{take(a.flat, argsort(a))} is the same as \code{sort(a)}... but slower.
\begin{verbatim}
>>> print data
[5 0 1 9 8]
>>> print sort(data)
[0 1 5 8 9]
>>> print argsort(data)
[1 2 0 4 3]
>>> print take(data, argsort(data))
[0 1 5 8 9]
\end{verbatim}
\end{funcdesc}


\begin{funcdesc}{argmax}{array, axis=-1}
\end{funcdesc}
\begin{funcdesc}{argmin}{array, axis=-1}
   The \function{argmax} function returns an array (or scalar for a 1D array)
   with the index(es) of the maximum value(s) of its input \var{array} along
   the given \var{axis}. The returned array will have one less dimension than
   \var{array}. \function{argmin} is just like \function{argmax}, except that
   it returns the indices of the minima along the given axis.
\begin{verbatim}
>>> print data
[[9 6 1 3 0]
 [0 0 8 9 1]
 [7 4 5 4 0]
 [5 2 7 7 1]
 [9 9 7 9 7]]
>>> print argmax(data)
[0 3 0 3 1]
>>> print argmax(data, 0)
[4 4 1 4 4]
>>> print argmin(data)
[4 1 4 4 4]
>>> print argmin(data, 0)
[1 1 0 0 2]
\end{verbatim}
\end{funcdesc}

\begin{funcdesc}{fromstring}{datastring, type, shape=None}
   Will return the array formed by the binary data given in 
   \var{datastring}, of the specified \var{type}. This is mainly 
   used for reading binary data to and from files, it can also be used to 
   exchange binary data with other modules that use python strings as 
   storage (e.g.\ PIL). Note that this representation is dependent on the 
   byte order. To find out the byte ordering used, use the 
   \method{isbyteswapped} method described on page 
   \pageref{arraymethod:byteswap}.  If \var{shape} is not specified, the 
   created array will be one dimensional.
\end{funcdesc}

\begin{funcdesc}{fromfile}{file, type, shape=None}
  If \var{file} is a string then it is interpreted as the name of a 
  file which is opened and read.  Otherwise, \var{file} must be a 
  Python file object which is read as a source of binary array data.  
  \function{fromfile} reads directly into the newly created array buffer 
  with no intermediate string, but otherwise is similar to fromstring, 
  treating the contents of the specified file as a binary data string.
\end{funcdesc}

\begin{funcdesc}{dot}{a, b}
   The \function{dot} function returns the dot product of \var{a} and
   \var{b}. This is equivalent to matrix multiply for rank-2 arrays (without
   the transposition).  This function is identical to the
   \function{matrixmultiply} function.
\begin{verbatim}
>>> print a
[1 2]
>>> print b
[10 11]
# kind of like vector inner product with implicit transposition 
>>> print dot(a,b), dot(b,a) 
32 32
>>> print a
[[1 2]
 [5 7]]
>>> print b
[[  0   1]
 [ 10 100]]
>>> print dot(a,b)
[[ 20 201]
 [ 70 705]]
>>> print dot(b,a)
[[  5   7]
 [510 720]]
\end{verbatim}
\end{funcdesc}

\begin{funcdesc}{matrixmultiply}{a, b}
   This function multiplies matrices or matrices and vectors as matrices rather
   than elementwise. This function is identical to \function{dot}.  Compare:
\begin{verbatim}
>>> print a
[[0 1 2]
 [3 4 5]]
>>> print b
[1 2 3]
>>> print a*b
[[ 0  2  6]
 [ 3  8 15]]
>>> print matrixmultiply(a,b)
[ 8 26]
\end{verbatim}
\end{funcdesc}


\begin{funcdesc}{clip}{m, m_min, m_max}
   The clip function creates an array with the same shape and type as 
   \var{m}, but where every entry in \var{m} that is less than 
   \var{m_min} is replaced by \var{m_min}, and every entry greater 
   than \var{m_max} is replaced by \var{m_max}.  Entries within 
   the range \var{[m_min, m_max]} are left unchanged.
\begin{verbatim}
>>> a = arange(9, type=Float32)
>>> print clip(a, 1.5, 7.5)
[1.5 1.5 2. 3. 4. 5. 6. 7. 7.5]
\end{verbatim}
\end{funcdesc}


\begin{funcdesc}{indices}{shape, type=None}
   The indices function returns an array corresponding to the \var{shape} 
   given. The array returned is an array of a new shape which is based on 
   the specified \var{shape}, but has an added dimension of length 
   the number of dimensions in the specified shape.  For example, if 
   \code{shape=(3,4)}, then the shape of the array returned will be
   \code{(2,3,4)} since the length of \code{(3,4)} is \var{2} and if 
   \code{shape=(5,6,7)}, the returned array's shape will be \code{(3,5,6,7)}. 
   The contents of the returned arrays are such that the \var{i}th subarray 
   (along index 0, the first dimension) contains the indices for that axis 
   of the elements in the array.  An example makes things clearer:
\begin{verbatim}
>>> i = indices((4,3))
>>> i.getshape()
(2, 4, 3)
>>> print i[0]
[[0 0 0]
 [1 1 1]
 [2 2 2]
 [3 3 3]]
>>> print i[1]
[[0 1 2]
 [0 1 2]
 [0 1 2]
 [0 1 2]]
\end{verbatim}
   So, \code{i[0]} has an array of the specified shape, and each element in
   that array specifies the index of that position in the subarray for axis 0.
   Similarly, each element in the subarray in \code{i[1]} contains the index of
   that position in the subarray for axis 1.
\end{funcdesc}


\begin{funcdesc}{swapaxes}{array, axis1, axis2}
   Returns a new array which \var{shares} the data of \var{array}, but 
   has the two axes specified by \var{axis1} and \var{axis2} 
   swapped. If \var{array} is of rank 0 or 1, swapaxes simply returns a 
   new reference to \var{array}.
\begin{verbatim}
>>> x = arange(10)
>>> x.setshape((5,2,1))
>>> print x
[[[0]
  [1]]

 [[2]
  [3]]

 [[4]
  [5]]

 [[6]
  [7]]

 [[8]
  [9]]]
>>> y = swapaxes(x, 0, 2)
>>> y.getshape()
(1, 2, 5)
>>> print y
[[[0 2 4 6 8]
 [1 3 5 7 9]]]
\end{verbatim}
\end{funcdesc}


\begin{funcdesc}{concatenate}{arrs, axis=0}
   Returns a new array containing copies of the data contained in all arrays
   of \var{arrs= (a0, a1, ... an)}.  The arrays \var{ai} will be 
   concatenated along the specified \var{axis} (default=0). All 
   arrays \var{ai} must have the same shape along every axis except for 
   the one specified in \var{axis}. To concatenate arrays along a
   newly created axis, you can use \code{array((a0, ..., an))}, as long as all
   arrays have the same shape.
\begin{verbatim}
>>> print x
[[ 0  1  2  3]
 [ 5  6  7  8]
 [10 11 12 13]]
>>> print concatenate((x,x))
[[ 0  1  2  3]
 [ 5  6  7  8]
 [10 11 12 13]
 [ 0  1  2  3]
 [ 5  6  7  8]
 [10 11 12 13]]
>>> print concatenate((x,x), 1)
[[ 0  1  2  3  0  1  2  3]
 [ 5  6  7  8  5  6  7  8]
 [10 11 12 13 10 11 12 13]]
>>> print array((x,x))   # Note: one extra dimension
[[[ 0  1  2  3]
  [ 5  6  7  8]
  [10 11 12 13]]
 [[ 0  1  2  3]
  [ 5  6  7  8]
  [10 11 12 13]]]
>>> print a
[[1 2]]
>>> print b
[[3 4 5]]
>>> print concatenate((a,b),1)
[[1 2 3 4 5]]
>>> print concatenate((a,b),0)
ValueError: _concat array shapes must match except 1st dimension
\end{verbatim}
\end{funcdesc}


\begin{funcdesc}{innerproduct}{a, b}
   \function{innerproduct} produces the inner product of arrays \var{a} and
   \var{b}. It is equivalent to \code{matrixmultiply(a, transpose(b))}.
\end{funcdesc}


\begin{funcdesc}{outerproduct}{a,b}
   \function{outerproduct} produces the outer product of vectors \var{a} and
   \var{b}, that is \code{result[i, j] = a[i] * b[j]}.
\end{funcdesc}


\begin{funcdesc}{array_repr}{a, max_line_width=None, precision=None, supress_small=None}
   See section \ref{TBD} on Textual Representations of arrays.
\end{funcdesc}


\begin{funcdesc}{array_str}{a, max_line_width=None, precision=None, supress_small=None}
   See section \ref{TBD} Textual Representations of arrays.
\begin{verbatim}
>>> print a
[  1.00000000e+00   1.10000000e+00   1.11600000e+00   1.11380000e+00
   1.20000000e-02   1.34560000e-04]
>>> print array_str(a,precision=4,suppress_small=1)
[ 1.      1.1     1.116   1.1138  0.012   0.0001]
>>> print array_str(a,precision=3,suppress_small=1)
[ 1.     1.1    1.116  1.114  0.012  0.   ]
>>> print array_str(a,precision=3)
[  1.000e+00   1.100e+00   1.116e+00   1.114e+00   1.200e-02
   1.346e-04]
\end{verbatim}
\end{funcdesc}


\begin{funcdesc}{resize}{array, shape}
  \label{func:resize}
   The \function{resize} function takes an array and a shape, and returns a new
   array with the specified \var{shape}, and filled with the data in 
   the input \var{array}.  Unlike the \function{reshape} function, the 
   new shape does not have to yield the same size as the original array. 
   If the new size of is less than that of the input \var{array}, the 
   returned array contains the appropriate data from the "beginning" of the 
   old array. If the new size is greater than that of the input array, the 
   data in the input \var{array} is repeated as many times as needed
   to fill the new array.
\begin{verbatim}
>>> x = arange(10)
>>> y = resize(x, (4,2))                # note that 4*2 < 10
>>> print x
[0 1 2 3 4 5 6 7 8 9]
>>> print y
[[0 1]
 [2 3]
 [4 5]
 [6 7]]
>>> print resize(array((0,1)), (5,5))   # note that 5*5 > 2
[[0 1 0 1 0]
 [1 0 1 0 1]
 [0 1 0 1 0]
 [1 0 1 0 1]
 [0 1 0 1 0]]
\end{verbatim}
\end{funcdesc}


\begin{funcdesc}{identity}{n, type=None}
   The identity function returns an \var{n} by \var{n} array 
   where the diagonal elements are 1, and the off-diagonal elements are 0.
\begin{verbatim}
>>> print identity(5)
[[1 0 0 0 0]
 [0 1 0 0 0]
 [0 0 1 0 0]
 [0 0 0 1 0]
 [0 0 0 0 1]]
\end{verbatim}
\end{funcdesc}


\begin{funcdesc}{sum}{a, axis=0}
   The sum function is a synonym for the \method{reduce} method of the
   \function{add} ufunc. It returns the sum of all of the elements in the
   sequence given along the specified axis (first axis by default).
\begin{verbatim}
>>> print x
[[ 0  1  2  3]
 [ 4  5  6  7]
 [ 8  9 10 11]
 [12 13 14 15]
 [16 17 18 19]]
>>> print sum(x)
[40 45 50 55]                           # 0+4+8+12+16, 1+5+9+13+17,
2+6+10+14+18, ...
>>> print sum(x, 1)
[ 6 22 38 54 70]                        # 0+1+2+3, 4+5+6+7, 8+9+10+11, ...
\end{verbatim}
\end{funcdesc}


\begin{funcdesc}{cumsum}{a, axis=0}
   The cumsum function is a synonym for the \method{accumulate} method of the
   \function{add} ufunc.
\end{funcdesc}


\begin{funcdesc}{product}{a, axis=0}
   The product function is a synonym for the \method{reduce} method of the
   \function{multiply} ufunc.
\end{funcdesc}


\begin{funcdesc}{cumproduct}{a, axis=0}
   The cumproduct function is a synonym for the \method{accumulate} method of
   the \function{multiply} ufunc.
\end{funcdesc}


\begin{funcdesc}{alltrue}{a, axis=0}
   The alltrue function is a synonym for the \method{reduce} method of the
   \function{logical_and} ufunc.
\end{funcdesc}


\begin{funcdesc}{sometrue}{a, axis=0}
   The sometrue function is a synonym for the \method{reduce} method of the
   \function{logical_or} ufunc.
\end{funcdesc}


\begin{funcdesc}{all}{a}
   \function{all} is a synonym for the \method{reduce} method of the
   \function{logical_and} ufunc, preceded by a \function{ravel} which converts
   arrays with \(rank>1\) to \(rank=1\).  Thus, \function{all} tests that all
   the elements of a multidimensional array are nonzero.
\end{funcdesc}


\begin{funcdesc}{any}{a}
   The \function{any} function is a synonym for the \method{reduce} method of
   the \function{logical_and} ufunc, preceded by a \function{ravel} which
   converts arrays with \(rank>1\) to \(rank=1\).  Thus, \function{any} tests
   that at least one of the elements of a multidimensional array is nonzero.
\end{funcdesc}


\begin{funcdesc}{allclose}{a, b, rtol=1.e-5, atol=1.e-8}
   This function tests whether or not arrays \var{x} and \var{y} 
   of an integer or real type are equal subject to the given relative and 
   absolute tolerances: \code{rtol, atol}. The formula used is:
   \begin{equation}
      \left| x - y \right| < atol + rtol * \left| y \right|
   \end{equation}
   This means essentially that both elements are small compared to \var{atol}
   or their difference divided by \var{y}'s value is small compared to
   \var{rtol}.
\end{funcdesc}



\begin{seealso}
   \seemodule{numarray.convolve}{The \function{convolve} function is implemented in the
      optional \module{numarray.convolve} package.}%
   \seemodule{numarray.convolve}{The \function{correlation} function is implemented in
      the optional \module{numarray.convolve} package.}%
\end{seealso} 




%% Local Variables:
%% mode: LaTeX
%% mode: auto-fill
%% fill-column: 79
%% indent-tabs-mode: nil
%% ispell-dictionary: "american"
%% reftex-fref-is-default: nil
%% TeX-auto-save: t
%% TeX-command-default: "pdfeLaTeX"
%% TeX-master: "numarray"
%% TeX-parse-self: t
%% End:


\chapter{Array Methods}
\label{cha:array-methods}

As we discussed at the beginning of the last chapter, there are very few array
methods for good reasons, and these all depend on the implementation
details. They're worth knowing, though.

\begin{methoddesc}[numarray]{argmax}{axis=-1}
  \label{arraymethod:argmax}
  The \method{argmax} method returns the index of the largest element in a 1D
  array.  In the case of a multi-dimensional array, it returns and array of
  indices.
\begin{verbatim}
>>> array([1,2,4,3]).argmax()
2
>>> arange(100, shape=(10,10)).argmax()
array([9, 9, 9, 9, 9, 9, 9, 9, 9, 9])
\end{verbatim}
\end{methoddesc}


\begin{methoddesc}[numarray]{argmin}{axis=-1}
  \label{arraymethod:argmin}
  The \method{argmin} method returns the index of the smallest element in a 1D
  array.  In the case of a multi-dimensional array, it returns and array of
  indices.
\end{methoddesc}


\begin{methoddesc}[numarray]{argsort}{axis=-1}
  \label{arraymethod:argsort}
  The \method{argsort} method returns the array of indices which if taken from
  the array using \function{take} would return a sorted copy of the array.  For
  multi-dimensional arrays, \method{argsort} computes the indices for each 1D
  subarray independently and aggregates them all into a single array result;
  The \method{argsort} of a multi-dimensional array does not produce a sorted
  copy of the array when applied directly to it using \function{take}; instead,
  each 1D subarray must be passed to \function{take} independently.
\begin{verbatim}
  >>> array([1,2,4,3]).argsort()
  array([0, 1, 3, 2])
  >>> take([1,2,4,3], argsort([1,2,4,3]))
  array([1, 2, 3, 4])
\end{verbatim}
\end{methoddesc}


\begin{methoddesc}[numarray]{astype}{type}
  \label{arraymethod:astype}
  The \method{astype} method returns a copy of the array converted to the
  specified type.  As with any copy, the new array is aligned, contiguous, and
  in native machine byte order.  If the specified type is the same as current
  type, a copy is \emph{still} made.
\begin{verbatim}
  >>> arange(5).astype('Float64')
  array([ 0.,  1.,  2.,  3.,  4.])
\end{verbatim}
\end{methoddesc}


\begin{methoddesc}[numarray]{byteswap}{}
   \label{arraymethod:byteswap}
   The \method{byteswap} method performs a byte swapping operation on all the
   elements in the array, working inplace (i.e.\ it returns None).
   \method{byteswap} does not affect the array's byte order state variable.
   See \method{togglebyteorder} for changing the array's byte order state
   in addition to or rather than physically swapping bytes.
\begin{verbatim}
>>> print a
[1 2 3]
>>> a.byteswap()
>>> print a
[16777216 33554432 50331648]
\end{verbatim}
\end{methoddesc}


\begin{methoddesc}[numarray]{byteswapped}{}
  \label{arraymethod:byteswapped} 
  The \method{byteswapped} method returns a byteswapped copy of the array.
  \method{byteswapped} does not affect the array's own byte order state
  variable.  The result of \method{byteswapped} is logically in native byte
  order.
\begin{verbatim}
>>> array([1,2,3]).byteswapped()
array([16777216, 33554432, 50331648])
\end{verbatim}
\end{methoddesc}


\begin{methoddesc}[numarray]{conjugate}{}
  \label{arraymethod:conjugate}
   The \method{conjugate} method returns the complex conjugate of an array.
\begin{verbatim}
>>> (arange(3) + 1j).conjugate()
array([ 0.-1.j,  1.-1.j,  2.-1.j])
\end{verbatim}
\end{methoddesc}


\begin{methoddesc}[numarray]{copy}{}
  \label{arraymethod:copy}
   The \method{copy} method returns a copy of an array. When making an
   assignment or taking a slice, a new array object is created and has its own
   attributes, except that the data attribute just points to the data of the
   first array (a "view").  The \method{copy} method is used when it is
   important to obtain an independent copy.  \method{copy} returns arrays which
   are contiguous, aligned, and not byteswapped, i.e. well behaved.
\begin{verbatim}
>>> c = a[3:8:2].copy()
>>> print c.iscontiguous()
1
\end{verbatim}
\end{methoddesc}


\begin{methoddesc}[numarray]{diagonal}{}
  \label{arraymethod:diagonal}
   The \method{diagonal} method returns the diagonal elements of the array,
   those elements where the row and column indices are equal.
\begin{verbatim}
>>> arange(25,shape=(5,5)).diagonal()
array([ 0,  6, 12, 18, 24])
\end{verbatim}
\end{methoddesc}


\begin{methoddesc}[numarray]{info}{}
   \label{arraymethod:info} Calling an array's \method{info}
   method prints out information about the array which is useful for debugging.
\begin{verbatim}
>>> arange(10).info()
class: <class 'numarray.numarraycore.NumArray'>
shape: (10,)
strides: (4,)
byteoffset: 0
bytestride: 4
itemsize: 4
aligned: 1
contiguous: 1
data: <memory at 0x08931d18 with size:0x00000028 held by object 0x3ff91bd8 aliasing object 0x00000000>
byteorder: little
byteswap: 0
type: Int32
\end{verbatim}
\end{methoddesc}


\begin{methoddesc}[numarray]{isaligned}{}
  \label{arraymethod:isaligned} \method{isaligned} returns 1 IFF the buffer
  address for an array modulo the array itemsize is 0.  When the array
  itemsize exceeds 8 (sizeof(double)) aligment is done modulo 8.
\end{methoddesc}


\begin{methoddesc}[numarray]{isbyteswapped}{}
  \label{arraymethod:isbyteswapped} \method{isbyteswapped} returns 1 IFF the 
  array's binary data is not in native machine byte order, possibly because it
  originated on a machine with a different native order.
\end{methoddesc}


\begin{methoddesc}[numarray]{iscontiguous}{}
  \label{arraymethod:iscontiguous} \method{iscontiguous} returns 1 IFF
   an array is C-contiguous and 0 otherwise.  An array is C-contiguous if its
   smallest stride corresponds to the innermost dimension and its other strides
   strictly increase in size from the innermost dimension to the outermost,
   with each stride being the product of the previous inner stride and shape.
   A non-contiguous array can be converted to a contiguous array by the
   \method{copy} method.
\begin{verbatim}
>>> a=arange(25, shape=(5,5))
>>> a
array([[ 0,  1,  2,  3,  4],
       [ 5,  6,  7,  8,  9],
       [10, 11, 12, 13, 14],
       [15, 16, 17, 18, 19],
       [20, 21, 22, 23, 24]])
>>> a.iscontiguous()
1
\end{verbatim}
\end{methoddesc}


\begin{methoddesc}[numarray]{is_c_array}{}
   \label{arraymethod:is-c-array} 
   \method{is_c_array} returns 1 IFF an array is C-contiguous, aligned, and
   not byteswapped, and returns 0 otherwise.
\begin{verbatim}
>>> a=arange(25, shape=(5,5))
>>> a.is_c_array()
1
>>> a.is_f_array()
0
\end{verbatim}
\end{methoddesc}


\begin{methoddesc}[numarray]{is_fortran_contiguous}{}
   \label{arraymethod:is-fortran-contiguous} 
   \method{is_fortran_contiguous} returns 1 IFF an array is Fortran-contiguous
   and 0 otherwise.  An array is Fortran-contiguous if its smallest stride
   corresponds to its outermost dimension and each succesive stride is the
   product of the previous stride and shape element.
\begin{verbatim}
>>> a=arange(25, shape=(5,5))
>>> a.transpose()
>>> a
array([[ 0,  5, 10, 15, 20],
       [ 1,  6, 11, 16, 21],
       [ 2,  7, 12, 17, 22],
       [ 3,  8, 13, 18, 23],
       [ 4,  9, 14, 19, 24]])
>>> a.iscontiguous()
0
>>> a.is_fortran_contiguous()
1
\end{verbatim}
\end{methoddesc}


\begin{methoddesc}[numarray]{is_f_array}{}
   \label{arraymethod:is-f-array} \method{is_f_array} returns 1 IFF
   an array is Fortran-contiguous, aligned, and not byteswapped, and returns 0
   otherwise.
\begin{verbatim}
>>> a=arange(25, shape=(5,5))
>>> a.transpose()
>>> a.is_f_array()
1
>>> a.is_c_array()
0
\end{verbatim}
\end{methoddesc}


\begin{methoddesc}[numarray]{itemsize}{}
  \label{arraymethod:itemsize} The \method{itemsize} method 
  returns the number of bytes used by any one of its elements.
\begin{verbatim}
>>> a = arange(10)
>>> a.itemsize()
4
>>> a = array([1.0])
>>> a.itemsize()
8
>>> a = array([1], type=Complex64)
>>> a.itemsize()
16
\end{verbatim}
\end{methoddesc}


\begin{methoddesc}[numarray]{max}{}
  \label{arraymethod:max}
  The \method{max} method returns the largest element in an array.
\begin{verbatim}
>>> arange(100, shape=(10,10)).max()
99
\end{verbatim}
\end{methoddesc}
\begin{methoddesc}[numarray]{mean}{}
  \label{arraymethod:mean}
  The \method{mean} method returns the average of all elements in an array.
\begin{verbatim}
>>> arange(10).mean() 4.5
\end{verbatim}
\end{methoddesc}
\begin{methoddesc}[numarray]{min}{}
  \label{arraymethod:min}
  The \method{min} method returns the smallest element in an array.
\begin{verbatim}
>>> arange(10).min()
0
\end{verbatim}
\end{methoddesc}


\begin{methoddesc}[numarray]{nelements}{}
  \label{arraymethod:nelements}
  \method{nelements} returns the total number of elements in this array.
  Synonymous with \method{size}.
\begin{verbatim}
>>> arange(100).nelements()
100
\end{verbatim}
\end{methoddesc}


\begin{methoddesc}[numarray]{new}{type=None}
  \label{arraymethod:new}
   \method{new} returns a new array of the specified type with the same shape
   as this array.  The new array is uninitialized.
\end{methoddesc}


\begin{methoddesc}[numarray]{nonzero}{axis=-1}
  \label{arraymethod:nonzero}
   \method{nonzero} returns a tuple of arrays containing the indices of the
   elements that are nonzero.
\begin{verbatim}
>>> arange(5).nonzero()
(array([1, 2, 3, 4]),)
>>> b = arange(9, shape=(3,3)) % 2; b
array([[0, 1, 0],
       [1, 0, 1],
       [0, 1, 0]])
>>>b.nonzero()
(array([0, 1, 1, 2]), array([1, 0, 2, 1]))
\end{verbatim}
\end{methoddesc}


\begin{methoddesc}[numarray]{repeat}{r, axis=0}
  \label{arraymethod:repeat}
   The \method{repeat} method returns a new array with each element self[i]
   (along the specified axis) repeated r[i] times.
\begin{verbatim}
>>> a=arange(25, shape=(5,5))
>>> a
array([[ 0,  1,  2,  3,  4],
       [ 5,  6,  7,  8,  9],
       [10, 11, 12, 13, 14],
       [15, 16, 17, 18, 19],
       [20, 21, 22, 23, 24]])
>>> a.repeat(arange(5)%2*2)
array([[ 5,  6,  7,  8,  9],
       [ 5,  6,  7,  8,  9],
       [15, 16, 17, 18, 19],
       [15, 16, 17, 18, 19]])
\end{verbatim}
\end{methoddesc}


\begin{methoddesc}[numarray]{resize}{shape}
  \label{arraymethod:resize}
   \method{resize} shrinks/grows the array to new \var{shape}, possibly
    replacing the underlying buffer object.
\begin{verbatim}
>>> a = array([0, 1, 2, 3])
>>> a.resize(10)
array([0, 1, 2, 3, 0, 1, 2, 3, 0, 1])
\end{verbatim}
\end{methoddesc}


\begin{methoddesc}[numarray]{size}{}
  \label{arraymethod:size}
  \method{size} returns the total number of elements in this array.
  Synonymous with \method{nelements}.
\begin{verbatim}
>>> arange(100).size()
100
\end{verbatim}
\end{methoddesc}


\begin{methoddesc}[numarray]{type}{}
  \label{arraymethod:type}
   The \method{type} method returns the type of the array it is applied to.
   While we've been talking about them as Float32, Int16, etc., it is important
   to note that they are not character strings, they are instances of
   NumericType classes. 
\begin{verbatim}
>>> a = array([1,2,3])
>>> a.type()
Int32
>>> a = array([1], type=Complex64)
>>> a.type()
Complex64
\end{verbatim}
\end{methoddesc}


\begin{methoddesc}[numarray]{typecode}{}
  \label{arraymethod:typecode}
   The \method{typecode} method returns the typecode character of the array it
   is applied to.  \method{typecode} exists for backward compatibility with
   Numeric but the \method{type} method is preferred.
\begin{verbatim}
>>> a = array([1,2,3])
>>> a.typecode()
'l'
>>> a = array([1], type=Complex64)
>>> a.typecode()
'D'
\end{verbatim}
\end{methoddesc}


\begin{methoddesc}[numarray]{tofile}{file}
  \label{arraymethod:tofile}
  The \method{tofile} method writes the binary data of the array into
  \constant{file}.  If \constant{file} is a Python string, it is interpreted 
  as the name of a file to be created.  Otherwise, \constant{file} must be 
  Python file object to which the data will be written.  
\begin{verbatim}
>>> a = arange(65,100)
>>> a.tofile('test.dat')   # writes a's binary data to file 'test.dat'.
>>> f = open('test2.dat', 'w')
>>> a.tofile(f)            # writes a's binary data to file 'test2.dat'
\end{verbatim}
   Note that the binary representation of array data depends on the platform,
   with some platforms being little endian (sys.byteorder == 'little') and
   others being big endian.  The byte order of the array data is \emph{not}
   recorded in the file, nor are the array's shape and type.
\end{methoddesc}


\begin{methoddesc}[numarray]{tolist}{}
  \label{arraymethod:tolist}
   Calling an array's \method{tolist} method returns a hierarchical python list
   version of the same array:
\begin{verbatim}
>>> print a
[[65 66 67 68 69 70 71]
 [72 73 74 75 76 77 78]
 [79 80 81 82 83 84 85]
 [86 87 88 89 90 91 92]
 [93 94 95 96 97 98 99]]
>>> print a.tolist()
[[65, 66, 67, 68, 69, 70, 71], [72, 73, 74, 75, 76, 77, 78], [79, 80,
81, 82, 83, 84, 85], [86, 87, 88, 89, 90, 91, 92], [93, 94, 95, 96, 97,
98, 99]]
\end{verbatim}
\end{methoddesc}


\begin{methoddesc}[numarray]{tostring}{}
  \label{arraymethod:tostring}
   The \method{tostring} method returns a string representation of the 
   array data.
\begin{verbatim}
>>> a = arange(65,70)
>>> a.tostring()
'A\x00\x00\x00B\x00\x00\x00C\x00\x00\x00D\x00\x00\x00E\x00\x00\x00'
\end{verbatim}
Note that the arangement of the printable characters and interspersed NULL
characters is dependent on machine architecture.  The layout shown here is
for little endian platform.
\end{methoddesc}


\begin{methoddesc}[numarray]{transpose}{axis=-1}
  \label{arraymethod:transpose}
  \method{transpose} re-shapes the array by permuting it's dimensions
  as specified by 'axes'.  If 'axes' is none, \method{transpose}
  reverses the array's dimensions.  \method{transpose} operates
  in-place and returns None.
\begin{verbatim}
>>> a = arange(9, shape=(3,3))
>>> a.transpose()
>>> a
array([[0, 3, 6],
       [1, 4, 7],
       [2, 5, 8]])
\end{verbatim}
\end{methoddesc}


\begin{methoddesc}[numarray]{stddev}{}
  \label{arraymethod:stddev}
  The \method{stddev} method returns the standard deviation of all elements in
  an array.
\begin{verbatim}
>>> arange(10).stddev()
3.0276503540974917
\end{verbatim}
\end{methoddesc}


\begin{methoddesc}[numarray]{sum}{}
  \label{arraymethod:sum}
  The \method{sum} method returns the sum of all elements in an array.
\begin{verbatim}
>>> arange(10).sum()
45
\end{verbatim}
\end{methoddesc}


\begin{methoddesc}[numarray]{swapaxes}{axis1, axis2}
  \label{arraymethod:swapaxes}
  The \method{swapaxes} method adjusts the strides of an array so that
  the two specified axes appear to be swapped.  \method{swapaxes} operates
  in place and returns None.
\begin{verbatim}
>>> a = arange(25, shape=(5,5))
>>> a.swapaxes(0,1)
>>> a
array([[ 0,  5, 10, 15, 20],
       [ 1,  6, 11, 16, 21],
       [ 2,  7, 12, 17, 22],
       [ 3,  8, 13, 18, 23],
       [ 4,  9, 14, 19, 24]])
\end{verbatim}
\end{methoddesc}


\begin{methoddesc}[numarray]{togglebyteorder}{}
  \label{arraymethod:togglebyteorder}
  The \method{togglebyteorder} method adjusts the byte order state 
  variable for an array, with ``little'' being replaced by ``big'' and ``big''
  being replaced by ``little''.  \method{togglebyteorder} just reinterprets
  the existing data, it does not actually rearrange bytes.
\begin{verbatim}
>>> a = arange(4)
>>> a.togglebyteorder()
>>> a
array([       0, 16777216, 33554432, 50331648])
\end{verbatim}
\end{methoddesc}

\begin{methoddesc}[numarray]{trace}{}
  \label{arraymethod:togglebyteorder}
  The \method{trace} method returns the sum of the diagonal elements
  of an array.
\begin{verbatim}
>>> a = arange(25, shape=(5,5))
>>> a.trace()
60
\end{verbatim}
\end{methoddesc}


\begin{methoddesc}[numarray]{view}{}
  \label{arraymethod:view} The \method{view} method returns a new
  state object for an array but does not actually copy the array's
  data; views are used to reinterpret an existing data buffer by 
  changing the array's properties.
\begin{verbatim}
>>> a = arange(4)
>>> b = a.view()
>>> b.shape = (2,2)
>>> a
array([0, 1, 2, 3])
>>> b
array([[0, 1],
       [2, 3]])
>>> a is b
False
>>> a._data is b._data
True
\end{verbatim}
\end{methoddesc}


When using Python 2.2 or later, there are four public attributes which
correspond to those of Numeric type objects. These are \member{shape},
\member{flat}, \member{real}, and \member{imag} (or \member{imaginary}). The
following methods are used to implement and provide an alternative to using
these attributes.


\begin{methoddesc}[numarray]{getshape}{}
\end{methoddesc}
\begin{methoddesc}[numarray]{setshape}{}
   The \method{getshape} method returns the tuple that gives the shape of the
   array.  \method{setshape} assigns its argument (a tuple) to the internal
   attribute which defines the array shape. When using Python 2.2 or later, the
   \member{shape} attribute can be accessed or assigned to, which is equivalent
   to using these methods.
\begin{verbatim}
>>> a = arange(12)
>>> a.setshape((3,4))
>>> print a.getshape()
(3, 4)
>>> print a
[[ 0  1  2  3]
 [ 4  5  6  7]
 [ 8  9 10 11]]
\end{verbatim}
\end{methoddesc}


\begin{methoddesc}[numarray]{getflat}{}
   The \method{getflat} method is equivalent to using the \member{flat}
   attribute of Numeric. For compatibility with Numeric, there is no
   \method{setflat} method, although the attribute can in fact be set using
   \method{setshape}.
\begin{verbatim}
>>> print a
[[ 0  1  2  3]
 [ 4  5  6  7]
 [ 8  9 10 11]]
>>> print a.getflat()
[ 0  1  2  3  4  5  6  7  8  9 10 11]
\end{verbatim}
\end{methoddesc}


\begin{methoddesc}[numarray]{getreal}{}
\end{methoddesc}
\begin{methoddesc}[numarray]{setreal}{}
   The \method{getreal} and \method{setreal} methods can be used to access or
   assign to the real part of an array containing imaginary elements.
\end{methoddesc}


\begin{methoddesc}[numarray]{getimag}{}
\end{methoddesc}
\begin{methoddesc}[numarray]{getimaginary}{}
\end{methoddesc}
\begin{methoddesc}[numarray]{setimag}{}
\end{methoddesc}
\begin{methoddesc}[numarray]{setimaginary}{}
   The \method{getimag} and \method{setimag} methods can be used to access or
   assign to the imaginary part of an array containing imaginary elements.
   \method{getimaginary} is equivalent to \method{getimag}, and
   \method{setimaginary} is equivalent to \method{setimag}.
\end{methoddesc}

%% Local Variables:
%% mode: LaTeX
%% mode: auto-fill
%% fill-column: 79
%% indent-tabs-mode: nil
%% ispell-dictionary: "american"
%% reftex-fref-is-default: nil
%% TeX-auto-save: t
%% TeX-command-default: "pdfeLaTeX"
%% TeX-master: "numarray"
%% TeX-parse-self: t
%% End:

\chapter{Array Attributes}
\label{cha:array-attributes}

There are four public array attributes; however, they are only available 
in Python 2.2 or later. There are array methods that may be used instead. The
attributes are \code{shape, flat, real,} and \code{imaginary}.


\begin{memberdesc}[numarray]{shape}
   Accessing the \member{shape} attribute is equivalent to calling the
   \method{getshape} method; it returns the shape tuple.  Assigning a value to
   the shape attribute is equivalent to calling the \method{setshape} method.
\begin{verbatim}
>>> print a
[[0 1 2]
 [3 4 5]
 [6 7 8]]
>>> print a.shape
(3,3)
>>> a.shape = ((9,))
>>> print a.shape
(9,)
\end{verbatim}
\end{memberdesc}


\begin{memberdesc}[numarray]{flat}
   \label{mem:numarray:flat}
   Accessing the flat attribute of an array returns the flattened, or
   \method{ravel}ed version of that array, without having to do a function
   call.  This is equivalent to calling the \method{getflat} method. The
   returned array has the same number of elements as the input array, but it is
   of rank-1. One cannot set the flat attribute of an array, but one can use
   the indexing and slicing notations to modify the contents of the array:
\begin{verbatim}
>> print a
[[0 1 2]
 [3 4 5]
 [6 7 8]]
>> print a.flat
0 1 2 3 4 5 6 7 8]
>> a.flat[4] = 100
>> print a
[[  0   1   2]
 [  3 100   5]
 [  6   7   8]]
>> a.flat = arange(9,18)
>> print a
[[ 9 10 11]
 [12 13 14]
 [15 16 17]]
\end{verbatim}
\end{memberdesc}


\begin{memberdesc}[numarray]{real}
\end{memberdesc}
\begin{memberdesc}[numarray]{imag}
\end{memberdesc}
\begin{memberdesc}[numarray]{imaginary}
   These attributes exist only for complex arrays. They return respectively
   arrays filled with the real and imaginary parts of the elements. The
   equivalent methods for getting and setting these values are
   \method{getreal}, \method{setreal}, \method{getimag}, and \method{setimag}.
   \method{getimaginary} and \method{setimaginary} are synonyms for
   \method{getimag} and \method{setimag} respectively, and \method{.imag} is a
   synonym for \method{.imaginary}.  The arrays returned are not contiguous
   (except for arrays of length 1, which are always contiguous).
   The attributes \member{real}, \member{imag}, and \member{imaginary} are 
   modifiable:
\begin{verbatim}
>>> print x
[ 0.             +1.j               0.84147098+0.54030231j 0.90929743-0.41614684j]
>>> print x.real
[ 0.          0.84147098  0.90929743]
>>> print x.imag
[ 1.          0.54030231 -0.41614684]
>>> x.imag = arange(3)
>>> print x
[ 0.        +0.j  0.84147098+1.j  0.90929743+2.j]
>>> x = reshape(arange(10), (2,5)) + 0j # make complex array
>>> print x
[[ 0.+0.j  1.+0.j  2.+0.j  3.+0.j  4.+0.j]
 [ 5.+0.j  6.+0.j  7.+0.j  8.+0.j  9.+0.j]]
>>> print x.real
[[ 0.  1.  2.  3.  4.]
 [ 5.  6.  7.  8.  9.]]
>>> print x.type(), x.real.type()
Complex64 Float64
>>> print x.itemsize(), x.imag.itemsize()
16 8
\end{verbatim}
\end{memberdesc}


%% Local Variables:
%% mode: LaTeX
%% mode: auto-fill
%% fill-column: 79
%% indent-tabs-mode: nil
%% ispell-dictionary: "american"
%% reftex-fref-is-default: nil
%% TeX-auto-save: t
%% TeX-command-default: "pdfeLaTeX"
%% TeX-master: "numarray"
%% TeX-parse-self: t
%% End:

\chapter{Character Array}
\label{cha:character-array}
\declaremodule{extension}{numarray.strings}
\index{character array}
\index{string array}

\section{Introduction}
\label{sec:chararray-intro}

\code{numarray}, like \code{Numeric}, has support for arrays of character data
(provided by the \code{numarray.strings} module) in addition to arrays of
numbers.  The support for character arrays in \code{Numeric} is relatively
limited, restricted to arrays of single characters.  In contrast,
\code{numarray} supports arrays of fixed length strings.  As an additional
enhancement, the \code{numarray} design supports interleaving arrays of
characters with arrays of numbers, with both occupying the same memory buffer.
This provides basic infrastructure for building the arrays of heterogenous
records as provided by \code{numarray.records} (see chapter
\ref{cha:record-array}).  Currently, neither \code{Numeric} nor \code{numarray}
provides support for unicode.

Each character array is a \index{CharArray} \code{CharArray} object in the
\code{numarray.strings} module.  The easiest way to construct a character array
is to use the \code{numarray.strings.array()} function.  For example:

\begin{verbatim}
  >>> import numarray.strings as str
  >>> s = str.array(['Smith', 'Johnson', 'Williams', 'Miller'])
  >>> print s
  ['Smith', 'Johnson', 'Williams', 'Miller']
  >>> s.itemsize()
  8
\end{verbatim}
In this example, this string array has 4 elements.  The maximum string length
is automatically determined from the data.  In this case, the created array will
support fixed length strings of 8 characters (since the longest name is 8
characters long).

The character array is just like an array in numarray, except that now each
element is conceptually a Python string rather than a number.  We can do the
usual indexing and slicing:

\begin{verbatim}
  >>> print s[0]
  'Smith'
  >>> print s[:2]
  ['Smith', 'Johnson']
  >>> s[:2] = 'changed'
  >>> print s
  ['changed', 'changed', 'Williams', 'Miller']
\end{verbatim}

\section{Character array stripping, padding, and truncation}
\label{sec:chararray-clip-pad-truncate}
CharArrays are designed to store fixed length strings of visible ASCII text.
You may have noticed that although a \code{CharArray} stores fixed length
strings, it displays variable length strings.  This is a result of the
stripping and padding policies of the CharArray class.  

When an element of a \code{CharArray} is fetched trailing whitespace is
stripped off.  The sole exception to this rule is that a single whitespace is
never stripped down to the empty string.  \code{numarray.strings} defines
whitespace as an ASCII space, formfeed, newline, carriage return, tab, or
vertical tab.

When a string is assigned to a \code{CharArray}, the string is considered
terminated by the first of any NULL characters it contains and is padded with
spaces to the full length of the \code{CharArray} itemsize.  Thus, the memory
image of a \code{CharArray} element does not include anything at or after the
first NULL in an assigned string; instead, there are spaces, and no terminating
NULL character at all.

When a string which is longer than the \code{itemsize()} is assigned to a
\code{CharArray}, it is silently truncated.

The \code{RawCharArray} baseclass of \code{CharArray} implements transparent
\code{strip()} and \code{pad()} methods, enabling the storage and retrieval of
arbitrary ASCII values within array elements.  For \code{RawCharArray}, all
array elements are identical in percieved length.  Alternate
stripping and padding policies can be implemented by subclassing
\code{CharArray} or \code{RawCharArray}.

\section{Character array functions}
\label{sec:chararray-func}
\begin{funcdesc}{array}{buffer=None, itemsize=None, shape=None, byteoffset=0,
    bytestride=None, kind=CharArray}
\label{func:str.array}
   The function \code{array} is, for most practical purposes, all a user needs 
   to know to construct a character array.

   The first argument, \code{buffer}, may be any one of the following:

   (1) \code{None} (default).  The constructor will allocate a writeable memory
   buffer which will be uninitialized.  The user must assign valid data before
   trying to read the contents or before writing the character array to a disk
   file.
   
   (2) a Python string containing binary data.  For example:
\begin{verbatim}
     >>> print str.array('abcdefg'*10, itemsize=10)
     ['abcdefgabc', 'defgabcdef', 'gabcdefgab', 'cdefgabcde', 'fgabcdefga',
      'bcdefgabcd', 'efgabcdefg']
\end{verbatim}
   
   (3) a Python file object for an open file.  The data will be copied from 
   the file, starting at the current position of the read pointer.
   
   (4) a character array.  This results in a deep copy of the input character
   array; any other arguments to \code{array()} will be silently ignored.

\begin{verbatim}
     >>> print str.array(s)
     ['abcdefgabc', 'defgabcdef', 'gabcdefgab', 'cdefgabcde', 'fgabcdefga', 
      'bcdefgabcd', 'efgabcdefg']
\end{verbatim}
   
   (5) a nested sequence of strings.  The sequence nesting implies the
   shape of the string array unless shape is specified.

\begin{verbatim}
     >>> print str.array([['Smith', 'Johnson'], ['Williams', 'Miller']])
     [['Smith', 'Johnson'],
      ['Williams', 'Miller']]
\end{verbatim}

   \code{itemsize} can be used to increase or decrease the fixed size of an
   array element relative to the natural itemsize implied by any literal data
   specified by the \code{buffer} parameter.

\begin{verbatim}
     >>> print str.array([['Smith', 'Johnson'], ['Williams', 'Miller']], 
                         itemsize=2)
     [['Sm', 'Jo'],
      ['Wi', 'Mi']])
     >>> print str.array([['Smith', 'Johnson'], ['Williams', 'Miller']], 
                         itemsize=20)
     [['Smith', 'Johnson'],
      ['Williams', 'Miller']]
\end{verbatim}
   
   \code{shape} is the shape of the character array.  It can be an integer, in
   which case it is equivalent to the number of \var{rows} in a table.  It can
   also be a tuple implying the character array is an N-D array with fixed
   length strings as its elements. \code{shape} should be consistent with
   the number of elements implied by the data buffer and itemsize.

   \code{byteoffset} indicates an offset, specified in bytes, from the start
   of the array buffer to where the array data actually begins.  
   \code{byteoffset} enables the character array to be offset from the
   beginning of a table record.  This is mainly useful for implementing
   record arrays.

   \code{bytestride} indicates the separation, specified in bytes, between
   successive elements in the last dimension of the character array.
   \code{bytestride} is used in the implementation of record arrays to space
   character array elements with the size of the total record rather than the
   size of a single string.
   
   \code{kind} is used to specify the class of the created array, and should be
   \code{RawCharArray}, \code{CharArray}, or a subclass of either.
\end{funcdesc}
   
\begin{funcdesc}{num2char}{n, format, itemsize=32}
\label{func:str.num2char}
\code{num2char} formats the numarray \code{n} using the Python string format
\code{format} and stores the result in a character array with the specified
\code{itemsize}
\begin{verbatim}
     >>> num2char(num.arange(0.0,5), '%2.2f')
     CharArray(['0.00', '1.00', '2.00', '3.00', '4.00'])
\end{verbatim}
\end{funcdesc}

\section{Character array methods}
\label{sec:recarray-methods}
CharArray object has these public methods:

\begin{methoddesc}[RawCharArray]{tolist}{}
  \code{tolist()} returns a nested list of strings corresponding to all the
  elements in the array.
\end{methoddesc}
\begin{methoddesc}[RawCharArray]{copy}{}
  \code{copy()} returns a deep copy of the character array.
\end{methoddesc}
\begin{methoddesc}[RawCharArray]{raw}{}
  \code{raw()} returns the corresponding \code{RawCharArray} view.
\begin{verbatim}
     >>> c=str.array(["this","that","another"])
     >>> c.raw()
     RawCharArray(['this   ', 'that   ', 'another'])
\end{verbatim}
\end{methoddesc}
\begin{methoddesc}{resized}{n, fill=' '}
  \code{resized(n)} returns a copy of the array, resized so that each element
  is of length \code{n} characters.  Extra characters are filled with value
  \code{fill}.  Caution: do not confuse this method with \code{resize()} which
  changes the number of elements rather than the size of each element.
\begin{verbatim}
     >>> c = str.array(["this","that","another"])
     >>> c.itemsize()
     7
     >>> d = c.resized(20)
     >>> print d
     ['this', 'that', 'another']
     >>> d.itemsize()
     20
\end{verbatim}
\end{methoddesc}
\begin{methoddesc}[RawCharArray]{concatenate}{other}
  \code{concatenate(other)} returns a new array which corresponds to the
  element by element concatenation of \code{other} to \code{self}.  The
  addition operator is also overloaded to perform concatenation.
\begin{verbatim}
     >>> print map(str, range(3)) + array(["this","that","another one"])
     ['0this', '1that', '2another one']
     >>> print "prefix with trailing whitespace   " + array(["."])
     ['prefix with trailing whitespace   .']
\end{verbatim}
\end{methoddesc}
\begin{methoddesc}[RawCharArray]{sort}{}
  \code{sort} modifies the \code{CharArray} inplace so that its elements are in
  sorted order. \code{sort} only works for 1D character arrays.  Like the
  \code{sort()} for the Python list, \code{CharArray.sort()} returns nothing.
\begin{verbatim}
     >>> a=str.array(["other","this","that","another"])
     >>> a.sort()
     >>> print a
     ['another', 'other', 'that', 'this']
\end{verbatim}
\end{methoddesc}
\begin{methoddesc}[RawCharArray]{argsort}{}
   \code{argsort} returns a numarray corresponding to the permutation which will
   put the character array \code{self} into sorted order.  \code{argsort} only
   works for 1D character arrays.
\begin{verbatim}
     >>> a=str.array(["other","that","this","another"])
     >>> a.argsort()
     array([3, 0, 1, 2])
     >>> print a[ a.argsort ] 
     ['another', 'other', 'that', 'this']
\end{verbatim}
\end{methoddesc}
\begin{methoddesc}[RawCharArray]{amap}{f}
  \code{amap} applies the function \code{f} to every element of \code{self} and
  returns the nested list of the results.  The function \code{f} should operate
  on a single string and may return any Python value.
\end{methoddesc}
\begin{verbatim}
     >>> c = str.array(['this','that','another'])
     >>> print c.amap(lambda x: x[-2:])
     ['is', 'at', 'er']
\end{verbatim}
\begin{methoddesc}[RawCharArray]{match}{pattern, flags=0}
  \code{match} uses Python regular expression matching over all elements of a
  character array and returns a tuple of numarrays corresponding to the indices
  of \code{self} where the pattern matches. \code{flags} are passed directly to
  the Python pattern matcher defined in the \code{re} module of the standard
  library.
\begin{verbatim}
     >>> a=str.array([["wo","what"],["wen","erewh"]])
     >>> print a.match("wh[aebd]")
     (array([0]), array([1]))
     >>> print a[ a.match("wh[aebd]") ]
     ['what']
\end{verbatim}
\end{methoddesc}
\begin{methoddesc}[RawCharArray]{search}{pattern,flags=0}
  \code{search} uses Python regular expression searching over all elements of a
  character array and returns a tuple of numarrays corresponding to the indices
  of \code{self} where the pattern was found. \code{flags} are passed directly
  to the Python pattern \code{search} method defined in the \code{re} module of
  the standard library.  \code{flags} should be an or'ed combination (use the
  $\vert$ operator) of the following \code{re} variables: \code{IGNORECASE},
  \code{LOCALE}, \code{MULTILINE}, \code{DOTALL}, \code{VERBOSE}.  See the
  \code{re} module documentation for more details.
\end{methoddesc}
\begin{methoddesc}[RawCharArray]{sub}{pattern,replacement,flags=0,count=0}
  \code{sub} performs Python regular expression pattern substitution
  to all elements of a character array. \code{flags} and \code{count} work
  as they do for \code{re.sub()}.
\begin{verbatim}
     >>> a=str.array([["who","what"],["when","where"]])
     >>> print a.sub("wh", "ph")
     [['pho', 'phat'],
      ['phen', 'phere']])
\end{verbatim}
\end{methoddesc}
\begin{methoddesc}[RawCharArray]{grep}{pattern, flags=0}
  \code{grep} is intended to be used interactively to search a \code{CharArray}
  for the array of strings which match the given \code{pattern}.
  \code{pattern} should be a Python regular expression (see the \code{re}
  module in the Python standard library, which can be as simple as a string
  constant as shown below.
\begin{verbatim}
     >>> a=str.array([["who","what"],["when","where"]])
     >>> print a.grep("whe")
     ['when', 'where']
\end{verbatim}
\end{methoddesc}
\begin{methoddesc}[RawCharArray]{eval}{}
  \code{eval} executes the Python eval function on each element of a character
  array and returns the resulting numarray.  \code{eval} is intended for use
  converting character arrays to the corresponding numeric arrays.  An
  exception is raised if any string element fails to evaluate.
\begin{verbatim}
     >>> print str.array([["1","2"],["3","4."]]).eval()
     [[1., 2.],
      [3., 4.]]
\end{verbatim}
\end{methoddesc}
\begin{methoddesc}[RawCharArray]{maxLen}{}
  \code{maxLen} returns the minimum element length required to store the
  stripped elements of the array \code{self}.
\begin{verbatim}
     >>> print str.array(["this","there"], itemsize=20).maxLen()
     5
\end{verbatim}
\end{methoddesc}
\begin{methoddesc}[RawCharArray]{truncated}{}
  \code{truncated} returns an array corresponding to \code{self} resized
  so that it uses a minimum amount of storage.
\begin{verbatim}
     >>> a = str.array(["this  ","that"])
     >>> print a.itemsize()
     6
     >>> print a.truncated().itemsize()
     4
\end{verbatim}
\end{methoddesc}
\begin{methoddesc}[RawCharArray]{count}{s}
  \code{count} counts the occurences of string \code{s} in array \code{self}.
\begin{verbatim}
     >>> print array(["this","that","another","this"]).count("this")
     2
\end{verbatim}
\end{methoddesc}
\begin{methoddesc}[RawCharArray]{info}{}
   This will display key attributes of the character array.
\end{methoddesc}

%% Local Variables:
%% mode: LaTeX
%% mode: auto-fill
%% fill-column: 79
%% indent-tabs-mode: nil
%% ispell-dictionary: "american"
%% reftex-fref-is-default: nil
%% TeX-auto-save: t
%% TeX-command-default: "pdfeLaTeX"
%% TeX-master: "numarray"
%% TeX-parse-self: t
%% End:

\chapter{Record Array}
\label{cha:record-array}
\declaremodule{extension}{numarray.records}
\index{record array}

\section{Introduction}
\label{sec:recarray-intro}
One of the enhancements of \code{numarray} over \code{Numeric} is its 
support for record arrays, i.e. arrays with heterogeneous data types: 
for example, tabulated data where each field (or \var{column}) has the 
same data type but different fields may not.

Each record array is a \index{RecArray} \code{RecArray} object in the 
\code{numarray.records} module.  Most of 
the time, the easiest way to construct a record array is to use the 
\code{array()} function in the \code{numarray.records} module.  For example:
\begin{verbatim}
>>> import numarray.records as rec
>>> r = rec.array([('Smith', 1234),\
                   ('Johnson', 1001),\
                   ('Williams', 1357),\
                   ('Miller', 2468)], \
                   names='Last_name, phone_number')
\end{verbatim}
In this example, we \var{manually} construct a record array by longhand input of
the information.  This record array has 4 records (or rows) and two fields (or 
columns).  The names of the fields are specified in the \code{names} argument.  
When using this longhand input, the data types (formats) are 
automatically determined from the data.  In this case the first field is a 
string of 8 characters (since the longest name is 8 characters long) and 
the second field is an integer.

The record array is just like an array in numarray, except that now each 
element is a \code{Record}.  We can do the usual indexing and slicing:
\begin{verbatim}
>>> print r[0]
('Smith', 1234)
>>> print r[:2]
RecArray[ 
('Smith', 1234),
('Johnson', 1001)
]
\end{verbatim}
To access the record array's fields, use the \code{field()} method:
\begin{verbatim}
>>> print r.field(0)
['Smith', 'Johnson', 'Williams', 'Miller']
>>> print r.field('Last_name')
['Smith', 'Johnson', 'Williams', 'Miller']
\end{verbatim}
these examples show that the \code{field} method can accept either the 
numeric index or the field name.

Since each field is simply a numarray of numbers or strings, all 
functionalities of numarray are available to them.  The record array is one 
single object which allows the user to have either field-wise or row-wise 
access.  The following example:
\begin{verbatim}
>>> r.field('phone_number')[1]=9999
>>> print r[:2]
RecArray[ 
('Smith', 1234),
('Johnson', 9999)
]
\end{verbatim}
shows that a change using the field view will cause the corresponding change 
in the row-wise view without additional copying or computing.

\section{Record array functions}
\label{sec:recarray-func}
\begin{funcdesc}{array}{buffer=None, formats=None, shape=0, 
names=None, byteorder=sys.byteorder}
\label{func:rec.array}
   The function \code{array} is, for most practical purposes, all a user needs 
   to know to construct a record array.

   \code{formats} is a string containing the format information of all fields.  
   Each format can be the \var{letter code}, such as \code{f4} or \code{i2}, 
   or longer name like \code{Float32} or \code{Int16}.  For a list of letter 
   codes or the longer names, see Table \ref{tab:type-specifiers} or use 
   the \code{letterCode()} function.  A field of strings is specified by the 
   letter \code{a}, followed by an integer giving the maximum length; thus 
   \code{a5} is the format for a field of strings of (maximum) length of 5.  

   The formats are separated by commas, and each \var{cell} 
   (element in a field) can be a numarray itself, by attaching a number or a 
   tuple in front of the format specification.  So if 
   \code{formats='i4,Float64,a5,3i2,(2,3)f4,Complex64,b1'}, the record array 
   will have:
   \begin{verbatim}
   1st field: (4-byte) integers
   2nd field: double precision floating point numbers
   3rd field: strings of length 5
   4th field: short (2-byte) integers, each element is an array of shape=(3,)
   5th field: single precision floating point numbers, each element is an 
       array of shape=(2,3)
   6th field: double precision complex numbers
   7th field: (1-byte) Booleans
   \end{verbatim}
   \code{formats} specification takes precedence over the data.  For 
   example, if a field is specified as integers in \code{buffer}, but is 
   specified as floats in \code{formats}, it will be floats in the record 
   array.  If a field in the \code{buffer} is not convertible to the 
   corresponding data type in the \code{formats} specification, e.g. from 
   strings to numbers (integers, floats, Booleans) or vice versa, an 
   exception will be raised.
   
   \code{shape} is the shape of the record array.  It can be an integer, 
   in which case it is equivalent to the number of \var{rows} in a table.  
   It can also be a tuple where the record array is an N-D array with 
   \code{Records} as its elements. \code{shape} must be consistent with the 
   data in \code{buffer} for buffer types (5) and (6), explained below.
   
   \code{names} is a string containing the names of the fields, separated by 
   commas.  If there are more formats specified than names, then default 
   names will be used: If there are five fields specified in \code{formats} 
   but \code{names=None} (default), then the field names will be: 
   \code{c1, c2, c3, c4, c5}.  If \code{names="a,b"}, then the field 
   names will be: \code{a, b, c3, c4, c5}.
   
   If more names have been specified than there are formats, the extra names 
   will be discarded.  If duplicate names are specified, a \code{ValueError} 
   will be raised.  Field names are case sensitive, e.g. column \code{ABC} will 
   not be found if it is referred to as \code{abc} or \code{Abc} 
   (for example) when using the \code{field()} method.
   
   \code{byteorder} is a string of the value \code{big} or \code{little}, 
   referring to big endian or little endian.  This is useful when reading 
   (binary) data from a string or a file.  If not specified, it will use the 
   \code{sys.byteorder} value and the result will be platform dependent for 
   string or file input.
   
   The first argument, \code{buffer}, may be any one of the following:

   (1) \code{None} (default).  The data block in the record array will not be 
   initialized.  The user must assign valid data before trying to read the 
   contents or before writing the record array to a disk file.
   
   (2) a Python string containing binary data.  For example:
   \begin{verbatim}
   >>> r=rec.array('abcdefg'*100, formats='i2,a3,i4', shape=3, byteorder='big')
   >>> print r
   RecArray[ 
   (24930, 'cde', 1718051170),
   (25444, 'efg', 1633837924),
   (25958, 'gab', 1667523942)
   ]
   \end{verbatim}
   
   (3) a Python file object for an open file.  The data will be copied from 
   the file, starting at the current position of the read pointer, with 
   byte order as specified in \code{byteorder}.
   
   (4) a record array.  This results in a deep copy of the input record array; 
   any other arguments to \code{array()} will be silently ignored.
   
   (5) a list of numarrays.  There must be one such numarray for each field.  
   The \code{formats} and \code{shape} arguments to \code{array()} are not 
   required, but if they are specified, they need to be consistent with the 
   input arrays.  The shapes of all the input numarrays also need to be 
   consistent to one another.
   \begin{verbatim}
   # this will have 3 rows, each cell in the 2nd field is an array of 4 elements
   # note that the formats sepcification needs to reflect the data shape
   >>> arr1=numarray.arange(3)
   >>> arr2=numarray.arange(12,shape=(3,4))
   >>> r=rec.array([arr1, arr2],formats='i2,4f4')
   \end{verbatim}
   
   In this example, \code{arr2} is cast up to float.
   
   (6) a list of sequences.  Each sequence contains the 
   number(s)/string(s) of a record.  The example in the introduction 
   uses such input, sometimes called \var{longhand} input.  The data 
   types are automatically determined after comparing all input data.  
   Data of the same field will be cast to the highest type:
   \begin{verbatim}
   # the first field uses the highest data type: Float64
   >>> r=rec.array([[1,'abc'],(3.5, 'xx')]); print r
   RecArray[ 
   (1.0, 'abc'),
   (3.5, 'xx')
   ]
   \end{verbatim}
   unless overruled by the \code{formats} argument:
   \begin{verbatim}
   # overrule the first field to short integers, second field to shorter strings
   >>> r=rec.array([[1,'abc'],(3.5, 'xx')],formats='i2,a1'); print r
   RecArray[ 
   (1, 'a'),
   (3, 'x')
   ]
   \end{verbatim}
   Inconsistent data in the same field will cause a \code{ValueError}:
   \begin{verbatim}
   >>> r=rec.array([[1,'abc'],('a', 'xx')])
   ValueError: inconsistent data at row 1,field 0
   \end{verbatim}
   
   A record array with multi-dimensional numarray cells in a field can also 
   be constructed by using nested sequences:
   \begin{verbatim}
   >>> r=rec.array([[(11,12,13),'abc'],[(2,3,4), 'xx']]); print r
   RecArray[ 
   (array([11, 12, 13]), 'abc'),
   (array([2, 3, 4]), 'xx')
   ]
   \end{verbatim}
\end{funcdesc}
   
\begin{funcdesc}{letterCode}{}
   This function will list the letter codes acceptable by the \code{formats} 
   argument in \code{array()}.
\end{funcdesc}

\section{Record array methods}
\label{sec:recarray-methods}
RecArray object has these public methods:

\begin{methoddesc}[RecArray]{field}{fieldName}
   \code{fieldName} can be either an integer (field index) or string 
   (field name).
   \begin{verbatim}
   >>> r=rec.array([[11,'abc',1.],[12, 'xx', 2.]])
   >>> print r.field('c1')
   [11 12]
   >>> print r.field(0)  # same as field('c1')
   [11 12]
   \end{verbatim}
   To set values, simply use indexing or slicing, since each field is a 
   numarray:
   \begin{verbatim}
   >>> r.field(2)[1]=1000; r.field(1)[1]='xyz'
   >>> r.field(0)[:]=999
   >>> print r
   RecArray[ 
   (999, 'abc', 1.0),
   (999, 'xyz', 1000.0)
   ]
   \end{verbatim}
\end{methoddesc}
\begin{methoddesc}[RecArray]{info}{}
   This will display key attributes of the record array.
\end{methoddesc}

\section{Record object}
\label{sec:recarray-record}
\index{Record object}
Each single record (or \var{row}) in the record array is a 
\code{records.Record} object.  It has these methods:

\begin{methoddesc}[Record]{field}{fieldName}
\end{methoddesc}
\begin{methoddesc}[Record]{setfield}{fieldname, value}
   Like the \code{RecArray}, a \code{Record} object has the \code{field} 
   method to \var{get} the field value.  But since a \code{Record} object 
   is not an array, it does not take an index or slice, so one cannot 
   assign a value to it.  So a separate \var{set} method, \code{setfield()}, 
   is necessary:
   \begin{verbatim}
   >>> r[1].field(0)
   999
   >>> r[1].setfield(0, -1)
   >>> print r[1]
   (-1, 'xy', 1000.0)
   \end{verbatim}
   Like the \code{field()} method in \code{RecArray}, \code{fieldName} in 
   \code{Record}'s \code{field()} and \code{setfield()} methods can be 
   either an integer (index) or a string (field name).
\end{methoddesc}


%% Local Variables:
%% mode: LaTeX
%% mode: auto-fill
%% fill-column: 79
%% indent-tabs-mode: nil
%% ispell-dictionary: "american"
%% reftex-fref-is-default: nil
%% TeX-auto-save: t
%% TeX-command-default: "pdfeLaTeX"
%% TeX-master: "numarray"
%% TeX-parse-self: t
%% End:

\chapter{Object Array}
\label{cha:object-array}
\declaremodule{extension}{numarray.objects}
\index{object array}

\section{Introduction}
\label{sec:objectarray-intro}

\code{numarray}, like \code{Numeric}, has support for arrays of objects in
addition to arrays of numbers.  Arrays of objects are supported by the
\code{numarray.objects} module.  The \index{ObjectArray} \code{ObjectArray}
class is used to represent object arrays.  

The easiest way to construct an object array is to use the
\code{numarray.objects.array()} function.  For example:

\begin{verbatim}
  >>> import numarray.objects as obj
  >>> o = obj.array(['S', 'J', 1, 'M'])
  >>> print o
  ['S' 'J' 1 'M']
  >>> print o + o
  ['SS' 'JJ' 2 'MM']
\end{verbatim}

In this example, the array contains 3 Python strings and an integer, but the
array elements can be any Python object.  For each pair of elements, the
\function{add} operator is applied.  For strings, \function{add} is defined as
string concatenation.  For integers, \function{add} is defined as numerical
addition.  For a class object, the \function{__add__} and \function{__radd__}
methods would define the result.

\class{ObjectArray} is defined as a subclass of numarray's structural array
class, \class{NDArray}.  As a result, we can do the usual indexing and slicing:

\begin{verbatim}
  >>> import numarray.objects as obj
  >>> print s[0]
  'S'
  >>> print s[:2]
  ['S' 'J']
  >>> s[:2] = 'changed'
  >>> print s
  ['changed' 'changed' 1 'M']
  >>> a = obj.fromlist(numarray.arange(100), shape=(10,10))
  >>> a[2:5, 2:5]
  ObjectArray([[22, 23, 24],
               [32, 33, 34],
               [42, 43, 44]])
\end{verbatim}

\section{Object array functions}
\label{sec:objectarray-func}
\begin{funcdesc}{array}{sequence=None, shape=None, typecode='O'}
\label{func:obj.array}
   The function \function{array} is, for most practical purposes, all a user needs 
   to know to construct an object array.

   The first argument, \code{sequence}, can be an arbitrary sequence of Python
   objects, such as a list, tuple, or another object array.  

\begin{verbatim}
  >>> import numarray.objects as obj
  >>> class C:
  ...     pass
  >>> c = C()
  >>> a = obj.array([c, c, c])
  >>> a
  ObjectArray([c, c, c])
\end{verbatim}
   
   Like objects in Python lists, objects in object arrays are referred to, not
   copied, so changes to the objects are reflected in the originals because
   they are one and the same.

\begin{verbatim}
     >>> a[0].attribute  = 'this'
     >>> c.attribute
     'this'
\end{verbatim}
   
   The second argument, \code{shape}, optionally specifies the shape of the
   array.  If no \code{shape} is specified, the shape is implied by the
   sequence.

\begin{verbatim}
  >>> import numarray.objects as obj
  >>> class C:
  ...     pass
  >>> c = C()
  >>> a = obj.fromlist([c, c, c])
  >>> a
  ObjectArray([c, c, c])
\end{verbatim}
   
   The last argument, \code{typecode}, is there for backward compatibility with
   Numeric; it must be specified as 'O'.

\end{funcdesc}
   
\begin{funcdesc}{asarray}{obj}
  \label{func:obj.asarray}
  \code{asarray} converts sequences which are not object arrays into object
  arrays.  If \code{obj} is already an \class{ObjectArray}, it is returned
  unaltered.
\begin{verbatim}
  >>> import numarray.objects as obj
  >>> a = obj.asarray([1,''this'',''that''])
  >>> a
  ObjectArray([1 'this' 'that'])
  >>> b = obj.asarray(a)
  >>> b is a
  True
\end{verbatim}
\end{funcdesc}

\begin{funcdesc}{choose}{selector, population, output=None}
  \label{func:obj.choose}
  \code{choose} selects elements from \var{population} based on the values in
  \var{selector}, either returning the selected array or storing it in the
  optional \code{ObjectArray} specified by \var{output}.  \var{selector} should
  be an integer sequence where each element is within the range 0 to
  \function{len}{population}.  \var{population} should be a sequence of
  \class{ObjectArray}s. The shapes of \var{selector} and each element of
  \var{population} must be mutually broadcastable.
\begin{verbatim}
  >>> import numarray.objects as obj
  >>> s = num.arange(25, shape=(5,5)) % 3
  >>> p = obj.fromlist(["foo", 1, {"this":"that"}])
  >>> obj.choose(s, p)
  ObjectArray([['foo', 1, {'this': 'that'}, 'foo', 1],
    [{'this': 'that'}, 'foo', 1, {'this': 'that'}, 'foo'],
    [1, {'this': 'that'}, 'foo', 1, {'this': 'that'}],
    ['foo', 1, {'this': 'that'}, 'foo', 1],
    [{'this': 'that'}, 'foo', 1, {'this': 'that'}, 'foo']])
  
\end{verbatim}
\end{funcdesc}

\begin{funcdesc}{sort}{objects, axis=-1, output=None}
  \label{func:obj.sort}
  \code{sort} sorts the elements from \var{objects} along the specified
  \var{axis}.  If an output array is specified, the result is stored there
  and the return value is None,  otherwise the sort is returned.
\begin{verbatim}
    >>> import numarray.objects as obj
    >>> a = obj.ObjectArray(shape=(5,5))
    >>> a[:] = range(5,0,-1)
    >>> obj.sort(a)
    ObjectArray([[1, 2, 3, 4, 5],
                 [1, 2, 3, 4, 5],
                 [1, 2, 3, 4, 5],
                 [1, 2, 3, 4, 5],
                 [1, 2, 3, 4, 5]])
    >>> a[:] = range(5,0,-1)
    >>> a.transpose()
    >>> obj.sort(a, axis=0)
    ObjectArray([[1, 1, 1, 1, 1],
                 [2, 2, 2, 2, 2],
                 [3, 3, 3, 3, 3],
                 [4, 4, 4, 4, 4],
                 [5, 5, 5, 5, 5]])
\end{verbatim}
\end{funcdesc}

\begin{funcdesc}{argsort}{objects, axis=-1, output=None}
  \label{func:obj.argsort}
  \code{argsort} returns the sort order for the elements from \var{objects}
  along the specified \var{axis}.  If an output array is specified, the result
  is stored there and the return value is None, otherwise the sort order is
  returned.
\begin{verbatim}
  >>> import numarray.objects as obj
  >>> a = obj.ObjectArray(shape=(5,5))
  >>> a[:] = ['e','d','c','b','a']
  >>> obj.argsort(a)
  array([[4, 3, 2, 1, 0],
         [4, 3, 2, 1, 0],
         [4, 3, 2, 1, 0],
         [4, 3, 2, 1, 0],
         [4, 3, 2, 1, 0]])
\end{verbatim}
\end{funcdesc}

\begin{funcdesc}{take}{objects, indices, axis=0}
  \label{func:obj.take}
  \code{take} returns elements of \var{objects} specified by tuple of index
  arrays \var{indices} along the specified \var{axis}.
\begin{verbatim}
  >>> import numarray.objects as obj
  >>> o = obj.fromlist(range(10))
  >>> a = obj.arange(5)*2
  >>> obj.take(o, a)
  ObjectArray([0, 2, 4, 6, 8])
\end{verbatim}
\end{funcdesc}

\begin{funcdesc}{put}{objects, indices, values, axis=-1}
  \label{func:obj.put}
  \function{put} stores \var{values} at the locations of \var{objects}
  specified by tuple of index arrays \var{indices}.
\begin{verbatim}
  >>> import numarray.objects as obj
  >>> o = obj.fromlist(range(10))
  >>> a = obj.arange(5)*2
  >>> obj.put(o, a, 0); o
  ObjectArray([0, 1, 0, 3, 0, 5, 0, 7, 0, 9])
\end{verbatim}
\end{funcdesc}

\begin{funcdesc}{add}{objects1, objects2, out=None}
  \label{func:obj.add}
  \code{numarray.objects} defines universal functions which are named after and
  use the operators defined in the standard library module operator.py.  In
  addition, the operator hooks of the \class{ObjectArray} class are defined to
  call the operators.  \code{add} applies the \code{add} operator to
  corresponding elements of \var{objects1} and \var{objects2}.  Like the ufuncs
  in the numerical side of numarray, the object ufuncs support reduction and
  accumulation.  In addition to add, there are ufuncs defined for every unary
  and binary operator function in the standard library module operator.py.
  Some of these are given additional synonyms so that they use numarray naming
  conventions, e.g. \function{sub} has an alias named \function{subtract}.
\begin{verbatim}
  >>> import numarray.objects as obj
  >>> a = obj.fromlist(["t","u","w"])
  >>> a
  ObjectArray(['t', 'u', 'w'])
  >>> a+a
  ObjectArray(['tt', 'uu', 'ww'])
  >>> obj.add(a,a)
  ObjectArray(['tt', 'uu', 'ww'])
  >>> obj.add.reduce(a)
  'tuw' # not, as in the docs, an ObjectArray
  >>> obj.add.accumulate(a)
  ObjectArray(['t', 'tu', 'tuw']) # w, not v

  >>> a = obj.fromlist(["t","u","w"])
  >>> a
  ObjectArray(['t', 'u', 'w'])
  >>> a+a
  ObjectArray(['tt', 'uu', 'ww'])
  >>> obj.add(a,a)
  ObjectArray(['tt', 'uu', 'ww'])
  >>> obj.add.reduce(a)
  ObjectArray('tuv')
  >>> obj.add.accumulate(a)
  ObjectArray(['t', 'tu', 'tuv'])
\end{verbatim}
\end{funcdesc}

\section{Object array methods}
\label{sec:objectarray-methods}
\class{ObjectArray} maps each of its operator hooks (e.g. \code{__add__}) onto
the corresponding object ufunc (e.g. \code{numarray.objects.add}).  In addition
to its hook methods,  \class{ObjectArray} has these public methods:

\begin{methoddesc}[ObjectArray]{tolist}{}
  \code{tolist} returns a nested list of objects corresponding to all the
  elements in the array.
\end{methoddesc}

\begin{methoddesc}[ObjectArray]{copy}{}
  \code{copy} returns a shallow copy of the object array.
\end{methoddesc}

\begin{methoddesc}[ObjectArray]{astype}{type}
  \code{astype} returns either a copy of the \class{ObjectArray} or converts it
  into a numerical array of the specified \var{type}.
\end{methoddesc}

\begin{methoddesc}[ObjectArray]{info}{}
   This will display key attributes of the object array.
\end{methoddesc}

%% Local Variables:
%% mode: LaTeX
%% mode: auto-fill
%% fill-column: 79
%% indent-tabs-mode: nil
%% ispell-dictionary: "american"
%% reftex-fref-is-default: nil
%% TeX-auto-save: t
%% TeX-command-default: "pdfeLaTeX"
%% TeX-master: "numarray"
%% TeX-parse-self: t
%% End:

\chapter{C extension API}
\label{cha:C-API}
\declaremodule{extension}{C-API}
\index{C-API}

\begin{quote}
   This chapter describes the different available C-APIs for \module{numarray}
   based extension modules.
\end{quote}

While this chapter describes the \module{\numarray}-specifics for writing
extension modules, a basic understanding of \python extension modules is
expected. See \python's \ulink{Extending and
   Embedding}{http://www.python.org/doc/current/ext/ext.html} tutorial and the
\ulink{Python/C API}{http://www.python.org/doc/current/api/api.html}.

The numarray C-API has several different facets, and the first three facets
each make different tradeoffs between memory use, speed, and ease of use.  An
additional facet provides backwards compatability with legacy Numeric code.
The final facet consists of miscellaneous function calls used to implement
and utilize numarray, that were not part of Numeric.

In addition to most of the basic functionality provided by Numeric, these APIs
provide access to misaligned, byteswapped, and discontiguous \class{numarray}s.
Byteswapped arrays arise in the context of portable binary data formats where
the byteorder specified by the data format is not the same as the host
processor byte order.  Misaligned arrays arise in the context of tabular data:
files of records where arrays are superimposed on the column formed by a single
field in the record.  Discontiguous arrays arise from operations which permute
the shape and strides of an array, such as reshape.

\begin{description}
\item[Numeric compatability] This API provides a reasonable (if not complete)
   simulation of the Numeric C-API.  It is written in terms of the numarray
   high level API (discussed below) so that misbehaved numarrays are copied
   prior to processing with legacy Numeric code.  This API was actually written
   last because of the extra considerations in numarray not found in Numeric.
   Nevertheless, it is perhaps the most important because it enables writing
   extension modules which can be compiled for either numarray or Numeric.  It
   is also very useful for porting existing Numeric code.  See section
   \ref{sec:C-API:numeric-simulation}.
\item[High-level] This is the cleanest and eaisiest to use API.  It creates
   temporary arrays to handle difficult cases (discontiguous, byteswapped,
   misaligned) in C code.  Code using this API is written in terms of a pointer
   to a contiguous 1D array of C data.  See section
   \ref{sec:C-API:high-level-api}.
\item[Element-wise] This API handles misbehaved arrays without creating
     temporaries.  Code using this API is written to access single elements of
     an array via macros or functions.  \note{These macros are relatively slow
     compared to raw access to C data, and the functions even slower.} See
     section \ref{sec:C-API:element-wise-api}.
\item[One-dimensional] Code using this API get/sets consecutive elements of the
   inner dimension of an array, enabling the API to factor out tests for
   aligment and byteswapping to one test per array rather than one test per
   element.  Fewer tests means better performance, but at a cost of some
   temporary data and more difficult usage.  See section
   \ref{sec:C-API:One-dimensional-api}.
\item[New numarray functions] This last facet of the C-API consists of function
  calls which have been added to numarray which are orthogonal to each of the 3
  native access APIs and not part of the original Numeric. See section
  \ref{sec:C-API:new-numarray-functions}
\end{description}

\section{Numarray extension basics}
There's a couple things you need to do in order to access numarray's C-API in
your own C extension module:

\subsection{Include libnumarray.h}
  Near the top of your extension module add the lines:
\begin{verbatim}
  #include "Python.h"
  #include "libnumarray.h"
\end{verbatim} 
  This gives your C-code access to the numarray typedefs, macros, and function
  prototypes as well as the Python C-API.

  \subsection{Alternate include method}
  There's an alternate form of including libnumarray.h or arrayobject.h some
  people may prefer provided that they're willing to ignore the case where the
  numarray includes are not installed in the standard location.  The advantage
  of the following approach is that it automatically works with the default
  path to the Python include files which the distutils always provide.
  \begin{verbatim}
    #include "Python.h"
    #include "numarray/libnumarray.h"
  \end{verbatim}

\subsection{Import libnumarray}
  In your extension module's initialization function, add the line:
\begin{verbatim}
  import_libnumarray();
\end{verbatim} 
  
  import_libnumarray() is actually a macro which sets up a pointer to the
  numarray C-API function pointer table. If you forget to call
  import_libnumarray(), your extension module will crash as soon as you call a
  numarray API function, because your application will attempt to dereference a
  NULL pointer.

  Note that for the Numeric compatible API you should substitute arrayobject.h
  for libnumarray.h and import_array() for import_libnumarray() respectively.
  Unlike other versions of numarray prior to 1.0, arrayobject.h now includes
  only the Numeric simulation API.  To use the rest of the numarray API, you
  \emph{must} include libnumarray.h.  To use both, you must include both
  arrayobject.h and libnumarray.h, and you must both import_array() and
  import_libnumarray() in your module initialization function.

  \subsection{Writing a simple setup.py file for a numarray extension}
  One important practice for writing an extension module is the creation of a
  distutils setup.py file which automates both extension installation from
  source and the creation of binary distributions.  Here is a simple setup.py
  which builds a single extension module from a single C source file:
  \begin{verbatim}
    from distutils.core import setup, Extension
    from numarray.numarrayext import NumarrayExtension
    import sys
    
    if not hasattr(sys, 'version_info') or sys.version_info < (2,2,0,'alpha',0):        raise SystemExit, "Python 2.2 or later required to build this module."
    
    setup(name = "buildHistogram",
       version = "0.1",
       description = "",
       packages=[""],
       package_dir={"":""},
       ext_modules=[NumarrayExtension("buildHistogram",['buildHistogram.c'],\
         include_dirs=["./"],
         library_dirs=["./"],
         libraries=['m'])])
\end{verbatim}
\class{NumarrayExtension} is recommended rather than it's distutils baseclass
\class{Extension} because \class{NumarrayExtension} knows where to find the
numarray headers regardless of where the numarray installer or setup.py command
line options put them.  A disadvantage of using NumarrayExtension is that it
is numarray specific, so it does not work for compiling Numeric versions of the
extension.

See the Python manuals ``Installing Python Modules'' and ``Distributing Python
Modules'' for more information on how to use distutils.

\section{Fundamental data structures}
\label{C-API:fundamental-data-structures}

\subsection{Numarray Numerical Data Types}

Numarray hides the C implementation of its basic array elements behind a set of
C typedefs which specify the absolute size of the type in bits.  This approach
enables a programmer to specify data items of arrays and extension functions in
an explicit yet portable manner.  In contrast, basic C types are platform
relative, and so less useful for describing real physical data.  Here are the
names of the concrete Numarray element types:

\begin{itemize}
\item Bool                
\item Int8,      UInt8
\item Int16,     UInt16
\item Int32,     UInt32
\item Int64,     UInt64
\item Float32,   Float64
\item Complex32, Complex64
\end{itemize}

\subsection{NumarrayType}

The type of a numarray is communicated in C via one of the following
enumeration constants.  Type codes which are backwards compatible with Numeric
are defined in terms of these constants, but use these if you're not already
using the Numeric codes.  These constants communicate type requirements between
one function and another, since in C, you cannot pass a typedef as a value.
tAny is used to specify both ``no type requirement'' and ``no known type''
depending on context.   

\begin{verbatim}
typedef enum 
{
  tAny,

  tBool,
  tInt8,      tUInt8,
  tInt16,     tUInt16,
  tInt32,     tUInt32, 
  tInt64,     tUInt64,
  tFloat32,   tFloat64,
  tComplex32, tComplex64,

  tDefault = tFloat64,

#if LP64
  tLong = tInt64
#else
  tLong = tInt32
#endif

} NumarrayType;
\end{verbatim}

\subsection{PyArray_Descr}

\ctype{PyArray_Descr} is used to hold a few parameters related to the type of
an array and exists mostly for backwards compatability with Numeric.
\var{type_num} is a NumarrayType value.  \var{elsize} indicates the number of
bytes in one element of an array of that type.  \var{type} is a Numeric
compatible character code.

Numarray's \ctype{PyArray_Descr} is currently missing the type-casting,
\function{ones}, and \function{zeroes} functions.  Extensions which use these
missing Numeric features will not yet compile.  Arrays of type Object are not
yet supported.

\begin{verbatim}

typedef struct {
        int  type_num;  /* PyArray_TYPES */
        int  elsize;    /* bytes for 1 element */
        char type;      /* One of "cb1silfdFD "  Object arrays not supported. */
} PyArray_Descr;

\end{verbatim}

\subsection{PyArrayObject}

The fundamental data structure of numarray is the PyArrayObject, which is named
and layed out to provide source compatibility with Numeric.  It is compatible
with most but not all Numeric code.  The constant MAXDIM, the maximum number of
dimensions in an array, is defined as 40.  It should be noted that unlike
earlier versions of numarray, the present PyArrayObject structure is a first
class python object, with full support for the number protocols in C.
Well-behaved arrays have mutable fields which will reflect modifications back
into \python ``for free''.

\begin{verbatim}

typedef int maybelong;          /* towards 64-bit without breaking extensions. */

typedef struct {
        /* Numeric compatible stuff */

        PyObject_HEAD
        char *data;              /* points to the actual C data for the array */
        int nd;                  /* number of array shape elements */
        maybelong *dimensions;   /* values of shape elements */
        maybelong *strides;      /* values of stride elements */
        PyObject *base;          /* unused, but don't touch! */
        PyArray_Descr *descr;    /* pointer to descriptor for this array's type */
        int flags;               /* bitmask defining various array properties */

        /* numarray extras */

        maybelong _dimensions[MAXDIM];  /* values of shape elements */
        maybelong _strides[MAXDIM];     /* values of stride elements */
        PyObject *_data;       /* object must meet buffer API */
        PyObject *_shadows;    /* ill-behaved original array. */
        int      nstrides;     /* elements in strides array */
        long     byteoffset;   /* offset into buffer where array data begins */
        long     bytestride;   /* basic seperation of elements in bytes */
        long     itemsize;     /* length of 1 element in bytes */

        char      byteorder;   /* NUM_BIG_ENDIAN, NUM_LITTLE_ENDIAN */

        char      _unused0; 
        char      _unused1; 
        
        /* Don't expect the following vars to stay around.  Never use them.
        They're an implementation detail of the get/set macros. */

        Complex64      temp;   /* temporary for get/set macros */
        char *         wptr;   /* working pointer for get/set macros */
} PyArrayObject;

\end{verbatim}

\subsection{Flag Bits}

The following are the definitions for the bit values in the \var{flags} field
of each numarray.  Low order bits are Numeric compatible,  higher order bits
were added by numarray.

\begin{verbatim}
/* Array flags */
#define CONTIGUOUS        1       /* compatible, depends */
#define OWN_DIMENSIONS    2       /* always false */
#define OWN_STRIDES       4       /* always false */
#define OWN_DATA          8       /* always false */
#define SAVESPACE      0x10       /* not used */

#define ALIGNED       0x100       /* roughly: data % itemsize == 0 */
#define NOTSWAPPED    0x200       /* byteorder == sys.byteorder    */
#define WRITABLE      0x400       /* data buffer is writable       */

#define IS_CARRAY (CONTIGUOUS | ALIGNED | NOTSWAPPED)
\end{verbatim}

\section{Numeric simulation API}
\label{sec:C-API:numeric-simulation}

These notes describe the Numeric compatability functions which enable numarray
to utilize a subset of the extensions written for Numeric (NumPy).  Not all
Numeric C-API features and therefore not all Numeric extensions are currently
supported.  Users should be able to utilize suitable extensions written for
Numeric within the numarray environment by:

\begin{enumerate}
\item Writing a numarray setup.py file.
\item Scanning the extension C-code for all instances of array creation and
  return and making corrections as needed and specified below. 
\item Re-compiling the Numeric C-extension for numarray.
\end{enumerate}

Numarray's compatability with Numeric consists of 3 things:
\begin{enumerate}
\item A replacement header file, "arrayobject.h" which supplies simulation
   functions and macros for numarray just as the original arrayobject.h
   supplies the C-API for Numeric.
\item Layout and naming of the fundamental numarray C-type,
\ctype{PyArrayObject}, in a Numeric source compatible way.
\item A set of "simulation" functions.  These functions have the same names and
   parameters as the original Numeric functions, but operate on numarrays.  The
   simulation functions are also incomplete; features not currently supported
   should result in compile time warnings.
\end{enumerate}

\subsection{Simulation Functions}
\label{sec:C-API:compat:simulation-functions}

The basic use of numarrays by Numeric extensions is achieved in the extension
function's wrapper code by:
\begin{enumerate}
\item Ensuring creation of array objects by calls to simulation functions.
\item DECREFing each array or calling PyArray_Return.
\end{enumerate}

Unlike prior versions of numarray, this version *does* support access to array
objects straight out of PyArg_ParseTuple.  This is a consequence of a change to
the underlying object model, where a class instance has been replaced by
PyArrayObject.  Nevertheless, the ``right'' way to access arrays is either via
the high level interface or via emulated Numeric factory functions.  That way,
access to other python sequences is supported as well.  Using the ``right'' way
for numarray is also more important than for Numeric because numarray arrays
may be byteswapped or misaligned and hence unusable from simple C-code.  It
should be noted that the numarray and Numeric are not completely compatible,
and therefore this API does not provide support for string arrays or object
arrays.

The creation of array objects is illustrated by the following of wrapper code
for a 2D convolution function:

\begin{verbatim}
#include "python.h"
#include "arrayobject.h"

static PyObject *
Py_Convolve2d(PyObject *obj, PyObject *args)
{
        PyObject   *okernel, *odata, *oconvolved=Py_None;
        PyArrayObject *kernel, *data, *convolved;

        if (!PyArg_ParseTuple(args, "OO|O", &okernel, &odata, &oconvolved)) {
                return PyErr_Format(_Error, 
                                    "Convove2d: Invalid parameters.");  
                goto _fail;
        }
\end{verbatim}

The first step was simply to get object pointers to the numarray parameters to
the convolution function: okernel, odata, and oconvolved.  Oconvolved is an
optional output parameter, specified with a default value of Py_None which is
used when only 2 parameters are supplied at the python level.  Each of the
``o'' parameters should be thought of as an arbitrary sequence object, not
necessarily an array.

The next step is to call simulation functions which convert sequence objects
into PyArrayObjects.  In a Numeric extension, these calls map tuples and lists
onto Numeric arrays and assert their dimensionality as 2D.  The Numeric
simulation functions first map tuples, lists, and misbehaved numarrays onto
well-behaved numarrays.  Calls to these functions transparently use the
numarray high level interface and provide visibility only to aligned and
non-byteswapped array objects.

\begin{verbatim}
        kernel = (PyArrayObject *) PyArray_ContiguousFromObject(
                okernel, PyArray_DOUBLE, 2, 2);
        data = (PyArrayObject *) PyArray_ContiguousFromObject(
                odata, PyArray_DOUBLE, 2, 2);

        if (!kernel || !data) goto _fail;
\end{verbatim}

Extra processing is required to handle the output array \var{convolved},
cloning it from \var{data} if it was not specified.  Code should be supplied,
but is not, to verify that convolved and data have the same shape.  

\begin{verbatim}
        if (convolved == Py_None)
                convolved = (PyArrayObject *) PyArray_FromDims(
                        data->nd, data->dimensions, PyArray_DOUBLE);
        else
                convolved = (PyArrayObject *) PyArray_ContiguousFromObject(
                        oconvolved, PyArray_DOUBLE, 2, 2);
        if (!convolved) goto _fail;
\end{verbatim}

After converting all of the input paramters into \ctype{PyArrayObject}s, the
actual convolution is performed by a seperate function.  This could just as
well be done inline:

\begin{verbatim}
        Convolve2d(kernel, data, convolved);
\end{verbatim}

After processing the arrays, they should be DECREF'ed or returned using
\cfunction{PyArray_Return}.  It is generally not possible to directly return a
numarray object using \cfunction{Py_BuildValue} because the shadowing of
mis-behaved arrays needs to be undone.  Calling \cfunction{PyArray_Return}
destroys any temporary and passes the numarray back to \python.

\begin{verbatim}
        Py_DECREF(kernel);
        Py_DECREF(data);
        if (convolved != Py_None) {
                Py_DECREF(convolved);
                Py_INCREF(Py_None);
                return Py_None;
        } else
                return PyArray_Return(convolved);
_fail:
        Py_XDECREF(kernel);
        Py_XDECREF(data);
        Py_XDECREF(convolved);
        return NULL;
}
\end{verbatim}

Byteswapped or misaligned arrays are handled by a process of shadowing which
works like this:
\begin{enumerate}
\item When a "misbehaved" numarray is accessed via the Numeric simulation
  functions, first a well-behaved temporary copy (shadow) is created by
  NA_IoArray.
\item Operations performed by the extension function modifiy the data buffer
  belonging to the shadow.
\item On extension function exit, the shadow array is copied back onto the 
  original and the shadow is freed.
\end{enumerate}
All of this is transparent to the user; if the original array is well-behaved,
it works much like it always did; if not, what would have failed altogether
works at the cost of extra temporary storage.  Users which cannot afford the
cost of shadowing need to use numarray's native elementwise or 1D APIs.
\subsection{Numeric Compatible Functions}
\label{sec:C-API:compat:implemented-functions}

The following functions are currently implemented:
\begin{cfuncdesc}{PyObject*}{PyArray_FromDims}{int nd, int *dims, int type}
   This function will allocate a new numarray.

   An array created with PyArray_FromDims can be used as a temporary or
   returned using PyArray_Return.
   
   Used as a temporary, calling Py_DECREF deallocates it.   
\end{cfuncdesc}

\begin{cfuncdesc}{PyObject*}{PyArray_FromDimsAndData}{int nd, int *dims, int type, char *data}
   This function will allocate a numarray of the specified shape and type
   which will refer to the data buffer specified by \var{data}.  The contents
   of \var{data} will not be copied nor will \var{data} be deallocated upon
   the deletion of the array.
\end{cfuncdesc}

\begin{cfuncdesc}{PyObject*}{PyArray_ContiguousFromObject}{%
      PyObject *op, int type, int min_dim, int max_dim}% Returns an simulation
   object for a contiguous numarray of 'type' created from the sequence object
   'op'.  If 'op' is a contiguous, aligned, non-byteswapped numarray, then the
   simulation object refers to it directly.  Otherwise a well-behaved numarray
   will be created from 'op' and the simulation object will refer to it.
   min_dim and max_dim bound the expected rank as in Numeric.
   \code{min_dim==max_dim} specifies an exact rank.  \code{min_dim==max_dim==0}
   specifies \emph{any} rank.
\end{cfuncdesc}

\begin{cfuncdesc}{PyObject*}{PyArray_CopyFromObject}{%
      PyObject *op, int type, int min_dim, int max_dim}% Returns a contiguous
   array, similar to PyArray_FromContiguousObject, but always returning an
   simulation object referring to a new numarray copied from the original
   sequence.
\end{cfuncdesc}

\begin{cfuncdesc}{PyObject*}{PyArray_FromObject}{%
      PyObject  *op, int type, int min_dim, int max_dim}%
   Returns and simulation object based on 'op', possibly discontiguous.  The
   strides array must be used to access elements of the simulation object.
   
   If 'op' is a byteswapped or misaligned numarray, FromObject creates a
   temporary copy and the simulation object refers to it.
   
   If 'op' is a nonswapped, aligned numarray, the simulation object refers to
   it.
   
   If 'op' is some other sequence, it is converted to a numarray and the
   simulation object refers to that.
\end{cfuncdesc}

\begin{cfuncdesc}{PyObject*}{PyArray_Return}{PyArrayObject *apr}
   Returns simulation object 'apr' to python.  The simulation object itself is
   destructed.  The numarray it refers to (base) is returned as the result of
   the function.
   
   An additional check is (or eventually will be) performed to guarantee that
   rank-0 arrays are converted to appropriate python scalars.
   
   PyArray_Return has no net effect on the reference count of the underlying
   numarray.
\end{cfuncdesc}

\begin{cfuncdesc}{int}{PyArray_As1D}{PyObject **op, char **ptr, int *d1, int typecode}
   Copied from Numeric verbatim.
\end{cfuncdesc}

\begin{cfuncdesc}{int}{PyArray_As2D}{PyObject **op, char ***ptr, int *d1, int *d2, int typecode}
   Copied from Numeric verbatim.
\end{cfuncdesc}

\begin{cfuncdesc}{int}{PyArray_Free}{PyObject *op, char *ptr}
   Copied from Numeric verbatim. \note{This means including bugs and all!}
\end{cfuncdesc}

\begin{cfuncdesc}{int}{PyArray_Check}{PyObject *op}
   This function returns 1 if op is a PyArrayObject.  
\end{cfuncdesc}

\begin{cfuncdesc}{int}{PyArray_Size}{PyObject *op}
   This function returns the total element count of the array.
\end{cfuncdesc}

\begin{cfuncdesc}{int}{PyArray_NBYTES}{PyArrayObject *op}
   This function returns the total size in bytes of the array, and assumes that
bytestride == itemsize, so that the size is product(shape)*itemsize.
\end{cfuncdesc}

\begin{cfuncdesc}{PyObject*}{PyArray_Copy}{PyArrayObject *op}
   This function returns a copy of the array 'op'.  The copy returned is
   guaranteed to be well behaved, i.e. neither byteswapped nor misaligned.
\end{cfuncdesc}

\begin{cfuncdesc}{int}{PyArray_CanCastSafely}{PyArrayObject *op, int type}
  This function returns 1 IFF the array 'op' can be safely cast to 'type',
otherwise it returns 0.
\end{cfuncdesc}

\begin{cfuncdesc}{PyArrayObject*}{PyArray_Cast}{PyArrayObject *op, int type}
  This function casts the array 'op' into an equivalent array of type 'type'.
\end{cfuncdesc}

\begin{cfuncdesc}{PyArray_Descr*}{PyArray_DescrFromType}{int type}
This function returns a pointer to the array descriptor for 'type'.  The
numarray version of PyArray_Descr is incomplete and does not support casting,
getitem, setitem, one, or zero.
\end{cfuncdesc}

\begin{cfuncdesc}{int}{PyArray_isArray(PyObject *o)}
  This macro is designed to fail safe and return 0 when numarray is not
  installed at all.  When numarray is installed, it returns 1 iff object 'o' is
  a numarray, and 0 otherwise.  This macro facilitates the optional use of
  numarray within an extension.
\end{cfuncdesc}

\subsection{Unsupported Numeric Features}
\label{sec:C-API:compat:unsupported}

\begin{itemize}
\item PyArrayError 
\item PyArray_ObjectType() 
\item PyArray_Reshape()
\item PyArray_SetStringFunction() 
\item PyArray_SetNumericOps() 
\item PyArray_Take()
\item UFunc API
\end{itemize}

\section{High-level API}
\label{sec:C-API:high-level-api}

The high-level native API accepts an object (which may or may not be an array)
and transforms the object into an array which satisfies a set of ``behaved-ness
requirements''.  The idea behind the high-level API is to transparently convert
misbehaved numarrays, ordinary sequences, and python scalars into C-arrays.  A
``misbehaved array'' is one which is byteswapped, misaligned, or discontiguous.
This API is the simplest and fastest, provided that your arrays are small.  If
you find your program is exhausting all available memory, it may be time to
look at one of the other APIs.

\subsection{High-level functions}
\label{sec:C-API:high-level-functions}

The high-level support functions for interchanging \class{numarray}s between
\python{} and C are as follows:

\begin{cfuncdesc}{PyArrayObject*}{NA_InputArray}{%
      PyObject *seq, NumarrayType t, int requires}
The purpose of NA_InputArray is to transfer array data from \python to C.
\end{cfuncdesc}

\begin{cfuncdesc}{PyArrayObject*}{NA_OutputArray}{%
      PyObject *seq, NumarrayType t, int requires} The purpose of
NA_OutputArray is to transfer data from C to \python.  The output array must be
a PyArrayObject, i.e. it cannot be an arbitrary Python sequence.
\end{cfuncdesc}

\begin{cfuncdesc}{PyArrayObject*}{NA_IoArray}{%
      PyObject *seq, NumarrayType t, int requires} NA_IoArray has fully
bidirectional data transfer, creating the illusion of call-by-reference.
\end{cfuncdesc}

  For a well-behaved writable array, there is no difference between the three,
  as no temporary is created and the returned object is identical to the
  original object (with an additional reference).  For a mis-behaved input
  array, a well-behaved temporary will be created and the data copied from the
  original to the temporary.  Since it is an input, modifications to its
  contents are not guaranteed to be reflected back to \python, and in the case
  where a temporary was created, won't be.  For a mis-behaved output array, any
  data side-effects generated by the C code will be safely communicated back to
  \python, but the initial array contents are undefined.  For an I/O array, any
  required temporary will be initialized to the same contents as the original
  array, and any side-effects caused by C-code will be copied back to the
  original array.  The array factory routines of the Numeric compatability API
  are written in terms of NA_IoArray.
   
   The return value of each function (\cfunction{NA_InputArray},
   \cfunction{NA_OutputArray}, or \cfunction{NA_IoArray}) is either a reference
   to the original array object, or a reference to a temporary array.
   Following execution of the C-code in the extension function body this
   pointer should \emph{always} be DECREFed.  When a temporary is DECREFed, it
   is deallocated, possibly after copying itself onto the original array.  The
   one exception to this rule is that you should not DECREF an array returned
   via the NA_ReturnOutput function.
   
   The \var{seq} parameter specifies the original numeric sequence to be
   interfaced.  Nested lists and tuples of numbers can be converted by
   \cfunction{NA_InputArray} and \cfunction{NA_IoArray} into a temporary array.
   The temporary is lost on function exit.  Strictly speaking, allowing
   NA_IoArray to accept a list or tuple is a wart, since it will lose any side
   effects.  In principle, communication back to lists and tuples can be
   supported but is not currently.
   
   The \var{t} parameter is an enumeration value which defines the type the
   array data should be converted to.  Arrays of the same type are passed
   through unaltered, while mis-matched arrays are cast into temporaries of the
   specified type.  The value \constant{tAny} may be specified to indicate that
   the extension function can handle any type correctly so no temporary should
   is required.
   
   The \var{requires} integer indicates under what conditions, other than type
   mismatch, a temporary should be made.  The simple way to specify it is to
   use \constant{NUM_C_ARRAY}.  This will cause the API function to make a
   well-behaved temporary if the original is byteswapped, misaligned, or
   discontiguous.  

There is one other pair of high level function which serves to return output
arrays as the function value: NA_OptionalOutputArray and NA_ReturnOutput.

\begin{cfuncdesc}{PyArrayObject*}{NA_OptionalOutputArray}{%
      PyObject *seq, NumarrayType t, int requires, PyObject *master}%
   \cfunction{NA_OptionalOutputArray} is essentially
   \cfunction{NA_OutputArray}, but with one twist: if the original array
   \var{seq} has the value \constant{NULL} or \constant{Py_None}, a copy of
   \var{master} is returned.  This facilitates writing functions where the
   output array may or may-not be specified by the \python{} user.  
\end{cfuncdesc}

\begin{cfuncdesc}{PyObject*}{NA_ReturnOutput}{PyObject *seq, PyObject *shadow}
   \cfunction{NA_ReturnOutput} accepts as parameters both the original
   \var{seq} and the value returned from
   \cfunction{NA_OptionalOutputArray}, \var{shadow}.  If \var{seq} is
   \constant{Py_None} or \constant{NULL}, then \var{shadow} is returned.
   Otherwise, an output array was specified by the user, and \constant{Py_None}
   is returned.  This facilitates writing functions in the numarray style
   where the specification of an output array renders the function ``mute'',
   with all side-effects in the output array and None as the return value.
\end{cfuncdesc}

\subsection{Behaved-ness Requirements}

Calls to the high level API specify a set of requirements that incoming arrays
must satisfy.  The requirements set is specified by a bit mask which is or'ed
together from bits representing individual array requirements.  An ordinary C
array satisfies all 3 requirements: it is contiguous, aligned, and not
byteswapped.  It is possible to request arrays satisfying any or none of the
behavedness requirements.  Arrays which do not satisfy the specified
requirements are transparently ``shadowed'' by temporary arrays which do
satisfy them.  By specifying \constant{NUM_UNCONVERTED}, a caller is certifying
that his extension function can correctly and directly handle the special cases
possible for a \class{NumArray}, excluding type differences.

\begin{verbatim}
typedef enum
{
        NUM_CONTIGUOUS=1,
        NUM_NOTSWAPPED=2,
        NUM_ALIGNED=4,
        NUM_WRITABLE=8,
        NUM_COPY=16,

        NUM_C_ARRAY  = (NUM_CONTIGUOUS | NUM_ALIGNED | NUM_NOTSWAPPED),
        NUM_UNCONVERTED = 0
}
\end{verbatim}

\function{NA_InputArray} will return a guaranteed writable result if
\constant{NUM_WRITABLE} is specified. A writable temporary will be made for
arrays which have readonly buffers.  Any changes made to a writable input array
\emph{may} be lost at extension exit time depending on whether or not a
temporary was required.  \function{NA_InputArray} will also return a guaranteed
writable result by specifying \constant{NUM_COPY}; with \constant{NUM_COPY}, a
temporary is \emph{always} made and changes to it are \emph{always} lost at
extension exit time.

Omitting \constant{NUM_WRITABLE} and \constant{NUM_COPY} from the
\var{requires} of \function{NA_InputArray} asserts that you will not modify the
array buffer in your C code.  Readonly arrays (e.g. from a readonly memory map)
which you attempt to modify can result in a segfault if \constant{NUM_WRITABLE}
or \constant{NUM_COPY} was not specified.

Arrays passed to \function{NA_IoArray} and \function{NA_OutputArray} must be
writable or they will raise an exception; specifing \constant{NUM_WRITABLE} or
\constant{NUM_COPY} to these functions has no effect.

\subsection{Example}
\label{sec:C-API:high-level:example}

A C wrapper function using the high-level API would typically look like the
following.\footnote{This function is taken from the convolve example in the
source distribution.}

\begin{verbatim}
#include "Python.h"
#include "libnumarray.h"

static PyObject *
Py_Convolve1d(PyObject *obj, PyObject *args)
{
        PyObject   *okernel, *odata, *oconvolved=Py_None;
        PyArrayObject *kernel, *data, *convolved;

        if (!PyArg_ParseTuple(args, "OO|O", &okernel, &odata, &oconvolved)) {
                PyErr_Format(_convolveError, 
                             "Convolve1d: Invalid parameters.");
                goto _fail;
        }

\end{verbatim}

First, define local variables and parse parameters.  \cfunction{Py_Convolve1d}
expects two or three array parameters in \var{args}: the convolution kernel,
the data, and optionally the return array.  We define two variables for each
array parameter, one which represents an arbitrary sequence object, and one
which represents a PyArrayObject which contains a conversion of the sequence.
If the sequence object was already a well-behaved numarray, it is returned
without making a copy.

\begin{verbatim}
        /* Align, Byteswap, Contiguous, Typeconvert */
        kernel  = NA_InputArray(okernel, tFloat64, NUM_C_ARRAY);
        data    = NA_InputArray(odata, tFloat64, NUM_C_ARRAY);
        convolved = NA_OptionalOutputArray(oconvolved, tFloat64, NUM_C_ARRAY, data);

        if (!kernel || !data || !convolved) {
                PyErr_Format( _convolveError, 
                             "Convolve1d: error converting array inputs.");
                goto _fail;
        }
\end{verbatim}

These calls to NA_InputArray and OptionalOutputArray require that the arrays be
aligned, contiguous, and not byteswapped, and of type Float64, or a temporary
will be created.  If the user hasn't provided a output array we ask
\cfunction{NA_OptionalOutputArray} to create a copy of the input \var{data}.
We also check that the array screening and conversion process succeeded by
verifying that none of the array pointers is NULL.

\begin{verbatim}
        if ((kernel->nd != 1) || (data->nd != 1)) {
                PyErr_Format(_convolveError,
                      "Convolve1d: arrays must have 1 dimension.");
                goto _fail;
        }

        if (!NA_ShapeEqual(data, convolved)) {
                PyErr_Format(_convolveError,
                "Convolve1d: data and output arrays need identitcal shapes.");
                goto _fail;
        }
\end{verbatim}

Make sure we were passed one-dimensional arrays, and data and output have the
same size.

\begin{verbatim}
        Convolve1d(kernel->dimensions[0], NA_OFFSETDATA(kernel),
                   data->dimensions[0],   NA_OFFSETDATA(data),
                   NA_OFFSETDATA(convolved));
\end{verbatim}

Call the C function actually performing the work.  NA_OFFSETDATA returns the
pointer to the first element of the array,  adjusting for any byteoffset.

\begin{verbatim}
        Py_XDECREF(kernel);
        Py_XDECREF(data);
\end{verbatim}

Decrease the reference counters of the input arrays.  These were increased by
\cfunction{NA_InputArray}.  Py_XDECREF tolerates NULL.  DECREF'ing the
PyArrayObject is how temporaries are released and in the case of
IO and Output arrays, copied back onto the original.

\begin{verbatim}
        /* Align, Byteswap, Contiguous, Typeconvert */
        return NA_ReturnOutput(oconvolved, convolved);
_fail:
        Py_XDECREF(kernel);
        Py_XDECREF(data);
        Py_XDECREF(convolved);
        return NULL;
}
\end{verbatim}

Now return the results, which are either stored in the user-supplied array
\var{oconvolved} and \constant{Py_None} is returned, or if the user didn't
supply an output array the temporary \var{convolved} is returned.

If your C function creates the output array you can use the following sequence
to pass this array back to \python{}:

\begin{verbatim}
        double *result;
        int m, n;
        .
        .
        .
        result = func(...);
        if(NULL == result)
            return NULL;
        return NA_NewArray((void *)result, tFloat64, 2, m, n);
}
\end{verbatim}

The C function \cfunction{func} returns a newly allocated (m, n) array in
\var{result}.  After we check that everything is ok, we create a new numarray
using \cfunction{NA_NewArray} and pass it back to \python.  \cfunction{NA_NewArray}
creates a \class{numarray} with \constant{NUM_C_ARRAY} properties.  If you wish to
create an array that is byte-swapped, or misaligned, you can use
\cfunction{NA_NewAll}.

The C-code of the core convolution function is uninteresting.  The main point
of the example is that when using the high-level API, numarray specific code is
confined to the wrapper function.  The interface for the core function can be
written in terms of primitive numarray/C data items, not objects.  This is
possible because the high level API can be used to deliver C arrays.

\begin{verbatim}
static void Convolve1d(long ksizex, Float64 *kernel, 
     long dsizex, Float64*data, Float64 *convolved) 
{ 
  long xc; long halfk = ksizex/2;

  for(xc=0; xc<halfk; xc++)
      convolved[xc] = data[xc];
  
  for(xc=halfk; xc<dsizex-halfk; xc++) {
      long xk;
      double temp = 0;
      for (xk=0; xk<ksizex; xk++)
         temp += kernel[xk]*data[xc-halfk+xk];
      convolved[xc] = temp;
  }
  
  for(xc=dsizex-halfk; xc<dsizex; xc++)
     convolved[xc] = data[xc];
}
\end{verbatim}

\section{Element-wise API}
\label{sec:C-API:element-wise-api}

The element-wise in-place API is a family of macros and functions designed to
get and set elements of arrays which might be byteswapped, misaligned,
discontiguous, or of a different type.  You can obtain \class{PyArrayObject}s
for these misbehaved arrays from the high-level API by specifying fewer
requirements (perhaps just 0, rather than NUM_C_ARRAY).  In this way, you can
avoid the creation of temporaries at a cost of accessing your array with these
macros and functions and a significant performance penalty.  Make no mistake,
if you have the memory, the high level API is the fastest.  The whole point of
this API is to support cases where the creation of temporaries exhausts either
the physical or virtual address space.  Exhausting physical memory will result
in thrashing, while exhausting the virtual address space will result in program
exception and failure.  This API supports avoiding the creation of the
temporaries, and thus avoids exhausting physical and virual memory, possibly
improving net performance or even enabling program success where simpler
methods would just fail.

\subsection{Element-wise functions}
\label{sec:C-API:element-wise:functions}

The single element macros each access one element of an array at a time, and
specify the array type in two places: as part of the PyArrayObject type
descriptor, and as ``type''.  The former defines what the array is, and the
latter is required to produce correct code from the macro.  They should
\emph{match}.  When you pass ``type'' into one of these macros, you are
defining the kind of array the code can operate on.  It is an error to pass a
non-matching array to one of these macros.  One last piece of advice: call
these macros carefully, because the resulting expansions and error messages are
a *obscene*.  Note: the type parameter for a macro is one of the Numarray
Numeric Data Types, not a NumarrayType enumeration value.

\subsubsection{Pointer based single element macros}
\label{sec:C-API:pointer-based-single}

\begin{cfuncdesc}{}{NA_GETPa}{PyArrayObject*, type, char*}
   aligning
\end{cfuncdesc}
\begin{cfuncdesc}{}{NA_GETPb}{PyArrayObject*, type, char*}
   byteswapping
\end{cfuncdesc}
\begin{cfuncdesc}{}{NA_GETPf}{PyArrayObject*, type, char*}
   fast (well-behaved)
\end{cfuncdesc}
\begin{cfuncdesc}{}{NA_GETP}{PyArrayObject*,  type, char*}
   testing: any of above
\end{cfuncdesc}
\begin{cfuncdesc}{}{NA_SETPa}{PyArrayObject*, type, char*, v}
\end{cfuncdesc}
\begin{cfuncdesc}{}{NA_SETPb}{PyArrayObject*, type, char*, v}
\end{cfuncdesc}
\begin{cfuncdesc}{}{NA_SETPf}{PyArrayObject*, type, char*, v}
\end{cfuncdesc}
\begin{cfuncdesc}{}{NA_SETP}{PyArrayObject*,  type, char*, v}
\end{cfuncdesc}

\subsubsection{One index single element macros}
\begin{cfuncdesc}{}{NA_GET1a}{PyArrayObject*, type, i}
\end{cfuncdesc}
\begin{cfuncdesc}{}{NA_GET1b}{PyArrayObject*, type, i}
\end{cfuncdesc}
\begin{cfuncdesc}{}{NA_GET1f}{PyArrayObject*, type, i}
\end{cfuncdesc}
\begin{cfuncdesc}{}{NA_GET1}{PyArrayObject*,  type, i}
\end{cfuncdesc}
\begin{cfuncdesc}{}{NA_SET1a}{PyArrayObject*, type, i, v}
\end{cfuncdesc}
\begin{cfuncdesc}{}{NA_SET1b}{PyArrayObject*, type, i, v}
\end{cfuncdesc}
\begin{cfuncdesc}{}{NA_SET1f}{PyArrayObject*, type, i, v}
\end{cfuncdesc}
\begin{cfuncdesc}{}{NA_SET1}{PyArrayObject*,  type, i, v}
\end{cfuncdesc}

\subsubsection{Two index single element macros}
\begin{cfuncdesc}{}{NA_GET2a}{PyArrayObject*, type, i, j}
\end{cfuncdesc}
\begin{cfuncdesc}{}{NA_GET2b}{PyArrayObject*, type, i, j}
\end{cfuncdesc}
\begin{cfuncdesc}{}{NA_GET2f}{PyArrayObject*, type, i, j}
\end{cfuncdesc}
\begin{cfuncdesc}{}{NA_GET2}{PyArrayObject*,  type, i, j}
\end{cfuncdesc}
\begin{cfuncdesc}{}{NA_SET2a}{PyArrayObject*, type, i, j, v}
\end{cfuncdesc}
\begin{cfuncdesc}{}{NA_SET2b}{PyArrayObject*, type, i, j, v}
\end{cfuncdesc}
\begin{cfuncdesc}{}{NA_SET2f}{PyArrayObject*, type, i, j, v}
\end{cfuncdesc}
\begin{cfuncdesc}{}{NA_SET2}{PyArrayObject*,  type, i, j, v}
\end{cfuncdesc}

\subsubsection{One and Two Index, Offset, Float64/Complex64/Int64 functions}

The \class{Int64}/\class{Float64}/\class{Complex64} functions require a
function call to access a single element of an array, making them slower than
the single element macros.  They have two advantages:
\begin{enumerate}
\item They're function calls, so they're a little more robust. 
\item They can handle \emph{any} input array type and behavior properties.
\end{enumerate}


While these functions have no error return status, they *can* alter the Python
error state, so well written extensions should call
\cfunction{PyErr_Occurred()} to determine if an error occurred and report it.
It's reasonable to do this check once at the end of an extension function,
rather than on a per-element basis.


\begin{cfuncdesc}{}{void NA_get_offset}{PyArrayObject *, int N, ...}
  \cfunction{NA_get_offset} computes the offset into an array object given a
  variable number of indices.  It is not especially robust, and it is
  considered an error to pass it more indices than the array has, or indices
  which are negative or out of range.
\end{cfuncdesc}

\begin{cfuncdesc}{Float64}{NA_get_Float64}{PyArrayObject *, long offset}
\end{cfuncdesc}
\begin{cfuncdesc}{void}{NA_set_Float64}{PyArrayObject *, long offset, Float64 v}
\end{cfuncdesc}
\begin{cfuncdesc}{Float64}{NA_get1_Float64}{PyArrayObject *, int i}
\end{cfuncdesc}
\begin{cfuncdesc}{void}{NA_set1_Float64}{PyArrayObject *, int i, Float64 v}
\end{cfuncdesc}
\begin{cfuncdesc}{Float64}{NA_get2_Float64}{PyArrayObject *, int i, int j}
\end{cfuncdesc}
\begin{cfuncdesc}{void}{NA_set2_Float64}{PyArrayObject *, int i, int j, Float64 v}
\end{cfuncdesc}

\begin{cfuncdesc}{Int64}{NA_get_Int64}{PyArrayObject *, long offset}
\end{cfuncdesc}
\begin{cfuncdesc}{void}{NA_set_Int64}{PyArrayObject *, long offset, Int64 v}
\end{cfuncdesc}
\begin{cfuncdesc}{Int64}{NA_get1_Int64}{PyArrayObject *, int i}
\end{cfuncdesc}
\begin{cfuncdesc}{void}{NA_set1_Int64}{PyArrayObject *, int i, Int64 v}
\end{cfuncdesc}
\begin{cfuncdesc}{Int64}{NA_get2_Int64}{PyArrayObject *, int i, int j}
\end{cfuncdesc}
\begin{cfuncdesc}{void}{NA_set2_Int64}{PyArrayObject *, int i, int j, Int64 v}
\end{cfuncdesc}

\begin{cfuncdesc}{Complex64}{NA_get_Complex64}{PyArrayObject *, long offset}
\end{cfuncdesc}
\begin{cfuncdesc}{void}{NA_set_Complex64}{PyArrayObject *, long offset, Complex64 v}
\end{cfuncdesc}
\begin{cfuncdesc}{Complex64}{NA_get1_Complex64}{PyArrayObject *, int i}
\end{cfuncdesc}
\begin{cfuncdesc}{void}{NA_set1_Complex64}{PyArrayObject *, int i, Complex64 v}
\end{cfuncdesc}
\begin{cfuncdesc}{Complex64}{NA_get2_Complex64}{PyArrayObject *, int i, int j}
\end{cfuncdesc}
\begin{cfuncdesc}{void}{NA_set2_Complex64}{PyArrayObject *, int i, int j, Complex64 v}
\end{cfuncdesc}

\subsection{Example}
\label{sec:C-API:element-wise:example}

The \cfunction{convolve1D} wrapper function corresponding to section
\ref{sec:C-API:high-level:example} using the element-wise API could look
like:\footnote{This function is also available as an example in the source
   distribution.}

\begin{verbatim}
static PyObject *
Py_Convolve1d(PyObject *obj, PyObject *args)
{
        PyObject   *okernel, *odata, *oconvolved=Py_None;
        PyArrayObject *kernel, *data, *convolved;

        if (!PyArg_ParseTuple(args, "OO|O", &okernel, &odata, &oconvolved)) {
                PyErr_Format(_Error, "Convolve1d: Invalid parameters.");
                goto _fail;
        }

        kernel = NA_InputArray(okernel, tAny, 0);
        data   = NA_InputArray(odata, tAny, 0);
\end{verbatim}

For the kernel and data arrays, \class{numarray}s of any type are accepted
without conversion.  Thus there is no copy of the data made except for lists or
tuples.  All types, byteswapping, misalignment, and discontiguity must be
handled by Convolve1d.  This can be done easily using the get/set functions.
Macros, while faster than the functions, can only handle a single type.

\begin{verbatim}
        convolved = NA_OptionalOutputArray(oconvolved, tFloat64, 0, data);
\end{verbatim}

Also for the output array we accept any variety of type tFloat without
conversion.  No copy is made except for non-tFloat.  Non-numarray sequences are
not permitted as output arrays.  Byteswaping, misaligment, and discontiguity
must be handled by Convolve1d.  If the \python caller did not specify the
oconvolved array, it initially retains the value Py_None.  In that case,
\var{convolved} is cloned from the array \var{data} using the specified type.
It is important to clone from \var{data} and not \var{odata}, since the latter
may be an ordinary \python sequence which was converted into numarray
\var{data}.  

\begin{verbatim}
        if (!kernel || !data || !convolved)
                goto _fail;

        if ((kernel->nd != 1) || (data->nd != 1)) {
                PyErr_Format(_Error,
                     "Convolve1d: arrays must have exactly 1 dimension.");
                goto _fail;
        }

        if (!NA_ShapeEqual(data, convolved)) {
                PyErr_Format(_Error,
                    "Convolve1d: data and output arrays must have identical length.");
                goto _fail;
        }
        if (!NA_ShapeLessThan(kernel, data)) {
                PyErr_Format(_Error,
                    "Convolve1d: kernel must be smaller than data in both dimensions");
                goto _fail;
        }
        
        if (Convolve1d(kernel, data, convolved) < 0)  /* Error? */
            goto _fail;
        else {
           Py_XDECREF(kernel);
           Py_XDECREF(data);
           return NA_ReturnOutput(oconvolved, convolved);
        }
_fail:
        Py_XDECREF(kernel);
        Py_XDECREF(data);
        Py_XDECREF(convolved);
        return NULL;
}

\end{verbatim}

This function is very similar to the high-level API wrapper, the notable
difference is that we ask for the unconverted arrays \var{kernel} and
\var{data} and \var{convolved}.  This requires some attention in their usage.
The function that does the actual convolution in the example has to use
\cfunction{NA_get*} to read and \cfunction{NA_set*} to set an element of these
arrays, instead of using straight array notation.  These functions perform any
necessary type conversion, byteswapping, and alignment.

\begin{verbatim}
static int
Convolve1d(PyArrayObject *kernel, PyArrayObject *data, PyArrayObject *convolved)
{
        int xc, xk;
        int ksizex = kernel->dimensions[0];
        int halfk = ksizex / 2;
        int dsizex = data->dimensions[0];

        for(xc=0; xc<halfk; xc++)
                NA_set1_Float64(convolved, xc, NA_get1_Float64(data, xc));
                     
        for(xc=dsizex-halfk; xc<dsizex; xc++)
                NA_set1_Float64(convolved, xc, NA_get1_Float64(data, xc));

        for(xc=halfk; xc<dsizex-halfk; xc++) {
                Float64 temp = 0;
                for (xk=0; xk<ksizex; xk++) {
                        int i = xc - halfk + xk;
                        temp += NA_get1_Float64(kernel, xk) * 
                                NA_get1_Float64(data, i);
                }
                NA_set1_Float64(convolved, xc, temp);
        }
        if (PyErr_Occurred())
           return -1;
        else 
           return 0;
}
\end{verbatim}

\section{One-dimensional API}
\label{sec:C-API:One-dimensional-api}

The 1D in-place API is a set of functions for getting/setting elements from the
innermost dimension of an array.  These functions improve speed by moving type
switches, ``behavior tests'', and function calls out of the per-element loop.
The functions get/set a series of consequtive array elements to/from arrays of
\class{Int64}, \class{Float64}, or \class{Complex64}.  These functions are
(even) more intrusive than the single element functions, but have better
performance in many cases.  They can operate on arrays of any type, with the
exception of the Complex64 functions, which only handle Complex64.  The
functions return 0 on success and -1 on failure, with the Python error state
already set.  To be used profitably, the 1D API requires either a large single
dimension which can be processeed in blocks or a multi-dimensional array such
as an image.  In the latter case, the 1D API is suitable for processing one (or
more) scanlines at a time rather than the entire image at once.  See the source
distribution Examples/convolve/one_dimensionalmodule.c for an example of usage.

\begin{cfuncdesc}{long}{NA_get_offset}{PyArrayObject *, int N, ...}
   This function applies a (variable length) set of \var{N} indices to an array
   and returns a byte offset into the array.
\end{cfuncdesc}

\begin{cfuncdesc}{int}{NA_get1D_Int64}{%
      PyArrayObject *, long offset, int cnt, Int64 *out}%
\end{cfuncdesc}

\begin{cfuncdesc}{int}{NA_set1D_Int64}{%
      PyArrayObject *, long offset, int cnt, Int64 *in}%
\end{cfuncdesc}

\begin{cfuncdesc}{int}{NA_get1D_Float64}{%
      PyArrayObject *, long offset, int cnt, Float64 *out}%
\end{cfuncdesc}

\begin{cfuncdesc}{int}{NA_set1D_Float64}{%
      PyArrayObject *, long offset, int cnt, Float64 *in}%
\end{cfuncdesc}

\begin{cfuncdesc}{int}{NA_get1D_Complex64}{%
      PyArrayObject *, long offset, int cnt, Complex64 *out}%
\end{cfuncdesc}

\begin{cfuncdesc}{int}{NA_set1D_Complex64}{%
      PyArrayObject *, long offset, int cnt, Complex64 *in}%
\end{cfuncdesc}

\section{New numarray functions}
\label{sec:C-API:new-numarray-functions}

The following array creation functions share similar behavior.  All but one
create a new \class{numarray} using the data specified by \var{data}.  If
\var{data} is NULL, the routine allocates a buffer internally based on the
array shape and type; internally allocated buffers have undefined contents.
The data type of the created array is specified by \var{type}.

There are several functions to create \class{numarray}s at the C level:

\begin{cfuncdesc}{static PyArrayObject*}{NA_NewArray}{%
    void *data, NumarrayType type, int ndim, ...}% 

  \var{ndim} specifies the rank of the array (number of dimensions), and the
  length of each dimension must be given as the remaining (variable length)
  list of \emph{int} parameters.  The following example allocates a 100x100
  uninitialized array of Int32.
\begin{verbatim}
  if (!(array = NA_NewArray(NULL, tInt32, 2, 100, 100)))
      return NULL;
\end{verbatim}
\end{cfuncdesc}

\begin{cfuncdesc}{static PyObject*}{NA_vNewArray}{%
    void *data, NumarrayType type, int ndim, maybelong *shape}% 

  For \function{NA_vNewArray} the length of each dimension must be given in an
  array of \var{maybelong} pointed to by \var{shape}. The following code
  allocates a 2x2 array initialized to a copy of the specified \var{data}.
  \begin{verbatim}
    Int32 data[4] = { 1, 2, 3, 4 };
    maybelong shape[2] = { 2, 2 };
    if (!(array = NA_vNewArray(data, tInt32, 2, shape)))
       return NULL;
  \end{verbatim}
\end{cfuncdesc}

\begin{cfuncdesc}{static PyArrayObject*}{NA_NewAll}{%
    int ndim, maybelong *shape, NumarrayType type, void *data, maybelong
    byteoffset, maybelong bytestride, int byteorder, int aligned, int
    writable}%

    \function{NA_NewAll} is similar to \function{NA_vNewArray} except it
    provides for the specification of additional parameters. \var{byteoffset}
    specifies the byte offset from the base of the data array at which the
    \var{real} data begins.  \var{bytestride} specifies the miminum stride to
    use, the seperation in bytes between adjacent elements in the
    array. \var{byteorder} takes one of the values \constant{NUM_BIG_ENDIAN} or
    \constant{NUM_LITTLE_ENDIAN}.  \var{writable} defines whether the buffer
    object associated with the resulting array is readonly or writable.
\end{cfuncdesc}

\begin{cfuncdesc}{static PyArrayObject*}{NA_NewAllStrides}{%
    int ndim, maybelong *shape, maybelong *strides, NumarrayType type, void
    *data, maybelong byteoffset, maybelong byteorder, int aligned, int
    writable}% 

    \function{NA_NewAllStrides} is a variant of \function{NA_vNewAll} which
    also permits the specification of the array strides.  The strides are not
    checked for correctness.
\end{cfuncdesc}

\begin{cfuncdesc}{static PyArrayObject*}{NA_NewAllFromBuffer}{%
    int ndim, maybelong *shape, NumarrayType type, PyObject *buffer, maybelong
    byteoffset, maybelong bytestride, int byteorder, int aligned, int
    writable}% 

   \function{NA_NewAllFromBuffer} is similar to \function{NA_NewAll} except it
   accepts a buffer object rather than a pointer to C data.  The \var{buffer}
   object must support the buffer protocol.  If \var{buffer} is non-NULL, the
   returned array object stores a reference to \var{buffer} and locates its
   data there.  If \var{buffer} is specified as NULL, a buffer object and
   associated data space are allocated internally and the returned array object
   refers to that.  It is possible to create a Python buffer object from an
   array of C data and then construct a \class{numarray} using this function
   which refers to the C data without making a copy.
\end{cfuncdesc}

\begin{cfuncdesc}{int}{NA_ShapeEqual}{PyArrayObject*a,PyArrayObject*b}
This function compares the shapes of two arrays, and returns 1 if they
are the same, 0 otherwise.
\end{cfuncdesc}

\begin{cfuncdesc}{int}{NA_ShapeLessThan}{PyArrayObject*a,PyArrayObject*b}
This function compares the shapes of two arrays, and returns 1 if each
dimension of 'a' is less than the corresponding dimension of 'b', 0 otherwise.
\end{cfuncdesc}

\begin{cfuncdesc}{int}{NA_ByteOrder}{}
This function returns the system byte order, either NUM_LITTLE_ENDIAN or
NUM_BIG_ENDIAN.
\end{cfuncdesc}

\begin{cfuncdesc}{Bool}{NA_IeeeMask32}{Float32 value, Int32 mask}
This function returns 1 IFF Float32 \var{value} matches any of the IEEE special
value criteria specified by \var{mask}.  See ieeespecial.h for the mask bit
values which can be or'ed together to specify mask.
\function{NA_IeeeSpecial32} has been deprecated and will eventually be removed.
\end{cfuncdesc}

\begin{cfuncdesc}{Bool}{NA_IeeeMask64}{Float64 value,Int32 mask}
This function returns 1 IFF Float64 \var{value} matches any of the IEEE special
value criteria specified by \var{mask}.  See ieeespecial.h for the mask bit
values which can be or'ed together to specify mask.
\function{NA_IeeeSpecial64} has been deprecated and will eventually be removed.
\end{cfuncdesc}

\begin{cfuncdesc}{PyArrayObject *}{NA_updateDataPtr}{PyArrayObject *}
This function updates the values derived from the ``_data'' buffer, namely the
data pointer and buffer WRITABLE flag bit.  This needs to be called upon
entering or re-entering C-code from Python, since it is possible for buffer
objects to move their data buffers as a result of executing arbitrary Python
and hence arbitrary C-code.  The high level interface routines,
e.g. \function{NA_InputArray}, call this routine automatically.
\end{cfuncdesc}

\begin{cfuncdesc}{char*}{NA_typeNoToName}{int}
NA_typeNoToName translates a NumarrayType into a character string which can be
used to display it:  e.g.  tInt32 converts to the string ``Int32''
\end{cfuncdesc}

\begin{cfuncdesc}{PyObject*}{NA_typeNoToTypeObject}{int}
This function converts a NumarrayType C type code into the NumericType object
which implements and represents it more fully.  tInt32 converts to the type
object numarray.Int32.  
\end{cfuncdesc}

\begin{cfuncdesc}{int}{NA_typeObjectToTypeNo}{PyObject*}
This function converts a numarray type object (e.g. numarray.Int32) into the
corresponding NumarrayType (e.g. tInt32) C type code. 
\end{cfuncdesc}

\begin{cfuncdesc} {PyObject*}{NA_intTupleFromMaybeLongs}{int,maybelong*}
This function creates a tuple of Python ints from an array of C maybelong integers.
\end{cfuncdesc}

\begin{cfuncdesc}{long}{NA_maybeLongsFromIntTuple}{int,maybelong*,PyObject*}
This function fills an array of C long integers with the converted values from
a tuple of Python ints.  It returns the number of conversions, or -1 for error.
\end{cfuncdesc}

\begin{cfuncdesc}{long}{NA_isIntegerSequence}{PyObject*}
This function returns 1 iff the single parameter is a sequence of Python
integers, and 0 otherwise.
\end{cfuncdesc}

\begin{cfuncdesc}{PyObject*}{NA_setArrayFromSequence}{PyArrayObject*,PyObject*}
This function copies the elementwise from a sequence object to a numarray.
\end{cfuncdesc}

\begin{cfuncdesc}{int}{NA_maxType}{PyObject*}
This function returns an integer code corresponding to the highest kind of
Python numeric object in a sequence.  INT(0) LONG(1) FLOAT(2) COMPLEX(3).
On error -1 is returned.
\end{cfuncdesc}

\begin{cfuncdesc}{PyObject*}{NA_getPythonScalar}{PyArrayObject *a, long offset}
This function returns the Python object corresponding to the single element of 
the array 'a' at the given byte offset.
\end{cfuncdesc}

\begin{cfuncdesc}{int}{NA_setFromPythonScalar}{PyArrayObject *a, long offset, PyObject*value}
This function sets the single element of the array 'a' at the given byte
offset to 'value'.
\end{cfuncdesc}

\begin{cfuncdesc}{int}{NA_NDArrayCheck}{PyObject*o}
This function returns 1 iff the 'o' is an instance of NDArray or an instance of
a subclass of NDArray, and 0 otherwise.
\end{cfuncdesc}

\begin{cfuncdesc}{int}{NA_NumArrayCheck}{PyObject*}
This function returns 1 iff the 'o' is an instance of NumArray or an instance of
a subclass of NumArray, and 0 otherwise.
\end{cfuncdesc}

\begin{cfuncdesc}{int}{NA_ComplexArrayCheck}{PyObject*}
This function returns 1 iff the 'o' is an instance of ComplexArray or an instance of
a subclass of ComplexArray, and 0 otherwise.
\end{cfuncdesc}

\begin{cfuncdesc}{unsigned long}{NA_elements}{PyArrayObject*}
This function returns the total count of elements in an array,  essentially the
product of the elements of the array's shape.
\end{cfuncdesc}

\begin{cfuncdesc}{PyArrayObject *}{NA_copy}{PyArrayObject*}
This function returns a copy of the given array.  The array copy is guaranteed
to be well-behaved, i.e. neither byteswapped, misaligned, nor discontiguous.
\end{cfuncdesc}

\begin{cfuncdesc}{int}{NA_copyArray}{PyArrayObject*to, const PyArrayObject *from}
This function returns a copies one array onto another;  used in f2py.
\end{cfuncdesc}

\begin{cfuncdesc}{int}{NA_swapAxes}{PyArrayObject*a, int dim1, int dim2}
This function mutates the specified array \var{a} to exchange the shape and
strides values for the two dimensions, \var{dim1} and \var{dim2}.  Negative
dimensions count backwards from the innermost, with -1 being the innermost
dimension.  Returns 0 on success and -1 on error.
\end{cfuncdesc}

%% Local Variables:
%% mode: LaTeX
%% mode: auto-fill
%% fill-column: 79
%% indent-tabs-mode: nil
%% ispell-dictionary: "american"
%% reftex-fref-is-default: nil
%% TeX-auto-save: t
%% TeX-command-default: "pdfeLaTeX"
%% TeX-master: "numarray"
%% TeX-parse-self: t
%% End:




\part{Extension modules}

\label{part:numarray-extensions}

\chapter{Convolution}
\label{cha:convolve}

%begin{latexonly}
\makeatletter
\py@reset
\makeatother
%end{latexonly}
\declaremodule{extension}{numarray.convolve}
\moduleauthor{The numarray team}{numpy-discussion@lists.sourceforge.net}
\modulesynopsis{Convolution,correlation}

\begin{quote}
   This package (numarray.convolve) provides functions for one- and 
   two-dimensional convolutions and correlations of \class{numarray}s.
   Each of the following examples assumes that the following code has been 
   executed:
\begin{verbatim}
import numarray.convolve as conv
\end{verbatim}
\end{quote}


\section{Convolution functions}
\label{sec:CONV:convolution-functions}

\begin{funcdesc}{boxcar}{data, boxshape, output=None, mode='nearest', cval=0.0}
   \function{boxcar} computes a 1-D or 2-D boxcar filter on every 1-D or
   2-D subarray of \constant{data}. \constant{boxshape} is a tuple of integers
   specifying the dimensions of the filter, e.g. \code{(3,3)}.  If
   \constant{output} is specified, it should be the same shape as
   \constant{data} and the result will be stored in it.  In that case
   \class{None} will be returned.
        
   \constant{mode} can be any of the following values:
   \begin{description}
   \item[\var{nearest}]: Elements beyond boundary come from nearest edge pixel.
   \item[\var{wrap}]: Elements beyond boundary come from the opposite array
      edge.
   \item[\var{reflect}]: Elements beyond boundary come from reflection on same
      array edge.
   \item[\var{constant}]: Elements beyond boundary are set to what is specified
      in \constant{cval}, an optional numerical parameter; the default value is
      \code{0.0}.
   \end{description}        
\end{funcdesc}
\begin{verbatim}
>>> print a
[1 5 4 7 2 9 3 6]
>>> print conv.boxcar(a,(3,))
[ 2.33333333  3.33333333  5.33333333  4.33333333  6.          4.66666667
  6.          5.        ]
# for even number box size, it will take the extra point from the lower end
>>> print conv.boxcar(a,(2,))
[ 1.   3.   4.5  5.5  4.5  5.5  6.   4.5]
\end{verbatim}

\begin{funcdesc}{convolve}{data, kernel, mode=FULL}
   \label{func:CONV:convolve}
   Returns the discrete, linear convolution of 1-D sequences \constant{data} 
   and \constant{kernel}; \constant{mode} can be \class{VALID}, \class{SAME}, 
   or \code{FULL} to specify the size of the resulting sequence.  See section
   \ref{sec:CONV:global-constants}.
\end{funcdesc}

\begin{funcdesc}{convolve2d}{data, kernel, output=None, fft=0, mode='nearest', 
    cval=0.0} Return the 2-dimensional convolution of \constant{data} and
  \constant{kernel}.  If \constant{output} is not \class{None}, the result is
  stored in \constant{output} and \class{None} is returned.  \constant{fft} is
  used to switch between FFT-based convolution and the naive algorithm,
  defaulting to naive.  Using \constant{fft} mode becomes more beneficial as
  the size of the kernel grows; for small kernels, the naive algorithm is more
  efficient.  \constant{mode} has the same choices as those of
  \function{boxcar}.  A number of storage considerations come into play with
  large arrays: (1) boundary modes are implemented by making an oversized
  temporary copy of the \constant{data} array which has a shape equal to the
  sum of the \constant{data} and \constant{kernel} shapes.  (2) likewise, the
  \constant{kernel} is copied into an array with the same shape as the
  oversized \constant{data} array.  (3) In FFT mode, the fourier transforms of
  the \constant{data} and \constant{kernel} arrays are stored in double
  precision complex temporaries. The aggregate effect is that storage roughly
  equal to a factor of eight (x2 from 2 and x4 from 3) times the size of the
  \constant{data} is required to compute the convolution of a Float32
  \constant{data} array.
\end{funcdesc}

\begin{funcdesc}{correlate}{data, kernel, mode=FULL}
   Return the cross-correlation of \constant{data} and \constant{kernel};
   \constant{mode} can be \class{VALID}, \class{SAME}, or \code{FULL} to 
   specify the size of the resulting sequence.  \function{correlate} is
   very closely related to \function{convolve} in implementation.
   See section \ref{sec:CONV:global-constants}.
\end{funcdesc}

\begin{funcdesc}{correlate2d}{data, kernel, output=None, fft=0, mode='nearest', cval=0.0}
   \label{func:CONV:correlate2d}
  Return the 2-dimensional convolution of \constant{data} and
  \constant{kernel}.  If \constant{output} is not \class{None}, the result is
  stored in \constant{output} and \class{None} is returned.  \constant{fft} is
  used to switch between FFT-based convolution and the naive algorithm,
  defaulting to naive.  Using \constant{fft} mode becomes more beneficial as
  the size of the kernel grows; for small kernels, the naive algorithm is more
  efficient.  \constant{mode} has the same choices as those of
  \function{boxcar}.  See also \function{convolve2d} for notes regarding 
  storage consumption.
\end{funcdesc}

\note{\function{cross_correlate} is deprecated and should not be used.}



\section{Global constants}
\label{sec:CONV:global-constants}

These constants specify what part of the result the \function{convolve} and
\function{correlate} functions of this module return.  Each of the following
examples assumes that the following code has been executed:

\begin{verbatim}
arr = numarray.arange(8)
\end{verbatim}

\begin{datadesc}{FULL}
   Return the full convolution or correlation of two arrays.
\begin{verbatim}
>>> conv.correlate(arr, [1, 2, 3], mode=conv.FULL)
array([ 0,  3,  8, 14, 20, 26, 32, 38, 20,  7])
\end{verbatim}
\end{datadesc}

\begin{datadesc}{PASS}
   Correlate the arrays without padding the data.
\begin{verbatim}
>>> conv.correlate(arr, [1, 2, 3], mode=conv.PASS)
array([ 0,  8, 14, 20, 26, 32, 38,  7])
\end{verbatim}
\end{datadesc}

\begin{datadesc}{SAME}
   Return the part of the convolution or correlation of two arrays that
   corresponds to an array of the same shape as the input data.
\begin{verbatim}
>>> conv.correlate(arr, [1, 2, 3], mode=conv.SAME)
array([ 3,  8, 14, 20, 26, 32, 38, 20])
\end{verbatim}
\end{datadesc}

\begin{datadesc}{VALID}
   Return the valid part of the convolution or correlation of two arrays.
\begin{verbatim}
>>> conv.correlate(arr, [1, 2, 3], mode=conv.VALID)
array([ 8, 14, 20, 26, 32, 38])
\end{verbatim}
\end{datadesc}



%% Local Variables:
%% mode: LaTeX
%% mode: auto-fill
%% fill-column: 79
%% indent-tabs-mode: nil
%% ispell-dictionary: "american"
%% reftex-fref-is-default: nil
%% TeX-auto-save: t
%% TeX-command-default: "pdfeLaTeX"
%% TeX-master: "numarray"
%% TeX-parse-self: t
%% End:

\chapter{Fast-Fourier-Transform}
\label{cha:fft}

%begin{latexonly}
\makeatletter \py@reset \makeatother
%end{latexonly}
\declaremodule{extension}{numarray.fft}
\moduleauthor{The numarray team}{numpy-discussion@lists.sourceforge.net}
\modulesynopsis{Fast Fourier Transform}

\begin{quote}
   This package provides functions for one- and two-dimensional
   Fast-Fourier-Transforms (FFT) and inverse FFTs.
\end{quote}


The \module{numarray.fft} module provides a simple interface to the FFTPACK
Fortran library, which is a powerful standard library for doing fast Fourier
transforms of real and complex data sets, or the C fftpack library, which is
algorithmically based on FFTPACK and provides a compatible interface.


\section{Installation}
\label{sec:FFT:installation}

The default installation uses the provided \module{numarray.fft.fftpack} C
implementation of these routines and this works without any further
interaction.


\subsection{Installation using FFTPACK}
\label{sec:FFT:install-lapack}

On some platforms, precompiled optimized versions of the FFTPACK libraries are
preinstalled on the operating system, and the setup procedure needs to be
modified to force the \module{numarray.fft} module to be linked against those
rather than the builtin replacement functions.




\section{FFT Python Interface}
\label{sec:FFT:python-interface}

The Python user imports the \module{numarray.fft} module, which provides 
a set of utility functions of the most commonly used FFT routines, and
allows the specification of which axes (dimensions) of the input arrays are to
be used for the FFT's. These routines are:

\begin{funcdesc}{fft}{a, n=None, axis=-1} 
   Performs a \constant{n}-point discrete Fourier transform of the array 
   \constant{a}, \constant{n} defaults to the size of \constant{a}. It is 
   most efficient for \constant{n} a power of two. If \constant{n} is 
   larger than \code{len(a)}, then \constant{a} will be
   zero-padded to make up the difference. If \constant{n} is smaller than
   \code{len(a)}, then \constant{a} will be aliased to reduce its size. This
   also stores a cache of working memory for different sizes of 
   \module{fft}'s, so you could theoretically run into memory problems if 
   you call this too many times with too many different \constant{n}'s.
   
   The FFT is performed along the axis indicated by the \constant{axis} 
   argument, which defaults to be the last dimension of \constant{a}.
   
   The format of the returned array is a complex array of the same shape as
   \constant{a}, where the first element in the result array contains the DC
   (steady-state) value of the FFT.
   \remark{missing: ..., and where each successive ...}

   Some examples are:
\begin{verbatim}
>>> a = array([1., 0., 1., 0., 1., 0., 1., 0.]) + 10
>>> b = array([0., 1., 0., 1., 0., 1., 0., 1.]) + 10
>>> c = array([0., 1., 0., 0., 0., 1., 0., 0.]) + 10
>>> print numarray.fft.fft(a).real
[ 84.   0.   0.   0.   4.   0.   0.   0.]
>>> print numarray.fft.fft(b).real
[ 84.   0.   0.   0.  -4.   0.   0.   0.]
>>> print numarray.fft.fft(c).real
[ 82.   0.   0.   0.  -2.   0.   0.   0.]
\end{verbatim}
\end{funcdesc}
       
\begin{funcdesc}{inverse_fft}{a, n=None, axis=-1}
   Will return the \constant{n} point inverse discrete Fourier transform of
   \constant{a}; \constant{n} defaults to the length of \constant{a}. 
   It is most efficient for \constant{n} a power of two.  If \constant{n} 
   is larger than \constant{a}, then \constant{a} will be zero-padded to 
   make up the difference.  If \constant{n} is smaller than \constant{a}, 
   then \constant{a} will be aliased to reduce its size.
   This also stores a cache of working memory for different sizes of FFT's, so
   you could theoretically run into memory problems if you call this too many
   times with too many different \constant{n}'s.
\end{funcdesc}
       
\begin{funcdesc}{real_fft}{a, n=None, axis=-1}
   Will return the \constant{n} point discrete Fourier transform of the 
   real valued array \constant{a}; \constant{n} defaults to the length of 
   \constant{a}.  It is most efficient for \constant{n} a power of two.  
   The returned array will be one half of the symmetric complex transform of 
   the real array.

\begin{verbatim}
>>> x = cos(arange(30.0)/30.0*2*pi)
>>> print numarray.fft.real_fft(x)
[ -5.82867088e-16 +0.00000000e+00j   1.50000000e+01 -3.08862614e-15j
   7.13643755e-16 -1.04457106e-15j   1.13047653e-15 -3.23843935e-15j
  -1.52158521e-15 +1.14787259e-15j   3.60822483e-16 +3.60555504e-16j
   1.34237661e-15 +2.05127011e-15j   1.98981960e-16 -1.02472357e-15j
   1.55899290e-15 -9.94619821e-16j  -1.05417678e-15 -2.33364171e-17j
  -2.08166817e-16 +1.00955541e-15j  -1.34094426e-15 +8.88633386e-16j
   5.67513742e-16 -2.24823896e-15j   2.13735778e-15 -5.68448962e-16j
  -9.55398954e-16 +7.76890265e-16j  -1.05471187e-15 +0.00000000e+00j]
\end{verbatim}
\end{funcdesc}
       
\begin{funcdesc}{inverse_real_fft}{a, n=None, axis=-1}
   Will return the inverse FFT of the real valued array \constant{a}.
\end{funcdesc}
       
\begin{funcdesc}{fft2d}{a, s=None, axes=(-2,-1)}
   Will return the 2-dimensional FFT of the array \constant{a}.  This
   is really just \function{fft_nd()} with different default behavior.
\end{funcdesc}
       
\begin{funcdesc}{inverse_fft2d}{a, s=None, axes=(-2,-1)}
  The inverse of \function{fft2d()}. This is really just
  \function{inverse_fftnd()} with different default behavior.
\end{funcdesc}
       
\begin{funcdesc}{real_fft2d}{a, s=None, axes=(-2,-1)}
   Will return the 2-D FFT of the real valued array \constant{a}.
\end{funcdesc}
            
\begin{funcdesc}{inverse_real_fft2d}{a, s=None, axes=(-2,-1)}
  The inverse of \function{real_fft2d()}. This is really just
  \function{inverse_real_fftnd()} with different default behavior.
\end{funcdesc}

            
\section{fftpack Python Interface}
\label{sec:FFT:c-api}

%begin{latexonly}
\makeatletter \py@reset \makeatother
%end{latexonly}
\declaremodule{extension}{numarray.fft.fftpack}
\moduleauthor{The numarray team}{numpy-discussion@lists.sourceforge.net}
\modulesynopsis{Fast Fourier Transform}

The interface to the FFTPACK library is performed via the \module{fftpack}
module, which is responsible for making sure that the arrays sent to the
FFTPACK routines are in the right format (contiguous memory locations, right
numerical storage format, etc). It provides interfaces to the following FFTPACK
routines, which are also the names of the Python functions:
\begin{funcdesc}{cffti}{i}
\end{funcdesc}
\begin{funcdesc}{cfftf}{data, savearea}
\end{funcdesc}
\begin{funcdesc}{cfftb}{data, savearea}
\end{funcdesc}
\begin{funcdesc}{rffti}{i}
\end{funcdesc}
\begin{funcdesc}{rfftf}{data, savearea}
\end{funcdesc}
\begin{funcdesc}{rfftb}{data, savearea}
\end{funcdesc}
The routines which start with \texttt{c} expect arrays of complex numbers, the
routines which start with \texttt{r} expect real numbers only. The routines
which end with \texttt{i} are the initalization functions, those which end with
\texttt{f} perform the forward FFTs and those which end with \texttt{b} perform
the backwards FFTs.

The initialization functions require a single integer argument corresponding to
the size of the dataset, and returns a work array. The forward and backwards
FFTs require two array arguments -- the first is the data array, the second is
the work array returned by the initialization function. They return arrays
corresponding to the coefficients of the FFT, with the first element in the
returned array corresponding to the DC component, the second one to the first
fundamental, etc.The length of the returned array is 1 + half the length of the
input array in the case of real FFTs, and the same size as the input array in
the case of complex data.
\begin{verbatim}
>>> import numarray.fft.fftpack as fftpack
>>> x = cos(arange(30.0)/30.0*2*pi)
>>> w = fftpack.rffti(30)
>>> f = fftpack.rfftf(x, w)
>>> print f[0:5]
[ -5.68989300e-16 +0.00000000e+00j   1.50000000e+01 -3.08862614e-15j
        6.86516117e-16 -1.00588467e-15j   1.12688689e-15 -3.19983494e-15j
       -1.52158521e-15 +1.14787259e-15j]
\end{verbatim}



%% Local Variables:
%% mode: LaTeX
%% mode: auto-fill
%% fill-column: 79
%% indent-tabs-mode: nil
%% ispell-dictionary: "american"
%% reftex-fref-is-default: nil
%% TeX-auto-save: t
%% TeX-command-default: "pdfeLaTeX"
%% TeX-master: "numarray"
%% TeX-parse-self: t
%% End:

\chapter{Linear Algebra}
\label{cha:linear-algebra}

%begin{latexonly}
\makeatletter
\py@reset
\makeatother
%end{latexonly}
\declaremodule[numarray.linearalgebra]{extension}{numarray.linear_algebra}
\moduleauthor{The numarray team}{numpy-discussion@lists.sourceforge.net}
\modulesynopsis{Linear Algebra}

\begin{quote}
  The numarray.linear\_algebra module provides a simple interface to some
  commonly used linear algebra routines.
\end{quote}

The \module{numarray.linear_algebra} module provides a simple high-level
interface to some common linear algebra problems. It uses either the LAPACK
Fortran library or the compatible
\module{\mbox{numarray.linear_algebra.lapack_lite}} C library shipped with
\module{numarray}.

\section{Installation}
\label{sec:LA:installation}

The default installation uses the provided
\module{numarray.linear_algebra.lapack_lite} implementation of these routines
and this works without any further interaction.

Nevertheless if LAPACK is installed already or you are concerned about the
performance of these routines you should consider installing
\module{numarray.linear_algebra} to take advantage of the real LAPACK library.
See the next section for instructions.

\subsection{Installation using LAPACK}
\label{sec:LA:install-lapack}

On some platforms, precompiled optimized versions of the LAPACK and BLAS
libraries are preinstalled on the operating system, and the setup procedure
needs to be modified to force the \module{lapack_lite} module to be linked
against those rather than the builtin replacement functions.

Here's a recipe for building using LAPACK:

\begin{verbatim}
% setenv USE_LAPACK 1
% setenv LINALG_LIB <where your lapack, blas, atlas, etc are>
% setenv LINALG_INCLUDE <where your lapack, blas, atlas headers are>
% python setup.py install --selftest
\end{verbatim}

For your particular system and library installations, you may need to edit
\texttt{addons.py} and adjust the variables \texttt{sourcelist},
\texttt{lapack_dirs}, and \texttt{lapack_libs}.

\note{A frequent request is that somehow the maintainers of Numerical Python
   invent a procedure which will automatically find and use the \emph{best}
   available versions of these libraries.  We welcome any patches that provide
   the functionality in a simple, platform independent, and reliable way.  The
   \ulink{scipy}{http://www.scipy.org} project has done some work to provide
   such functionality, but is probably not mature enough for use by
   \module{numarray} yet.}


\section{Python Interface}
\label{sec:LA:python-interface}

All examples in this section assume that you performed a
\begin{verbatim}
from numarray import *
import numarray.linear_algebra as la
\end{verbatim}

\begin{funcdesc}{cholesky_decomposition}{a}
   This function returns a lower triangular matrix L which, when multiplied by
   its transpose yields the original matrix \code{a}; \code{a} must be 
   square, Hermitian, and positive definite. L is often referred to as the 
   Cholesky lower-triangular square-root of \code{a}.
\end{funcdesc}
 
\begin{funcdesc}{determinant}{a}
   This function returns the determinant of the square matrix \code{a}.
\begin{verbatim}
>>> print a
[[ 1  2]
 [ 3 15]]
>>> print la.determinant(a)
9.0
\end{verbatim}
\end{funcdesc}
 
\begin{funcdesc}{eigenvalues}{a}
   This function returns the eigenvalues of the square matrix \code{a}.
\begin{verbatim}
>>> print a
[[ 1.    0.    0.    0.  ]
 [ 0.    2.    0.    0.01]
 [ 0.    0.    5.    0.  ]
 [ 0.    0.01  0.    2.5 ]]
>>> print la.eigenvalues(a)
[ 2.50019992  1.99980008  1.          5.        ]
\end{verbatim}
\end{funcdesc}
 
\begin{funcdesc}{eigenvectors}{a}
   This function returns both the eigenvalues and the eigenvectors, the latter
   as a two-dimensional array (i.e. a sequence of vectors).
\begin{verbatim}
>>> print a
[[ 1.    0.    0.    0.  ]
 [ 0.    2.    0.    0.01]
 [ 0.    0.    5.    0.  ]
 [ 0.    0.01  0.    2.5 ]]
>>> eval, evec = la.eigenvectors(a)
>>> print eval  # same as eigenvalues()
[ 2.50019992  1.99980008  1.          5.        ]
>>> print transpose(evec)
[[ 0.          0.          1.          0.        ]
 [ 0.01998801  0.99980022  0.          0.        ]
 [ 0.          0.          0.          1.        ]
 [ 0.99980022 -0.01998801  0.          0.        ]]
\end{verbatim}
\end{funcdesc}
 
\begin{funcdesc}{generalized_inverse}{a, rcond=1e-10}
   This function returns the generalized inverse (also known as pseudo-inverse
   or Moore-Penrose-inverse) of the matrix \code{a}. It has numerous 
   applications related to linear equations and least-squares problems.
\begin{verbatim}
>>> ainv = la.generalized_inverse(a)
>>> print array_str(innerproduct(a,ainv),suppress_small=1,precision=8)
[[ 1.  0.  0.  0.]
 [ 0.  1.  0. -0.]
 [ 0.  0.  1.  0.]
 [ 0. -0.  0.  1.]]
\end{verbatim}
\end{funcdesc}
 
\begin{funcdesc}{Heigenvalues}{a}
   returns the (real positive) eigenvalues of the square, Hermitian positive
   definite matrix a.
\end{funcdesc}
 
\begin{funcdesc}{Heigenvectors}{a}
   returns both the (real positive) eigenvalues and the eigenvectors of a
   square, Hermitian positive definite matrix a. The eigenvectors are returned
   in an (orthornormal) two-dimensional matrix.
\end{funcdesc}

\begin{funcdesc}{inverse}{a}
   This function returns the inverse of the specified matrix a which must be
   square and non-singular. To within floating point precision, it should
   always be true that \code{matrixmultiply(a, inverse(a)) ==
      identity(len(a))}.  To test this claim, one can do e.g.:
\begin{verbatim}
>>> a = reshape(arange(25.0), (5,5)) + identity(5)
>>> print a
[[  1.   1.   2.   3.   4.]
 [  5.   7.   7.   8.   9.]
 [ 10.  11.  13.  13.  14.]
 [ 15.  16.  17.  19.  19.]
 [ 20.  21.  22.  23.  25.]]
>>> inv_a = la.inverse(a)
>>> print inv_a
[[ 0.20634921 -0.52380952 -0.25396825  0.01587302  0.28571429]
 [-0.5026455   0.63492063 -0.22751323 -0.08994709  0.04761905]
 [-0.21164021 -0.20634921  0.7989418  -0.1957672  -0.19047619]
 [ 0.07936508 -0.04761905 -0.17460317  0.6984127  -0.42857143]
 [ 0.37037037  0.11111111 -0.14814815 -0.40740741  0.33333333]]
\end{verbatim}
   Verify the inverse by printing the largest absolute element of
   $a\, a^{-1} - identity(5)$:
\begin{verbatim}
>>> print "Inversion error:", maximum.reduce(fabs(ravel(dot(a, inv_a)-identity(5))))
Inversion error: 8.18789480661e-16
\end{verbatim}
\end{funcdesc}
 
\begin{funcdesc}{linear_least_squares}{a, b, rcond=1e-10}
   This function returns the least-squares solution of an overdetermined system
   of linear equations. An optional third argument indicates the cutoff for the
   range of singular values (defaults to $10^{-10}$). There are four return
   values: the least-squares solution itself, the sum of the squared residuals
   (i.e.  the quantity minimized by the solution), the rank of the matrix a,
   and the singular values of a in descending order.
\begin{verbatim}

\end{verbatim}
\end{funcdesc}
 
\begin{funcdesc}{solve_linear_equations}{a, b}
   This function solves a system of linear equations with a square non-singular
   matrix a and a right-hand-side vector b. Several right-hand-side vectors can
   be treated simultaneously by making b a two-dimensional array (i.e. a
   sequence of vectors). The function inverse(a) calculates the inverse of the
   square non-singular matrix a by calling solve_linear_equations(a, b) with a
   suitable b.
\end{funcdesc}
 
\begin{funcdesc}{singular_value_decomposition}{a, full_matrices=0}
   This function returns three arrays V, S, and WT whose matrix product is the
   original matrix a. V and WT are unitary matrices (rank-2 arrays), whereas S
   is the vector (rank-1 array) of diagonal elements of the singular-value
   matrix. This function is mainly used to check whether (and in what way) a
   matrix is ill-conditioned.
\end{funcdesc}
 



%% Local Variables:
%% mode: LaTeX
%% mode: auto-fill
%% fill-column: 79
%% indent-tabs-mode: nil
%% ispell-dictionary: "american"
%% reftex-fref-is-default: nil
%% TeX-auto-save: t
%% TeX-command-default: "pdfeLaTeX"
%% TeX-master: "numarray"
%% TeX-parse-self: t
%% End:

\chapter{Masked Arrays}
\label{cha:masked-arrays}

%begin{latexonly}
\makeatletter
\py@reset
\makeatother
%end{latexonly}
\declaremodule{extension}{numarray.ma}
\moduleauthor{The numarray team}{numpy-discussion@lists.sourceforge.net}
\modulesynopsis{Masked Arrays}
\index{MaskedArray|see{numarray.ma}}
\index{observations, dealing with missing}

\begin{quote}
   Masked arrays are arrays that may have missing or invalid entries. Module
   \module{numarray.ma} provides a nearly work-alike replacement for numarray
   that supports data arrays with masks.
\end{quote}

\section{What is a masked array?}
\label{sec:numarray.ma:what-is-a-masked-array}

Masked arrays are arrays that may have missing or invalid entries. Module
\module{numarray.ma} provides a work-alike replacement for \module{\numarray}
that supports data arrays with masks. A mask is either None or an array of ones
and zeros, that determines for each element of the masked array whether or not
it contains an invalid entry.  The package assures that invalid entries are not
used in computations.  A particular element is said to be masked
(\index{numarray.ma!invalid}invalid) if the mask is not None and the
corresponding element of the mask is 1; otherwise it is unmasked
(\index{numarray.ma!valid}valid).

This package was written by \index{Dubois, Paul F.}Paul F.\ Dubois at Lawrence
Livermore National Laboratory. Please see the legal notice in the software and
section \ref{sec:legal-notice} ``License and disclaimer for packages
numarray.ma''. 

\section{Using numarray.ma}
\label{sec:numarray.ma:using}

Use numarray.ma as a replacement for numarray:
\begin{verbatim}
from numarray.ma import *
>>> x = array([1, 2, 3])
\end{verbatim}
To create an array with the second element invalid, we would do:
\begin{verbatim}
>>> y = array([1, 2, 3], mask = [0, 1, 0])
\end{verbatim}
To create a masked array where all values ``near'' 1.e20 are invalid, we can
do:
\begin{verbatim}
>>> z = masked_values([1.0, 1.e20, 3.0, 4.0], 1.e20)
\end{verbatim}
For a complete discussion of creation methods for masked arrays please see
section \ref{sec:numarray.ma:constructing-mask-arrays} ``Constructing masked
arrays''.

The \module{\numarray} module is an attribute in \module{numarray.ma}, so to
execute a method \method{foo} from numarray, you can reference it as
\method{numarray.foo}.

Usually people use both numarray.ma and numarray this way, but of course you can
always fully-qualify the names:
\begin{verbatim}
>>> import numarray.ma
>>> x = numarray.ma.array([1, 2, 3])
\end{verbatim}

The principal feature of module \module{numarray.ma} is class
\class{MaskedArray}, the class whose instances are returned by the array
constructors and most functions in module \module{numarray.ma}. We will discuss
this class first, and later cover the attributes and functions in module
\module{numarray.ma}. For now suffice it to say that among the attributes of
the module are the constants from module \module{\numarray} including those for
declaring typecodes, \constant{NewAxis}, and the mathematical constants such as
\constant{pi} and \constant{e}.  An additional typecode, \class{MaskType}, is
the typecode used for masks.


\section{Class MaskedArray}
\label{sec:numarray.ma:class-maskedarray}
\index{numarray.ma!MaskedArray@\class{MaskedArray}}

In Module \module{numarray.ma}, an array is an instance of class
\class{MaskedArray}, which is defined in the module \module{numarray.ma}. An
instance of class \class{MaskedArray} can be thought of as containing the
following parts:
\begin{itemize}
\item An array of data, of any shape;
\item A mask of ones and zeros of the same shape as the data where a one value
  (true) indicates that the element is masked and the corresponding data is
  invalid.
\item A ``fill value'' --- this is a value that may be used to replace the
   invalid entries in order to return a plain \module{\numarray} array. The
   chief method that does this is the method \method{filled} discussed below.
\end{itemize}
We will use the terms ``invalid value'' and ``invalid entry'' to refer to the
data value at a place corresponding to a mask value of 1. It should be
emphasized that the invalid values are \emph{never} used in any computation,
and that the fill value is not used for \emph{any} computational purpose. When
an instance \var{x} of class \class{MaskedArray} is converted to its string
representation, it is the result returned by \code{filled(x)} that is converted
to a string.


\subsection{Attributes of masked arrays}
\label{sec:numarray.ma:attr-mask-arrays}

\begin{memberdesc}[MaskedArray]{flat}
   (deprecated) \remark{why deprecated in numarray?}
   Returns the masked array as a one-dimensional one. This is
   provided for compatibility with \module{\numarray}. \method{ravel} is
   preferred.  \member{flat} can be assigned to: \samp{x.flat = value} will
   change the values of \var{x}.
\end{memberdesc}

\begin{memberdesc}[MaskedArray]{real}
   Returns the real part of the array if complex. It can be assigned to:
   \samp{x.real = value} will change the real parts of \var{x}.
\end{memberdesc}

\begin{memberdesc}[MaskedArray]{imaginary}
   Returns the imaginary part of the array if complex. It can be assigned to:
   \samp{x.imaginary = value} will change the imaginary parts of x.
\end{memberdesc}

\begin{memberdesc}[MaskedArray]{shape}
   The shape of a masked array can be accessed or changed by using the special
   attribute \member{shape}, as with \module{\numarray} arrays. It can be
   assigned to: \samp{x.shape = newshape} will change the shape of \var{x}. The
   new shape has to describe the same total number of elements.
   \remark{Correct?}
\end{memberdesc}

\begin{memberdesc}[MaskedArray]{shared_data}
   This read-only flag if true indicates that the masked array shared a
   reference with the original data used to construct it at the time of
   construction. Changes to the original array will affect the masked array.
   (This is not the default behavior; see ``Copying or not''.) This flag is
   informational only.
\end{memberdesc}

\begin{memberdesc}[MaskedArray]{shared_mask}
   This read-only flag if true indicates that the masked array \emph{currently}
   shares a reference to the mask used to create it. Unlike
   \member{shared_data}, this flag may change as the result of modifying the
   array contents, as the mask uses copy on write semantics if it is shared.
\end{memberdesc}



\subsection{Methods on masked arrays}
\label{sec:numarray.ma:meth-mask-arrays}

\begin{methoddesc}[MaskedArray]{__array__}
   A special method allows conversion to a \module{\numarray} array if no
   element is actually masked. If there is a masked element, an
   \exception{numarray.maError} exception is thrown. Many \module{\numarray}
   functions, such as \function{numarray.sqrt}, will attempt this conversion on
   their arguments. See also module function \function{filled} in section
   \ref{sec:numarray.ma:meth-mask-arrays}.
\begin{verbatim}
yn = numarray.array(x)
\end{verbatim}
\end{methoddesc}

\begin{methoddesc}[MaskedArray]{astype}{type}
   Return \var{self} as array of given \var{type}. 
\begin{verbatim}
y = x.astype(Float32)
\end{verbatim}
\end{methoddesc}

\begin{methoddesc}[MaskedArray]{byte_swapped}{}
   Returns the raw data \class{\numarray} byte-swapped; included for
   consistency with \module{\numarray} but probably meaningless. 
\begin{verbatim}
y = x.byte_swapped()
\end{verbatim}
\end{methoddesc}

\begin{methoddesc}[MaskedArray]{compressed}{}
   Return an array of the valid elements. Result is one-dimensional.  
\begin{verbatim}
y = x.compressed()
\end{verbatim}
\end{methoddesc}

\begin{methoddesc}[MaskedArray]{count}{axis=None}
   If \var{axis} is \constant{None} return the count of non-masked elements in
   the whole array.  Otherwise return an array of such counts along the axis
   given.
\begin{verbatim}
n = x.count()
y = x.count(0)
\end{verbatim}
\end{methoddesc}

\begin{methoddesc}[MaskedArray]{fill_value}{}
   Get the current fill value. 
\begin{verbatim}
v = x.fill_value()
\end{verbatim}
\end{methoddesc}

\begin{methoddesc}[MaskedArray]{filled}{fill_value=None}
   Returns a \module{\numarray} array with the masked values replaced by the
   fill value.  See also the description of module function filled in section
   \ref{sec:numarray.ma:meth-mask-arrays}.
\begin{verbatim}
yn = x.filled()
\end{verbatim}
\end{methoddesc}

\begin{methoddesc}[MaskedArray]{ids}{}
   Return the ids of the data and mask areas. 
\begin{verbatim}
id1, id2 = x.ids()
\end{verbatim}
\end{methoddesc}

\begin{methoddesc}[MaskedArray]{iscontiguous}{}
   Is the data area contiguous? See \method{numarray.scontiguous} in section
   \ref{arraymethod:iscontiguous}.
\begin{verbatim}
if x.iscontiguous():
\end{verbatim}
\end{methoddesc}

\begin{methoddesc}[MaskedArray]{itemsize}{}
   Size of individual data items in bytes. \samp{n = x.itemsize()}
\end{methoddesc}

\begin{methoddesc}[MaskedArray]{mask}{}
   Return the data mask, or \constant{None}. 
\begin{verbatim}
m = x.mask()
\end{verbatim}
\end{methoddesc}

\begin{methoddesc}[MaskedArray]{put}{values}
   Set the value at each non-masked entry to the corresponding entry in
   \var{values}. The mask is unchanged. See also module function
   \function{put}. 
\begin{verbatim}
x.put(values)
\end{verbatim}
\end{methoddesc}

\begin{methoddesc}[MaskedArray]{putmask}{values}
   Eliminate any masked values by setting the value at each masked entry to the
   corresponding entry in \var{values}. Set the mask to \constant{None}.
\begin{verbatim}
x.putmask(values)
assert getmask(x) is None
\end{verbatim}
\end{methoddesc}

\begin{methoddesc}[MaskedArray]{raw_data}{}
   A reference to the non-filled data; portions may be meaningless. Expert use
   only. 
\begin{verbatim}
d = x.raw_data ()
\end{verbatim}
\end{methoddesc}

\begin{methoddesc}[MaskedArray]{savespace}{v}
   Set the spacesaver attribute to \var{v}. 
\begin{verbatim}
x.savespace (1)
\end{verbatim}
\end{methoddesc}

\begin{methoddesc}[MaskedArray]{set_fill_value}{v}
   Set the fill value to \var{v}. Omit v to restore default.
   \samp{x.set_fill_value(1.e21)} \remark{Give correct default value for v.}
\end{methoddesc}

\begin{methoddesc}[MaskedArray]{set_shape}{args...}
   Set the shape. 
\begin{verbatim}
x.set_shape (3, 12)
\end{verbatim}
\end{methoddesc}

\begin{methoddesc}[MaskedArray]{size}{axis}
   Number of elements in array, or along a particular \var{axis}. 
\begin{verbatim}
totalsize = x.size ()
col_len = x.size (1)
\end{verbatim}
\end{methoddesc}

\begin{methoddesc}[MaskedArray]{spacesaver}{}
   Query the spacesave flag.
\begin{verbatim}
flag = x.spacesaver()
\end{verbatim}
\end{methoddesc}

\begin{methoddesc}[MaskedArray]{tolist}{fill_value=None}
   Return the Python \class{list} \code{self.filled(fill_value).tolist()}; note
   that masked values are filled. 
\begin{verbatim}
alist=x.tolist()
\end{verbatim}
\end{methoddesc}

\begin{methoddesc}[MaskedArray]{tostring}{fill_value=None}
   Return the string \code{self.filled(fill_value).tostring()s = x.tostring()}
\end{methoddesc}

\begin{methoddesc}[MaskedArray]{typecode}{}
   Return the type of the data. See module \module{Precision}, section \ref{TBD}.
\begin{verbatim}
z = x.typecode()
\end{verbatim}
\end{methoddesc}

\begin{methoddesc}[MaskedArray]{unmask}{}
   Replaces the mask by \constant{None} if possible. Subsequent operations may
   be faster if the array previously had an all-zero mask.
\begin{verbatim}
x.unmask()
\end{verbatim}
\end{methoddesc}

\begin{methoddesc}[MaskedArray]{unshare_mask}{}
   If shared_mask is currently true, replaces the reference to it with a
   copy. 
\begin{verbatim}
x.unshare_mask()
\end{verbatim}
\end{methoddesc}


\subsection{Constructing masked arrays}
\label{sec:numarray.ma:constructing-mask-arrays}

\index{numarray.ma!constructor}
\begin{methoddesc}[MaskedArray]{array}
   {data, type=None, copy=1, savespace=0, mask=None, fill_value=None}
   Creates a masked array with the given \var{data} and
   \var{mask}.  The name \class{array} is simply an alias for the class name,
   \class{MaskedArray}.  The fill value is set to \var{fill_value}, and the
   \var{savespace} flag is applied. If \var{data} is a \class{MaskedArray}, its
   \constant{mask}, \constant{typecode}, \constant{spacesaver} flag, and
   \constant{fill_value} will be used unless specifically overridden by one of
   the remaining arguments. In particular, if \var{d} is a masked array,
   \code{array(d, copy=0)} is \var{d}.
\end{methoddesc}

\index{numarray.ma!constructor}
\begin{methoddesc}[MaskedArray]{masked_array}{data, mask=None, fill_value=None}
   This is an easier-to-use version of \method{array},
   for the common case of \code{typecode = None}, \code{copy = 0}. When
   \var{data} is newly-created this function can be used to make it a masked
   array without copying the data if \var{data} is already a \module{\numarray}
   array.
\end{methoddesc}

\index{numarray.ma!constructor}
\begin{methoddesc}[MaskedArray]{masked_values}{data, value, rtol=1.e-5, atol=1.e-8, type=None, copy=1, savespace=0)}
   Constructs a masked array whose mask is set at those places where 
   \begin{equation}
      \abs(\var{data} - \var{value}) < \var{atol} + \var{rtol} * \abs(\var{data})
   \end{equation}
   That is a careful way of saying that those elements of the \var{data} that
   have a value of \var{value} (to within a tolerance) are to be treated as
   invalid.  If data is not of a floating point type, calls
   \method{masked_object} instead.
\end{methoddesc}

\index{numarray.ma!constructor}
\begin{methoddesc}[MaskedArray]{masked_object}{data, value, copy=1, savespace=0}
   Creates a masked array with those entries marked invalid that are equal to
   \var{value}. Again, \var{copy} and \var{/savespace} are passed on to the
   \module{\numarray} array constructor.
\end{methoddesc}

\index{numarray.ma!constructor}
\begin{methoddesc}[MaskedArray]{asarray}{data, type=None}
   This is the same as \code{array(data, typecode, copy=0)}. It is a short way
   of ensuring that something is an instance of \class{MaskedArray} of a given
   \var{type} before proceeding, as in \samp{data = asarray(data)}.
   
   If \var{data} already is a masked array and \var{type} is \constant{None}
   then the return value is \var{data}; nothing is copied in that case.
\end{methoddesc}

\index{numarray.ma!constructor}
\begin{methoddesc}[MaskedArray]{masked_where}{condition, data, copy=1)}
   Creates a masked array whose shape is that of \var{condition}, whose values
   are those of \var{data}, and which is masked where elements of
   \var{condition} are true.
\end{methoddesc}

\index{numarray.ma!constructor}
\begin{datadesc}{masked}
   This is a module constant that represents a scalar masked value. For
   example, if \var{x} is a masked array and a particular location such as
   \code{x[1]} is masked, the quantity \code{x[1]} will be this special
   constant. This special element is discussed more fully in section
   \ref{sec:numarray.ma:constant-masked} ``The constant \constant{masked}''.
\end{datadesc}


The following additional constructors are provided for convenience.

\index{numarray.ma!constructor}
\begin{methoddesc}[MaskedArray]{masked_equal}{data, value, copy=1}
\end{methoddesc} \index{numarray.ma!constructor}
\begin{methoddesc}[MaskedArray]{masked_greater}{data, value, copy=1}
\end{methoddesc} \index{numarray.ma!constructor}
\begin{methoddesc}[MaskedArray]{masked_greater_equal}{data, value, copy=1}
\end{methoddesc} \index{numarray.ma!constructor}
\begin{methoddesc}[MaskedArray]{masked_less}{data, value, copy=1}
\end{methoddesc} \index{numarray.ma!constructor}
\begin{methoddesc}[MaskedArray]{masked_less_equal}{data, value, copy=1}
\end{methoddesc} \index{numarray.ma!constructor}
\begin{methoddesc}[MaskedArray]{masked_not_equal}{data, value, copy=1}
   \method{masked_greater} is equivalent to \code{masked_where(greater(data,
      value), data))}.  Similarly, \method{masked_greater_equal},
   \method{masked_equal}, \method{masked_not_equal}, \method{masked_less},
   \method{masked_less_equal} are called in the same way with the obvious
   meanings.  Note that for floating point data, \method{masked_values} is
   preferable to \method{masked_equal} in most cases.  \remark{because...}
\end{methoddesc}

\index{numarray.ma!constructor}
\begin{methoddesc}[MaskedArray]{masked_inside}{data, v1, v2, copy=1}
   Creates an array with values in the closed interval \code{[v1, v2]} masked.
   \var{v1} and \var{v2} may be in either order.
\end{methoddesc}

\index{numarray.ma!constructor}
\begin{methoddesc}[MaskedArray]{masked_outside}{data, v1, v2, copy=1}
   Creates an array with values outside the closed interval \code{[v1, v2]}
   masked.  \var{v1} and \var{v2} may be in either order.
\end{methoddesc}

On entry to any of these constructors, \var{data} must be any object which the
\module{\numarray} package can accept to create an array (with the desired
\var{type}, if specified). The \var{mask}, if given, must be \constant{None} or
any object that can be turned into a \module{\numarray} array of integer type
(it will be converted to type \class{MaskType}, if necessary), have the same
shape as \var{data}, and contain only values of 0 or 1.

If the \var{mask} is not \constant{None} but its shape does not match that of
\var{data}, an exception will be thrown, unless one of the two is of length 1,
in which case the scalar will be resized (using \method{numarray.resize}) to
match the other.

See section \ref{sec:numarray.ma:copying-or-not} ``Copying or not'' for a
discussion of whether or not the resulting array shares its data or its mask
with the arguments given to these constructors.


\paragraph*{Important Tip} \method{filled} is very important. It converts its
argument to a plain \module{\numarray} array.

\begin{funcdesc}{filled}{x, value=None}
   Returns \var{x} with any invalid locations replaced by a fill \var{value}.
   \function{filled} is guaranteed to return a plain \module{\numarray} array.
   The argument \var{x} does not have to be a masked array or even an array,
   just something that \module{\numarray}/\module{numarray.ma} can turn into
   one.
   \begin{itemize}
   \item If \var{x} is not a masked array, and not a \module{\numarray} array,
      \code{numarray.array(x)} is returned.
   \item If \var{x} is a contiguous \module{\numarray} array then \var{x} is
      returned. (A \module{\numarray} array is contiguous if its data storage
      region is layed out in column-major order; \module{\numarray} allows
      non-contiguous arrays to exist but they are not allowed in certain
      operations).
   \item If \var{x} is a masked array, but the mask is \constant{None}, and
      \var{x}'s data array is contiguous, then it is returned. If the data
      array is not contiguous, a (contiguous) copy of it is returned.
   \item If \var{x} is a masked array with an actual mask, then an array formed
      by replacing the invalid entries with \var{value}, or
      \code{fill_value(x)} if \var{value} is \constant{None}, is returned. If
      the fill value used is of a different type or precision than \var{x}, the
      result may be of a different type or precision than \var{x}.
\end{itemize}
Note that a new array is created only if necessary to create a correctly
filled, contiguous, \module{\numarray} array.

The function \method{filled} plays a central role in our design. It is the
``exit'' back to \module{\numarray}, and is used whenever the invalid values
must be replaced before an operation. For example, adding two masked arrays
\var{a} and \var{b} is roughly:
\begin{verbatim}
masked_array(filled(a, 0) + filled(b, 0), mask_or(getmask(a), getmask(b))
\end{verbatim}
That is, fill the invalid entries of \var{a} and \var{b} with zeros, add them
up, and declare any entry of the result invalid if either \var{a} or \var{b}
was invalid at that spot. The functions \function{getmask} and
\function{mask_or} are discussed later.

\function{filled} also can be used to simply be certain that some expression is
a contiguous \module{\numarray} array at little cost. If its argument is a
\module{\numarray} array already, it is returned without copying.

If you are certain that a masked array \var{x} contains a mask that is None or
is all zeros, you can convert it to a numarray array with the
\method{numarray.array(x)} constructor. If you turn out to be wrong, an
\exception{MAError} exception is raised.
\end{funcdesc}

\begin{funcdesc}{fill_value}{x}
\end{funcdesc}
\begin{methoddesc}[MaskedArray]{fill_value}{}
   \code{fill_value(x)} and the method \code{x.fill_value()} on masked arrays,
   return a value suitable for filling \var{x} based on its type.  If \var{x}
   is a masked array, then \var{x.fill_value()} results. The returned value for
   a given type can be changed by assigning to the following names in module
   \module{numarray.ma}. They should be set to scalars or one element arrays.
   \index{numarray.ma!default_real_fill_value@\constant{default_real_fill_value}}
   \index{numarray.ma!default_complex_fill_value@\constant{default_complex_fill_value}}
   \index{numarray.ma!default_character_fill_value@\constant{default_character_fill_value}}
   \index{numarray.ma!default_integer_fill_value@\constant{default_integer_fill_value}}
   \index{numarray.ma!default_object_fill_value@\constant{default_object_fill_value}}
\begin{verbatim}
default_real_fill_value = numarray.array([1.0e20], Float32)
default_complex_fill_value = numarray.array([1.0e20 + 0.0j], Complex32)
default_character_fill_value = masked
default_integer_fill_value = numarray.array([0]).astype(UnsignedInt8)
default_object_fill_value = masked
\end{verbatim}
   The variable \var{masked} is a module variable of \module{numarray.ma} and
   is discussed in section \ref{sec:numarray.ma:constant-masked}. Calling
   \function{filled} with a \var{fill_value} of \constant{masked} sometimes
   produces a useful printed representation of a masked array.  The function
   \function{fill_value} works on any kind of object.
\end{methoddesc}

\index{numarray.ma!set_fill_value@\method{set_fill_value}}\code{set_fill_value(a,
   fill_value)} is the same as \code{a.set_fill_value (fill_value)} if \var{a}
   is a masked array; otherwise it does nothing. Please note that the fill
   value is mostly cosmetic; it is used when it is needed to convert the masked
   array to a plain \module{\numarray} array but not involved in most
   operations. In particular, setting the \member{fill_value} to
   \constant{1.e20} will \emph{not}, repeat not, cause elements of the array
   whose values are currently 1.e20 to be masked. For that sort of behavior use
   the \method{masked_value} constructor.



\subsection{What are masks?}
\label{sec:numarray.ma:what-are-masks}
\index{masks, description of}
\index{masks, savespace attribute}

Masks are either \constant{None} or 1-byte \module{\numarray} arrays of 1's and
0's. To avoid excessive performance penalties, mask arrays are never checked to
be sure that the values are 1's and 0's, and supplying a \var{mask} argument to
a constructor with an illegal mask will have undefined consequences later.

\emph{Masks have the savespace attribute set.}  This attribute, discussed in
part \ref{part:numerical-python}, may have surprising consequences if you
attempt to do any operations on them other than those supplied by this package.
In particular, do not add or multiply a quantity involving a mask. For example,
if \var{m} is a mask consisting of 1080 1 values, \code{sum(m)} is 56, not
1080. Oops.


\subsection{Working with masks}

\begin{funcdesc}{is_mask}{m}
   Returns true if \var{m} is of a type and precision that would be allowed as
   the mask field of a masked array (that is, it is an array of integers with
   \module{\numarray}'s typecode \class{MaskType}, or it is \constant{None}).
   To be a legal mask, \var{m} should contain only zeros or ones, but this is
   not checked.
\end{funcdesc}

\begin{funcdesc}{make_mask}{m, copy=0, flag=0}
   Returns an object whose entries are equal to \var{m} and for which
   \function{is_mask} would return true. If \var{m} is already a mask or
   \constant{None}, it returns \var{m} or a copy of it. Otherwise it will
   attempt to make a mask, so it will accept any sequence of integers for
   \var{m}. If \var{flag} is true, \method{make_mask} returns \constant{None}
   if its return value otherwise would contain no true elements. To make a
   legal mask, \var{m} should contain only zeros or ones, but this is not
   checked.
\end{funcdesc}

\begin{funcdesc}{make_mask_none}{s}
   Returns a mask of all zeros of shape \var{s} (deprecated name:
   \index{numarray.ma!create_mask@\method{create_mask}
      (deprecated)|see{\method{make_mask_none}}}create_mask).
\end{funcdesc}

\begin{funcdesc}{getmask}{x}
   Returns \index{numarray.ma!mask@\method{mask}}\code{x.mask()}, the mask of
   \var{x}, if \var{x} is a masked array, and \constant{None} otherwise.
   \note{\function{getmask} may return \constant{None} if \var{x} is a masked
   array but has a mask of \constant{None}.  (Please see caution above about
   operating on the result).}
\end{funcdesc}

\begin{funcdesc}{getmaskarray}{x}
   Returns \code{x.mask()} if \var{x} is a masked array and has a mask that is
   not \constant{None}; otherwise it returns a zero mask array of the same
   shape as \var{x}.  Unlike \method{getmask}, \method{getmaskarray} always
   returns an \module{\numarray} array of typecode \class{MaskType}. (Please
   see caution above about operating on the result).
\end{funcdesc}

\begin{funcdesc}{mask_or}{m1, m2}
   Returns an object which when used as a mask behaves like the element-wise
   ``logical or'' of \var{m1} and \var{m2}, where \var{m1} and \var{/m2} are
   either masks or \constant{None} (e.g., they are the results of calling
   \method{getmask}). A \constant{None} is treated as everywhere false. If both
   \var{m1} and \var{m2} are \constant{None}, it returns \constant{None}. If
   just one of them is \constant{None}, it returns the other. If \var{m1} and
   \var{m2} refer to the same object, a reference to that object is returned.
\end{funcdesc}


\subsection{Operations}
\label{sec:numarray.ma:operations}

Masked arrays support the operators $+$, $*$, $/$, $-$, $**$, and unary plus
and minus.  The other operand can be another masked array, a scalar, a
\module{\numarray} array, or something \method{numarray.array} can convert to a
\module{\numarray} array. The results are masked arrays.

In addition masked arrays support the in-place operators $+=$, $-=$, $*=$, and
$/=$.  Implementation of in-place operators differs from \module{\numarray}
semantics in being more generous about converting the right-hand side to the
required type: any kind or lesser type accepted via an \method{astype}
conversion.  In-place operators truly operate in-place when the target is not
masked.



\subsection{Copying or not?}
\label{sec:numarray.ma:copying-or-not}

Depending on the arguments results of constructors may or may not contain a
separate copy of the data or mask arguments. The easiest way to think about
this is as follows: the given field, be it data or a mask, is required to be a
\module{\numarray} array, possibly with a given typecode, and a mask's shape
must match that of the data. If the copy argument is zero, and the candidate
array otherwise qualifies, a reference will be made instead of a copy. If for
any reason the data is unsuitable as is, an attempt will be made to make a copy
that is suitable. Should that fail, an exception will be thrown. Thus, a
\code{copy=0} argument is more of a hope than a command.

If the basic array \index{numarray.ma!constructor}constructor is given a masked
array as the first argument, its mask, typecode, spacesaver flag, and fill
value will be used unless specifically specified by one of the remaining
arguments. In particular, if \var{d} is a masked array, \code{array(d, copy=0)}
is \var{d}.

Since the default behavior for masks is to use a reference if possible, rather
than a copy, which produces a sizeable time and space savings, it is especially
important not to modify something you used as a mask argument to a masked array
creation routine, if it was a \module{\numarray} array of typecode
\class{MaskType}.





\subsection{Behaviors}
\label{sec:numarray.ma:behaviors}
\begin{funcdesc}{float}{a}
\end{funcdesc}
\begin{funcdesc}{int}{a}
  The conversion operators \function{float}, and \function{int} are defined
  to operate on masked arrays consisting of a single unmasked element.
  Masked values and multi-element arrays are not convertible.  
\end{funcdesc}
\begin{funcdesc}{repr}{a}
\end{funcdesc}
\begin{funcdesc}{str}{a}
  A masked array defines the conversion operators \function{str} and
  \function{repr} by applying the corresponding operator to the
  \module{\numarray} array \code{filled(a)}.  
\end{funcdesc}


\subsection{Indexing and Slicing}
\label{sec:numarray.ma:indexing-slicing}

Indexing and slicing differ from Numeric: while generally the same, they return
a copy, not a reference, when used in an expression that produces a non-scalar
result. Consider this example:
\begin{verbatim}
from Numeric import *
x = array([1.,2.,3.])
y = x[1:]
y[0] = 9.
print x
\end{verbatim}
This will print \code{[1., 9., 3.]} since \code{x[1:]} returns a reference to a
portion of \var{x}.  Doing the same operation using \module{numarray.ma},
\begin{verbatim}
from numarray.ma import *
x = array([1.,2.,3.])
y = x[1:]
y[0] = 9.
print x
\end{verbatim}
will print \code{[1., 2., 3.]}, while \var{y} will be a separate array whose
present value would be \code{[9., 3.]}. While sentiment on the correct
semantics here is divided amongst the Numeric Python community as a whole, it
is not divided amongst the author's community, on whose behalf this package is
written.


\subsection{Indexing in assignments}
\label{sec:numarray.ma:indexing-assignments}

Using multiple sets of square brackets on the left side of an assignment
statement will not produce the desired result:
\begin{verbatim}
x = array([[1,2],[3,4]])
x[1][1] = 20.                           # Error, does not change x
x[1,1] = 20.                            # Correct, changes x
\end{verbatim}
The reason is that \code{x[1]} is a copy, so changing it changes that copy, not
\var{x}.  Always use just one single square bracket for assignments.


\subsection{Operations that produce a scalar result}
\label{sec:numarray.ma:operations-producing-scalars}

If indexing or another operation on a masked array produces a scalar result,
then a scalar value is returned rather than a one-element masked array. This
raises the issue of what to return if that result is masked. The answer is that
the module constant
\index{numarray.ma!masked@\constant{masked}}\constant{masked} is returned. This
constant is discussed in section \ref{sec:numarray.ma:constant-masked}.  While
this most frequently occurs from indexing, you can also get such a result from
other functions. For example, averaging a 1-D array, all of whom's values are
invalid, would result in \constant{masked}.


\subsection{Assignment to elements and slices}
\label{sec:numarray.ma:assignments-elements-slices}

Assignment of a normal value to a single element or slice of a masked array has
the effect of clearing the mask in those locations. In this way previously
\index{numarray.ma!invalid}invalid elements become
\index{numarray.ma!valid}valid. The value being assigned is filled first, so
that you are guaranteed that all the elements on the left-hand side are now
valid.  \remark{???}

Assignment of \constant{None} to a single element or slice of a masked array
has the effect of setting the mask in those locations, and the locations become
invalid.

Since these operations change the mask, the result afterwards will no longer
share a mask, since masks have copy-on-write semantics.



\section{MaskedArray Attributes}
\label{sec:numarray.ma:attributes}

\begin{datadesc}{e}
\end{datadesc}
\begin{datadesc}{pi}
\end{datadesc}
\begin{datadesc}{NewAxis}
   Constants \constant{e}, \constant{pi}, \constant{NewAxis} from
   \module{\numarray}, and the constants from module \module{Precision} that
   define nice names for the typecodes.
\end{datadesc}

The special variables \index{numarray.ma!masked@\constant{masked}}\constant{masked} and
\index{numarray.ma!masked@\constant{masked}}masked_print_option are discussed in section
\ref{sec:numarray.ma:constant-masked}.

The module \module{\numarray} is an element of \module{numarray.ma}, so after \samp{from
   numarray.ma import *}, you can refer to the functions in \module{\numarray} such as
\constant{numarray.ones}; see part \ref{part:numerical-python} for the
constants available in \module{\numarray}.




\section{MaskedArray Functions}
\label{sec:numarray.ma:functions}

Each of the operations discussed below returns an instance of \module{numarray.ma} class
\index{numarray.ma!MaskedArray@\class{MaskedArray}}\class{MaskedArray}, having performed
the desired operation element-wise.  In most cases the array arguments can be
masked arrays or \module{\numarray} arrays or something that \module{\numarray}
can turn into a \module{\numarray} array, such as a list of real numbers.

In most cases, if \module{\numarray} has a function of the same name, the
behavior of the one in \module{numarray.ma} is the same, except that it ``respects'' the
mask.


\subsection{Unary functions}
\label{sec:numarray.ma:unary-functions}

The result of a unary operation will be masked wherever the original operand
was masked. It may also be masked if the argument is not in the domain of the
function.  The following functions have their standard meaning:
\begin{quote}
   \index{absolute@\function{absolute} (in module numarray.ma)}\function{absolute}, 
   \index{arccos@\function{arccos} (in module numarray.ma)}\function{arccos}, 
   \index{arcsin@\function{arcsin} (in module numarray.ma)}\function{arcsin}, 
   \index{arctan@\function{arctan} (in module numarray.ma)}\function{arctan}, 
   \index{around@\function{around} (in module numarray.ma)}\function{around}, 
   \index{conjugate@\function{conjugate} (in module numarray.ma)}\function{conjugate}, 
   \index{cos@\function{cos} (in module numarray.ma)}\function{cos}, 
   \index{cosh@\function{cosh} (in module numarray.ma)}\function{cosh}, 
   \index{exp@\function{exp} (in module numarray.ma)}\function{exp},
   \index{floor@\function{floor} (in module numarray.ma)}\function{floor},
   \index{log@\function{log} (in module numarray.ma)}\function{log}, 
   \index{log10@\function{log10} (in module numarray.ma)}\function{log10}, 
   \index{negative@\function{negative} (in module numarray.ma)}\function{negative}
   (also as operator \index{- (in module numarray.ma)}\index{numarray.ma!-}-),
   \index{nonzero@\function{nonzero} (in module numarray.ma)}\function{nonzero}, 
   \index{sin@\function{sin} (in module numarray.ma)}\function{sin}, 
   \index{sinh@\function{sinh} (in module numarray.ma)}\function{sinh}, 
   \index{sqrt@\function{sqrt} (in module numarray.ma)}\function{sqrt}, 
   \index{tan@\function{tan} (in module numarray.ma)}\function{tan}, 
   \index{tanh@\function{tanh} (in module numarray.ma)}\function{tanh}.
\end{quote}

\begin{funcdesc}{fabs}{x}
   The absolute value of \var{x} as a \constant{Float32} array.
   \remark{What happens when you pass \constant{Float64} ?}
\end{funcdesc}


\subsection{Binary functions}
\label{sec:numarray.ma:binary-functions}

Binary functions return a result that is masked wherever either of the operands
were masked; it may also be masked where the arguments are not in the domain of
the function.

\begin{quote}
   \index{add@\function{add} (in module numarray.ma)}\function{add}
   (also as operator \index{+}\index{numarray.ma!+}+),
   \index{subtract@\function{subtract} (in module numarray.ma)}\function{subtract}
   \index{- (in module numarray.ma)}\index{numarray.ma!-}(also as operator -),
   \index{multiply@\function{multiply} (in module numarray.ma)}\function{multiply}
   \index{* (in module numarray.ma)}\index{numarray.ma!*}(also as operator *), 
   \index{divide@\function{divide} (in module numarray.ma)}\function{divide}
   \index{/ (in module numarray.ma)}\index{numarray.ma!/}(also as operator / ), 
   \index{power@\function{power} (in module numarray.ma)}\function{power}
   \index{** (in module numarray.ma)}\index{numarray.ma!**}(also as operator **), 
   \index{remainder@\function{remainder} (in module numarray.ma)}\function{remainder},
   \index{fmod@\function{fmod} (in module numarray.ma)}\function{fmod},
   \index{hypot@\function{hypot} (in module numarray.ma)}\function{hypot},
   \index{arctan2@\function{arctan2} (in module numarray.ma)}\function{arctan2},
   \index{bitwise_and@\function{bitwise_and} (in module numarray.ma)}\function{bitwise_and},
   \index{bitwise_or@\function{bitwise_or} (in module numarray.ma)}\function{bitwise_or},
   \index{bitwise_xor@\function{bitwise_xor} (in module numarray.ma)}\function{bitwise_xor}.
\end{quote}



\subsection{Comparison operators}

To compare arrays, use the following binary functions. Each of them returns a
masked array of 1's and 0's.

\begin{quote}
   \index{equal@\function{equal} (in module numarray.ma)}\function{equal},
   \index{greater@\function{greater} (in module numarray.ma)}\function{greater},
   \index{greater_equal@\function{greater_equal} (in module numarray.ma)}\function{greater_equal},
   \index{less@\function{less} (in module numarray.ma)}\function{less},
   \index{less_equal@\function{less_equal} (in module numarray.ma)}\function{less_equal},
   \index{not_equal@\function{not_equal} (in module numarray.ma)}\function{not_equal}.
\end{quote}

Note that as in \module{\numarray}, you can use a scalar for one argument and
an array for the other. \note{See section \ref{TBD} why operators and comparison
   functions are not excatly equivalent.}



\subsection{Logical operators}

Arrays of logical values can be manipulated with:

\begin{quote}
   \index{logical_and@\function{logical_and} (in module numarray.ma)}\function{logical_and},
   \index{logical_not@\function{logical_not} (in module numarray.ma)}\function{logical_not (unary)},
   \index{logical_or@\function{logical_or} (in module numarray.ma)}\function{logical_or},
   \index{logical_xor@\function{logical_xor} (in module numarray.ma)}\function{logical_xor}.
\end{quote}

\begin{funcdesc}{alltrue}{x}
   Returns 1 if all elements of \var{x} are true. Masked elements are treated
   as true.
\end{funcdesc}

\begin{funcdesc}{sometrue}{x}
   Returns 1 if any element of \var{x} is true. Masked elements are treated as
   false.
\end{funcdesc}



\subsection{Special array operators}

\begin{funcdesc}{isarray}{x}
   Return true \var{x} is a masked array.
   \remark{What is about \numarray's?}
\end{funcdesc}

\begin{funcdesc}{rank}{x} 
   The number of dimensions in \var{x}.
\end{funcdesc}

\begin{funcdesc}{shape}{x}
   Returns the shape of \var{x}, a tuple of array extents.
\end{funcdesc}

\begin{funcdesc}{resize}{x, shape}
   Returns a new array with specified \var{shape}.
\end{funcdesc}

\begin{funcdesc}{reshape}{x, shape}
   Returns a copy of \var{x} with the given new \var{shape}.
\end{funcdesc}

\begin{funcdesc}{ravel}{x}
   Returns \var{x} as one-dimensional \class{MaskedArray}.
\end{funcdesc}

\begin{funcdesc}{concatenate}{(a0, ... an), axis=0}
   Concatenates the arrays \code{a0, ... an} along the specified \var{axis}.
\end{funcdesc}

\begin{funcdesc}{repeat}{a, repeats, axis=0}
   Repeat elements \var{i} of \var{a} \code{repeats[i]} times along \var{axis}.
   \var{repeats} is a sequence of length \code{a.shape[axis]} telling how many
   times to repeat each element.
\end{funcdesc}

\begin{funcdesc}{identity}{n}
   Returns the identity matrix of shape \var{n} by \var{n}.
\end{funcdesc}

\begin{funcdesc}{indices}{dimensions, type=None}
   Returns an array representing a grid of indices with row-only and
   column-only variation.
\end{funcdesc}

\begin{funcdesc}{len}{x}
   This is defined to be the length of the first dimension of \var{x}. This
   definition, peculiar from the array point of view, is required by the way
   Python implements slicing. Use \function{size} for the total length of
   \var{x}.
\end{funcdesc}

\begin{funcdesc}{size}{x, axis=None}
   This is the total size of \var{x}, or the length of a particular dimension
   \var{axis} whose index is given. When axis is given the dimension of the
   result is one less than the dimension of \var{x}.
\end{funcdesc}

\begin{funcdesc}{count}{x, axis=None}
   Count the number of (non-masked) elements in the array, or in the array
   along a certain \var{axis}.  When \var{axis} is given the dimension of the
   result is one less than the dimension of \var{x}.
\end{funcdesc}

\begin{funcdesc}{arange}{}
\end{funcdesc}
\begin{funcdesc}{arrayrange}{}
\end{funcdesc}
\begin{funcdesc}{diagonal}{}
\end{funcdesc}
\begin{funcdesc}{fromfunction}{}
\end{funcdesc}
\begin{funcdesc}{ones}{}
\end{funcdesc}
\begin{funcdesc}{zeros}{}
   are the same as in numarray, but return masked arrays.
\end{funcdesc}

\begin{funcdesc}{sum}{}
\end{funcdesc}
\begin{funcdesc}{product}{}
   are called the same way as count; the difference is that the result is the
   sum or product of the unmasked element.
\end{funcdesc}

\begin{funcdesc}{average}{x, axis=0, weights=None, returned=0}
   Compute the average value of the non-masked elements of \var{x} along the
   selected \var{axis}. If \var{weights} is given, it must match the size and
   shape of \var{x}, and the value returned is:
   \begin{equation}
      \text{average} = \frac{\sum{}weights_i\cdot{}x_i}{\sum{}weights_i}
   \end{equation}
   In computing these sums, elements that correspond to those that are masked
   in \var{x} or \var{weights} are ignored. If successful a 2-tuple consisting
   of the average and the sum of the weights is returned.
\end{funcdesc}

\begin{funcdesc}{allclose}{x, y, fill_value=1, rtol=1.e-5, atol=1.e-8}
   Test whether or not arrays \var{x} and \var{y} are equal subject to the
   given relative and absolute tolerances. If \var{fill_value} is 1, masked
   values are considered equal, otherwise they are considered different. The
   formula used for elements where both \var{x} and \var{y} have a valid value
   is:
   \begin{equation}
      |x-y| < \var{atol} + \var{rtol} \cdot{} |y|
   \end{equation}
   This means essentially that both elements are small compared to \var{atol}
   or their difference divided by their value is small compared to \var{rtol}.
\end{funcdesc}

\begin{funcdesc}{allequal}{x, y, fill_value=1}
   This function is similar to \function{allclose}, except that exact equality
   is demanded. \note{Consider the problems of floating-point representations
      when using this function on non-integer numbers arrays.}
\end{funcdesc}

\begin{funcdesc}{take}{a, indices, axis=0}
   Returns a selection of items from \var{a}. See the documentation of
   \function{numarray.take} in section \ref{sec:array-functions:take}.
\end{funcdesc}

\begin{funcdesc}{transpose}{a, axes=None}
   Performs a reordering of the axes depending on the tuple of indices
   \var{axes}; the default is to reverse the order of the axes.
\end{funcdesc}

\begin{funcdesc}{put}{a, indices, values}
   The opposite of \function{take}. The values of the array \var{a} at the
   locations specified in \var{indices} are set to the corresponding value of
   \var{values}.  The array \var{a} must be a contiguous array. The argument
   \var{indices} can be any integer sequence object with values suitable for
   indexing into the flat form of \var{a}.  The argument \var{values} must be
   any sequence of values that can be converted to the typecode of \var{a}.
\begin{verbatim}
>>> x = arange(6)
>>> put(x, [2,4], [20,40])
>>> print x
[ 0  1 20  3 40  5 ]
\end{verbatim}
   Note that the target array \var{a} is not required to be one-dimensional.
   Since it is contiguous and stored in row-major order, the array indices can
   be treated as indexing \var{a}s elements in storage order.
   
   The wrinkle on this for masked arrays is that if the locations being set by
   \function{put} are masked, the mask is cleared in those locations.
\end{funcdesc}

\begin{funcdesc}{choose}{condition, t}
   This function has a result shaped like \var{condition}. \var{t} must be a
   tuple. Each element of the tuple can be an array, a scalar, or the constant
   element \constant{masked} (See section \ref{sec:numarray.ma:constant-masked}). Each
   element of the result is the corresponding element of \code{t[i]} where
   \var{condition} has the value \var{i}. The result is masked where
   \var{condition} is masked or where the selected element is masked or the
   selected element of \var{t} is the constant \constant{masked}.
\end{funcdesc}

\begin{funcdesc}{where}{condition, x, y}
   Returns an array that is \code{filled(x)} where \var{condition} is true,
   \code{filled(y)} where the condition is false. One of \var{x} or \var{y} can
   be the constant element \constant{masked} (See section
   \ref{sec:numarray.ma:constant-masked}). The result is masked where \var{condition} is
   masked, where the element selected from \var{x} or \var{y} is masked, or
   where \var{x} or \var{y} itself is the constant \constant{masked} and it is
   selected.
\end{funcdesc}

\begin{funcdesc}{innerproduct}{a, b}
\end{funcdesc}
\begin{funcdesc}{dot}{a, b}
   These functions work as in \module{\numarray}, but missing values don't
   contribute. The result is always a masked array, possibly of length one,
   because of the possibility that one or more entries in it may be invalid
   since all the data contributing to that entry was invalid.
\end{funcdesc}

\begin{funcdesc}{outerproduct}{a, b}
   Produces a masked array such that \code{result[i, j] = a[i] * b[j]}. The
   result will be masked where \code{a[i]} or \code{b[j]} is masked.
\end{funcdesc}

\begin{funcdesc}{compress}{condition, x, dimension=-1}
   Compresses out only those valid values where \var{condition} is true. Masked
   values in \var{condition} are considered false.
\end{funcdesc}

\begin{funcdesc}{maximum}{x, y=None}
\end{funcdesc}
\begin{funcdesc}{minimum}{x, y=None}
   Compute the maximum (minimum) valid values of \var{x} if \var{y} is
   \constant{None}; with two arguments, they return the element-wise larger or
   smaller of valid values, and mask the result where either \var{x} or \var{y}
   is masked.  If both arguments are scalars a scalar is returned.
\end{funcdesc}

\begin{funcdesc}{sort}{x, axis=-1, value=None}
   Returns the array \var{x} sorted along the given axis, with masked values
   treated as if they have a sort value of \var{value} but locations containing
   \var{value} are masked in the result if \var{x} had a mask to start with.
   \note{Thus if \var{x} contains \var{value} at a non-masked spot, but has
      other spots masked, the result may not be what you want.}
\end{funcdesc}

\begin{funcdesc}{argsort}{x, axis=-1, fill_value=None}
   This function is unusual in that it returns a \module{\numarray} array,
   equal to \code{numarray.argsort(filled(x, fill_value), axis)}; this is an
   array of indices for sorting along a given axis.
\end{funcdesc}



\subsection{Controlling the size of the string representations}
\label{sec:numarray.ma:contr-size-string}


\begin{funcdesc}{get_print_limit}{}
\end{funcdesc}
\begin{funcdesc}{set_print_limit}{n=0}
   These functions are used to limit printing of large arrays; query and set
   the limit for converting arrays using \function{str} or \function{repr}.
   
   If an array is printed that is larger than this, the values are not printed;
   rather you are informed of the type and size of the array. If \var{n} is
   zero, the standard \module{\numarray} conversion functions are used.
   
   When imported, \module{numarray.ma} sets this limit to 300, and the limit is also
   made to apply to standard \module{\numarray} arrays as well.
\end{funcdesc}



\section{Helper classes}
\label{sec:numarray.ma:helper-classes}

\begin{quote}
   This section discusses other classes defined in module numarray.ma.
\end{quote}

\begin{classdesc}{MAError}
   This class inherits from Exception, used to raise exceptions in the
   \module{numarray.ma} module. Other exceptions are possible, such as errors from the
   underlying \module{\numarray} module.
\end{classdesc}


\subsection{The constant masked}
\label{sec:numarray.ma:constant-masked}
\index{numarray.ma!masked@\constant{masked} (constant)}

A constant named \index{numarray.ma!masked@\constant{masked}}\constant{masked} in
\module{numarray.ma} serves several purposes.
\begin{enumerate}
\item When a indexing operation on an \class{MaskedArray} instance returns a
   scalar result, but the location indexed was masked, then \constant{masked}
   is returned. For example, given a one-dimensional array \var{x} such that
   \code{x.mask()[3]} is 1, then \code{x[3]} is \constant{masked}.
\item When \constant{masked} is assigned to elements of an array via indexing
   or slicing, those elements become masked. So after \code{x[3] = masked},
   \code{x[3]} is masked.
\item Some other operations that may return scalar values, such as
   \function{average}, may return \constant{masked} if given only invalid data.
\item To test whether or not a variable is this element, use the \function{is}
   or \function{is not} operator, not \code{==} or \code{!=}.
\item Operations involving the constant \constant{masked} may result in an
   exception.  In operations, \constant{masked} behaves as an integer array of
   shape \code{()} with one masked element. For example, using \code{+} for
   illustration,
   \begin{itemize}
   \item \constant{masked} + \constant{masked} is \constant{masked}.
   \item \constant{masked} + numeric scalar or numeric scalar +
      \constant{masked} is \constant{masked}.
   \item \constant{masked} + array or array + \constant{masked} is a masked
      array with all elements \constant{masked} if array is of a numeric type.
      The same is true if array is a \module{\numarray} array.
   \end{itemize}
\end{enumerate}



\subsection{The constant masked_print_option}
\index{numarray.ma!masked_print_option@\constant{masked_print_option} (constant)}


Another constant, \constant{masked_print_option} controls what happens when
masked arrays and the constant
\index{numarray.ma!masked@\constant{masked}}\constant{masked} are printed:

\begin{methoddesc}[masked_print_option]{display}{} 
   Returns a string that may be used to indicate those elements of an array
   that are masked when the array is converted to a string, as happens with the
   print statement.
\end{methoddesc}

\begin{methoddesc}[masked_print_option]{set_display}{string} 
   This functions can be used to set the string that is used to indicate those
   elements of an array that are masked when the array is converted to a
   string, as happens with the print statement.
\end{methoddesc}

\begin{methoddesc}[masked_print_option]{enable}{flag}
   can be used to enable (\var{flag} = 1, default) the use of the display
   string. If disabled (\var{flag} = 0), the conversion to string becomes
   equivalent to \code{str(self.filled())}.
\end{methoddesc}

\begin{methoddesc}[masked_print_option]{enabled}{}
   Returns the state of the display-enabling flag.
\end{methoddesc}


\paragraph*{Example of masked behavior}
\label{sec:numarray.ma:example-mask-behavior}
\begin{verbatim}
>>> from numarray.ma import *
>>> x=arange(5)
>>> x[3] = masked
>>> print x
[0 ,1 ,2 ,-- ,4 ,]
>>> print repr(x)
array(data = 
 [0,1,2,0,4,],
      mask = 
 [0,0,0,1,0,],
      fill_value=[0,])
>>> print x[3]
--
>>> print x[3] + 1.0
--
>>> print masked + x
[-- ,-- ,-- ,-- ,-- ,]
>>> masked_print_option.enable(0)
>>> print x
[0,1,2,0,4,]
>>> print x + masked
[0,0,0,0,0,]
>>> print filled(x+masked, -99)
[-99,-99,-99,-99,-99,]
\end{verbatim}


\begin{classdesc}{masked_unary_function}{f, fill=0, domain=None}
   Given a \index{unary}unary array function \function{f}, give a function
   which when applied to an argument \var{x} returns \function{f} applied to
   the array \code{filled(x, fill)}, with a mask equal to
   \code{mask_or(getmask(x), domain(x))}.
   
   The argument domain therefore should be a callable object that returns true
   where \var{x} is not in the domain of \function{f}. 
\end{classdesc}

The following domains are also supplied as members of module \module{numarray.ma}:
\begin{classdesc}{domain_check_interval}{a, b)(x}
   Returns true where \code{x < a or y > b}.
\end{classdesc}

\begin{classdesc}{domain_tan}{eps}{x}
   This is true where \code{abs(cos (x)) < eps}, that is, a domain suitable for
   the tangent function.
\end{classdesc}

\begin{classdesc}{domain_greater}{v)(x}
   True where \code{x <= v}.
\end{classdesc}

\begin{classdesc}{domain_greater_equal}{v)(x}
   True where x < v.
\end{classdesc}


\begin{classdesc}{masked_binary_function}{f, fillx=0, filly=0}
   Given a binary array function \function{f}, \code{masked_binary_function(f,
      fillx=0, filly=0)} defines a function whose value at \var{x} is
   \code{f(filled(x, fillx), filled (y, filly))} with a resulting mask of
   \code{mask_or(getmask (x), getmask(y))}. The values \var{fillx} and
   \var{filly} must be chosen so that \code{(fillx, filly)} is in the domain of
   \function{f}.
\end{classdesc}

In addition, an instance of
\index{numarray.ma!masked_binary_function@\class{masked_binary_function}}\class{masked_binary_function}
has two methods defined upon it:

\begin{methoddesc}[masked_binary_function]{reduce}{target, axis = 0}
\end{methoddesc}

\begin{methoddesc}[masked_binary_function]{accumulate}{target, axis = 0}
\end{methoddesc}

\begin{methoddesc}[masked_binary_function]{outer}{a, b}
   These methods perform reduction, accumulation, and applying the function in
   an outer-product-like manner, as discussed in the section
   \ref{sec:ufuncs-have-special-methods}.
\end{methoddesc}


\begin{classdesc}{domained_binary_function}{}
   This class exists to implement division-related operations. It is the same
   as \class{masked_binary_function}, except that a new second argument is a
   domain which is used to mask operations that would otherwise cause failure,
   such as dividing by zero. The functions that are created from this class are
   \function{divide}, \function{remainder} (\function{mod}), and
   \function{fmod}.
\end{classdesc}

The following domains are available for use as the domain argument:

\begin{classdesc}{domain_safe_divide}{)(x, y}
   True where \code{absolute(x)*divide_tolerance > absolute(y)}.  As the
   comments in the code say, \emph{better ideas welcome}. The constant
   \index{numarray.ma!divide_tolerance@\constant{divide_tolerance}}\constant{divide_tolerance}
   is set to \constant{1.e-35} in the source and can be changed by editing its
   value in \file{MA.py} and reinstalling. This domain is used for the divide
   operator.
\end{classdesc}


\section{Examples of Using numarray.ma}
\label{sec:numarray.ma:examples-using-ma}


\subsection{Data with a given value representing missing data}
\label{sec:numarray.ma:data-with-given-repr-miss-data}

Suppose we have read a one-dimensional list of elements named \var{x}. We also
know that if any of the values are \constant{1.e20}, they represent missing
data. We want to compute the average value of the data and the vector of
deviations from average.
\begin{verbatim}
>>> from numarray.ma import *
>>> x = array([0.,1.,2.,3.,4.])
>>> x[2] = 1.e20
>>> y = masked_values (x, 1.e20)
>>> print average(y)
2.0
>>> print y-average(y)
[ -2.00000000e+00, -1.00000000e+00,  --,  1.00000000e+00,
        2.00000000e+00,]
\end{verbatim}


\subsection{Filling in the missing data}
\label{sec:numarray.ma:filling-missing-data}

Suppose now that we wish to print that same data, but with the missing values
replaced by the average value.
\begin{verbatim}
>>> print filled (y, average(y))
\end{verbatim}


\subsection{Numerical operations}
\label{sec:numarray.ma:numerical-operations}

We can do numerical operations without worrying about missing values, dividing
by zero, square roots of negative numbers, etc.
\begin{verbatim}
>>> from numarray.ma import *
>>> x=array([1., -1., 3., 4., 5., 6.], mask=[0,0,0,0,1,0])
>>> y=array([1., 2., 0., 4., 5., 6.], mask=[0,0,0,0,0,1])
>>> print sqrt(x/y)
[  1.00000000e+00,  --,  --,  1.00000000e+00, --,  --,]
\end{verbatim}
Note that four values in the result are invalid: one from a negative square
root, one from a divide by zero, and two more where the two arrays \var{x} and
\var{y} had invalid data. Since the result was of a real type, the print
command printed \code{str(filled(sqrt (x/y)))}.



\subsection{Seeing the mask}
\label{sec:numarray.ma:seeing-mask}

There are various ways to see the mask. One is to print it directly, the other
is to convert to the \function{repr} representation, and a third is get the
mask itself.  Use of \function{getmask} is more robust than \code{x.mask()},
since it will work (returning \constant{None}) if \var{x} is a
\module{\numarray} array or list.
\begin{verbatim}
>>> x = arange(10)
>>> x[3:5] = masked
>>> print x
[0 ,1 ,2 ,-- ,-- ,5 ,6 ,7 ,8 ,9 ,]
>>> print repr(x)
*** Masked array, mask present ***
Data:
[0 ,1 ,2 ,-- ,-- ,5 ,6 ,7 ,8 ,9 ,]
Mask (fill value [0,])
[0,0,0,1,1,0,0,0,0,0,]
>>> print getmask(x)
[0,0,0,1,1,0,0,0,0,0,]
\end{verbatim}



\subsection{Filling it your way}
\label{sec:numarray.ma:filling-it-your-way}

If we want to print the data with \constant{-1}'s where the elements are
masked, we use \function{filled}.
\begin{verbatim}
>>> print filled(z, -1)
[ 1.,-1.,-1., 1.,-1.,-1.,]
\end{verbatim}



\subsection{Ignoring extreme values}
\label{sec:numarray.ma:ignore-extreme-values}

Suppose we have an array \var{d} and we wish to compute the average of the
values in \var{d} but ignore any data outside the range -100. to 100.
\begin{verbatim}
v = masked_outside(d, -100., 100.)
print average(v)
\end{verbatim}


\subsection{Averaging an entire multidimensional array}
\label{sec:numarray.ma:averaging-an-entire}

The problem with averaging over an entire array is that the average function
only reduces one dimension at a time. So to average the entire array,
\function{ravel} it first.
\begin{verbatim}
>>> x
*** Masked array, no mask ***
Data:
[[ 0, 1, 2,]
 [ 3, 4, 5,]
 [ 6, 7, 8,]
 [ 9,10,11,]]
>>> average(x)
*** Masked array, no mask ***
Data:
[ 4.5, 5.5, 6.5,]
>>> average(ravel(x))
5.5
\end{verbatim}




%% Local Variables:
%% mode: LaTeX
%% mode: auto-fill
%% fill-column: 79
%% indent-tabs-mode: nil
%% ispell-dictionary: "american"
%% reftex-fref-is-default: nil
%% TeX-auto-save: t
%% TeX-command-default: "pdfeLaTeX"
%% TeX-master: "numarray"
%% TeX-parse-self: t
%% End:

\chapter{Mlab}
\label{cha:mlab}

%begin{latexonly}
\makeatletter
\py@reset
\makeatother
%end{latexonly}
\declaremodule[numarray.mlab]{extension}{numarray.mlab}
\moduleauthor{The numarray team}{numpy-discussion@lists.sourceforge.net}
\modulesynopsis{mlab}

\section{Matlab(tm) compatible functions}
\label{sec:Matlab-compatible-functions}

\begin{quote}
  \module{numarray.mlab} provides a set of Matlab(tm) compatible functions.
\end{quote}

This will hopefully become a complete set of the basic functions available in
Matlab.  The syntax is kept as close to the Matlab syntax as possible.  One
fundamental change is that the first index in Matlab varies the fastest (as in
FORTRAN).  That means that it will usually perform reductions over columns,
whereas with this object the most natural reductions are over rows.  It's
perfectly possible to make this work the way it does in Matlab if that's
desired.

\begin{funcdesc}{mean}{m, axis=0}
   \label{cha:mlab:mean}
   \label{func:mean}
   returns the mean along the axis'th dimension of m.  Note: if m is an integer
   array, the result will be floating point. This was changed in release 10.1;
   previously, a meaningless integer divide was used.
\end{funcdesc}
\begin{funcdesc}{median}{m}
   \label{cha:mlab:median}
   \label{func:median}
   returns a mean of m along the first dimension of m.
\end{funcdesc}
\begin{funcdesc}{min}{m, axis=0}
   \label{cha:mlab:min}
   \label{func:min}
   returns the minimum along the axis'th dimension of m.
\end{funcdesc}
\begin{funcdesc}{msort}{m}
   \label{cha:mlab:msort}
   \label{func:msort}
   returns a sort along the first dimension of m as in MATLAB.
\end{funcdesc}
\begin{funcdesc}{prod}{m, axis=0}
   \label{cha:mlab:prod}
   \label{func:prod}
   returns the product of the elements along the axis'th dimension of m.
\end{funcdesc}
\begin{funcdesc}{ptp}{m, axis = 0}
   \label{cha:mlab:ptp}
   \label{func:ptp}
   returns the maximum - minimum along the axis'th dimension of m.
\end{funcdesc}
\begin{funcdesc}{rand}{d1, ..., dn}
   \label{cha:mlab:rand}
   \label{func:rand}
   returns a matrix of the given dimensions which is initialized to random numbers from a uniform distribution in
the range [0,1).
\end{funcdesc}
\begin{funcdesc}{rot90}{m,k=1}
   \label{cha:mlab:rot90}
   \label{func:rot90}
   returns the matrix found by rotating m by k*90 degrees in the counterclockwise direction.
\end{funcdesc}
\begin{funcdesc}{sinc}{x}
   \label{cha:mlab:sinc}
   \label{func:sinc}
   returns sin(pi*x)
\end{funcdesc}
\begin{funcdesc}{squeeze}{a}
   \label{cha:mlab:squeeze}
   \label{func:squeeze}
   removes any ones from the shape of a
\end{funcdesc}
\begin{funcdesc}{std}{m, axis = 0}
   \label{cha:mlab:std}
   \label{func:std}
   returns the unbiased estimate of the population standard deviation from a
   sample along the axis'th dimension of m. (That is, the denominator for the
   calculation is n-1, not n.) 
\end{funcdesc}
\newpage
\begin{funcdesc}{sum}{m, axis=0}
   \label{cha:mlab:sum}
   \label{func:sum}
   returns the sum of the elements along the axis'th dimension of m.
\end{funcdesc}
\begin{funcdesc}{svd}{m}
   \label{cha:mlab:svd}
   \label{func:svd}
   return the singular value decomposition of m [u,x,v]
\end{funcdesc}
\begin{funcdesc}{trapz}{y,x=None}
   \label{cha:mlab:trapz}
   \label{func:trapz}
   integrates y = f(x) using the trapezoidal rule
\end{funcdesc}
\begin{funcdesc}{tri}{N, M=N, k=0, typecode=None}
   \label{cha:mlab:tri}
   \label{func:tri}
   returns a N-by-M matrix where all the diagonals starting from lower left corner up to the k-th are all ones.
\end{funcdesc}
\begin{funcdesc}{tril}{m,k=0}
   \label{cha:mlab:tril}
   \label{func:tril}
   returns the elements on and below the k-th diagonal of m. k=0 is the main
   diagonal, \begin{math} k > 0\end{math} is above and \begin{math}k < 0
   \end{math} is below the main diagonal.
\end{funcdesc}

\begin{funcdesc}{triu}{m,k=0}
   \label{sec:mlab-functions:triu}
   \label{func:triu}
   returns the elements on and above the k-th diagonal of m. k=0 is the main
   diagonal, \begin{math}k > 0\end{math} is above and \begin{math}k <
   0\end{math} is below the main diagonal.
\end{funcdesc}

%% Local Variables:
%% mode: LaTeX
%% mode: auto-fill
%% fill-column: 79
%% indent-tabs-mode: nil
%% ispell-dictionary: "american"
%% reftex-fref-is-default: nil
%% TeX-auto-save: t
%% TeX-command-default: "pdfeLaTeX"
%% TeX-master: "numarray"
%% TeX-parse-self: t
%% End:

\chapter{Random Numbers}
\label{cha:random-array}

%begin{latexonly}
\makeatletter \py@reset \makeatother
%end{latexonly}
\declaremodule[numarray.randomarray]{extension}{numarray.random_array}
\moduleauthor{The numarray team}{numpy-discussion@lists.sourceforge.net}
\modulesynopsis{Random Numbers}

\begin{quote}
  The \module{numarray.random_array} module (in conjunction with the
  \module{numarray.random_array.ranlib} submodule) provides a high-level
  interface to ranlib, which provides a good quality C implementation of a
  random-number generator.
\end{quote}

\section{General functions}
\label{sec:RA:general-functions}

\begin{funcdesc}{seed}{x=0, y=0}
The \function{seed} function takes two integers and sets the two seeds of the
random number generator to those values. If the default values of 0 are used
for both \var{x} and \var{y}, then a seed is generated from the current time,
providing a pseudo-random seed.
\end{funcdesc}

\begin{funcdesc}{get_seed}{}
This function returns the two seeds used by the current random-number
generator. It is most often used to find out what seeds the \function{seed}
function chose at the last iteration.  \remark{Are there any thread-safety
issues?}
\end{funcdesc}

\begin{funcdesc}{random}{shape=[]}
   The \function{random} function takes a \var{shape}, and returns an array of
   \class{Float} numbers between 0.0 and 1.0.  Neither 0.0 nor 1.0 is ever
   returned by this function.  The array is filled from the generator following
   the canonical array organization.
   
   If no argument is specified, the function returns a single floating point
   number, not an array.
   
   \note{See discussion of the \member{flat} attribute in section
      \ref{mem:numarray:flat}.}
\end{funcdesc}

\begin{funcdesc}{uniform}{minimum, maximum, shape=[]}
   The \function{uniform} function returns an array of the specified
   \var{shape} and containing \class{Float} random numbers strictly between
   \var{minimum} and \var{maximum}.
   
   The \var{minimum} and \var{maximum} arguments can be arrays. If this is the
   case, and the output \var{shape} is specified, \var{minimum} and
   \var{maximum} are broadcasted if their dimensions are not equal to
   \var{shape}. If \var{shape} is not specified, the shape of the output is
   equal to the shape of \var{minimum} and \var{maximum} after broadcasting.
   
   If no \var{shape} is specified, and \var{minimum} and \var{maximum} are
   scalars, a single value is returned.
\end{funcdesc}

\begin{funcdesc}{randint}{minimum, maximum, shape=[]}
   The \function{randint} function returns an array of the specified
   \var{shape} and containing random (standard) integers greater than or equal
   to \var{minimum} and strictly less than \var{maximum}. 
   
   The \var{minimum} and \var{maximum} arguments can be arrays. If this is the
   case, and the output \var{shape} is specified, \var{minimum} and
   \var{maximum} are broadcasted if their dimensions are not equal to
   \var{shape}. If \var{shape} is not specified, the shape of the output is
   equal to the shape of \var{minimum} and \var{maximum} after broadcasting.
   
   If no \var{shape} is specified, and \var{minimum} and \var{maximum} are
   scalars, a single value is returned.
\end{funcdesc}

\begin{funcdesc}{permutation}{n}
   The \function{permutation} function returns an array of the integers between
   \code{0} and \code{\var{n}-1}, in an array of shape \code{(n,)} with its
   elements randomly permuted.
\end{funcdesc}


\section{Special random number distributions}
\label{sec:RA:special-distribution}



\subsection{Random floating point number distributions}
\label{sec:RA:float-distribution}

\begin{funcdesc}{beta}{a, b, shape=[]}
   The \function{beta} function returns an array of the specified shape that
   contains \class{Float} numbers $\beta$-distributed with $\alpha$-parameter
   \var{a} and $\beta$-parameter \var{b}. 
   
   The \var{a} and \var{b} arguments can be arrays. If this is the case, and
   the output \var{shape} is specified, \var{a} and \var{b} are broadcasted if
   their dimensions are not equal to \var{shape}. If \var{shape} is not
   specified, the shape of the output is equal to the shape of \var{a} and
   \var{b} after broadcasting.
   
   If no \var{shape} is specified, and \var{a} and \var{b} are
   scalars, a single value is returned.
\end{funcdesc}

\begin{funcdesc}{chi_square}{df, shape=[]}
   The \function{chi_square} function returns an array of the specified
   \var{shape} that contains \class{Float} numbers with the
   $\chi^2$-distribution with \var{df} degrees of freedom.
   
   The \var{df} argument can be an array. If this is the case, and the output
   \var{shape} is specified, \var{df} is broadcasted if its dimensions are not
   equal to \var{shape}. If \var{shape} is not specified, the shape of the
   output is equal to the shape of \var{df}.
   
   If no \var{shape} is specified, and \var{df} is a scalar, a single value is
   returned.
\end{funcdesc}

\begin{funcdesc}{exponential}{mean, shape=[]}
   The \function{exponential} function returns an array of the specified
   \var{shape} that contains \class{Float} numbers exponentially distributed
   with the specified \var{mean}. 
   
   The \var{mean} argument can be an array. If this is the case, and the output
   \var{shape} is specified, \var{mean} is broadcasted if its dimensions are
   not equal to \var{shape}. If \var{shape} is not specified, the shape of the
   output is equal to the shape of \var{mean}.
   
   If no \var{shape} is specified, and \var{mean} is a scalar, a single value
   is returned.
\end{funcdesc}

\begin{funcdesc}{F}{dfn, dfd, shape=[]}
  The \function{F} function returns an array of the specified \var{shape} that
  contains \class{Float} numbers with the F-distribution with \var{dfn} degrees
  of freedom in the numerator and \var{dfd} degrees of freedom in the
  denominator.
   
  The \var{dfn} and \var{dfd} arguments can be arrays. If this is the case, and
  the output \var{shape} is specified, \var{dfn} and \var{dfd} are broadcasted
  if their dimensions are not equal to \var{shape}. If \var{shape} is not
  specified, the shape of the output is equal to the shape of \var{dfn} and
  \var{dfd} after broadcasting.
   
  If no \var{shape} is specified, and \var{dfn} and \var{dfd} are scalars, a
  single value is returned.
\end{funcdesc}

\begin{funcdesc}{gamma}{a, r, shape=[]}
   The \function{gamma} function returns an array of the specified \var{shape}
   that contains \class{Float} numbers $\beta$-distributed with location
   parameter \var{a} and distribution shape parameter \var{r}.
   
   The \var{a} and \var{r} arguments can be arrays. If this is the case, and
   the output \var{shape} is specified, \var{a} and \var{r} are broadcasted if
   their dimensions are not equal to \var{shape}. If \var{shape} is not
   specified, the shape of the output is equal to the shape of \var{a} and
   \var{r} after broadcasting.
   
   If no \var{shape} is specified, and \var{a} and \var{r} are scalars, a
   single value is returned.
\end{funcdesc}

\begin{funcdesc}{multivariate_normal}{mean, cov, shape=[]}
   The multivariate_normal function takes a one dimensional array argument
   \var{mean} and a two dimensional array argument \var{cov}. Suppose
   the shape of \var{mean} is \code{(n,)}. Then the shape of \var{cov}
   must be \code{(n,n)}. The function returns an array of \class{Float}s.
   
   The effect of the \var{shape} parameter is:
   \begin{itemize}
   \item If no \var{shape} is specified, then an array with shape \code{(n,)}
      is returned containing a vector of numbers with a multivariate normal
      distribution with the specified mean and covariance.
   \item If \var{shape} is specified, then an array of such vectors is
      returned.  The shape of the output is \code{shape.append((n,))}. The
      leading indices into the output array select a multivariate normal from
      the array. The final index selects one number from within the
      multivariate normal.
 \end{itemize}
 In either case, the behavior of \function{multivariate_normal} is undefined if
 \var{cov} is not symmetric and positive definite.
\end{funcdesc}

\begin{funcdesc}{normal}{mean, std, shape=[]}
   The \function{normal} function returns an array of the specified \var{shape}
   that contains \class{Float} numbers normally distributed with the specified
   \var{mean} and standard deviation \var{std}. 
   
   The \var{mean} and \var{std} arguments can be arrays. If this is the
   case, and the output \var{shape} is specified, \var{mean} and \var{std}
   are broadcasted if their dimensions are not equal to \var{shape}. If
   \var{shape} is not specified, the shape of the output is equal to the shape
   of \var{mean} and \var{std} after broadcasting.
   
   If no \var{shape} is specified, and \var{mean} and \var{std} are scalars, a
   single value is returned.
\end{funcdesc}

\begin{funcdesc}{noncentral_chi_square}{df, nonc, shape=[]}
   The \function{noncentral_chi_square} function returns an array of the
   specified \var{shape} that contains \class{Float} numbers with
   the$\chi^2$-distribution with \var{df} degrees of freedom and noncentrality
   parameter \var{nconc}.
   
   The \var{df} and \var{nonc} arguments can be arrays. If this is the case,
   and the output \var{shape} is specified, \var{df} and \var{nonc} are
   broadcasted if their dimensions are not equal to \var{shape}. If \var{shape}
   is not specified, the shape of the output is equal to the shape of \var{df}
   and \var{nonc} after broadcasting.
   
   If no \var{shape} is specified, and \var{df} and \var{nonc} are scalars, a
   single value is returned.
\end{funcdesc}

\begin{funcdesc}{noncentral_F}{dfn, dfd, nconc, shape=[]}
   The \function{noncentral_F} function returns an array of the specified
   \var{shape} that contains \class{Float} numbers with the F-distribution with
   \var{dfn} degrees of freedom in the numerator, \var{dfd} degrees of freedom
   in the denominator, and noncentrality parameter \var{nconc}.
   
   The \var{dfn}, \var{dfd} and \var{nonc} arguments can be arrays. If this is
   the case, and the output \var{shape} is specified, \var{dfn}, \var{dfd} and
   \var{nonc} are broadcasted if their dimensions are not equal to \var{shape}.
   If \var{shape} is not specified, the shape of the output is equal to the
   shape of \var{dfn}, \var{dfd} and \var{nonc} after broadcasting.
   
   If no \var{shape} is specified, and \var{dfn}, \var{dfd} and \var{nonc} are
   scalars, a single value is returned.
\end{funcdesc}

\begin{funcdesc}{standard_normal}{shape=[]}
   The \function{standard_normal} function returns an array of the specified
   \var{shape} that contains \class{Float} numbers normally (Gaussian)
   distributed with mean zero and variance and standard deviation one. 
   
   If no \var{shape} is specified, a single number is returned.
\end{funcdesc}

\begin{funcdesc}{F}{dfn, dfd, shape=[]}
Returns array of F distributed random numbers with \var{dfn} degrees of freedom
in the numerator and \var{dfd} degrees of freedom in the denominator.
\end{funcdesc}

\begin{funcdesc}{noncentral_F}{dfn, dfd, nconc, shape=[]}
 Returns array of noncentral F distributed random numbers with dfn degrees of
 freedom in the numerator and dfd degrees of freedom in the denominator, and
 noncentrality parameter nconc.
\end{funcdesc}



\subsection{Random integer number distributions }
\label{sec:RA:int-distributions}

\begin{funcdesc}{binomial}{trials, p, shape=[]}
   The \function{binomial} function returns an array with the specified
   \var{shape} that contains \class{Integer} numbers with the binomial
   distribution with \var{trials} and event probability \var{p}. In other
   words, each value in the returned array is the number of times an event with
   probability \var{p} occurred within \var{trials} repeated trials.
   
   The \var{trials} and \var{p} arguments can be arrays. If this is the
   case, and the output \var{shape} is specified, \var{trials} and \var{p}
   are broadcasted if their dimensions are not equal to \var{shape}. If
   \var{shape} is not specified, the shape of the output is equal to the shape
   of \var{trials} and \var{p} after broadcasting.
   
   If no \var{shape} is specified, and \var{trials} and \var{p} are scalars,
   a single value is returned.
\end{funcdesc}

\begin{funcdesc}{negative_binomial}{trials, p, shape=[]}
  The \function{negative_binomial} function returns an array with the specified
  \var{shape} that contains \class{Integer} numbers with the negative binomial
  distribution with \var{trials} and event probability \var{p}.
   
  The \var{trials} and \var{p} arguments can be arrays. If this is the case,
  and the output \var{shape} is specified, \var{trials} and \var{p} are
  broadcasted if their dimensions are not equal to \var{shape}. If \var{shape}
  is not specified, the shape of the output is equal to the shape of
  \var{trials} and \var{p} after broadcasting.
   
   If no \var{shape} is specified, and \var{trials} and \var{p} are scalars,
   a single value is returned.
\end{funcdesc}

\begin{funcdesc}{multinomial}{trials, probs, shape=[]}
   The \function{multinomial} function returns an array with that contains
   integer numbers with the multinomial distribution with \var{trials} and
   event probabilities given in \var{probs}.  \var{probs} must be a one
   dimensional array.  There are \code{len(probs)+1} events. \code{probs[i]} is
   the probability of the i-th event for \code{0<=i<len(probs)}. The
   probability of event \code{len(probs)} is \code{1.-Numeric.sum(prob)}.
   
   The function returns an integer array of shape
   \code{shape~+~(len(probs)+1,)}.  If \var{shape} is not specified this is one
   multinomially distributed vector of shape \code{(len(prob)+1,)}.  Otherwise
   each \code{returnarray[i,j,...,:]} is an integer array of shape
   \code{(len(prob)+1,)} containing one multinomially distributed vector.
\end{funcdesc}

\begin{funcdesc}{poisson}{mean, shape=[]}
   The \function{poisson} function returns an array with the specified shape
   that contains \class{Integer} numbers with the Poisson distribution with the
   specified \var{mean}.
   
   The \var{mean} argument can be an array. If this is the case, and the output
   \var{shape} is specified, \var{mean} is broadcasted if its dimensions are
   not equal to \var{shape}. If \var{shape} is not specified, the shape of the
   output is equal to the shape of \var{mean}.
   
   If no \var{shape} is specified, and \var{mean} is a scalar, a single value
   is returned.
\end{funcdesc}



\section{Examples}
\label{sec:examples}

Some example uses of the \module{numarray.random_array} module. \note{Naturally the exact
   output of running these examples will be different each time!} \remark{Make
   sure these examples are correct!}
\begin{verbatim}
>>> from numarray.random_array import *
>>> seed() # Set seed based on current time
>>> print get_seed() # Find out what seeds were used
(897800491, 192000)
>>> print random()
0.0528018975065
>>> print random((5,2))
[[ 0.14833829 0.99031458]
[ 0.7526806 0.09601787]
[ 0.1895229 0.97674777]
[ 0.46134511 0.25420982]
[ 0.66132009 0.24864472]]
>>> print uniform(-1,1,(10,))
[ 0.72168852 -0.75374185 -0.73590945 0.50488248 -0.74462822 0.09293685
-0.65898308 0.9718067 -0.03252475 0.99611011]
>>> print randint(0,100, (12,))
[28 5 96 19 1 32 69 40 56 69 53 44]
>>> print permutation(10)
[4 2 8 9 1 7 3 6 5 0]
>>> seed(897800491, 192000) # resetting the same seeds
>>> print random() # yields the same numbers
0.0528018975065
\end{verbatim}
Most of the functions in this package take zero or more distribution specific
parameters plus an optional \var{shape} parameter. The \var{shape} parameter
gives the shape of the output array:
\begin{verbatim}
>>> from numarray.random_array import *
>>> print standard_normal()
-0.435568600893
>>> print standard_normal(5)
[-1.36134553 0.78617644 -0.45038718 0.18508556 0.05941355]
>>> print standard_normal((5,2))
[[ 1.33448863 -0.10125473]
[ 0.66838062 0.24691346]
[-0.95092064 0.94168913]
[-0.23919107 1.89288616]
[ 0.87651485 0.96400219]]
>>> print normal(7., 4., (5,2)) #mean=7, std. dev.=4
[[ 2.66997623 11.65832615]
[ 6.73916003 6.58162862]
[ 8.47180378 4.30354905]
[ 1.35531998 -2.80886841]
[ 7.07408469 11.39024973]]
>>> print exponential(10., 5) #mean=10
[ 18.03347754 7.11702306 9.8587961 32.49231603 28.55408891]
>>> print beta(3.1, 9.1, 5) # alpha=3.1, beta=9.1
[ 0.1175056 0.17504358 0.3517828 0.06965593 0.43898219]
>>> print chi_square(7, 5) # 7 degrees of freedom (dfs)
[ 11.99046516 3.00741053 4.72235727 6.17056274 8.50756836]
>>> print noncentral_chi_square(7, 3, 5) # 7 dfs, noncentrality 3
[ 18.28332138 4.07550335 16.0425396 9.51192093 9.80156231]
>>> F(5, 7, 5) # 5 and 7 dfs
array([ 0.24693671, 3.76726145, 0.66883826, 0.59169068, 1.90763224])
>>> noncentral_F(5, 7, 3., 5) # 5 and 7 dfs, noncentrality 3
array([ 1.17992553, 0.7500126 , 0.77389943, 9.26798989, 1.35719634])
>>> binomial(32, .5, 5) # 32 trials, prob of an event = .5
array([12, 20, 21, 19, 17])
>>> negative_binomial(32, .5, 5) # 32 trials: prob of an event = .5
array([21, 38, 29, 32, 36])
\end{verbatim}
Two functions that return generate multivariate random numbers (that is, random
vectors with some known relationship between the elements of each vector,
defined by the distribution). They are \function{multivariate_normal} and
\function{multinomial}. For these two functions, the lengths of the leading
axes of the output may be specified. The length of the last axis is determined
by the length of some other parameter.
\begin{verbatim}
>>> multivariate_normal([1,2], [[1,2],[2,1]], [2,3])
array([[[ 0.14157988, 1.46232224],
[-1.11820295, -0.82796288],
[ 1.35251635, -0.2575901 ]],
[[-0.61142141, 1.0230465 ],
[-1.08280948, -0.55567217],
[ 2.49873002, 3.28136372]]])
>>> x = multivariate_normal([10,100], [[1,2],[2,1]], 10000)
>>> x_mean = sum(x)/10000
>>> print x_mean
[ 9.98599893 100.00032416]
>>> x_minus_mean = x - x_mean
>>> cov = matrixmultiply(transpose(x_minus_mean), x_minus_mean) / 9999.
>>> cov
array([[ 2.01737122, 1.00474408],
[ 1.00474408, 2.0009806 ]])
\end{verbatim}
The a priori probabilities for a multinomial distribution must sum to one. The
prior probability argument to \function{multinomial} doesn't give the prior
probability of the last event: it is computed to be one minus the sum of the
others.
\begin{verbatim}
>>> multinomial(16, [.1, .4, .2]) # prior probabilities [.1, .4, .2, .3]
array([2, 7, 1, 6])
>>> multinomial(16, [.1, .4, .2], [2,3]) # output shape [2,3,4]
array([[[ 1, 9, 1, 5],
[ 0, 10, 3, 3],
[ 4, 9, 3, 0]],
[[ 1, 6, 1, 8],
[ 3, 4, 5, 4],
[ 1, 5, 2, 8]]])
\end{verbatim}
Many of the functions accept arrays or sequences for the distribution
arguments. If no \var{shape} argument is given, then the shape of the output is
determined by the shape of the parameter argument. For instance:
\begin{verbatim}
>>> beta([5.0, 50.0], [10.0, 100.])
array([ 0.54379648,  0.35352072])
\end{verbatim}
Broadcasting rules apply if two or more arguments are arrays:
\begin{verbatim}
>>> beta([5.0, 50.0], [[10.0, 100.], [20.0, 200.0]])
array([[ 0.30204576,  0.32154009],
       [ 0.10851908,  0.19207685]])
\end{verbatim}
The \var{shape} argument can still be used to specify the output shape. Any
array argument will be broadcasted to have the given shape:
\begin{verbatim}
>>> beta(5.0, [10.0, 100.0], shape = (3, 2))
array([[ 0.49521708,  0.02218186],
       [ 0.21000148,  0.04366644],
       [ 0.43169656,  0.05285903]])
\end{verbatim}
%% Local Variables:
%% mode: LaTeX
%% mode: auto-fill
%% fill-column: 79
%% indent-tabs-mode: nil
%% ispell-dictionary: "american"
%% reftex-fref-is-default: nil
%% TeX-auto-save: t
%% TeX-command-default: "pdfeLaTeX"
%% TeX-master: "numarray"
%% TeX-parse-self: t
%% End:

\chapter{Multi-dimensional image processing}
\label{cha:ndimage}

%begin{latexonly}
\makeatletter
\py@reset
\makeatother
%end{latexonly}
\declaremodule[numarray.ndimage]{extension}{numarray.nd_image}
\moduleauthor{Peter Verveer}{verveer@users.sourceforge.net}
\modulesynopsis{Multidimensional image analysis functions}

\begin{quote}
  The \module{numarray.nd\_image} module provides functions for
  multidimensional image analysis.
\end{quote}

\section{Introduction}
Image processing and analysis are generally seen as operations on
two-dimensional arrays of values. There are however a number of fields
where images of higher dimensionality must be analyzed. Good examples
of these are medical imaging and biological imaging. \module{numarray}
is suited very well for this type of applications due its inherent
multi-dimensional nature. The \module{numarray.nd\_image} packages
provides a number of general image processing and analysis functions
that are designed to operate with arrays of arbitrary dimensionality.
The packages currently includes functions for linear and non-linear
filtering, binary morphology, B-spline interpolation, and object
measurements.

\section{Properties shared by all functions}
All functions share some common properties. Notably, all functions allow the 
specification of an output array with the \var{output} argument. With this 
argument you can specify an array that will be changed in-place with the 
result with the operation. In this case the result is not returned. Usually, 
using the \var{output} argument is more efficient, since an existing array 
is used to store the result.

The type of arrays returned is dependent on the type of operation, but it is  in most cases equal to the type of the input. If, however, the \var{output} argument is used, the type of the result is equal to the type of the specified output argument. If no output argument is given, it is still possible to specify what the result of the output should be. This is done by simply assigning the desired numarray type object to the output argument. For example:
\begin{verbatim}
>>> print correlate(arange(10), [1, 2.5])
[ 0  2  6  9 13 16 20 23 27 30]
>>> print correlate(arange(10), [1, 2.5], output = Float64)
[  0.    2.5   6.    9.5  13.   16.5  20.   23.5  27.   30.5]                   
\end{verbatim}
\note{In previous versions of \module{numarray.nd\_image}, some functions accepted the \var{output_type} argument to achieve the same effect. This argument is still supported, but its use will generate an deprecation warning. In a future version all instances of this argument will be removed. The preferred way to specify an output type, is by using the \var{output} argument, either by specifying an output array of the desired type, or by specifying the type of the output that is to be returned.}
%
\section{Filter functions}
\label{sec:ndimage:filter-functions}
The functions described in this section all perform some type of spatial
filtering of the the input array: the elements in the output are some 
function of the values in the neighborhood of the corresponding input 
element. We refer to this neighborhood of elements as the filter kernel, 
which is often rectangular in shape but may also have an arbitrary 
footprint. Many of the functions described below allow you to define the 
footprint of the kernel, by passing a mask through the \var{footprint} 
parameter. For example a cross shaped kernel can be defined as follows:
\begin{verbatim}
>>> footprint = array([[0,1,0],[1,1,1],[0,1,0]])
>>> print footprint
[[0 1 0]
 [1 1 1]
 [0 1 0]]
\end{verbatim}
Usually the origin of the kernel is at the center calculated by dividing 
the dimensions of the kernel shape by two.  For instance, the origin of a
one-dimensional kernel of length three is at the second element. Take for
example the correlation of a one-dimensional array with a filter of
length 3 consisting of ones:
\begin{verbatim}
>>> a = [0, 0, 0, 1, 0, 0, 0]
>>> correlate1d(a, [1, 1, 1])
[0 0 1 1 1 0 0]
\end{verbatim}
Sometimes it is convenient to choose a different origin for the kernel. For
this reason most functions support the \var{origin} parameter which gives 
the origin of the filter relative to its center. For example:
\begin{verbatim}
>>> a = [0, 0, 0, 1, 0, 0, 0]
>>> print correlate1d(a, [1, 1, 1], origin = -1)
[0 1 1 1 0 0 0]
\end{verbatim}
The effect is a shift of the result towards the left. This feature will not 
be needed very often, but it may be useful especially for filters that have 
an even size.  A good example is the calculation of backward and forward
differences:
\begin{verbatim}
>>> a = [0, 0, 1, 1, 1, 0, 0]
>>> print correlate1d(a, [-1, 1])              ## backward difference
[ 0  0  1  0  0 -1  0]
>>> print correlate1d(a, [-1, 1], origin = -1) ## forward difference
[ 0  1  0  0 -1  0  0]
\end{verbatim}
We could also have calculated the  forward difference as follows:
\begin{verbatim}
>>> print correlate1d(a, [0, -1, 1])
[ 0  1  0  0 -1  0  0]
\end{verbatim}
however, using the origin parameter instead of a larger kernel is more
efficient. For multi-dimensional kernels \var{origin} can be a number, in 
which case the origin is assumed to be equal along all axes, or a sequence  
giving the origin along each axis.

Since the output elements are a function of elements in the neighborhood of 
the input elements, the borders of the array need to be dealt with 
appropriately by providing the values outside the borders. This is done by 
assuming that the arrays are extended beyond their boundaries according 
certain boundary conditions. In the functions described below, the boundary 
conditions can be selected using the \var{mode} parameter which must be a 
string with the name of the boundary condition.  Following boundary 
conditions are currently supported:
\begin{tableiii}{l|l|l}{constant}{Boundary condition}{Description}{Example}
  \lineiii{"nearest"}{Use the value at the boundary}
  {\constant{[1 2 3]->[1 1 2 3 3]}}
  \lineiii{"wrap"}{Periodically replicate the array}
  {\constant{[1 2 3]->[3 1 2 3 1]}}
  \lineiii{"reflect"}{Reflect the array at the boundary}
  {\constant{[1 2 3]->[1 1 2 3 3]}}
  \lineiii{"constant"}{Use a constant value, default value is 0.0}
  {\constant{[1 2 3]->[0 1 2 3 0]}}
\end{tableiii}
The \constant{"constant"} mode is special since it needs an additional
parameter to specify the constant value that should be used.

\note{The easiest way to implement such boundary conditions would be to 
copy the data to a larger array and extend the data at the borders 
according to the boundary conditions. For large arrays and large filter 
kernels, this would be very memory consuming, and the functions described 
below therefore use a different approach that does not require allocating 
large temporary buffers.}

\subsection{Correlation and convolution}

\begin{funcdesc}{correlate1d}{input, weights, axis=-1, output=None, 
    mode='reflect', cval=0.0, origin=0, output_type=None} The
  \function{correlate1d} function calculates a one-dimensional correlation
  along the given axis. The lines of the array along the given axis are
  correlated with the given \var{weights}. The \var{weights} parameter must 
  be a one-dimensional sequences of numbers.
\end{funcdesc}

\begin{funcdesc}{correlate}{input, weights, output=None, mode='reflect', 
    cval=0.0, origin=0, output_type=None} The function \function{correlate}
  implements multi-dimensional correlation of the input array with a given
  kernel.
\end{funcdesc}

\begin{funcdesc}{convolve1d}{input, weights, axis=-1, output=None, 
    mode='reflect', cval=0.0, origin=0, output_type=None} The
  \function{convolve1d} function calculates a one-dimensional convolution 
  along the given axis. The lines of the array along the given axis are 
  convoluted with the given \var{weights}. The \var{weights} parameter must 
  be a one-dimensional sequences of numbers.
  
  \note{A convolution is essentially a correlation after mirroring the 
  kernel. As a result, the \var{origin} parameter behaves differently than 
  in the case of a correlation: the result is shifted in the opposite 
  directions.}
\end{funcdesc}

\begin{funcdesc}{convolve}{input, weights, output=None, mode='reflect', 
    cval=0.0, origin=0, output_type=None} The function \function{convolve}
  implements multi-dimensional convolution of the input array with a given
  kernel.
  
  \note{A convolution is essentially a correlation after mirroring the 
  kernel. As a result, the \var{origin} parameter behaves differently than 
  in the case of a correlation: the results is shifted in the opposite 
  direction.}
\end{funcdesc}

\subsection{Smoothing filters}
\label{sec:ndimage:filter-functions:smoothing}

\begin{funcdesc}{gaussian_filter1d}{input, sigma, axis=-1, order=0, 
    output=None, mode='reflect', cval=0.0, output_type=None} The
  \function{gaussian_filter1d} function implements a one-dimensional 
  Gaussian
  filter. The standard-deviation of the Gaussian filter is passed through 
  the parameter \var{sigma}. Setting \var{order}=0 corresponds to 
  convolution with a Gaussian kernel.  An order of 1, 2, or 3 corresponds 
  to convolution with the first, second or third derivatives of a Gaussian. 
  Higher order derivatives are not implemented.
\end{funcdesc}

\begin{funcdesc}{gaussian_filter}{input, sigma, order=0, output=None, 
  mode='reflect', cval=0.0, output_type=None} The 
  \function{gaussian_filter} function implements a multi-dimensional 
  Gaussian filter. The standard-deviations of the Gaussian filter along 
  each axis are passed through the parameter \var{sigma} as a sequence or 
  numbers.  If \var{sigma} is not a sequence but a single number, the 
  standard deviation of the filter is equal along all directions. The 
  order of the filter can be specified separately for each axis. An order 
  of 0 corresponds to convolution with a Gaussian kernel. An order of 1, 
  2, or 3 corresponds to convolution with the first, second or
  third derivatives of a Gaussian. Higher order derivatives are not
  implemented. The \var{order} parameter must be a number, to specify the 
  same order for all axes, or a sequence of numbers to specify a different 
  order for each axis.
  
  \note{The multi-dimensional filter is implemented as a sequence of
    one-dimensional Gaussian filters. The intermediate arrays are stored in 
    the same data type as the output.  Therefore, for output types with a 
    lower precision, the results may be imprecise because intermediate 
    results may be stored with insufficient precision. This can be 
    prevented by specifying a more precise output type.}
\end{funcdesc}

\begin{funcdesc}{uniform_filter1d}{input, size, axis=-1, output=None,
    mode='reflect', cval=0.0, origin=0, output_type=None} The
  \function{uniform_filter1d} function calculates a one-dimensional uniform
  filter of the given \var{size} along the given axis.
\end{funcdesc}

\begin{funcdesc}{uniform_filter}{input, size, output=None, mode='reflect', 
    cval=0.0, origin=0, output_type=None} The \function{uniform_filter}
  implements a multi-dimensional uniform filter.  The sizes of the uniform 
  filter are given for each axis as a sequence of integers by the 
  \var{size} parameter. If \var{size} is not a sequence, but a single 
  number, the sizes along all axis are assumed to be equal.
    
  \note{The multi-dimensional filter is implemented as a sequence of
    one-dimensional uniform filters. The intermediate arrays are stored in 
    the same data type as the output. Therefore, for output types with a 
    lower precision, the results may be imprecise because intermediate 
    results may be stored with insufficient precision. This can be 
    prevented by specifying a
    more precise output type.}
  \end{funcdesc}

\subsection{Filters based on order statistics}

\begin{funcdesc}{minimum_filter1d}{input, size, axis=-1, output=None, 
    mode='reflect', cval=0.0, origin=0} The \function{minimum_filter1d}
  function calculates a one-dimensional minimum filter of given \var{size}
  along the given axis.
\end{funcdesc}

\begin{funcdesc}{maximum_filter1d}{input, size, axis=-1, output=None, 
    mode='reflect', cval=0.0, origin=0} The \function{maximum_filter1d}
  function calculates a one-dimensional maximum filter of given \var{size}
  along the given axis.
\end{funcdesc}

\begin{funcdesc}{minimum_filter}{input,  size=None, footprint=None, 
    output=None, mode='reflect', cval=0.0, origin=0} The
  \function{minimum_filter} function calculates a multi-dimensional minimum
  filter. Either the sizes of a rectangular kernel or the footprint of the
  kernel must be provided. The \var{size} parameter, if provided, must be a
  sequence of sizes or a single number in which case the size of the filter 
  is assumed to be equal along each axis. The \var{footprint}, if provided, 
  must be an array that defines the shape of the kernel by its non-zero 
  elements.
\end{funcdesc}

\begin{funcdesc}{maximum_filter}{input,  size=None, footprint=None, 
    output=None, mode='reflect', cval=0.0, origin=0} The
  \function{maximum_filter} function calculates a multi-dimensional maximum
  filter. Either the sizes of a rectangular kernel or the footprint of the
  kernel must be provided. The \var{size} parameter, if provided, must be a
  sequence of sizes or a single number in which case the size of the filter 
  is assumed to be equal along each axis. The \var{footprint}, if provided, 
  must be an array that defines the shape of the kernel by its non-zero 
  elements.
\end{funcdesc}

\begin{funcdesc}{rank_filter}{input, rank, size=None, footprint=None,
  output=None, mode='reflect', cval=0.0, origin=0} The 
  \function{rank_filter}
  function calculates a multi-dimensional rank filter.  The \var{rank} may 
  be less then zero, i.e., \var{rank}=-1 indicates the largest element. 
  Either the sizes of a rectangular kernel or the footprint of the kernel 
  must be provided. The \var{size} parameter, if provided, must be a 
  sequence of sizes or a single number in which case the size of the filter 
  is assumed to be equal along each axis. The \var{footprint}, if provided, 
  must be an array that defines the shape of the kernel by its non-zero 
  elements.
\end{funcdesc}

\begin{funcdesc}{percentile_filter}{input, percentile, size=None, 
  footprint=None, output=None, mode='reflect', cval=0.0, origin=0} The
  \function{percentile_filter} function calculates a multi-dimensional
  percentile filter.  The \var{percentile} may be less then zero, i.e.,
  \var{percentile}=-20 equals \var{percentile}=80. Either the sizes of a 
  rectangular kernel or the footprint of the kernel must be provided. The 
  \var{size} parameter, if provided, must be a sequence of sizes or a 
  single number in which case the size of the filter is assumed to be equal 
  along each axis. The \var{footprint}, if provided, must be an array that 
  defines the shape of the kernel by its non-zero elements.
\end{funcdesc}

\begin{funcdesc}{median_filter}{input, size=None, footprint=None, 
  output=None, mode='reflect', cval=0.0, origin=0} The 
  \function{median_filter} function calculates a multi-dimensional median 
  filter. Either the sizes of a rectangular kernel or the footprint of the 
  kernel must be provided. The \var{size} parameter, if provided, must be a 
  sequence of sizes or a single number in which case the size of the filter 
  is assumed to be equal along each axis. The \var{footprint} if provided, 
  must be an array that defines the shape of the kernel by its non-zero 
  elements.
\end{funcdesc}

\subsection{Derivatives}

Derivative filters can be constructed in several ways. The function
\function{gaussian_filter1d} described in section
\ref{sec:ndimage:filter-functions:smoothing} can be used to calculate
derivatives along a given axis using the \var{order} parameter. Other
derivative filters are the Prewitt and Sobel filters:

\begin{funcdesc}{prewitt}{input, axis=-1, output=None, mode='reflect', 
  cval=0.0} The \function{prewitt} function calculates a derivative along 
  the given axis.
\end{funcdesc}

\begin{funcdesc}{sobel}{input, axis=-1, output=None, mode='reflect', 
  cval=0.0} The \function{sobel} function calculates a derivative along 
  the given axis.
\end{funcdesc}

The Laplace filter is calculated by the sum of the second derivatives along 
all axes. Thus, different Laplace filters can be constructed using 
different second derivative functions. Therefore we provide a general 
function that takes a function argument to calculate the second derivative 
along a given direction and to construct the Laplace filter:

\begin{funcdesc}{generic_laplace}{input, derivative2, output=None,
  mode='reflect', cval=0.0, output_type=None, extra_arguments = (), 
  extra_keywords = {}} The function 
  \function{generic_laplace} calculates a laplace filter using the
  function passed through \var{derivative2} to calculate second 
  derivatives. The function \function{derivative2} should have the 
  following signature:

  \function{derivative2(input, axis, output, mode, cval, *extra_arguments, **extra_keywords)}
  
  It should calculate the second derivative along the dimension \var{axis}. 
  If \var{output} is not \constant{None} it should use that for the output 
  and return \constant{None}, otherwise it should return the result. 
  \var{mode}, \var{cval} have the usual meaning.
  
  The \var{extra_arguments} and \var{extra_keywords} arguments can be used 
  to pass a tuple of extra arguments and a dictionary of named 
  arguments that are passed to \function{derivative2} at each call.

  For example:
\begin{verbatim}
>>> def d2(input, axis, output, mode, cval):
...     return correlate1d(input, [1, -2, 1], axis, output, mode, cval, 0)
... 
>>> a = zeros((5, 5))
>>> a[2, 2] = 1
>>> print generic_laplace(a, d2)
[[ 0  0  0  0  0]
 [ 0  0  1  0  0]
 [ 0  1 -4  1  0]
 [ 0  0  1  0  0]
 [ 0  0  0  0  0]]
\end{verbatim}
To demonstrate the use of the \var{extra_arguments} argument we could do:
\begin{verbatim}
>>> def d2(input, axis, output, mode, cval, weights):
...     return correlate1d(input, weights, axis, output, mode, cval, 0,)
... 
>>> a = zeros((5, 5))
>>> a[2, 2] = 1
>>> print generic_laplace(a, d2, extra_arguments = ([1, -2, 1],))
[[ 0  0  0  0  0]
 [ 0  0  1  0  0]
 [ 0  1 -4  1  0]
 [ 0  0  1  0  0]
 [ 0  0  0  0  0]]
\end{verbatim}
or:
\begin{verbatim}
>>> print generic_laplace(a, d2, extra_keywords = {'weights': [1, -2, 1]})
[[ 0  0  0  0  0]
 [ 0  0  1  0  0]
 [ 0  1 -4  1  0]
 [ 0  0  1  0  0]
 [ 0  0  0  0  0]]
\end{verbatim}
\end{funcdesc}

The following two functions are implemented using 
\function{generic_laplace} by providing appropriate functions for the 
second derivative function:

\begin{funcdesc}{laplace}{input, output=None, mode='reflect', 
  cval=0.0, output_type=None} 
  The function \function{laplace} calculates 
  the Laplace using discrete differentiation for the second derivative 
  (i.e. convolution with \constant{[1, -2, 1]}).
\end{funcdesc}

\begin{funcdesc}{gaussian_laplace}{input, sigma, output=None, 
  mode='reflect', cval=0.0, output_type=None} The function 
  \function{gaussian_laplace} calculates the Laplace using 
  \function{gaussian_filter} to calculate the
  second derivatives. The standard-deviations of the Gaussian filter along 
  each axis are passed through the parameter \var{sigma} as a sequence or 
  numbers.  If \var{sigma} is not a sequence but a single number, the 
  standard deviation of the filter is equal along all directions.
  \end{funcdesc}

The gradient magnitude is defined as the square root of the sum of the 
squares of the gradients in all directions. Similar to the generic Laplace 
function there is a \function{generic_gradient_magnitude} function that 
calculated the gradient magnitude of an array:

\begin{funcdesc}{generic_gradient_magnitude}{input, derivative,
  output=None, mode='reflect', cval=0.0, output_type=None, 
  extra_arguments = (), extra_keywords = {}} The 
  function \function{generic_gradient_magnitude} calculates a gradient 
  magnitude using the function passed through \var{derivative} to calculate 
  first derivatives. The function \function{derivative} should have the 
  following signature:

  \function{derivative(input, axis, output, mode, cval, *extra_arguments, **extra_keywords)}
  
  It should calculate the derivative along the dimension \var{axis}. If
  \var{output} is not \constant{None} it should use that for the output and
  return \constant{None}, otherwise it should return the result. 
  \var{mode}, \var{cval} have the usual meaning.
  
  The \var{extra_arguments} and \var{extra_keywords} arguments can be used 
  to pass a tuple of extra arguments and a dictionary of named 
  arguments that are passed to \function{derivative} at each call.

  For example, the \function{sobel} function fits the required signature:
\begin{verbatim}
>>> a = zeros((5, 5))
>>> a[2, 2] = 1
>>> print generic_gradient_magnitude(a, sobel)
[[0 0 0 0 0]
 [0 1 2 1 0]
 [0 2 0 2 0]
 [0 1 2 1 0]
 [0 0 0 0 0]]
\end{verbatim}
See the documentation of \function{generic_laplace} for examples of using the \var{extra_arguments} and \var{extra_keywords} arguments.
\end{funcdesc}

The \function{sobel} and \function{prewitt} functions fit the required
signature and can therefore directly be used with
\function{generic_gradient_magnitude}. The following function implements 
the gradient magnitude using Gaussian derivatives:

\begin{funcdesc}{gaussian_gradient_magnitude}{input, sigma, output=None, 
  mode='reflect', cval=0.0, output_type=None} The function
  \function{gaussian_gradient_magnitude} calculates the gradient magnitude
  using \function{gaussian_filter} to calculate the first derivatives. The
  standard-deviations of the Gaussian filter along each axis are passed 
  through the parameter \var{sigma} as a sequence or numbers.  If 
  \var{sigma} is not a sequence but a single number, the standard deviation 
  of the filter is equal along all directions.
\end{funcdesc}

\subsection{Generic filter functions}
\label{sec:ndimage:genericfilters}
To implement filter functions, generic functions can be used that accept a 
callable object that implements the filtering operation. The iteration over 
the input and output arrays is handled by these generic functions, along 
with such details as the implementation of the boundary conditions. Only a 
callable object implementing a callback function that does the actual 
filtering work must be provided. The callback function can also be written 
in C and passed using a CObject (see \ref{sec:ndimage:ccallbacks} for more 
information).

\begin{funcdesc}{generic_filter1d}{input, function, filter_size, axis=-1,
  output=None, mode="reflect", cval=0.0, origin=0, output_type=None,
  extra_arguments = (), extra_keywords = {}}
  The \function{generic_filter1d} function implements a generic 
  one-dimensional filter function, where the actual filtering operation 
  must be supplied as a python function (or other callable object). The 
  \function{generic_filter1d} function iterates over the lines of an array 
  and calls \var{function} at each line. The arguments that are passed to 
  \var{function} are one-dimensional arrays of the \constant{tFloat64} 
  type. The first contains the values of the current line. It is extended 
  at the beginning end the end, according to the \var{filter_size} and 
  \var{origin} arguments. The second array should be modified in-place to 
  provide the output values of the line. For example 
  consider a correlation along one dimension:

\begin{verbatim}
>>> a = arange(12, shape = (3,4))
>>> print correlate1d(a, [1, 2, 3])
[[ 3  8 14 17]
 [27 32 38 41]
 [51 56 62 65]]
\end{verbatim}
The same operation can be implemented using \function{generic_filter1d} as 
follows:
\begin{verbatim} 
>>> def fnc(iline, oline):
...     oline[...] = iline[:-2] + 2 * iline[1:-1] + 3 * iline[2:]
... 
>>> print generic_filter1d(a, fnc, 3)
[[ 3  8 14 17]
 [27 32 38 41]
 [51 56 62 65]]
\end{verbatim}
  Here the origin of the kernel was (by default) assumed to be in the 
  middle of the filter of length 3. Therefore, each input line was
  extended by one value at the beginning and at the end, before the 
  function was called.
  
  Optionally extra arguments can be defined and passed to the filter 
  function. The \var{extra_arguments} and \var{extra_keywords} arguments 
  can be used to pass a tuple of extra arguments and/or a dictionary of 
  named arguments that are passed to derivative at each call. For example, 
  we can pass the parameters of our filter as an argument:
\begin{verbatim} 
>>> def fnc(iline, oline, a, b):
...     oline[...] = iline[:-2] + a * iline[1:-1] + b * iline[2:]
... 
>>> print generic_filter1d(a, fnc, 3, extra_arguments = (2, 3))
[[ 3  8 14 17]
 [27 32 38 41]
 [51 56 62 65]]
\end{verbatim}
or
\begin{verbatim} 
>>> print generic_filter1d(a, fnc, 3, extra_keywords = {'a':2, 'b':3})
[[ 3  8 14 17]
 [27 32 38 41]
 [51 56 62 65]]
\end{verbatim}
\end{funcdesc}

\begin{funcdesc}{generic_filter}{input, function, size=None,
  footprint=None, output=None, mode='reflect', cval=0.0, origin=0, 
  output_type=None, extra_arguments = (), extra_keywords = {}}
  The \function{generic_filter} function implements a generic filter  
  function,  where the actual filtering operation must be supplied as a 
  python function (or other callable object). The \function{generic_filter} 
  function iterates over the array and calls \var{function} at each 
  element. The argument of \var{function} is a one-dimensional array of the 
  \constant{tFloat64} type, that contains the values around the current
  element that are within the footprint of the filter. The function should 
  return a single value that can be converted to a double precision 
  number. For example consider a correlation:

\begin{verbatim}
>>> a = arange(12, shape = (3,4))
>>> print correlate(a, [[1, 0], [0, 3]])
[[ 0  3  7 11]
 [12 15 19 23]
 [28 31 35 39]]
\end{verbatim}
The same operation can be implemented using \function{generic_filter} as 
follows:
\begin{verbatim} 
>>> def fnc(buffer): 
...     return (buffer * array([1, 3])).sum()
... 
>>> print generic_filter(a, fnc, footprint = [[1, 0], [0, 1]])
[[ 0  3  7 11]
 [12 15 19 23]
 [28 31 35 39]]
\end{verbatim}
  Here a kernel footprint was specified that contains only two elements.
  Therefore the filter function receives a buffer of length equal to two,
  which was multiplied with the proper weights and the result summed.

  When calling \function{generic_filter}, either the sizes of a rectangular 
  kernel or the footprint of the kernel must be provided. The \var{size} 
  parameter, if provided, must be a sequence of sizes or a single number in 
  which case the size of the filter is assumed to be equal along each axis. 
  The \var{footprint}, if provided, must be an array that defines the shape 
  of the kernel by its non-zero elements.

  Optionally extra arguments can be defined and passed to the filter 
  function. The \var{extra_arguments} and \var{extra_keywords} arguments 
  can be used to pass a tuple of extra arguments and/or a dictionary of 
  named arguments that are passed to derivative at each call. For example, 
  we can pass the parameters of our filter as an argument:
\begin{verbatim} 
>>> def fnc(buffer, weights): 
...     weights = asarray(weights)
...     return (buffer * weights).sum()
... 
>>> print generic_filter(a, fnc, footprint = [[1, 0], [0, 1]], extra_arguments = ([1, 3],))
[[ 0  3  7 11]
 [12 15 19 23]
 [28 31 35 39]]
\end{verbatim}
or
\begin{verbatim} 
>>> print generic_filter(a, fnc, footprint = [[1, 0], [0, 1]], extra_keywords= {'weights': [1, 3]})
[[ 0  3  7 11]
 [12 15 19 23]
 [28 31 35 39]]
\end{verbatim}
\end{funcdesc}

These functions iterate over the lines or elements starting at the 
last axis, i.e. the last index changest the fastest. This order of iteration 
is garantueed for the case that it is important to adapt the filter 
dependening on spatial location. Here is an example of using a class that 
implements the filter and keeps track of the current coordinates while 
iterating. It performs the same filter operation as described above for 
\function{generic_filter}, but additionally prints the current coordinates:
\begin{verbatim}
>>> a = arange(12, shape = (3,4))
>>> 
>>> class fnc_class:
...     def __init__(self, shape):
...         # store the shape:
...         self.shape = shape
...         # initialize the coordinates:
...         self.coordinates = [0] * len(shape)
...         
...     def filter(self, buffer):
...         result = (buffer * array([1, 3])).sum()
...         print self.coordinates
...         # calculate the next coordinates:
...         axes = range(len(self.shape))
...         axes.reverse()
...         for jj in axes:
...             if self.coordinates[jj] < self.shape[jj] - 1:
...                 self.coordinates[jj] += 1
...                 break
...             else:
...                 self.coordinates[jj] = 0
...         return result
... 
>>> fnc = fnc_class(shape = (3,4))
>>> print generic_filter(a, fnc.filter, footprint = [[1, 0], [0, 1]]) 
[0, 0]
[0, 1]
[0, 2]
[0, 3]
[1, 0]
[1, 1]
[1, 2]
[1, 3]
[2, 0]
[2, 1]
[2, 2]
[2, 3]
[[ 0  3  7 11]
 [12 15 19 23]
 [28 31 35 39]]
\end{verbatim}

For the \function{generic_filter1d} function the same approach works, except that this function does not iterate over the axis that is being filtered. The example for \function{generic_filte1d} then becomes this:
\begin{verbatim}
>>> a = arange(12, shape = (3,4))
>>> 
>>> class fnc1d_class:
...     def __init__(self, shape, axis = -1):
...         # store the filter axis:
...         self.axis = axis
...         # store the shape:
...         self.shape = shape
...         # initialize the coordinates:
...         self.coordinates = [0] * len(shape)
...         
...     def filter(self, iline, oline):
...         oline[...] = iline[:-2] + 2 * iline[1:-1] + 3 * iline[2:]
...         print self.coordinates
...         # calculate the next coordinates:
...         axes = range(len(self.shape))
...         # skip the filter axis:
...         del axes[self.axis]
...         axes.reverse()
...         for jj in axes:
...             if self.coordinates[jj] < self.shape[jj] - 1:
...                 self.coordinates[jj] += 1
...                 break
...             else:
...                 self.coordinates[jj] = 0
... 
>>> fnc = fnc1d_class(shape = (3,4))
>>> print generic_filter1d(a, fnc.filter, 3)
[0, 0]
[1, 0]
[2, 0]
[[ 3  8 14 17]
 [27 32 38 41]
 [51 56 62 65]]
\end{verbatim}

\section{Fourier domain filters}
The functions described in this section perform filtering operations in the
Fourier domain. Thus, the input array of such a function should be 
compatible with an inverse Fourier transform function, such as the 
functions from the \module{numarray.fft} module. We therefore have to deal 
with arrays that may be the result of a real or a complex Fourier 
transform. In the case of a real Fourier transform only half of the of the 
symmetric complex transform is stored. Additionally, it needs to be known 
what the length of the axis was that was transformed by the real fft.  The 
functions described here provide a parameter \var{n} that in the case of a 
real transform must be equal to the length of the real transform axis 
before transformation. If this parameter is less than zero, it is assumed 
that the input array was the result of a complex Fourier transform. The 
parameter \var{axis} can be used to indicate along which axis the real 
transform was executed.

\begin{funcdesc}{fourier_shift}{input, shift, n=-1, axis=-1, output=None}
  The \function{fourier_shift} function multiplies the input array with the
  multi-dimensional Fourier transform of a shift operation for the given 
  shift. The \var{shift} parameter is a sequences of shifts for each 
  dimension, or a single value for all dimensions.
\end{funcdesc}

\begin{funcdesc}{fourier_gaussian}{input, sigma, n=-1, axis=-1, output=None}
  The \function{fourier_gaussian} function multiplies the input array with 
  the multi-dimensional Fourier transform of a Gaussian filter with given
  standard-deviations \var{sigma}. The \var{sigma} parameter is a sequences 
  of values for each dimension, or a single value for all dimensions.
\end{funcdesc}

\begin{funcdesc}{fourier_uniform}{input, size, n=-1, axis=-1, output=None}
  The \function{fourier_uniform} function multiplies the input array with 
  the multi-dimensional Fourier transform of a uniform filter with given
  sizes \var{size}. The \var{size} parameter is a sequences of
  values for each dimension, or a single value for all dimensions.
\end{funcdesc}

\begin{funcdesc}{fourier_ellipsoid}{input, size, n=-1, axis=-1, 
  output=None}
  The \function{fourier_ellipsoid} function multiplies the input array with 
  the multi-dimensional Fourier transform of a elliptically shaped filter 
  with given sizes \var{size}. The \var{size} parameter is a sequences of 
  values for each dimension, or a single value for all dimensions.  
  \note{This function is
    only implemented for dimensions 1, 2, and 3.}
\end{funcdesc}

\section{Interpolation functions}
This section describes various interpolation functions that are based on
B-spline theory. A good introduction to B-splines can be found in: M. 
Unser, "Splines: A Perfect Fit for Signal and Image Processing," IEEE 
Signal Processing Magazine, vol. 16, no. 6, pp. 22-38, November 1999.
\subsection{Spline pre-filters}
Interpolation using splines of an order larger than 1 requires a pre-
filtering step. The interpolation functions described in section
\ref{sec:ndimage:interpolation} apply pre-filtering by calling
\function{spline_filter}, but they can be instructed not to do this by 
setting the \var{prefilter} keyword equal to \constant{False}.  This is 
useful if more than one interpolation operation is done on the same array. 
In this case it is more efficient to do the pre-filtering only once and use 
a prefiltered array as the input of the interpolation functions. The 
following two functions implement the pre-filtering:

\begin{funcdesc}{spline_filter1d}{input, order=3, axis=-1, output=None,
    output_type=numarray.Float64} The \function{spline_filter1d} function
  calculates a one-dimensional spline filter along the given axis. An 
  output array can optionally be provided. The order of the spline must be 
  larger then 1 and less than 6.
\end{funcdesc}

\begin{funcdesc}{spline_filter}{input, order=3, output=None, 
    output_type=numarray.Float64} The \function{spline_filter} function
  calculates a multi-dimensional spline filter.
  
  \note{The multi-dimensional filter is implemented as a sequence of
    one-dimensional spline filters. The intermediate arrays are stored in 
    the same data type as the output. Therefore, if an output 
    with a limited precision is requested, the results may be imprecise 
    because intermediate results may be stored with insufficient precision. 
    This can be prevented by specifying a output type of high precision.}
\end{funcdesc}

\subsection{Interpolation functions}
\label{sec:ndimage:interpolation}
Following functions all employ spline interpolation to effect some type of
geometric transformation of the input array. This requires a mapping of the
output coordinates to the input coordinates, and therefore the possibility
arises that input values outside the boundaries are needed. This problem 
is solved in the same way as described in section
\ref{sec:ndimage:filter-functions} for the multi-dimensional filter 
functions. Therefore these functions all support a \var{mode} parameter 
that determines how the boundaries are handled, and a \var{cval} parameter 
that gives a constant value in case that the \constant{'constant'} mode is 
used.

\begin{funcdesc}{geometric_transform}{input, mapping, output_shape=None,
    output_type=None, output=None, order=3, mode='constant', cval=0.0,
    prefilter=True, extra_arguments = (), extra_keywords = {}} The \function{geometric_transform} function applies an
  arbitrary geometric transform to the input. The given \var{mapping} 
  function is called at each point in the output to find the corresponding 
  coordinates in the input.  \var{mapping} must be a callable object that 
  accepts a tuple of length equal to the output array rank and returns the 
  corresponding input coordinates as a tuple of length equal to the input 
  array rank. The output shape and output type can optionally be provided. 
  If not given they are equal to the input shape and type.
  
  For example:
\begin{verbatim}
>>> a = arange(12, shape=(4,3), type = Float64)
>>> def shift_func(output_coordinates):
...     return (output_coordinates[0] - 0.5, output_coordinates[1] - 0.5)
... 
>>> print geometric_transform(a, shift_func)
[[ 0.      0.      0.    ]
 [ 0.      1.3625  2.7375]
 [ 0.      4.8125  6.1875]
 [ 0.      8.2625  9.6375]]  
\end{verbatim}  

  Optionally extra arguments can be defined and passed to the filter 
  function. The \var{extra_arguments} and \var{extra_keywords} arguments 
  can be used to pass a tuple of extra arguments and/or a dictionary of 
  named arguments that are passed to derivative at each call. For example, 
  we can pass the shifts in our example as arguments:

\begin{verbatim}
>>> def shift_func(output_coordinates, s0, s1):
...     return (output_coordinates[0] - s0, output_coordinates[1] - s1)
... 
>>> print geometric_transform(a, shift_func, extra_arguments = (0.5, 0.5))
[[ 0.      0.      0.    ]
 [ 0.      1.3625  2.7375]
 [ 0.      4.8125  6.1875]
 [ 0.      8.2625  9.6375]]  
\end{verbatim}  
or
\begin{verbatim}
>>> print geometric_transform(a, shift_func, extra_keywords = {'s0': 0.5, 's1': 0.5})
[[ 0.      0.      0.    ]
 [ 0.      1.3625  2.7375]
 [ 0.      4.8125  6.1875]
 [ 0.      8.2625  9.6375]]  
\end{verbatim}  

\note{The mapping function can also be written in C and passed using a CObject. See \ref{sec:ndimage:ccallbacks} for more information.}
\end{funcdesc}

\begin{funcdesc}{map_coordinates}{input, coordinates, output_type=None, 
    output=None, order=3, mode='constant', cval=0.0, prefilter=True} 
  The function \function{map_coordinates} applies an arbitrary coordinate
  transformation using the given array of coordinates. The shape of the 
  output is derived from that of the coordinate array by dropping the first 
  axis. The parameter \var{coordinates} is used to find for each point in 
  the output the corresponding coordinates in the input. The values of 
  \var{coordinates} along the first axis are the coordinates in the input 
  array at which the output value is found. (See also the numarray 
  \function{coordinates} function.) Since the coordinates may be non-
  integer coordinates, the value of the input at these coordinates is 
  determined by spline interpolation of the requested order. Here is an 
  example that interpolates a 2D array at (0.5, 0.5) and (1, 2):
\begin{verbatim}
>>> a = arange(12, shape=(4,3), type = numarray.Float64)
>>> print a
[[  0.   1.   2.]
 [  3.   4.   5.]
 [  6.   7.   8.]
 [  9.  10.  11.]]
>>> print map_coordinates(a, [[0.5, 2], [0.5, 1]])
[ 1.3625  7.    ]
\end{verbatim}
\end{funcdesc}

\begin{funcdesc}{affine_transform}{input, matrix, offset=0.0, 
  output_shape=None, output_type=None, output=None, order=3, 
  mode='constant', cval=0.0, prefilter=True} The 
  \function{affine_transform} function applies an affine transformation to 
  the input array. The given transformation \var{matrix} and \var{offset} 
  are used to find for each point in the output the corresponding 
  coordinates in the input.  The value of the input at the
  calculated coordinates is determined by spline interpolation of the 
  requested order. The transformation \var{matrix} must be two-dimensional 
  or can also be given as a one-dimensional sequence or array.  In the 
  latter case, it is assumed that the matrix is diagonal. A more efficient 
  interpolation algorithm is then applied that exploits the separability of 
  the problem.  The output shape and output type can optionally be 
  provided. If not given they are equal to the input shape and type.
\end{funcdesc}

\begin{funcdesc}{shift}{input, shift, output_type=None, output=None, 
  order=3, mode='constant', cval=0.0, prefilter=True} The \function{shift} 
  function returns a shifted version of the input, using spline 
  interpolation of the requested \var{order}.
\end{funcdesc}

\begin{funcdesc}{zoom}{input, zoom, output_type=None, output=None, order=3, 
    mode='constant', cval=0.0, prefilter=True} The \function{zoom} function
  returns a rescaled version of the input, using spline interpolation of 
  the requested \var{order}.
\end{funcdesc}

\begin{funcdesc}{rotate}{input, angle, axes=(-1, -2), reshape=1,
    output_type=None, output=None, order=3, mode='constant', cval=0.0,
    prefilter=True} The \function{rotate} function returns the input array
  rotated in the plane defined by the two axes given by the parameter
  \var{axes}, using spline interpolation of the requested \var{order}. The
  angle must be given in degrees. If \var{reshape} is true, then the size 
  of the output array is adapted to contain the rotated input.
\end{funcdesc}

\section{Binary morphology}
\label{sec:ndimage:binary-morphology}

\begin{funcdesc}{generate_binary_structure}{rank, connectivity}
  The \function{generate_binary_structure} functions generates a binary
  structuring element for use in binary morphology operations. The 
  \var{rank} of the structure must be provided. The size of the structure 
  that is returned is equal to three in each direction. The value of each 
  element is equal to one if the square of the Euclidean distance from the 
  element to the center is less or equal to \var{connectivity}. For 
  instance, two dimensional 4-connected and 8-connected structures are 
  generated as follows:
\begin{verbatim}
>>> print generate_binary_structure(2, 1)
[[0 1 0]
 [1 1 1]
 [0 1 0]]
>>> print generate_binary_structure(2, 2)
[[1 1 1]
 [1 1 1]
 [1 1 1]]
\end{verbatim}
\end{funcdesc}

Most binary morphology functions can be expressed in terms of the basic
operations erosion and dilation:

\begin{funcdesc}{binary_erosion}{input, structure=None, iterations=1,
    mask=None, output=None, border_value=0, origin=0} The
  \function{binary_erosion} function implements binary erosion of arrays of
  arbitrary rank with the given structuring element. The origin parameter
  controls the placement of the structuring element as described in section
  \ref{sec:ndimage:filter-functions}. If no structuring element is 
  provided, an element with connectivity equal to one is generated using
  \function{generate_binary_structure}. The \var{border_value} parameter 
  gives the value of the array outside boundaries. The erosion is repeated
  \var{iterations} times. If \var{iterations} is less than one, the erosion 
  is repeated until the result does not change anymore. If a \var{mask} 
  array is given, only those elements with a true value at the 
  corresponding mask element are modified at each iteration.
\end{funcdesc}

\begin{funcdesc}{binary_dilation}{input, structure=None, iterations=1,
    mask=None, output=None, border_value=0, origin=0} The
  \function{binary_dilation} function implements binary dilation of arrays 
  of arbitrary rank with the given structuring element. The origin 
  parameter controls the placement of the structuring element as described 
  in section \ref{sec:ndimage:filter-functions}. If no structuring element 
  is provided, an element with connectivity equal to one is generated using
  \function{generate_binary_structure}. The \var{border_value} parameter 
  gives the value of the array outside boundaries. The dilation is repeated
  \var{iterations} times.  If \var{iterations} is less than one, the 
  dilation is repeated until the result does not change anymore. If a 
  \var{mask} array is given, only those elements with a true value at the 
  corresponding mask element are modified at each iteration.

  Here is an example of using \function{binary_dilation} to find all 
  elements that touch the border, by repeatedly dilating an empty array 
  from the border using the data array as the mask:
\begin{verbatim}
>>> struct = array([[0, 1, 0], [1, 1, 1], [0, 1, 0]])
>>> a = array([[1,0,0,0,0], [1,1,0,1,0], [0,0,1,1,0], [0,0,0,0,0]])
>>> print a
[[1 0 0 0 0]
 [1 1 0 1 0]
 [0 0 1 1 0]
 [0 0 0 0 0]]
>>> print binary_dilation(zeros(a.shape), struct, -1, a, border_value=1)
[[1 0 0 0 0]
 [1 1 0 0 0]
 [0 0 0 0 0]
 [0 0 0 0 0]]
\end{verbatim}
\end{funcdesc}

The \function{binary_erosion} and \function{binary_dilation} functions both
have an \var{iterations} parameter which allows the erosion or dilation to 
be repeated a number of times. Repeating an erosion or a dilation with a 
given structure \constant{n} times is equivalent to an erosion or a 
dilation with a structure that is \constant{n-1} times dilated with itself. 
A function is provided that allows the calculation of a structure that is 
dilated a number of times with itself:

\begin{funcdesc}{iterate_structure}{structure, iterations, origin=None} 
  The \function{iterate_structure} function returns a structure by dilation 
  of the input structure \var{iteration} - 1 times with itself. For 
  instance:
  \begin{verbatim}
>>> struct = generate_binary_structure(2, 1)
>>> print struct
[[0 1 0]
 [1 1 1]
 [0 1 0]]
>>> print iterate_structure(struct, 2)
[[0 0 1 0 0]
 [0 1 1 1 0]
 [1 1 1 1 1]
 [0 1 1 1 0]
 [0 0 1 0 0]]
\end{verbatim}
  If the origin of the original structure is equal to 0, then it is also 
  equal to 0 for the iterated structure. If not, the origin must also be 
  adapted if the equivalent of the \var{iterations} erosions or dilations 
  must be achieved with the iterated structure. The adapted origin is 
  simply obtained by multiplying with the number of iterations. For 
  convenience the
  \function{iterate_structure} also returns the adapted origin if the
  \var{origin} parameter is not \constant{None}:
\begin{verbatim}
>>> print iterate_structure(struct, 2, -1)
(array([[0, 0, 1, 0, 0],
       [0, 1, 1, 1, 0],
       [1, 1, 1, 1, 1],
       [0, 1, 1, 1, 0],
       [0, 0, 1, 0, 0]], type=Bool), [-2, -2])
\end{verbatim}
\end{funcdesc}

Other morphology operations can be defined in terms of erosion and d
dilation. Following functions provide a few of these operations for 
convenience:

\begin{funcdesc}{binary_opening}{input, structure=None, iterations=1,
  output=None, origin=0} The \function{binary_opening} function implements
  binary opening of arrays of arbitrary rank with the given structuring
  element. Binary opening is equivalent to a binary erosion followed by a
  binary dilation with the same structuring element. The origin parameter
  controls the placement of the structuring element as described in section
  \ref{sec:ndimage:filter-functions}. If no structuring element is 
  provided, an element with connectivity equal to one is generated using
  \function{generate_binary_structure}. The \var{iterations} parameter 
  gives the number of erosions that is performed followed by the same 
  number of dilations.
\end{funcdesc}

\begin{funcdesc}{binary_closing}{input, structure=None, iterations=1,
  output=None, origin=0} The \function{binary_closing} function implements
  binary closing of arrays of arbitrary rank with the given structuring
  element. Binary closing is equivalent to a binary dilation followed by a
  binary erosion with the same structuring element. The origin parameter
  controls the placement of the structuring element as described in section
  \ref{sec:ndimage:filter-functions}. If no structuring element is 
  provided, an element with connectivity equal to one is generated using
  \function{generate_binary_structure}. The \var{iterations} parameter   
  gives the number of dilations that is performed followed by the same 
  number of erosions.
\end{funcdesc}

\begin{funcdesc}{binary_fill_holes}{input, structure = None, output = None, 
origin = 0} The \function{binary_fill_holes} function is used to close 
holes in objects in a binary image, where the structure defines the 
connectivity of the holes. The origin parameter controls the placement of 
the structuring element as described in section \ref{sec:ndimage:filter-
functions}. If no structuring element is provided, an element with 
connectivity equal to one is generated using 
\function{generate_binary_structure}. 
\end{funcdesc}

\begin{funcdesc}{binary_hit_or_miss}{input, structure1=None, 
  structure2=None, output=None, origin1=0, origin2=None} The 
  \function{binary_hit_or_miss}
  function implements a binary hit-or-miss transform of arrays of arbitrary
  rank with the given structuring elements.  The hit-or-miss transform is
  calculated by erosion of the input with the first structure, erosion of   
  the logical \emph{not} of the input with the second structure, followed 
  by the logical \emph{and} of these two erosions.  The origin parameters 
  control the placement of the structuring elements as described in section
  \ref{sec:ndimage:filter-functions}. If \var{origin2} equals 
  \constant{None} it is set equal to the \var{origin1} parameter. If the 
  first structuring element is not provided, a structuring element with 
  connectivity equal to one is generated using 
  \function{generate_binary_structure}, if \var{structure2} is not 
  provided, it is set equal to the logical \emph{not} of \var{structure1}.
\end{funcdesc}

\section{Grey-scale morphology}
\label{sec:ndimage:grey-morphology}

Grey-scale morphology operations are the equivalents of binary morphology
operations that operate on arrays with arbitrary values. Below we describe 
the grey-scale equivalents of erosion, dilation, opening and closing. These
operations are implemented in a similar fashion as the filters described in
section \ref{sec:ndimage:filter-functions}, and we refer to this section 
for the description of filter kernels and footprints, and the handling of 
array borders. The grey-scale morphology operations optionally take a 
\var{structure} parameter that gives the values of the structuring element. 
If this parameter is not given the structuring element is assumed to be 
flat with a value equal to zero. The shape of the structure can optionally 
be defined by the \var{footprint} parameter. If this parameter is not 
given, the structure is assumed to be rectangular, with sizes equal to the 
dimensions of the \var{structure} array, or by the \var{size} parameter if 
\var{structure} is not given. The \var{size} parameter is only used if both 
\var{structure} and \var{footprint} are not given, in which case the 
structuring element is assumed to be rectangular and flat with the 
dimensions given by \var{size}. The \var{size} parameter, if provided, must 
be a sequence of sizes or a single number in which case the size of the 
filter is assumed to be equal along each axis. The \var{footprint} 
parameter, if provided, must be an array that defines the shape of the 
kernel by its non-zero elements.

Similar to binary erosion and dilation there are operations for grey-scale
erosion and dilation:

\begin{funcdesc}{grey_erosion}{input, size=None, footprint=None, 
    structure=None, output=None, mode='reflect', cval=0.0, origin=0} The
  \function{grey_erosion} function calculates a multi-dimensional grey-
  scale erosion.
\end{funcdesc}

\begin{funcdesc}{grey_dilation}{input, size=None, footprint=None, 
    structure=None, output=None, mode='reflect', cval=0.0, origin=0} The
  \function{grey_dilation} function calculates a multi-dimensional grey-
  scale dilation.
\end{funcdesc}

Grey-scale opening and closing operations can be defined similar to their
binary counterparts:

\begin{funcdesc}{grey_opening}{input, size=None, footprint=None, 
    structure=None, output=None, mode='reflect', cval=0.0, origin=0} The
  \function{grey_opening} function implements grey-scale opening of arrays 
  of arbitrary rank. Grey-scale opening is equivalent to a grey-scale 
  erosion followed by a grey-scale dilation.
\end{funcdesc}

\begin{funcdesc}{grey_closing}{input, size=None, footprint=None, 
    structure=None, output=None, mode='reflect', cval=0.0, origin=0} The
  \function{grey_closing} function implements grey-scale closing of arrays 
  of arbitrary rank. Grey-scale opening is equivalent to a grey-scale 
  dilation followed by a grey-scale erosion.
\end{funcdesc}

\begin{funcdesc}{morphological_gradient}{input, size=None, footprint=None, 
    structure=None, output=None, mode='reflect', cval=0.0, origin=0} The
  \function{morphological_gradient} function implements a grey-scale
  morphological gradient of arrays of arbitrary rank. The grey-scale
  morphological gradient is equal to the difference of a grey-scale 
  dilation and a grey-scale erosion.
\end{funcdesc}

\begin{funcdesc}{morphological_laplace}{input, size=None, footprint=None, 
    structure=None, output=None, mode='reflect', cval=0.0, origin=0} The
  \function{morphological_laplace} function implements a grey-scale
  morphological laplace of arrays of arbitrary rank. The grey-scale
  morphological laplace is equal to the sum of a grey-scale dilation and a
  grey-scale erosion minus twice the input.
\end{funcdesc}

\begin{funcdesc}{white_tophat}{input, size=None, footprint=None, 
    structure=None, output=None, mode='reflect', cval=0.0, origin=0} The
  \function{white_tophat} function implements a white top-hat filter of 
  arrays of arbitrary rank. The white top-hat is equal to the difference of 
  the input and a grey-scale opening.
\end{funcdesc}

\begin{funcdesc}{black_tophat}{input, size=None, footprint=None, 
    structure=None, output=None, mode='reflect', cval=0.0, origin=0} The
  \function{black_tophat} function implements a black top-hat filter of 
  arrays of arbitrary rank. The black top-hat is equal to the difference of 
  the a grey-scale closing and the input.
\end{funcdesc}

\section{Distance transforms}
\label{sec:ndimage:grey-morphology}
Distance transforms are used to calculate the minimum distance from each
element of an object to the background. The following functions implement
distance transforms for three different distance metrics: Euclidean, City
Block, and Chessboard distances.

\begin{funcdesc}{distance_transform_cdt}{input, structure="chessboard",
  return_distances=True, return_indices=False, distances=None, 
  indices=None} The function \function{distance_transform_cdt} uses a 
  chamfer type algorithm to calculate the distance transform of the input, 
  by replacing each object element (defined by values larger than zero) 
  with the shortest distance to the background (all non-object elements). 
  The structure determines the type of chamfering that is done. If the 
  structure is equal to 'cityblock' a structure is generated using 
  \function{generate_binary_structure} with a squared distance equal to 1. 
  If the structure is equal to 'chessboard', a structure is generated using 
  \function{generate_binary_structure} with a squared distance equal to the 
  rank of the array. These choices correspond to the common interpretations 
  of the cityblock and the chessboard distancemetrics in two dimensions.
  
  In addition to the distance transform, the feature transform can be
  calculated. In this case the index of the closest background element is
  returned along the first axis of the result.  The \var{return_distances}, 
  and \var{return_indices} flags can be used to indicate if the distance 
  transform, the feature transform, or both must be returned.
  
  The \var{distances} and \var{indices} arguments can be used to give 
  optional output arrays that must be of the correct size and type (both
  \constant{Int32}).

  The basics of the algorithm used to implement this function is described
  in: G. Borgefors, "Distance transformations in arbitrary dimensions.",
  Computer Vision, Graphics, and Image Processing, 27:321--345, 1984.
\end{funcdesc}

\begin{funcdesc}{distance_transform_edt}{input, sampling=None,
  return_distances=True, return_indices=False, distances=None, 
  indices=None} The function \function{distance_transform_edt} calculates 
  the exact euclidean distance transform of the input, by replacing each 
  object element (defined by values larger than zero) with the shortest 
  euclidean distance to the background (all non-object elements).
  
  In addition to the distance transform, the feature transform can be
  calculated. In this case the index of the closest background element is
  returned along the first axis of the result.  The \var{return_distances}, 
  and \var{return_indices} flags can be used to indicate if the distance 
  transform, the feature transform, or both must be returned.
  
  Optionally the sampling along each axis can be given by the 
  \var{sampling} parameter which should be a sequence of length equal to 
  the input rank, or a single number in which the sampling is assumed to be 
  equal along all axes.

  The \var{distances} and \var{indices} arguments can be used to give 
  optional output arrays that must be of the correct size and type 
  (\constant{Float64} and \constant{Int32}).
  
  The algorithm used to implement this function is described in: C. R. 
  Maurer, Jr., R. Qi, and V. Raghavan, "A linear time algorithm for 
  computing exact euclidean distance transforms of binary images in 
  arbitrary dimensions. IEEE Trans. PAMI 25, 265-270, 2003.
\end{funcdesc}

\begin{funcdesc}{distance_transform_bf}{input, metric="euclidean",
  sampling=None, return_distances=True, return_indices=False, 
  distances=None, indices=None} The function  
  \function{distance_transform_bf} uses a brute-force algorithm to 
  calculate the distance transform of the input, by replacing each object 
  element (defined by values larger than zero) with the shortest distance 
  to the background (all non-object elements).  The metric must be one of 
  \constant{"euclidean"}, \constant{"cityblock"}, or 
  \constant{"chessboard"}.
  
  In addition to the distance transform, the feature transform can be
  calculated. In this case the index of the closest background element is
  returned along the first axis of the result.  The \var{return_distances}, 
  and \var{return_indices} flags can be used to indicate if the distance 
  transform, the feature transform, or both must be returned.
  
  Optionally the sampling along each axis can be given by the 
  \var{sampling} parameter which should be a sequence of length equal to 
  the input rank, or a single number in which the sampling is assumed to be 
  equal along all axes. This parameter is only used in the case of the 
  euclidean distance transform.

  The \var{distances} and \var{indices} arguments can be used to give 
  optional output arrays that must be of the correct size and type 
  (\constant{Float64} and \constant{Int32}).

  \note{This function uses a slow brute-force algorithm, the function
    \function{distance_transform_cdt} can be used to more efficiently 
    calculate cityblock and chessboard distance transforms. The function
    \function{distance_transform_edt} can be used to more efficiently 
    calculate the exact euclidean distance transform.}
\end{funcdesc}

\section{Segmentation and labeling}
Segmentation is the process of separating objects of interest from the
background. The most simple approach is probably intensity thresholding, 
which is easily done with \module{numarray} functions:
\begin{verbatim}
>>> a = array([[1,2,2,1,1,0],
...            [0,2,3,1,2,0],
...            [1,1,1,3,3,2],
...            [1,1,1,1,2,1]])
>>> print where(a > 1, 1, 0)
[[0 1 1 0 0 0]
 [0 1 1 0 1 0]
 [0 0 0 1 1 1]
 [0 0 0 0 1 0]]
\end{verbatim}

The result is a binary image, in which the individual objects still need to 
be identified and labeled.  The function \function{label} generates an 
array where each object is assigned a unique number:

\begin{funcdesc}{label}{input, structure=None, output=None}
  The \function{label} function generates an array where the objects in the
  input are labeled with an integer index. It returns a tuple consisting of 
  the array of object labels and the number of objects found, unless the
  \var{output} parameter is given, in which case only the number of objects 
  is returned. The connectivity of the objects is defined by a structuring
  element. For instance, in two dimensions using a four-connected 
  structuring element gives:
\begin{verbatim}
>>> a = array([[0,1,1,0,0,0],[0,1,1,0,1,0],[0,0,0,1,1,1],[0,0,0,0,1,0]])
>>> s = [[0, 1, 0], [1,1,1], [0,1,0]]
>>> print label(a, s)
(array([[0, 1, 1, 0, 0, 0],
       [0, 1, 1, 0, 2, 0],
       [0, 0, 0, 2, 2, 2],
       [0, 0, 0, 0, 2, 0]]), 2)
\end{verbatim}
These two objects are not connected because there is no way in which we can
place the structuring element such that it overlaps with both objects. 
However, an 8-connected structuring element results in only a single 
object:
\begin{verbatim}
>>> a = array([[0,1,1,0,0,0],[0,1,1,0,1,0],[0,0,0,1,1,1],[0,0,0,0,1,0]])
>>> s = [[1,1,1], [1,1,1], [1,1,1]]
>>> print label(a, s)[0]
[[0 1 1 0 0 0]
 [0 1 1 0 1 0]
 [0 0 0 1 1 1]
 [0 0 0 0 1 0]]
\end{verbatim}
If no structuring element is provided, one is generated by calling
\function{generate_binary_structure} (see section \ref{sec:ndimage:
morphology}) using a connectivity of one (which in 2D is the 4-connected 
structure of the first example).  The input can be of any type, any value 
not equal to zero is taken to be part of an object. This is useful if you 
need to 're-label' an array of object indices, for instance after removing 
unwanted objects. Just apply the label function again to the index array. 
For instance:
\begin{verbatim}
>>> l, n = label([1, 0, 1, 0, 1])
>>> print l
[1 0 2 0 3]
>>> l = where(l != 2, l, 0)
>>> print l
[1 0 0 0 3]
>>> print label(l)[0]
[1 0 0 0 2]
\end{verbatim}

\note{The structuring element used by \function{label} is assumed to be
  symmetric.}
\end{funcdesc}

There is a large number of other approaches for segmentation, for instance 
from an estimation of the borders of the objects that can be obtained for 
instance by derivative filters. One such an approach is watershed 
segmentation.  The function \function{watershed_ift} generates an array 
where each object is assigned a unique label, from an array that localizes 
the object borders, generated for instance by a gradient magnitude filter. 
It uses an array containing initial markers for the objects:
\begin{funcdesc}{watershed_ift}{input, markers, structure=None, 
  output=None} The \function{watershed_ift} function applies a watershed 
  from markers algorithm, using an Iterative Forest Transform, as described 
  in: P. Felkel, R.  Wegenkittl, and M. Bruckschwaiger, "Implementation and 
  Complexity of the Watershed-from-Markers Algorithm Computed as a Minimal 
  Cost Forest.", Eurographics 2001, pp. C:26-35.
  
  The inputs of this function are the array to which the transform is 
  applied, and an array of markers that designate the objects by a unique 
  label, where any non-zero value is a marker. For instance:
\begin{verbatim}
>>> input = array([[0, 0, 0, 0, 0, 0, 0],
...                [0, 1, 1, 1, 1, 1, 0],
...                [0, 1, 0, 0, 0, 1, 0],
...                [0, 1, 0, 0, 0, 1, 0],
...                [0, 1, 0, 0, 0, 1, 0],
...                [0, 1, 1, 1, 1, 1, 0],
...                [0, 0, 0, 0, 0, 0, 0]], numarray.UInt8)
>>> markers = array([[1, 0, 0, 0, 0, 0, 0],
...                  [0, 0, 0, 0, 0, 0, 0],
...                  [0, 0, 0, 0, 0, 0, 0],
...                  [0, 0, 0, 2, 0, 0, 0],
...                  [0, 0, 0, 0, 0, 0, 0],
...                  [0, 0, 0, 0, 0, 0, 0],
...                  [0, 0, 0, 0, 0, 0, 0]], numarray.Int8)
>>> print watershed_ift(input, markers)
[[1 1 1 1 1 1 1]
 [1 1 2 2 2 1 1]
 [1 2 2 2 2 2 1]
 [1 2 2 2 2 2 1]
 [1 2 2 2 2 2 1]
 [1 1 2 2 2 1 1]
 [1 1 1 1 1 1 1]]
\end{verbatim}
  
  Here two markers were used to designate an object (marker=2) and the
  background (marker=1).  The order in which these are processed is 
  arbitrary: moving the marker for the background to the lower right corner 
  of the array yields a different result:
\begin{verbatim}
>>> markers = array([[0, 0, 0, 0, 0, 0, 0],
...                  [0, 0, 0, 0, 0, 0, 0],
...                  [0, 0, 0, 0, 0, 0, 0],
...                  [0, 0, 0, 2, 0, 0, 0],
...                  [0, 0, 0, 0, 0, 0, 0],
...                  [0, 0, 0, 0, 0, 0, 0],
...                  [0, 0, 0, 0, 0, 0, 1]], numarray.Int8)
>>> print watershed_ift(input, markers)
[[1 1 1 1 1 1 1]
 [1 1 1 1 1 1 1]
 [1 1 2 2 2 1 1]
 [1 1 2 2 2 1 1]
 [1 1 2 2 2 1 1]
 [1 1 1 1 1 1 1]
 [1 1 1 1 1 1 1]]
\end{verbatim}
  The result is that the object (marker=2) is smaller because the second 
  marker was processed earlier. This may not be the desired effect if the 
  first marker was supposed to designate a background object. Therefore
  \function{watershed_ift} treats markers with a negative value explicitly 
  as background markers and processes them after the normal markers. For 
  instance, replacing the first marker by a negative marker gives a result 
  similar to the first example:
\begin{verbatim}
>>> markers = array([[0, 0, 0, 0, 0, 0, 0],
...                  [0, 0, 0, 0, 0, 0, 0],
...                  [0, 0, 0, 0, 0, 0, 0],
...                  [0, 0, 0, 2, 0, 0, 0],
...                  [0, 0, 0, 0, 0, 0, 0],
...                  [0, 0, 0, 0, 0, 0, 0],
...                  [0, 0, 0, 0, 0, 0, -1]], numarray.Int8)
>>> print watershed_ift(input, markers)
[[-1 -1 -1 -1 -1 -1 -1]
 [-1 -1  2  2  2 -1 -1]
 [-1  2  2  2  2  2 -1]
 [-1  2  2  2  2  2 -1]
 [-1  2  2  2  2  2 -1]
 [-1 -1  2  2  2 -1 -1]
 [-1 -1 -1 -1 -1 -1 -1]]
\end{verbatim}
  
  The connectivity of the objects is defined by a structuring element. If 
  no structuring element is provided, one is generated by calling
  \function{generate_binary_structure} (see section
  \ref{sec:ndimage:morphology}) using a connectivity of one (which in 2D is 
  a 4-connected structure.) For example, using an 8-connected structure 
  with the last example yields a different object:
\begin{verbatim}
>>> print watershed_ift(input, markers,
...                     structure = [[1,1,1], [1,1,1], [1,1,1]])
[[-1 -1 -1 -1 -1 -1 -1]
 [-1  2  2  2  2  2 -1]
 [-1  2  2  2  2  2 -1]
 [-1  2  2  2  2  2 -1]
 [-1  2  2  2  2  2 -1]
 [-1  2  2  2  2  2 -1]
 [-1 -1 -1 -1 -1 -1 -1]]
\end{verbatim}

\note{The implementation of \function{watershed_ift} limits the data types 
of the input to \constant{UInt8} and \constant{UInt16}.}
\end{funcdesc}

\section{Object measurements}
Given an array of labeled objects, the properties of the individual objects 
can be measured. The \function{find_objects} function can be used to 
generate a list of slices that for each object, give the smallest sub-array 
that fully contains the object:

\begin{funcdesc}{find_objects}{input, max_label=0}
  The \function{find_objects} finds all objects in a labeled array and 
  returns a list of slices that correspond to the smallest regions in the 
  array that contains the object. For instance:
\begin{verbatim}
>>> a = array([[0,1,1,0,0,0],[0,1,1,0,1,0],[0,0,0,1,1,1],[0,0,0,0,1,0]])
>>> l, n = label(a)
>>> f = find_objects(l)
>>> print a[f[0]]
[[1 1]
 [1 1]]
>>> print a[f[1]]
[[0 1 0]
 [1 1 1]
 [0 1 0]]
\end{verbatim}
\function{find_objects} returns slices for all objects, unless the
\var{max_label} parameter is larger then zero, in which case only the first
\var{max_label} objects are returned. If an index is missing in the 
\var{label} array, \constant{None} is return instead of a slice. For 
example:
\begin{verbatim}
>>> print find_objects([1, 0, 3, 4], max_label = 3)
[(slice(0, 1, None),), None, (slice(2, 3, None),)]
\end{verbatim}
\end{funcdesc}

The list of slices generated by \function{find_objects} is useful to find 
the position and dimensions of the objects in the array, but can also be 
used to perform measurements on the individual objects. Say we want to find 
the sum of the intensities of an object in image:
\begin{verbatim}
>>> image = arange(4*6,shape=(4,6))
>>> mask = array([[0,1,1,0,0,0],[0,1,1,0,1,0],[0,0,0,1,1,1],[0,0,0,0,1,0]])
>>> labels = label(mask)[0]
>>> slices = find_objects(labels)
\end{verbatim}
Then we can calculate the sum of the elements in the second object:
\begin{verbatim}
>>> print where(labels[slices[1]] == 2, image[slices[1]], 0).sum()
80
\end{verbatim}
That is however not particularly efficient, and may also be more 
complicated for other types of measurements. Therefore a few measurements 
functions are defined that accept the array of object labels and the index 
of the object to be measured. For instance calculating the sum of the 
intensities can be done by:
\begin{verbatim}
>>> print sum(image, labels, 2)
80.0
\end{verbatim}
For large arrays and small objects it is more efficient to call the 
measurement functions after slicing the array:
\begin{verbatim}
>>> print sum(image[slices[1]], labels[slices[1]], 2)
80.0
\end{verbatim}
Alternatively, we can do the measurements for a number of labels with a 
single function call, returning a list of results. For instance, to measure 
the sum of the values of the background and the second object in our 
example we give a list of labels:
\begin{verbatim}
>>> print sum(image, labels, [0, 2])
[178.0, 80.0]
\end{verbatim}

The measurement functions described below all support the \var{index} 
parameter to indicate which object(s) should be measured. The default value 
of \var{index} is \constant{None}. This indicates that all elements where 
the label is larger than zero should be treated as a single object and 
measured. Thus, in this case the \var{labels} array is treated as a mask 
defined by the elements that are larger than zero. If \var{index} is a 
number or a sequence of numbers it gives the labels of the objects that are 
measured. If \var{index} is a sequence, a list of the results is returned. 
Functions that return more than one result, return their result as a tuple 
if \var{index} is a single number, or as a tuple of lists, if \var{index} 
is a sequence.

\begin{funcdesc}{sum}{input, labels=None, index=None}
  The \function{sum} function calculates the sum of the elements of the 
  object with label(s) given by \var{index}, using the \var{labels} array 
  for the object labels. If \var{index} is \constant{None}, all elements 
  with a non-zero label value are treated as a single object. If 
  \var{label} is \constant{None}, all elements of \var{input} are used in 
  the calculation.
\end{funcdesc}

\begin{funcdesc}{mean}{input, labels=None, index=None}
  The \function{mean} function calculates the mean of the elements of the
  object with label(s) given by \var{index}, using the \var{labels} array 
  for the object labels. If \var{index} is \constant{None}, all elements 
  with a non-zero label value are treated as a single object. If 
  \var{label} is \constant{None}, all elements of \var{input} are used in 
  the calculation.
\end{funcdesc}

\begin{funcdesc}{variance}{input, labels=None, index=None}
  The \function{variance} function calculates the variance of the elements 
  of the object with label(s) given by \var{index}, using the \var{labels} 
  array for the object labels. If \var{index} is \constant{None}, all 
  elements with a non-zero label value are treated as a single object. If 
  \var{label} is \constant{None}, all elements of \var{input} are used in 
  the calculation.
\end{funcdesc}

\begin{funcdesc}{standard_deviation}{input, labels=None, index=None}
  The \function{standard_deviation} function calculates the standard 
  deviation of the elements of the object with label(s) given by 
  \var{index}, using the \var{labels} array for the object labels. If 
  \var{index} is \constant{None}, all elements with a non-zero label value 
  are treated as a single object. If \var{label} is \constant{None}, all 
  elements of \var{input} are used in the calculation.
\end{funcdesc}

\begin{funcdesc}{minimum}{input, labels=None, index=None}
  The \function{minimum} function calculates the minimum of the elements of 
  the object with label(s) given by \var{index}, using the \var{labels} 
  array for the object labels. If \var{index} is \constant{None}, all 
  elements with a non-zero label value are treated as a single object. If 
  \var{label} is \constant{None}, all elements of \var{input} are used in 
  the calculation.
\end{funcdesc}

\begin{funcdesc}{maximum}{input, labels=None, index=None}
  The \function{maximum} function calculates the maximum of the elements of 
  the object with label(s) given by \var{index}, using the \var{labels} 
  array for the object labels. If \var{index} is \constant{None}, all 
  elements with a non-zero label value are treated as a single object. If 
  \var{label} is \constant{None}, all elements of \var{input} are used in 
  the calculation.
\end{funcdesc}

\begin{funcdesc}{minimum_position}{input, labels=None, index=None}
  The \function{minimum_position} function calculates the position of the
  minimum of the elements of the object with label(s) given by \var{index},
  using the \var{labels} array for the object labels. If \var{index} is
  \constant{None}, all elements with a non-zero label value are treated as 
  a single object. If \var{label} is \constant{None}, all elements of 
  \var{input} are used in the calculation.
\end{funcdesc}

\begin{funcdesc}{maximum_position}{input, labels=None, index=None}
  The \function{maximum_position} function calculates the position of the
  maximum of the elements of the object with label(s) given by \var{index},
  using the \var{labels} array for the object labels. If \var{index} is
  \constant{None}, all elements with a non-zero label value are treated as 
  a single object. If \var{label} is \constant{None}, all elements of 
  \var{input} are used in the calculation.
\end{funcdesc}

\begin{funcdesc}{extrema}{input, labels=None, index=None}
  The \function{extrema} function calculates the minimum, the maximum, and 
  their positions, of the elements of the object with label(s) given by 
  \var{index}, using the \var{labels} array for the object labels. If 
  \var{index} is \constant{None}, all elements with a non-zero label value 
  are treated as a single object. If \var{label} is \constant{None}, all 
  elements of \var{input} are used in the calculation. The result is a 
  tuple giving the minimum, the maximum, the position of the mininum and 
  the postition of the maximum. The result is the same as a tuple formed by 
  the results of the functions \function{minimum}, \function{maximum}, 
  \function{minimum_position}, and \function{maximum_position} that are 
  described above.
\end{funcdesc}

\begin{funcdesc}{center_of_mass}{input, labels=None, index=None}
  The \function{center_of_mass} function calculates the center of mass of 
  the of the object with label(s) given by \var{index}, using the 
  \var{labels} array for the object labels. If \var{index} is 
  \constant{None}, all elements with a non-zero label value are treated as 
  a single object. If \var{label} is \constant{None}, all elements of 
  \var{input} are used in the calculation.
\end{funcdesc}

\begin{funcdesc}{histogram}{input, min, max, bins, labels=None, index=None}
  The \function{histogram} function calculates a histogram of 
  the of the object with label(s) given by \var{index}, using the 
  \var{labels} array for the object labels. If \var{index} is 
  \constant{None}, all elements with a non-zero label value are treated as 
  a single object. If \var{label} is \constant{None}, all elements of 
  \var{input} are used in the calculation. Histograms are defined by their 
  minimum (\var{min}), maximum (\var{max}) and the number of bins 
  (\var{bins}). They are returned as one-dimensional arrays of type Int32. 
\end{funcdesc}

\section{Extending \module{nd\_image} in C}
\label{sec:ndimage:ccallbacks}
\subsection{C callback functions}
A few functions in the \module{numarray.nd\_image} take a call-back 
argument. This can be a python function, but also a CObject containing a 
pointer to a C function. To use this feature, you must write your own C 
extension that defines the function, and define a python function that 
returns a CObject containing a pointer to this function.

An example of a function that supports this is 
\function{geometric_transform} (see section \ref{sec:ndimage:
interpolation}). You can pass it a python callable object that defines a 
mapping from all output coordinates to corresponding coordinates in the 
input array. This mapping function can also be a C function, which 
generally will be much more efficient, since the overhead of calling a
python function at each element is avoided.

For example to implement a simple shift function we define the following 
function:
\begin{verbatim}
static int 
_shift_function(int *output_coordinates, double* input_coordinates,
                int output_rank, int input_rank, void *callback_data)
{
  int ii;
  /* get the shift from the callback data pointer: */
  double shift = *(double*)callback_data;
  /* calculate the coordinates: */
  for(ii = 0; ii < irank; ii++)
    icoor[ii] = ocoor[ii] - shift;
  /* return OK status: */
  return 1;
}
\end{verbatim}
This function is called at every element of the output array, passing the 
current coordinates in the \var{output_coordinates} array. On return, the 
\var{input_coordinates} array must contain the coordinates at which the 
input is interpolated. The ranks of the input and output array are passed 
through \var{output_rank} and \var{input_rank}. The value of the shift is 
passed through the \var{callback_data} argument, which is a pointer to 
void. The function returns an error status, in this case always 1, since no 
error can occur.

A pointer to this function and a pointer to the shift value must be passed 
to \function{geometric_transform}. Both are passed by a single CObject 
which is created by the following python extension function:
\begin{verbatim}
static PyObject *
py_shift_function(PyObject *obj, PyObject *args)
{
  double shift = 0.0;
  if (!PyArg_ParseTuple(args, "d", &shift)) {
    PyErr_SetString(PyExc_RuntimeError, "invalid parameters");
    return NULL;
  } else {
    /* assign the shift to a dynamically allocated location: */
    double *cdata = (double*)malloc(sizeof(double));
    *cdata = shift;
    /* wrap function and callback_data in a CObject: */
    return PyCObject_FromVoidPtrAndDesc(_shift_function, cdata,
                                        _destructor);
  }
}
\end{verbatim}
The value of the shift is obtained and then assigned to a dynamically 
allocated memory location. Both this data pointer and the function pointer 
are then wrapped in a CObject, which is returned. Additionally, a pointer 
to a destructor function is given, that will free the memory we allocated 
for the shift value when the CObject is destroyed. This destructor is very 
simple:
\begin{verbatim}
static void
_destructor(void* cobject, void *cdata)
{
  if (cdata)
    free(cdata);
}
\end{verbatim}
To use these functions, an extension module is build:
\begin{verbatim}
static PyMethodDef methods[] = {
  {"shift_function", (PyCFunction)py_shift_function, METH_VARARGS, ""},
  {NULL, NULL, 0, NULL}
};

void
initexample(void)
{
  Py_InitModule("example", methods);
}
\end{verbatim}
This extension can then be used in Python, for example:
\begin{verbatim}
>>> import example
>>> array = arange(12, shape=(4,3), type = Float64)
>>> fnc = example.shift_function(0.5)
>>> print geometric_transform(array, fnc)
[[ 0.      0.      0.    ]
 [ 0.      1.3625  2.7375]
 [ 0.      4.8125  6.1875]
 [ 0.      8.2625  9.6375]]
\end{verbatim}

C Callback functions for use with \module{nd\_image} functions must all be 
written according to this scheme. The next section lists the 
\module{nd\_image} functions that acccept a C callback function and gives 
the prototype of the callback function.

\subsection{Functions that support C callback functions}
The \module{nd\_image} functions that support C callback functions are 
described here. Obviously, the prototype of the function that is provided 
to these functions must match exactly that what they expect. Therefore we 
give here the prototypes of the callback functions. All these callback 
functions accept a void \var{callback_data} pointer that must be wrapped in 
a CObject using the Python \cfunction{PyCObject_FromVoidPtrAndDesc} 
function, which can also accept a pointer to a destructor function to free 
any memory allocated for \var{callback_data}. If \var{callback_data} is not 
needed, \cfunction{PyCObject_FromVoidPtr} may be used instead. The callback 
functions must return an integer error status that is equal to zero if 
something went wrong, or 1 otherwise. If an error occurs, you should 
normally set the python error status with an informative message before 
returning, otherwise, a default error message is set by the calling 
function.

The function \function{generic_filter} (see section 
\ref{sec:ndimage:genericfilters}) accepts a callback function with the 
following prototype:
\begin{cfuncdesc}{int}{FilterFunction}{double *buffer, int filter_size,
double *return_value, void *callback_data} The calling function iterates 
over the elements of the input and output arrays, calling the callback 
function at each element. The elements within the footprint of the filter 
at the current element are passed through the \var{buffer} parameter, and 
the number of elements within the footprint through \var{filter_size}. The 
calculated valued should be returned in the \var{return_value} argument.
\end{cfuncdesc}

The function \function{generic_filter1d} (see section 
\ref{sec:ndimage:genericfilters}) accepts a callback function with the 
following prototype: 
\begin{cfuncdesc}{int}{FilterFunction1D}{double *input_line, int 
input_length, double *output_line, int output_length, void *callback_data} 
The calling function iterates over the lines of the input and output 
arrays, calling the callback function at each line. The current line is 
extended according to the border conditions set by the calling function, 
and the result is copied into the array that is passed through the 
\var{input_line} array. The length of the input line (after extension) is 
passed through \var{input_length}. The callback function should apply the 
1D filter and store the result in the array passed through 
\var{output_line}. The length of the output line is passed through 
\var{output_length}.
\end{cfuncdesc}

The function \function{geometric_transform} (see section 
\ref{sec:ndimage:interpolation}) expects a function with the following 
prototype: 
\begin{cfuncdesc}{int}{MapCoordinates}{int *output_coordinates, 
double* input_coordinates, int output_rank, int input_rank, 
void *callback_data} The calling function iterates over the elements of the 
output array, calling the callback function at each element. The 
coordinates of the current output element are passed through 
\var{output_coordinates}. The callback function must return the coordinates 
at which the input must be interpolated in \var{input_coordinates}. The 
rank of the input and output arrays are given by \var{input_rank} and 
\var{output_rank} respectively.
\end{cfuncdesc}


\chapter{Memory Mapping}
\label{cha:memmap}
\declaremodule{extension}{numarray.memmap}
\index{character array}
\index{string array}

\section{Introduction}
\label{sec:memmap-intro}

\code{numarray} provides support for the creation of arrays which are
mapped directly onto files with the \code{numarray.memmap} module.
Much of \code{numarray}'s design, the ability to handle misaligned and
byteswapped arrays for instance, was motivated by the desire to create
arrays from portable files which contain binary array data.  One
advantage of memory mapping is efficient random access to small
regions of a large file: only the region of the mapped file which is
actually used in array operations needs to be paged into system
memory; the rest of the file remains unread and unwritten.

\code{numarray.memmap} is pure Python and is layered on top of
Python's \code{mmap} module.  The basic idea behind \code{numarray}'s
memory mapping is to create a ``buffer'' referring to a region in a
mapped file and to use it as the data store for an array.  The
\code{numarray.memmap} module contains two classes, one which
corresponds to an entire mapped file (\class{Memmap}) and one which
corresponds to a contiguous region within a file
(\class{MemmapSlice}).  \class{MemmapSlice} objects have these
properties:

\begin{itemize}
\item MemmapSlices can be used as NumArray buffers.
\item MemmapSlices are non-overlapping.
\item MemmapSlices are resizable.
\item Changing the size of a MemmapSlice changes the parent Memmap.
\end{itemize}

\section{Opening a Memmap}
\label{sec:memmap-open}

You can create a \class{Memmap} object by calling the \function{open} function,
as in:

\begin{verbatim}
>>> m = open("memmap.tst","w+",len=48)
>>> m
<Memmap on file 'memmap.tst' with mode='w+', length=48, 0 slices>
\end{verbatim}

Here, the file ``memmap.tst'' is created/truncated to a length of 48
bytes and used to construct a Memmap object in write mode whose
contents are considered undefined.  

\section{Slicing a Memmap}
\label{sec:memmap-slicing}

Once opened, a \class{Memmap} object can be sliced into regions.

\begin{verbatim}
# Slice m into the buffers "n" and "p" which will correspond to numarray:

>>> n = m[0:16]
>>> n
<MemmapSlice of length:16 writable>

>>> p = m[24:48]
>>> p
<MemmapSlice of length:24 writable>
\end{verbatim}

NOTE: You cannot make \emph{overlapping} slices of a Memmap:

\begin{verbatim}
>>> q = m[20:28]
Traceback (most recent call last):
...
IndexError: Slice overlaps prior slice of same file.
\end{verbatim}

Deletion of a slice is possible once all other references to it are
forgotten, e.g. all arrays that used it have themselves been deleted.
Deletion of a slice of a Memmap "un-registers" the slice, making that
region of the Memmap available for reallocation.  Delete directly from
the Memmap without referring to the MemmapSlice:

\begin{verbatim}
>>> m = Memmap("memmap.tst",mode="w+",len=100)
>>> m1 = m[0:50]
>>> del m[0:50]      # note: delete from m, not m1
>>> m2 = m[0:70]
\end{verbatim}

Note that since the region of m1 was deleted, there is no overlap when
m2 is created.  However, deleting the region of m1 has invalidated it:

\begin{verbatim}
>>> m1
Traceback (most recent call last):
...
RuntimeError: A deleted MemmapSlice has been used.
\end{verbatim}

Don't mix operations on a Memmap which modify its data or file
structure with slice deletions.  In this case, the status of the
modifications is undefined; the underlying map may or may not reflect
the modifications after the deletion.

\section{Creating an array from a MemmapSlice}
\label{sec:memmap-array-construction}

Arrays are created from \class{MemmapSlice}s simply by specifying the
slice as the \var{buffer} parameter of the array.  Since the slice is
essentially just a byte string, it's necessary to specify the
\var{type} of the binary data as well.

\begin{verbatim}
>>> a = num.NumArray(buffer=n, shape=(len(n)/4,), type=num.Int32)
>>> a[:] = 0  # Since the initial contents of 'n' are undefined.
>>> a += 1
array([1, 1, 1, 1], type=Int32)
\end{verbatim}

\section{Resizing a MemmapSlice}
\label{sec:memmap-slice}

Arrays based on \class{MemmapSlice} objects are resizable.  As soon as
they're resized, slices become un-mapped or ``free floating''.
Resizing a slice affects the parent \class{Memmap}.

\begin{verbatim}
>>> a.resize(6)
array([1, 1, 1, 1, 1, 1], type=Int32)
\end{verbatim}

\section{Forcing file updates and closing the Memmap}
\label{sec:memmap-flushing-closing}

After doing slice resizes or inserting new slices, call
\function{flush} to synchronize the underlying map file with any free
floating slices.  This explicit step is required to avoid implicitly
shuffling huge amounts of file space for every \function{resize} or
\function{insert}.  After calling \function{flush}, all slices are
once again memory mapped rather than free floating.

\begin{verbatim}
>>> m.flush()
\end{verbatim}

A related concept is ``syncing'' which applies even to arrays which
have not been resized.  Since memory maps don't guarantee when the
underlying file will be updated with the values you have written to
the map, call \function{sync} when you want to be sure your changes
are on disk.  This is similar to syncing a UNIX file system.  Note
that \function{sync} does not consolidate the mapfile with any free
floating slices (newly inserted or resized), it merely ensures that
mapped slices whose contents have been altered are written to disk.

\begin{verbatim}
>>> m.sync()
\end{verbatim}

Now "a" and "b" are both memory mapped on "memmap.tst" again.

When you're done with the memory map and numarray, call
\function{close}. \function{close} calls \function{flush} which will
consolidate resized or inserted slices as necessary.

\begin{verbatim}
>>> m.close()
\end{verbatim}

It is an error to use "m" (or slices of m) any further after closing
it.

\section{numarray.memmap functions}
\label{sec:memmap-functions}

\begin{funcdesc}{open}{filename, mode='r+', len=None}
\label{func:memmap-open}
\function{open} opens a \class{Memmap} object on the file
\var{filename} with the specified \var{mode}.  Available \var{mode}
values include 'readonly' ('r'), 'copyonwrite' ('c'), 'readwrite'
('r+'), and 'write' ('w+'), all but the last of which have contents
defined by the file.

Neither mode 'r' nor mode 'c' can affect the underlying map file.
Readonly maps impose no requirement on system swap space and raise
exceptions when their contents are modified.  Copy-on-write maps
require system swap space corresponding to their size, but have
modifiable pages which become reassociated with system swap as they
are changed leaving the original map file unaltered.  Insufficient
swap space can prevent the creation of a copy-on-write memory map.
Modifications to readwrite memory maps are eventually reflected onto
the map file;  see flushing and syncing.
\end{funcdesc}
   
\begin{funcdesc}{close}{map}
\label{func:memmap-close}
\function{close} closes the \class{Memmap} object specified by
\var{map}.
\end{funcdesc}
   
\section{Memmap methods}
\label{sec:memmap-methods}
A Memmap object represents an entire mapped file and is sliced to
create objects which can be used as array buffers.  It has these
public methods:

\begin{methoddesc}[Memmap]{close}{}
  \function{close} unites the \class{Memmap} and any RAM based slices with
  its underlying file and removes the mapping and all references to
  its slices.  Once a \class{Memmap} has been closed, all of its
  slices become unusable.
\end{methoddesc}

\begin{methoddesc}[Memmap]{find}{string, offset=0}
  find(string, offset=0) returns the first index at which string
  is found, or -1 on failure.
  \begin{verbatim}
    >>> _open("memmap.tst","w+").write("this is a test")
    >>> Memmap("memmap.tst",len=14).find("is")
    2
    >>> Memmap("memmap.tst",len=14).find("is", 3)
    5
    >>> _open("memmap.tst","w+").write("x")
    >>> Memmap("memmap.tst",len=1).find("is")
    -1
  \end{verbatim}
\end{methoddesc}

\begin{methoddesc}[Memmap]{insert}{offset, size=None, buffer=None}
  \function{insert} places a new slice at the specified \var{offset} of
  the \class{Memmap}.  \var{size} indicates the length in bytes of the
  inserted slice when \var{buffer} is not specified.  If \function{buffer}
  is specified, it should refer to an existing memory object created
  using \code{numarray.memory.new_memory} and \function{size} should not
  be specified.
\end{methoddesc}

\begin{methoddesc}[Memmap]{flush}{}
  \function{flush} writes a \class{Memmap} out to its associated file,
  reconciling any inserted or resized slices by backing them directly
  on the map file rather than a system swap file.  \function{flush}
  only makes sense for write and readwrite memory maps.
\end{methoddesc}

\begin{methoddesc}[Memmap]{sync}{}
  \function{sync} forces slices which are backed on the map file to be
  immediately written to disk.  Resized or newly inserted slices are
  not affected.  \function{sync} only makes sense for write and
  readwrite memory maps.
\end{methoddesc}

\section{MemmapSlice methods}
\label{sec:memmap-methods}
A \class{MemmapSlice} object represents a subregion of a
\class{Memmap} and has these public methods:

\begin{methoddesc}[MemmapSlice]{__buffer__}{}
  Returns an object which supports the Python buffer protocol and
  represents this slice.  The Python buffer protocol enables a C
  function to obtain the pointer and size corresponding to the data
  region of the slice.
\end{methoddesc}

\begin{methoddesc}[MemmapSlice]{resize}{newsize}
  \function{resize} expands or contracts this slice to the specified
  \var{newsize}.
\end{methoddesc}

%% mode: LaTeX
%% mode: auto-fill
%% fill-column: 79
%% indent-tabs-mode: nil
%% ispell-dictionary: "american"
%% reftex-fref-is-default: nil
%% TeX-auto-save: t
%% TeX-command-default: "pdfeLaTeX"
%% TeX-master: "numarray"
%% TeX-parse-self: t
%% End:


\appendix
\part*{Appendix}
%begin{latexonly}
\makeatletter
\py@reset
\makeatother
%end{latexonly}

\chapter{Glossary}
\label{cha:glossary}

\begin{quote} 
   This chapter provides a glossary of terms.\footnote{Please let us know of
      any additions to this list which you feel would be helpful.}
\end{quote}

\begin{description}
\item[array] An array refers to the Python object type defined by the NumPy
   extensions to store and manipulate numbers efficiently.
\item[byteswapped]
\item[discontiguous]  
\item[misaligned] 
\item[misbehaved array] A \class{\numarray} which is byteswapped, misaligned,
   or discontiguous.
\item[rank] The rank of an array is the number of dimensions it has, or the
   number of integers in its shape tuple.
\item[shape] Array objects have an attribute called shape which is necessarily
   a tuple. An array with an empty tuple shape is treated like a scalar (it
   holds one element).
\item[ufunc] A callable object which performs operations on all of the elements
   of its arguments, which can be lists, tuples, or arrays. Many ufuncs are
   defined in the umath module.
\item[universal function] See ufunc.
\end{description}
 


%% Local Variables:
%% mode: LaTeX
%% mode: auto-fill
%% fill-column: 79
%% indent-tabs-mode: nil
%% ispell-dictionary: "american"
%% reftex-fref-is-default: nil
%% TeX-auto-save: t
%% TeX-command-default: "pdfeLaTeX"
%% TeX-master: "numarray"
%% TeX-parse-self: t
%% End:

% Complete documentation on the extended LaTeX markup used for Python
% documentation is available in ``Documenting Python'', which is part
% of the standard documentation for Python.  It may be found online
% at:
%
%     http://www.python.org/doc/current/doc/doc.html

\documentclass[hyperref]{manual}
\pagestyle{plain}

% latex2html doesn't know [T1]{fontenc}, so we cannot use that:(

\usepackage{amsmath}
\usepackage[latin1]{inputenc}
\usepackage{textcomp}


% The commands of this document do not reset module names at section level
% (nor at chapter level).
% --> You have to do that manually when a new module starts!
%     (use \py@reset)
%begin{latexonly}
\makeatletter
\renewcommand{\section}{\@startsection{section}{1}{\z@}%
   {-3.5ex \@plus -1ex \@minus -.2ex}%
   {2.3ex \@plus.2ex}%
   {\reset@font\Large\py@HeaderFamily}}
\makeatother
%end{latexonly}


% additional mathematical functions
\DeclareMathOperator{\abs}{abs}

% provide a cross-linking command for the index
%begin{latexonly}
\newcommand*\see[2]{\protect\seename #1}
\newcommand*{\seename}{$\to$}
%end{latexonly}


% some convenience declarations
\newcommand{\numarray}{numarray}
\newcommand{\Numarray}{Numarray}  % Only beginning of sentence, otherwise use \numarray
\newcommand{\NUMARRAY}{NumArray}
\newcommand{\numpy}{Numeric}
\newcommand{\NUMPY}{Numerical Python}
\newcommand{\python}{Python}


% mark internal comments
% for any published version switch to the second (empty) definition of the macro!
% \newcommand{\remark}[1]{(\textbf{Note to authors: #1})}
\newcommand{\remark}[1]{}


\title{numarray\\User's Manual}

\author{Perry Greenfield \\
   Todd Miller \\
   Rick White \\
   J.C. Hsu \\
   Paul Barrett \\
   Jochen K�pper \\
   Peter J. Verveer \\[1ex]
   Previously authored by: \\
   David Ascher \\
   Paul F. Dubois \\
   Konrad Hinsen \\
   Jim Hugunin \\
   Travis Oliphant \\[1ex]
   with contributions from the Numerical Python community}

\authoraddress{Space Telescope Science Institute, 3700 San Martin Dr,
   Baltimore, MD 21218 \\ UCRL-MA-128569}

% I use date to indicate the manual-updates,
% release below gives the matching software version.
\date{November 2, 2005}        % update before release!
                                % Use an explicit date so that reformatting
                                % doesn't cause a new date to be used.  Setting
                                % the date to \today can be used during draft
                                % stages to make it easier to handle versions.

\release{1.5}                 % (software) release version;
\setshortversion{1.5}         % this is used to define the \version macro

\makeindex                      % tell \index to actually write the .idx file



\begin{document}

\maketitle

% This makes the contents more accessible from the front page of the HTML.
\ifhtml
\part*{General}
\chapter*{Front Matter}
\label{front}
\fi

\section*{Legal Notice}
\label{sec:legal-notice}

Please see file LICENSE.txt in the source distribution.  

This open source project has been contributed to by many people, including
personnel of the Lawrence Livermore National Laboratory, Livermore, CA, USA.
The following notice covers those contributions, including contributions to
this this manual.

Copyright (c) 1999, 2000, 2001.  The Regents of the University of California.
All rights reserved.

Permission to use, copy, modify, and distribute this software for any purpose
without fee is hereby granted, provided that this entire notice is included in
all copies of any software which is or includes a copy or modification of this
software and in all copies of the supporting documentation for such software.

This work was produced at the University of California, Lawrence Livermore
National Laboratory under contract no. W-7405-ENG-48 between the U.S.
Department of Energy and The Regents of the University of California for the
operation of UC LLNL.



\subsection*{Special license for package numarray.ma}
\label{sec:license-numarray.ma}


The package \module{numarray.ma} was written by Paul Dubois, Lawrence Livermore
National Laboratory, Livermore, CA, USA.

Copyright (c) 1999, 2000. The Regents of the University of California. All
rights reserved.

Permission to use, copy, modify, and distribute this software for any purpose
without fee is hereby granted, provided that this entire notice is included in
all copies of any software which is or includes a copy or modification of this
software and in all copies of the supporting documentation for such software.

This work was produced at the University of California, Lawrence Livermore
National Laboratory under contract no. W-7405-ENG-48 between the U.S.
Department of Energy and The Regents of the University of California for the
operation of UC LLNL.



\subsection*{Disclaimer}

This software was prepared as an account of work sponsored by an agency of the
United States Government. Neither the United States Government nor the
University of California nor any of their employees, makes any warranty,
express or implied, or assumes any liability or responsibility for the
accuracy, completeness, or usefulness of any information, apparatus, product,
or process disclosed, or represents that its use would not infringe
privately-owned rights. Reference herein to any specific commercial products,
process, or service by trade name, trademark, manufacturer, or otherwise, does
not necessarily constitute or imply its endorsement, recommendation, or
favoring by the United States Government or the University of California. The
views and opinions of authors expressed herein do not necessarily state or
reflect those of the United States Government or the University of California,
and shall not be used for advertising or product endorsement purposes.




%% Local Variables:
%% mode: LaTeX
%% mode: auto-fill
%% fill-column: 79
%% indent-tabs-mode: nil
%% ispell-dictionary: "american"
%% reftex-fref-is-default: nil
%% TeX-auto-save: t
%% TeX-command-default: "pdfeLaTeX"
%% TeX-master: "numarray"
%% TeX-parse-self: t
%% End:
  \cleardoublepage


\tableofcontents


\part{Numerical Python}

\NUMARRAY{} (``\numarray{}'') adds a fast multidimensional array facility to
Python.  This part contains all you need to know about ``\numarray{}'' arrays
and the functions that operate upon them.

\label{part:numerical-python}

\declaremodule{extension}{numarray}
\moduleauthor{The numarray team}{numpy-discussion@lists.sourceforge.net}
\modulesynopsis{Numerics}

\chapter{Introduction}
\label{cha:introduction}

\begin{quote}
   This chapter introduces the numarray Python extension and outlines the rest
   of the document.
\end{quote}

Numarray is a set of extensions to the Python programming language which allows
Python programmers to efficiently manipulate large sets of objects organized in
grid-like fashion. These sets of objects are called arrays, and they can have
any number of dimensions. One-dimensional arrays are similar to standard Python
sequences, and two-dimensional arrays are similar to matrices from linear
algebra. Note that one-dimensional arrays are also different from any other
Python sequence, and that two-dimensional matrices are also different from the
matrices of linear algebra. One significant difference is that numarray objects
must contain elements of homogeneous type, while standard Python sequences can
contain elements of mixed type. Two-dimensional arrays differ from matrices
primarily in the way multiplication is performed; 2-D arrays are multiplied
element-by-element.

This is a reimplementation of the earlier Numeric module (aka numpy). For the
most part, the syntax of numarray is identical to that of Numeric, although
there are significant differences. The differences are primarily in new
features. For Python 2.2 and later, the syntax is completely backwards
compatible. See the High-Level Overview (chapter \ref{cha:high-level-overview})
for incompatibilities for earlier versions of Python. The reasons for rewriting
Numeric and a comparison between Numeric and numarray are also described in
chapter \ref{cha:high-level-overview}. Portions of the present document are
almost word-for-word identical to the Numeric manual. It has been updated to
reflect the syntax and behavior of numarray, and there is a new section
(~\ref{sec:diff-numarray-numpy}) on differences between Numeric and numarray.

Why are these extensions needed? The core reason is a very prosaic one:
manipulating a set of a million numbers in Python with the
standard data structures such as lists, tuples or classes is much too slow and
uses too much space. A more subtle
reason for these extensions, however, is that the kinds of operations that
programmers typically want to do on arrays, while sometimes very complex, can
often be decomposed into a set of fairly standard operations. This
decomposition has been similarly developed in many array languages. In some
ways, numarray is simply the application of this experience to the Python
language.  Thus many of the operations described in numarray work the way they
do because experience has shown that way to be a good one, in a variety of
contexts. The languages which were used to guide the development of numarray
include the infamous APL family of languages, Basis, MATLAB, FORTRAN, S and S+,
and others.  This heritage will be obvious to users of numarray who already
have experience with these other languages.  This manual, however, does not
assume any such background, and all that is expected of the reader is a
reasonable working knowledge of the standard Python language.

This document is the ``official'' documentation for numarray. It is both a
tutorial and the most authoritative source of information about numarray with
the exception of the source code. The tutorial material will walk you through a
set of manipulations of simple, small arrays of numbers. This choice was made
because:
\begin{itemize}
\item A concrete data set makes explaining the behavior of some functions much
   easier to motivate than simply talking about abstract operations on abstract
   data sets.
\item Every reader will have at least an intuition as to the meaning of the
   data and organization of image files.  \remark{These ``image files'' are not
      mentioned anywhere before, and not really used later...?}
\end{itemize}
All users of numarray, whether interested in image processing or not, are
encouraged to follow the tutorial with a working numarray installed,
testing the examples, and more importantly, transferring the
understanding gained by working on arrays to their specific domain. The best
way to learn is by doing --- the aim of this tutorial is to guide you along
this "doing."

This manual contains:
\begin{description}
\item[Installing numarray] Chapter \ref{cha:installation} provides information
   on testing Python, numarray, and compiling and installing numarray if
   necessary.
\item[High-Level Overview] Chapter \ref{cha:high-level-overview} gives a
   high-level overview of the components of the numarray system as a whole.
\item[Array Basics] Chapter \ref{cha:array-basics} provides a detailed
   step-by-step introduction to the most important aspect of numarray, the
   multidimensional array objects.
\item[Ufuncs] Chapter \ref{cha:ufuncs} provides information on universal
   functions, the mathematical functions which operate on arrays and other
   sequences elementwise.
\item[Pseudo Indices] Chapter \ref{cha:pseudo-indices} covers syntax for some
   special indexing operators.
\item[Array Functions] Chapter \ref{cha:array-functions} is a catalog of each
   of the utility functions which allow easy algorithmic processing of arrays.
\item[Array Methods] Chapter \ref{cha:array-methods} discusses the methods of
   array objects.
\item[Array Attributes] Chapter \ref{cha:array-attributes} presents the
   attributes of array objects.
\item[Character Array] Chapter \ref{cha:character-array} describes the
  \code{numarray.strings} module that provides support for arrays of fixed
  length strings.
\item[Record Array] Chapter \ref{cha:record-array} describes the
   \code{numarray.records} module that supports arrays of fixed length records
   of string or numerical data.
\item[Object Array] Chapter \ref{cha:object-array} describes the
   \code{numarray.objects} module that supports arrays of Python objects.
\item[C extension API] Chapter \ref{cha:C-API} describes the C-APIs provided
   for \module{numarray} based extension modules.
\item[Convolution] Chapter \ref{cha:convolve} describes the
   \module{numarray.convolve} module for computing one-D and two-D convolutions
   and correlations of \class{numarray} objects.
\item[Fast-Fourier-Transform] Chapter \ref{cha:fft} describes the
   \module{numarray.fft} module for computing Fast-Fourier-Transforms
   (FFT) and Inverse FFTs over \class{numarray} objects in one- or
   two-dimensional manner.  Ported from Numeric.
\item[Linear Algebra] Chapter \ref{cha:linear-algebra} describes the
   \module{numarray.linear_algebra} module which provides a simple
   interface to some commonly used linear algebra routines; 
   \program{LAPACK}.   Ported from Numeric.
\item[Masked Arrays] Chapter \ref{cha:masked-arrays} describes the
   \module{numarray.ma} module which supports Masked Arrays: arrays which
   potentially have missing or invalid elements.  Ported from Numeric.
\item[Random Numbers] Chapter \ref{cha:random-array} describes the
   \module{numarray.random_array} module which supports generation of arrays of
   random numbers.  Ported from Numeric.
\item[Multidimentional image analysis functions] Chapter \ref{cha:ndimage}
   describes the \module{numarray.ndimage} module which provides
   functions for multidimensional image analysis such as filtering,
   morphology or interpolation.
\item[Glossary] Appendix \ref{cha:glossary} gives a glossary of terms.
\end{description}


\section{Where to get information and code}

Numarray and its documentation are available at SourceForge
(\ulink{sourceforge.net}{http://sourceforge.net}; SourceForge addresses can
also be abbreviated as \ulink{sf.net}{http://sf.net}). The main web site is:
\url{http://numpy.sourceforge.net}. Downloads, bug reports, a patch facility,
and releases are at the main project page, reachable from the above site or
directly at: \url{http://sourceforge.net/projects/numpy} (see Numarray under
"Latest File Releases").  The Python web site is \url{http://www.python.org}.
For up-to-date status on compatible modules available for numarray, please 
check \url{http://www.stsci.edu/resources/software_hardware/numarray/}.

NOTE: because numarray shares the numpy Source Forge project with Numeric and
Numeric3, there are dedicated Source Forge ``Trackers'' for numarray, .e.g.
``Numarray Bugs'' rather than just ``Bugs''.  When submitting bug reports,
patches, or requests, please look for the numarray version of the tracker under
the top level menu item ``Tracker'', nominally here:
\url{http://sourceforge.net/tracker/?group_id=1369}.

\section{Acknowledgments}

Numerical Python was the outgrowth of a long collaborative design process
carried out by the Matrix SIG of the Python Software Activity (PSA). Jim
Hugunin, while a graduate student at MIT, wrote most of the code and initial
documentation. When Jim joined CNRI and began working on JPython, he didn't
have the time to maintain Numerical Python so Paul Dubois at LLNL agreed to
become the maintainer of Numerical Python. David Ascher, working as a
consultant to LLNL, wrote most of the Numerical Python version of this
document, incorporating contributions from Konrad Hinsen and Travis Oliphant,
both of whom are major contributors to Numerical Python.  The reimplementation
of Numeric as numarray was done primarily by Perry Greenfield, Todd Miller, and
Rick White, with some assistance from J.C. Hsu and Paul Barrett. Although
numarray is almost a completely new implementation, it owes a great deal to the
ideas, interface and behavior expressed in the Numeric implementation. It is
not an overstatement to say that the existence of Numeric made the
implementation of numarray far, far easier that it would otherwise have been.
Since the source for the original Numeric module was moved to SourceForge, the
numarray user community has become a significant part of the process.
Numeric/numarray illustrates the power of the open source software concept.
Please send comments and corrections to this manual to
\ulink{perry@stsci.edu}{mailto:perry@stsci.edu}, or to Perry Greenfield, 3700
San Martin Dr, Baltimore, MD 21218, U.S.A.




%% Local Variables:
%% mode: LaTeX
%% mode: auto-fill
%% fill-column: 79
%% indent-tabs-mode: nil
%% ispell-dictionary: "american"
%% reftex-fref-is-default: nil
%% TeX-auto-save: t
%% TeX-command-default: "pdfeLaTeX"
%% TeX-master: "numarray"
%% TeX-parse-self: t
%% End:

\chapter{Installing numarray}
\label{cha:installation}

\begin{quote}
   This chapter explains how to install and test numarray, from either the
   source distribution or from the binary distribution.
\end{quote}

Before we start with the actual tutorial, we will describe the steps needed for
you to be able to follow along the examples step by step. These steps include
installing and testing Python, the numarray extensions, and some tools and
sample files used in the examples of this tutorial.


\section{Testing the Python installation}

The first step is to install Python if you haven't already. Python is available
from the Python project page at \url{http://sourceforge.net/projects/python}.
Click on the link corresponding to your platform, and follow the instructions
described there. Unlike earlier versions of numarray, version 0.7 and later
require Python version 2.2.2 at a minimum.  When installed, starting Python by
typing python at the shell or double-clicking on the Python interpreter should
give a prompt such as:
\begin{verbatim}
Python 2.3 (#2, Aug 22 2003, 13:47:10) [C] on sunos5
Type "help", "copyright", "credits" or "license" for more information.
\end{verbatim}
If you have problems getting this part to work, consider contacting a local
support person or emailing
\ulink{python-help@python.org}{mailto:python-help@python.org} for help. If
neither solution works, consider posting on the
\ulink{comp.lang.python}{news:comp.lang.python} newsgroup (details on the
newsgroup/mailing list are available at
\url{http://www.python.org/psa/MailingLists.html\#clp}).


\section{Testing the Numarray Python Extension Installation}

The standard Python distribution does not come, as of this writing, with the
numarray Python extensions installed, but your system administrator may have
installed them already. To find out if your Python interpreter has numarray
installed, type \samp{import numarray} at the Python prompt. You'll see one of
two behaviors (throughout this document user input and python interpreter
output will be emphasized as shown in the block below):
\begin{verbatim}
>>> import numarray
Traceback (innermost last):
File "<stdin>", line 1, in ?
ImportError: No module named numarray
\end{verbatim}
indicating that you don't have numarray installed, or:
\begin{verbatim}
>>> import numarray
>>> numarray.__version__
'0.6'
\end{verbatim}
indicating that you do. If you do, you can skip the next section
and go ahead to section \ref{sec:at-sourceforge}.  If you don't, you have to
get and install the numarray extensions as described in section
\ref{sec:installing-numarray}.

\section{Installing numarray}
\label{sec:installing-numarray}

The release facility at SourceForge is accessed through the project page,
\url{http://sourceforge.net/projects/numpy}.  Click on the "Numarray" release
and you will be presented with a list of the available files. The files whose
names end in ".tar.gz" are source code releases. The other files are binaries
for a given platform (if any are available).

It is possible to get the latest sources directly from our CVS repository using
the facilities described at SourceForge. Note that while every effort is made
to ensure that the repository is always ``good'', direct use of the repository
is subject to more errors than using a standard release.


\subsection{Installing on Unix, Linux, and Mac OSX}
\label{sec:installing-unix}

The source distribution should be uncompressed and unpacked as follows (for
example):
\begin{verbatim}
gunzip numarray-0.6.tar.gz
tar xf numarray-0.6.tar
\end{verbatim}
Follow the instructions in the top-level directory for compilation and
installation. Note that there are options you must consider before beginning.
Installation is usually as simple as:
\begin{verbatim}
python setup.py install
\end{verbatim}
or:
\begin{verbatim}
python setupall.py install
\end{verbatim}
if you want to install all additional packages, which include
\module{\mbox{numarray.convolve}}, \module{\mbox{numarray.fft}},
\module{\mbox{numarray.linear_algebra}}, and
\module{\mbox{numarray.random_array}}.

See numarray-X.XX/Doc/INSTALL.txt for the latest details (X.XX is the version 
number).

\paragraph*{Important Tip} \label{sec:tip:from-numarray-import} Just like all
Python modules and packages, the numarray module can be invoked using either
the \samp{import numarray} form, or the \samp{from numarray import ...} form.
Because most of the functions we'll talk about are in the numarray module, in
this document, all of the code samples will assume that they have been preceded
by a statement:
\begin{verbatim}
>>> from numarray import *
\end{verbatim}
Note the lowercase name in \module{\numarray} as opposed to \module{\numpy}.


\subsection{Installing on Windows}
\label{sec:installing-windows}

To install numarray, you need to be in an account with Administrator
privileges.  As a general rule, always remove (or hide) any old version of
numarray before installing the next version.

We have tested Numarrray on several Win-32 platforms including:
   
\begin{itemize}
\item Windows-XP-Pro-x86 ( MSVC-6.0) 
\item Windows-NT-x86 (MSVC-6.0) 
\item Windows-98-x86 (MSVC-6.0)
\end{itemize}

\subsubsection{Installation from source}

\begin{enumerate}
\item Unpack the distribution: (NOTE: You may have to download an "unzipping"
   utility)
\begin{verbatim}
C:\> unzip numarray.zip 
C:\> cd numarray
\end{verbatim}
\item Build it using the distutils defaults:
\begin{verbatim}
C:\numarray> python setup.py install
\end{verbatim}
   This installs numarray in \texttt{C:\textbackslash{}pythonXX} where XX is the version
   number of your python installation, e.g. 20, 21, etc.
\end{enumerate}


\subsubsection{Installation from self-installing executable}

\begin{enumerate}
\item Click on the executable's icon to run the installer.
\item Click "next" several times.  I have not experimented with customizing the
   installation directory and don't recommend changing any of the installation
   defaults.  If you do and have problems, let us know.
\item Assuming everything else goes smoothly, click "finish".
\end{enumerate}


\subsubsection{Testing your Installation}

Once you have installed numarray, test it with:
\begin{verbatim}
C:\numarray> python
Python 2.2.2 (#18, Dec 30 2002, 02:26:03) [MSC 32 bit (Intel)] on win32
Type "copyright", "credits" or "license" for more information.
>>> import numarray.testall as testall
>>> testall.test()
numeric:  (0, 1115)
records:  (0, 48)
strings:  (0, 166)
objects:  (0, 72)
memmap:  (0, 75)
\end{verbatim}
Each line in the above output indicates that 0 of X tests failed.  X grows
steadily with each release, so the numbers shown above may not be current.


\subsubsection{Installation on Cygwin}

For an installation of numarray for python running on Cygwin, see section
\ref{sec:installing-unix}.



\section{At the SourceForge...}
\label{sec:at-sourceforge}

The SourceForge project page for numarray is at
\url{http://sourceforge.net/projects/numpy}. On this project page you will find
links to:
\begin{description}
\item[The Numpy Discussion List] You can subscribe to a discussion list about
   numarray using the project page at SourceForge. The list is a good place to
   ask questions and get help. Send mail to
   numpy-discussion@lists.sourceforge.net.  Note that there is no
   numarray-discussion group, we share the list created by the numeric community.

\item[The Web Site] Click on "home page" to get to the Numarray Home Page,
   which has links to documentation and other resources, including tools for
   connecting numarray to Fortran.
\item[Bugs and Patches] Bug tracking and patch-management facilities is
   provided on the SourceForge project page.
\item[CVS Repository] You can get the latest and greatest (albeit less tested
   and trustworthy) version of numarray directly from our CVS repository.
\item[FTP Site] The FTP Site contains this documentation in several formats,
   plus maybe some other goodies we have lying around.
\end{description}



%% Local Variables:
%% mode: LaTeX
%% mode: auto-fill
%% fill-column: 79
%% indent-tabs-mode: nil
%% ispell-dictionary: "american"
%% reftex-fref-is-default: nil
%% TeX-auto-save: t
%% TeX-command-default: "pdfeLaTeX"
%% TeX-master: "numarray"
%% TeX-parse-self: t
%% End:

\chapter{High-Level Overview}
\label{cha:high-level-overview}

\begin{quote} 
   In this chapter, a high-level overview of the extensions is provided, giving
   the reader the definitions of the key components of the system. This section
   defines the concepts used by the remaining sections.
\end{quote}

Numarray makes available a set of universal functions (technically ufunc
objects), used in the same way they were used in Numeric. These are discussed
in some detail in chapter \ref{cha:ufuncs}.


\section{Numarray Objects}
\label{sec:numarray-objects}

The array objects are generally homogeneous collections of potentially large
numbers of numbers. All numbers in a numarray are the same kind (i.e. number
representation, such as double-precision floating point). Array objects must be
full (no empty cells are allowed), and their size is immutable. The specific
numbers within them can change throughout the life of the array, however.
There is a "mask array" package ("MA") for Numeric, which has been ported
to numarray as ``numarray.ma''.

Mathematical operations on arrays return new arrays containing the results of
these operations performed element-wise on the arguments of the operation.

The size of an array is the total number of elements therein (it can be 0 or
more). It does not change throughout the life of the array, unless the array
is explicitly resized using the resize function.

The shape of an array is the number of dimensions of the array and its extent
in each of these dimensions (it can be 0, 1 or more). It can change throughout
the life of the array. In Python terms, the shape of an array is a tuple of
integers, one integer for each dimension that represents the extent in that
dimension.  The rank of an array is the number of dimensions along which it is
defined. It can change throughout the life of the array. Thus, the rank is the
length of the shape (except for rank 0). \note{This is not the same meaning of
rank as in linear algebra.}

Use more familiar mathematicial examples: A vector is a rank-1 array
(it has only one dimension along which it can be indexed). A matrix as used in
linear algebra is a rank-2 array (it has two dimensions along which it can be
indexed). It is possible to create a rank-0 array which is just a scalar of 
one single value --- it has no dimension along which it can be indexed.

The type of an array is a description of the kind of element it contains. It
determines the itemsize of the array.  In contrast to Numeric, an array type in
numarray is an instance of a NumericType class, rather than a single character
code. However, it has been implemented in such a way that one may use aliases,
such as `\constant{u1}', `\constant{i1}', `\constant{i2}', `\constant{i4}',
`\constant{f4}', `\constant{f8}', etc., as well as the original character
codes, to set array types.  The itemsize of an array is the number of 8-bit
bytes used to store a single element in the array. The total memory used by an
array tends to be its size times its itemsize, when the size is large (there
is a fixed overhead per array, as well as a fixed overhead per dimension).

Here is an example of Python code using the array objects:
\begin{verbatim}
>>> vector1 = array([1,2,3,4,5])
>>> print vector1
[1 2 3 4 5]
>>> matrix1 = array([[0,1],[1,3]])
>>> print matrix1
[[0 1]
 [1 3]]
>>> print vector1.shape, matrix1.shape
(5,) (2,2)
>>> print vector1 + vector1
[ 2  4  6  8  10]
>>> print matrix1 * matrix1
[[0 1]                                  # note that this is not the matrix
 [1 9]]                                 # multiplication of linear algebra
\end{verbatim}
If this example complains of an unknown name "array", you forgot to begin
your session with
\begin{verbatim}
>>> from numarray import *
\end{verbatim}
See section \ref{sec:tip:from-numarray-import}.


\section{Universal Functions}
\label{sec:universal-functions}

Universal functions (ufuncs) are functions which operate on arrays and other
sequences. Most ufuncs perform mathematical operations on their arguments, also
elementwise.

Here is an example of Python code using the ufunc objects:
\begin{verbatim}
>>> print sin([pi/2., pi/4., pi/6.])
[ 1. 0.70710678 0.5       ]
>>> print greater([1,2,4,5], [5,4,3,2])
[0 0 1 1]
>>> print add([1,2,4,5], [5,4,3,2])
[6 6 7 7]
>>> print add.reduce([1,2,4,5])
12                                      # 1 + 2 + 4 + 5
\end{verbatim}
Ufuncs are covered in detail in "Ufuncs" on page~\pageref{cha:ufuncs}.


\section{Convenience Functions}
\label{sec:conv-funct}

The numarray module provides, in addition to the functions which are needed to
create the objects above, a set of powerful functions to manipulate arrays,
select subsets of arrays based on the contents of other arrays, and other
array-processing operations.
\begin{verbatim}
>>> data = arange(10)                   # analogous to builtin range()
>>> print data
[0 1 2 3 4 5 6 7 8 9]
>>> print where(greater(data, 5), -1, data)
[ 0  1  2  3  4  5 -1 -1 -1 -1]         # selection facility
>>> data = resize(array([0,1]), (9, 9)) # or just: data=resize([0,1], (9,9))
>>> print data
[[0 1 0 1 0 1 0 1 0]
 [1 0 1 0 1 0 1 0 1]
 [0 1 0 1 0 1 0 1 0]
 [1 0 1 0 1 0 1 0 1]
 [0 1 0 1 0 1 0 1 0]
 [1 0 1 0 1 0 1 0 1]
 [0 1 0 1 0 1 0 1 0]
 [1 0 1 0 1 0 1 0 1]
 [0 1 0 1 0 1 0 1 0]]
\end{verbatim}
All of the functions which operate on numarray arrays are described in chapter
\ref{cha:array-functions}.  See page \pageref{func:where} for more information
about \function{where} and page \pageref{func:resize} for
information on \function{resize}.

\section{Differences between numarray and Numeric.}
\label{sec:diff-numarray-numpy}

This new module numarray was developed for a number of reasons. To 
summarize, we regularly deal with large datasets and numarray gives us the
capabilities that we feel are necessary for working with such datasets. In
particular:
\begin{enumerate}
\item Avoid promotion of array types in expressions involving Python scalars
   (e.g., \code{2.*<Float32 array>} should not result in a \code{Float64}
   array).
\item Ability to use memory mapped files.
\item Ability to access fields in arrays of records as numeric arrays without
   copying the data to a new array.
\item Ability to reference byteswapped data or non-aligned data (as might be
   found in record arrays) without producing new temporary arrays.
\item Reuse temporary arrays in expressions when possible.
\item Provide more convenient use of index arrays (put and take).
\end{enumerate}
We decided to implement a new module since many of the existing Numeric
developers agree that the existing Numeric implementation is not suitable 
for massive changes and enhancements.

This version has nearly the full functionality of the basic Numeric.
\emph{Numarray is not fully compatible with Numeric}.
(But it is very similar in most respects).

The incompatibilities are listed below. 
\begin{enumerate}
\item Coercion rules are different. Expressions involving scalars may not
   produce the same type of arrays.  
\item Types are represented by Type Objects rather than character codes (though
   the old character codes may still be used as arguments to the functions).
\item For versions of Python prior to 2.2, arrays have no public attributes.
   Accessor functions must be used instead (e.g., to get shape for array x, one
   must use x.getshape() instead of x.shape). When using Python 2.2 or later,
   however, the attributes of Numarray are in fact available.
\end{enumerate}
A further comment on type is appropriate here. In numarray, types are
represented by type objects and not character codes. As with Numeric there is a
module variable Float32, but now it represents an instance of a FloatingType
class. For example, if x is a Float32 array, x.type() will return a
FloatingType instance associated with 32-bit floats (instead of using
x.typecode() as is done in Numeric). The following will still work in
numarray, to be backward compatible:
\begin{verbatim}
>>> if x.typecode() == 'f':
\end{verbatim}
or use:
\begin{verbatim}
>>> if x.type() == Float32:
\end{verbatim}
(All examples presume ``\code{from numarray import *}'' has been used instead
of ``\code{import numarray}'', see section \ref{sec:tip:from-numarray-import}.)
The advantage of the new scheme is that other kinds of tests become simpler.
The type classes are hierarchical so one can easily test to see if the array is
an integer array. For example:
\begin{verbatim}
>>> if isinstance(x.type(), IntegralType): 
\end{verbatim}
or:
\begin{verbatim}
>>> if isinstance(x.type(), UnsignedIntegralType):
\end{verbatim}



%% Local Variables:
%% mode: LaTeX
%% mode: auto-fill
%% fill-column: 79
%% indent-tabs-mode: nil
%% ispell-dictionary: "american"
%% reftex-fref-is-default: nil
%% TeX-auto-save: t
%% TeX-command-default: "pdfeLaTeX"
%% TeX-master: "numarray"
%% TeX-parse-self: t
%% End:

\chapter{Array Basics}
\label{cha:array-basics}

\begin{quote} 
   This chapter introduces some of the basic functions which will be used
   throughout the text.
\end{quote}

\section{Basics}
\label{sec:arraybasics:basics}

Before we explore the world of image manipulation as a case-study in array
manipulation, we should first define a few terms which we'll use over and
over again. Discussions of arrays and matrices and vectors can get confusing
due to differences in nomenclature. Here is a brief definition of the terms
used in this tutorial, and more or less consistently in the error messages of
\numarray{}.

The Python objects under discussion are formally called ``\NUMARRAY{}'' (or
even more correctly ``\numarray{}'') objects (N-dimensional arrays), but
informally we'll just call them ``array objects'' or just ``arrays''. These are
different from the array objects defined in the standard Python \module{array}
module (which is an older module designed for processing one-dimensional data
such as sound files).

These array objects hold their data in a fixed length, homogeneous (but not
necessarily contiguous) block of elements, i.e.\ their elements all have the
same C type (such as a 64-bit floating-point number). This is quite different
from most Python container objects, which are variable length heterogeneous
collections.

Any given array object has a \index{rank}rank, which is the number of
``dimensions'' or ``axes'' it has. For example, a point in 3D space \code{[1,
   2, 1]} is an array of rank 1 --- it has one dimension. That dimension has a
length of 3.  As another example, the array
\begin{verbatim}
1.0 0.0 0.0
0.0 1.0 2.0
\end{verbatim}
is an array of rank 2 (it is 2-dimensional). The first dimension has a length
of 2, the second dimension has a length of 3. Because the word ``dimension''
has many different meanings to different folks, in general the word ``axis''
will be used instead. Axes are numbered just like Python list indices: they
start at 0, and can also be counted from the end, so that \code{axis=-1} is the
last axis of an array, \code{axis=-2} is the penultimate axis, etc.  

There are there important and potentially unintuitive behaviors of
\module{numarray} arrays which take some getting used to. The first is that by
default, operations on arrays are performed elementwise.\footnote{This is
common to IDL behavior but contrary to Matlab behavior.}  This means that when
adding two arrays, the resulting array has as elements the pairwise sums of the
two operand arrays.  This is true for all operations, including multiplication.
Thus, array multiplication using the * operator will default to elementwise
multiplication, not matrix multiplication as used in linear algebra. Many
people will want to use arrays as linear algebra-type matrices (including their
\index{rank}rank-1 versions, vectors). For those users, the matrixmultiply
function will be useful.

The second behavior which will catch many users by surprise is that
certain operations, such as slicing, return arrays which are simply different
views of the same data; that is, they will in fact share their data. This will
be discussed at length in examples later.  Now that these definitions and 
warnings are laid out, let's see what we can do with these arrays.

The third behavior which may catch Matlab or Fortran users unaware is the use
of row-major data storage as is done in C.  So while a Fortran array might 
be indexed a[x,y],  numarray is indexed a[y,x].

\newpage
\section{Creating arrays from scratch}
\label{sec:creating-arrays-from}


\subsection{array() and types}
\label{sec:array-types}

\begin{funcdesc}{array}{sequence=None, typecode=None, copy=1, savespace=0,
    type=None, shape=None}
   There are many ways to create arrays. The most basic one is the use of the
   \function{array} function:
\begin{verbatim}
>>> a = array([1.2, 3.5, -1])
\end{verbatim}
   to make sure this worked, do:
\begin{verbatim}
>>> print a
[ 1.2  3.5 -1. ]
\end{verbatim}
   The \function{array} function takes several arguments --- the first one is
   the values, which can be a Python sequence object (such as a list or a
   tuple).  If the optional argument \code{type} is omitted, numarray tries to
   find the best data type which can represent all the elements. 
   
   Since the elements we gave our example were two floats and one integer, it
   chose \class{Float64} as the type of the resulting array. One can specify
   unequivocally the \code{type} of the elements --- this is especially 
   useful when, for example, one wants an array contains floats even
   though all of its input elements are integers:
\begin{verbatim}
>>> x,y,z = 1,2,3
>>> a = array([x,y,z])                  # integers are enough for 1, 2 and 3
>>> print a
[1 2 3]
>>> a = array([x,y,z], type=Float32)    # not the default type
>>> print a
[ 1.  2.  3.]
\end{verbatim}
    Another optional argument is the \code{shape} to use for the array.  When
    passed a \class{NumArray} instance, by default \function{array} will make
    an independent, aligned, contiguous, non-byteswapped copy.  If also passed
    a shape or different type, the resulting ``copy'' will be reshaped or
    cast as the new type.
\end{funcdesc}

\begin{funcdesc}{asarray}{seq, type=None, typecode=None}
   This function converts scalars, lists and tuples to a \class{numarray}, when
   possible. It passes \class{numarray}s through, making copies only to
   convert types.  In any other case a \class{TypeError} is raised.
\end{funcdesc}

\begin{funcdesc}{inputarray}{seq, type=None, typecode=None}
  This is an obosolete alias for \function{asarray}.
\end{funcdesc}


\paragraph*{Important Tip} \label{sec:important-tips} 
Pop Quiz: What will be the type of the array below:
\begin{verbatim}
>>> mystery = array([1, 2.0, -3j])
\end{verbatim}
Hint: -3j is an imaginary number. \\
Answer: Complex64
         
A very common mistake is to call \function{array} with a set of numbers as
arguments, as in \code{array(1, 2, 3, 4, 5)}. This doesn't produce the expected
result if at least two numbers are used, because the first argument to
\function{array} must be the entire data for the array --- thus, in most cases,
a sequence of numbers. The correct way to write the preceding invocation is
most likely \code{array([1, 2, 3, 4, 5])}.

Possible values for the type \index{type argument}argument to the
\function{array} creator function (and indeed to any function which accepts a
so-called type for arrays) are:
\begin{enumerate}
\item Elements that can have values true or false: \index{Bool}\class{Bool}.
\item Unsigned numeric types: \index{UInt8}\class{UInt8},
  \index{UInt16}\class{UInt16}, \index{UInt32}\class{UInt32}, and
  \index{UInt64}\class{UInt64}\footnote[1]{UInt64 is unsupported on Windows}.
\item Signed numeric types: 
   \begin{itemize}
   \item Signed integer choices: \index{Int8}\class{Int8},
      \index{Int16}\class{Int16}, \index{Int32}\class{Int32}, \index{Int64}\class{Int64}.
   \item Floating point choices: \index{Float32}\class{Float32},
      \index{Float64}\class{Float64}.
   \end{itemize}
\item Complex number types: \index{Complex32}\class{Complex32},
   \index{Complex64}\class{Complex64}.
\end{enumerate}

To specify a type, e.g. \class{UInt8}, etc, the easiest method is just to
specify it as a string:
\begin{verbatim}
a = array([10], type = 'UInt8')
\end{verbatim}

The various means for specifying types are defined in table
\ref{tab:type-specifiers}, with each item in a row being equivalent.  The
\emph{preferred} methods are in the first 3 columns: numarray type object, type
string, or type code.  The last two columns were added for backwards
compatabililty with Numeric and are not recommended for new code.  Numarray
type object and string names denote the size of the type in bits.  The numarray
type code names denote the size of the type in bytes.  The type objects must be
imported from or referenced via the numerictypes module.  All type strings and
type codes are specified using ordinary Python strings, and hence don't require
an import.  Complex type names denote the size of one component, real or
imaginary, in bits/bytes, but the letter code is the total size of the 
whole number ('c8' and 'c16').

\begin{table}[h]
  \centering
  \caption{Type specifiers}
  \label{tab:type-specifiers}
  \begin{tabular}{|l|l|l|l|l|}
    \hline
    Numarray Type&Numarray String&Numarray Code&Numeric String&Numeric Code\\
    \hline
    Int8&'Int8'&'i1'&'Byte'&'1'\\
    \hline
    UInt8&'UInt8'&'u1'&'UByte'& \\
    \hline
    Int16&'Int16'&'i2'&'Short'&'s'\\
    \hline
    UInt16&'UInt16'&'u2'&'UShort'& \\
    \hline
    Int32&'Int32'&'i4'&'Int'&'i'\\
    \hline
    UInt32&'UInt32'&'u4'&'UInt'&'u'\\
    \hline
    Int64&'Int64'&'i8'& & \\
    \hline
    UInt64\footnotemark[1]&'UInt64'&'u8'& & \\
    \hline
    Float32&'Float32'&'f4'&'Float'&'f'\\
    \hline
    Float64&'Float64'&'f8'&'Double'&'d'\\
    \hline
    Complex32&'Complex32'&'c8'& &'F'\\
    \hline
    Complex64&'Complex64'&'c16'&'Complex'&'D'\\
    \hline
    Bool&'Bool'& & & \\
    \hline
  \end{tabular}
\end{table}

\subsection{Multidimensional Arrays}
\label{sec:multi-dim-arrays}

The following example shows one way of creating \index{multidimensional
   arrays}multidimensional arrays:
\begin{verbatim}
>>> ma = array([[1,2,3],[4,5,6]])
>>> print ma
[[1 2 3]
 [4 5 6]]
\end{verbatim}
The first argument to \function{array} in the code above is a single
\class{list} containing two lists, each containing three elements. If we wanted
floats instead, we could specify, as discussed in the previous section, the
optional type we wished:
\begin{verbatim}
>>> ma_floats = array([[1,2,3],[4,5,6]], type=Float32)
>>> print ma_floats
[[ 1.  2.  3.]
 [ 4.  5.  6.]]
\end{verbatim}
This array allows us to introduce the notion of \index{shape}``shape''. The
shape of an array is the set of numbers which define its dimensions. The shape
of the array \var{ma} defined above is 2 by 3. More precisely, all arrays have
an attribute which is a tuple of integers giving the shape. The
\index{getshape}\method{getshape} method returns this tuple.  In general, one
can directly use the \member{shape} attribute (but only for Python 2.2 and
later) to get or set its value. Since it isn't supported for earlier versions
of Python, subsequent examples will use \method{getshape} and
\index{setshape}\method{setshape} only. So, in this case:
\begin{verbatim}
>>> print ma.shape                      # works only with Python 2.2 or later
>>> print ma.getshape()                 # works with all Python versions
(2, 3)
\end{verbatim}
Using the earlier definitions, this is a shape of \index{rank}rank 2, where the
first axis has length 2, and the second axis has length 3. The rank of an array
\code{A} is always equal to \code{len(A.getshape())}.  Note that shape is an
attribute and \method{getshape} is a method of array objects. They are the
first of several that we will see throughout this tutorial. If you're not used
to object-oriented programming, you can think of attributes as ``features'' or
``qualities'' of individual arrays, and methods are functions that operate on
individual arrays.  The relation between an array and its shape is similar to
the relation between a person and their hair color. In Python, it's called an
object/attribute relation.

\begin{funcdesc}{reshape}{a, shape}
   What if one wants to change the dimensions of an array? For now, let us
   consider changing the shape of an array without making it ``grow''. Say, for
   example, we want to make the 2x3 array defined above (\var{ma}) an array of
   rank 1:
\begin{verbatim}
>>> flattened_ma = reshape(ma, (6,))
>>> print flattened_ma
[1 2 3 4 5 6]
\end{verbatim}
   One can change the shape of arrays to any shape as long as the product of
   all the lengths of all the axes is kept constant (in other words, as long as
   the number of elements in the array doesn't change):
\begin{verbatim}
>>> a = array([1,2,3,4,5,6,7,8])
>>> print a
[1 2 3 4 5 6 7 8]
>>> b = reshape(a, (2,4))               # 2*4 == 8
>>> print b
[[1 2 3 4]
 [5 6 7 8]]
>>> c = reshape(b, (4,2))               # 4*2 == 8
>>> print c
[[1 2]
 [3 4]
 [5 6]
 [7 8]]
\end{verbatim}
   The function \function{reshape} expects an array/sequence as its 
   first argument, and a shape as its second argument.
   The shape has to be a sequence of integers (a \class{list} or a
   \class{tuple}).  There is also a \method{setshape}
   method, which changes the shape of an array in-place (see below).
   
   One nice feature of shape tuples is that one entry in the shape tuple is
   allowed to be -1. The -1 will be automatically replaced by whatever number
   is needed to build a shape which does not change the size of the array.
   Thus:
\begin{verbatim}
>>> a = reshape(array(range(25)), (5,-1))
>>> print a, a.getshape()
[[ 0  1  2  3  4]
 [ 5  6  7  8  9]
 [10 11 12 13 14]
 [15 16 17 18 19]
 [20 21 22 23 24]] (5, 5)
\end{verbatim}
   The \member{shape} of an array is a modifiable attribute of the array, but
   it is an internal attribute. You can change the shape of an array by calling
   the \method{setshape} method (or by assigning a \class{tuple} to the shape
   attribute, in Python 2.2 and later), which assigns a new shape to the array:
\begin{verbatim}
>>> a = array([1,2,3,4,5,6,7,8,9,10])
>>> a.getshape()
(10,)
>>> a.setshape((2,5))
>>> a.shape = (2,5)                     # for Python 2.2 and later
>>> print a
[[ 1  2  3  4  5]
 [ 6  7  8  9 10]]
>>> a.setshape((10,1))                  # second axis has length 1
>>> print a
[[ 1]
 [ 2]
 [ 3]
 [ 4]
 [ 5]
 [ 6]
 [ 7]
 [ 8]
 [ 9]
 [10]]
>>> a.setshape((5,-1))                  # note the -1 trick described above
>>> print a
[[ 1  2]
 [ 3  4]
 [ 5  6]
 [ 7  8]
 [ 9 10]]
\end{verbatim}
   As in the rest of Python, violating rules (such as the one about which
   shapes are allowed) results in exceptions:
\begin{verbatim}
>>> a.setshape((6,-1))
Traceback (innermost last):
  File "<stdin>", line 1, in ?
ValueError: New shape is not consistent with the old shape
\end{verbatim}
\end{funcdesc}


\paragraph*{For Advanced Users: Printing arrays}

\begin{quote}
   Sections denoted ``For Advanced Users'' indicates 
   function aspects which may not be needed for a first introduction of
   \numarray{}, but is mentioned for the sake of completeness.
\end{quote}

The default \index{printing arrays}printing routine provided by the
\module{\numarray} module prints arrays as follows:
\begin{enumerate}
\item The last axis is always printed left to right.
\item The next-to-last axis is printed top to bottom.
\end{enumerate}
The remaining axes are printed top to bottom with increasing numbers of
separators.

This explains why rank-1 arrays are printed from left to right, rank-2 arrays
have the first dimension going down the screen and the second dimension going
from left to right, etc.

If you want to change the shape of an array so that it has more elements than
it started with (i.e. grow it), then you have several options: One solution is
to use the \index{concatenate}\function{concatenate} function discussed later.
\begin{verbatim}
>>> print a
[0 1 2 3 4 5 6 6 7]
>>> print concatenate([[a],[a]])
>>> print b
[[0 1 2 3 4 5 6 7]
 [0 1 2 3 4 5 6 7]]
>>> print b.getshape()
(2, 8)
\end{verbatim}


\begin{funcdesc}{resize}{array, shape}
   A final possibility is the \function{resize} function, which takes a
   \var{base} array as its first argument and the desired \var{shape} as the
   second argument.  Unlike \function{reshape}, the shape argument to
   \function{resize} can be a smaller or larger shape than the input
   array. Smaller shapes will result in arrays with the data at the
   ``beginning'' of the input array, and larger shapes result in arrays with
   data containing as many replications of the input array as are needed to
   fill the shape. For example, starting with a simple array
\begin{verbatim}
>>> base = array([0,1])
\end{verbatim}
   one can quickly build a large array with replicated data:
\begin{verbatim}
>>> big = resize(base, (9,9))
>>> print big
[[0 1 0 1 0 1 0 1 0]
 [1 0 1 0 1 0 1 0 1]
 [0 1 0 1 0 1 0 1 0]
 [1 0 1 0 1 0 1 0 1]
 [0 1 0 1 0 1 0 1 0]
 [1 0 1 0 1 0 1 0 1]
 [0 1 0 1 0 1 0 1 0]
 [1 0 1 0 1 0 1 0 1]
 [0 1 0 1 0 1 0 1 0]]
\end{verbatim}
\end{funcdesc}

\newpage
\section{Creating arrays with values specified ``on-the-fly''}
\label{sec:creating-arrays-on-the-fly}

\begin{funcdesc}{zeros}{shape, type}
\end{funcdesc}
\begin{funcdesc}{ones}{shape, type}
   Often, one needs to manipulate arrays filled with numbers which aren't
   available beforehand. The \module{\numarray} module provides a few functions
   which create arrays from scratch: \function{zeros} and \function{ones}
   simply create arrays of a given \var{shape} filled with zeros and ones
   respectively:
\begin{verbatim}
>>> z = zeros((3,3))
>>> print z
[[0 0 0]
 [0 0 0]
 [0 0 0]]
>>> o = ones([2,3])
>>> print o
[[1 1 1]
 [1 1 1]]
\end{verbatim}
   Note that the first argument is a shape --- it needs to be a \class{tuple} or
   a \class{list} of integers. Also note that the default type for the
   returned arrays is \class{Int}, which you can override, e. g.: 
\begin{verbatim}
>>> o = ones((2,3), Float32)
>>> print o
[[ 1.  1.  1.]
 [ 1.  1.  1.]]
\end{verbatim}
\end{funcdesc}


\begin{funcdesc}{arrayrange}{a1, a2=None, stride=1, type=None, shape=None}
\end{funcdesc}
\begin{funcdesc}{arange}{a1, a2=None, stride=1, type=None, shape=None}
   The \function{arange} function is similar to the \function{range} function
   in Python, except that it returns an \class{array} as opposed to a
   \class{list}.
   \function{arange} and \function{arrayrange} are equivalent.
\begin{verbatim}
>>> r = arange(10)
>>> print r
[0 1 2 3 4 5 6 7 8 9]
\end{verbatim}
   Combining the \function{arange} with the \function{reshape} function, we can
   get:
\begin{verbatim}
>>> big = reshape(arange(100),(10,10))
>>> print big
[[ 0  1  2  3  4  5  6  7  8  9]
 [10 11 12 13 14 15 16 17 18 19]
 [20 21 22 23 24 25 26 27 28 29]
 [30 31 32 33 34 35 36 37 38 39]
 [40 41 42 43 44 45 46 47 48 49]
 [50 51 52 53 54 55 56 57 58 59]
 [60 61 62 63 64 65 66 67 68 69]
 [70 71 72 73 74 75 76 77 78 79]
 [80 81 82 83 84 85 86 87 88 89]
 [90 91 92 93 94 95 96 97 98 99]]
\end{verbatim}
   One can set the \code{a1}, \code{a2}, and \code{stride} arguments, which 
   allows for more varied ranges:
\begin{verbatim}
>>> print arange(10,-10,-2)
[10  8  6  4  2  0  -2  -4  -6  -8]
\end{verbatim}
   An important feature of arange is that it can be used with non-integer
   starting points and strides:
\begin{verbatim}
>>> print arange(5.0)
[ 0. 1. 2. 3. 4.]
>>> print arange(0, 1, .2)
[ 0.   0.2  0.4  0.6  0.8]
\end{verbatim}
   If you want to create an array with just one value, repeated over and over,
   you can use the * operator applied to lists
\begin{verbatim}
>>> a = array([[3]*5]*5)
>>> print a
[[3 3 3 3 3]
 [3 3 3 3 3]
 [3 3 3 3 3]
 [3 3 3 3 3]
 [3 3 3 3 3]]
\end{verbatim}
   but that is relatively slow, since the duplication is done on Python lists.
   A quicker way would be to start with 0's and add 3:
\begin{verbatim}
         >>> a = zeros([5,5]) + 3
         >>> print a
         [[3 3 3 3 3]
          [3 3 3 3 3]
          [3 3 3 3 3]
          [3 3 3 3 3]
          [3 3 3 3 3]]
\end{verbatim}
   The optional \code{type} argument forces the type of the resulting array,
   which is otherwise the ``highest'' of the \code{a1}, \code{a2}, and 
   \code{stride} arguments.  The \code{a1} argument defaults to 0 if not 
   specified. Note that if the specified \code{type} is
   is ``lower'' than what \function{arange} would
   normally use, the array is the result of a precision-losing cast (a
   round-down, as that used in the \method{astype} method for arrays.)
\end{funcdesc}


\subsection{Creating an array from a function}
\label{sec:creating-array-from-function}

\begin{funcdesc}{fromfunction}{object, shape} 
   Finally, one may want to create an array whose elements are the result
   of a function evaluation. This is done using the \function{fromfunction}
   function, which takes two arguments, a \var{shape} and a callable
   \var{object} (usually a function).  For example:
\begin{verbatim}
>>> def dist(x,y):
...   return (x-5)**2+(y-5)**2          # distance from (5,5) squared
...
>>> m = fromfunction(dist, (10,10))
>>> print m
[[50 41 34 29 26 25 26 29 34 41]
 [41 32 25 20 17 16 17 20 25 32]
 [34 25 18 13 10  9 10 13 18 25]
 [29 20 13  8  5  4  5  8 13 20]
 [26 17 10  5  2  1  2  5 10 17]
 [25 16  9  4  1  0  1  4  9 16]
 [26 17 10  5  2  1  2  5 10 17]
 [29 20 13  8  5  4  5  8 13 20]
 [34 25 18 13 10  9 10 13 18 25]
 [41 32 25 20 17 16 17 20 25 32]]
>>> m = fromfunction(lambda i,j,k: 100*(i+1)+10*(j+1)+(k+1), (4,2,3))
>>> print m
[[[111 112 113]
  [121 122 123]]
 [[211 212 213]
  [221 222 223]]
 [[311 312 313]
  [321 322 323]]
 [[411 412 413]
  [421 422 423]]]
\end{verbatim}
   These examples show that \function{fromfunction}
   creates an array of the shape specified by its second argument, and with the
   contents corresponding to the value of the function argument (the first
   argument) evaluated at the indices of the array. Thus the value of
   \code{m[3, 4]} in the first example above is the value of dist when
   \code{x=3} and \code{y=4}.  Similarly for the lambda function in the second
   example, but with a rank-3 array.  The implementation of
   \function{fromfunction} consists of:
\begin{verbatim}
def fromfunction(function, dimensions):
    return apply(function, tuple(indices(dimensions)))
\end{verbatim}
   which means that the function \function{function} is called with arguments
   given by the sequence \code{indices(dimensions)}. As described in the
   definition of indices, this consists of arrays of indices which will be of
   rank one less than that specified by dimensions. This means that the
   function argument must accept the same number of arguments as there are
   dimensions in \var{dimensions}, and that each argument will be an array of
   the same shape as that specified by dimensions.  Furthermore, the array
   which is passed as the first argument corresponds to the indices of each
   element in the resulting array along the first axis, that which is passed as
   the second argument corresponds to the indices of each element in the
   resulting array along the second axis, etc. A consequence of this is that
   the function which is used with \function{fromfunction} will work as
   expected only if it performs a separable computation on its arguments, and
   expects its arguments to be indices along each axis. Thus, no logical
   operation on the arguments can be performed, or any non-shape preserving
   operation. Thus, the following will not work as expected:
\begin{verbatim}
>>> def buggy(test):
...     if test > 4: return 1
...     else: return 0
...
>>> print fromfunction(buggy,(10,))
Traceback (most recent call last):
...
RuntimeError: An array doesn't make sense as a truth value.  Use any(a) or
all(a).
\end{verbatim}
The reason \function{buggy()} failed is that indices((10,)) results in an array
passed as \var{test}.  The result of comparing \var{test} with 4 is also an
array which has no unambiguous meaning as a truth value.

Here is how to do it properly. We add a print statement to the
   function for clarity:
\begin{verbatim}
>>> def notbuggy(test):                 # only works in Python 2.1 & later
...     print test
...     return where(test>4,1,0)
...
>>> fromfunction(notbuggy,(10,))
[0 1 2 3 4 5 6 7 8 9]
array([0, 0, 0, 0, 0, 1, 1, 1, 1, 1])
\end{verbatim}
   We leave it as an excercise for the reader to figure out why the ``buggy''
   example gave the result 1.
\end{funcdesc}


\begin{funcdesc}{identity}{size}
   The \function{identity} function takes a single integer argument and returns
   a square identity array (in the ``matrix'' sense) of that \var{size} of
   integers:
\begin{verbatim}
      >>> print identity(5)
      [[1 0 0 0 0]
       [0 1 0 0 0]
       [0 0 1 0 0]
       [0 0 0 1 0]
       [0 0 0 0 1]]
\end{verbatim}
\end{funcdesc}



\newpage
\section{Coercion and Casting}
\label{sec:coercion-casting}

We've mentioned the types of arrays, and how to create arrays with the right
type.  But what happens when arrays with different types interact?  For 
some operations, the behavior of \numarray{} is significantly different 
from Numeric.

\subsection{Automatic Coercions and Binary Operations}
\label{sec:automatic-coercion-binary-casting}

In \numarray{} (in contrast to Numeric), there is now a distinction between how
coercion is treated in two basic cases: array/scalar operations and array/array
operations. In the array/array case, the coercion rules are nearly identical to
those of Numeric, the only difference being combining signed and unsigned
integers of the same size.  The array/array result types are enumerated in
table \ref{tab:array-array-result-types}.
\begin{table}[h]
\footnotesize
\centering
\caption{Array/Array Result Types}
\label{tab:array-array-result-types}
\begin{tabular}{|l|l|l|l|l|l|l|l|l|l|l|l|l|l|}
\hline
 &Bool&Int8&UInt8&Int16&UInt16&Int32&UInt32&Int64&UInt64&Float32&Float64&Complex32&Complex64\\
\hline
Bool&Int8&Int8&UInt8&Int16&UInt16&Int32&UInt32&Int64&UInt64&Float32&Float64&Complex32&Complex64\\
\hline
Int8& &Int8&Int16&Int16&Int32&Int32&Int64&Int64&Int64&Float32&Float64&Complex32&Complex64\\
\hline
UInt8& & &UInt8&Int16&UInt16&Int32&UInt32&Int64&UInt64&Float32&Float64&Complex32&Complex64\\
\hline
Int16& & & &Int16&Int32&Int32&Int64&Int64&Int64&Float32&Float64&Complex32&Complex64\\
\hline
UInt16& & & & &UInt16&Int32&UInt32&Int64&UInt64&Float32&Float64&Complex32&Complex64\\
\hline
Int32& & & & & &Int32&Int64&Int64&Int64&Float32&Float64&Complex32&Complex64\\
\hline
UInt32& & & & & & &UInt32&Int64&UInt64&Float32&Float64&Complex32&Complex64\\
\hline
Int64& & & & & & & &Int64&Int64&Float64&Float64&Complex64&Complex64\\
\hline
UInt64& & & & & & & & &UInt64&Float64&Float64&Complex64&Complex64\\
\hline
Float32& & & & & & & & & &Float32&Float64&Complex32&Complex64\\
\hline
Float64& & & & & & & & & & &Float64&Complex64&Complex64\\
\hline
Complex32& & & & & & & & & & & &Complex32&Complex64\\
\hline
Complex64& & & & & & & & & & & & &Complex64\\
\hline
\end{tabular}
\end{table}

Scalars, however, are treated differently. If the scalar is of the same
``kind'' as the array (for example, the array and scalar are both integer
types) then the output is the type of the array, even if it is of a normally
``lower'' type than the scalar.  Adding an \class{Int16} array with an integer
scalar results in an \class{Int16} array, not an \class{Int32} array as is the
case in Numeric.  Likewise adding a \class{Float32} array to a float scalar
results in a \class{Float32} array rather than a \class{Float64} array as is
the case with Numeric.  Adding an \class{Int16} array and a float scalar will
result in a \class{Float64} array, however, since the scalar is of a higher
kind than the array.  Finally, when scalars and arrays are operated on
together, the scalar is converted to a rank-0 array first.  Thus, adding a
``small'' integer to a ``large'' floating point array is equivalent to first
casting the integer ``up'' to the type of the array.
\begin{verbatim}
>>> print (array ((1, 2, 3), type=Int16) * 2).type()
numarray type: Int16
>>> arange(0, 1.0, .1) + 12
array([ 12. , 12.1, 12.2, 12.3, 12.4, 12.5, 12.6, 12.7, 12.8, 12.9]
\end{verbatim}

The results of array/scalar operations are enumerated in table
\ref{tab:Array-Scalar-Result-Types}.  Entries marked with " are identical to
their neighbors on the same row.
\begin{table}[h]
\footnotesize
\centering
\caption{Array/Scalar Result Types}
\label{tab:Array-Scalar-Result-Types}
\begin{tabular}{|l|l|l|l|l|l|l|l|l|l|l|l|l|l|}
\hline
 &Bool&Int8&UInt8&Int16&UInt16&Int32&UInt32&Int64&UInt64&Float32&Float64&Complex32&Complex64\\
\hline
int&Int32&Int8&UInt8&Int16&UInt16&Int32&UInt32&Int64&UInt64&Float32&Float64&Complex32&Complex64\\
\hline
long&Int32&Int8&UInt8&Int16&UInt16&Int32&UInt32&Int64&UInt64&Float32&Float64&Complex32&Complex64\\
\hline
float&Float64&"&"&"&"&"&"&"&Float64&Float32&Float64&Complex32&Complex64\\
\hline
complex&Complex64&"&"&"&"&"&"&"&"&"&"&"&Complex64\\
\hline
\end{tabular}
\end{table}

\footnotetext[10]{Float64}
\footnotetext[20]{Complex64}

\subsection{The type value table}
\label{sec:type-value-table}

The type identifiers (\class{Float32}, etc.) are \class{NumericType} instances.
The mapping between type and the equivalent C variable is machine dependent.
The correspondences between types and C variables for 32-bit architectures are
shown in Table \ref{tab:type-identifiers}.

\begin{table}[h]
   \centering
   \caption{Type identifier table on a x86 computer.}
   \label{tab:type-identifiers}
   \begin{tabular}{ccl}
      \# of bytes & \# of bits      & Identifier \\
           1      &       8         &   Bool \\
           1      &       8         &   Int8 \\
           1      &       8         &   UInt8 \\
           2      &       16        &   Int16 \\
           2      &       16        &   UInt16 \\
           4      &       32        &   Int32 \\
           4      &       32        &   UInt32 \\
           8      &       64        &   Int64 \\
           8      &       64        &   UInt64 \\
           4      &       32        &   Float32 \\
           8      &       64        &   Float64 \\
           8      &       64        &   Complex32 \\
           16     &      128        &   Complex64 
   \end{tabular}
\end{table}

\subsection{Long: the platform relative type}
The type identifier \class{Long} is aliased to either \class{Int32} or
\class{Int64}, depending on the machine architecture where numarray is
installed.  On 32-bit platforms, \class{Long} is defined as \class{Int32}.  On
64-bit (LP64) platforms, \class{Long} is defined as \class{Int64}. \class{Long}
is used as the default integer type for arrays and for index values, such as
those returned by \function{nonzero}.  

\subsection{Deliberate casts (potentially down)}
\label{sec:deliberate-casts}

\begin{methoddesc}{astype}{type}
   You may also force \module{numarray} to cast any number array to another
   number array.  For example, to take an array of any numeric type
   (\class{IntX} or \class{FloatX} or \class{ComplexX} or \class{UIntX}) and
   convert it to a 64-bit float, one can do:
\begin{verbatim}
>>> floatarray = otherarray.astype(Float64)
\end{verbatim}
   The \var{type} can be any of the number types, ``larger'' or ``smaller''. If
   it is larger, this is a cast-up. If it is smaller, the standard casting
   rules of the underlying language (C) are used, which means that truncation
   or integer wrap-around can occur:
\begin{verbatim}
>>> print x
[   0.     0.4    0.8    1.2  300.6]
>>> print x.astype(Int32)
[  0   0   0   1 300]
>>> print x.astype(Int8)      # wrap-around
[ 0  0  0  1 44]
\end{verbatim}
   If the \var{type} used with \method{astype} is the original array's type,
   then a copy of the original array is returned.
\end{methoddesc}


\newpage
\section{Operating on Arrays}
\label{sec:operating-arrays}

\subsection{Simple operations}
\label{sec:simple-operations}

If you have a keen eye, you have noticed that some of the previous examples did
something new: they added a number to an array. Indeed, most Python operations
applicable to numbers are directly applicable to arrays:
\begin{verbatim}
>>> print a
[1 2 3]
>>> print a * 3
[3 6 9]
>>> print a + 3
[4 5 6]
\end{verbatim}
Note that the mathematical operators behave differently depending on the types
of their operands. When one of the operands is an array and the other a
number, the number is added to all the elements of the array, and the resulting
array is returned. This is called \index{broadcasting}\var{broadcasting}. 
This also occurs for unary mathematical operations such as sine and the 
negative sign:
\begin{verbatim}
>>> print sin(a)
[ 0.84147096 0.90929741 0.14112 ]
>>> print -a
[-1 -2 -3]
\end{verbatim}
When both elements are arrays of the same shape, then a new array is created,
where each element is the operation result of the corresponding elements in 
the original arrays:
\begin{verbatim}
>>> print a + a
[2 4 6]
\end{verbatim}
If the operands of operations such as addition, are arrays having the same
rank but different dimensions, then an exception is generated:
\begin{verbatim}
>>> a = array([1,2,3])
>>> b = array([4,5,6,7])                # note this has four elements
>>> print a + b
Traceback (innermost last):
  File "<stdin>", line 1, in ?
ValueError: Arrays have incompatible shapes
\end{verbatim}
This is because there is no reasonable way for numarray to interpret addition
of a \code{(3,)} shaped array and a \code{(4,)} shaped array.

Note what happens when adding arrays with different rank:
\begin{verbatim}
>>> print a
[1 2 3]
>>> print b
[[ 4  8 12]
 [ 5  9 13]
 [ 6 10 14]
 [ 7 11 15]]
>>> print a + b
[[ 5 10 15]
 [ 6 11 16]
 [ 7 12 17]
 [ 8 13 18]]
\end{verbatim}
This is another form of \index{broadcasting}broadcasting. To understand this,
one needs to look carefully at the shapes of \code{a} and \code{b}:
\begin{verbatim}
>>> a.getshape()
(3,)
>>> b.getshape()
(4,3)
\end{verbatim}
Note that the last axis of \code{a} is the same length as that of \code{b}
(i.e.\ compare the last elements in their shape tuples).  Because \code{a}'s
and \code{b}'s last dimensions both have length 3, those two dimensions were
``matched'', and a new dimension was created and automatically ``assumed'' for
array \code{a}. The data already in \code{a} were ``replicated'' as many 
times as needed (4, in this case) to make the shapes of the two operand 
arrays conform. This
replication (\index{broadcasting}broadcasting) occurs when arrays are operands
to binary operations and their shapes differ, based on the following algorithm:
\begin{itemize}
\item starting from the last axis, the axis lengths (dimensions) of the
   operands are compared,
\item if both arrays have axis lengths greater than 1, but the lengths differ,
   an exception is raised,
\item if one array has an axis length greater than 1, then the other array's
   axis is ``stretched'' to match the length of the first axis; if the other
   array's axis is not present (i.e., if the other array has smaller rank),
   then a new axis of the same length is created.
\end{itemize}

Operands with the following shapes will work:
\begin{verbatim}
(3, 2, 4) and (3, 2, 4)
(3, 2, 4) and (2, 4)
(3, 2, 4) and (4,)
(2, 1, 2) and (2, 2)
\end{verbatim}

But not these:
\begin{verbatim}
(3, 2, 4) and (2, 3, 4)
(3, 2, 4) and (3, 4)
(4,) and (0,)
(2, 1, 2) and (0, 2)
\end{verbatim}

This algorithm is complex to describe, but intuitive in practice.


\subsection{In-place operations}
\label{sec:inplace-operations}

Beginning with Python 2.0, Python supports the in-place operators
\index{+=}\code{+=}, \index{+=}\code{-=}, \index{*=}\code{*=}, and
\index{/=}\code{/=}. \Numarray{} supports these operations, but you need to be
careful. The right-hand side should be of the same type. Some violation of this
is possible, but in general contortions may be necessary for using the smaller
``kinds'' of types.
\begin{verbatim}
>>> x = array ([1, 2, 3], type=Int16)
>>> x += 3.5
>>> print x
[4 5 6]
\end{verbatim}


%% Local Variables:
%% mode: LaTeX
%% mode: auto-fill
%% fill-column: 79
%% indent-tabs-mode: nil
%% ispell-dictionary: "american"
%% reftex-fref-is-default: nil
%% TeX-auto-save: t
%% TeX-command-default: "pdfeLaTeX"
%% TeX-master: "numarray"
%% TeX-parse-self: t
%% End:

\chapter{Array Indexing}
\label{cha:array-indexing}

This chapter discusses the rich and varied ways of indexing numarray
objects to specify individual elements, sub-arrays, sub-samplings, and
even random collections of elements.

\section{Getting and Setting array values}
\label{sec:get-set-array-values}

Just like other Python sequences, array contents are manipulated with the
\code{[]} notation. For rank-1 arrays, there are no differences between list
and array notations:
\begin{verbatim}
>>> a = arange(10)
>>> print a[0]                          # get first element
0
>>> print a[1:5]                        # get second through fifth elements
[1 2 3 4]
>>> print a[-1]                         # get last element
9
>>> print a[:-1]                        # get all but last element
[0 1 2 3 4 5 6 7 8]
\end{verbatim}
If an array is multidimensional (of rank > 1), then specifying a single 
integer index will return an array of
dimension one less than the original array.

\begin{verbatim}
>>> a = arange(9, shape=(3,3))
>>> print a
[[0 1 2]
 [3 4 5]
 [6 7 8]]
>>> print a[0]                          # get first row, not first element!
[0 1 2]
>>> print a[1]                          # get second row
[3 4 5]
\end{verbatim}
To get to individual elements in a rank-2 array, one specifies both indices
separated by commas:
\begin{verbatim}
>>> print a[0,0]                        # get element at first row, first column
0
>>> print a[0,1]                        # get element at first row, second column
1
>>> print a[1,0]                        # get element at second row, first column
3
>>> print a[2,-1]                       # get element at third row, last column
8
\end{verbatim}
Of course, the \code{[]} notation can be used to set values as well:
\begin{verbatim}
>>> a[0,0] = 123
>>> print a
[[123   1   2]
 [  3   4   5]
 [  6   7   8]]
\end{verbatim}
Note that when referring to rows, the right hand side of the equal sign needs
to be a sequence which ``fits'' in the referred array subset, as described 
by the broadcast rule (in the code sample below, a 3-element row):
\begin{verbatim}
>>> a[1] = [10,11,12] ; print a
[[123   1   2]
 [ 10  11  12]
 [  6   7   8]]
>>> a[2] = 99 ; print a
[[123   1   2]
 [ 10  11  12]
 [ 99  99  99]]
\end{verbatim}

Note also that when assigning floating point values to integer arrays that
the values are silently truncated:
\begin{verbatim}
>>> a[1] = 93.999432
[[123   1   2]
 [ 93  93  93]
 [ 99  99  99]]
\end{verbatim}

\newpage
\section{Slicing Arrays}
\label{sec:slicing-arrays}

The standard rules of Python slicing apply to arrays, on a per-dimension basis.
Assuming a 3x3 array:
\begin{verbatim}
>>> a = reshape(arange(9),(3,3))
>>> print a
[[0 1 2]
 [3 4 5]
 [6 7 8]]
\end{verbatim}
The plain \code{[:]} operator slices from beginning to end:
\begin{verbatim}
>>> print a[:,:]
[[0 1 2]
 [3 4 5]
 [6 7 8]]
\end{verbatim}
In other words, \code{[:]} with no arguments is the same as \code{[:]} for
lists --- it can be read ``all indices along this axis''.  (Actually, there is
an important distinction; see below.) So, to get the second row along the
second dimension:
\begin{verbatim}
>>> print a[:,1]
[1 4 7]
\end{verbatim}
Note that what was a ``column'' vector is now a ``row'' vector.  Any ``integer
slice'' (as in the 1 in the example above) results in a returned array with
rank one less than the input array.  

There is one important distinction between
slicing arrays and slicing standard Python sequence objects. A slice of a
\class{list} is a new copy of that subset of the \class{list}; a slice of an
array is just a view into the data of the first array.  To force a copy, you
can use the \function{copy} method. For example:
\begin{verbatim}
>>> a = arange (20)
>>> b = a[3:8]
>>> c = a[3:8].copy()
>>> a[5] = -99
>>> print b
[  3   4 -99   6   7]
>>> print c
[3 4 5 6 7]
\end{verbatim}
If one does not specify as many slices as there are dimensions in an array,
then the remaining slices are assumed to be ``all''. If \var{A} is a rank-3
array, then
\begin{verbatim}
A[1] == A[1,:] == A[1,:,:]
\end{verbatim}
An additional slice notation for arrays which does not exist for Python
lists (before Python 2.3), i. e. the optional third argument, meaning 
the ``step size'', also called \index{stride}stride or increment. Its 
default value is 1, meaning return every element in the specified range.  
Alternate values allow one to skip some of the elements in the slice:
\begin{verbatim}
>>> a = arange(12)
>>> print a
[ 0  1  2  3  4  5  6  7  8  9 10 11]
>>> print a[::2]                        # return every *other* element
[ 0  2  4  6  8 10]
\end{verbatim}
\index{stride!Negative}Negative strides are allowed as long as the starting
index is greater than the stopping index:
\begin{verbatim}
>>> a = reshape(arange(9),(3,3))                                                                                          Array Basics
>>> print a
[[0 1 2]
 [3 4 5]
 [6 7 8]]
>>> print a[:, 0]
[0 3 6]
>>> print a[0:3, 0]
[0 3 6]
>>> print a[2::-1, 0]
[6 3 0]
\end{verbatim}
If a negative stride is specified and the starting or stopping indices are
omitted, they default to ``end of axis'' and ``beginning of axis''
respectively.  Thus, the following two statements are equivalent for the array
given:
\begin{verbatim}
>>> print a[2::-1, 0]
[6 3 0]
>>> print a[::-1, 0]
[6 3 0]
>>> print a[::-1]                       # this reverses only the first axis
[[6 7 8]
 [3 4 5]
 [0 1 2]]
>>> print a[::-1,::-1]                  # this reverses both axes
[[8 7 6]
 [5 4 3]
 [2 1 0]]
\end{verbatim}
One final way of slicing arrays is with the keyword \samp{...} This keyword is
somewhat complicated. It stands for ``however many `:' I need depending on the
rank of the object I'm indexing, so that the indices I \emph{do} specify are at
the end of the index list as opposed to the usual beginning''.

So, if one has a rank-3 array \var{A}, then \code{A[...,0]} is the same thing
as \code{A[:,:,0]}, but if \var{B} is rank-4, then \code{B[...,0]} is the same
thing as: \code{B[:,:,:,0]}. Only one \samp{...} is expanded in an index
expression, so if one has a rank-5 array \var{C}, then \code{C[...,0,...]} is
the same thing as \code{C[:,:,:,0,:]}.

When assigment source and destination locations overlap, i.e. when an array is
assigned onto itself at overlapping indices, it may produce something
"surprising":

\begin{verbatim}
>>> n=numarray.arange(36)
>>> n[11:18]=n[7:14]
>>> n
array([ 0,  1,  2,  3,  4,  5,  6,  7,  8,  9, 10,  7,  8,  9, 10,  7,
        8,  9, 18, 19, 20, 21, 22, 23, 24, 25, 26, 27, 28, 29, 30, 31,
       32, 33, 34, 35])
\end{verbatim}

If the slice on the right hand side (RHS) is AFTER that on the left hand side
(LHS) for 1-D array, then it works fine:

\begin{verbatim}
>>> n=numarray.arange(36)
>>> n[1:8]=n[7:14]       
>>> n
array([ 0,  7,  8,  9, 10, 11, 12, 13,  8,  9, 10, 11, 12, 13, 14, 15,
       16, 17, 18, 19, 20, 21, 22, 23, 24, 25, 26, 27, 28, 29, 30, 31,
       32, 33, 34, 35])
\end{verbatim}

Actually, this behavior can be undedrstood if we follow the pixel by pixel
copying logic.  Parts of the slice start to get the "updated" values when the
RHS is before the LHS.

An easy solution which is guaranteed to work is to use the copy() method on the
righ hand side:

\begin{verbatim}
>>> n=numarray.arange(36)
>>> n[11:18]=n[7:14].copy()
>>> n
array([ 0,  1,  2,  3,  4,  5,  6,  7,  8,  9, 10,  7,  8,  9, 10, 11,
       12, 13, 18, 19, 20, 21, 22, 23, 24, 25, 26, 27, 28, 29, 30, 31,
       32, 33, 34, 35])
\end{verbatim}

\newpage
\section{Pseudo Indices}
This section discusses pseudo-indices, which allow arrays to have their shapes
modified by adding axes, sometimes only for the duration of the evaluation of a
Python expression.

Consider multiplication of a rank-1 array by a scalar:
\begin{verbatim}
>>> a = array([1,2,3])
>>> print a * 2
[2 4 6]
\end{verbatim}
This should be trivial by now; we've just multiplied a rank-1 array by a
scalar . The scalar was converted to a rank-0 array which was then broadcast to
the next rank. This works for adding some two rank-1 arrays as well:
\begin{verbatim}
>>> print a
[1 2 3]
>>> a + array([4])
[5 6 7]
\end{verbatim}
but it won't work if either of the two rank-1 arrays have non-matching
dimensions which aren't 1.  In other words, broadcast only works for
dimensions which are either missing (e.g. a lower-rank array) or for dimensions
of 1.

With this in mind, consider a classic task, matrix multiplication. Suppose we
want to multiply the row vector [10,20] by the column vector [1,2,3].
\begin{verbatim}
>>> a = array([10,20])
>>> b = array([1,2,3])
>>> a * b
ValueError: Arrays have incompatible shapes
\end{verbatim}
% In "This makes sense - we're ..." the hyphen disappears in the PDF.
This makes sense: we're trying to multiply a rank-1 array of shape (2,) with a
rank-1 array of shape (3,). This violates the laws of broadcast. What we really
want to do is make the second vector a vector of shape (3,1), so that the first
vector can be broadcast across the second axis of the second vector. One way to
do this is to use the reshape function:
\begin{verbatim}
>>> a.getshape()
(2,)
>>> b.getshape()
(3,)
>>> b2 = reshape(b, (3,1))
>>> print b2
[[1]
 [2]
 [3]]
>>> b2.getshape()
(3, 1)
>>> print a * b2    # Note: b2 * a gives the same result
[[10 20]
 [20 40]
 [30 60]]
\end{verbatim}
This is such a common operation that a special feature was added (it turns out
to be useful in many other places as well) --� the NewAxis "pseudo-index",
originally developed in the Yorick language. NewAxis is an index, just like
integers, so it is used inside of the slice brackets []. It can be thought of
as meaning "add a new axis here," in much the same ways as adding a 1 to an
array's shape adds an axis. Again, examples help clarify the situation:
\begin{verbatim}
>>> print b
[1 2 3]
>>> b.getshape()
(3,)
>>> c = b[:, NewAxis]
>>> print c
[[1]
 [2]
 [3]]
>>> c.getshape()
(3,1)
\end{verbatim}
Why use such a pseudo-index over the reshape function or setshape assignments?
Often one doesn't really want a new array with a new axis, one just wants it
for an intermediate computation. Witness the array multiplication mentioned
above, without and with pseudo-indices:
\begin{verbatim}
>>> without = a * reshape(b, (3,1))
>>> with = a * b[:,NewAxis]
\end{verbatim}
The second is much more readable (once you understand how NewAxis works), and
it's much closer to the intended meaning. Also, it's independent of the
dimensions of the array b. You might counter that using something like
reshape(b, (-1,1)) is also dimension-independent, but 
it's less readable and impossible with rank-3 or higher arrays? The
NewAxis-based idiom also works nicely with higher rank arrays, and with the ...
"rubber index" mentioned earlier. Adding an axis before the last axis in an
array can be done simply with:
\begin{verbatim}
>>> a[...,NewAxis,:]
\end{verbatim}
Note that \code{NewAxis} is a \code{numarray} object, so if you used 
\code{import numarray} instead of \code{from numarray import *}, you'll 
need \code{numarray.NewAxis}.

\newpage
\section{Index Arrays}
\label{sec:index-arrays}

Arrays used as subscripts have special meanings which implicitly invoke the
functions \function{put} (page \pageref{func:put}), \function{take} (page
\pageref{func:take}), or \function{compress} (page \pageref{func:compress}). If
the array is of \class{Bool} type, then the indexing will be treated as the
equivalent of the compress function. If the array is of an integer type, then a
\function{take} or \function{put} operation is implied. We will generalize the
existing take and put as follows: If \var{ind1}, \var{ind2}, ...  \var{indN}
are index arrays (arrays of integers whose values indicate the index into
another array), then \code{x[ind1, ind2]} forms a new array with the same shape
as \var{ind1}, \var{ind2} (they all must be broadcastable to the same shape)
and values such: \samp{result[i,j,k] = x[ind1[i,j,k], ind2[i,j,k]]} In this
example, \var{ind1}, \var{ind2} are index arrays with 3 dimensions (but they
could have an arbitrary number of dimensions).  To illustrate with some
specific examples:
\begin{verbatim}
>>> x=2*arange(10)
>>> ind1=[0,4,3,7]
>>> x[ind1]
array([ 0,  8,  6, 14])
>>> ind1=[[0,4],[3,7]]
>>> x[ind1]
array([[ 0,  8],
       [ 6, 14]])
\end{verbatim}
This shows that the same elements in the same order are extracted from x by
both forms of ind1, but the result shares the shape of ind1 Something similar
happens in two dimensions:
\begin{verbatim}
>>> x=reshape(arange(12),(3,4))
>>> x
array([[ 0,  1,  2,  3],
       [ 4,  5,  6,  7],
       [ 8,  9, 10, 11]])
>>> ind1=array([2,1])
>>> ind2=array([0,3])
>>> x[ind1,ind2]
array([8, 7])
\end{verbatim}
Notice this pulls out x[2,0] and x[1,3] as a one-dimensional array.
\begin{verbatim}
>>> ind1=array([[2,2],[1,0]])
>>> ind2=array([[0,1],[3,2]])
>>> x[ind1,ind2]
array([[8, 9],
       [7, 2]])
\end{verbatim}
This pulls out x[2,0], x[2,1], x[1,3], and x[0,2], reading the ind1 and ind2
arrays left to right, and then reshapes the result to the same (2,2) shape as
ind1 and ind2 have.
\begin{verbatim}
>>> ind1.shape=(4,)
>>> ind2.shape=(4,)
>>> x[ind1,ind2]
array([8, 9, 7, 2])
\end{verbatim}

\newpage
Notice this is the same values in the same order, but now as a one-d array.
One index array does a broadcast:
\begin{verbatim}
>>> x[ind1]
array([[ 8,  9, 10, 11],
       [ 8,  9, 10, 11],
       [ 4,  5,  6,  7],
       [ 0,  1,  2,  3]])
>>> ind1.shape=(2,2)
>>> x[ind1]
array([[[ 8,  9, 10, 11],
        [ 8,  9, 10, 11]],

       [[ 4,  5,  6,  7],
        [ 0,  1,  2,  3]]])
\end{verbatim}

Again, note that the same 'elements', in this case rows of x, are returned in
both cases.  But in the second case, ind1 had two dimensions, and so using it
to index only one dimension of a two-d array results in a three-d output of
shape (2,2,4);  i.e., a 2 by 2 'array' of 4-element rows.

When using constants for some of the index positions, then the result uses that
constant for all values. Slices and strides (at least initially) will not be
permitted in the same subscript as index arrays. So
\begin{verbatim}
>>> x[ind1, 2]
array([[10, 10],
  [ 6,  2]])
\end{verbatim}
would be legal, but
\begin{verbatim}
>>> x[ind1, 1:3]
Traceback (most recent call last):
...
IndexError: Cannot mix arrays and slices as indices
\end{verbatim}
would not be.  Similarly for assignment:
\begin{verbatim}
x[ind1, ind2, ind3] = values
\end{verbatim}
will form a new array such that:
\begin{verbatim}
x[ind1[i,j,k], ind2[i,j,k], ind3[i,j,k]] = values[i,j,k]
\end{verbatim}

The index arrays and the value array must be broadcast consistently. (As an
example: \code{ind1.setshape((5,4))}, \code{ind2.setshape((5,))},
\code{ind3.setshape((1,4))}, and \code{values.setshape((1,))}.)
\begin{verbatim}
>>> x=zeros((10,10))
>>> x[[2,5,6],array([0,1,9,3])[:,NewAxis]]=array([1,2,3,4])[:,NewAxis]
>>> x
array([[0, 0, 0, 0, 0, 0, 0, 0, 0, 0],
       [0, 0, 0, 0, 0, 0, 0, 0, 0, 0],
       [1, 2, 0, 4, 0, 0, 0, 0, 0, 3],
       [0, 0, 0, 0, 0, 0, 0, 0, 0, 0],
       [0, 0, 0, 0, 0, 0, 0, 0, 0, 0],
       [1, 2, 0, 4, 0, 0, 0, 0, 0, 3],
       [1, 2, 0, 4, 0, 0, 0, 0, 0, 3],
       [0, 0, 0, 0, 0, 0, 0, 0, 0, 0],
       [0, 0, 0, 0, 0, 0, 0, 0, 0, 0],
       [0, 0, 0, 0, 0, 0, 0, 0, 0, 0]])
\end{verbatim}
If indices are repeated, the last value encountered will be stored.  When an
index is too large, Numarray raises an IndexError exception. When an index is
negative, Numarray will interpret it in the usual Python style, counting
backwards from the end.  Use of the equivalent \index{take}\function{take} and
\index{put}\function{put} functions will allow other interpretations of the
indices (clip out of bounds indices, allow negative indices to work backwards
as they do when used individually, or for indices to wrap around). The same
behavior applies for functions such as choose and where.

%% Local Variables:
%% mode: LaTeX
%% mode: auto-fill
%% fill-column: 79
%% indent-tabs-mode: nil
%% ispell-dictionary: "american"
%% reftex-fref-is-default: nil
%% TeX-auto-save: t
%% TeX-command-default: "pdfeLaTeX"
%% TeX-master: "numarray"
%% TeX-parse-self: t
%% End:

\chapter{Intermediate Topics}
\label{cha:intermediate-topics}

This chapter discusses a few of the more esoteric features of numarray which
are certainly useful but probably not a top priority for new users.

\section{Rank-0 Arrays}
\label{sec:rank-0-arrays}
numarray provides limited support for dimensionless arrays which represent
single values, also known as rank-0 arrays.  Rank-0 arrays are the array
representation of a scalar value.  They have the advantage over scalars that
they include array specific type information, e.g. \var{Int16}.  Rank-0 arrays
can be created as follows:
\begin{verbatim}
>>> a = array(1); a
array(1)
\end{verbatim}
A rank-0 array has a 0-length or empty shape:
\begin{verbatim}
>>> a.shape
()
\end{verbatim}
numarray's rank-0 array handling differs from that of Numeric in two ways.
First, numarray's rank-0 arrays cannot be indexed by 0:
\begin{verbatim}
>>> array(1)[0]
Traceback (most recent call last):
...
IndexError: Too many indices
\end{verbatim}
Second, numarray's rank-0 arrays do not have a length.
\begin{verbatim}
>>> len(array(1))
Traceback (most recent call last):
...
ValueError: Rank-0 array has no length.
\end{verbatim}
Finally, numarray's rank-0 arrays can be converted to a Python scalar by
subscripting with an empty tuple as follows:
\begin{verbatim}
>>> a = array(1)
>>> a[()]
1
\end{verbatim}

\newpage
\section{Exception Handling}
\label{sec:exception-handling}

We desired better control over exception handling than currently exists in
Numeric. This has traditionally been a problem area (see the numerous posts in
\ulink{comp.lang.python}{news:comp.lang.python} regarding floating point
exceptions, especially those by Tim Peters). Numeric raises an exception for
integer computations that result in a divide by zero or multiplications that
result in overflows. The exception is raised after that operation has completed
on all the array elements. No exceptions are raised for floating point errors
(divide by zero, overflow, underflow, and invalid results), the compiler and
processor are left to their default behavior (which is usually to return Infs
and NaNs as values).

The approach for numarray is to provide customizable error handling behavior.
It should be possible to specify three different behaviors for each of the four
error types independently. These are:
\begin{itemize}
\item Ignore the error.
\item Print a warning.
\item Raise a Python exception.
\end{itemize}
The current implementation does that and has been tested successfully on
Windows, Solaris, Redhat and Tru64.  The implementation uses the floating point
processor ``sticky status flags'' to detect errors. One can set the error mode
by calling the error object's setMode method. For example:
\begin{verbatim}
>>> Error.setMode(all="warn") # the default mode
>>> Error.setMode(dividebyzero="raise", underflow="ignore", invalid="warn")
\end{verbatim}

The Error object can also be used in a stacking manner, by using the \function{pushMode}
and \function{popMode} methods rather than \function{setMode}.  For example:
\begin{verbatim}
>>> Error.getMode()
_NumErrorMode(overflow='warn', underflow='warn', dividebyzero='warn', invalid='warn')
>>> Error.pushMode(all="raise") # get really picky...
>>> Error.getMode()
_NumErrorMode(overflow='raise', underflow='raise', dividebyzero='raise', invalid='raise')
>>> Error.popMode()  # pop and return the ``new'' mode
_NumErrorMode(overflow='raise', underflow='raise', dividebyzero='raise', invalid='raise')
>>> Error.getMode()  # verify the original mode is back
_NumErrorMode(overflow='warn', underflow='warn', dividebyzero='warn', invalid='warn')
\end{verbatim}
Integer exception modes work the same way. Although integer computations do not
affect the floating point status flag directly, our code checks the denominator
of 0 in divisions (in much the same way Numeric does) and then performs a
floating point divide by zero to set the status flag (overflows are handled
similarly). So even integer exceptions use the floating point status flags
indirectly.

\newpage
\section{IEEE-754 Not a Number (NAN) and Infinity}
\label{sec:ieee-special-values}
\module{numarray.ieeespecial} has support for manipulating IEEE-754 floating
point special values NaN (Not a Number), Inf (infinity), etc.  The special
values are denoted using lower case as follows:
\begin{verbatim}
>>> import numarray.ieeespecial as ieee
>>> ieee.inf
inf
>>> ieee.plus_inf
inf
>>> ieee.minus_inf
-inf
>>> ieee.nan
nan
>>> ieee.plus_zero
0.0
>>> ieee.minus_zero
-0.0
\end{verbatim}
Note that the representation of IEEE special values is platform dependent so
your Python might for instance say \var{Infinity} rather than \var{inf}.
Below, \var{inf} is seen to arise as the result of floating point division by 0
and \var{nan} is seen to arise from 0 divided by 0:
\begin{verbatim}
>>> a = array([0.0, 1.0])
>>> b = a/0.0
Warning: Encountered invalid numeric result(s)  in divide
Warning: Encountered divide by zero(s)  in divide
>>> b
array([              nan,               inf])
\end{verbatim}
A curious property of \var{nan} is that it does not compare to \emph{itself} as
equal:
\begin{verbatim}
>>> b == ieee.nan
array([0, 0], type=Bool)
\end{verbatim}
The \function{isnan}, \function{isinf}, and \function{isfinite} functions
return boolean arrays which have the value True where the corresponding
predicate holds.  These functions detect bit ranges and are therefore more
robust than simple equality checks.
\begin{verbatim}
>>> ieee.isnan(b)
array([1, 0], type=Bool)
>>> ieee.isinf(b)
array([0, 1], type=Bool)
>>> ieee.isfinite(b)
array([0, 0], type=Bool)
\end{verbatim}
Array based indexing provides a convenient way to replace special values:
\begin{verbatim}
>>> b[ieee.isnan(b)] = 999
>>> b[ieee.isinf(b)] = 5
>>> b
array([ 999.,    5.])
\end{verbatim}

Here's an easy approach for compressing your data arrays to remove
NaNs:
\begin{verbatim}
>>> x, y = arange(10.), arange(10.)
>>> x[5] = ieee.nan
>>> y[6] = ieee.nan
>>> keep = ~ieee.isnan(x) & ~ieee.isnan(y)
>>> x[keep]
array([ 0.,  1.,  2.,  3.,  4.,  7.,  8.,  9.])
>>> y[keep]
array([ 0.,  1.,  2.,  3.,  4.,  7.,  8.,  9.])
\end{verbatim}

%% Local Variables:
%% mode: LaTeX
%% mode: auto-fill
%% fill-column: 79
%% indent-tabs-mode: nil
%% ispell-dictionary: "american"
%% reftex-fref-is-default: nil
%% TeX-auto-save: t
%% TeX-command-default: "pdfeLaTeX"
%% TeX-master: "numarray"
%% TeX-parse-self: t
%% End:

\chapter{Ufuncs}
\label{cha:ufuncs}

\section{What are Ufuncs?}
\label{sec:what-are-ufuncs}

The operations on arrays that were mentioned in the previous section
(element-wise addition, multiplication, etc.) all share some features --- they
all follow similar rules for broadcasting, coercion and ``element-wise
operation''. Just as standard addition is available in Python through the add
function in the operator module, array operations are available through
callable objects as well. Thus, the following objects are available in the
numarray module:

\begin{table}[htbp]
   \centering
   \caption{Universal Functions, or ufuncs. The operators which invoke them
   when applied to arrays are indicated in parentheses. The entries in slanted
   typeface refer to unary ufuncs, while the others refer to binary ufuncs.} 
   \label{tab:ufuncs}
   \begin{tabular}{llll}
      add ($+$)         & subtract ($-$)   & multiply (*)       & divide ($/$) \\
      remainder (\%)    & power (**)       & \textsl{arccos}    & \textsl{arccosh} \\
      \textsl{arcsin}   & \textsl{arcsinh} & \textsl{arctan}    & \textsl{arctanh} \\
      \textsl{cos}      & \textsl{cosh}    & \textsl{tan}       & \textsl{tanh} \\
      \textsl{log10}    & \textsl{sin}     & \textsl{sinh}      & \textsl{sqrt} \\
      \textsl{absolute (abs)} & \textsl{fabs}    & \textsl{floor}     & \textsl{ceil} \\
      fmod              & \textsl{exp}     & \textsl{log}       & \textsl{conjugate} \\
      maximum           & minimum \\
      greater ($>$)     & greater\_equal ($>=$) & equal ($==$)  \\
      less ($<$)        & less\_equal ($<=$)  & not\_equal ($!=$) \\
      logical\_or       & logical\_xor     & logical\_not  & logical\_and \\
      bitwise\_or ($|$) & bitwise\_xor (\^{}) 
                        & bitwise\_not (\textasciitilde)  & bitwise\_and (\&)
      \\
      rshift ($>>$)       & lshift ($<<$)
   \end{tabular}
\end{table}

\remark{Add a remark here on how numarray does (or will) handle 'true'
and 'floor' division, which can be activated in Python 2.2 with
\samp{from __future__ import division}?.
Note: with 'true' division, \samp{1/2 == 0.5} and not \samp{0}.}

All of these ufuncs can be used as functions. For example, to use
\function{add}, which is a binary ufunc (i.e.\ it takes two arguments), one can
do either of:
\begin{verbatim}
>>> a = arange(10)
>>> print add(a,a)
[ 0  2  4  6  8 10 12 14 16 18]
>>> print a + a
[ 0  2  4  6  8 10 12 14 16 18]
\end{verbatim}
In other words, the \code{+} operator on arrays performs exactly the same thing
as the \function{add} ufunc when operated on arrays.  For a unary ufunc such as
\function{sin}, one can do, e.g.:
\begin{verbatim}
>>> a = arange(10)
>>> print sin(a)
[ 0.          0.84147096  0.90929741  0.14112    -0.7568025
      -0.95892429 -0.27941549  0.65698659  0.98935825  0.41211849]
\end{verbatim}
A unary ufunc returns an array with the same shape as its argument array, but
with each element replaced by the application of the function to that element
(\code{sin(0)=0}, \code{sin(1)=0.84147098}, etc.).

There are three additional features of ufuncs which make them different from
standard Python functions. They can operate on any Python sequence in addition
to arrays; they can take an ``output'' argument; they have methods which are
themselves callable with arrays and sequences. Each of these will be described
in turn.

Ufuncs can operate on any Python sequence. Ufuncs have so far been described as
callable objects which take either one or two arrays as arguments (depending on
whether they are unary or binary). In fact, any Python sequence which can be
the input to the \function{array} constructor can be used.  The return value
from ufuncs is always an array.  Thus:
\begin{verbatim}
>>> add([1,2,3,4], (1,2,3,4))
array([2, 4, 6, 8])
\end{verbatim}


\subsection{Ufuncs can take output arguments}
\label{sec:ufuncs-can-take}

In many computations with large sets of numbers, arrays are often used only
once. For example, a computation on a large set of numbers could involve the
following step
\begin{verbatim}
dataset = dataset * 1.20 
\end{verbatim}
This can also be written as the following using the Ufunc form:
\begin{verbatim}
dataset = multiply(dataset, 1.20)
\end{verbatim}
In both cases, a temporary array is created to store the results of the
computation before it is finally copied into \var{dataset}. It is
more efficient, both in terms of memory and computation time, to do an
``in-place'' operation.  This can be done by specifying an existing array as
the place to store the result of the ufunc. In this example, one can 
write:\footnote[1]{for Python-2.2.2 or later: `dataset *= 1.20' also works}
\begin{verbatim}
multiply(dataset, 1.20, dataset) 
\end{verbatim}
This is not a step to take lightly, however. For example, the ``big and slow''
version (\code{dataset = dataset * 1.20}) and the ``small and fast'' version
above will yield different results in at least one case:
\begin{itemize}
\item If the type of the target array is not that which would normally be
   computed, the operation will not coerce the array to the expected data type.
   (The result is done in the expected data type, but coerced back to the
   original array type.)
\item Example:
\begin{verbatim}
\>>> a=arange(5,type=Int32)
>>> print a[::-1]*1.7
[ 6.8  5.1  3.4  1.7  0. ]
>>> multiply(a[::-1],1.7,a)
>>> print a
[6 5 3 1 0]
>>> a *= 1.7
>>> print a
[0 1 3 5 6]
\end{verbatim}
\end{itemize}

The output array does not need to be the same variable as the input array. In
numarray, in contrast to Numeric, the output array may have any type (automatic
conversion is performed on the output).

\subsection{Ufuncs have special methods}
\label{sec:ufuncs-have-special-methods}


\begin{methoddesc}{reduce}{a, axis=0}
   If you don't know about the \function{reduce} command in Python, review
   section 5.1.3 of the Python Tutorial
   (\url{http://www.python.org/doc/current/tut/}). Briefly,
   \function{reduce} is most often used with two arguments, a callable object
   (such as a function), and a sequence. It calls the callable object with the
   first two elements of the sequence, then with the result of that operation
   and the third element, and so on, returning at the end the successive
   ``reduction'' of the specified callable object over the sequence elements.
   Similarly, the \method{reduce} method of ufuncs is called with a sequence as
   an argument, and performs the reduction of that ufunc on the sequence. As an
   example, adding all of the elements in a rank-1 array can be done with:
\begin{verbatim}
>>> a = array([1,2,3,4])
>>> print add.reduce(a)   # with Python's reduce, same as reduce(add, a)
10
\end{verbatim}
   When applied to arrays which are of rank greater than one, the reduction
   proceeds by default along the first axis:
\begin{verbatim}
>>> b = array([[1,2,3,4],[6,7,8,9]])
>>> print b
[[1 2 3 4]
 [6 7 8 9]]
>>> print add.reduce(b)
[ 7  9 11 13]
\end{verbatim}
   A different axis of reduction can be specified with a second integer
   argument:
\begin{verbatim}
>>> print b
[[1 2 3 4]
 [6 7 8 9]]
>>> print add.reduce(b, axis=1)
[10 30]
\end{verbatim}
\end{methoddesc}


\begin{methoddesc}{accumulate}{a}
   The \method{accumulate} ufunc method is simular to \method{reduce}, except
   that it returns an array containing the intermediate results of the
   reduction:
\begin{verbatim}
>>> a = arange(10)
>>> print a
[0 1 2 3 4 5 6 7 8 9]
>>> print add.accumulate(a)
[ 0  1  3  6 10 15 21 28 36 45] # 0, 0+1, 0+1+2, 0+1+2+3, ... 0+...+9
>>> print add.reduce(a) # same as add.accumulate(a)[-1] w/o side effects on a
45                                      
\end{verbatim}
\end{methoddesc}


\begin{methoddesc}{outer}{a, b}
   The third ufunc method is \method{outer}, which takes two arrays as
   arguments and returns the ``outer ufunc'' of the two arguments. Thus the
   \method{outer} method of the \function{multiply} ufunc, results in the outer
   product. The \method{outer} method is only supported for binary methods.
\begin{verbatim}
>>> print a
[0 1 2 3 4]
>>> print b
[0 1 2 3]
>>> print add.outer(a,b)
[[0 1 2 3]
 [1 2 3 4]
 [2 3 4 5]
 [3 4 5 6]
 [4 5 6 7]]
>>> print multiply.outer(b,a)
[[ 0  0  0  0  0]
 [ 0  1  2  3  4]
 [ 0  2  4  6  8]
 [ 0  3  6  9 12]]
>>> print power.outer(a,b)
[[ 1  0  0  0]
 [ 1  1  1  1]
 [ 1  2  4  8]
 [ 1  3  9 27]
 [ 1  4 16 64]]
\end{verbatim}
\end{methoddesc}


\begin{methoddesc}{reduceat}{}
   The reduceat method of Numeric has not been implemented in numarray.
\end{methoddesc}

\subsection{Ufuncs always return new arrays}
\label{sec:ufuncs-always-return}

Except when the output argument is used as described above, ufuncs always
return new arrays which do not share any data with the input arrays.


\section{Which are the Ufuncs?}
\label{sec:which-are-ufuncs}

Table \ref{tab:ufuncs} lists all the ufuncs. We will first discuss the
mathematical ufuncs, which perform operations very similar to the functions in
the \module{math} and \module{cmath} modules, albeit elementwise, on arrays.
Originally,  numarray ufuncs came in two forms, unary and binary.  More
recently (1.3) numarray has added support for ufuncs with up to 16 total
input or output parameters.  These newer ufuncs are called N-ary ufuncs.

\subsection{Unary Mathematical Ufuncs}
\label{sec:unary-math-ufuncs}

Unary ufuncs take only one argument.  The following ufuncs apply the
predictable functions on their single array arguments, one element at a time:
\function{arccos}, \function{arccosh}, \function{arcsin}, \function{arcsinh},
\function{arctan}, \function{arctanh}, \function{cos}, \function{cosh},
\function{exp}, \function{log}, \function{log10}, \function{sin},
\function{sinh}, \function{sqrt}, \function{tan}, \function{tanh},
\function{abs}, \function{fabs}, \function{floor}, \function{ceil},
\function{conjugate}.  As an example:
\begin{verbatim}
>>> print x
[0 1 2 3 4]
>>> print cos(x)
[ 1.          0.54030231 -0.41614684 -0.9899925  -0.65364362]
>>> print arccos(cos(x))
[ 0.          1.          2.          3.          2.28318531]
# not a bug, but wraparound: 2*pi%4 is 2.28318531
\end{verbatim}


\subsection{Binary Mathematical Ufuncs}
\label{sec:binary-math-ufuncs}

These ufuncs take two arrays as arguments, and perform the specified
mathematical operation on them, one pair of elements at a time: \function{add},
\function{subtract}, \function{multiply}, \function{divide},
\function{remainder}, \function{power}, \function{fmod}.


\subsection{Logical and bitwise ufuncs}
\label{sec:logical-ufuncs}

The ``logical'' ufuncs also perform their operations on arrays or numbers 
in elementwise fashion, just like the "mathematical" ones.  Two are special
(\function{maximum} and \function{miminum}) in that they return arrays with
entries taken from their input arrays:
\begin{verbatim}
>>> print x
[0 1 2 3 4]
>>> print y
[ 2.   2.5  3.   3.5  4. ]
>>> print maximum(x, y)
[ 2.   2.5  3.   3.5  4. ]
>>> print minimum(x, y)
[ 0.  1.  2.  3.  4.]
\end{verbatim}
The others logical ufuncs return arrays of 0's or 1's and of type Bool:
\function{logical_and}, \function{logical_or}, \function{logical_xor},
\function{logical_not}, 
These are fairly
self-explanatory, especially with the associated symbols from the standard
Python version of the same operations in Table \ref{tab:ufuncs} above. 
The bitwise ufuncs,
\function{bitwise_and}, \function{bitwise_or},
\function{bitwise_xor}, \function{bitwise_not},  
\function{lshift}, \function{rshift},  
on the other hand, only work with integer arrays (of any word size), and
will return integer arrays of the larger bit size of the two input arrays:
\begin{verbatim}
>>> x
array([7, 7, 0], type=Int8)
>>> y
array([4, 5, 6])
>>> x & y          # bitwise_and(x,y)
array([4, 5, 0])
>>> x | y          # bitwise_or(x,y)
array([7, 7, 6])   
>>> x ^ y          # bitwise_xor(x,y)
array([3, 2, 6]) 
>>> ~ x            # bitwise_not(x)
array([-8, -8, -1], type=Int8)

\end{verbatim}
We discussed finding contents of arrays based on arrays' indices by using slice.
Often, especially when dealing with the result of computations or data
analysis, one needs to ``pick out'' parts of matrices based on the content of
those matrices. For example, it might be useful to find out which elements of
an array are negative, and which are positive. The comparison ufuncs are
designed for such operation. Assume an array with various positive
and negative numbers in it (for the sake of the example we'll generate it from
scratch):
\begin{verbatim}
>>> print a
[[ 0  1  2  3  4]
 [ 5  6  7  8  9]
 [10 11 12 13 14]
 [15 16 17 18 19]
 [20 21 22 23 24]]
>>> b = sin(a)
>>> print b
[[ 0.          0.84147098  0.90929743  0.14112001 -0.7568025 ]
 [-0.95892427 -0.2794155   0.6569866   0.98935825  0.41211849]
 [-0.54402111 -0.99999021 -0.53657292  0.42016704  0.99060736]
 [ 0.65028784 -0.28790332 -0.96139749 -0.75098725  0.14987721]
 [ 0.91294525  0.83665564 -0.00885131 -0.8462204  -0.90557836]]
>>> print greater(b, .3)
[[0 1 1 0 0]
 [0 0 1 1 1]
 [0 0 0 1 1]
 [1 0 0 0 0]
 [1 1 0 0 0]]
\end{verbatim}


\subsection{Comparisons}
\label{sec:comparisons}

The comparison functions \function{equal}, \function{not_equal},
\function{greater}, \function{greater_equal}, \function{less}, and
\function{less_equal} are invoked by the operators \code{==}, \code{!=},
\code{>}, \code{>=}, \code{<}, and \code{<=} respectively, but they can also be
called directly as functions. Continuing with the preceding example,
\begin{verbatim}
>>> print less_equal(b, 0)
[[1 0 0 0 1]
 [1 1 0 0 0]
 [1 1 1 0 0]
 [0 1 1 1 0]
 [0 0 1 1 1]]
\end{verbatim}
This last example has 1's where the corresponding elements are less than or
equal to 0, and 0's everywhere else.

The operators and the comparison functions are not exactly equivalent.  To
compare an array a with an object b, if b can be converted to an array, the
result of the comparison is returned. Otherwise, zero is returned. This means
that comparing a list and comparing an array can return quite different
answers. Since the functional forms such as equal will try to make arrays from
their arguments, using equal can result in a different result than using
\code{==}.
\begin{verbatim}
>>> a = array([1, 2, 3])
>>> b = [1, 2, 3]
>>> print a == 2
[0 1 0]
>>> print b == 2  
0          # (False since 2.3)
>>> print equal(a, 2)
[0 1 0]
>>> print equal(b, 2)
[0 1 0]
\end{verbatim}

\subsection{Ufunc shorthands}
\label{sec:ufunc-shorthands}

Numarray defines a few functions which correspond to popular ufunc methods:
for example, \function{add.reduce} is synonymous with the \function{sum}
utility function:
\begin{verbatim}
>>> a = arange(5)                       # [0 1 2 3 4]
>>> print sum(a)                        # 0 + 1 + 2 + 3 + 4
10
\end{verbatim}
Similarly, \function{cumsum} is equivalent to \function{add.accumulate} (for
``cumulative sum''), \function{product} to \function{multiply.reduce}, and
\function{cumproduct} to \function{multiply.accumulate}.  Additional useful
``utility'' functions are \function{all} and \function{any}:
\begin{verbatim}
>>> a = array([0,1,2,3,4])
>>> print greater(a,0)
[0 1 1 1 1]
>>> all(greater(a,0))
0
>>> any(greater(a,0))
1
\end{verbatim}

\section{Writing your own ufuncs!}

This section describes a new process for defining your own universal functions.
It explains a new interface that enables the description of N-ary ufuncs, those
that use semi-arbitrary numbers \((<= 16)\) of inputs and outputs.

\subsection{Runtime components of a ufunc}

A numarray universal function maps from a Python function name to a set of C
functions.  Ufuncs are polymorphic and figure out what to do in C when passed a
particular set of input parameter types.  C functions, on the other hand, can
only be run on parameters which match their type signatures.  The task of
defining a universal function is one of describing how different parameter
sequences are mapped from Python array types to C function signatures and back.

At runtime, there are three principle kinds of things used to define a
universal function.

\begin {enumerate}
\item Ufunc 

The universal function is itself a callable Python object.  Ufuncs organize a
collection of Cfuncs to be called based on the actual parameter types seen at
runtime.  The same Ufunc is typically associated with several Cfuncs each of
which handles a unique Ufunc type signature.  Because a Ufunc typically has
more than one C func, it can also be implemented using more than one library
function.

\item Library function

A pre-existing function written in C or Fortran which will ultimately be called
for each element of the ufunc parameter arrays.  

\item Cfunc

Cfuncs are binding objects that map C library functions safely into Python.
It's the job of a Cfunc to interpret typeless pointers corresponding to the
parameter arrays as particular C data types being passed down from the ufunc.
Further, the Cfunc casts array elements from the input type to the Libraray
function parameter type.  This process lets the ufunc implementer describe the
ufunc type signatures which will be processed most efficiently by the
underlying Library function by enabling per-call element-by-element type casts.
Ufunc calling signatures which are not represented directly by a Cfunc result
in blockwise coercion to the closest matching Cfunc, which is slower.

\end {enumerate}

\subsection{Source components of a ufunc}
There are 4 source components required to define numarray ufuncs, one of which
is hand written, two are generated, and the last is assumed to be pre-existing:

\begin {enumerate}
\item Code generation script

The primary source component for defining new universal functions is a Python
script used to generate the other components.  For a standalone set of
functions, putting the code generation directives in setup.py can be done as in
the example numarray/Examples/ufunc/setup_airy.py.  The code generation script
only executes at install time.

\item Extension module

A private extension module is generated which contains a collection of Cfuncs
for the package being created.  The extension module contains a dictionary
mapping from ufuncs/types to Cfuncs.

\item Ufunc init file 

A Python script used at ufunc import time is required to construct Ufunc
objects from Cfuncs.  This code is boilerplate created with the code generation
directive \function{make_stub()}.

\item Library functions

The C functions which are ultimately called by a Ufunc need to be defined
somewhere, typically in a third party C or Fortran library which is linked
to the Extension module.
\end{enumerate}

\subsection{Ufunc code generation}
There are several code generation directives provided by package
numarray.codegenerator which are called at installation time to generate the
Cfunc extension module and Ufunc init file.

\begin{funcdesc}{UfuncModule}{module_name}
The \class{UfuncModule} constructor creates a module object which collects code
which is later output to form the Cfunc extension module.  The name passed to
the constructor defines the name of the Python extension module, not the source
code.
\begin{verbatim}
m = UfuncModule("_na_special")
\end{verbatim}
\end{funcdesc}

\begin{methoddesc}{add_code}{code_string}
The \method{add_code()} method of a \class{UfuncModule} object is used to add
arbitrary code to the module at the point that \method{add_code()} is
called. Here it includes a header file used to define prototypes for the C
library functions which this extension will ultimately call.
\begin{verbatim}
m.add_code('#include "airy.h"')
\end{verbatim}
\end{methoddesc}

\begin{methoddesc}{add_nary_ufunc}{ufunc_name, c_name,
    ufunc_signatures, c_signature, forms=None} 
The \method{add_nary_ufunc()} method declares a Ufunc and relates it to one
library function and a collection of Cfunc bindings for it.  The
\var{signatures} parameter defines which ufunc type signatures receive Cfunc
bindings. Input types which don't match those signature are blockwise coerced
to the best matching signature.  \method{add_nary_ufunc()} can be called for
the same Ufunc name more than once and can thus be used to associate multiple
library functions with the same Ufunc.
\begin{verbatim}
m.add_nary_ufunc(ufunc_name = "airy",
                 c_function  = "airy",    
                 signatures  =["dxdddd",
                               "fxffff"],
                 c_signature = "dxdddd")
\end{verbatim}
\end{methoddesc}

\begin{methoddesc}{generate}{source_filename}
The \method{generate()} method asks the \class{UfuncModule} object to emit the
code for an extension module to the specified \var{source_filename}.
\begin{verbatim}
m.generate("Src/_na_specialmodule.c")
\end{verbatim}
\end{methoddesc}

\begin{funcdesc}{make_stub}{stub_filename, cfunc_extension, add_code=None}
The \function{make_stub()} function is used to generate the boilerplate Python
code which constructs universal functions from a Cfunc extension module at
import time.  \function{make_stub()} accepts a \var{add_code} parameter which
should be a string containing any additional Python code to be injected into
the stub module.  Here \function{make_stub()} creates the init file
``Lib/__init__.py'' associated with the Cfunc extension ``_na_special'' and
includes some extra Python code to define the \function{plot_airy()} function.
\begin{verbatim}
extra_stub_code = '''

import matplotlib.pylab as mpl

def plot_airy(start=-10,stop=10,step=0.1,which=1):
    a = mpl.arange(start, stop, step)
    mpl.plot(a, airy(a)[which])

    b = 1.j*a + a
    ba = airy(b)[which]

    h = mpl.figure(2)
    mpl.plot(b.real, ba.real)

    i = mpl.figure(3)
    mpl.plot(b.imag, ba.imag)
    
    mpl.show()
'''

make_stub("Lib/__init__", "_na_special", add_code=extra_stub_code)
\end{verbatim}

\end{funcdesc}

\subsection{Type signatures and signature ordering}

Type signatures are described using the single character typecodes from
Numeric.  Since the type signature and form of a Cfunc need to be encoded in
its name for later identification, it must be brief.  

\begin{verbatim}
typesignature ::= <inputtypes> + ``x'' + <outputtypes>
inputtypes ::= [<typecode>]+
outputtypes ::= [<typecode>]+
typecode ::= "B" | "1" | "b" | "s" | "w" | "i" | "u" |
             "N" | "U" | "f" | "d" | "F" | "D"
\end{verbatim}

For example,  the type signature corresponding to one Int32 input and one Int16
output is "ixs".

A type signature is a sequence of ordered types.  One signature can be compared
to another by comparing corresponding elements, in left to right order.
Individual elements are ranked using the order from the previous section.  A
ufunc maintains its associated Cfuncs as a sorted sequence and selects the
first Cfunc which is \(>=\) the input type signature;  this defines the notion
of ``best matching''.

\subsection{Forms}

The \method{add_nary_ufunc()} method has a parameter \var{forms} which enables
generation of code with some extra properties.  It specifies the list of
function forms for which dedicated code will be generated.  If you don't
specify \var{forms}, it defaults to a (list of a) single form which specifies
that all inputs and outputs corresponding to the type signature are vectors.
Input vectors are passed by value, output vectors are passed by reference.  The
default form implies that the library function return value, if there is one,
is ignored.  The following Python code shows the default form:

\begin{verbatim}
["v"*n_inputs + "x" + "v"*n_outputs] 
\end{verbatim}

Forms are denoted using a syntax very similar to, and typically symmetric with,
type signatures.

\begin{verbatim}

form ::=  <inputs> "x" <outputs>
inputs ::= ["v"|"s"]*
outputs ::= ["f"]?["v"]* | "A" | "R"

The form character values have different meanings than for type
signatures:

'v'  :   vector,  an array of input or output values
's'  :   scalar,  a non-array input value
'f'  :   function,  the c_function returns a value
'R'  :   reduce,    this binary ufunc needs a reduction method
'A'  :   accumulate this binary ufunc needs an accumulate method
'x'  :   separator  delineates inputs from outputs

\end{verbatim}

So, a form consists of some input codes followed by a lower case "x" followed
by some output codes.  

The form for a C function which takes 4 input values, the last of which is
assumed to be a scalar, returns one value, and fills in 2 additional output
values is:  "vvvsxfvv".

Using "s" to designate scalar parameters is a useful performance
optimization for cases where it is known that only a single value is
passed in from Python to be used in all calls to the c function.  This
prevents the blockwise expansion of the scalar value into a vector.

Use "f" to specify that the C function return value should be kept; it must
always be the first output and will therefore appear as the first element of
the result tuple.

For ufuncs of two input parameters (binary ufuncs), two additional form
characters are possible: A (accumulate) and R (reduce).  Each of these
characters constitutes the *entire* ufunc form, so the form is denoted "R" or
"A".  For these kinds of cfuncs, the type signature always reads \code{<t>x<t>}
where \code{<t>} is one of the type characters.  

One reason for all these codes is so that the many Cfuncs generated for Ufuncs
can be easily named.  The name for the Cfunc which implements \function{add()}
for two Int32 inputs and one Int32 output and where all parameters are arrays
is: "add_iixi_vvxv".  The cfunc name for \method{add.reduce()} with two integer
parameters would be written as "add_ixi_R" and for \method{add.accumulate()}
as "add_ixi_A".

The set of Cfuncs generated is based on the signatures \emph{crossed} with the
forms.  Multiple calls to \method{add_nary_ufunc()} can be used the reduce the
effects of signature/form crossing.

\newpage
\subsection{Ufunc Generation Example}

This section includes code from Examples/ufunc/setup_airy.py in the numarray
source distribution to illustrate how to create a package which defines your
own universal functions.  

This script eventually generates two files: _na_airymodule.c and
__init__.py.  The former defines an extension module which creates
numarray cfuncs, c helpers for the numarray airy() ufunc.  The latter
file includes Python code which automatically constructs numarray
universal functions (ufuncs) from the cfuncs in _na_airymodule.c.

\begin{verbatim}

import distutils, os, sys
from distutils.core import setup
from numarray.codegenerator import UfuncModule, make_stub
from numarray.numarrayext import NumarrayExtension

m = UfuncModule("_na_special")

m.add_code('#include "airy.h"')

m.add_nary_ufunc(ufunc_name = "airy",
                 c_function  = "airy",    
                 signatures  =["dxdddd",
                               "fxffff"],
                 c_signature = "dxdddd")

m.add_nary_ufunc(ufunc_name = "airy",
                 c_function  ="cairy_fake",
                 signatures  =["DxDDDD",
                               "FxFFFF"],
                 c_signature = "DxDDDD")

m.generate("Src/_na_specialmodule.c")

\end{verbatim}

\begin{verbatim}

extra_stub_code = '''
def plot_airy(start=-10,stop=10,step=0.1,which=1):
    import matplotlib.pylab as mpl;

    a = mpl.arange(start, stop, step);
    mpl.plot(a, airy(a)[which]);

    b = 1.j*a + a
    ba = airy(b)[which]

    h = mpl.figure(2)
    mpl.plot(b.real, ba.real)

    i = mpl.figure(3)
    mpl.plot(b.imag, ba.imag)
    
    mpl.show()
'''

make_stub("Lib/__init__", "_na_special", 
          add_code=extra_stub_code)

dist = setup(name = "na_special",
      version = "0.1",
      maintainer = "Todd Miller",
      maintainer_email = "jmiller@stsci.edu",
      description = "airy() universal function for numarray",
      url = "http://www.scipy.org/",
      packages = ["numarray.special"],
      package_dir = { "numarray.special":"Lib" },
      ext_modules = [ NumarrayExtension( 'numarray.special._na_special',
                                         ['Src/_na_specialmodule.c',
                                          'Src/airy.c',
                                          'Src/const.c',
                                          'Src/polevl.c']
                                        )
                     ]
     )

\end{verbatim}

Additional explanatory text is available in
numarray/Examples/ufunc/setup_airy.py.  Scripts used to extract
numarray ufunc specs from the existing Numeric ufunc definitions
in scipy.special are in numarray/Examples/ufunc/RipNumeric as an
example of how to convert existing Numeric code to numarray.



%% Local Variables:
%% mode: LaTeX
%% mode: auto-fill
%% fill-column: 79
%% indent-tabs-mode: nil
%% ispell-dictionary: "american"
%% reftex-fref-is-default: nil
%% TeX-auto-save: t
%% TeX-command-default: "pdfeLaTeX"
%% TeX-master: "numarray"
%% TeX-parse-self: t
%% End:

\chapter{Array Functions}
\label{cha:array-functions}

Most of the useful manipulations on arrays are done with functions. This might
be surprising given Python's object-oriented framework, and that many of these
functions could have been implemented using methods instead. Choosing functions
means that the same procedures can be applied to arbitrary python sequences,
not just to arrays. For example, while \code{transpose([[1,2],[3,4]])} works
just fine, \code{[[1,2],[3,4]].transpose()} does not. This approach also allows
uniformity in interface between functions defined in the numarray Python
system, whether implemented in C or in Python, and functions defined in
extension modules. We've already covered two functions which operate on arrays:
\code{reshape} and \code{resize}.

\begin{funcdesc}{take}{array, indices, axis=0, clipmode=CLIP}
   \label{sec:array-functions:take}
   \label{func:take}
   The function \code{take} is a generalized indexing/slicing of the array.  In 
   its simplest form, it is equivalent to indexing:
\begin{verbatim}
>>> a1 = array([10,20,30,40])
>>> print a1[[1,3]]
[20 40]
>>> print take(a1,[1,3])
[20 40]
\end{verbatim}
   See the description of index
   arrays in the Array Basics section for an alternative to \code{take} 
   and \code{put}. \code{take}
   selects the elements of the array (the first argument) based on the
   indices (the second argument). Unlike slicing, however, the array
   returned by \code{take} has the same rank as the input array.  
   Some 2-D examples:
\begin{verbatim}
>>> print a2
[[ 0  1  2  3  4]
 [ 5  6  7  8  9]
 [10 11 12 13 14]
 [15 16 17 18 19]]
>>> print take(a2, (0,))                 # first row
[[0 1 2 3 4]]
>>> print take(a2, (0,1))                # first and second row
[[0 1 2 3 4]
 [5 6 7 8 9]]
>>> print take (a2, (0, -1))             # index relative to dimension end
[[ 0 1 2 3 4]
[15 16 17 18 19]]
\end{verbatim}
   The optional third argument specifies the axis along which the selection
   occurs, and the default value (as in examples above) is 0, the first
   axis.  If you want a different axis, then you need to specify it:
\begin{verbatim}
>>> print take(a2, (0,), axis=1)         # first column
[[ 0]
 [ 5]
 [10]
 [15]]
>>> print take(a2, (0,1), axis=1)        # first and second column
[[ 0  1]
 [ 5  6]
 [10 11]
 [15 16]]
>>> print take(a2, (0,4), axis=1)        # first and last column
[[ 0  4]
 [ 5  9]
 [10 14]
 [15 19]]
\end{verbatim}
   
   This is considered to be a \var{structural} operation, because its 
   result does
   not depend on the content of the arrays or the result of a computation on
   those contents but uniquely on the structure of the array. Like all such
   structural operations, the default axis is 0 (the first rank). 
   Later in this tutorial, we will see other functions with a default axis 
   of -1.
   
   \function{take} is often used to create multidimensional arrays with the
   indices from a rank-1 array. As in the earlier examples, the shape of the
   array returned by \function{take} is a combination of the shape of its first
   argument and the shape of the array that elements are "taken" from �-- when
   that array is rank-1, the shape of the returned array has the same shape as
   the index sequence. This, as with many other facets of numarray, is best
   understood by experiment.
\begin{verbatim}
>>> x = arange(10) * 100
>>> print x
[  0 100 200 300 400 500 600 700 800 900]
>>> print take(x, [[2,4],[1,2]])
[[200 400]
 [100 200]]
\end{verbatim}
   A typical example of using \function{take} is to replace the grey values in
   an image according to a "translation table."  For example, suppose \code{m51}
   is a 2-D array of type \code{UInt8} containing a greyscale image. We can
   create a table mapping the integers 0--255 to integers 0--255 using a
   "compressive nonlinearity":
\begin{verbatim}
>>> table = (255 - arange(256)**2 / 256).astype(UInt8)
\end{verbatim}
   Then we can perform the take() operation
\begin{verbatim}
>>> m51b = take(table, m51)
\end{verbatim}
The numarray version of \function{take} can also take a sequence as its 
axis value:
\begin{verbatim}
>>> print take(a2, [0,1], [0,1])
1
>>> print take(a2, [0,1], [1,0])
5
\end{verbatim}
In this case, it functions like indexing: a2[0,1] and a2[1,0] respectively.
\end{funcdesc}


\begin{funcdesc}{put}{array, indices, values, axis=0, clipmode=CLIP}
  \label{func:put}
   \function{put} is the opposite of \function{take}. The values of \var{array}
   at the locations specified in \var{indices} are set to the corresponding
   \var{values}.  The \var{array} must be a contiguous array. The \var{indices}
   can be any integer sequence object with values suitable for indexing into
   the flat form of \var{array}.  The \var{values} must be any sequence of
   values that can be converted to the type of \var{a}.
\begin{verbatim}
>>> x = arange(6)
>>> put(x, [2,4], [20,40])
>>> print x
[ 0  1 20  3 40  5]
\end{verbatim}
   Note that the target \var{array} is not required to be one-dimensional.
   Since \var{array} is contiguous and stored in row-major order, the
   \var{indices} can be treated as indexing \var{array}'s elements in storage
   order.  The routine \function{put} is thus equivalent to the following
   (although the loop is in C for speed):
\begin{verbatim}
ind = array(indices, copy=0)
v = array(values, copy=0).astype(a.type())
for i in range(len(ind)): a.flat[i] = v[i]
\end{verbatim}
\end{funcdesc}


\begin{funcdesc}{putmask}{array, mask, values}
   \function{putmask} sets those elements of \var{array} for which 
   \var{mask} is true to the corresponding value in \var{values}. 
   The array \var{array} must be contiguous. The argument \var{mask} 
   must be an integer sequence of the same size (but not necessarily the 
   same shape) as \var{array}. The argument \var{values} will be 
   repeated as necessary; in particular it can be a
   scalar. The array values must be convertible to the type of \var{array}.
\begin{verbatim}
>>> x=arange(5)
>>> putmask(x, [1,0,1,0,1], [10,20,30,40,50])
>>> print x
[10  1 30  3 50]
>>> putmask(x, [1,0,1,0,1], [-1,-2])
>>> print x
[-1  1 -1  3 -1]
\end{verbatim}
   Note how in the last example, the third argument was treated as if it were
   \code{[-1, -2, -1, -2, -1]}.
\end{funcdesc}


\begin{funcdesc}{transpose}{array, axes=None}
   \function{transpose} takes an array \var{array} and returns a new 
   array which corresponds to \var{a} with the order of axes specified 
   by the second argument \var{axes} which is a sequence of dimension 
   indices.  The default is to reverse the order of all axes, i.e. 
   \code{axes=arange(a.rank)[::-1]}.
\begin{verbatim}
>>> a2=arange(6,shape=(2,3)); print a2
[[0 1 2]
 [3 4 5]]
>>> print transpose(a2)  # same as transpose(a2, axes=(1,0))
[[0 3]
 [1 4]
 [2 5]]
>>> a3=arange(24,shape=(2,3,4)); print a3
[[[ 0  1  2  3]
  [ 4  5  6  7]
  [ 8  9 10 11]]

 [[12 13 14 15]
  [16 17 18 19]
  [20 21 22 23]]]
>>> print transpose(a3)  # same as transpose(a3, axes=(2,1,0))
[[[ 0 12]
  [ 4 16]
  [ 8 20]]

 [[ 1 13]
  [ 5 17]
  [ 9 21]]

 [[ 2 14]
  [ 6 18]
  [10 22]]

 [[ 3 15]
  [ 7 19]
  [11 23]]]
>>> print transpose(a3, axes=(1,0,2))
[[[ 0  1  2  3]
  [12 13 14 15]]

 [[ 4  5  6  7]
  [16 17 18 19]]

 [[ 8  9 10 11]
  [20 21 22 23]]]
\end{verbatim}
\end{funcdesc}


\begin{funcdesc}{repeat}{array, repeats, axis=0}
   \function{repeat} takes an array \var{array} and returns an array 
   with each element in the input array repeated as often as indicated by the
   corresponding elements in the second array. It operates along the specified
   axis. So, to stretch an array evenly, one needs the repeats array to contain
   as many instances of the integer scaling factor as the size of the specified
   axis:
\begin{verbatim}
>>> print a
[[0 1 2]
 [3 4 5]]
>>> print repeat(a, 2*ones(a.shape[0]))   # i.e. repeat(a, (2,2)), broadcast 
                   # rules apply, so this is also equivalent to repeat(a, 2)
[[0 1 2]
 [0 1 2]
 [3 4 5]
 [3 4 5]]
>>> print repeat(a, 2*ones(a.shape[1]), 1)  # i.e. repeat(a, (2,2,2), 1), or
                                            # repeat(a, 2, 1)
[[0 0 1 1 2 2]
 [3 3 4 4 5 5]]
>>> print repeat(a, (1,2))
[[0 1 2]
 [3 4 5]
 [3 4 5]]
\end{verbatim}
\end{funcdesc}


\begin{funcdesc}{where}{condition, x, y, out=None}
  \label{func:where}
   The \function{where} function creates an array whose values are those of
   \var{x} at those indices where \var{condition} is true, and those of \var{y}
   otherwise.  The shape of the result is the shape of \var{condition}. The
   type of the result is determined by the types of \var{x} and \var{y}. Either
   \var{x} or \var{y} (or both) can be a scalar, which is then used for all
   appropriate elements of condition.  \var{out} can be used to specify an
   output array.
\begin{verbatim}
>>> where(arange(10) >= 5, 1, 2)
array([2, 2, 2, 2, 2, 1, 1, 1, 1, 1])
\end{verbatim}

   Starting with numarray-0.6, \function{where} supports a one parameter form
   that is equivalent to the \var{nonzero} function but reads better:

\begin{verbatim}
>>> where(arange(10) % 2)
(array([1, 3, 5, 7, 9]),)   # indices where expression is true 
\end{verbatim}
   Note that in this case, the output is a tuple.

   Like \function{nonzero}, the one parameter form of \function{where} can be
   used to do array indexing:

\begin{verbatim}
>>> a = arange(10,20)
>>> a[ where( a % 2 ) ]
array([11, 13, 15, 17, 19])
\end{verbatim}

   Note that for array indices which are boolean arrays, using \function{where}
   is not necessary but is still OK:

\begin{verbatim}
>>> a[(a % 2) != 0]
array([11, 13, 15, 17, 19])
>>> a[where((a%2) != 0)]
array([11, 13, 15, 17, 19])
\end{verbatim}
\end{funcdesc}

\begin{funcdesc}{choose}{selector, population, outarr=None, clipmode=RAISE}
   The function \function{choose} provides a more general mechanism for
   selecting members of a \var{population} based on a \var{selector} array.
   Unlike \function{where}, \function{choose} can select values from more than
   two arrays.  \var{selector} is an array of integers between \constant{0} and
   \constant{n}. The resulting array will have the same shape as
   \var{selector}, with element selected from \code{population=(b0, ..., bn)}
   as indicated by the value of the corresponding element in \var{selector}.
   Assume \var{selector} is an array that you want to "clip" so that no values
   are greater than \constant{100.0}.
\begin{verbatim}
>>> choose(greater(a, 100.0), (a, 100.0))
\end{verbatim}
   Everywhere that \code{greater(a, 100.0)} is false (i.e.\ \constant{0}) this
   will ``choose'' the corresponding value in \var{a}. Everywhere else 
   it will ``choose'' \constant{100.0}.  This works as well with arrays. 
   Try to figure out what the following does:
\begin{verbatim}
>>> ret = choose(greater(a,b), (c,d))
\end{verbatim}
\end{funcdesc}

\begin{funcdesc}{ravel}{array}
   Returns the argument \var{array} as a 1-D array. It is 
   equivalent to \code{reshape(a, (-1,))}. There is a \method{ravel} 
   method which reshapes the array in-place. Unlike \code{a.ravel()}, 
   however, the \function{ravel} function works with non-contiguous arrays.
\begin{verbatim}
>>> a=arange(25)
>>> a.setshape(5,5)
>>> a.transpose()
>>> a.iscontiguous()
0
>>> a
array([[ 0,  5, 10, 15, 20],
  [ 1,  6, 11, 16, 21],
  [ 2,  7, 12, 17, 22],
  [ 3,  8, 13, 18, 23],
  [ 4,  9, 14, 19, 24]])
>>> a.ravel()
Traceback (most recent call last):
...
TypeError: Can't reshape non-contiguous arrays
>>> ravel(a)
array([ 0,  5, 10, 15, 20,  1,  6, 11, 16, 21,  2,  7, 12, 17, 22,  3,
        8, 13, 18, 23,  4,  9, 14, 19, 24])
\end{verbatim}
\end{funcdesc}


\begin{funcdesc}{nonzero}{a}
   \function{nonzero} returns a tuple of arrays containing the indices of the
   elements in \var{a} that are nonzero.

\begin{verbatim}
>>> a = array([-1, 0, 1, 2])
>>> nonzero(a)
(array([0, 2, 3]),)
>>> print a2
[[-1  0  1  2]
 [ 9  0  4  0]]
>>> print nonzero(a2)
(array([0, 0, 0, 1, 1]), array([0, 2, 3, 0, 2]))
\end{verbatim}
\end{funcdesc}

\begin{funcdesc}{compress}{condition, a, axis=0}
  \label{func:compress}
   Returns those elements of a corresponding to those elements of condition
   that are nonzero. \var{condition} must be the same size as the given axis of
   \var{a}.  Alternately, \var{condition} may match \var{a} in shape; in this
   case the result is a 1D array and \var{axis} should not be specified.
\begin{verbatim}
>>> print x
[1 0 6 2 3 4]
>>> print greater(x, 2)
[0 0 1 0 1 1]
>>> print compress(greater(x, 2), x)
[6 3 4]
>>> print a2
[[-1  0  1  2]
 [ 9  0  4  0]]
>>> print compress(a2>1, a2)
[2 9 4]
>>> a = array([[1,2],[3,4]])
>>> print compress([1,0], a, axis = 1)
[[1]
[3]]
>>> print compress([[1,0],[0,1]], a)
[1, 4]
\end{verbatim}
\end{funcdesc}


\begin{funcdesc}{diagonal}{a, offset=0, axis1=0, axis2=1}
   Returns the entries along the diagonal of \var{a} specified by \var{offset}.
   The \var{offset} is relative to the \var{axis2} axis.  This is designed for
   2-D arrays. For arrays of higher ranks, it will return the diagonal of each
   2-D sub-array.  The 2-D array does not have to be square.

   Warning:  in Numeric (and numarray 0.7 or before), there is a bug in 
   the \function{diagonal} function which will give erronous result for 
   arrays of 3-D or higher.
\begin{verbatim}
>>> print x
[[ 0  1  2  3  4]
 [ 5  6  7  8  9]
 [10 11 12 13 14]
 [15 16 17 18 19]
 [20 21 22 23 24]]
>>> print diagonal(x)
[ 0  6 12 18 24]
>>> print diagonal(x, 1)
[ 1  7 13 19]
>>> print diagonal(x, -1)
[ 5 11 17 23]
\end{verbatim}
\end{funcdesc}


\begin{funcdesc}{trace}{a, offset=0, axis1=0, axis2=1}
   Returns the sum of the elements in a along the diagonal specified by offset.

   Warning:  in Numeric (and numarray 0.7 or before), there is a bug in 
   the \function{trace} function which will give erronous result for 
   arrays of 3-D or higher.
\begin{verbatim}
>>> print x
[[ 0  1  2  3  4]
 [ 5  6  7  8  9]
 [10 11 12 13 14]
 [15 16 17 18 19]
 [20 21 22 23 24]]
>>> print trace(x)                      # 0 + 6 + 12 + 18 + 24
60
>>> print trace(x, -1)                  # 5 + 11 + 17 + 23
56
>>> print trace(x, 1)                   # 1 + 7 + 13 + 19
40
\end{verbatim}
\end{funcdesc}


\begin{funcdesc}{searchsorted}{bin, values}
   Called with a rank-1 array sorted in ascending order,
   \function{searchsorted} will return the indices of the positions in 
   \var{bin} where the corresponding \var{values} would fit.
\begin{verbatim}
>>> print bin_boundaries
[ 0.   0.1  0.2  0.3  0.4  0.5  0.6  0.7  0.8  0.9  1. ]
>>> print data
[ 0.31  0.79  0.82  5.  -2.  -0.1 ]
>>> print searchsorted(bin_boundaries, data)
[4 8 9 11 0 0]
\end{verbatim}
   This can be used for example to write a simple histogramming function:
\begin{verbatim}
>>> def histogram(a, bins):
...         # Note that the argument names below are reverse of the 
...         # searchsorted argument names
...         n = searchsorted(sort(a), bins)
...         n = concatenate([n, [len(a)]])
...         return n[1:]-n[:-1]
...
>>> print histogram([0,0,0,0,0,0,0,.33,.33,.33], arange(0,1.0,.1))
[7 0 0 3 0 0 0 0 0 0]
>>> print histogram(sin(arange(0,10,.2)), arange(-1.2, 1.2, .1))
[0 0 4 2 2 2 0 2 1 2 1 3 1 3 1 3 2 3 2 3 4 9 0 0]
\end{verbatim}
\end{funcdesc}


\begin{funcdesc}{sort}{array, axis=-1}
   This function returns an array containing a copy of the data in 
   \var{array}, with the same shape as \var{array}, but with the 
   order of the elements along the specified \var{axis} sorted. The shape 
   of the returned array is the same as \var{array}'s.  Thus, 
   \code{sort(a, 3)} will be an array of the same shape as \var{array}, 
   where the elements of \var{array} have been sorted along the fourth
   axis.
\begin{verbatim}
>>> print data
[[5 0 1 9 8]
 [2 5 8 3 2]
 [8 0 3 7 0]
 [9 6 9 5 0]
 [9 0 9 7 7]]
>>> print sort(data)                    # Axis -1 by default
[[0 1 5 8 9]
 [2 2 3 5 8]
 [0 0 3 7 8]
 [0 5 6 9 9]
 [0 7 7 9 9]]
>>> print sort(data, 0)
[[2 0 1 3 0]
 [5 0 3 5 0]
 [8 0 8 7 2]
 [9 5 9 7 7]
 [9 6 9 9 8]]
\end{verbatim}
\end{funcdesc}


\begin{funcdesc}{argsort}{array, axis=-1}
   \function{argsort} will return the indices of the elements of the array
   needed to produce \code{sort(array)}. In other words, for a 1-D array,
   \code{take(a.flat, argsort(a))} is the same as \code{sort(a)}... but slower.
\begin{verbatim}
>>> print data
[5 0 1 9 8]
>>> print sort(data)
[0 1 5 8 9]
>>> print argsort(data)
[1 2 0 4 3]
>>> print take(data, argsort(data))
[0 1 5 8 9]
\end{verbatim}
\end{funcdesc}


\begin{funcdesc}{argmax}{array, axis=-1}
\end{funcdesc}
\begin{funcdesc}{argmin}{array, axis=-1}
   The \function{argmax} function returns an array (or scalar for a 1D array)
   with the index(es) of the maximum value(s) of its input \var{array} along
   the given \var{axis}. The returned array will have one less dimension than
   \var{array}. \function{argmin} is just like \function{argmax}, except that
   it returns the indices of the minima along the given axis.
\begin{verbatim}
>>> print data
[[9 6 1 3 0]
 [0 0 8 9 1]
 [7 4 5 4 0]
 [5 2 7 7 1]
 [9 9 7 9 7]]
>>> print argmax(data)
[0 3 0 3 1]
>>> print argmax(data, 0)
[4 4 1 4 4]
>>> print argmin(data)
[4 1 4 4 4]
>>> print argmin(data, 0)
[1 1 0 0 2]
\end{verbatim}
\end{funcdesc}

\begin{funcdesc}{fromstring}{datastring, type, shape=None}
   Will return the array formed by the binary data given in 
   \var{datastring}, of the specified \var{type}. This is mainly 
   used for reading binary data to and from files, it can also be used to 
   exchange binary data with other modules that use python strings as 
   storage (e.g.\ PIL). Note that this representation is dependent on the 
   byte order. To find out the byte ordering used, use the 
   \method{isbyteswapped} method described on page 
   \pageref{arraymethod:byteswap}.  If \var{shape} is not specified, the 
   created array will be one dimensional.
\end{funcdesc}

\begin{funcdesc}{fromfile}{file, type, shape=None}
  If \var{file} is a string then it is interpreted as the name of a 
  file which is opened and read.  Otherwise, \var{file} must be a 
  Python file object which is read as a source of binary array data.  
  \function{fromfile} reads directly into the newly created array buffer 
  with no intermediate string, but otherwise is similar to fromstring, 
  treating the contents of the specified file as a binary data string.
\end{funcdesc}

\begin{funcdesc}{dot}{a, b}
   The \function{dot} function returns the dot product of \var{a} and
   \var{b}. This is equivalent to matrix multiply for rank-2 arrays (without
   the transposition).  This function is identical to the
   \function{matrixmultiply} function.
\begin{verbatim}
>>> print a
[1 2]
>>> print b
[10 11]
# kind of like vector inner product with implicit transposition 
>>> print dot(a,b), dot(b,a) 
32 32
>>> print a
[[1 2]
 [5 7]]
>>> print b
[[  0   1]
 [ 10 100]]
>>> print dot(a,b)
[[ 20 201]
 [ 70 705]]
>>> print dot(b,a)
[[  5   7]
 [510 720]]
\end{verbatim}
\end{funcdesc}

\begin{funcdesc}{matrixmultiply}{a, b}
   This function multiplies matrices or matrices and vectors as matrices rather
   than elementwise. This function is identical to \function{dot}.  Compare:
\begin{verbatim}
>>> print a
[[0 1 2]
 [3 4 5]]
>>> print b
[1 2 3]
>>> print a*b
[[ 0  2  6]
 [ 3  8 15]]
>>> print matrixmultiply(a,b)
[ 8 26]
\end{verbatim}
\end{funcdesc}


\begin{funcdesc}{clip}{m, m_min, m_max}
   The clip function creates an array with the same shape and type as 
   \var{m}, but where every entry in \var{m} that is less than 
   \var{m_min} is replaced by \var{m_min}, and every entry greater 
   than \var{m_max} is replaced by \var{m_max}.  Entries within 
   the range \var{[m_min, m_max]} are left unchanged.
\begin{verbatim}
>>> a = arange(9, type=Float32)
>>> print clip(a, 1.5, 7.5)
[1.5 1.5 2. 3. 4. 5. 6. 7. 7.5]
\end{verbatim}
\end{funcdesc}


\begin{funcdesc}{indices}{shape, type=None}
   The indices function returns an array corresponding to the \var{shape} 
   given. The array returned is an array of a new shape which is based on 
   the specified \var{shape}, but has an added dimension of length 
   the number of dimensions in the specified shape.  For example, if 
   \code{shape=(3,4)}, then the shape of the array returned will be
   \code{(2,3,4)} since the length of \code{(3,4)} is \var{2} and if 
   \code{shape=(5,6,7)}, the returned array's shape will be \code{(3,5,6,7)}. 
   The contents of the returned arrays are such that the \var{i}th subarray 
   (along index 0, the first dimension) contains the indices for that axis 
   of the elements in the array.  An example makes things clearer:
\begin{verbatim}
>>> i = indices((4,3))
>>> i.getshape()
(2, 4, 3)
>>> print i[0]
[[0 0 0]
 [1 1 1]
 [2 2 2]
 [3 3 3]]
>>> print i[1]
[[0 1 2]
 [0 1 2]
 [0 1 2]
 [0 1 2]]
\end{verbatim}
   So, \code{i[0]} has an array of the specified shape, and each element in
   that array specifies the index of that position in the subarray for axis 0.
   Similarly, each element in the subarray in \code{i[1]} contains the index of
   that position in the subarray for axis 1.
\end{funcdesc}


\begin{funcdesc}{swapaxes}{array, axis1, axis2}
   Returns a new array which \var{shares} the data of \var{array}, but 
   has the two axes specified by \var{axis1} and \var{axis2} 
   swapped. If \var{array} is of rank 0 or 1, swapaxes simply returns a 
   new reference to \var{array}.
\begin{verbatim}
>>> x = arange(10)
>>> x.setshape((5,2,1))
>>> print x
[[[0]
  [1]]

 [[2]
  [3]]

 [[4]
  [5]]

 [[6]
  [7]]

 [[8]
  [9]]]
>>> y = swapaxes(x, 0, 2)
>>> y.getshape()
(1, 2, 5)
>>> print y
[[[0 2 4 6 8]
 [1 3 5 7 9]]]
\end{verbatim}
\end{funcdesc}


\begin{funcdesc}{concatenate}{arrs, axis=0}
   Returns a new array containing copies of the data contained in all arrays
   of \var{arrs= (a0, a1, ... an)}.  The arrays \var{ai} will be 
   concatenated along the specified \var{axis} (default=0). All 
   arrays \var{ai} must have the same shape along every axis except for 
   the one specified in \var{axis}. To concatenate arrays along a
   newly created axis, you can use \code{array((a0, ..., an))}, as long as all
   arrays have the same shape.
\begin{verbatim}
>>> print x
[[ 0  1  2  3]
 [ 5  6  7  8]
 [10 11 12 13]]
>>> print concatenate((x,x))
[[ 0  1  2  3]
 [ 5  6  7  8]
 [10 11 12 13]
 [ 0  1  2  3]
 [ 5  6  7  8]
 [10 11 12 13]]
>>> print concatenate((x,x), 1)
[[ 0  1  2  3  0  1  2  3]
 [ 5  6  7  8  5  6  7  8]
 [10 11 12 13 10 11 12 13]]
>>> print array((x,x))   # Note: one extra dimension
[[[ 0  1  2  3]
  [ 5  6  7  8]
  [10 11 12 13]]
 [[ 0  1  2  3]
  [ 5  6  7  8]
  [10 11 12 13]]]
>>> print a
[[1 2]]
>>> print b
[[3 4 5]]
>>> print concatenate((a,b),1)
[[1 2 3 4 5]]
>>> print concatenate((a,b),0)
ValueError: _concat array shapes must match except 1st dimension
\end{verbatim}
\end{funcdesc}


\begin{funcdesc}{innerproduct}{a, b}
   \function{innerproduct} produces the inner product of arrays \var{a} and
   \var{b}. It is equivalent to \code{matrixmultiply(a, transpose(b))}.
\end{funcdesc}


\begin{funcdesc}{outerproduct}{a,b}
   \function{outerproduct} produces the outer product of vectors \var{a} and
   \var{b}, that is \code{result[i, j] = a[i] * b[j]}.
\end{funcdesc}


\begin{funcdesc}{array_repr}{a, max_line_width=None, precision=None, supress_small=None}
   See section \ref{TBD} on Textual Representations of arrays.
\end{funcdesc}


\begin{funcdesc}{array_str}{a, max_line_width=None, precision=None, supress_small=None}
   See section \ref{TBD} Textual Representations of arrays.
\begin{verbatim}
>>> print a
[  1.00000000e+00   1.10000000e+00   1.11600000e+00   1.11380000e+00
   1.20000000e-02   1.34560000e-04]
>>> print array_str(a,precision=4,suppress_small=1)
[ 1.      1.1     1.116   1.1138  0.012   0.0001]
>>> print array_str(a,precision=3,suppress_small=1)
[ 1.     1.1    1.116  1.114  0.012  0.   ]
>>> print array_str(a,precision=3)
[  1.000e+00   1.100e+00   1.116e+00   1.114e+00   1.200e-02
   1.346e-04]
\end{verbatim}
\end{funcdesc}


\begin{funcdesc}{resize}{array, shape}
  \label{func:resize}
   The \function{resize} function takes an array and a shape, and returns a new
   array with the specified \var{shape}, and filled with the data in 
   the input \var{array}.  Unlike the \function{reshape} function, the 
   new shape does not have to yield the same size as the original array. 
   If the new size of is less than that of the input \var{array}, the 
   returned array contains the appropriate data from the "beginning" of the 
   old array. If the new size is greater than that of the input array, the 
   data in the input \var{array} is repeated as many times as needed
   to fill the new array.
\begin{verbatim}
>>> x = arange(10)
>>> y = resize(x, (4,2))                # note that 4*2 < 10
>>> print x
[0 1 2 3 4 5 6 7 8 9]
>>> print y
[[0 1]
 [2 3]
 [4 5]
 [6 7]]
>>> print resize(array((0,1)), (5,5))   # note that 5*5 > 2
[[0 1 0 1 0]
 [1 0 1 0 1]
 [0 1 0 1 0]
 [1 0 1 0 1]
 [0 1 0 1 0]]
\end{verbatim}
\end{funcdesc}


\begin{funcdesc}{identity}{n, type=None}
   The identity function returns an \var{n} by \var{n} array 
   where the diagonal elements are 1, and the off-diagonal elements are 0.
\begin{verbatim}
>>> print identity(5)
[[1 0 0 0 0]
 [0 1 0 0 0]
 [0 0 1 0 0]
 [0 0 0 1 0]
 [0 0 0 0 1]]
\end{verbatim}
\end{funcdesc}


\begin{funcdesc}{sum}{a, axis=0}
   The sum function is a synonym for the \method{reduce} method of the
   \function{add} ufunc. It returns the sum of all of the elements in the
   sequence given along the specified axis (first axis by default).
\begin{verbatim}
>>> print x
[[ 0  1  2  3]
 [ 4  5  6  7]
 [ 8  9 10 11]
 [12 13 14 15]
 [16 17 18 19]]
>>> print sum(x)
[40 45 50 55]                           # 0+4+8+12+16, 1+5+9+13+17,
2+6+10+14+18, ...
>>> print sum(x, 1)
[ 6 22 38 54 70]                        # 0+1+2+3, 4+5+6+7, 8+9+10+11, ...
\end{verbatim}
\end{funcdesc}


\begin{funcdesc}{cumsum}{a, axis=0}
   The cumsum function is a synonym for the \method{accumulate} method of the
   \function{add} ufunc.
\end{funcdesc}


\begin{funcdesc}{product}{a, axis=0}
   The product function is a synonym for the \method{reduce} method of the
   \function{multiply} ufunc.
\end{funcdesc}


\begin{funcdesc}{cumproduct}{a, axis=0}
   The cumproduct function is a synonym for the \method{accumulate} method of
   the \function{multiply} ufunc.
\end{funcdesc}


\begin{funcdesc}{alltrue}{a, axis=0}
   The alltrue function is a synonym for the \method{reduce} method of the
   \function{logical_and} ufunc.
\end{funcdesc}


\begin{funcdesc}{sometrue}{a, axis=0}
   The sometrue function is a synonym for the \method{reduce} method of the
   \function{logical_or} ufunc.
\end{funcdesc}


\begin{funcdesc}{all}{a}
   \function{all} is a synonym for the \method{reduce} method of the
   \function{logical_and} ufunc, preceded by a \function{ravel} which converts
   arrays with \(rank>1\) to \(rank=1\).  Thus, \function{all} tests that all
   the elements of a multidimensional array are nonzero.
\end{funcdesc}


\begin{funcdesc}{any}{a}
   The \function{any} function is a synonym for the \method{reduce} method of
   the \function{logical_and} ufunc, preceded by a \function{ravel} which
   converts arrays with \(rank>1\) to \(rank=1\).  Thus, \function{any} tests
   that at least one of the elements of a multidimensional array is nonzero.
\end{funcdesc}


\begin{funcdesc}{allclose}{a, b, rtol=1.e-5, atol=1.e-8}
   This function tests whether or not arrays \var{x} and \var{y} 
   of an integer or real type are equal subject to the given relative and 
   absolute tolerances: \code{rtol, atol}. The formula used is:
   \begin{equation}
      \left| x - y \right| < atol + rtol * \left| y \right|
   \end{equation}
   This means essentially that both elements are small compared to \var{atol}
   or their difference divided by \var{y}'s value is small compared to
   \var{rtol}.
\end{funcdesc}



\begin{seealso}
   \seemodule{numarray.convolve}{The \function{convolve} function is implemented in the
      optional \module{numarray.convolve} package.}%
   \seemodule{numarray.convolve}{The \function{correlation} function is implemented in
      the optional \module{numarray.convolve} package.}%
\end{seealso} 




%% Local Variables:
%% mode: LaTeX
%% mode: auto-fill
%% fill-column: 79
%% indent-tabs-mode: nil
%% ispell-dictionary: "american"
%% reftex-fref-is-default: nil
%% TeX-auto-save: t
%% TeX-command-default: "pdfeLaTeX"
%% TeX-master: "numarray"
%% TeX-parse-self: t
%% End:


\chapter{Array Methods}
\label{cha:array-methods}

As we discussed at the beginning of the last chapter, there are very few array
methods for good reasons, and these all depend on the implementation
details. They're worth knowing, though.

\begin{methoddesc}[numarray]{argmax}{axis=-1}
  \label{arraymethod:argmax}
  The \method{argmax} method returns the index of the largest element in a 1D
  array.  In the case of a multi-dimensional array, it returns and array of
  indices.
\begin{verbatim}
>>> array([1,2,4,3]).argmax()
2
>>> arange(100, shape=(10,10)).argmax()
array([9, 9, 9, 9, 9, 9, 9, 9, 9, 9])
\end{verbatim}
\end{methoddesc}


\begin{methoddesc}[numarray]{argmin}{axis=-1}
  \label{arraymethod:argmin}
  The \method{argmin} method returns the index of the smallest element in a 1D
  array.  In the case of a multi-dimensional array, it returns and array of
  indices.
\end{methoddesc}


\begin{methoddesc}[numarray]{argsort}{axis=-1}
  \label{arraymethod:argsort}
  The \method{argsort} method returns the array of indices which if taken from
  the array using \function{take} would return a sorted copy of the array.  For
  multi-dimensional arrays, \method{argsort} computes the indices for each 1D
  subarray independently and aggregates them all into a single array result;
  The \method{argsort} of a multi-dimensional array does not produce a sorted
  copy of the array when applied directly to it using \function{take}; instead,
  each 1D subarray must be passed to \function{take} independently.
\begin{verbatim}
  >>> array([1,2,4,3]).argsort()
  array([0, 1, 3, 2])
  >>> take([1,2,4,3], argsort([1,2,4,3]))
  array([1, 2, 3, 4])
\end{verbatim}
\end{methoddesc}


\begin{methoddesc}[numarray]{astype}{type}
  \label{arraymethod:astype}
  The \method{astype} method returns a copy of the array converted to the
  specified type.  As with any copy, the new array is aligned, contiguous, and
  in native machine byte order.  If the specified type is the same as current
  type, a copy is \emph{still} made.
\begin{verbatim}
  >>> arange(5).astype('Float64')
  array([ 0.,  1.,  2.,  3.,  4.])
\end{verbatim}
\end{methoddesc}


\begin{methoddesc}[numarray]{byteswap}{}
   \label{arraymethod:byteswap}
   The \method{byteswap} method performs a byte swapping operation on all the
   elements in the array, working inplace (i.e.\ it returns None).
   \method{byteswap} does not affect the array's byte order state variable.
   See \method{togglebyteorder} for changing the array's byte order state
   in addition to or rather than physically swapping bytes.
\begin{verbatim}
>>> print a
[1 2 3]
>>> a.byteswap()
>>> print a
[16777216 33554432 50331648]
\end{verbatim}
\end{methoddesc}


\begin{methoddesc}[numarray]{byteswapped}{}
  \label{arraymethod:byteswapped} 
  The \method{byteswapped} method returns a byteswapped copy of the array.
  \method{byteswapped} does not affect the array's own byte order state
  variable.  The result of \method{byteswapped} is logically in native byte
  order.
\begin{verbatim}
>>> array([1,2,3]).byteswapped()
array([16777216, 33554432, 50331648])
\end{verbatim}
\end{methoddesc}


\begin{methoddesc}[numarray]{conjugate}{}
  \label{arraymethod:conjugate}
   The \method{conjugate} method returns the complex conjugate of an array.
\begin{verbatim}
>>> (arange(3) + 1j).conjugate()
array([ 0.-1.j,  1.-1.j,  2.-1.j])
\end{verbatim}
\end{methoddesc}


\begin{methoddesc}[numarray]{copy}{}
  \label{arraymethod:copy}
   The \method{copy} method returns a copy of an array. When making an
   assignment or taking a slice, a new array object is created and has its own
   attributes, except that the data attribute just points to the data of the
   first array (a "view").  The \method{copy} method is used when it is
   important to obtain an independent copy.  \method{copy} returns arrays which
   are contiguous, aligned, and not byteswapped, i.e. well behaved.
\begin{verbatim}
>>> c = a[3:8:2].copy()
>>> print c.iscontiguous()
1
\end{verbatim}
\end{methoddesc}


\begin{methoddesc}[numarray]{diagonal}{}
  \label{arraymethod:diagonal}
   The \method{diagonal} method returns the diagonal elements of the array,
   those elements where the row and column indices are equal.
\begin{verbatim}
>>> arange(25,shape=(5,5)).diagonal()
array([ 0,  6, 12, 18, 24])
\end{verbatim}
\end{methoddesc}


\begin{methoddesc}[numarray]{info}{}
   \label{arraymethod:info} Calling an array's \method{info}
   method prints out information about the array which is useful for debugging.
\begin{verbatim}
>>> arange(10).info()
class: <class 'numarray.numarraycore.NumArray'>
shape: (10,)
strides: (4,)
byteoffset: 0
bytestride: 4
itemsize: 4
aligned: 1
contiguous: 1
data: <memory at 0x08931d18 with size:0x00000028 held by object 0x3ff91bd8 aliasing object 0x00000000>
byteorder: little
byteswap: 0
type: Int32
\end{verbatim}
\end{methoddesc}


\begin{methoddesc}[numarray]{isaligned}{}
  \label{arraymethod:isaligned} \method{isaligned} returns 1 IFF the buffer
  address for an array modulo the array itemsize is 0.  When the array
  itemsize exceeds 8 (sizeof(double)) aligment is done modulo 8.
\end{methoddesc}


\begin{methoddesc}[numarray]{isbyteswapped}{}
  \label{arraymethod:isbyteswapped} \method{isbyteswapped} returns 1 IFF the 
  array's binary data is not in native machine byte order, possibly because it
  originated on a machine with a different native order.
\end{methoddesc}


\begin{methoddesc}[numarray]{iscontiguous}{}
  \label{arraymethod:iscontiguous} \method{iscontiguous} returns 1 IFF
   an array is C-contiguous and 0 otherwise.  An array is C-contiguous if its
   smallest stride corresponds to the innermost dimension and its other strides
   strictly increase in size from the innermost dimension to the outermost,
   with each stride being the product of the previous inner stride and shape.
   A non-contiguous array can be converted to a contiguous array by the
   \method{copy} method.
\begin{verbatim}
>>> a=arange(25, shape=(5,5))
>>> a
array([[ 0,  1,  2,  3,  4],
       [ 5,  6,  7,  8,  9],
       [10, 11, 12, 13, 14],
       [15, 16, 17, 18, 19],
       [20, 21, 22, 23, 24]])
>>> a.iscontiguous()
1
\end{verbatim}
\end{methoddesc}


\begin{methoddesc}[numarray]{is_c_array}{}
   \label{arraymethod:is-c-array} 
   \method{is_c_array} returns 1 IFF an array is C-contiguous, aligned, and
   not byteswapped, and returns 0 otherwise.
\begin{verbatim}
>>> a=arange(25, shape=(5,5))
>>> a.is_c_array()
1
>>> a.is_f_array()
0
\end{verbatim}
\end{methoddesc}


\begin{methoddesc}[numarray]{is_fortran_contiguous}{}
   \label{arraymethod:is-fortran-contiguous} 
   \method{is_fortran_contiguous} returns 1 IFF an array is Fortran-contiguous
   and 0 otherwise.  An array is Fortran-contiguous if its smallest stride
   corresponds to its outermost dimension and each succesive stride is the
   product of the previous stride and shape element.
\begin{verbatim}
>>> a=arange(25, shape=(5,5))
>>> a.transpose()
>>> a
array([[ 0,  5, 10, 15, 20],
       [ 1,  6, 11, 16, 21],
       [ 2,  7, 12, 17, 22],
       [ 3,  8, 13, 18, 23],
       [ 4,  9, 14, 19, 24]])
>>> a.iscontiguous()
0
>>> a.is_fortran_contiguous()
1
\end{verbatim}
\end{methoddesc}


\begin{methoddesc}[numarray]{is_f_array}{}
   \label{arraymethod:is-f-array} \method{is_f_array} returns 1 IFF
   an array is Fortran-contiguous, aligned, and not byteswapped, and returns 0
   otherwise.
\begin{verbatim}
>>> a=arange(25, shape=(5,5))
>>> a.transpose()
>>> a.is_f_array()
1
>>> a.is_c_array()
0
\end{verbatim}
\end{methoddesc}


\begin{methoddesc}[numarray]{itemsize}{}
  \label{arraymethod:itemsize} The \method{itemsize} method 
  returns the number of bytes used by any one of its elements.
\begin{verbatim}
>>> a = arange(10)
>>> a.itemsize()
4
>>> a = array([1.0])
>>> a.itemsize()
8
>>> a = array([1], type=Complex64)
>>> a.itemsize()
16
\end{verbatim}
\end{methoddesc}


\begin{methoddesc}[numarray]{max}{}
  \label{arraymethod:max}
  The \method{max} method returns the largest element in an array.
\begin{verbatim}
>>> arange(100, shape=(10,10)).max()
99
\end{verbatim}
\end{methoddesc}
\begin{methoddesc}[numarray]{mean}{}
  \label{arraymethod:mean}
  The \method{mean} method returns the average of all elements in an array.
\begin{verbatim}
>>> arange(10).mean() 4.5
\end{verbatim}
\end{methoddesc}
\begin{methoddesc}[numarray]{min}{}
  \label{arraymethod:min}
  The \method{min} method returns the smallest element in an array.
\begin{verbatim}
>>> arange(10).min()
0
\end{verbatim}
\end{methoddesc}


\begin{methoddesc}[numarray]{nelements}{}
  \label{arraymethod:nelements}
  \method{nelements} returns the total number of elements in this array.
  Synonymous with \method{size}.
\begin{verbatim}
>>> arange(100).nelements()
100
\end{verbatim}
\end{methoddesc}


\begin{methoddesc}[numarray]{new}{type=None}
  \label{arraymethod:new}
   \method{new} returns a new array of the specified type with the same shape
   as this array.  The new array is uninitialized.
\end{methoddesc}


\begin{methoddesc}[numarray]{nonzero}{axis=-1}
  \label{arraymethod:nonzero}
   \method{nonzero} returns a tuple of arrays containing the indices of the
   elements that are nonzero.
\begin{verbatim}
>>> arange(5).nonzero()
(array([1, 2, 3, 4]),)
>>> b = arange(9, shape=(3,3)) % 2; b
array([[0, 1, 0],
       [1, 0, 1],
       [0, 1, 0]])
>>>b.nonzero()
(array([0, 1, 1, 2]), array([1, 0, 2, 1]))
\end{verbatim}
\end{methoddesc}


\begin{methoddesc}[numarray]{repeat}{r, axis=0}
  \label{arraymethod:repeat}
   The \method{repeat} method returns a new array with each element self[i]
   (along the specified axis) repeated r[i] times.
\begin{verbatim}
>>> a=arange(25, shape=(5,5))
>>> a
array([[ 0,  1,  2,  3,  4],
       [ 5,  6,  7,  8,  9],
       [10, 11, 12, 13, 14],
       [15, 16, 17, 18, 19],
       [20, 21, 22, 23, 24]])
>>> a.repeat(arange(5)%2*2)
array([[ 5,  6,  7,  8,  9],
       [ 5,  6,  7,  8,  9],
       [15, 16, 17, 18, 19],
       [15, 16, 17, 18, 19]])
\end{verbatim}
\end{methoddesc}


\begin{methoddesc}[numarray]{resize}{shape}
  \label{arraymethod:resize}
   \method{resize} shrinks/grows the array to new \var{shape}, possibly
    replacing the underlying buffer object.
\begin{verbatim}
>>> a = array([0, 1, 2, 3])
>>> a.resize(10)
array([0, 1, 2, 3, 0, 1, 2, 3, 0, 1])
\end{verbatim}
\end{methoddesc}


\begin{methoddesc}[numarray]{size}{}
  \label{arraymethod:size}
  \method{size} returns the total number of elements in this array.
  Synonymous with \method{nelements}.
\begin{verbatim}
>>> arange(100).size()
100
\end{verbatim}
\end{methoddesc}


\begin{methoddesc}[numarray]{type}{}
  \label{arraymethod:type}
   The \method{type} method returns the type of the array it is applied to.
   While we've been talking about them as Float32, Int16, etc., it is important
   to note that they are not character strings, they are instances of
   NumericType classes. 
\begin{verbatim}
>>> a = array([1,2,3])
>>> a.type()
Int32
>>> a = array([1], type=Complex64)
>>> a.type()
Complex64
\end{verbatim}
\end{methoddesc}


\begin{methoddesc}[numarray]{typecode}{}
  \label{arraymethod:typecode}
   The \method{typecode} method returns the typecode character of the array it
   is applied to.  \method{typecode} exists for backward compatibility with
   Numeric but the \method{type} method is preferred.
\begin{verbatim}
>>> a = array([1,2,3])
>>> a.typecode()
'l'
>>> a = array([1], type=Complex64)
>>> a.typecode()
'D'
\end{verbatim}
\end{methoddesc}


\begin{methoddesc}[numarray]{tofile}{file}
  \label{arraymethod:tofile}
  The \method{tofile} method writes the binary data of the array into
  \constant{file}.  If \constant{file} is a Python string, it is interpreted 
  as the name of a file to be created.  Otherwise, \constant{file} must be 
  Python file object to which the data will be written.  
\begin{verbatim}
>>> a = arange(65,100)
>>> a.tofile('test.dat')   # writes a's binary data to file 'test.dat'.
>>> f = open('test2.dat', 'w')
>>> a.tofile(f)            # writes a's binary data to file 'test2.dat'
\end{verbatim}
   Note that the binary representation of array data depends on the platform,
   with some platforms being little endian (sys.byteorder == 'little') and
   others being big endian.  The byte order of the array data is \emph{not}
   recorded in the file, nor are the array's shape and type.
\end{methoddesc}


\begin{methoddesc}[numarray]{tolist}{}
  \label{arraymethod:tolist}
   Calling an array's \method{tolist} method returns a hierarchical python list
   version of the same array:
\begin{verbatim}
>>> print a
[[65 66 67 68 69 70 71]
 [72 73 74 75 76 77 78]
 [79 80 81 82 83 84 85]
 [86 87 88 89 90 91 92]
 [93 94 95 96 97 98 99]]
>>> print a.tolist()
[[65, 66, 67, 68, 69, 70, 71], [72, 73, 74, 75, 76, 77, 78], [79, 80,
81, 82, 83, 84, 85], [86, 87, 88, 89, 90, 91, 92], [93, 94, 95, 96, 97,
98, 99]]
\end{verbatim}
\end{methoddesc}


\begin{methoddesc}[numarray]{tostring}{}
  \label{arraymethod:tostring}
   The \method{tostring} method returns a string representation of the 
   array data.
\begin{verbatim}
>>> a = arange(65,70)
>>> a.tostring()
'A\x00\x00\x00B\x00\x00\x00C\x00\x00\x00D\x00\x00\x00E\x00\x00\x00'
\end{verbatim}
Note that the arangement of the printable characters and interspersed NULL
characters is dependent on machine architecture.  The layout shown here is
for little endian platform.
\end{methoddesc}


\begin{methoddesc}[numarray]{transpose}{axis=-1}
  \label{arraymethod:transpose}
  \method{transpose} re-shapes the array by permuting it's dimensions
  as specified by 'axes'.  If 'axes' is none, \method{transpose}
  reverses the array's dimensions.  \method{transpose} operates
  in-place and returns None.
\begin{verbatim}
>>> a = arange(9, shape=(3,3))
>>> a.transpose()
>>> a
array([[0, 3, 6],
       [1, 4, 7],
       [2, 5, 8]])
\end{verbatim}
\end{methoddesc}


\begin{methoddesc}[numarray]{stddev}{}
  \label{arraymethod:stddev}
  The \method{stddev} method returns the standard deviation of all elements in
  an array.
\begin{verbatim}
>>> arange(10).stddev()
3.0276503540974917
\end{verbatim}
\end{methoddesc}


\begin{methoddesc}[numarray]{sum}{}
  \label{arraymethod:sum}
  The \method{sum} method returns the sum of all elements in an array.
\begin{verbatim}
>>> arange(10).sum()
45
\end{verbatim}
\end{methoddesc}


\begin{methoddesc}[numarray]{swapaxes}{axis1, axis2}
  \label{arraymethod:swapaxes}
  The \method{swapaxes} method adjusts the strides of an array so that
  the two specified axes appear to be swapped.  \method{swapaxes} operates
  in place and returns None.
\begin{verbatim}
>>> a = arange(25, shape=(5,5))
>>> a.swapaxes(0,1)
>>> a
array([[ 0,  5, 10, 15, 20],
       [ 1,  6, 11, 16, 21],
       [ 2,  7, 12, 17, 22],
       [ 3,  8, 13, 18, 23],
       [ 4,  9, 14, 19, 24]])
\end{verbatim}
\end{methoddesc}


\begin{methoddesc}[numarray]{togglebyteorder}{}
  \label{arraymethod:togglebyteorder}
  The \method{togglebyteorder} method adjusts the byte order state 
  variable for an array, with ``little'' being replaced by ``big'' and ``big''
  being replaced by ``little''.  \method{togglebyteorder} just reinterprets
  the existing data, it does not actually rearrange bytes.
\begin{verbatim}
>>> a = arange(4)
>>> a.togglebyteorder()
>>> a
array([       0, 16777216, 33554432, 50331648])
\end{verbatim}
\end{methoddesc}

\begin{methoddesc}[numarray]{trace}{}
  \label{arraymethod:togglebyteorder}
  The \method{trace} method returns the sum of the diagonal elements
  of an array.
\begin{verbatim}
>>> a = arange(25, shape=(5,5))
>>> a.trace()
60
\end{verbatim}
\end{methoddesc}


\begin{methoddesc}[numarray]{view}{}
  \label{arraymethod:view} The \method{view} method returns a new
  state object for an array but does not actually copy the array's
  data; views are used to reinterpret an existing data buffer by 
  changing the array's properties.
\begin{verbatim}
>>> a = arange(4)
>>> b = a.view()
>>> b.shape = (2,2)
>>> a
array([0, 1, 2, 3])
>>> b
array([[0, 1],
       [2, 3]])
>>> a is b
False
>>> a._data is b._data
True
\end{verbatim}
\end{methoddesc}


When using Python 2.2 or later, there are four public attributes which
correspond to those of Numeric type objects. These are \member{shape},
\member{flat}, \member{real}, and \member{imag} (or \member{imaginary}). The
following methods are used to implement and provide an alternative to using
these attributes.


\begin{methoddesc}[numarray]{getshape}{}
\end{methoddesc}
\begin{methoddesc}[numarray]{setshape}{}
   The \method{getshape} method returns the tuple that gives the shape of the
   array.  \method{setshape} assigns its argument (a tuple) to the internal
   attribute which defines the array shape. When using Python 2.2 or later, the
   \member{shape} attribute can be accessed or assigned to, which is equivalent
   to using these methods.
\begin{verbatim}
>>> a = arange(12)
>>> a.setshape((3,4))
>>> print a.getshape()
(3, 4)
>>> print a
[[ 0  1  2  3]
 [ 4  5  6  7]
 [ 8  9 10 11]]
\end{verbatim}
\end{methoddesc}


\begin{methoddesc}[numarray]{getflat}{}
   The \method{getflat} method is equivalent to using the \member{flat}
   attribute of Numeric. For compatibility with Numeric, there is no
   \method{setflat} method, although the attribute can in fact be set using
   \method{setshape}.
\begin{verbatim}
>>> print a
[[ 0  1  2  3]
 [ 4  5  6  7]
 [ 8  9 10 11]]
>>> print a.getflat()
[ 0  1  2  3  4  5  6  7  8  9 10 11]
\end{verbatim}
\end{methoddesc}


\begin{methoddesc}[numarray]{getreal}{}
\end{methoddesc}
\begin{methoddesc}[numarray]{setreal}{}
   The \method{getreal} and \method{setreal} methods can be used to access or
   assign to the real part of an array containing imaginary elements.
\end{methoddesc}


\begin{methoddesc}[numarray]{getimag}{}
\end{methoddesc}
\begin{methoddesc}[numarray]{getimaginary}{}
\end{methoddesc}
\begin{methoddesc}[numarray]{setimag}{}
\end{methoddesc}
\begin{methoddesc}[numarray]{setimaginary}{}
   The \method{getimag} and \method{setimag} methods can be used to access or
   assign to the imaginary part of an array containing imaginary elements.
   \method{getimaginary} is equivalent to \method{getimag}, and
   \method{setimaginary} is equivalent to \method{setimag}.
\end{methoddesc}

%% Local Variables:
%% mode: LaTeX
%% mode: auto-fill
%% fill-column: 79
%% indent-tabs-mode: nil
%% ispell-dictionary: "american"
%% reftex-fref-is-default: nil
%% TeX-auto-save: t
%% TeX-command-default: "pdfeLaTeX"
%% TeX-master: "numarray"
%% TeX-parse-self: t
%% End:

\chapter{Array Attributes}
\label{cha:array-attributes}

There are four public array attributes; however, they are only available 
in Python 2.2 or later. There are array methods that may be used instead. The
attributes are \code{shape, flat, real,} and \code{imaginary}.


\begin{memberdesc}[numarray]{shape}
   Accessing the \member{shape} attribute is equivalent to calling the
   \method{getshape} method; it returns the shape tuple.  Assigning a value to
   the shape attribute is equivalent to calling the \method{setshape} method.
\begin{verbatim}
>>> print a
[[0 1 2]
 [3 4 5]
 [6 7 8]]
>>> print a.shape
(3,3)
>>> a.shape = ((9,))
>>> print a.shape
(9,)
\end{verbatim}
\end{memberdesc}


\begin{memberdesc}[numarray]{flat}
   \label{mem:numarray:flat}
   Accessing the flat attribute of an array returns the flattened, or
   \method{ravel}ed version of that array, without having to do a function
   call.  This is equivalent to calling the \method{getflat} method. The
   returned array has the same number of elements as the input array, but it is
   of rank-1. One cannot set the flat attribute of an array, but one can use
   the indexing and slicing notations to modify the contents of the array:
\begin{verbatim}
>> print a
[[0 1 2]
 [3 4 5]
 [6 7 8]]
>> print a.flat
0 1 2 3 4 5 6 7 8]
>> a.flat[4] = 100
>> print a
[[  0   1   2]
 [  3 100   5]
 [  6   7   8]]
>> a.flat = arange(9,18)
>> print a
[[ 9 10 11]
 [12 13 14]
 [15 16 17]]
\end{verbatim}
\end{memberdesc}


\begin{memberdesc}[numarray]{real}
\end{memberdesc}
\begin{memberdesc}[numarray]{imag}
\end{memberdesc}
\begin{memberdesc}[numarray]{imaginary}
   These attributes exist only for complex arrays. They return respectively
   arrays filled with the real and imaginary parts of the elements. The
   equivalent methods for getting and setting these values are
   \method{getreal}, \method{setreal}, \method{getimag}, and \method{setimag}.
   \method{getimaginary} and \method{setimaginary} are synonyms for
   \method{getimag} and \method{setimag} respectively, and \method{.imag} is a
   synonym for \method{.imaginary}.  The arrays returned are not contiguous
   (except for arrays of length 1, which are always contiguous).
   The attributes \member{real}, \member{imag}, and \member{imaginary} are 
   modifiable:
\begin{verbatim}
>>> print x
[ 0.             +1.j               0.84147098+0.54030231j 0.90929743-0.41614684j]
>>> print x.real
[ 0.          0.84147098  0.90929743]
>>> print x.imag
[ 1.          0.54030231 -0.41614684]
>>> x.imag = arange(3)
>>> print x
[ 0.        +0.j  0.84147098+1.j  0.90929743+2.j]
>>> x = reshape(arange(10), (2,5)) + 0j # make complex array
>>> print x
[[ 0.+0.j  1.+0.j  2.+0.j  3.+0.j  4.+0.j]
 [ 5.+0.j  6.+0.j  7.+0.j  8.+0.j  9.+0.j]]
>>> print x.real
[[ 0.  1.  2.  3.  4.]
 [ 5.  6.  7.  8.  9.]]
>>> print x.type(), x.real.type()
Complex64 Float64
>>> print x.itemsize(), x.imag.itemsize()
16 8
\end{verbatim}
\end{memberdesc}


%% Local Variables:
%% mode: LaTeX
%% mode: auto-fill
%% fill-column: 79
%% indent-tabs-mode: nil
%% ispell-dictionary: "american"
%% reftex-fref-is-default: nil
%% TeX-auto-save: t
%% TeX-command-default: "pdfeLaTeX"
%% TeX-master: "numarray"
%% TeX-parse-self: t
%% End:

\chapter{Character Array}
\label{cha:character-array}
\declaremodule{extension}{numarray.strings}
\index{character array}
\index{string array}

\section{Introduction}
\label{sec:chararray-intro}

\code{numarray}, like \code{Numeric}, has support for arrays of character data
(provided by the \code{numarray.strings} module) in addition to arrays of
numbers.  The support for character arrays in \code{Numeric} is relatively
limited, restricted to arrays of single characters.  In contrast,
\code{numarray} supports arrays of fixed length strings.  As an additional
enhancement, the \code{numarray} design supports interleaving arrays of
characters with arrays of numbers, with both occupying the same memory buffer.
This provides basic infrastructure for building the arrays of heterogenous
records as provided by \code{numarray.records} (see chapter
\ref{cha:record-array}).  Currently, neither \code{Numeric} nor \code{numarray}
provides support for unicode.

Each character array is a \index{CharArray} \code{CharArray} object in the
\code{numarray.strings} module.  The easiest way to construct a character array
is to use the \code{numarray.strings.array()} function.  For example:

\begin{verbatim}
  >>> import numarray.strings as str
  >>> s = str.array(['Smith', 'Johnson', 'Williams', 'Miller'])
  >>> print s
  ['Smith', 'Johnson', 'Williams', 'Miller']
  >>> s.itemsize()
  8
\end{verbatim}
In this example, this string array has 4 elements.  The maximum string length
is automatically determined from the data.  In this case, the created array will
support fixed length strings of 8 characters (since the longest name is 8
characters long).

The character array is just like an array in numarray, except that now each
element is conceptually a Python string rather than a number.  We can do the
usual indexing and slicing:

\begin{verbatim}
  >>> print s[0]
  'Smith'
  >>> print s[:2]
  ['Smith', 'Johnson']
  >>> s[:2] = 'changed'
  >>> print s
  ['changed', 'changed', 'Williams', 'Miller']
\end{verbatim}

\section{Character array stripping, padding, and truncation}
\label{sec:chararray-clip-pad-truncate}
CharArrays are designed to store fixed length strings of visible ASCII text.
You may have noticed that although a \code{CharArray} stores fixed length
strings, it displays variable length strings.  This is a result of the
stripping and padding policies of the CharArray class.  

When an element of a \code{CharArray} is fetched trailing whitespace is
stripped off.  The sole exception to this rule is that a single whitespace is
never stripped down to the empty string.  \code{numarray.strings} defines
whitespace as an ASCII space, formfeed, newline, carriage return, tab, or
vertical tab.

When a string is assigned to a \code{CharArray}, the string is considered
terminated by the first of any NULL characters it contains and is padded with
spaces to the full length of the \code{CharArray} itemsize.  Thus, the memory
image of a \code{CharArray} element does not include anything at or after the
first NULL in an assigned string; instead, there are spaces, and no terminating
NULL character at all.

When a string which is longer than the \code{itemsize()} is assigned to a
\code{CharArray}, it is silently truncated.

The \code{RawCharArray} baseclass of \code{CharArray} implements transparent
\code{strip()} and \code{pad()} methods, enabling the storage and retrieval of
arbitrary ASCII values within array elements.  For \code{RawCharArray}, all
array elements are identical in percieved length.  Alternate
stripping and padding policies can be implemented by subclassing
\code{CharArray} or \code{RawCharArray}.

\section{Character array functions}
\label{sec:chararray-func}
\begin{funcdesc}{array}{buffer=None, itemsize=None, shape=None, byteoffset=0,
    bytestride=None, kind=CharArray}
\label{func:str.array}
   The function \code{array} is, for most practical purposes, all a user needs 
   to know to construct a character array.

   The first argument, \code{buffer}, may be any one of the following:

   (1) \code{None} (default).  The constructor will allocate a writeable memory
   buffer which will be uninitialized.  The user must assign valid data before
   trying to read the contents or before writing the character array to a disk
   file.
   
   (2) a Python string containing binary data.  For example:
\begin{verbatim}
     >>> print str.array('abcdefg'*10, itemsize=10)
     ['abcdefgabc', 'defgabcdef', 'gabcdefgab', 'cdefgabcde', 'fgabcdefga',
      'bcdefgabcd', 'efgabcdefg']
\end{verbatim}
   
   (3) a Python file object for an open file.  The data will be copied from 
   the file, starting at the current position of the read pointer.
   
   (4) a character array.  This results in a deep copy of the input character
   array; any other arguments to \code{array()} will be silently ignored.

\begin{verbatim}
     >>> print str.array(s)
     ['abcdefgabc', 'defgabcdef', 'gabcdefgab', 'cdefgabcde', 'fgabcdefga', 
      'bcdefgabcd', 'efgabcdefg']
\end{verbatim}
   
   (5) a nested sequence of strings.  The sequence nesting implies the
   shape of the string array unless shape is specified.

\begin{verbatim}
     >>> print str.array([['Smith', 'Johnson'], ['Williams', 'Miller']])
     [['Smith', 'Johnson'],
      ['Williams', 'Miller']]
\end{verbatim}

   \code{itemsize} can be used to increase or decrease the fixed size of an
   array element relative to the natural itemsize implied by any literal data
   specified by the \code{buffer} parameter.

\begin{verbatim}
     >>> print str.array([['Smith', 'Johnson'], ['Williams', 'Miller']], 
                         itemsize=2)
     [['Sm', 'Jo'],
      ['Wi', 'Mi']])
     >>> print str.array([['Smith', 'Johnson'], ['Williams', 'Miller']], 
                         itemsize=20)
     [['Smith', 'Johnson'],
      ['Williams', 'Miller']]
\end{verbatim}
   
   \code{shape} is the shape of the character array.  It can be an integer, in
   which case it is equivalent to the number of \var{rows} in a table.  It can
   also be a tuple implying the character array is an N-D array with fixed
   length strings as its elements. \code{shape} should be consistent with
   the number of elements implied by the data buffer and itemsize.

   \code{byteoffset} indicates an offset, specified in bytes, from the start
   of the array buffer to where the array data actually begins.  
   \code{byteoffset} enables the character array to be offset from the
   beginning of a table record.  This is mainly useful for implementing
   record arrays.

   \code{bytestride} indicates the separation, specified in bytes, between
   successive elements in the last dimension of the character array.
   \code{bytestride} is used in the implementation of record arrays to space
   character array elements with the size of the total record rather than the
   size of a single string.
   
   \code{kind} is used to specify the class of the created array, and should be
   \code{RawCharArray}, \code{CharArray}, or a subclass of either.
\end{funcdesc}
   
\begin{funcdesc}{num2char}{n, format, itemsize=32}
\label{func:str.num2char}
\code{num2char} formats the numarray \code{n} using the Python string format
\code{format} and stores the result in a character array with the specified
\code{itemsize}
\begin{verbatim}
     >>> num2char(num.arange(0.0,5), '%2.2f')
     CharArray(['0.00', '1.00', '2.00', '3.00', '4.00'])
\end{verbatim}
\end{funcdesc}

\section{Character array methods}
\label{sec:recarray-methods}
CharArray object has these public methods:

\begin{methoddesc}[RawCharArray]{tolist}{}
  \code{tolist()} returns a nested list of strings corresponding to all the
  elements in the array.
\end{methoddesc}
\begin{methoddesc}[RawCharArray]{copy}{}
  \code{copy()} returns a deep copy of the character array.
\end{methoddesc}
\begin{methoddesc}[RawCharArray]{raw}{}
  \code{raw()} returns the corresponding \code{RawCharArray} view.
\begin{verbatim}
     >>> c=str.array(["this","that","another"])
     >>> c.raw()
     RawCharArray(['this   ', 'that   ', 'another'])
\end{verbatim}
\end{methoddesc}
\begin{methoddesc}{resized}{n, fill=' '}
  \code{resized(n)} returns a copy of the array, resized so that each element
  is of length \code{n} characters.  Extra characters are filled with value
  \code{fill}.  Caution: do not confuse this method with \code{resize()} which
  changes the number of elements rather than the size of each element.
\begin{verbatim}
     >>> c = str.array(["this","that","another"])
     >>> c.itemsize()
     7
     >>> d = c.resized(20)
     >>> print d
     ['this', 'that', 'another']
     >>> d.itemsize()
     20
\end{verbatim}
\end{methoddesc}
\begin{methoddesc}[RawCharArray]{concatenate}{other}
  \code{concatenate(other)} returns a new array which corresponds to the
  element by element concatenation of \code{other} to \code{self}.  The
  addition operator is also overloaded to perform concatenation.
\begin{verbatim}
     >>> print map(str, range(3)) + array(["this","that","another one"])
     ['0this', '1that', '2another one']
     >>> print "prefix with trailing whitespace   " + array(["."])
     ['prefix with trailing whitespace   .']
\end{verbatim}
\end{methoddesc}
\begin{methoddesc}[RawCharArray]{sort}{}
  \code{sort} modifies the \code{CharArray} inplace so that its elements are in
  sorted order. \code{sort} only works for 1D character arrays.  Like the
  \code{sort()} for the Python list, \code{CharArray.sort()} returns nothing.
\begin{verbatim}
     >>> a=str.array(["other","this","that","another"])
     >>> a.sort()
     >>> print a
     ['another', 'other', 'that', 'this']
\end{verbatim}
\end{methoddesc}
\begin{methoddesc}[RawCharArray]{argsort}{}
   \code{argsort} returns a numarray corresponding to the permutation which will
   put the character array \code{self} into sorted order.  \code{argsort} only
   works for 1D character arrays.
\begin{verbatim}
     >>> a=str.array(["other","that","this","another"])
     >>> a.argsort()
     array([3, 0, 1, 2])
     >>> print a[ a.argsort ] 
     ['another', 'other', 'that', 'this']
\end{verbatim}
\end{methoddesc}
\begin{methoddesc}[RawCharArray]{amap}{f}
  \code{amap} applies the function \code{f} to every element of \code{self} and
  returns the nested list of the results.  The function \code{f} should operate
  on a single string and may return any Python value.
\end{methoddesc}
\begin{verbatim}
     >>> c = str.array(['this','that','another'])
     >>> print c.amap(lambda x: x[-2:])
     ['is', 'at', 'er']
\end{verbatim}
\begin{methoddesc}[RawCharArray]{match}{pattern, flags=0}
  \code{match} uses Python regular expression matching over all elements of a
  character array and returns a tuple of numarrays corresponding to the indices
  of \code{self} where the pattern matches. \code{flags} are passed directly to
  the Python pattern matcher defined in the \code{re} module of the standard
  library.
\begin{verbatim}
     >>> a=str.array([["wo","what"],["wen","erewh"]])
     >>> print a.match("wh[aebd]")
     (array([0]), array([1]))
     >>> print a[ a.match("wh[aebd]") ]
     ['what']
\end{verbatim}
\end{methoddesc}
\begin{methoddesc}[RawCharArray]{search}{pattern,flags=0}
  \code{search} uses Python regular expression searching over all elements of a
  character array and returns a tuple of numarrays corresponding to the indices
  of \code{self} where the pattern was found. \code{flags} are passed directly
  to the Python pattern \code{search} method defined in the \code{re} module of
  the standard library.  \code{flags} should be an or'ed combination (use the
  $\vert$ operator) of the following \code{re} variables: \code{IGNORECASE},
  \code{LOCALE}, \code{MULTILINE}, \code{DOTALL}, \code{VERBOSE}.  See the
  \code{re} module documentation for more details.
\end{methoddesc}
\begin{methoddesc}[RawCharArray]{sub}{pattern,replacement,flags=0,count=0}
  \code{sub} performs Python regular expression pattern substitution
  to all elements of a character array. \code{flags} and \code{count} work
  as they do for \code{re.sub()}.
\begin{verbatim}
     >>> a=str.array([["who","what"],["when","where"]])
     >>> print a.sub("wh", "ph")
     [['pho', 'phat'],
      ['phen', 'phere']])
\end{verbatim}
\end{methoddesc}
\begin{methoddesc}[RawCharArray]{grep}{pattern, flags=0}
  \code{grep} is intended to be used interactively to search a \code{CharArray}
  for the array of strings which match the given \code{pattern}.
  \code{pattern} should be a Python regular expression (see the \code{re}
  module in the Python standard library, which can be as simple as a string
  constant as shown below.
\begin{verbatim}
     >>> a=str.array([["who","what"],["when","where"]])
     >>> print a.grep("whe")
     ['when', 'where']
\end{verbatim}
\end{methoddesc}
\begin{methoddesc}[RawCharArray]{eval}{}
  \code{eval} executes the Python eval function on each element of a character
  array and returns the resulting numarray.  \code{eval} is intended for use
  converting character arrays to the corresponding numeric arrays.  An
  exception is raised if any string element fails to evaluate.
\begin{verbatim}
     >>> print str.array([["1","2"],["3","4."]]).eval()
     [[1., 2.],
      [3., 4.]]
\end{verbatim}
\end{methoddesc}
\begin{methoddesc}[RawCharArray]{maxLen}{}
  \code{maxLen} returns the minimum element length required to store the
  stripped elements of the array \code{self}.
\begin{verbatim}
     >>> print str.array(["this","there"], itemsize=20).maxLen()
     5
\end{verbatim}
\end{methoddesc}
\begin{methoddesc}[RawCharArray]{truncated}{}
  \code{truncated} returns an array corresponding to \code{self} resized
  so that it uses a minimum amount of storage.
\begin{verbatim}
     >>> a = str.array(["this  ","that"])
     >>> print a.itemsize()
     6
     >>> print a.truncated().itemsize()
     4
\end{verbatim}
\end{methoddesc}
\begin{methoddesc}[RawCharArray]{count}{s}
  \code{count} counts the occurences of string \code{s} in array \code{self}.
\begin{verbatim}
     >>> print array(["this","that","another","this"]).count("this")
     2
\end{verbatim}
\end{methoddesc}
\begin{methoddesc}[RawCharArray]{info}{}
   This will display key attributes of the character array.
\end{methoddesc}

%% Local Variables:
%% mode: LaTeX
%% mode: auto-fill
%% fill-column: 79
%% indent-tabs-mode: nil
%% ispell-dictionary: "american"
%% reftex-fref-is-default: nil
%% TeX-auto-save: t
%% TeX-command-default: "pdfeLaTeX"
%% TeX-master: "numarray"
%% TeX-parse-self: t
%% End:

\chapter{Record Array}
\label{cha:record-array}
\declaremodule{extension}{numarray.records}
\index{record array}

\section{Introduction}
\label{sec:recarray-intro}
One of the enhancements of \code{numarray} over \code{Numeric} is its 
support for record arrays, i.e. arrays with heterogeneous data types: 
for example, tabulated data where each field (or \var{column}) has the 
same data type but different fields may not.

Each record array is a \index{RecArray} \code{RecArray} object in the 
\code{numarray.records} module.  Most of 
the time, the easiest way to construct a record array is to use the 
\code{array()} function in the \code{numarray.records} module.  For example:
\begin{verbatim}
>>> import numarray.records as rec
>>> r = rec.array([('Smith', 1234),\
                   ('Johnson', 1001),\
                   ('Williams', 1357),\
                   ('Miller', 2468)], \
                   names='Last_name, phone_number')
\end{verbatim}
In this example, we \var{manually} construct a record array by longhand input of
the information.  This record array has 4 records (or rows) and two fields (or 
columns).  The names of the fields are specified in the \code{names} argument.  
When using this longhand input, the data types (formats) are 
automatically determined from the data.  In this case the first field is a 
string of 8 characters (since the longest name is 8 characters long) and 
the second field is an integer.

The record array is just like an array in numarray, except that now each 
element is a \code{Record}.  We can do the usual indexing and slicing:
\begin{verbatim}
>>> print r[0]
('Smith', 1234)
>>> print r[:2]
RecArray[ 
('Smith', 1234),
('Johnson', 1001)
]
\end{verbatim}
To access the record array's fields, use the \code{field()} method:
\begin{verbatim}
>>> print r.field(0)
['Smith', 'Johnson', 'Williams', 'Miller']
>>> print r.field('Last_name')
['Smith', 'Johnson', 'Williams', 'Miller']
\end{verbatim}
these examples show that the \code{field} method can accept either the 
numeric index or the field name.

Since each field is simply a numarray of numbers or strings, all 
functionalities of numarray are available to them.  The record array is one 
single object which allows the user to have either field-wise or row-wise 
access.  The following example:
\begin{verbatim}
>>> r.field('phone_number')[1]=9999
>>> print r[:2]
RecArray[ 
('Smith', 1234),
('Johnson', 9999)
]
\end{verbatim}
shows that a change using the field view will cause the corresponding change 
in the row-wise view without additional copying or computing.

\section{Record array functions}
\label{sec:recarray-func}
\begin{funcdesc}{array}{buffer=None, formats=None, shape=0, 
names=None, byteorder=sys.byteorder}
\label{func:rec.array}
   The function \code{array} is, for most practical purposes, all a user needs 
   to know to construct a record array.

   \code{formats} is a string containing the format information of all fields.  
   Each format can be the \var{letter code}, such as \code{f4} or \code{i2}, 
   or longer name like \code{Float32} or \code{Int16}.  For a list of letter 
   codes or the longer names, see Table \ref{tab:type-specifiers} or use 
   the \code{letterCode()} function.  A field of strings is specified by the 
   letter \code{a}, followed by an integer giving the maximum length; thus 
   \code{a5} is the format for a field of strings of (maximum) length of 5.  

   The formats are separated by commas, and each \var{cell} 
   (element in a field) can be a numarray itself, by attaching a number or a 
   tuple in front of the format specification.  So if 
   \code{formats='i4,Float64,a5,3i2,(2,3)f4,Complex64,b1'}, the record array 
   will have:
   \begin{verbatim}
   1st field: (4-byte) integers
   2nd field: double precision floating point numbers
   3rd field: strings of length 5
   4th field: short (2-byte) integers, each element is an array of shape=(3,)
   5th field: single precision floating point numbers, each element is an 
       array of shape=(2,3)
   6th field: double precision complex numbers
   7th field: (1-byte) Booleans
   \end{verbatim}
   \code{formats} specification takes precedence over the data.  For 
   example, if a field is specified as integers in \code{buffer}, but is 
   specified as floats in \code{formats}, it will be floats in the record 
   array.  If a field in the \code{buffer} is not convertible to the 
   corresponding data type in the \code{formats} specification, e.g. from 
   strings to numbers (integers, floats, Booleans) or vice versa, an 
   exception will be raised.
   
   \code{shape} is the shape of the record array.  It can be an integer, 
   in which case it is equivalent to the number of \var{rows} in a table.  
   It can also be a tuple where the record array is an N-D array with 
   \code{Records} as its elements. \code{shape} must be consistent with the 
   data in \code{buffer} for buffer types (5) and (6), explained below.
   
   \code{names} is a string containing the names of the fields, separated by 
   commas.  If there are more formats specified than names, then default 
   names will be used: If there are five fields specified in \code{formats} 
   but \code{names=None} (default), then the field names will be: 
   \code{c1, c2, c3, c4, c5}.  If \code{names="a,b"}, then the field 
   names will be: \code{a, b, c3, c4, c5}.
   
   If more names have been specified than there are formats, the extra names 
   will be discarded.  If duplicate names are specified, a \code{ValueError} 
   will be raised.  Field names are case sensitive, e.g. column \code{ABC} will 
   not be found if it is referred to as \code{abc} or \code{Abc} 
   (for example) when using the \code{field()} method.
   
   \code{byteorder} is a string of the value \code{big} or \code{little}, 
   referring to big endian or little endian.  This is useful when reading 
   (binary) data from a string or a file.  If not specified, it will use the 
   \code{sys.byteorder} value and the result will be platform dependent for 
   string or file input.
   
   The first argument, \code{buffer}, may be any one of the following:

   (1) \code{None} (default).  The data block in the record array will not be 
   initialized.  The user must assign valid data before trying to read the 
   contents or before writing the record array to a disk file.
   
   (2) a Python string containing binary data.  For example:
   \begin{verbatim}
   >>> r=rec.array('abcdefg'*100, formats='i2,a3,i4', shape=3, byteorder='big')
   >>> print r
   RecArray[ 
   (24930, 'cde', 1718051170),
   (25444, 'efg', 1633837924),
   (25958, 'gab', 1667523942)
   ]
   \end{verbatim}
   
   (3) a Python file object for an open file.  The data will be copied from 
   the file, starting at the current position of the read pointer, with 
   byte order as specified in \code{byteorder}.
   
   (4) a record array.  This results in a deep copy of the input record array; 
   any other arguments to \code{array()} will be silently ignored.
   
   (5) a list of numarrays.  There must be one such numarray for each field.  
   The \code{formats} and \code{shape} arguments to \code{array()} are not 
   required, but if they are specified, they need to be consistent with the 
   input arrays.  The shapes of all the input numarrays also need to be 
   consistent to one another.
   \begin{verbatim}
   # this will have 3 rows, each cell in the 2nd field is an array of 4 elements
   # note that the formats sepcification needs to reflect the data shape
   >>> arr1=numarray.arange(3)
   >>> arr2=numarray.arange(12,shape=(3,4))
   >>> r=rec.array([arr1, arr2],formats='i2,4f4')
   \end{verbatim}
   
   In this example, \code{arr2} is cast up to float.
   
   (6) a list of sequences.  Each sequence contains the 
   number(s)/string(s) of a record.  The example in the introduction 
   uses such input, sometimes called \var{longhand} input.  The data 
   types are automatically determined after comparing all input data.  
   Data of the same field will be cast to the highest type:
   \begin{verbatim}
   # the first field uses the highest data type: Float64
   >>> r=rec.array([[1,'abc'],(3.5, 'xx')]); print r
   RecArray[ 
   (1.0, 'abc'),
   (3.5, 'xx')
   ]
   \end{verbatim}
   unless overruled by the \code{formats} argument:
   \begin{verbatim}
   # overrule the first field to short integers, second field to shorter strings
   >>> r=rec.array([[1,'abc'],(3.5, 'xx')],formats='i2,a1'); print r
   RecArray[ 
   (1, 'a'),
   (3, 'x')
   ]
   \end{verbatim}
   Inconsistent data in the same field will cause a \code{ValueError}:
   \begin{verbatim}
   >>> r=rec.array([[1,'abc'],('a', 'xx')])
   ValueError: inconsistent data at row 1,field 0
   \end{verbatim}
   
   A record array with multi-dimensional numarray cells in a field can also 
   be constructed by using nested sequences:
   \begin{verbatim}
   >>> r=rec.array([[(11,12,13),'abc'],[(2,3,4), 'xx']]); print r
   RecArray[ 
   (array([11, 12, 13]), 'abc'),
   (array([2, 3, 4]), 'xx')
   ]
   \end{verbatim}
\end{funcdesc}
   
\begin{funcdesc}{letterCode}{}
   This function will list the letter codes acceptable by the \code{formats} 
   argument in \code{array()}.
\end{funcdesc}

\section{Record array methods}
\label{sec:recarray-methods}
RecArray object has these public methods:

\begin{methoddesc}[RecArray]{field}{fieldName}
   \code{fieldName} can be either an integer (field index) or string 
   (field name).
   \begin{verbatim}
   >>> r=rec.array([[11,'abc',1.],[12, 'xx', 2.]])
   >>> print r.field('c1')
   [11 12]
   >>> print r.field(0)  # same as field('c1')
   [11 12]
   \end{verbatim}
   To set values, simply use indexing or slicing, since each field is a 
   numarray:
   \begin{verbatim}
   >>> r.field(2)[1]=1000; r.field(1)[1]='xyz'
   >>> r.field(0)[:]=999
   >>> print r
   RecArray[ 
   (999, 'abc', 1.0),
   (999, 'xyz', 1000.0)
   ]
   \end{verbatim}
\end{methoddesc}
\begin{methoddesc}[RecArray]{info}{}
   This will display key attributes of the record array.
\end{methoddesc}

\section{Record object}
\label{sec:recarray-record}
\index{Record object}
Each single record (or \var{row}) in the record array is a 
\code{records.Record} object.  It has these methods:

\begin{methoddesc}[Record]{field}{fieldName}
\end{methoddesc}
\begin{methoddesc}[Record]{setfield}{fieldname, value}
   Like the \code{RecArray}, a \code{Record} object has the \code{field} 
   method to \var{get} the field value.  But since a \code{Record} object 
   is not an array, it does not take an index or slice, so one cannot 
   assign a value to it.  So a separate \var{set} method, \code{setfield()}, 
   is necessary:
   \begin{verbatim}
   >>> r[1].field(0)
   999
   >>> r[1].setfield(0, -1)
   >>> print r[1]
   (-1, 'xy', 1000.0)
   \end{verbatim}
   Like the \code{field()} method in \code{RecArray}, \code{fieldName} in 
   \code{Record}'s \code{field()} and \code{setfield()} methods can be 
   either an integer (index) or a string (field name).
\end{methoddesc}


%% Local Variables:
%% mode: LaTeX
%% mode: auto-fill
%% fill-column: 79
%% indent-tabs-mode: nil
%% ispell-dictionary: "american"
%% reftex-fref-is-default: nil
%% TeX-auto-save: t
%% TeX-command-default: "pdfeLaTeX"
%% TeX-master: "numarray"
%% TeX-parse-self: t
%% End:

\chapter{Object Array}
\label{cha:object-array}
\declaremodule{extension}{numarray.objects}
\index{object array}

\section{Introduction}
\label{sec:objectarray-intro}

\code{numarray}, like \code{Numeric}, has support for arrays of objects in
addition to arrays of numbers.  Arrays of objects are supported by the
\code{numarray.objects} module.  The \index{ObjectArray} \code{ObjectArray}
class is used to represent object arrays.  

The easiest way to construct an object array is to use the
\code{numarray.objects.array()} function.  For example:

\begin{verbatim}
  >>> import numarray.objects as obj
  >>> o = obj.array(['S', 'J', 1, 'M'])
  >>> print o
  ['S' 'J' 1 'M']
  >>> print o + o
  ['SS' 'JJ' 2 'MM']
\end{verbatim}

In this example, the array contains 3 Python strings and an integer, but the
array elements can be any Python object.  For each pair of elements, the
\function{add} operator is applied.  For strings, \function{add} is defined as
string concatenation.  For integers, \function{add} is defined as numerical
addition.  For a class object, the \function{__add__} and \function{__radd__}
methods would define the result.

\class{ObjectArray} is defined as a subclass of numarray's structural array
class, \class{NDArray}.  As a result, we can do the usual indexing and slicing:

\begin{verbatim}
  >>> import numarray.objects as obj
  >>> print s[0]
  'S'
  >>> print s[:2]
  ['S' 'J']
  >>> s[:2] = 'changed'
  >>> print s
  ['changed' 'changed' 1 'M']
  >>> a = obj.fromlist(numarray.arange(100), shape=(10,10))
  >>> a[2:5, 2:5]
  ObjectArray([[22, 23, 24],
               [32, 33, 34],
               [42, 43, 44]])
\end{verbatim}

\section{Object array functions}
\label{sec:objectarray-func}
\begin{funcdesc}{array}{sequence=None, shape=None, typecode='O'}
\label{func:obj.array}
   The function \function{array} is, for most practical purposes, all a user needs 
   to know to construct an object array.

   The first argument, \code{sequence}, can be an arbitrary sequence of Python
   objects, such as a list, tuple, or another object array.  

\begin{verbatim}
  >>> import numarray.objects as obj
  >>> class C:
  ...     pass
  >>> c = C()
  >>> a = obj.array([c, c, c])
  >>> a
  ObjectArray([c, c, c])
\end{verbatim}
   
   Like objects in Python lists, objects in object arrays are referred to, not
   copied, so changes to the objects are reflected in the originals because
   they are one and the same.

\begin{verbatim}
     >>> a[0].attribute  = 'this'
     >>> c.attribute
     'this'
\end{verbatim}
   
   The second argument, \code{shape}, optionally specifies the shape of the
   array.  If no \code{shape} is specified, the shape is implied by the
   sequence.

\begin{verbatim}
  >>> import numarray.objects as obj
  >>> class C:
  ...     pass
  >>> c = C()
  >>> a = obj.fromlist([c, c, c])
  >>> a
  ObjectArray([c, c, c])
\end{verbatim}
   
   The last argument, \code{typecode}, is there for backward compatibility with
   Numeric; it must be specified as 'O'.

\end{funcdesc}
   
\begin{funcdesc}{asarray}{obj}
  \label{func:obj.asarray}
  \code{asarray} converts sequences which are not object arrays into object
  arrays.  If \code{obj} is already an \class{ObjectArray}, it is returned
  unaltered.
\begin{verbatim}
  >>> import numarray.objects as obj
  >>> a = obj.asarray([1,''this'',''that''])
  >>> a
  ObjectArray([1 'this' 'that'])
  >>> b = obj.asarray(a)
  >>> b is a
  True
\end{verbatim}
\end{funcdesc}

\begin{funcdesc}{choose}{selector, population, output=None}
  \label{func:obj.choose}
  \code{choose} selects elements from \var{population} based on the values in
  \var{selector}, either returning the selected array or storing it in the
  optional \code{ObjectArray} specified by \var{output}.  \var{selector} should
  be an integer sequence where each element is within the range 0 to
  \function{len}{population}.  \var{population} should be a sequence of
  \class{ObjectArray}s. The shapes of \var{selector} and each element of
  \var{population} must be mutually broadcastable.
\begin{verbatim}
  >>> import numarray.objects as obj
  >>> s = num.arange(25, shape=(5,5)) % 3
  >>> p = obj.fromlist(["foo", 1, {"this":"that"}])
  >>> obj.choose(s, p)
  ObjectArray([['foo', 1, {'this': 'that'}, 'foo', 1],
    [{'this': 'that'}, 'foo', 1, {'this': 'that'}, 'foo'],
    [1, {'this': 'that'}, 'foo', 1, {'this': 'that'}],
    ['foo', 1, {'this': 'that'}, 'foo', 1],
    [{'this': 'that'}, 'foo', 1, {'this': 'that'}, 'foo']])
  
\end{verbatim}
\end{funcdesc}

\begin{funcdesc}{sort}{objects, axis=-1, output=None}
  \label{func:obj.sort}
  \code{sort} sorts the elements from \var{objects} along the specified
  \var{axis}.  If an output array is specified, the result is stored there
  and the return value is None,  otherwise the sort is returned.
\begin{verbatim}
    >>> import numarray.objects as obj
    >>> a = obj.ObjectArray(shape=(5,5))
    >>> a[:] = range(5,0,-1)
    >>> obj.sort(a)
    ObjectArray([[1, 2, 3, 4, 5],
                 [1, 2, 3, 4, 5],
                 [1, 2, 3, 4, 5],
                 [1, 2, 3, 4, 5],
                 [1, 2, 3, 4, 5]])
    >>> a[:] = range(5,0,-1)
    >>> a.transpose()
    >>> obj.sort(a, axis=0)
    ObjectArray([[1, 1, 1, 1, 1],
                 [2, 2, 2, 2, 2],
                 [3, 3, 3, 3, 3],
                 [4, 4, 4, 4, 4],
                 [5, 5, 5, 5, 5]])
\end{verbatim}
\end{funcdesc}

\begin{funcdesc}{argsort}{objects, axis=-1, output=None}
  \label{func:obj.argsort}
  \code{argsort} returns the sort order for the elements from \var{objects}
  along the specified \var{axis}.  If an output array is specified, the result
  is stored there and the return value is None, otherwise the sort order is
  returned.
\begin{verbatim}
  >>> import numarray.objects as obj
  >>> a = obj.ObjectArray(shape=(5,5))
  >>> a[:] = ['e','d','c','b','a']
  >>> obj.argsort(a)
  array([[4, 3, 2, 1, 0],
         [4, 3, 2, 1, 0],
         [4, 3, 2, 1, 0],
         [4, 3, 2, 1, 0],
         [4, 3, 2, 1, 0]])
\end{verbatim}
\end{funcdesc}

\begin{funcdesc}{take}{objects, indices, axis=0}
  \label{func:obj.take}
  \code{take} returns elements of \var{objects} specified by tuple of index
  arrays \var{indices} along the specified \var{axis}.
\begin{verbatim}
  >>> import numarray.objects as obj
  >>> o = obj.fromlist(range(10))
  >>> a = obj.arange(5)*2
  >>> obj.take(o, a)
  ObjectArray([0, 2, 4, 6, 8])
\end{verbatim}
\end{funcdesc}

\begin{funcdesc}{put}{objects, indices, values, axis=-1}
  \label{func:obj.put}
  \function{put} stores \var{values} at the locations of \var{objects}
  specified by tuple of index arrays \var{indices}.
\begin{verbatim}
  >>> import numarray.objects as obj
  >>> o = obj.fromlist(range(10))
  >>> a = obj.arange(5)*2
  >>> obj.put(o, a, 0); o
  ObjectArray([0, 1, 0, 3, 0, 5, 0, 7, 0, 9])
\end{verbatim}
\end{funcdesc}

\begin{funcdesc}{add}{objects1, objects2, out=None}
  \label{func:obj.add}
  \code{numarray.objects} defines universal functions which are named after and
  use the operators defined in the standard library module operator.py.  In
  addition, the operator hooks of the \class{ObjectArray} class are defined to
  call the operators.  \code{add} applies the \code{add} operator to
  corresponding elements of \var{objects1} and \var{objects2}.  Like the ufuncs
  in the numerical side of numarray, the object ufuncs support reduction and
  accumulation.  In addition to add, there are ufuncs defined for every unary
  and binary operator function in the standard library module operator.py.
  Some of these are given additional synonyms so that they use numarray naming
  conventions, e.g. \function{sub} has an alias named \function{subtract}.
\begin{verbatim}
  >>> import numarray.objects as obj
  >>> a = obj.fromlist(["t","u","w"])
  >>> a
  ObjectArray(['t', 'u', 'w'])
  >>> a+a
  ObjectArray(['tt', 'uu', 'ww'])
  >>> obj.add(a,a)
  ObjectArray(['tt', 'uu', 'ww'])
  >>> obj.add.reduce(a)
  'tuw' # not, as in the docs, an ObjectArray
  >>> obj.add.accumulate(a)
  ObjectArray(['t', 'tu', 'tuw']) # w, not v

  >>> a = obj.fromlist(["t","u","w"])
  >>> a
  ObjectArray(['t', 'u', 'w'])
  >>> a+a
  ObjectArray(['tt', 'uu', 'ww'])
  >>> obj.add(a,a)
  ObjectArray(['tt', 'uu', 'ww'])
  >>> obj.add.reduce(a)
  ObjectArray('tuv')
  >>> obj.add.accumulate(a)
  ObjectArray(['t', 'tu', 'tuv'])
\end{verbatim}
\end{funcdesc}

\section{Object array methods}
\label{sec:objectarray-methods}
\class{ObjectArray} maps each of its operator hooks (e.g. \code{__add__}) onto
the corresponding object ufunc (e.g. \code{numarray.objects.add}).  In addition
to its hook methods,  \class{ObjectArray} has these public methods:

\begin{methoddesc}[ObjectArray]{tolist}{}
  \code{tolist} returns a nested list of objects corresponding to all the
  elements in the array.
\end{methoddesc}

\begin{methoddesc}[ObjectArray]{copy}{}
  \code{copy} returns a shallow copy of the object array.
\end{methoddesc}

\begin{methoddesc}[ObjectArray]{astype}{type}
  \code{astype} returns either a copy of the \class{ObjectArray} or converts it
  into a numerical array of the specified \var{type}.
\end{methoddesc}

\begin{methoddesc}[ObjectArray]{info}{}
   This will display key attributes of the object array.
\end{methoddesc}

%% Local Variables:
%% mode: LaTeX
%% mode: auto-fill
%% fill-column: 79
%% indent-tabs-mode: nil
%% ispell-dictionary: "american"
%% reftex-fref-is-default: nil
%% TeX-auto-save: t
%% TeX-command-default: "pdfeLaTeX"
%% TeX-master: "numarray"
%% TeX-parse-self: t
%% End:

\chapter{C extension API}
\label{cha:C-API}
\declaremodule{extension}{C-API}
\index{C-API}

\begin{quote}
   This chapter describes the different available C-APIs for \module{numarray}
   based extension modules.
\end{quote}

While this chapter describes the \module{\numarray}-specifics for writing
extension modules, a basic understanding of \python extension modules is
expected. See \python's \ulink{Extending and
   Embedding}{http://www.python.org/doc/current/ext/ext.html} tutorial and the
\ulink{Python/C API}{http://www.python.org/doc/current/api/api.html}.

The numarray C-API has several different facets, and the first three facets
each make different tradeoffs between memory use, speed, and ease of use.  An
additional facet provides backwards compatability with legacy Numeric code.
The final facet consists of miscellaneous function calls used to implement
and utilize numarray, that were not part of Numeric.

In addition to most of the basic functionality provided by Numeric, these APIs
provide access to misaligned, byteswapped, and discontiguous \class{numarray}s.
Byteswapped arrays arise in the context of portable binary data formats where
the byteorder specified by the data format is not the same as the host
processor byte order.  Misaligned arrays arise in the context of tabular data:
files of records where arrays are superimposed on the column formed by a single
field in the record.  Discontiguous arrays arise from operations which permute
the shape and strides of an array, such as reshape.

\begin{description}
\item[Numeric compatability] This API provides a reasonable (if not complete)
   simulation of the Numeric C-API.  It is written in terms of the numarray
   high level API (discussed below) so that misbehaved numarrays are copied
   prior to processing with legacy Numeric code.  This API was actually written
   last because of the extra considerations in numarray not found in Numeric.
   Nevertheless, it is perhaps the most important because it enables writing
   extension modules which can be compiled for either numarray or Numeric.  It
   is also very useful for porting existing Numeric code.  See section
   \ref{sec:C-API:numeric-simulation}.
\item[High-level] This is the cleanest and eaisiest to use API.  It creates
   temporary arrays to handle difficult cases (discontiguous, byteswapped,
   misaligned) in C code.  Code using this API is written in terms of a pointer
   to a contiguous 1D array of C data.  See section
   \ref{sec:C-API:high-level-api}.
\item[Element-wise] This API handles misbehaved arrays without creating
     temporaries.  Code using this API is written to access single elements of
     an array via macros or functions.  \note{These macros are relatively slow
     compared to raw access to C data, and the functions even slower.} See
     section \ref{sec:C-API:element-wise-api}.
\item[One-dimensional] Code using this API get/sets consecutive elements of the
   inner dimension of an array, enabling the API to factor out tests for
   aligment and byteswapping to one test per array rather than one test per
   element.  Fewer tests means better performance, but at a cost of some
   temporary data and more difficult usage.  See section
   \ref{sec:C-API:One-dimensional-api}.
\item[New numarray functions] This last facet of the C-API consists of function
  calls which have been added to numarray which are orthogonal to each of the 3
  native access APIs and not part of the original Numeric. See section
  \ref{sec:C-API:new-numarray-functions}
\end{description}

\section{Numarray extension basics}
There's a couple things you need to do in order to access numarray's C-API in
your own C extension module:

\subsection{Include libnumarray.h}
  Near the top of your extension module add the lines:
\begin{verbatim}
  #include "Python.h"
  #include "libnumarray.h"
\end{verbatim} 
  This gives your C-code access to the numarray typedefs, macros, and function
  prototypes as well as the Python C-API.

  \subsection{Alternate include method}
  There's an alternate form of including libnumarray.h or arrayobject.h some
  people may prefer provided that they're willing to ignore the case where the
  numarray includes are not installed in the standard location.  The advantage
  of the following approach is that it automatically works with the default
  path to the Python include files which the distutils always provide.
  \begin{verbatim}
    #include "Python.h"
    #include "numarray/libnumarray.h"
  \end{verbatim}

\subsection{Import libnumarray}
  In your extension module's initialization function, add the line:
\begin{verbatim}
  import_libnumarray();
\end{verbatim} 
  
  import_libnumarray() is actually a macro which sets up a pointer to the
  numarray C-API function pointer table. If you forget to call
  import_libnumarray(), your extension module will crash as soon as you call a
  numarray API function, because your application will attempt to dereference a
  NULL pointer.

  Note that for the Numeric compatible API you should substitute arrayobject.h
  for libnumarray.h and import_array() for import_libnumarray() respectively.
  Unlike other versions of numarray prior to 1.0, arrayobject.h now includes
  only the Numeric simulation API.  To use the rest of the numarray API, you
  \emph{must} include libnumarray.h.  To use both, you must include both
  arrayobject.h and libnumarray.h, and you must both import_array() and
  import_libnumarray() in your module initialization function.

  \subsection{Writing a simple setup.py file for a numarray extension}
  One important practice for writing an extension module is the creation of a
  distutils setup.py file which automates both extension installation from
  source and the creation of binary distributions.  Here is a simple setup.py
  which builds a single extension module from a single C source file:
  \begin{verbatim}
    from distutils.core import setup, Extension
    from numarray.numarrayext import NumarrayExtension
    import sys
    
    if not hasattr(sys, 'version_info') or sys.version_info < (2,2,0,'alpha',0):        raise SystemExit, "Python 2.2 or later required to build this module."
    
    setup(name = "buildHistogram",
       version = "0.1",
       description = "",
       packages=[""],
       package_dir={"":""},
       ext_modules=[NumarrayExtension("buildHistogram",['buildHistogram.c'],\
         include_dirs=["./"],
         library_dirs=["./"],
         libraries=['m'])])
\end{verbatim}
\class{NumarrayExtension} is recommended rather than it's distutils baseclass
\class{Extension} because \class{NumarrayExtension} knows where to find the
numarray headers regardless of where the numarray installer or setup.py command
line options put them.  A disadvantage of using NumarrayExtension is that it
is numarray specific, so it does not work for compiling Numeric versions of the
extension.

See the Python manuals ``Installing Python Modules'' and ``Distributing Python
Modules'' for more information on how to use distutils.

\section{Fundamental data structures}
\label{C-API:fundamental-data-structures}

\subsection{Numarray Numerical Data Types}

Numarray hides the C implementation of its basic array elements behind a set of
C typedefs which specify the absolute size of the type in bits.  This approach
enables a programmer to specify data items of arrays and extension functions in
an explicit yet portable manner.  In contrast, basic C types are platform
relative, and so less useful for describing real physical data.  Here are the
names of the concrete Numarray element types:

\begin{itemize}
\item Bool                
\item Int8,      UInt8
\item Int16,     UInt16
\item Int32,     UInt32
\item Int64,     UInt64
\item Float32,   Float64
\item Complex32, Complex64
\end{itemize}

\subsection{NumarrayType}

The type of a numarray is communicated in C via one of the following
enumeration constants.  Type codes which are backwards compatible with Numeric
are defined in terms of these constants, but use these if you're not already
using the Numeric codes.  These constants communicate type requirements between
one function and another, since in C, you cannot pass a typedef as a value.
tAny is used to specify both ``no type requirement'' and ``no known type''
depending on context.   

\begin{verbatim}
typedef enum 
{
  tAny,

  tBool,
  tInt8,      tUInt8,
  tInt16,     tUInt16,
  tInt32,     tUInt32, 
  tInt64,     tUInt64,
  tFloat32,   tFloat64,
  tComplex32, tComplex64,

  tDefault = tFloat64,

#if LP64
  tLong = tInt64
#else
  tLong = tInt32
#endif

} NumarrayType;
\end{verbatim}

\subsection{PyArray_Descr}

\ctype{PyArray_Descr} is used to hold a few parameters related to the type of
an array and exists mostly for backwards compatability with Numeric.
\var{type_num} is a NumarrayType value.  \var{elsize} indicates the number of
bytes in one element of an array of that type.  \var{type} is a Numeric
compatible character code.

Numarray's \ctype{PyArray_Descr} is currently missing the type-casting,
\function{ones}, and \function{zeroes} functions.  Extensions which use these
missing Numeric features will not yet compile.  Arrays of type Object are not
yet supported.

\begin{verbatim}

typedef struct {
        int  type_num;  /* PyArray_TYPES */
        int  elsize;    /* bytes for 1 element */
        char type;      /* One of "cb1silfdFD "  Object arrays not supported. */
} PyArray_Descr;

\end{verbatim}

\subsection{PyArrayObject}

The fundamental data structure of numarray is the PyArrayObject, which is named
and layed out to provide source compatibility with Numeric.  It is compatible
with most but not all Numeric code.  The constant MAXDIM, the maximum number of
dimensions in an array, is defined as 40.  It should be noted that unlike
earlier versions of numarray, the present PyArrayObject structure is a first
class python object, with full support for the number protocols in C.
Well-behaved arrays have mutable fields which will reflect modifications back
into \python ``for free''.

\begin{verbatim}

typedef int maybelong;          /* towards 64-bit without breaking extensions. */

typedef struct {
        /* Numeric compatible stuff */

        PyObject_HEAD
        char *data;              /* points to the actual C data for the array */
        int nd;                  /* number of array shape elements */
        maybelong *dimensions;   /* values of shape elements */
        maybelong *strides;      /* values of stride elements */
        PyObject *base;          /* unused, but don't touch! */
        PyArray_Descr *descr;    /* pointer to descriptor for this array's type */
        int flags;               /* bitmask defining various array properties */

        /* numarray extras */

        maybelong _dimensions[MAXDIM];  /* values of shape elements */
        maybelong _strides[MAXDIM];     /* values of stride elements */
        PyObject *_data;       /* object must meet buffer API */
        PyObject *_shadows;    /* ill-behaved original array. */
        int      nstrides;     /* elements in strides array */
        long     byteoffset;   /* offset into buffer where array data begins */
        long     bytestride;   /* basic seperation of elements in bytes */
        long     itemsize;     /* length of 1 element in bytes */

        char      byteorder;   /* NUM_BIG_ENDIAN, NUM_LITTLE_ENDIAN */

        char      _unused0; 
        char      _unused1; 
        
        /* Don't expect the following vars to stay around.  Never use them.
        They're an implementation detail of the get/set macros. */

        Complex64      temp;   /* temporary for get/set macros */
        char *         wptr;   /* working pointer for get/set macros */
} PyArrayObject;

\end{verbatim}

\subsection{Flag Bits}

The following are the definitions for the bit values in the \var{flags} field
of each numarray.  Low order bits are Numeric compatible,  higher order bits
were added by numarray.

\begin{verbatim}
/* Array flags */
#define CONTIGUOUS        1       /* compatible, depends */
#define OWN_DIMENSIONS    2       /* always false */
#define OWN_STRIDES       4       /* always false */
#define OWN_DATA          8       /* always false */
#define SAVESPACE      0x10       /* not used */

#define ALIGNED       0x100       /* roughly: data % itemsize == 0 */
#define NOTSWAPPED    0x200       /* byteorder == sys.byteorder    */
#define WRITABLE      0x400       /* data buffer is writable       */

#define IS_CARRAY (CONTIGUOUS | ALIGNED | NOTSWAPPED)
\end{verbatim}

\section{Numeric simulation API}
\label{sec:C-API:numeric-simulation}

These notes describe the Numeric compatability functions which enable numarray
to utilize a subset of the extensions written for Numeric (NumPy).  Not all
Numeric C-API features and therefore not all Numeric extensions are currently
supported.  Users should be able to utilize suitable extensions written for
Numeric within the numarray environment by:

\begin{enumerate}
\item Writing a numarray setup.py file.
\item Scanning the extension C-code for all instances of array creation and
  return and making corrections as needed and specified below. 
\item Re-compiling the Numeric C-extension for numarray.
\end{enumerate}

Numarray's compatability with Numeric consists of 3 things:
\begin{enumerate}
\item A replacement header file, "arrayobject.h" which supplies simulation
   functions and macros for numarray just as the original arrayobject.h
   supplies the C-API for Numeric.
\item Layout and naming of the fundamental numarray C-type,
\ctype{PyArrayObject}, in a Numeric source compatible way.
\item A set of "simulation" functions.  These functions have the same names and
   parameters as the original Numeric functions, but operate on numarrays.  The
   simulation functions are also incomplete; features not currently supported
   should result in compile time warnings.
\end{enumerate}

\subsection{Simulation Functions}
\label{sec:C-API:compat:simulation-functions}

The basic use of numarrays by Numeric extensions is achieved in the extension
function's wrapper code by:
\begin{enumerate}
\item Ensuring creation of array objects by calls to simulation functions.
\item DECREFing each array or calling PyArray_Return.
\end{enumerate}

Unlike prior versions of numarray, this version *does* support access to array
objects straight out of PyArg_ParseTuple.  This is a consequence of a change to
the underlying object model, where a class instance has been replaced by
PyArrayObject.  Nevertheless, the ``right'' way to access arrays is either via
the high level interface or via emulated Numeric factory functions.  That way,
access to other python sequences is supported as well.  Using the ``right'' way
for numarray is also more important than for Numeric because numarray arrays
may be byteswapped or misaligned and hence unusable from simple C-code.  It
should be noted that the numarray and Numeric are not completely compatible,
and therefore this API does not provide support for string arrays or object
arrays.

The creation of array objects is illustrated by the following of wrapper code
for a 2D convolution function:

\begin{verbatim}
#include "python.h"
#include "arrayobject.h"

static PyObject *
Py_Convolve2d(PyObject *obj, PyObject *args)
{
        PyObject   *okernel, *odata, *oconvolved=Py_None;
        PyArrayObject *kernel, *data, *convolved;

        if (!PyArg_ParseTuple(args, "OO|O", &okernel, &odata, &oconvolved)) {
                return PyErr_Format(_Error, 
                                    "Convove2d: Invalid parameters.");  
                goto _fail;
        }
\end{verbatim}

The first step was simply to get object pointers to the numarray parameters to
the convolution function: okernel, odata, and oconvolved.  Oconvolved is an
optional output parameter, specified with a default value of Py_None which is
used when only 2 parameters are supplied at the python level.  Each of the
``o'' parameters should be thought of as an arbitrary sequence object, not
necessarily an array.

The next step is to call simulation functions which convert sequence objects
into PyArrayObjects.  In a Numeric extension, these calls map tuples and lists
onto Numeric arrays and assert their dimensionality as 2D.  The Numeric
simulation functions first map tuples, lists, and misbehaved numarrays onto
well-behaved numarrays.  Calls to these functions transparently use the
numarray high level interface and provide visibility only to aligned and
non-byteswapped array objects.

\begin{verbatim}
        kernel = (PyArrayObject *) PyArray_ContiguousFromObject(
                okernel, PyArray_DOUBLE, 2, 2);
        data = (PyArrayObject *) PyArray_ContiguousFromObject(
                odata, PyArray_DOUBLE, 2, 2);

        if (!kernel || !data) goto _fail;
\end{verbatim}

Extra processing is required to handle the output array \var{convolved},
cloning it from \var{data} if it was not specified.  Code should be supplied,
but is not, to verify that convolved and data have the same shape.  

\begin{verbatim}
        if (convolved == Py_None)
                convolved = (PyArrayObject *) PyArray_FromDims(
                        data->nd, data->dimensions, PyArray_DOUBLE);
        else
                convolved = (PyArrayObject *) PyArray_ContiguousFromObject(
                        oconvolved, PyArray_DOUBLE, 2, 2);
        if (!convolved) goto _fail;
\end{verbatim}

After converting all of the input paramters into \ctype{PyArrayObject}s, the
actual convolution is performed by a seperate function.  This could just as
well be done inline:

\begin{verbatim}
        Convolve2d(kernel, data, convolved);
\end{verbatim}

After processing the arrays, they should be DECREF'ed or returned using
\cfunction{PyArray_Return}.  It is generally not possible to directly return a
numarray object using \cfunction{Py_BuildValue} because the shadowing of
mis-behaved arrays needs to be undone.  Calling \cfunction{PyArray_Return}
destroys any temporary and passes the numarray back to \python.

\begin{verbatim}
        Py_DECREF(kernel);
        Py_DECREF(data);
        if (convolved != Py_None) {
                Py_DECREF(convolved);
                Py_INCREF(Py_None);
                return Py_None;
        } else
                return PyArray_Return(convolved);
_fail:
        Py_XDECREF(kernel);
        Py_XDECREF(data);
        Py_XDECREF(convolved);
        return NULL;
}
\end{verbatim}

Byteswapped or misaligned arrays are handled by a process of shadowing which
works like this:
\begin{enumerate}
\item When a "misbehaved" numarray is accessed via the Numeric simulation
  functions, first a well-behaved temporary copy (shadow) is created by
  NA_IoArray.
\item Operations performed by the extension function modifiy the data buffer
  belonging to the shadow.
\item On extension function exit, the shadow array is copied back onto the 
  original and the shadow is freed.
\end{enumerate}
All of this is transparent to the user; if the original array is well-behaved,
it works much like it always did; if not, what would have failed altogether
works at the cost of extra temporary storage.  Users which cannot afford the
cost of shadowing need to use numarray's native elementwise or 1D APIs.
\subsection{Numeric Compatible Functions}
\label{sec:C-API:compat:implemented-functions}

The following functions are currently implemented:
\begin{cfuncdesc}{PyObject*}{PyArray_FromDims}{int nd, int *dims, int type}
   This function will allocate a new numarray.

   An array created with PyArray_FromDims can be used as a temporary or
   returned using PyArray_Return.
   
   Used as a temporary, calling Py_DECREF deallocates it.   
\end{cfuncdesc}

\begin{cfuncdesc}{PyObject*}{PyArray_FromDimsAndData}{int nd, int *dims, int type, char *data}
   This function will allocate a numarray of the specified shape and type
   which will refer to the data buffer specified by \var{data}.  The contents
   of \var{data} will not be copied nor will \var{data} be deallocated upon
   the deletion of the array.
\end{cfuncdesc}

\begin{cfuncdesc}{PyObject*}{PyArray_ContiguousFromObject}{%
      PyObject *op, int type, int min_dim, int max_dim}% Returns an simulation
   object for a contiguous numarray of 'type' created from the sequence object
   'op'.  If 'op' is a contiguous, aligned, non-byteswapped numarray, then the
   simulation object refers to it directly.  Otherwise a well-behaved numarray
   will be created from 'op' and the simulation object will refer to it.
   min_dim and max_dim bound the expected rank as in Numeric.
   \code{min_dim==max_dim} specifies an exact rank.  \code{min_dim==max_dim==0}
   specifies \emph{any} rank.
\end{cfuncdesc}

\begin{cfuncdesc}{PyObject*}{PyArray_CopyFromObject}{%
      PyObject *op, int type, int min_dim, int max_dim}% Returns a contiguous
   array, similar to PyArray_FromContiguousObject, but always returning an
   simulation object referring to a new numarray copied from the original
   sequence.
\end{cfuncdesc}

\begin{cfuncdesc}{PyObject*}{PyArray_FromObject}{%
      PyObject  *op, int type, int min_dim, int max_dim}%
   Returns and simulation object based on 'op', possibly discontiguous.  The
   strides array must be used to access elements of the simulation object.
   
   If 'op' is a byteswapped or misaligned numarray, FromObject creates a
   temporary copy and the simulation object refers to it.
   
   If 'op' is a nonswapped, aligned numarray, the simulation object refers to
   it.
   
   If 'op' is some other sequence, it is converted to a numarray and the
   simulation object refers to that.
\end{cfuncdesc}

\begin{cfuncdesc}{PyObject*}{PyArray_Return}{PyArrayObject *apr}
   Returns simulation object 'apr' to python.  The simulation object itself is
   destructed.  The numarray it refers to (base) is returned as the result of
   the function.
   
   An additional check is (or eventually will be) performed to guarantee that
   rank-0 arrays are converted to appropriate python scalars.
   
   PyArray_Return has no net effect on the reference count of the underlying
   numarray.
\end{cfuncdesc}

\begin{cfuncdesc}{int}{PyArray_As1D}{PyObject **op, char **ptr, int *d1, int typecode}
   Copied from Numeric verbatim.
\end{cfuncdesc}

\begin{cfuncdesc}{int}{PyArray_As2D}{PyObject **op, char ***ptr, int *d1, int *d2, int typecode}
   Copied from Numeric verbatim.
\end{cfuncdesc}

\begin{cfuncdesc}{int}{PyArray_Free}{PyObject *op, char *ptr}
   Copied from Numeric verbatim. \note{This means including bugs and all!}
\end{cfuncdesc}

\begin{cfuncdesc}{int}{PyArray_Check}{PyObject *op}
   This function returns 1 if op is a PyArrayObject.  
\end{cfuncdesc}

\begin{cfuncdesc}{int}{PyArray_Size}{PyObject *op}
   This function returns the total element count of the array.
\end{cfuncdesc}

\begin{cfuncdesc}{int}{PyArray_NBYTES}{PyArrayObject *op}
   This function returns the total size in bytes of the array, and assumes that
bytestride == itemsize, so that the size is product(shape)*itemsize.
\end{cfuncdesc}

\begin{cfuncdesc}{PyObject*}{PyArray_Copy}{PyArrayObject *op}
   This function returns a copy of the array 'op'.  The copy returned is
   guaranteed to be well behaved, i.e. neither byteswapped nor misaligned.
\end{cfuncdesc}

\begin{cfuncdesc}{int}{PyArray_CanCastSafely}{PyArrayObject *op, int type}
  This function returns 1 IFF the array 'op' can be safely cast to 'type',
otherwise it returns 0.
\end{cfuncdesc}

\begin{cfuncdesc}{PyArrayObject*}{PyArray_Cast}{PyArrayObject *op, int type}
  This function casts the array 'op' into an equivalent array of type 'type'.
\end{cfuncdesc}

\begin{cfuncdesc}{PyArray_Descr*}{PyArray_DescrFromType}{int type}
This function returns a pointer to the array descriptor for 'type'.  The
numarray version of PyArray_Descr is incomplete and does not support casting,
getitem, setitem, one, or zero.
\end{cfuncdesc}

\begin{cfuncdesc}{int}{PyArray_isArray(PyObject *o)}
  This macro is designed to fail safe and return 0 when numarray is not
  installed at all.  When numarray is installed, it returns 1 iff object 'o' is
  a numarray, and 0 otherwise.  This macro facilitates the optional use of
  numarray within an extension.
\end{cfuncdesc}

\subsection{Unsupported Numeric Features}
\label{sec:C-API:compat:unsupported}

\begin{itemize}
\item PyArrayError 
\item PyArray_ObjectType() 
\item PyArray_Reshape()
\item PyArray_SetStringFunction() 
\item PyArray_SetNumericOps() 
\item PyArray_Take()
\item UFunc API
\end{itemize}

\section{High-level API}
\label{sec:C-API:high-level-api}

The high-level native API accepts an object (which may or may not be an array)
and transforms the object into an array which satisfies a set of ``behaved-ness
requirements''.  The idea behind the high-level API is to transparently convert
misbehaved numarrays, ordinary sequences, and python scalars into C-arrays.  A
``misbehaved array'' is one which is byteswapped, misaligned, or discontiguous.
This API is the simplest and fastest, provided that your arrays are small.  If
you find your program is exhausting all available memory, it may be time to
look at one of the other APIs.

\subsection{High-level functions}
\label{sec:C-API:high-level-functions}

The high-level support functions for interchanging \class{numarray}s between
\python{} and C are as follows:

\begin{cfuncdesc}{PyArrayObject*}{NA_InputArray}{%
      PyObject *seq, NumarrayType t, int requires}
The purpose of NA_InputArray is to transfer array data from \python to C.
\end{cfuncdesc}

\begin{cfuncdesc}{PyArrayObject*}{NA_OutputArray}{%
      PyObject *seq, NumarrayType t, int requires} The purpose of
NA_OutputArray is to transfer data from C to \python.  The output array must be
a PyArrayObject, i.e. it cannot be an arbitrary Python sequence.
\end{cfuncdesc}

\begin{cfuncdesc}{PyArrayObject*}{NA_IoArray}{%
      PyObject *seq, NumarrayType t, int requires} NA_IoArray has fully
bidirectional data transfer, creating the illusion of call-by-reference.
\end{cfuncdesc}

  For a well-behaved writable array, there is no difference between the three,
  as no temporary is created and the returned object is identical to the
  original object (with an additional reference).  For a mis-behaved input
  array, a well-behaved temporary will be created and the data copied from the
  original to the temporary.  Since it is an input, modifications to its
  contents are not guaranteed to be reflected back to \python, and in the case
  where a temporary was created, won't be.  For a mis-behaved output array, any
  data side-effects generated by the C code will be safely communicated back to
  \python, but the initial array contents are undefined.  For an I/O array, any
  required temporary will be initialized to the same contents as the original
  array, and any side-effects caused by C-code will be copied back to the
  original array.  The array factory routines of the Numeric compatability API
  are written in terms of NA_IoArray.
   
   The return value of each function (\cfunction{NA_InputArray},
   \cfunction{NA_OutputArray}, or \cfunction{NA_IoArray}) is either a reference
   to the original array object, or a reference to a temporary array.
   Following execution of the C-code in the extension function body this
   pointer should \emph{always} be DECREFed.  When a temporary is DECREFed, it
   is deallocated, possibly after copying itself onto the original array.  The
   one exception to this rule is that you should not DECREF an array returned
   via the NA_ReturnOutput function.
   
   The \var{seq} parameter specifies the original numeric sequence to be
   interfaced.  Nested lists and tuples of numbers can be converted by
   \cfunction{NA_InputArray} and \cfunction{NA_IoArray} into a temporary array.
   The temporary is lost on function exit.  Strictly speaking, allowing
   NA_IoArray to accept a list or tuple is a wart, since it will lose any side
   effects.  In principle, communication back to lists and tuples can be
   supported but is not currently.
   
   The \var{t} parameter is an enumeration value which defines the type the
   array data should be converted to.  Arrays of the same type are passed
   through unaltered, while mis-matched arrays are cast into temporaries of the
   specified type.  The value \constant{tAny} may be specified to indicate that
   the extension function can handle any type correctly so no temporary should
   is required.
   
   The \var{requires} integer indicates under what conditions, other than type
   mismatch, a temporary should be made.  The simple way to specify it is to
   use \constant{NUM_C_ARRAY}.  This will cause the API function to make a
   well-behaved temporary if the original is byteswapped, misaligned, or
   discontiguous.  

There is one other pair of high level function which serves to return output
arrays as the function value: NA_OptionalOutputArray and NA_ReturnOutput.

\begin{cfuncdesc}{PyArrayObject*}{NA_OptionalOutputArray}{%
      PyObject *seq, NumarrayType t, int requires, PyObject *master}%
   \cfunction{NA_OptionalOutputArray} is essentially
   \cfunction{NA_OutputArray}, but with one twist: if the original array
   \var{seq} has the value \constant{NULL} or \constant{Py_None}, a copy of
   \var{master} is returned.  This facilitates writing functions where the
   output array may or may-not be specified by the \python{} user.  
\end{cfuncdesc}

\begin{cfuncdesc}{PyObject*}{NA_ReturnOutput}{PyObject *seq, PyObject *shadow}
   \cfunction{NA_ReturnOutput} accepts as parameters both the original
   \var{seq} and the value returned from
   \cfunction{NA_OptionalOutputArray}, \var{shadow}.  If \var{seq} is
   \constant{Py_None} or \constant{NULL}, then \var{shadow} is returned.
   Otherwise, an output array was specified by the user, and \constant{Py_None}
   is returned.  This facilitates writing functions in the numarray style
   where the specification of an output array renders the function ``mute'',
   with all side-effects in the output array and None as the return value.
\end{cfuncdesc}

\subsection{Behaved-ness Requirements}

Calls to the high level API specify a set of requirements that incoming arrays
must satisfy.  The requirements set is specified by a bit mask which is or'ed
together from bits representing individual array requirements.  An ordinary C
array satisfies all 3 requirements: it is contiguous, aligned, and not
byteswapped.  It is possible to request arrays satisfying any or none of the
behavedness requirements.  Arrays which do not satisfy the specified
requirements are transparently ``shadowed'' by temporary arrays which do
satisfy them.  By specifying \constant{NUM_UNCONVERTED}, a caller is certifying
that his extension function can correctly and directly handle the special cases
possible for a \class{NumArray}, excluding type differences.

\begin{verbatim}
typedef enum
{
        NUM_CONTIGUOUS=1,
        NUM_NOTSWAPPED=2,
        NUM_ALIGNED=4,
        NUM_WRITABLE=8,
        NUM_COPY=16,

        NUM_C_ARRAY  = (NUM_CONTIGUOUS | NUM_ALIGNED | NUM_NOTSWAPPED),
        NUM_UNCONVERTED = 0
}
\end{verbatim}

\function{NA_InputArray} will return a guaranteed writable result if
\constant{NUM_WRITABLE} is specified. A writable temporary will be made for
arrays which have readonly buffers.  Any changes made to a writable input array
\emph{may} be lost at extension exit time depending on whether or not a
temporary was required.  \function{NA_InputArray} will also return a guaranteed
writable result by specifying \constant{NUM_COPY}; with \constant{NUM_COPY}, a
temporary is \emph{always} made and changes to it are \emph{always} lost at
extension exit time.

Omitting \constant{NUM_WRITABLE} and \constant{NUM_COPY} from the
\var{requires} of \function{NA_InputArray} asserts that you will not modify the
array buffer in your C code.  Readonly arrays (e.g. from a readonly memory map)
which you attempt to modify can result in a segfault if \constant{NUM_WRITABLE}
or \constant{NUM_COPY} was not specified.

Arrays passed to \function{NA_IoArray} and \function{NA_OutputArray} must be
writable or they will raise an exception; specifing \constant{NUM_WRITABLE} or
\constant{NUM_COPY} to these functions has no effect.

\subsection{Example}
\label{sec:C-API:high-level:example}

A C wrapper function using the high-level API would typically look like the
following.\footnote{This function is taken from the convolve example in the
source distribution.}

\begin{verbatim}
#include "Python.h"
#include "libnumarray.h"

static PyObject *
Py_Convolve1d(PyObject *obj, PyObject *args)
{
        PyObject   *okernel, *odata, *oconvolved=Py_None;
        PyArrayObject *kernel, *data, *convolved;

        if (!PyArg_ParseTuple(args, "OO|O", &okernel, &odata, &oconvolved)) {
                PyErr_Format(_convolveError, 
                             "Convolve1d: Invalid parameters.");
                goto _fail;
        }

\end{verbatim}

First, define local variables and parse parameters.  \cfunction{Py_Convolve1d}
expects two or three array parameters in \var{args}: the convolution kernel,
the data, and optionally the return array.  We define two variables for each
array parameter, one which represents an arbitrary sequence object, and one
which represents a PyArrayObject which contains a conversion of the sequence.
If the sequence object was already a well-behaved numarray, it is returned
without making a copy.

\begin{verbatim}
        /* Align, Byteswap, Contiguous, Typeconvert */
        kernel  = NA_InputArray(okernel, tFloat64, NUM_C_ARRAY);
        data    = NA_InputArray(odata, tFloat64, NUM_C_ARRAY);
        convolved = NA_OptionalOutputArray(oconvolved, tFloat64, NUM_C_ARRAY, data);

        if (!kernel || !data || !convolved) {
                PyErr_Format( _convolveError, 
                             "Convolve1d: error converting array inputs.");
                goto _fail;
        }
\end{verbatim}

These calls to NA_InputArray and OptionalOutputArray require that the arrays be
aligned, contiguous, and not byteswapped, and of type Float64, or a temporary
will be created.  If the user hasn't provided a output array we ask
\cfunction{NA_OptionalOutputArray} to create a copy of the input \var{data}.
We also check that the array screening and conversion process succeeded by
verifying that none of the array pointers is NULL.

\begin{verbatim}
        if ((kernel->nd != 1) || (data->nd != 1)) {
                PyErr_Format(_convolveError,
                      "Convolve1d: arrays must have 1 dimension.");
                goto _fail;
        }

        if (!NA_ShapeEqual(data, convolved)) {
                PyErr_Format(_convolveError,
                "Convolve1d: data and output arrays need identitcal shapes.");
                goto _fail;
        }
\end{verbatim}

Make sure we were passed one-dimensional arrays, and data and output have the
same size.

\begin{verbatim}
        Convolve1d(kernel->dimensions[0], NA_OFFSETDATA(kernel),
                   data->dimensions[0],   NA_OFFSETDATA(data),
                   NA_OFFSETDATA(convolved));
\end{verbatim}

Call the C function actually performing the work.  NA_OFFSETDATA returns the
pointer to the first element of the array,  adjusting for any byteoffset.

\begin{verbatim}
        Py_XDECREF(kernel);
        Py_XDECREF(data);
\end{verbatim}

Decrease the reference counters of the input arrays.  These were increased by
\cfunction{NA_InputArray}.  Py_XDECREF tolerates NULL.  DECREF'ing the
PyArrayObject is how temporaries are released and in the case of
IO and Output arrays, copied back onto the original.

\begin{verbatim}
        /* Align, Byteswap, Contiguous, Typeconvert */
        return NA_ReturnOutput(oconvolved, convolved);
_fail:
        Py_XDECREF(kernel);
        Py_XDECREF(data);
        Py_XDECREF(convolved);
        return NULL;
}
\end{verbatim}

Now return the results, which are either stored in the user-supplied array
\var{oconvolved} and \constant{Py_None} is returned, or if the user didn't
supply an output array the temporary \var{convolved} is returned.

If your C function creates the output array you can use the following sequence
to pass this array back to \python{}:

\begin{verbatim}
        double *result;
        int m, n;
        .
        .
        .
        result = func(...);
        if(NULL == result)
            return NULL;
        return NA_NewArray((void *)result, tFloat64, 2, m, n);
}
\end{verbatim}

The C function \cfunction{func} returns a newly allocated (m, n) array in
\var{result}.  After we check that everything is ok, we create a new numarray
using \cfunction{NA_NewArray} and pass it back to \python.  \cfunction{NA_NewArray}
creates a \class{numarray} with \constant{NUM_C_ARRAY} properties.  If you wish to
create an array that is byte-swapped, or misaligned, you can use
\cfunction{NA_NewAll}.

The C-code of the core convolution function is uninteresting.  The main point
of the example is that when using the high-level API, numarray specific code is
confined to the wrapper function.  The interface for the core function can be
written in terms of primitive numarray/C data items, not objects.  This is
possible because the high level API can be used to deliver C arrays.

\begin{verbatim}
static void Convolve1d(long ksizex, Float64 *kernel, 
     long dsizex, Float64*data, Float64 *convolved) 
{ 
  long xc; long halfk = ksizex/2;

  for(xc=0; xc<halfk; xc++)
      convolved[xc] = data[xc];
  
  for(xc=halfk; xc<dsizex-halfk; xc++) {
      long xk;
      double temp = 0;
      for (xk=0; xk<ksizex; xk++)
         temp += kernel[xk]*data[xc-halfk+xk];
      convolved[xc] = temp;
  }
  
  for(xc=dsizex-halfk; xc<dsizex; xc++)
     convolved[xc] = data[xc];
}
\end{verbatim}

\section{Element-wise API}
\label{sec:C-API:element-wise-api}

The element-wise in-place API is a family of macros and functions designed to
get and set elements of arrays which might be byteswapped, misaligned,
discontiguous, or of a different type.  You can obtain \class{PyArrayObject}s
for these misbehaved arrays from the high-level API by specifying fewer
requirements (perhaps just 0, rather than NUM_C_ARRAY).  In this way, you can
avoid the creation of temporaries at a cost of accessing your array with these
macros and functions and a significant performance penalty.  Make no mistake,
if you have the memory, the high level API is the fastest.  The whole point of
this API is to support cases where the creation of temporaries exhausts either
the physical or virtual address space.  Exhausting physical memory will result
in thrashing, while exhausting the virtual address space will result in program
exception and failure.  This API supports avoiding the creation of the
temporaries, and thus avoids exhausting physical and virual memory, possibly
improving net performance or even enabling program success where simpler
methods would just fail.

\subsection{Element-wise functions}
\label{sec:C-API:element-wise:functions}

The single element macros each access one element of an array at a time, and
specify the array type in two places: as part of the PyArrayObject type
descriptor, and as ``type''.  The former defines what the array is, and the
latter is required to produce correct code from the macro.  They should
\emph{match}.  When you pass ``type'' into one of these macros, you are
defining the kind of array the code can operate on.  It is an error to pass a
non-matching array to one of these macros.  One last piece of advice: call
these macros carefully, because the resulting expansions and error messages are
a *obscene*.  Note: the type parameter for a macro is one of the Numarray
Numeric Data Types, not a NumarrayType enumeration value.

\subsubsection{Pointer based single element macros}
\label{sec:C-API:pointer-based-single}

\begin{cfuncdesc}{}{NA_GETPa}{PyArrayObject*, type, char*}
   aligning
\end{cfuncdesc}
\begin{cfuncdesc}{}{NA_GETPb}{PyArrayObject*, type, char*}
   byteswapping
\end{cfuncdesc}
\begin{cfuncdesc}{}{NA_GETPf}{PyArrayObject*, type, char*}
   fast (well-behaved)
\end{cfuncdesc}
\begin{cfuncdesc}{}{NA_GETP}{PyArrayObject*,  type, char*}
   testing: any of above
\end{cfuncdesc}
\begin{cfuncdesc}{}{NA_SETPa}{PyArrayObject*, type, char*, v}
\end{cfuncdesc}
\begin{cfuncdesc}{}{NA_SETPb}{PyArrayObject*, type, char*, v}
\end{cfuncdesc}
\begin{cfuncdesc}{}{NA_SETPf}{PyArrayObject*, type, char*, v}
\end{cfuncdesc}
\begin{cfuncdesc}{}{NA_SETP}{PyArrayObject*,  type, char*, v}
\end{cfuncdesc}

\subsubsection{One index single element macros}
\begin{cfuncdesc}{}{NA_GET1a}{PyArrayObject*, type, i}
\end{cfuncdesc}
\begin{cfuncdesc}{}{NA_GET1b}{PyArrayObject*, type, i}
\end{cfuncdesc}
\begin{cfuncdesc}{}{NA_GET1f}{PyArrayObject*, type, i}
\end{cfuncdesc}
\begin{cfuncdesc}{}{NA_GET1}{PyArrayObject*,  type, i}
\end{cfuncdesc}
\begin{cfuncdesc}{}{NA_SET1a}{PyArrayObject*, type, i, v}
\end{cfuncdesc}
\begin{cfuncdesc}{}{NA_SET1b}{PyArrayObject*, type, i, v}
\end{cfuncdesc}
\begin{cfuncdesc}{}{NA_SET1f}{PyArrayObject*, type, i, v}
\end{cfuncdesc}
\begin{cfuncdesc}{}{NA_SET1}{PyArrayObject*,  type, i, v}
\end{cfuncdesc}

\subsubsection{Two index single element macros}
\begin{cfuncdesc}{}{NA_GET2a}{PyArrayObject*, type, i, j}
\end{cfuncdesc}
\begin{cfuncdesc}{}{NA_GET2b}{PyArrayObject*, type, i, j}
\end{cfuncdesc}
\begin{cfuncdesc}{}{NA_GET2f}{PyArrayObject*, type, i, j}
\end{cfuncdesc}
\begin{cfuncdesc}{}{NA_GET2}{PyArrayObject*,  type, i, j}
\end{cfuncdesc}
\begin{cfuncdesc}{}{NA_SET2a}{PyArrayObject*, type, i, j, v}
\end{cfuncdesc}
\begin{cfuncdesc}{}{NA_SET2b}{PyArrayObject*, type, i, j, v}
\end{cfuncdesc}
\begin{cfuncdesc}{}{NA_SET2f}{PyArrayObject*, type, i, j, v}
\end{cfuncdesc}
\begin{cfuncdesc}{}{NA_SET2}{PyArrayObject*,  type, i, j, v}
\end{cfuncdesc}

\subsubsection{One and Two Index, Offset, Float64/Complex64/Int64 functions}

The \class{Int64}/\class{Float64}/\class{Complex64} functions require a
function call to access a single element of an array, making them slower than
the single element macros.  They have two advantages:
\begin{enumerate}
\item They're function calls, so they're a little more robust. 
\item They can handle \emph{any} input array type and behavior properties.
\end{enumerate}


While these functions have no error return status, they *can* alter the Python
error state, so well written extensions should call
\cfunction{PyErr_Occurred()} to determine if an error occurred and report it.
It's reasonable to do this check once at the end of an extension function,
rather than on a per-element basis.


\begin{cfuncdesc}{}{void NA_get_offset}{PyArrayObject *, int N, ...}
  \cfunction{NA_get_offset} computes the offset into an array object given a
  variable number of indices.  It is not especially robust, and it is
  considered an error to pass it more indices than the array has, or indices
  which are negative or out of range.
\end{cfuncdesc}

\begin{cfuncdesc}{Float64}{NA_get_Float64}{PyArrayObject *, long offset}
\end{cfuncdesc}
\begin{cfuncdesc}{void}{NA_set_Float64}{PyArrayObject *, long offset, Float64 v}
\end{cfuncdesc}
\begin{cfuncdesc}{Float64}{NA_get1_Float64}{PyArrayObject *, int i}
\end{cfuncdesc}
\begin{cfuncdesc}{void}{NA_set1_Float64}{PyArrayObject *, int i, Float64 v}
\end{cfuncdesc}
\begin{cfuncdesc}{Float64}{NA_get2_Float64}{PyArrayObject *, int i, int j}
\end{cfuncdesc}
\begin{cfuncdesc}{void}{NA_set2_Float64}{PyArrayObject *, int i, int j, Float64 v}
\end{cfuncdesc}

\begin{cfuncdesc}{Int64}{NA_get_Int64}{PyArrayObject *, long offset}
\end{cfuncdesc}
\begin{cfuncdesc}{void}{NA_set_Int64}{PyArrayObject *, long offset, Int64 v}
\end{cfuncdesc}
\begin{cfuncdesc}{Int64}{NA_get1_Int64}{PyArrayObject *, int i}
\end{cfuncdesc}
\begin{cfuncdesc}{void}{NA_set1_Int64}{PyArrayObject *, int i, Int64 v}
\end{cfuncdesc}
\begin{cfuncdesc}{Int64}{NA_get2_Int64}{PyArrayObject *, int i, int j}
\end{cfuncdesc}
\begin{cfuncdesc}{void}{NA_set2_Int64}{PyArrayObject *, int i, int j, Int64 v}
\end{cfuncdesc}

\begin{cfuncdesc}{Complex64}{NA_get_Complex64}{PyArrayObject *, long offset}
\end{cfuncdesc}
\begin{cfuncdesc}{void}{NA_set_Complex64}{PyArrayObject *, long offset, Complex64 v}
\end{cfuncdesc}
\begin{cfuncdesc}{Complex64}{NA_get1_Complex64}{PyArrayObject *, int i}
\end{cfuncdesc}
\begin{cfuncdesc}{void}{NA_set1_Complex64}{PyArrayObject *, int i, Complex64 v}
\end{cfuncdesc}
\begin{cfuncdesc}{Complex64}{NA_get2_Complex64}{PyArrayObject *, int i, int j}
\end{cfuncdesc}
\begin{cfuncdesc}{void}{NA_set2_Complex64}{PyArrayObject *, int i, int j, Complex64 v}
\end{cfuncdesc}

\subsection{Example}
\label{sec:C-API:element-wise:example}

The \cfunction{convolve1D} wrapper function corresponding to section
\ref{sec:C-API:high-level:example} using the element-wise API could look
like:\footnote{This function is also available as an example in the source
   distribution.}

\begin{verbatim}
static PyObject *
Py_Convolve1d(PyObject *obj, PyObject *args)
{
        PyObject   *okernel, *odata, *oconvolved=Py_None;
        PyArrayObject *kernel, *data, *convolved;

        if (!PyArg_ParseTuple(args, "OO|O", &okernel, &odata, &oconvolved)) {
                PyErr_Format(_Error, "Convolve1d: Invalid parameters.");
                goto _fail;
        }

        kernel = NA_InputArray(okernel, tAny, 0);
        data   = NA_InputArray(odata, tAny, 0);
\end{verbatim}

For the kernel and data arrays, \class{numarray}s of any type are accepted
without conversion.  Thus there is no copy of the data made except for lists or
tuples.  All types, byteswapping, misalignment, and discontiguity must be
handled by Convolve1d.  This can be done easily using the get/set functions.
Macros, while faster than the functions, can only handle a single type.

\begin{verbatim}
        convolved = NA_OptionalOutputArray(oconvolved, tFloat64, 0, data);
\end{verbatim}

Also for the output array we accept any variety of type tFloat without
conversion.  No copy is made except for non-tFloat.  Non-numarray sequences are
not permitted as output arrays.  Byteswaping, misaligment, and discontiguity
must be handled by Convolve1d.  If the \python caller did not specify the
oconvolved array, it initially retains the value Py_None.  In that case,
\var{convolved} is cloned from the array \var{data} using the specified type.
It is important to clone from \var{data} and not \var{odata}, since the latter
may be an ordinary \python sequence which was converted into numarray
\var{data}.  

\begin{verbatim}
        if (!kernel || !data || !convolved)
                goto _fail;

        if ((kernel->nd != 1) || (data->nd != 1)) {
                PyErr_Format(_Error,
                     "Convolve1d: arrays must have exactly 1 dimension.");
                goto _fail;
        }

        if (!NA_ShapeEqual(data, convolved)) {
                PyErr_Format(_Error,
                    "Convolve1d: data and output arrays must have identical length.");
                goto _fail;
        }
        if (!NA_ShapeLessThan(kernel, data)) {
                PyErr_Format(_Error,
                    "Convolve1d: kernel must be smaller than data in both dimensions");
                goto _fail;
        }
        
        if (Convolve1d(kernel, data, convolved) < 0)  /* Error? */
            goto _fail;
        else {
           Py_XDECREF(kernel);
           Py_XDECREF(data);
           return NA_ReturnOutput(oconvolved, convolved);
        }
_fail:
        Py_XDECREF(kernel);
        Py_XDECREF(data);
        Py_XDECREF(convolved);
        return NULL;
}

\end{verbatim}

This function is very similar to the high-level API wrapper, the notable
difference is that we ask for the unconverted arrays \var{kernel} and
\var{data} and \var{convolved}.  This requires some attention in their usage.
The function that does the actual convolution in the example has to use
\cfunction{NA_get*} to read and \cfunction{NA_set*} to set an element of these
arrays, instead of using straight array notation.  These functions perform any
necessary type conversion, byteswapping, and alignment.

\begin{verbatim}
static int
Convolve1d(PyArrayObject *kernel, PyArrayObject *data, PyArrayObject *convolved)
{
        int xc, xk;
        int ksizex = kernel->dimensions[0];
        int halfk = ksizex / 2;
        int dsizex = data->dimensions[0];

        for(xc=0; xc<halfk; xc++)
                NA_set1_Float64(convolved, xc, NA_get1_Float64(data, xc));
                     
        for(xc=dsizex-halfk; xc<dsizex; xc++)
                NA_set1_Float64(convolved, xc, NA_get1_Float64(data, xc));

        for(xc=halfk; xc<dsizex-halfk; xc++) {
                Float64 temp = 0;
                for (xk=0; xk<ksizex; xk++) {
                        int i = xc - halfk + xk;
                        temp += NA_get1_Float64(kernel, xk) * 
                                NA_get1_Float64(data, i);
                }
                NA_set1_Float64(convolved, xc, temp);
        }
        if (PyErr_Occurred())
           return -1;
        else 
           return 0;
}
\end{verbatim}

\section{One-dimensional API}
\label{sec:C-API:One-dimensional-api}

The 1D in-place API is a set of functions for getting/setting elements from the
innermost dimension of an array.  These functions improve speed by moving type
switches, ``behavior tests'', and function calls out of the per-element loop.
The functions get/set a series of consequtive array elements to/from arrays of
\class{Int64}, \class{Float64}, or \class{Complex64}.  These functions are
(even) more intrusive than the single element functions, but have better
performance in many cases.  They can operate on arrays of any type, with the
exception of the Complex64 functions, which only handle Complex64.  The
functions return 0 on success and -1 on failure, with the Python error state
already set.  To be used profitably, the 1D API requires either a large single
dimension which can be processeed in blocks or a multi-dimensional array such
as an image.  In the latter case, the 1D API is suitable for processing one (or
more) scanlines at a time rather than the entire image at once.  See the source
distribution Examples/convolve/one_dimensionalmodule.c for an example of usage.

\begin{cfuncdesc}{long}{NA_get_offset}{PyArrayObject *, int N, ...}
   This function applies a (variable length) set of \var{N} indices to an array
   and returns a byte offset into the array.
\end{cfuncdesc}

\begin{cfuncdesc}{int}{NA_get1D_Int64}{%
      PyArrayObject *, long offset, int cnt, Int64 *out}%
\end{cfuncdesc}

\begin{cfuncdesc}{int}{NA_set1D_Int64}{%
      PyArrayObject *, long offset, int cnt, Int64 *in}%
\end{cfuncdesc}

\begin{cfuncdesc}{int}{NA_get1D_Float64}{%
      PyArrayObject *, long offset, int cnt, Float64 *out}%
\end{cfuncdesc}

\begin{cfuncdesc}{int}{NA_set1D_Float64}{%
      PyArrayObject *, long offset, int cnt, Float64 *in}%
\end{cfuncdesc}

\begin{cfuncdesc}{int}{NA_get1D_Complex64}{%
      PyArrayObject *, long offset, int cnt, Complex64 *out}%
\end{cfuncdesc}

\begin{cfuncdesc}{int}{NA_set1D_Complex64}{%
      PyArrayObject *, long offset, int cnt, Complex64 *in}%
\end{cfuncdesc}

\section{New numarray functions}
\label{sec:C-API:new-numarray-functions}

The following array creation functions share similar behavior.  All but one
create a new \class{numarray} using the data specified by \var{data}.  If
\var{data} is NULL, the routine allocates a buffer internally based on the
array shape and type; internally allocated buffers have undefined contents.
The data type of the created array is specified by \var{type}.

There are several functions to create \class{numarray}s at the C level:

\begin{cfuncdesc}{static PyArrayObject*}{NA_NewArray}{%
    void *data, NumarrayType type, int ndim, ...}% 

  \var{ndim} specifies the rank of the array (number of dimensions), and the
  length of each dimension must be given as the remaining (variable length)
  list of \emph{int} parameters.  The following example allocates a 100x100
  uninitialized array of Int32.
\begin{verbatim}
  if (!(array = NA_NewArray(NULL, tInt32, 2, 100, 100)))
      return NULL;
\end{verbatim}
\end{cfuncdesc}

\begin{cfuncdesc}{static PyObject*}{NA_vNewArray}{%
    void *data, NumarrayType type, int ndim, maybelong *shape}% 

  For \function{NA_vNewArray} the length of each dimension must be given in an
  array of \var{maybelong} pointed to by \var{shape}. The following code
  allocates a 2x2 array initialized to a copy of the specified \var{data}.
  \begin{verbatim}
    Int32 data[4] = { 1, 2, 3, 4 };
    maybelong shape[2] = { 2, 2 };
    if (!(array = NA_vNewArray(data, tInt32, 2, shape)))
       return NULL;
  \end{verbatim}
\end{cfuncdesc}

\begin{cfuncdesc}{static PyArrayObject*}{NA_NewAll}{%
    int ndim, maybelong *shape, NumarrayType type, void *data, maybelong
    byteoffset, maybelong bytestride, int byteorder, int aligned, int
    writable}%

    \function{NA_NewAll} is similar to \function{NA_vNewArray} except it
    provides for the specification of additional parameters. \var{byteoffset}
    specifies the byte offset from the base of the data array at which the
    \var{real} data begins.  \var{bytestride} specifies the miminum stride to
    use, the seperation in bytes between adjacent elements in the
    array. \var{byteorder} takes one of the values \constant{NUM_BIG_ENDIAN} or
    \constant{NUM_LITTLE_ENDIAN}.  \var{writable} defines whether the buffer
    object associated with the resulting array is readonly or writable.
\end{cfuncdesc}

\begin{cfuncdesc}{static PyArrayObject*}{NA_NewAllStrides}{%
    int ndim, maybelong *shape, maybelong *strides, NumarrayType type, void
    *data, maybelong byteoffset, maybelong byteorder, int aligned, int
    writable}% 

    \function{NA_NewAllStrides} is a variant of \function{NA_vNewAll} which
    also permits the specification of the array strides.  The strides are not
    checked for correctness.
\end{cfuncdesc}

\begin{cfuncdesc}{static PyArrayObject*}{NA_NewAllFromBuffer}{%
    int ndim, maybelong *shape, NumarrayType type, PyObject *buffer, maybelong
    byteoffset, maybelong bytestride, int byteorder, int aligned, int
    writable}% 

   \function{NA_NewAllFromBuffer} is similar to \function{NA_NewAll} except it
   accepts a buffer object rather than a pointer to C data.  The \var{buffer}
   object must support the buffer protocol.  If \var{buffer} is non-NULL, the
   returned array object stores a reference to \var{buffer} and locates its
   data there.  If \var{buffer} is specified as NULL, a buffer object and
   associated data space are allocated internally and the returned array object
   refers to that.  It is possible to create a Python buffer object from an
   array of C data and then construct a \class{numarray} using this function
   which refers to the C data without making a copy.
\end{cfuncdesc}

\begin{cfuncdesc}{int}{NA_ShapeEqual}{PyArrayObject*a,PyArrayObject*b}
This function compares the shapes of two arrays, and returns 1 if they
are the same, 0 otherwise.
\end{cfuncdesc}

\begin{cfuncdesc}{int}{NA_ShapeLessThan}{PyArrayObject*a,PyArrayObject*b}
This function compares the shapes of two arrays, and returns 1 if each
dimension of 'a' is less than the corresponding dimension of 'b', 0 otherwise.
\end{cfuncdesc}

\begin{cfuncdesc}{int}{NA_ByteOrder}{}
This function returns the system byte order, either NUM_LITTLE_ENDIAN or
NUM_BIG_ENDIAN.
\end{cfuncdesc}

\begin{cfuncdesc}{Bool}{NA_IeeeMask32}{Float32 value, Int32 mask}
This function returns 1 IFF Float32 \var{value} matches any of the IEEE special
value criteria specified by \var{mask}.  See ieeespecial.h for the mask bit
values which can be or'ed together to specify mask.
\function{NA_IeeeSpecial32} has been deprecated and will eventually be removed.
\end{cfuncdesc}

\begin{cfuncdesc}{Bool}{NA_IeeeMask64}{Float64 value,Int32 mask}
This function returns 1 IFF Float64 \var{value} matches any of the IEEE special
value criteria specified by \var{mask}.  See ieeespecial.h for the mask bit
values which can be or'ed together to specify mask.
\function{NA_IeeeSpecial64} has been deprecated and will eventually be removed.
\end{cfuncdesc}

\begin{cfuncdesc}{PyArrayObject *}{NA_updateDataPtr}{PyArrayObject *}
This function updates the values derived from the ``_data'' buffer, namely the
data pointer and buffer WRITABLE flag bit.  This needs to be called upon
entering or re-entering C-code from Python, since it is possible for buffer
objects to move their data buffers as a result of executing arbitrary Python
and hence arbitrary C-code.  The high level interface routines,
e.g. \function{NA_InputArray}, call this routine automatically.
\end{cfuncdesc}

\begin{cfuncdesc}{char*}{NA_typeNoToName}{int}
NA_typeNoToName translates a NumarrayType into a character string which can be
used to display it:  e.g.  tInt32 converts to the string ``Int32''
\end{cfuncdesc}

\begin{cfuncdesc}{PyObject*}{NA_typeNoToTypeObject}{int}
This function converts a NumarrayType C type code into the NumericType object
which implements and represents it more fully.  tInt32 converts to the type
object numarray.Int32.  
\end{cfuncdesc}

\begin{cfuncdesc}{int}{NA_typeObjectToTypeNo}{PyObject*}
This function converts a numarray type object (e.g. numarray.Int32) into the
corresponding NumarrayType (e.g. tInt32) C type code. 
\end{cfuncdesc}

\begin{cfuncdesc} {PyObject*}{NA_intTupleFromMaybeLongs}{int,maybelong*}
This function creates a tuple of Python ints from an array of C maybelong integers.
\end{cfuncdesc}

\begin{cfuncdesc}{long}{NA_maybeLongsFromIntTuple}{int,maybelong*,PyObject*}
This function fills an array of C long integers with the converted values from
a tuple of Python ints.  It returns the number of conversions, or -1 for error.
\end{cfuncdesc}

\begin{cfuncdesc}{long}{NA_isIntegerSequence}{PyObject*}
This function returns 1 iff the single parameter is a sequence of Python
integers, and 0 otherwise.
\end{cfuncdesc}

\begin{cfuncdesc}{PyObject*}{NA_setArrayFromSequence}{PyArrayObject*,PyObject*}
This function copies the elementwise from a sequence object to a numarray.
\end{cfuncdesc}

\begin{cfuncdesc}{int}{NA_maxType}{PyObject*}
This function returns an integer code corresponding to the highest kind of
Python numeric object in a sequence.  INT(0) LONG(1) FLOAT(2) COMPLEX(3).
On error -1 is returned.
\end{cfuncdesc}

\begin{cfuncdesc}{PyObject*}{NA_getPythonScalar}{PyArrayObject *a, long offset}
This function returns the Python object corresponding to the single element of 
the array 'a' at the given byte offset.
\end{cfuncdesc}

\begin{cfuncdesc}{int}{NA_setFromPythonScalar}{PyArrayObject *a, long offset, PyObject*value}
This function sets the single element of the array 'a' at the given byte
offset to 'value'.
\end{cfuncdesc}

\begin{cfuncdesc}{int}{NA_NDArrayCheck}{PyObject*o}
This function returns 1 iff the 'o' is an instance of NDArray or an instance of
a subclass of NDArray, and 0 otherwise.
\end{cfuncdesc}

\begin{cfuncdesc}{int}{NA_NumArrayCheck}{PyObject*}
This function returns 1 iff the 'o' is an instance of NumArray or an instance of
a subclass of NumArray, and 0 otherwise.
\end{cfuncdesc}

\begin{cfuncdesc}{int}{NA_ComplexArrayCheck}{PyObject*}
This function returns 1 iff the 'o' is an instance of ComplexArray or an instance of
a subclass of ComplexArray, and 0 otherwise.
\end{cfuncdesc}

\begin{cfuncdesc}{unsigned long}{NA_elements}{PyArrayObject*}
This function returns the total count of elements in an array,  essentially the
product of the elements of the array's shape.
\end{cfuncdesc}

\begin{cfuncdesc}{PyArrayObject *}{NA_copy}{PyArrayObject*}
This function returns a copy of the given array.  The array copy is guaranteed
to be well-behaved, i.e. neither byteswapped, misaligned, nor discontiguous.
\end{cfuncdesc}

\begin{cfuncdesc}{int}{NA_copyArray}{PyArrayObject*to, const PyArrayObject *from}
This function returns a copies one array onto another;  used in f2py.
\end{cfuncdesc}

\begin{cfuncdesc}{int}{NA_swapAxes}{PyArrayObject*a, int dim1, int dim2}
This function mutates the specified array \var{a} to exchange the shape and
strides values for the two dimensions, \var{dim1} and \var{dim2}.  Negative
dimensions count backwards from the innermost, with -1 being the innermost
dimension.  Returns 0 on success and -1 on error.
\end{cfuncdesc}

%% Local Variables:
%% mode: LaTeX
%% mode: auto-fill
%% fill-column: 79
%% indent-tabs-mode: nil
%% ispell-dictionary: "american"
%% reftex-fref-is-default: nil
%% TeX-auto-save: t
%% TeX-command-default: "pdfeLaTeX"
%% TeX-master: "numarray"
%% TeX-parse-self: t
%% End:




\part{Extension modules}

\label{part:numarray-extensions}

\chapter{Convolution}
\label{cha:convolve}

%begin{latexonly}
\makeatletter
\py@reset
\makeatother
%end{latexonly}
\declaremodule{extension}{numarray.convolve}
\moduleauthor{The numarray team}{numpy-discussion@lists.sourceforge.net}
\modulesynopsis{Convolution,correlation}

\begin{quote}
   This package (numarray.convolve) provides functions for one- and 
   two-dimensional convolutions and correlations of \class{numarray}s.
   Each of the following examples assumes that the following code has been 
   executed:
\begin{verbatim}
import numarray.convolve as conv
\end{verbatim}
\end{quote}


\section{Convolution functions}
\label{sec:CONV:convolution-functions}

\begin{funcdesc}{boxcar}{data, boxshape, output=None, mode='nearest', cval=0.0}
   \function{boxcar} computes a 1-D or 2-D boxcar filter on every 1-D or
   2-D subarray of \constant{data}. \constant{boxshape} is a tuple of integers
   specifying the dimensions of the filter, e.g. \code{(3,3)}.  If
   \constant{output} is specified, it should be the same shape as
   \constant{data} and the result will be stored in it.  In that case
   \class{None} will be returned.
        
   \constant{mode} can be any of the following values:
   \begin{description}
   \item[\var{nearest}]: Elements beyond boundary come from nearest edge pixel.
   \item[\var{wrap}]: Elements beyond boundary come from the opposite array
      edge.
   \item[\var{reflect}]: Elements beyond boundary come from reflection on same
      array edge.
   \item[\var{constant}]: Elements beyond boundary are set to what is specified
      in \constant{cval}, an optional numerical parameter; the default value is
      \code{0.0}.
   \end{description}        
\end{funcdesc}
\begin{verbatim}
>>> print a
[1 5 4 7 2 9 3 6]
>>> print conv.boxcar(a,(3,))
[ 2.33333333  3.33333333  5.33333333  4.33333333  6.          4.66666667
  6.          5.        ]
# for even number box size, it will take the extra point from the lower end
>>> print conv.boxcar(a,(2,))
[ 1.   3.   4.5  5.5  4.5  5.5  6.   4.5]
\end{verbatim}

\begin{funcdesc}{convolve}{data, kernel, mode=FULL}
   \label{func:CONV:convolve}
   Returns the discrete, linear convolution of 1-D sequences \constant{data} 
   and \constant{kernel}; \constant{mode} can be \class{VALID}, \class{SAME}, 
   or \code{FULL} to specify the size of the resulting sequence.  See section
   \ref{sec:CONV:global-constants}.
\end{funcdesc}

\begin{funcdesc}{convolve2d}{data, kernel, output=None, fft=0, mode='nearest', 
    cval=0.0} Return the 2-dimensional convolution of \constant{data} and
  \constant{kernel}.  If \constant{output} is not \class{None}, the result is
  stored in \constant{output} and \class{None} is returned.  \constant{fft} is
  used to switch between FFT-based convolution and the naive algorithm,
  defaulting to naive.  Using \constant{fft} mode becomes more beneficial as
  the size of the kernel grows; for small kernels, the naive algorithm is more
  efficient.  \constant{mode} has the same choices as those of
  \function{boxcar}.  A number of storage considerations come into play with
  large arrays: (1) boundary modes are implemented by making an oversized
  temporary copy of the \constant{data} array which has a shape equal to the
  sum of the \constant{data} and \constant{kernel} shapes.  (2) likewise, the
  \constant{kernel} is copied into an array with the same shape as the
  oversized \constant{data} array.  (3) In FFT mode, the fourier transforms of
  the \constant{data} and \constant{kernel} arrays are stored in double
  precision complex temporaries. The aggregate effect is that storage roughly
  equal to a factor of eight (x2 from 2 and x4 from 3) times the size of the
  \constant{data} is required to compute the convolution of a Float32
  \constant{data} array.
\end{funcdesc}

\begin{funcdesc}{correlate}{data, kernel, mode=FULL}
   Return the cross-correlation of \constant{data} and \constant{kernel};
   \constant{mode} can be \class{VALID}, \class{SAME}, or \code{FULL} to 
   specify the size of the resulting sequence.  \function{correlate} is
   very closely related to \function{convolve} in implementation.
   See section \ref{sec:CONV:global-constants}.
\end{funcdesc}

\begin{funcdesc}{correlate2d}{data, kernel, output=None, fft=0, mode='nearest', cval=0.0}
   \label{func:CONV:correlate2d}
  Return the 2-dimensional convolution of \constant{data} and
  \constant{kernel}.  If \constant{output} is not \class{None}, the result is
  stored in \constant{output} and \class{None} is returned.  \constant{fft} is
  used to switch between FFT-based convolution and the naive algorithm,
  defaulting to naive.  Using \constant{fft} mode becomes more beneficial as
  the size of the kernel grows; for small kernels, the naive algorithm is more
  efficient.  \constant{mode} has the same choices as those of
  \function{boxcar}.  See also \function{convolve2d} for notes regarding 
  storage consumption.
\end{funcdesc}

\note{\function{cross_correlate} is deprecated and should not be used.}



\section{Global constants}
\label{sec:CONV:global-constants}

These constants specify what part of the result the \function{convolve} and
\function{correlate} functions of this module return.  Each of the following
examples assumes that the following code has been executed:

\begin{verbatim}
arr = numarray.arange(8)
\end{verbatim}

\begin{datadesc}{FULL}
   Return the full convolution or correlation of two arrays.
\begin{verbatim}
>>> conv.correlate(arr, [1, 2, 3], mode=conv.FULL)
array([ 0,  3,  8, 14, 20, 26, 32, 38, 20,  7])
\end{verbatim}
\end{datadesc}

\begin{datadesc}{PASS}
   Correlate the arrays without padding the data.
\begin{verbatim}
>>> conv.correlate(arr, [1, 2, 3], mode=conv.PASS)
array([ 0,  8, 14, 20, 26, 32, 38,  7])
\end{verbatim}
\end{datadesc}

\begin{datadesc}{SAME}
   Return the part of the convolution or correlation of two arrays that
   corresponds to an array of the same shape as the input data.
\begin{verbatim}
>>> conv.correlate(arr, [1, 2, 3], mode=conv.SAME)
array([ 3,  8, 14, 20, 26, 32, 38, 20])
\end{verbatim}
\end{datadesc}

\begin{datadesc}{VALID}
   Return the valid part of the convolution or correlation of two arrays.
\begin{verbatim}
>>> conv.correlate(arr, [1, 2, 3], mode=conv.VALID)
array([ 8, 14, 20, 26, 32, 38])
\end{verbatim}
\end{datadesc}



%% Local Variables:
%% mode: LaTeX
%% mode: auto-fill
%% fill-column: 79
%% indent-tabs-mode: nil
%% ispell-dictionary: "american"
%% reftex-fref-is-default: nil
%% TeX-auto-save: t
%% TeX-command-default: "pdfeLaTeX"
%% TeX-master: "numarray"
%% TeX-parse-self: t
%% End:

\chapter{Fast-Fourier-Transform}
\label{cha:fft}

%begin{latexonly}
\makeatletter \py@reset \makeatother
%end{latexonly}
\declaremodule{extension}{numarray.fft}
\moduleauthor{The numarray team}{numpy-discussion@lists.sourceforge.net}
\modulesynopsis{Fast Fourier Transform}

\begin{quote}
   This package provides functions for one- and two-dimensional
   Fast-Fourier-Transforms (FFT) and inverse FFTs.
\end{quote}


The \module{numarray.fft} module provides a simple interface to the FFTPACK
Fortran library, which is a powerful standard library for doing fast Fourier
transforms of real and complex data sets, or the C fftpack library, which is
algorithmically based on FFTPACK and provides a compatible interface.


\section{Installation}
\label{sec:FFT:installation}

The default installation uses the provided \module{numarray.fft.fftpack} C
implementation of these routines and this works without any further
interaction.


\subsection{Installation using FFTPACK}
\label{sec:FFT:install-lapack}

On some platforms, precompiled optimized versions of the FFTPACK libraries are
preinstalled on the operating system, and the setup procedure needs to be
modified to force the \module{numarray.fft} module to be linked against those
rather than the builtin replacement functions.




\section{FFT Python Interface}
\label{sec:FFT:python-interface}

The Python user imports the \module{numarray.fft} module, which provides 
a set of utility functions of the most commonly used FFT routines, and
allows the specification of which axes (dimensions) of the input arrays are to
be used for the FFT's. These routines are:

\begin{funcdesc}{fft}{a, n=None, axis=-1} 
   Performs a \constant{n}-point discrete Fourier transform of the array 
   \constant{a}, \constant{n} defaults to the size of \constant{a}. It is 
   most efficient for \constant{n} a power of two. If \constant{n} is 
   larger than \code{len(a)}, then \constant{a} will be
   zero-padded to make up the difference. If \constant{n} is smaller than
   \code{len(a)}, then \constant{a} will be aliased to reduce its size. This
   also stores a cache of working memory for different sizes of 
   \module{fft}'s, so you could theoretically run into memory problems if 
   you call this too many times with too many different \constant{n}'s.
   
   The FFT is performed along the axis indicated by the \constant{axis} 
   argument, which defaults to be the last dimension of \constant{a}.
   
   The format of the returned array is a complex array of the same shape as
   \constant{a}, where the first element in the result array contains the DC
   (steady-state) value of the FFT.
   \remark{missing: ..., and where each successive ...}

   Some examples are:
\begin{verbatim}
>>> a = array([1., 0., 1., 0., 1., 0., 1., 0.]) + 10
>>> b = array([0., 1., 0., 1., 0., 1., 0., 1.]) + 10
>>> c = array([0., 1., 0., 0., 0., 1., 0., 0.]) + 10
>>> print numarray.fft.fft(a).real
[ 84.   0.   0.   0.   4.   0.   0.   0.]
>>> print numarray.fft.fft(b).real
[ 84.   0.   0.   0.  -4.   0.   0.   0.]
>>> print numarray.fft.fft(c).real
[ 82.   0.   0.   0.  -2.   0.   0.   0.]
\end{verbatim}
\end{funcdesc}
       
\begin{funcdesc}{inverse_fft}{a, n=None, axis=-1}
   Will return the \constant{n} point inverse discrete Fourier transform of
   \constant{a}; \constant{n} defaults to the length of \constant{a}. 
   It is most efficient for \constant{n} a power of two.  If \constant{n} 
   is larger than \constant{a}, then \constant{a} will be zero-padded to 
   make up the difference.  If \constant{n} is smaller than \constant{a}, 
   then \constant{a} will be aliased to reduce its size.
   This also stores a cache of working memory for different sizes of FFT's, so
   you could theoretically run into memory problems if you call this too many
   times with too many different \constant{n}'s.
\end{funcdesc}
       
\begin{funcdesc}{real_fft}{a, n=None, axis=-1}
   Will return the \constant{n} point discrete Fourier transform of the 
   real valued array \constant{a}; \constant{n} defaults to the length of 
   \constant{a}.  It is most efficient for \constant{n} a power of two.  
   The returned array will be one half of the symmetric complex transform of 
   the real array.

\begin{verbatim}
>>> x = cos(arange(30.0)/30.0*2*pi)
>>> print numarray.fft.real_fft(x)
[ -5.82867088e-16 +0.00000000e+00j   1.50000000e+01 -3.08862614e-15j
   7.13643755e-16 -1.04457106e-15j   1.13047653e-15 -3.23843935e-15j
  -1.52158521e-15 +1.14787259e-15j   3.60822483e-16 +3.60555504e-16j
   1.34237661e-15 +2.05127011e-15j   1.98981960e-16 -1.02472357e-15j
   1.55899290e-15 -9.94619821e-16j  -1.05417678e-15 -2.33364171e-17j
  -2.08166817e-16 +1.00955541e-15j  -1.34094426e-15 +8.88633386e-16j
   5.67513742e-16 -2.24823896e-15j   2.13735778e-15 -5.68448962e-16j
  -9.55398954e-16 +7.76890265e-16j  -1.05471187e-15 +0.00000000e+00j]
\end{verbatim}
\end{funcdesc}
       
\begin{funcdesc}{inverse_real_fft}{a, n=None, axis=-1}
   Will return the inverse FFT of the real valued array \constant{a}.
\end{funcdesc}
       
\begin{funcdesc}{fft2d}{a, s=None, axes=(-2,-1)}
   Will return the 2-dimensional FFT of the array \constant{a}.  This
   is really just \function{fft_nd()} with different default behavior.
\end{funcdesc}
       
\begin{funcdesc}{inverse_fft2d}{a, s=None, axes=(-2,-1)}
  The inverse of \function{fft2d()}. This is really just
  \function{inverse_fftnd()} with different default behavior.
\end{funcdesc}
       
\begin{funcdesc}{real_fft2d}{a, s=None, axes=(-2,-1)}
   Will return the 2-D FFT of the real valued array \constant{a}.
\end{funcdesc}
            
\begin{funcdesc}{inverse_real_fft2d}{a, s=None, axes=(-2,-1)}
  The inverse of \function{real_fft2d()}. This is really just
  \function{inverse_real_fftnd()} with different default behavior.
\end{funcdesc}

            
\section{fftpack Python Interface}
\label{sec:FFT:c-api}

%begin{latexonly}
\makeatletter \py@reset \makeatother
%end{latexonly}
\declaremodule{extension}{numarray.fft.fftpack}
\moduleauthor{The numarray team}{numpy-discussion@lists.sourceforge.net}
\modulesynopsis{Fast Fourier Transform}

The interface to the FFTPACK library is performed via the \module{fftpack}
module, which is responsible for making sure that the arrays sent to the
FFTPACK routines are in the right format (contiguous memory locations, right
numerical storage format, etc). It provides interfaces to the following FFTPACK
routines, which are also the names of the Python functions:
\begin{funcdesc}{cffti}{i}
\end{funcdesc}
\begin{funcdesc}{cfftf}{data, savearea}
\end{funcdesc}
\begin{funcdesc}{cfftb}{data, savearea}
\end{funcdesc}
\begin{funcdesc}{rffti}{i}
\end{funcdesc}
\begin{funcdesc}{rfftf}{data, savearea}
\end{funcdesc}
\begin{funcdesc}{rfftb}{data, savearea}
\end{funcdesc}
The routines which start with \texttt{c} expect arrays of complex numbers, the
routines which start with \texttt{r} expect real numbers only. The routines
which end with \texttt{i} are the initalization functions, those which end with
\texttt{f} perform the forward FFTs and those which end with \texttt{b} perform
the backwards FFTs.

The initialization functions require a single integer argument corresponding to
the size of the dataset, and returns a work array. The forward and backwards
FFTs require two array arguments -- the first is the data array, the second is
the work array returned by the initialization function. They return arrays
corresponding to the coefficients of the FFT, with the first element in the
returned array corresponding to the DC component, the second one to the first
fundamental, etc.The length of the returned array is 1 + half the length of the
input array in the case of real FFTs, and the same size as the input array in
the case of complex data.
\begin{verbatim}
>>> import numarray.fft.fftpack as fftpack
>>> x = cos(arange(30.0)/30.0*2*pi)
>>> w = fftpack.rffti(30)
>>> f = fftpack.rfftf(x, w)
>>> print f[0:5]
[ -5.68989300e-16 +0.00000000e+00j   1.50000000e+01 -3.08862614e-15j
        6.86516117e-16 -1.00588467e-15j   1.12688689e-15 -3.19983494e-15j
       -1.52158521e-15 +1.14787259e-15j]
\end{verbatim}



%% Local Variables:
%% mode: LaTeX
%% mode: auto-fill
%% fill-column: 79
%% indent-tabs-mode: nil
%% ispell-dictionary: "american"
%% reftex-fref-is-default: nil
%% TeX-auto-save: t
%% TeX-command-default: "pdfeLaTeX"
%% TeX-master: "numarray"
%% TeX-parse-self: t
%% End:

\chapter{Linear Algebra}
\label{cha:linear-algebra}

%begin{latexonly}
\makeatletter
\py@reset
\makeatother
%end{latexonly}
\declaremodule[numarray.linearalgebra]{extension}{numarray.linear_algebra}
\moduleauthor{The numarray team}{numpy-discussion@lists.sourceforge.net}
\modulesynopsis{Linear Algebra}

\begin{quote}
  The numarray.linear\_algebra module provides a simple interface to some
  commonly used linear algebra routines.
\end{quote}

The \module{numarray.linear_algebra} module provides a simple high-level
interface to some common linear algebra problems. It uses either the LAPACK
Fortran library or the compatible
\module{\mbox{numarray.linear_algebra.lapack_lite}} C library shipped with
\module{numarray}.

\section{Installation}
\label{sec:LA:installation}

The default installation uses the provided
\module{numarray.linear_algebra.lapack_lite} implementation of these routines
and this works without any further interaction.

Nevertheless if LAPACK is installed already or you are concerned about the
performance of these routines you should consider installing
\module{numarray.linear_algebra} to take advantage of the real LAPACK library.
See the next section for instructions.

\subsection{Installation using LAPACK}
\label{sec:LA:install-lapack}

On some platforms, precompiled optimized versions of the LAPACK and BLAS
libraries are preinstalled on the operating system, and the setup procedure
needs to be modified to force the \module{lapack_lite} module to be linked
against those rather than the builtin replacement functions.

Here's a recipe for building using LAPACK:

\begin{verbatim}
% setenv USE_LAPACK 1
% setenv LINALG_LIB <where your lapack, blas, atlas, etc are>
% setenv LINALG_INCLUDE <where your lapack, blas, atlas headers are>
% python setup.py install --selftest
\end{verbatim}

For your particular system and library installations, you may need to edit
\texttt{addons.py} and adjust the variables \texttt{sourcelist},
\texttt{lapack_dirs}, and \texttt{lapack_libs}.

\note{A frequent request is that somehow the maintainers of Numerical Python
   invent a procedure which will automatically find and use the \emph{best}
   available versions of these libraries.  We welcome any patches that provide
   the functionality in a simple, platform independent, and reliable way.  The
   \ulink{scipy}{http://www.scipy.org} project has done some work to provide
   such functionality, but is probably not mature enough for use by
   \module{numarray} yet.}


\section{Python Interface}
\label{sec:LA:python-interface}

All examples in this section assume that you performed a
\begin{verbatim}
from numarray import *
import numarray.linear_algebra as la
\end{verbatim}

\begin{funcdesc}{cholesky_decomposition}{a}
   This function returns a lower triangular matrix L which, when multiplied by
   its transpose yields the original matrix \code{a}; \code{a} must be 
   square, Hermitian, and positive definite. L is often referred to as the 
   Cholesky lower-triangular square-root of \code{a}.
\end{funcdesc}
 
\begin{funcdesc}{determinant}{a}
   This function returns the determinant of the square matrix \code{a}.
\begin{verbatim}
>>> print a
[[ 1  2]
 [ 3 15]]
>>> print la.determinant(a)
9.0
\end{verbatim}
\end{funcdesc}
 
\begin{funcdesc}{eigenvalues}{a}
   This function returns the eigenvalues of the square matrix \code{a}.
\begin{verbatim}
>>> print a
[[ 1.    0.    0.    0.  ]
 [ 0.    2.    0.    0.01]
 [ 0.    0.    5.    0.  ]
 [ 0.    0.01  0.    2.5 ]]
>>> print la.eigenvalues(a)
[ 2.50019992  1.99980008  1.          5.        ]
\end{verbatim}
\end{funcdesc}
 
\begin{funcdesc}{eigenvectors}{a}
   This function returns both the eigenvalues and the eigenvectors, the latter
   as a two-dimensional array (i.e. a sequence of vectors).
\begin{verbatim}
>>> print a
[[ 1.    0.    0.    0.  ]
 [ 0.    2.    0.    0.01]
 [ 0.    0.    5.    0.  ]
 [ 0.    0.01  0.    2.5 ]]
>>> eval, evec = la.eigenvectors(a)
>>> print eval  # same as eigenvalues()
[ 2.50019992  1.99980008  1.          5.        ]
>>> print transpose(evec)
[[ 0.          0.          1.          0.        ]
 [ 0.01998801  0.99980022  0.          0.        ]
 [ 0.          0.          0.          1.        ]
 [ 0.99980022 -0.01998801  0.          0.        ]]
\end{verbatim}
\end{funcdesc}
 
\begin{funcdesc}{generalized_inverse}{a, rcond=1e-10}
   This function returns the generalized inverse (also known as pseudo-inverse
   or Moore-Penrose-inverse) of the matrix \code{a}. It has numerous 
   applications related to linear equations and least-squares problems.
\begin{verbatim}
>>> ainv = la.generalized_inverse(a)
>>> print array_str(innerproduct(a,ainv),suppress_small=1,precision=8)
[[ 1.  0.  0.  0.]
 [ 0.  1.  0. -0.]
 [ 0.  0.  1.  0.]
 [ 0. -0.  0.  1.]]
\end{verbatim}
\end{funcdesc}
 
\begin{funcdesc}{Heigenvalues}{a}
   returns the (real positive) eigenvalues of the square, Hermitian positive
   definite matrix a.
\end{funcdesc}
 
\begin{funcdesc}{Heigenvectors}{a}
   returns both the (real positive) eigenvalues and the eigenvectors of a
   square, Hermitian positive definite matrix a. The eigenvectors are returned
   in an (orthornormal) two-dimensional matrix.
\end{funcdesc}

\begin{funcdesc}{inverse}{a}
   This function returns the inverse of the specified matrix a which must be
   square and non-singular. To within floating point precision, it should
   always be true that \code{matrixmultiply(a, inverse(a)) ==
      identity(len(a))}.  To test this claim, one can do e.g.:
\begin{verbatim}
>>> a = reshape(arange(25.0), (5,5)) + identity(5)
>>> print a
[[  1.   1.   2.   3.   4.]
 [  5.   7.   7.   8.   9.]
 [ 10.  11.  13.  13.  14.]
 [ 15.  16.  17.  19.  19.]
 [ 20.  21.  22.  23.  25.]]
>>> inv_a = la.inverse(a)
>>> print inv_a
[[ 0.20634921 -0.52380952 -0.25396825  0.01587302  0.28571429]
 [-0.5026455   0.63492063 -0.22751323 -0.08994709  0.04761905]
 [-0.21164021 -0.20634921  0.7989418  -0.1957672  -0.19047619]
 [ 0.07936508 -0.04761905 -0.17460317  0.6984127  -0.42857143]
 [ 0.37037037  0.11111111 -0.14814815 -0.40740741  0.33333333]]
\end{verbatim}
   Verify the inverse by printing the largest absolute element of
   $a\, a^{-1} - identity(5)$:
\begin{verbatim}
>>> print "Inversion error:", maximum.reduce(fabs(ravel(dot(a, inv_a)-identity(5))))
Inversion error: 8.18789480661e-16
\end{verbatim}
\end{funcdesc}
 
\begin{funcdesc}{linear_least_squares}{a, b, rcond=1e-10}
   This function returns the least-squares solution of an overdetermined system
   of linear equations. An optional third argument indicates the cutoff for the
   range of singular values (defaults to $10^{-10}$). There are four return
   values: the least-squares solution itself, the sum of the squared residuals
   (i.e.  the quantity minimized by the solution), the rank of the matrix a,
   and the singular values of a in descending order.
\begin{verbatim}

\end{verbatim}
\end{funcdesc}
 
\begin{funcdesc}{solve_linear_equations}{a, b}
   This function solves a system of linear equations with a square non-singular
   matrix a and a right-hand-side vector b. Several right-hand-side vectors can
   be treated simultaneously by making b a two-dimensional array (i.e. a
   sequence of vectors). The function inverse(a) calculates the inverse of the
   square non-singular matrix a by calling solve_linear_equations(a, b) with a
   suitable b.
\end{funcdesc}
 
\begin{funcdesc}{singular_value_decomposition}{a, full_matrices=0}
   This function returns three arrays V, S, and WT whose matrix product is the
   original matrix a. V and WT are unitary matrices (rank-2 arrays), whereas S
   is the vector (rank-1 array) of diagonal elements of the singular-value
   matrix. This function is mainly used to check whether (and in what way) a
   matrix is ill-conditioned.
\end{funcdesc}
 



%% Local Variables:
%% mode: LaTeX
%% mode: auto-fill
%% fill-column: 79
%% indent-tabs-mode: nil
%% ispell-dictionary: "american"
%% reftex-fref-is-default: nil
%% TeX-auto-save: t
%% TeX-command-default: "pdfeLaTeX"
%% TeX-master: "numarray"
%% TeX-parse-self: t
%% End:

\chapter{Masked Arrays}
\label{cha:masked-arrays}

%begin{latexonly}
\makeatletter
\py@reset
\makeatother
%end{latexonly}
\declaremodule{extension}{numarray.ma}
\moduleauthor{The numarray team}{numpy-discussion@lists.sourceforge.net}
\modulesynopsis{Masked Arrays}
\index{MaskedArray|see{numarray.ma}}
\index{observations, dealing with missing}

\begin{quote}
   Masked arrays are arrays that may have missing or invalid entries. Module
   \module{numarray.ma} provides a nearly work-alike replacement for numarray
   that supports data arrays with masks.
\end{quote}

\section{What is a masked array?}
\label{sec:numarray.ma:what-is-a-masked-array}

Masked arrays are arrays that may have missing or invalid entries. Module
\module{numarray.ma} provides a work-alike replacement for \module{\numarray}
that supports data arrays with masks. A mask is either None or an array of ones
and zeros, that determines for each element of the masked array whether or not
it contains an invalid entry.  The package assures that invalid entries are not
used in computations.  A particular element is said to be masked
(\index{numarray.ma!invalid}invalid) if the mask is not None and the
corresponding element of the mask is 1; otherwise it is unmasked
(\index{numarray.ma!valid}valid).

This package was written by \index{Dubois, Paul F.}Paul F.\ Dubois at Lawrence
Livermore National Laboratory. Please see the legal notice in the software and
section \ref{sec:legal-notice} ``License and disclaimer for packages
numarray.ma''. 

\section{Using numarray.ma}
\label{sec:numarray.ma:using}

Use numarray.ma as a replacement for numarray:
\begin{verbatim}
from numarray.ma import *
>>> x = array([1, 2, 3])
\end{verbatim}
To create an array with the second element invalid, we would do:
\begin{verbatim}
>>> y = array([1, 2, 3], mask = [0, 1, 0])
\end{verbatim}
To create a masked array where all values ``near'' 1.e20 are invalid, we can
do:
\begin{verbatim}
>>> z = masked_values([1.0, 1.e20, 3.0, 4.0], 1.e20)
\end{verbatim}
For a complete discussion of creation methods for masked arrays please see
section \ref{sec:numarray.ma:constructing-mask-arrays} ``Constructing masked
arrays''.

The \module{\numarray} module is an attribute in \module{numarray.ma}, so to
execute a method \method{foo} from numarray, you can reference it as
\method{numarray.foo}.

Usually people use both numarray.ma and numarray this way, but of course you can
always fully-qualify the names:
\begin{verbatim}
>>> import numarray.ma
>>> x = numarray.ma.array([1, 2, 3])
\end{verbatim}

The principal feature of module \module{numarray.ma} is class
\class{MaskedArray}, the class whose instances are returned by the array
constructors and most functions in module \module{numarray.ma}. We will discuss
this class first, and later cover the attributes and functions in module
\module{numarray.ma}. For now suffice it to say that among the attributes of
the module are the constants from module \module{\numarray} including those for
declaring typecodes, \constant{NewAxis}, and the mathematical constants such as
\constant{pi} and \constant{e}.  An additional typecode, \class{MaskType}, is
the typecode used for masks.


\section{Class MaskedArray}
\label{sec:numarray.ma:class-maskedarray}
\index{numarray.ma!MaskedArray@\class{MaskedArray}}

In Module \module{numarray.ma}, an array is an instance of class
\class{MaskedArray}, which is defined in the module \module{numarray.ma}. An
instance of class \class{MaskedArray} can be thought of as containing the
following parts:
\begin{itemize}
\item An array of data, of any shape;
\item A mask of ones and zeros of the same shape as the data where a one value
  (true) indicates that the element is masked and the corresponding data is
  invalid.
\item A ``fill value'' --- this is a value that may be used to replace the
   invalid entries in order to return a plain \module{\numarray} array. The
   chief method that does this is the method \method{filled} discussed below.
\end{itemize}
We will use the terms ``invalid value'' and ``invalid entry'' to refer to the
data value at a place corresponding to a mask value of 1. It should be
emphasized that the invalid values are \emph{never} used in any computation,
and that the fill value is not used for \emph{any} computational purpose. When
an instance \var{x} of class \class{MaskedArray} is converted to its string
representation, it is the result returned by \code{filled(x)} that is converted
to a string.


\subsection{Attributes of masked arrays}
\label{sec:numarray.ma:attr-mask-arrays}

\begin{memberdesc}[MaskedArray]{flat}
   (deprecated) \remark{why deprecated in numarray?}
   Returns the masked array as a one-dimensional one. This is
   provided for compatibility with \module{\numarray}. \method{ravel} is
   preferred.  \member{flat} can be assigned to: \samp{x.flat = value} will
   change the values of \var{x}.
\end{memberdesc}

\begin{memberdesc}[MaskedArray]{real}
   Returns the real part of the array if complex. It can be assigned to:
   \samp{x.real = value} will change the real parts of \var{x}.
\end{memberdesc}

\begin{memberdesc}[MaskedArray]{imaginary}
   Returns the imaginary part of the array if complex. It can be assigned to:
   \samp{x.imaginary = value} will change the imaginary parts of x.
\end{memberdesc}

\begin{memberdesc}[MaskedArray]{shape}
   The shape of a masked array can be accessed or changed by using the special
   attribute \member{shape}, as with \module{\numarray} arrays. It can be
   assigned to: \samp{x.shape = newshape} will change the shape of \var{x}. The
   new shape has to describe the same total number of elements.
   \remark{Correct?}
\end{memberdesc}

\begin{memberdesc}[MaskedArray]{shared_data}
   This read-only flag if true indicates that the masked array shared a
   reference with the original data used to construct it at the time of
   construction. Changes to the original array will affect the masked array.
   (This is not the default behavior; see ``Copying or not''.) This flag is
   informational only.
\end{memberdesc}

\begin{memberdesc}[MaskedArray]{shared_mask}
   This read-only flag if true indicates that the masked array \emph{currently}
   shares a reference to the mask used to create it. Unlike
   \member{shared_data}, this flag may change as the result of modifying the
   array contents, as the mask uses copy on write semantics if it is shared.
\end{memberdesc}



\subsection{Methods on masked arrays}
\label{sec:numarray.ma:meth-mask-arrays}

\begin{methoddesc}[MaskedArray]{__array__}
   A special method allows conversion to a \module{\numarray} array if no
   element is actually masked. If there is a masked element, an
   \exception{numarray.maError} exception is thrown. Many \module{\numarray}
   functions, such as \function{numarray.sqrt}, will attempt this conversion on
   their arguments. See also module function \function{filled} in section
   \ref{sec:numarray.ma:meth-mask-arrays}.
\begin{verbatim}
yn = numarray.array(x)
\end{verbatim}
\end{methoddesc}

\begin{methoddesc}[MaskedArray]{astype}{type}
   Return \var{self} as array of given \var{type}. 
\begin{verbatim}
y = x.astype(Float32)
\end{verbatim}
\end{methoddesc}

\begin{methoddesc}[MaskedArray]{byte_swapped}{}
   Returns the raw data \class{\numarray} byte-swapped; included for
   consistency with \module{\numarray} but probably meaningless. 
\begin{verbatim}
y = x.byte_swapped()
\end{verbatim}
\end{methoddesc}

\begin{methoddesc}[MaskedArray]{compressed}{}
   Return an array of the valid elements. Result is one-dimensional.  
\begin{verbatim}
y = x.compressed()
\end{verbatim}
\end{methoddesc}

\begin{methoddesc}[MaskedArray]{count}{axis=None}
   If \var{axis} is \constant{None} return the count of non-masked elements in
   the whole array.  Otherwise return an array of such counts along the axis
   given.
\begin{verbatim}
n = x.count()
y = x.count(0)
\end{verbatim}
\end{methoddesc}

\begin{methoddesc}[MaskedArray]{fill_value}{}
   Get the current fill value. 
\begin{verbatim}
v = x.fill_value()
\end{verbatim}
\end{methoddesc}

\begin{methoddesc}[MaskedArray]{filled}{fill_value=None}
   Returns a \module{\numarray} array with the masked values replaced by the
   fill value.  See also the description of module function filled in section
   \ref{sec:numarray.ma:meth-mask-arrays}.
\begin{verbatim}
yn = x.filled()
\end{verbatim}
\end{methoddesc}

\begin{methoddesc}[MaskedArray]{ids}{}
   Return the ids of the data and mask areas. 
\begin{verbatim}
id1, id2 = x.ids()
\end{verbatim}
\end{methoddesc}

\begin{methoddesc}[MaskedArray]{iscontiguous}{}
   Is the data area contiguous? See \method{numarray.scontiguous} in section
   \ref{arraymethod:iscontiguous}.
\begin{verbatim}
if x.iscontiguous():
\end{verbatim}
\end{methoddesc}

\begin{methoddesc}[MaskedArray]{itemsize}{}
   Size of individual data items in bytes. \samp{n = x.itemsize()}
\end{methoddesc}

\begin{methoddesc}[MaskedArray]{mask}{}
   Return the data mask, or \constant{None}. 
\begin{verbatim}
m = x.mask()
\end{verbatim}
\end{methoddesc}

\begin{methoddesc}[MaskedArray]{put}{values}
   Set the value at each non-masked entry to the corresponding entry in
   \var{values}. The mask is unchanged. See also module function
   \function{put}. 
\begin{verbatim}
x.put(values)
\end{verbatim}
\end{methoddesc}

\begin{methoddesc}[MaskedArray]{putmask}{values}
   Eliminate any masked values by setting the value at each masked entry to the
   corresponding entry in \var{values}. Set the mask to \constant{None}.
\begin{verbatim}
x.putmask(values)
assert getmask(x) is None
\end{verbatim}
\end{methoddesc}

\begin{methoddesc}[MaskedArray]{raw_data}{}
   A reference to the non-filled data; portions may be meaningless. Expert use
   only. 
\begin{verbatim}
d = x.raw_data ()
\end{verbatim}
\end{methoddesc}

\begin{methoddesc}[MaskedArray]{savespace}{v}
   Set the spacesaver attribute to \var{v}. 
\begin{verbatim}
x.savespace (1)
\end{verbatim}
\end{methoddesc}

\begin{methoddesc}[MaskedArray]{set_fill_value}{v}
   Set the fill value to \var{v}. Omit v to restore default.
   \samp{x.set_fill_value(1.e21)} \remark{Give correct default value for v.}
\end{methoddesc}

\begin{methoddesc}[MaskedArray]{set_shape}{args...}
   Set the shape. 
\begin{verbatim}
x.set_shape (3, 12)
\end{verbatim}
\end{methoddesc}

\begin{methoddesc}[MaskedArray]{size}{axis}
   Number of elements in array, or along a particular \var{axis}. 
\begin{verbatim}
totalsize = x.size ()
col_len = x.size (1)
\end{verbatim}
\end{methoddesc}

\begin{methoddesc}[MaskedArray]{spacesaver}{}
   Query the spacesave flag.
\begin{verbatim}
flag = x.spacesaver()
\end{verbatim}
\end{methoddesc}

\begin{methoddesc}[MaskedArray]{tolist}{fill_value=None}
   Return the Python \class{list} \code{self.filled(fill_value).tolist()}; note
   that masked values are filled. 
\begin{verbatim}
alist=x.tolist()
\end{verbatim}
\end{methoddesc}

\begin{methoddesc}[MaskedArray]{tostring}{fill_value=None}
   Return the string \code{self.filled(fill_value).tostring()s = x.tostring()}
\end{methoddesc}

\begin{methoddesc}[MaskedArray]{typecode}{}
   Return the type of the data. See module \module{Precision}, section \ref{TBD}.
\begin{verbatim}
z = x.typecode()
\end{verbatim}
\end{methoddesc}

\begin{methoddesc}[MaskedArray]{unmask}{}
   Replaces the mask by \constant{None} if possible. Subsequent operations may
   be faster if the array previously had an all-zero mask.
\begin{verbatim}
x.unmask()
\end{verbatim}
\end{methoddesc}

\begin{methoddesc}[MaskedArray]{unshare_mask}{}
   If shared_mask is currently true, replaces the reference to it with a
   copy. 
\begin{verbatim}
x.unshare_mask()
\end{verbatim}
\end{methoddesc}


\subsection{Constructing masked arrays}
\label{sec:numarray.ma:constructing-mask-arrays}

\index{numarray.ma!constructor}
\begin{methoddesc}[MaskedArray]{array}
   {data, type=None, copy=1, savespace=0, mask=None, fill_value=None}
   Creates a masked array with the given \var{data} and
   \var{mask}.  The name \class{array} is simply an alias for the class name,
   \class{MaskedArray}.  The fill value is set to \var{fill_value}, and the
   \var{savespace} flag is applied. If \var{data} is a \class{MaskedArray}, its
   \constant{mask}, \constant{typecode}, \constant{spacesaver} flag, and
   \constant{fill_value} will be used unless specifically overridden by one of
   the remaining arguments. In particular, if \var{d} is a masked array,
   \code{array(d, copy=0)} is \var{d}.
\end{methoddesc}

\index{numarray.ma!constructor}
\begin{methoddesc}[MaskedArray]{masked_array}{data, mask=None, fill_value=None}
   This is an easier-to-use version of \method{array},
   for the common case of \code{typecode = None}, \code{copy = 0}. When
   \var{data} is newly-created this function can be used to make it a masked
   array without copying the data if \var{data} is already a \module{\numarray}
   array.
\end{methoddesc}

\index{numarray.ma!constructor}
\begin{methoddesc}[MaskedArray]{masked_values}{data, value, rtol=1.e-5, atol=1.e-8, type=None, copy=1, savespace=0)}
   Constructs a masked array whose mask is set at those places where 
   \begin{equation}
      \abs(\var{data} - \var{value}) < \var{atol} + \var{rtol} * \abs(\var{data})
   \end{equation}
   That is a careful way of saying that those elements of the \var{data} that
   have a value of \var{value} (to within a tolerance) are to be treated as
   invalid.  If data is not of a floating point type, calls
   \method{masked_object} instead.
\end{methoddesc}

\index{numarray.ma!constructor}
\begin{methoddesc}[MaskedArray]{masked_object}{data, value, copy=1, savespace=0}
   Creates a masked array with those entries marked invalid that are equal to
   \var{value}. Again, \var{copy} and \var{/savespace} are passed on to the
   \module{\numarray} array constructor.
\end{methoddesc}

\index{numarray.ma!constructor}
\begin{methoddesc}[MaskedArray]{asarray}{data, type=None}
   This is the same as \code{array(data, typecode, copy=0)}. It is a short way
   of ensuring that something is an instance of \class{MaskedArray} of a given
   \var{type} before proceeding, as in \samp{data = asarray(data)}.
   
   If \var{data} already is a masked array and \var{type} is \constant{None}
   then the return value is \var{data}; nothing is copied in that case.
\end{methoddesc}

\index{numarray.ma!constructor}
\begin{methoddesc}[MaskedArray]{masked_where}{condition, data, copy=1)}
   Creates a masked array whose shape is that of \var{condition}, whose values
   are those of \var{data}, and which is masked where elements of
   \var{condition} are true.
\end{methoddesc}

\index{numarray.ma!constructor}
\begin{datadesc}{masked}
   This is a module constant that represents a scalar masked value. For
   example, if \var{x} is a masked array and a particular location such as
   \code{x[1]} is masked, the quantity \code{x[1]} will be this special
   constant. This special element is discussed more fully in section
   \ref{sec:numarray.ma:constant-masked} ``The constant \constant{masked}''.
\end{datadesc}


The following additional constructors are provided for convenience.

\index{numarray.ma!constructor}
\begin{methoddesc}[MaskedArray]{masked_equal}{data, value, copy=1}
\end{methoddesc} \index{numarray.ma!constructor}
\begin{methoddesc}[MaskedArray]{masked_greater}{data, value, copy=1}
\end{methoddesc} \index{numarray.ma!constructor}
\begin{methoddesc}[MaskedArray]{masked_greater_equal}{data, value, copy=1}
\end{methoddesc} \index{numarray.ma!constructor}
\begin{methoddesc}[MaskedArray]{masked_less}{data, value, copy=1}
\end{methoddesc} \index{numarray.ma!constructor}
\begin{methoddesc}[MaskedArray]{masked_less_equal}{data, value, copy=1}
\end{methoddesc} \index{numarray.ma!constructor}
\begin{methoddesc}[MaskedArray]{masked_not_equal}{data, value, copy=1}
   \method{masked_greater} is equivalent to \code{masked_where(greater(data,
      value), data))}.  Similarly, \method{masked_greater_equal},
   \method{masked_equal}, \method{masked_not_equal}, \method{masked_less},
   \method{masked_less_equal} are called in the same way with the obvious
   meanings.  Note that for floating point data, \method{masked_values} is
   preferable to \method{masked_equal} in most cases.  \remark{because...}
\end{methoddesc}

\index{numarray.ma!constructor}
\begin{methoddesc}[MaskedArray]{masked_inside}{data, v1, v2, copy=1}
   Creates an array with values in the closed interval \code{[v1, v2]} masked.
   \var{v1} and \var{v2} may be in either order.
\end{methoddesc}

\index{numarray.ma!constructor}
\begin{methoddesc}[MaskedArray]{masked_outside}{data, v1, v2, copy=1}
   Creates an array with values outside the closed interval \code{[v1, v2]}
   masked.  \var{v1} and \var{v2} may be in either order.
\end{methoddesc}

On entry to any of these constructors, \var{data} must be any object which the
\module{\numarray} package can accept to create an array (with the desired
\var{type}, if specified). The \var{mask}, if given, must be \constant{None} or
any object that can be turned into a \module{\numarray} array of integer type
(it will be converted to type \class{MaskType}, if necessary), have the same
shape as \var{data}, and contain only values of 0 or 1.

If the \var{mask} is not \constant{None} but its shape does not match that of
\var{data}, an exception will be thrown, unless one of the two is of length 1,
in which case the scalar will be resized (using \method{numarray.resize}) to
match the other.

See section \ref{sec:numarray.ma:copying-or-not} ``Copying or not'' for a
discussion of whether or not the resulting array shares its data or its mask
with the arguments given to these constructors.


\paragraph*{Important Tip} \method{filled} is very important. It converts its
argument to a plain \module{\numarray} array.

\begin{funcdesc}{filled}{x, value=None}
   Returns \var{x} with any invalid locations replaced by a fill \var{value}.
   \function{filled} is guaranteed to return a plain \module{\numarray} array.
   The argument \var{x} does not have to be a masked array or even an array,
   just something that \module{\numarray}/\module{numarray.ma} can turn into
   one.
   \begin{itemize}
   \item If \var{x} is not a masked array, and not a \module{\numarray} array,
      \code{numarray.array(x)} is returned.
   \item If \var{x} is a contiguous \module{\numarray} array then \var{x} is
      returned. (A \module{\numarray} array is contiguous if its data storage
      region is layed out in column-major order; \module{\numarray} allows
      non-contiguous arrays to exist but they are not allowed in certain
      operations).
   \item If \var{x} is a masked array, but the mask is \constant{None}, and
      \var{x}'s data array is contiguous, then it is returned. If the data
      array is not contiguous, a (contiguous) copy of it is returned.
   \item If \var{x} is a masked array with an actual mask, then an array formed
      by replacing the invalid entries with \var{value}, or
      \code{fill_value(x)} if \var{value} is \constant{None}, is returned. If
      the fill value used is of a different type or precision than \var{x}, the
      result may be of a different type or precision than \var{x}.
\end{itemize}
Note that a new array is created only if necessary to create a correctly
filled, contiguous, \module{\numarray} array.

The function \method{filled} plays a central role in our design. It is the
``exit'' back to \module{\numarray}, and is used whenever the invalid values
must be replaced before an operation. For example, adding two masked arrays
\var{a} and \var{b} is roughly:
\begin{verbatim}
masked_array(filled(a, 0) + filled(b, 0), mask_or(getmask(a), getmask(b))
\end{verbatim}
That is, fill the invalid entries of \var{a} and \var{b} with zeros, add them
up, and declare any entry of the result invalid if either \var{a} or \var{b}
was invalid at that spot. The functions \function{getmask} and
\function{mask_or} are discussed later.

\function{filled} also can be used to simply be certain that some expression is
a contiguous \module{\numarray} array at little cost. If its argument is a
\module{\numarray} array already, it is returned without copying.

If you are certain that a masked array \var{x} contains a mask that is None or
is all zeros, you can convert it to a numarray array with the
\method{numarray.array(x)} constructor. If you turn out to be wrong, an
\exception{MAError} exception is raised.
\end{funcdesc}

\begin{funcdesc}{fill_value}{x}
\end{funcdesc}
\begin{methoddesc}[MaskedArray]{fill_value}{}
   \code{fill_value(x)} and the method \code{x.fill_value()} on masked arrays,
   return a value suitable for filling \var{x} based on its type.  If \var{x}
   is a masked array, then \var{x.fill_value()} results. The returned value for
   a given type can be changed by assigning to the following names in module
   \module{numarray.ma}. They should be set to scalars or one element arrays.
   \index{numarray.ma!default_real_fill_value@\constant{default_real_fill_value}}
   \index{numarray.ma!default_complex_fill_value@\constant{default_complex_fill_value}}
   \index{numarray.ma!default_character_fill_value@\constant{default_character_fill_value}}
   \index{numarray.ma!default_integer_fill_value@\constant{default_integer_fill_value}}
   \index{numarray.ma!default_object_fill_value@\constant{default_object_fill_value}}
\begin{verbatim}
default_real_fill_value = numarray.array([1.0e20], Float32)
default_complex_fill_value = numarray.array([1.0e20 + 0.0j], Complex32)
default_character_fill_value = masked
default_integer_fill_value = numarray.array([0]).astype(UnsignedInt8)
default_object_fill_value = masked
\end{verbatim}
   The variable \var{masked} is a module variable of \module{numarray.ma} and
   is discussed in section \ref{sec:numarray.ma:constant-masked}. Calling
   \function{filled} with a \var{fill_value} of \constant{masked} sometimes
   produces a useful printed representation of a masked array.  The function
   \function{fill_value} works on any kind of object.
\end{methoddesc}

\index{numarray.ma!set_fill_value@\method{set_fill_value}}\code{set_fill_value(a,
   fill_value)} is the same as \code{a.set_fill_value (fill_value)} if \var{a}
   is a masked array; otherwise it does nothing. Please note that the fill
   value is mostly cosmetic; it is used when it is needed to convert the masked
   array to a plain \module{\numarray} array but not involved in most
   operations. In particular, setting the \member{fill_value} to
   \constant{1.e20} will \emph{not}, repeat not, cause elements of the array
   whose values are currently 1.e20 to be masked. For that sort of behavior use
   the \method{masked_value} constructor.



\subsection{What are masks?}
\label{sec:numarray.ma:what-are-masks}
\index{masks, description of}
\index{masks, savespace attribute}

Masks are either \constant{None} or 1-byte \module{\numarray} arrays of 1's and
0's. To avoid excessive performance penalties, mask arrays are never checked to
be sure that the values are 1's and 0's, and supplying a \var{mask} argument to
a constructor with an illegal mask will have undefined consequences later.

\emph{Masks have the savespace attribute set.}  This attribute, discussed in
part \ref{part:numerical-python}, may have surprising consequences if you
attempt to do any operations on them other than those supplied by this package.
In particular, do not add or multiply a quantity involving a mask. For example,
if \var{m} is a mask consisting of 1080 1 values, \code{sum(m)} is 56, not
1080. Oops.


\subsection{Working with masks}

\begin{funcdesc}{is_mask}{m}
   Returns true if \var{m} is of a type and precision that would be allowed as
   the mask field of a masked array (that is, it is an array of integers with
   \module{\numarray}'s typecode \class{MaskType}, or it is \constant{None}).
   To be a legal mask, \var{m} should contain only zeros or ones, but this is
   not checked.
\end{funcdesc}

\begin{funcdesc}{make_mask}{m, copy=0, flag=0}
   Returns an object whose entries are equal to \var{m} and for which
   \function{is_mask} would return true. If \var{m} is already a mask or
   \constant{None}, it returns \var{m} or a copy of it. Otherwise it will
   attempt to make a mask, so it will accept any sequence of integers for
   \var{m}. If \var{flag} is true, \method{make_mask} returns \constant{None}
   if its return value otherwise would contain no true elements. To make a
   legal mask, \var{m} should contain only zeros or ones, but this is not
   checked.
\end{funcdesc}

\begin{funcdesc}{make_mask_none}{s}
   Returns a mask of all zeros of shape \var{s} (deprecated name:
   \index{numarray.ma!create_mask@\method{create_mask}
      (deprecated)|see{\method{make_mask_none}}}create_mask).
\end{funcdesc}

\begin{funcdesc}{getmask}{x}
   Returns \index{numarray.ma!mask@\method{mask}}\code{x.mask()}, the mask of
   \var{x}, if \var{x} is a masked array, and \constant{None} otherwise.
   \note{\function{getmask} may return \constant{None} if \var{x} is a masked
   array but has a mask of \constant{None}.  (Please see caution above about
   operating on the result).}
\end{funcdesc}

\begin{funcdesc}{getmaskarray}{x}
   Returns \code{x.mask()} if \var{x} is a masked array and has a mask that is
   not \constant{None}; otherwise it returns a zero mask array of the same
   shape as \var{x}.  Unlike \method{getmask}, \method{getmaskarray} always
   returns an \module{\numarray} array of typecode \class{MaskType}. (Please
   see caution above about operating on the result).
\end{funcdesc}

\begin{funcdesc}{mask_or}{m1, m2}
   Returns an object which when used as a mask behaves like the element-wise
   ``logical or'' of \var{m1} and \var{m2}, where \var{m1} and \var{/m2} are
   either masks or \constant{None} (e.g., they are the results of calling
   \method{getmask}). A \constant{None} is treated as everywhere false. If both
   \var{m1} and \var{m2} are \constant{None}, it returns \constant{None}. If
   just one of them is \constant{None}, it returns the other. If \var{m1} and
   \var{m2} refer to the same object, a reference to that object is returned.
\end{funcdesc}


\subsection{Operations}
\label{sec:numarray.ma:operations}

Masked arrays support the operators $+$, $*$, $/$, $-$, $**$, and unary plus
and minus.  The other operand can be another masked array, a scalar, a
\module{\numarray} array, or something \method{numarray.array} can convert to a
\module{\numarray} array. The results are masked arrays.

In addition masked arrays support the in-place operators $+=$, $-=$, $*=$, and
$/=$.  Implementation of in-place operators differs from \module{\numarray}
semantics in being more generous about converting the right-hand side to the
required type: any kind or lesser type accepted via an \method{astype}
conversion.  In-place operators truly operate in-place when the target is not
masked.



\subsection{Copying or not?}
\label{sec:numarray.ma:copying-or-not}

Depending on the arguments results of constructors may or may not contain a
separate copy of the data or mask arguments. The easiest way to think about
this is as follows: the given field, be it data or a mask, is required to be a
\module{\numarray} array, possibly with a given typecode, and a mask's shape
must match that of the data. If the copy argument is zero, and the candidate
array otherwise qualifies, a reference will be made instead of a copy. If for
any reason the data is unsuitable as is, an attempt will be made to make a copy
that is suitable. Should that fail, an exception will be thrown. Thus, a
\code{copy=0} argument is more of a hope than a command.

If the basic array \index{numarray.ma!constructor}constructor is given a masked
array as the first argument, its mask, typecode, spacesaver flag, and fill
value will be used unless specifically specified by one of the remaining
arguments. In particular, if \var{d} is a masked array, \code{array(d, copy=0)}
is \var{d}.

Since the default behavior for masks is to use a reference if possible, rather
than a copy, which produces a sizeable time and space savings, it is especially
important not to modify something you used as a mask argument to a masked array
creation routine, if it was a \module{\numarray} array of typecode
\class{MaskType}.





\subsection{Behaviors}
\label{sec:numarray.ma:behaviors}
\begin{funcdesc}{float}{a}
\end{funcdesc}
\begin{funcdesc}{int}{a}
  The conversion operators \function{float}, and \function{int} are defined
  to operate on masked arrays consisting of a single unmasked element.
  Masked values and multi-element arrays are not convertible.  
\end{funcdesc}
\begin{funcdesc}{repr}{a}
\end{funcdesc}
\begin{funcdesc}{str}{a}
  A masked array defines the conversion operators \function{str} and
  \function{repr} by applying the corresponding operator to the
  \module{\numarray} array \code{filled(a)}.  
\end{funcdesc}


\subsection{Indexing and Slicing}
\label{sec:numarray.ma:indexing-slicing}

Indexing and slicing differ from Numeric: while generally the same, they return
a copy, not a reference, when used in an expression that produces a non-scalar
result. Consider this example:
\begin{verbatim}
from Numeric import *
x = array([1.,2.,3.])
y = x[1:]
y[0] = 9.
print x
\end{verbatim}
This will print \code{[1., 9., 3.]} since \code{x[1:]} returns a reference to a
portion of \var{x}.  Doing the same operation using \module{numarray.ma},
\begin{verbatim}
from numarray.ma import *
x = array([1.,2.,3.])
y = x[1:]
y[0] = 9.
print x
\end{verbatim}
will print \code{[1., 2., 3.]}, while \var{y} will be a separate array whose
present value would be \code{[9., 3.]}. While sentiment on the correct
semantics here is divided amongst the Numeric Python community as a whole, it
is not divided amongst the author's community, on whose behalf this package is
written.


\subsection{Indexing in assignments}
\label{sec:numarray.ma:indexing-assignments}

Using multiple sets of square brackets on the left side of an assignment
statement will not produce the desired result:
\begin{verbatim}
x = array([[1,2],[3,4]])
x[1][1] = 20.                           # Error, does not change x
x[1,1] = 20.                            # Correct, changes x
\end{verbatim}
The reason is that \code{x[1]} is a copy, so changing it changes that copy, not
\var{x}.  Always use just one single square bracket for assignments.


\subsection{Operations that produce a scalar result}
\label{sec:numarray.ma:operations-producing-scalars}

If indexing or another operation on a masked array produces a scalar result,
then a scalar value is returned rather than a one-element masked array. This
raises the issue of what to return if that result is masked. The answer is that
the module constant
\index{numarray.ma!masked@\constant{masked}}\constant{masked} is returned. This
constant is discussed in section \ref{sec:numarray.ma:constant-masked}.  While
this most frequently occurs from indexing, you can also get such a result from
other functions. For example, averaging a 1-D array, all of whom's values are
invalid, would result in \constant{masked}.


\subsection{Assignment to elements and slices}
\label{sec:numarray.ma:assignments-elements-slices}

Assignment of a normal value to a single element or slice of a masked array has
the effect of clearing the mask in those locations. In this way previously
\index{numarray.ma!invalid}invalid elements become
\index{numarray.ma!valid}valid. The value being assigned is filled first, so
that you are guaranteed that all the elements on the left-hand side are now
valid.  \remark{???}

Assignment of \constant{None} to a single element or slice of a masked array
has the effect of setting the mask in those locations, and the locations become
invalid.

Since these operations change the mask, the result afterwards will no longer
share a mask, since masks have copy-on-write semantics.



\section{MaskedArray Attributes}
\label{sec:numarray.ma:attributes}

\begin{datadesc}{e}
\end{datadesc}
\begin{datadesc}{pi}
\end{datadesc}
\begin{datadesc}{NewAxis}
   Constants \constant{e}, \constant{pi}, \constant{NewAxis} from
   \module{\numarray}, and the constants from module \module{Precision} that
   define nice names for the typecodes.
\end{datadesc}

The special variables \index{numarray.ma!masked@\constant{masked}}\constant{masked} and
\index{numarray.ma!masked@\constant{masked}}masked_print_option are discussed in section
\ref{sec:numarray.ma:constant-masked}.

The module \module{\numarray} is an element of \module{numarray.ma}, so after \samp{from
   numarray.ma import *}, you can refer to the functions in \module{\numarray} such as
\constant{numarray.ones}; see part \ref{part:numerical-python} for the
constants available in \module{\numarray}.




\section{MaskedArray Functions}
\label{sec:numarray.ma:functions}

Each of the operations discussed below returns an instance of \module{numarray.ma} class
\index{numarray.ma!MaskedArray@\class{MaskedArray}}\class{MaskedArray}, having performed
the desired operation element-wise.  In most cases the array arguments can be
masked arrays or \module{\numarray} arrays or something that \module{\numarray}
can turn into a \module{\numarray} array, such as a list of real numbers.

In most cases, if \module{\numarray} has a function of the same name, the
behavior of the one in \module{numarray.ma} is the same, except that it ``respects'' the
mask.


\subsection{Unary functions}
\label{sec:numarray.ma:unary-functions}

The result of a unary operation will be masked wherever the original operand
was masked. It may also be masked if the argument is not in the domain of the
function.  The following functions have their standard meaning:
\begin{quote}
   \index{absolute@\function{absolute} (in module numarray.ma)}\function{absolute}, 
   \index{arccos@\function{arccos} (in module numarray.ma)}\function{arccos}, 
   \index{arcsin@\function{arcsin} (in module numarray.ma)}\function{arcsin}, 
   \index{arctan@\function{arctan} (in module numarray.ma)}\function{arctan}, 
   \index{around@\function{around} (in module numarray.ma)}\function{around}, 
   \index{conjugate@\function{conjugate} (in module numarray.ma)}\function{conjugate}, 
   \index{cos@\function{cos} (in module numarray.ma)}\function{cos}, 
   \index{cosh@\function{cosh} (in module numarray.ma)}\function{cosh}, 
   \index{exp@\function{exp} (in module numarray.ma)}\function{exp},
   \index{floor@\function{floor} (in module numarray.ma)}\function{floor},
   \index{log@\function{log} (in module numarray.ma)}\function{log}, 
   \index{log10@\function{log10} (in module numarray.ma)}\function{log10}, 
   \index{negative@\function{negative} (in module numarray.ma)}\function{negative}
   (also as operator \index{- (in module numarray.ma)}\index{numarray.ma!-}-),
   \index{nonzero@\function{nonzero} (in module numarray.ma)}\function{nonzero}, 
   \index{sin@\function{sin} (in module numarray.ma)}\function{sin}, 
   \index{sinh@\function{sinh} (in module numarray.ma)}\function{sinh}, 
   \index{sqrt@\function{sqrt} (in module numarray.ma)}\function{sqrt}, 
   \index{tan@\function{tan} (in module numarray.ma)}\function{tan}, 
   \index{tanh@\function{tanh} (in module numarray.ma)}\function{tanh}.
\end{quote}

\begin{funcdesc}{fabs}{x}
   The absolute value of \var{x} as a \constant{Float32} array.
   \remark{What happens when you pass \constant{Float64} ?}
\end{funcdesc}


\subsection{Binary functions}
\label{sec:numarray.ma:binary-functions}

Binary functions return a result that is masked wherever either of the operands
were masked; it may also be masked where the arguments are not in the domain of
the function.

\begin{quote}
   \index{add@\function{add} (in module numarray.ma)}\function{add}
   (also as operator \index{+}\index{numarray.ma!+}+),
   \index{subtract@\function{subtract} (in module numarray.ma)}\function{subtract}
   \index{- (in module numarray.ma)}\index{numarray.ma!-}(also as operator -),
   \index{multiply@\function{multiply} (in module numarray.ma)}\function{multiply}
   \index{* (in module numarray.ma)}\index{numarray.ma!*}(also as operator *), 
   \index{divide@\function{divide} (in module numarray.ma)}\function{divide}
   \index{/ (in module numarray.ma)}\index{numarray.ma!/}(also as operator / ), 
   \index{power@\function{power} (in module numarray.ma)}\function{power}
   \index{** (in module numarray.ma)}\index{numarray.ma!**}(also as operator **), 
   \index{remainder@\function{remainder} (in module numarray.ma)}\function{remainder},
   \index{fmod@\function{fmod} (in module numarray.ma)}\function{fmod},
   \index{hypot@\function{hypot} (in module numarray.ma)}\function{hypot},
   \index{arctan2@\function{arctan2} (in module numarray.ma)}\function{arctan2},
   \index{bitwise_and@\function{bitwise_and} (in module numarray.ma)}\function{bitwise_and},
   \index{bitwise_or@\function{bitwise_or} (in module numarray.ma)}\function{bitwise_or},
   \index{bitwise_xor@\function{bitwise_xor} (in module numarray.ma)}\function{bitwise_xor}.
\end{quote}



\subsection{Comparison operators}

To compare arrays, use the following binary functions. Each of them returns a
masked array of 1's and 0's.

\begin{quote}
   \index{equal@\function{equal} (in module numarray.ma)}\function{equal},
   \index{greater@\function{greater} (in module numarray.ma)}\function{greater},
   \index{greater_equal@\function{greater_equal} (in module numarray.ma)}\function{greater_equal},
   \index{less@\function{less} (in module numarray.ma)}\function{less},
   \index{less_equal@\function{less_equal} (in module numarray.ma)}\function{less_equal},
   \index{not_equal@\function{not_equal} (in module numarray.ma)}\function{not_equal}.
\end{quote}

Note that as in \module{\numarray}, you can use a scalar for one argument and
an array for the other. \note{See section \ref{TBD} why operators and comparison
   functions are not excatly equivalent.}



\subsection{Logical operators}

Arrays of logical values can be manipulated with:

\begin{quote}
   \index{logical_and@\function{logical_and} (in module numarray.ma)}\function{logical_and},
   \index{logical_not@\function{logical_not} (in module numarray.ma)}\function{logical_not (unary)},
   \index{logical_or@\function{logical_or} (in module numarray.ma)}\function{logical_or},
   \index{logical_xor@\function{logical_xor} (in module numarray.ma)}\function{logical_xor}.
\end{quote}

\begin{funcdesc}{alltrue}{x}
   Returns 1 if all elements of \var{x} are true. Masked elements are treated
   as true.
\end{funcdesc}

\begin{funcdesc}{sometrue}{x}
   Returns 1 if any element of \var{x} is true. Masked elements are treated as
   false.
\end{funcdesc}



\subsection{Special array operators}

\begin{funcdesc}{isarray}{x}
   Return true \var{x} is a masked array.
   \remark{What is about \numarray's?}
\end{funcdesc}

\begin{funcdesc}{rank}{x} 
   The number of dimensions in \var{x}.
\end{funcdesc}

\begin{funcdesc}{shape}{x}
   Returns the shape of \var{x}, a tuple of array extents.
\end{funcdesc}

\begin{funcdesc}{resize}{x, shape}
   Returns a new array with specified \var{shape}.
\end{funcdesc}

\begin{funcdesc}{reshape}{x, shape}
   Returns a copy of \var{x} with the given new \var{shape}.
\end{funcdesc}

\begin{funcdesc}{ravel}{x}
   Returns \var{x} as one-dimensional \class{MaskedArray}.
\end{funcdesc}

\begin{funcdesc}{concatenate}{(a0, ... an), axis=0}
   Concatenates the arrays \code{a0, ... an} along the specified \var{axis}.
\end{funcdesc}

\begin{funcdesc}{repeat}{a, repeats, axis=0}
   Repeat elements \var{i} of \var{a} \code{repeats[i]} times along \var{axis}.
   \var{repeats} is a sequence of length \code{a.shape[axis]} telling how many
   times to repeat each element.
\end{funcdesc}

\begin{funcdesc}{identity}{n}
   Returns the identity matrix of shape \var{n} by \var{n}.
\end{funcdesc}

\begin{funcdesc}{indices}{dimensions, type=None}
   Returns an array representing a grid of indices with row-only and
   column-only variation.
\end{funcdesc}

\begin{funcdesc}{len}{x}
   This is defined to be the length of the first dimension of \var{x}. This
   definition, peculiar from the array point of view, is required by the way
   Python implements slicing. Use \function{size} for the total length of
   \var{x}.
\end{funcdesc}

\begin{funcdesc}{size}{x, axis=None}
   This is the total size of \var{x}, or the length of a particular dimension
   \var{axis} whose index is given. When axis is given the dimension of the
   result is one less than the dimension of \var{x}.
\end{funcdesc}

\begin{funcdesc}{count}{x, axis=None}
   Count the number of (non-masked) elements in the array, or in the array
   along a certain \var{axis}.  When \var{axis} is given the dimension of the
   result is one less than the dimension of \var{x}.
\end{funcdesc}

\begin{funcdesc}{arange}{}
\end{funcdesc}
\begin{funcdesc}{arrayrange}{}
\end{funcdesc}
\begin{funcdesc}{diagonal}{}
\end{funcdesc}
\begin{funcdesc}{fromfunction}{}
\end{funcdesc}
\begin{funcdesc}{ones}{}
\end{funcdesc}
\begin{funcdesc}{zeros}{}
   are the same as in numarray, but return masked arrays.
\end{funcdesc}

\begin{funcdesc}{sum}{}
\end{funcdesc}
\begin{funcdesc}{product}{}
   are called the same way as count; the difference is that the result is the
   sum or product of the unmasked element.
\end{funcdesc}

\begin{funcdesc}{average}{x, axis=0, weights=None, returned=0}
   Compute the average value of the non-masked elements of \var{x} along the
   selected \var{axis}. If \var{weights} is given, it must match the size and
   shape of \var{x}, and the value returned is:
   \begin{equation}
      \text{average} = \frac{\sum{}weights_i\cdot{}x_i}{\sum{}weights_i}
   \end{equation}
   In computing these sums, elements that correspond to those that are masked
   in \var{x} or \var{weights} are ignored. If successful a 2-tuple consisting
   of the average and the sum of the weights is returned.
\end{funcdesc}

\begin{funcdesc}{allclose}{x, y, fill_value=1, rtol=1.e-5, atol=1.e-8}
   Test whether or not arrays \var{x} and \var{y} are equal subject to the
   given relative and absolute tolerances. If \var{fill_value} is 1, masked
   values are considered equal, otherwise they are considered different. The
   formula used for elements where both \var{x} and \var{y} have a valid value
   is:
   \begin{equation}
      |x-y| < \var{atol} + \var{rtol} \cdot{} |y|
   \end{equation}
   This means essentially that both elements are small compared to \var{atol}
   or their difference divided by their value is small compared to \var{rtol}.
\end{funcdesc}

\begin{funcdesc}{allequal}{x, y, fill_value=1}
   This function is similar to \function{allclose}, except that exact equality
   is demanded. \note{Consider the problems of floating-point representations
      when using this function on non-integer numbers arrays.}
\end{funcdesc}

\begin{funcdesc}{take}{a, indices, axis=0}
   Returns a selection of items from \var{a}. See the documentation of
   \function{numarray.take} in section \ref{sec:array-functions:take}.
\end{funcdesc}

\begin{funcdesc}{transpose}{a, axes=None}
   Performs a reordering of the axes depending on the tuple of indices
   \var{axes}; the default is to reverse the order of the axes.
\end{funcdesc}

\begin{funcdesc}{put}{a, indices, values}
   The opposite of \function{take}. The values of the array \var{a} at the
   locations specified in \var{indices} are set to the corresponding value of
   \var{values}.  The array \var{a} must be a contiguous array. The argument
   \var{indices} can be any integer sequence object with values suitable for
   indexing into the flat form of \var{a}.  The argument \var{values} must be
   any sequence of values that can be converted to the typecode of \var{a}.
\begin{verbatim}
>>> x = arange(6)
>>> put(x, [2,4], [20,40])
>>> print x
[ 0  1 20  3 40  5 ]
\end{verbatim}
   Note that the target array \var{a} is not required to be one-dimensional.
   Since it is contiguous and stored in row-major order, the array indices can
   be treated as indexing \var{a}s elements in storage order.
   
   The wrinkle on this for masked arrays is that if the locations being set by
   \function{put} are masked, the mask is cleared in those locations.
\end{funcdesc}

\begin{funcdesc}{choose}{condition, t}
   This function has a result shaped like \var{condition}. \var{t} must be a
   tuple. Each element of the tuple can be an array, a scalar, or the constant
   element \constant{masked} (See section \ref{sec:numarray.ma:constant-masked}). Each
   element of the result is the corresponding element of \code{t[i]} where
   \var{condition} has the value \var{i}. The result is masked where
   \var{condition} is masked or where the selected element is masked or the
   selected element of \var{t} is the constant \constant{masked}.
\end{funcdesc}

\begin{funcdesc}{where}{condition, x, y}
   Returns an array that is \code{filled(x)} where \var{condition} is true,
   \code{filled(y)} where the condition is false. One of \var{x} or \var{y} can
   be the constant element \constant{masked} (See section
   \ref{sec:numarray.ma:constant-masked}). The result is masked where \var{condition} is
   masked, where the element selected from \var{x} or \var{y} is masked, or
   where \var{x} or \var{y} itself is the constant \constant{masked} and it is
   selected.
\end{funcdesc}

\begin{funcdesc}{innerproduct}{a, b}
\end{funcdesc}
\begin{funcdesc}{dot}{a, b}
   These functions work as in \module{\numarray}, but missing values don't
   contribute. The result is always a masked array, possibly of length one,
   because of the possibility that one or more entries in it may be invalid
   since all the data contributing to that entry was invalid.
\end{funcdesc}

\begin{funcdesc}{outerproduct}{a, b}
   Produces a masked array such that \code{result[i, j] = a[i] * b[j]}. The
   result will be masked where \code{a[i]} or \code{b[j]} is masked.
\end{funcdesc}

\begin{funcdesc}{compress}{condition, x, dimension=-1}
   Compresses out only those valid values where \var{condition} is true. Masked
   values in \var{condition} are considered false.
\end{funcdesc}

\begin{funcdesc}{maximum}{x, y=None}
\end{funcdesc}
\begin{funcdesc}{minimum}{x, y=None}
   Compute the maximum (minimum) valid values of \var{x} if \var{y} is
   \constant{None}; with two arguments, they return the element-wise larger or
   smaller of valid values, and mask the result where either \var{x} or \var{y}
   is masked.  If both arguments are scalars a scalar is returned.
\end{funcdesc}

\begin{funcdesc}{sort}{x, axis=-1, value=None}
   Returns the array \var{x} sorted along the given axis, with masked values
   treated as if they have a sort value of \var{value} but locations containing
   \var{value} are masked in the result if \var{x} had a mask to start with.
   \note{Thus if \var{x} contains \var{value} at a non-masked spot, but has
      other spots masked, the result may not be what you want.}
\end{funcdesc}

\begin{funcdesc}{argsort}{x, axis=-1, fill_value=None}
   This function is unusual in that it returns a \module{\numarray} array,
   equal to \code{numarray.argsort(filled(x, fill_value), axis)}; this is an
   array of indices for sorting along a given axis.
\end{funcdesc}



\subsection{Controlling the size of the string representations}
\label{sec:numarray.ma:contr-size-string}


\begin{funcdesc}{get_print_limit}{}
\end{funcdesc}
\begin{funcdesc}{set_print_limit}{n=0}
   These functions are used to limit printing of large arrays; query and set
   the limit for converting arrays using \function{str} or \function{repr}.
   
   If an array is printed that is larger than this, the values are not printed;
   rather you are informed of the type and size of the array. If \var{n} is
   zero, the standard \module{\numarray} conversion functions are used.
   
   When imported, \module{numarray.ma} sets this limit to 300, and the limit is also
   made to apply to standard \module{\numarray} arrays as well.
\end{funcdesc}



\section{Helper classes}
\label{sec:numarray.ma:helper-classes}

\begin{quote}
   This section discusses other classes defined in module numarray.ma.
\end{quote}

\begin{classdesc}{MAError}
   This class inherits from Exception, used to raise exceptions in the
   \module{numarray.ma} module. Other exceptions are possible, such as errors from the
   underlying \module{\numarray} module.
\end{classdesc}


\subsection{The constant masked}
\label{sec:numarray.ma:constant-masked}
\index{numarray.ma!masked@\constant{masked} (constant)}

A constant named \index{numarray.ma!masked@\constant{masked}}\constant{masked} in
\module{numarray.ma} serves several purposes.
\begin{enumerate}
\item When a indexing operation on an \class{MaskedArray} instance returns a
   scalar result, but the location indexed was masked, then \constant{masked}
   is returned. For example, given a one-dimensional array \var{x} such that
   \code{x.mask()[3]} is 1, then \code{x[3]} is \constant{masked}.
\item When \constant{masked} is assigned to elements of an array via indexing
   or slicing, those elements become masked. So after \code{x[3] = masked},
   \code{x[3]} is masked.
\item Some other operations that may return scalar values, such as
   \function{average}, may return \constant{masked} if given only invalid data.
\item To test whether or not a variable is this element, use the \function{is}
   or \function{is not} operator, not \code{==} or \code{!=}.
\item Operations involving the constant \constant{masked} may result in an
   exception.  In operations, \constant{masked} behaves as an integer array of
   shape \code{()} with one masked element. For example, using \code{+} for
   illustration,
   \begin{itemize}
   \item \constant{masked} + \constant{masked} is \constant{masked}.
   \item \constant{masked} + numeric scalar or numeric scalar +
      \constant{masked} is \constant{masked}.
   \item \constant{masked} + array or array + \constant{masked} is a masked
      array with all elements \constant{masked} if array is of a numeric type.
      The same is true if array is a \module{\numarray} array.
   \end{itemize}
\end{enumerate}



\subsection{The constant masked_print_option}
\index{numarray.ma!masked_print_option@\constant{masked_print_option} (constant)}


Another constant, \constant{masked_print_option} controls what happens when
masked arrays and the constant
\index{numarray.ma!masked@\constant{masked}}\constant{masked} are printed:

\begin{methoddesc}[masked_print_option]{display}{} 
   Returns a string that may be used to indicate those elements of an array
   that are masked when the array is converted to a string, as happens with the
   print statement.
\end{methoddesc}

\begin{methoddesc}[masked_print_option]{set_display}{string} 
   This functions can be used to set the string that is used to indicate those
   elements of an array that are masked when the array is converted to a
   string, as happens with the print statement.
\end{methoddesc}

\begin{methoddesc}[masked_print_option]{enable}{flag}
   can be used to enable (\var{flag} = 1, default) the use of the display
   string. If disabled (\var{flag} = 0), the conversion to string becomes
   equivalent to \code{str(self.filled())}.
\end{methoddesc}

\begin{methoddesc}[masked_print_option]{enabled}{}
   Returns the state of the display-enabling flag.
\end{methoddesc}


\paragraph*{Example of masked behavior}
\label{sec:numarray.ma:example-mask-behavior}
\begin{verbatim}
>>> from numarray.ma import *
>>> x=arange(5)
>>> x[3] = masked
>>> print x
[0 ,1 ,2 ,-- ,4 ,]
>>> print repr(x)
array(data = 
 [0,1,2,0,4,],
      mask = 
 [0,0,0,1,0,],
      fill_value=[0,])
>>> print x[3]
--
>>> print x[3] + 1.0
--
>>> print masked + x
[-- ,-- ,-- ,-- ,-- ,]
>>> masked_print_option.enable(0)
>>> print x
[0,1,2,0,4,]
>>> print x + masked
[0,0,0,0,0,]
>>> print filled(x+masked, -99)
[-99,-99,-99,-99,-99,]
\end{verbatim}


\begin{classdesc}{masked_unary_function}{f, fill=0, domain=None}
   Given a \index{unary}unary array function \function{f}, give a function
   which when applied to an argument \var{x} returns \function{f} applied to
   the array \code{filled(x, fill)}, with a mask equal to
   \code{mask_or(getmask(x), domain(x))}.
   
   The argument domain therefore should be a callable object that returns true
   where \var{x} is not in the domain of \function{f}. 
\end{classdesc}

The following domains are also supplied as members of module \module{numarray.ma}:
\begin{classdesc}{domain_check_interval}{a, b)(x}
   Returns true where \code{x < a or y > b}.
\end{classdesc}

\begin{classdesc}{domain_tan}{eps}{x}
   This is true where \code{abs(cos (x)) < eps}, that is, a domain suitable for
   the tangent function.
\end{classdesc}

\begin{classdesc}{domain_greater}{v)(x}
   True where \code{x <= v}.
\end{classdesc}

\begin{classdesc}{domain_greater_equal}{v)(x}
   True where x < v.
\end{classdesc}


\begin{classdesc}{masked_binary_function}{f, fillx=0, filly=0}
   Given a binary array function \function{f}, \code{masked_binary_function(f,
      fillx=0, filly=0)} defines a function whose value at \var{x} is
   \code{f(filled(x, fillx), filled (y, filly))} with a resulting mask of
   \code{mask_or(getmask (x), getmask(y))}. The values \var{fillx} and
   \var{filly} must be chosen so that \code{(fillx, filly)} is in the domain of
   \function{f}.
\end{classdesc}

In addition, an instance of
\index{numarray.ma!masked_binary_function@\class{masked_binary_function}}\class{masked_binary_function}
has two methods defined upon it:

\begin{methoddesc}[masked_binary_function]{reduce}{target, axis = 0}
\end{methoddesc}

\begin{methoddesc}[masked_binary_function]{accumulate}{target, axis = 0}
\end{methoddesc}

\begin{methoddesc}[masked_binary_function]{outer}{a, b}
   These methods perform reduction, accumulation, and applying the function in
   an outer-product-like manner, as discussed in the section
   \ref{sec:ufuncs-have-special-methods}.
\end{methoddesc}


\begin{classdesc}{domained_binary_function}{}
   This class exists to implement division-related operations. It is the same
   as \class{masked_binary_function}, except that a new second argument is a
   domain which is used to mask operations that would otherwise cause failure,
   such as dividing by zero. The functions that are created from this class are
   \function{divide}, \function{remainder} (\function{mod}), and
   \function{fmod}.
\end{classdesc}

The following domains are available for use as the domain argument:

\begin{classdesc}{domain_safe_divide}{)(x, y}
   True where \code{absolute(x)*divide_tolerance > absolute(y)}.  As the
   comments in the code say, \emph{better ideas welcome}. The constant
   \index{numarray.ma!divide_tolerance@\constant{divide_tolerance}}\constant{divide_tolerance}
   is set to \constant{1.e-35} in the source and can be changed by editing its
   value in \file{MA.py} and reinstalling. This domain is used for the divide
   operator.
\end{classdesc}


\section{Examples of Using numarray.ma}
\label{sec:numarray.ma:examples-using-ma}


\subsection{Data with a given value representing missing data}
\label{sec:numarray.ma:data-with-given-repr-miss-data}

Suppose we have read a one-dimensional list of elements named \var{x}. We also
know that if any of the values are \constant{1.e20}, they represent missing
data. We want to compute the average value of the data and the vector of
deviations from average.
\begin{verbatim}
>>> from numarray.ma import *
>>> x = array([0.,1.,2.,3.,4.])
>>> x[2] = 1.e20
>>> y = masked_values (x, 1.e20)
>>> print average(y)
2.0
>>> print y-average(y)
[ -2.00000000e+00, -1.00000000e+00,  --,  1.00000000e+00,
        2.00000000e+00,]
\end{verbatim}


\subsection{Filling in the missing data}
\label{sec:numarray.ma:filling-missing-data}

Suppose now that we wish to print that same data, but with the missing values
replaced by the average value.
\begin{verbatim}
>>> print filled (y, average(y))
\end{verbatim}


\subsection{Numerical operations}
\label{sec:numarray.ma:numerical-operations}

We can do numerical operations without worrying about missing values, dividing
by zero, square roots of negative numbers, etc.
\begin{verbatim}
>>> from numarray.ma import *
>>> x=array([1., -1., 3., 4., 5., 6.], mask=[0,0,0,0,1,0])
>>> y=array([1., 2., 0., 4., 5., 6.], mask=[0,0,0,0,0,1])
>>> print sqrt(x/y)
[  1.00000000e+00,  --,  --,  1.00000000e+00, --,  --,]
\end{verbatim}
Note that four values in the result are invalid: one from a negative square
root, one from a divide by zero, and two more where the two arrays \var{x} and
\var{y} had invalid data. Since the result was of a real type, the print
command printed \code{str(filled(sqrt (x/y)))}.



\subsection{Seeing the mask}
\label{sec:numarray.ma:seeing-mask}

There are various ways to see the mask. One is to print it directly, the other
is to convert to the \function{repr} representation, and a third is get the
mask itself.  Use of \function{getmask} is more robust than \code{x.mask()},
since it will work (returning \constant{None}) if \var{x} is a
\module{\numarray} array or list.
\begin{verbatim}
>>> x = arange(10)
>>> x[3:5] = masked
>>> print x
[0 ,1 ,2 ,-- ,-- ,5 ,6 ,7 ,8 ,9 ,]
>>> print repr(x)
*** Masked array, mask present ***
Data:
[0 ,1 ,2 ,-- ,-- ,5 ,6 ,7 ,8 ,9 ,]
Mask (fill value [0,])
[0,0,0,1,1,0,0,0,0,0,]
>>> print getmask(x)
[0,0,0,1,1,0,0,0,0,0,]
\end{verbatim}



\subsection{Filling it your way}
\label{sec:numarray.ma:filling-it-your-way}

If we want to print the data with \constant{-1}'s where the elements are
masked, we use \function{filled}.
\begin{verbatim}
>>> print filled(z, -1)
[ 1.,-1.,-1., 1.,-1.,-1.,]
\end{verbatim}



\subsection{Ignoring extreme values}
\label{sec:numarray.ma:ignore-extreme-values}

Suppose we have an array \var{d} and we wish to compute the average of the
values in \var{d} but ignore any data outside the range -100. to 100.
\begin{verbatim}
v = masked_outside(d, -100., 100.)
print average(v)
\end{verbatim}


\subsection{Averaging an entire multidimensional array}
\label{sec:numarray.ma:averaging-an-entire}

The problem with averaging over an entire array is that the average function
only reduces one dimension at a time. So to average the entire array,
\function{ravel} it first.
\begin{verbatim}
>>> x
*** Masked array, no mask ***
Data:
[[ 0, 1, 2,]
 [ 3, 4, 5,]
 [ 6, 7, 8,]
 [ 9,10,11,]]
>>> average(x)
*** Masked array, no mask ***
Data:
[ 4.5, 5.5, 6.5,]
>>> average(ravel(x))
5.5
\end{verbatim}




%% Local Variables:
%% mode: LaTeX
%% mode: auto-fill
%% fill-column: 79
%% indent-tabs-mode: nil
%% ispell-dictionary: "american"
%% reftex-fref-is-default: nil
%% TeX-auto-save: t
%% TeX-command-default: "pdfeLaTeX"
%% TeX-master: "numarray"
%% TeX-parse-self: t
%% End:

\chapter{Mlab}
\label{cha:mlab}

%begin{latexonly}
\makeatletter
\py@reset
\makeatother
%end{latexonly}
\declaremodule[numarray.mlab]{extension}{numarray.mlab}
\moduleauthor{The numarray team}{numpy-discussion@lists.sourceforge.net}
\modulesynopsis{mlab}

\section{Matlab(tm) compatible functions}
\label{sec:Matlab-compatible-functions}

\begin{quote}
  \module{numarray.mlab} provides a set of Matlab(tm) compatible functions.
\end{quote}

This will hopefully become a complete set of the basic functions available in
Matlab.  The syntax is kept as close to the Matlab syntax as possible.  One
fundamental change is that the first index in Matlab varies the fastest (as in
FORTRAN).  That means that it will usually perform reductions over columns,
whereas with this object the most natural reductions are over rows.  It's
perfectly possible to make this work the way it does in Matlab if that's
desired.

\begin{funcdesc}{mean}{m, axis=0}
   \label{cha:mlab:mean}
   \label{func:mean}
   returns the mean along the axis'th dimension of m.  Note: if m is an integer
   array, the result will be floating point. This was changed in release 10.1;
   previously, a meaningless integer divide was used.
\end{funcdesc}
\begin{funcdesc}{median}{m}
   \label{cha:mlab:median}
   \label{func:median}
   returns a mean of m along the first dimension of m.
\end{funcdesc}
\begin{funcdesc}{min}{m, axis=0}
   \label{cha:mlab:min}
   \label{func:min}
   returns the minimum along the axis'th dimension of m.
\end{funcdesc}
\begin{funcdesc}{msort}{m}
   \label{cha:mlab:msort}
   \label{func:msort}
   returns a sort along the first dimension of m as in MATLAB.
\end{funcdesc}
\begin{funcdesc}{prod}{m, axis=0}
   \label{cha:mlab:prod}
   \label{func:prod}
   returns the product of the elements along the axis'th dimension of m.
\end{funcdesc}
\begin{funcdesc}{ptp}{m, axis = 0}
   \label{cha:mlab:ptp}
   \label{func:ptp}
   returns the maximum - minimum along the axis'th dimension of m.
\end{funcdesc}
\begin{funcdesc}{rand}{d1, ..., dn}
   \label{cha:mlab:rand}
   \label{func:rand}
   returns a matrix of the given dimensions which is initialized to random numbers from a uniform distribution in
the range [0,1).
\end{funcdesc}
\begin{funcdesc}{rot90}{m,k=1}
   \label{cha:mlab:rot90}
   \label{func:rot90}
   returns the matrix found by rotating m by k*90 degrees in the counterclockwise direction.
\end{funcdesc}
\begin{funcdesc}{sinc}{x}
   \label{cha:mlab:sinc}
   \label{func:sinc}
   returns sin(pi*x)
\end{funcdesc}
\begin{funcdesc}{squeeze}{a}
   \label{cha:mlab:squeeze}
   \label{func:squeeze}
   removes any ones from the shape of a
\end{funcdesc}
\begin{funcdesc}{std}{m, axis = 0}
   \label{cha:mlab:std}
   \label{func:std}
   returns the unbiased estimate of the population standard deviation from a
   sample along the axis'th dimension of m. (That is, the denominator for the
   calculation is n-1, not n.) 
\end{funcdesc}
\newpage
\begin{funcdesc}{sum}{m, axis=0}
   \label{cha:mlab:sum}
   \label{func:sum}
   returns the sum of the elements along the axis'th dimension of m.
\end{funcdesc}
\begin{funcdesc}{svd}{m}
   \label{cha:mlab:svd}
   \label{func:svd}
   return the singular value decomposition of m [u,x,v]
\end{funcdesc}
\begin{funcdesc}{trapz}{y,x=None}
   \label{cha:mlab:trapz}
   \label{func:trapz}
   integrates y = f(x) using the trapezoidal rule
\end{funcdesc}
\begin{funcdesc}{tri}{N, M=N, k=0, typecode=None}
   \label{cha:mlab:tri}
   \label{func:tri}
   returns a N-by-M matrix where all the diagonals starting from lower left corner up to the k-th are all ones.
\end{funcdesc}
\begin{funcdesc}{tril}{m,k=0}
   \label{cha:mlab:tril}
   \label{func:tril}
   returns the elements on and below the k-th diagonal of m. k=0 is the main
   diagonal, \begin{math} k > 0\end{math} is above and \begin{math}k < 0
   \end{math} is below the main diagonal.
\end{funcdesc}

\begin{funcdesc}{triu}{m,k=0}
   \label{sec:mlab-functions:triu}
   \label{func:triu}
   returns the elements on and above the k-th diagonal of m. k=0 is the main
   diagonal, \begin{math}k > 0\end{math} is above and \begin{math}k <
   0\end{math} is below the main diagonal.
\end{funcdesc}

%% Local Variables:
%% mode: LaTeX
%% mode: auto-fill
%% fill-column: 79
%% indent-tabs-mode: nil
%% ispell-dictionary: "american"
%% reftex-fref-is-default: nil
%% TeX-auto-save: t
%% TeX-command-default: "pdfeLaTeX"
%% TeX-master: "numarray"
%% TeX-parse-self: t
%% End:

\chapter{Random Numbers}
\label{cha:random-array}

%begin{latexonly}
\makeatletter \py@reset \makeatother
%end{latexonly}
\declaremodule[numarray.randomarray]{extension}{numarray.random_array}
\moduleauthor{The numarray team}{numpy-discussion@lists.sourceforge.net}
\modulesynopsis{Random Numbers}

\begin{quote}
  The \module{numarray.random_array} module (in conjunction with the
  \module{numarray.random_array.ranlib} submodule) provides a high-level
  interface to ranlib, which provides a good quality C implementation of a
  random-number generator.
\end{quote}

\section{General functions}
\label{sec:RA:general-functions}

\begin{funcdesc}{seed}{x=0, y=0}
The \function{seed} function takes two integers and sets the two seeds of the
random number generator to those values. If the default values of 0 are used
for both \var{x} and \var{y}, then a seed is generated from the current time,
providing a pseudo-random seed.
\end{funcdesc}

\begin{funcdesc}{get_seed}{}
This function returns the two seeds used by the current random-number
generator. It is most often used to find out what seeds the \function{seed}
function chose at the last iteration.  \remark{Are there any thread-safety
issues?}
\end{funcdesc}

\begin{funcdesc}{random}{shape=[]}
   The \function{random} function takes a \var{shape}, and returns an array of
   \class{Float} numbers between 0.0 and 1.0.  Neither 0.0 nor 1.0 is ever
   returned by this function.  The array is filled from the generator following
   the canonical array organization.
   
   If no argument is specified, the function returns a single floating point
   number, not an array.
   
   \note{See discussion of the \member{flat} attribute in section
      \ref{mem:numarray:flat}.}
\end{funcdesc}

\begin{funcdesc}{uniform}{minimum, maximum, shape=[]}
   The \function{uniform} function returns an array of the specified
   \var{shape} and containing \class{Float} random numbers strictly between
   \var{minimum} and \var{maximum}.
   
   The \var{minimum} and \var{maximum} arguments can be arrays. If this is the
   case, and the output \var{shape} is specified, \var{minimum} and
   \var{maximum} are broadcasted if their dimensions are not equal to
   \var{shape}. If \var{shape} is not specified, the shape of the output is
   equal to the shape of \var{minimum} and \var{maximum} after broadcasting.
   
   If no \var{shape} is specified, and \var{minimum} and \var{maximum} are
   scalars, a single value is returned.
\end{funcdesc}

\begin{funcdesc}{randint}{minimum, maximum, shape=[]}
   The \function{randint} function returns an array of the specified
   \var{shape} and containing random (standard) integers greater than or equal
   to \var{minimum} and strictly less than \var{maximum}. 
   
   The \var{minimum} and \var{maximum} arguments can be arrays. If this is the
   case, and the output \var{shape} is specified, \var{minimum} and
   \var{maximum} are broadcasted if their dimensions are not equal to
   \var{shape}. If \var{shape} is not specified, the shape of the output is
   equal to the shape of \var{minimum} and \var{maximum} after broadcasting.
   
   If no \var{shape} is specified, and \var{minimum} and \var{maximum} are
   scalars, a single value is returned.
\end{funcdesc}

\begin{funcdesc}{permutation}{n}
   The \function{permutation} function returns an array of the integers between
   \code{0} and \code{\var{n}-1}, in an array of shape \code{(n,)} with its
   elements randomly permuted.
\end{funcdesc}


\section{Special random number distributions}
\label{sec:RA:special-distribution}



\subsection{Random floating point number distributions}
\label{sec:RA:float-distribution}

\begin{funcdesc}{beta}{a, b, shape=[]}
   The \function{beta} function returns an array of the specified shape that
   contains \class{Float} numbers $\beta$-distributed with $\alpha$-parameter
   \var{a} and $\beta$-parameter \var{b}. 
   
   The \var{a} and \var{b} arguments can be arrays. If this is the case, and
   the output \var{shape} is specified, \var{a} and \var{b} are broadcasted if
   their dimensions are not equal to \var{shape}. If \var{shape} is not
   specified, the shape of the output is equal to the shape of \var{a} and
   \var{b} after broadcasting.
   
   If no \var{shape} is specified, and \var{a} and \var{b} are
   scalars, a single value is returned.
\end{funcdesc}

\begin{funcdesc}{chi_square}{df, shape=[]}
   The \function{chi_square} function returns an array of the specified
   \var{shape} that contains \class{Float} numbers with the
   $\chi^2$-distribution with \var{df} degrees of freedom.
   
   The \var{df} argument can be an array. If this is the case, and the output
   \var{shape} is specified, \var{df} is broadcasted if its dimensions are not
   equal to \var{shape}. If \var{shape} is not specified, the shape of the
   output is equal to the shape of \var{df}.
   
   If no \var{shape} is specified, and \var{df} is a scalar, a single value is
   returned.
\end{funcdesc}

\begin{funcdesc}{exponential}{mean, shape=[]}
   The \function{exponential} function returns an array of the specified
   \var{shape} that contains \class{Float} numbers exponentially distributed
   with the specified \var{mean}. 
   
   The \var{mean} argument can be an array. If this is the case, and the output
   \var{shape} is specified, \var{mean} is broadcasted if its dimensions are
   not equal to \var{shape}. If \var{shape} is not specified, the shape of the
   output is equal to the shape of \var{mean}.
   
   If no \var{shape} is specified, and \var{mean} is a scalar, a single value
   is returned.
\end{funcdesc}

\begin{funcdesc}{F}{dfn, dfd, shape=[]}
  The \function{F} function returns an array of the specified \var{shape} that
  contains \class{Float} numbers with the F-distribution with \var{dfn} degrees
  of freedom in the numerator and \var{dfd} degrees of freedom in the
  denominator.
   
  The \var{dfn} and \var{dfd} arguments can be arrays. If this is the case, and
  the output \var{shape} is specified, \var{dfn} and \var{dfd} are broadcasted
  if their dimensions are not equal to \var{shape}. If \var{shape} is not
  specified, the shape of the output is equal to the shape of \var{dfn} and
  \var{dfd} after broadcasting.
   
  If no \var{shape} is specified, and \var{dfn} and \var{dfd} are scalars, a
  single value is returned.
\end{funcdesc}

\begin{funcdesc}{gamma}{a, r, shape=[]}
   The \function{gamma} function returns an array of the specified \var{shape}
   that contains \class{Float} numbers $\beta$-distributed with location
   parameter \var{a} and distribution shape parameter \var{r}.
   
   The \var{a} and \var{r} arguments can be arrays. If this is the case, and
   the output \var{shape} is specified, \var{a} and \var{r} are broadcasted if
   their dimensions are not equal to \var{shape}. If \var{shape} is not
   specified, the shape of the output is equal to the shape of \var{a} and
   \var{r} after broadcasting.
   
   If no \var{shape} is specified, and \var{a} and \var{r} are scalars, a
   single value is returned.
\end{funcdesc}

\begin{funcdesc}{multivariate_normal}{mean, cov, shape=[]}
   The multivariate_normal function takes a one dimensional array argument
   \var{mean} and a two dimensional array argument \var{cov}. Suppose
   the shape of \var{mean} is \code{(n,)}. Then the shape of \var{cov}
   must be \code{(n,n)}. The function returns an array of \class{Float}s.
   
   The effect of the \var{shape} parameter is:
   \begin{itemize}
   \item If no \var{shape} is specified, then an array with shape \code{(n,)}
      is returned containing a vector of numbers with a multivariate normal
      distribution with the specified mean and covariance.
   \item If \var{shape} is specified, then an array of such vectors is
      returned.  The shape of the output is \code{shape.append((n,))}. The
      leading indices into the output array select a multivariate normal from
      the array. The final index selects one number from within the
      multivariate normal.
 \end{itemize}
 In either case, the behavior of \function{multivariate_normal} is undefined if
 \var{cov} is not symmetric and positive definite.
\end{funcdesc}

\begin{funcdesc}{normal}{mean, std, shape=[]}
   The \function{normal} function returns an array of the specified \var{shape}
   that contains \class{Float} numbers normally distributed with the specified
   \var{mean} and standard deviation \var{std}. 
   
   The \var{mean} and \var{std} arguments can be arrays. If this is the
   case, and the output \var{shape} is specified, \var{mean} and \var{std}
   are broadcasted if their dimensions are not equal to \var{shape}. If
   \var{shape} is not specified, the shape of the output is equal to the shape
   of \var{mean} and \var{std} after broadcasting.
   
   If no \var{shape} is specified, and \var{mean} and \var{std} are scalars, a
   single value is returned.
\end{funcdesc}

\begin{funcdesc}{noncentral_chi_square}{df, nonc, shape=[]}
   The \function{noncentral_chi_square} function returns an array of the
   specified \var{shape} that contains \class{Float} numbers with
   the$\chi^2$-distribution with \var{df} degrees of freedom and noncentrality
   parameter \var{nconc}.
   
   The \var{df} and \var{nonc} arguments can be arrays. If this is the case,
   and the output \var{shape} is specified, \var{df} and \var{nonc} are
   broadcasted if their dimensions are not equal to \var{shape}. If \var{shape}
   is not specified, the shape of the output is equal to the shape of \var{df}
   and \var{nonc} after broadcasting.
   
   If no \var{shape} is specified, and \var{df} and \var{nonc} are scalars, a
   single value is returned.
\end{funcdesc}

\begin{funcdesc}{noncentral_F}{dfn, dfd, nconc, shape=[]}
   The \function{noncentral_F} function returns an array of the specified
   \var{shape} that contains \class{Float} numbers with the F-distribution with
   \var{dfn} degrees of freedom in the numerator, \var{dfd} degrees of freedom
   in the denominator, and noncentrality parameter \var{nconc}.
   
   The \var{dfn}, \var{dfd} and \var{nonc} arguments can be arrays. If this is
   the case, and the output \var{shape} is specified, \var{dfn}, \var{dfd} and
   \var{nonc} are broadcasted if their dimensions are not equal to \var{shape}.
   If \var{shape} is not specified, the shape of the output is equal to the
   shape of \var{dfn}, \var{dfd} and \var{nonc} after broadcasting.
   
   If no \var{shape} is specified, and \var{dfn}, \var{dfd} and \var{nonc} are
   scalars, a single value is returned.
\end{funcdesc}

\begin{funcdesc}{standard_normal}{shape=[]}
   The \function{standard_normal} function returns an array of the specified
   \var{shape} that contains \class{Float} numbers normally (Gaussian)
   distributed with mean zero and variance and standard deviation one. 
   
   If no \var{shape} is specified, a single number is returned.
\end{funcdesc}

\begin{funcdesc}{F}{dfn, dfd, shape=[]}
Returns array of F distributed random numbers with \var{dfn} degrees of freedom
in the numerator and \var{dfd} degrees of freedom in the denominator.
\end{funcdesc}

\begin{funcdesc}{noncentral_F}{dfn, dfd, nconc, shape=[]}
 Returns array of noncentral F distributed random numbers with dfn degrees of
 freedom in the numerator and dfd degrees of freedom in the denominator, and
 noncentrality parameter nconc.
\end{funcdesc}



\subsection{Random integer number distributions }
\label{sec:RA:int-distributions}

\begin{funcdesc}{binomial}{trials, p, shape=[]}
   The \function{binomial} function returns an array with the specified
   \var{shape} that contains \class{Integer} numbers with the binomial
   distribution with \var{trials} and event probability \var{p}. In other
   words, each value in the returned array is the number of times an event with
   probability \var{p} occurred within \var{trials} repeated trials.
   
   The \var{trials} and \var{p} arguments can be arrays. If this is the
   case, and the output \var{shape} is specified, \var{trials} and \var{p}
   are broadcasted if their dimensions are not equal to \var{shape}. If
   \var{shape} is not specified, the shape of the output is equal to the shape
   of \var{trials} and \var{p} after broadcasting.
   
   If no \var{shape} is specified, and \var{trials} and \var{p} are scalars,
   a single value is returned.
\end{funcdesc}

\begin{funcdesc}{negative_binomial}{trials, p, shape=[]}
  The \function{negative_binomial} function returns an array with the specified
  \var{shape} that contains \class{Integer} numbers with the negative binomial
  distribution with \var{trials} and event probability \var{p}.
   
  The \var{trials} and \var{p} arguments can be arrays. If this is the case,
  and the output \var{shape} is specified, \var{trials} and \var{p} are
  broadcasted if their dimensions are not equal to \var{shape}. If \var{shape}
  is not specified, the shape of the output is equal to the shape of
  \var{trials} and \var{p} after broadcasting.
   
   If no \var{shape} is specified, and \var{trials} and \var{p} are scalars,
   a single value is returned.
\end{funcdesc}

\begin{funcdesc}{multinomial}{trials, probs, shape=[]}
   The \function{multinomial} function returns an array with that contains
   integer numbers with the multinomial distribution with \var{trials} and
   event probabilities given in \var{probs}.  \var{probs} must be a one
   dimensional array.  There are \code{len(probs)+1} events. \code{probs[i]} is
   the probability of the i-th event for \code{0<=i<len(probs)}. The
   probability of event \code{len(probs)} is \code{1.-Numeric.sum(prob)}.
   
   The function returns an integer array of shape
   \code{shape~+~(len(probs)+1,)}.  If \var{shape} is not specified this is one
   multinomially distributed vector of shape \code{(len(prob)+1,)}.  Otherwise
   each \code{returnarray[i,j,...,:]} is an integer array of shape
   \code{(len(prob)+1,)} containing one multinomially distributed vector.
\end{funcdesc}

\begin{funcdesc}{poisson}{mean, shape=[]}
   The \function{poisson} function returns an array with the specified shape
   that contains \class{Integer} numbers with the Poisson distribution with the
   specified \var{mean}.
   
   The \var{mean} argument can be an array. If this is the case, and the output
   \var{shape} is specified, \var{mean} is broadcasted if its dimensions are
   not equal to \var{shape}. If \var{shape} is not specified, the shape of the
   output is equal to the shape of \var{mean}.
   
   If no \var{shape} is specified, and \var{mean} is a scalar, a single value
   is returned.
\end{funcdesc}



\section{Examples}
\label{sec:examples}

Some example uses of the \module{numarray.random_array} module. \note{Naturally the exact
   output of running these examples will be different each time!} \remark{Make
   sure these examples are correct!}
\begin{verbatim}
>>> from numarray.random_array import *
>>> seed() # Set seed based on current time
>>> print get_seed() # Find out what seeds were used
(897800491, 192000)
>>> print random()
0.0528018975065
>>> print random((5,2))
[[ 0.14833829 0.99031458]
[ 0.7526806 0.09601787]
[ 0.1895229 0.97674777]
[ 0.46134511 0.25420982]
[ 0.66132009 0.24864472]]
>>> print uniform(-1,1,(10,))
[ 0.72168852 -0.75374185 -0.73590945 0.50488248 -0.74462822 0.09293685
-0.65898308 0.9718067 -0.03252475 0.99611011]
>>> print randint(0,100, (12,))
[28 5 96 19 1 32 69 40 56 69 53 44]
>>> print permutation(10)
[4 2 8 9 1 7 3 6 5 0]
>>> seed(897800491, 192000) # resetting the same seeds
>>> print random() # yields the same numbers
0.0528018975065
\end{verbatim}
Most of the functions in this package take zero or more distribution specific
parameters plus an optional \var{shape} parameter. The \var{shape} parameter
gives the shape of the output array:
\begin{verbatim}
>>> from numarray.random_array import *
>>> print standard_normal()
-0.435568600893
>>> print standard_normal(5)
[-1.36134553 0.78617644 -0.45038718 0.18508556 0.05941355]
>>> print standard_normal((5,2))
[[ 1.33448863 -0.10125473]
[ 0.66838062 0.24691346]
[-0.95092064 0.94168913]
[-0.23919107 1.89288616]
[ 0.87651485 0.96400219]]
>>> print normal(7., 4., (5,2)) #mean=7, std. dev.=4
[[ 2.66997623 11.65832615]
[ 6.73916003 6.58162862]
[ 8.47180378 4.30354905]
[ 1.35531998 -2.80886841]
[ 7.07408469 11.39024973]]
>>> print exponential(10., 5) #mean=10
[ 18.03347754 7.11702306 9.8587961 32.49231603 28.55408891]
>>> print beta(3.1, 9.1, 5) # alpha=3.1, beta=9.1
[ 0.1175056 0.17504358 0.3517828 0.06965593 0.43898219]
>>> print chi_square(7, 5) # 7 degrees of freedom (dfs)
[ 11.99046516 3.00741053 4.72235727 6.17056274 8.50756836]
>>> print noncentral_chi_square(7, 3, 5) # 7 dfs, noncentrality 3
[ 18.28332138 4.07550335 16.0425396 9.51192093 9.80156231]
>>> F(5, 7, 5) # 5 and 7 dfs
array([ 0.24693671, 3.76726145, 0.66883826, 0.59169068, 1.90763224])
>>> noncentral_F(5, 7, 3., 5) # 5 and 7 dfs, noncentrality 3
array([ 1.17992553, 0.7500126 , 0.77389943, 9.26798989, 1.35719634])
>>> binomial(32, .5, 5) # 32 trials, prob of an event = .5
array([12, 20, 21, 19, 17])
>>> negative_binomial(32, .5, 5) # 32 trials: prob of an event = .5
array([21, 38, 29, 32, 36])
\end{verbatim}
Two functions that return generate multivariate random numbers (that is, random
vectors with some known relationship between the elements of each vector,
defined by the distribution). They are \function{multivariate_normal} and
\function{multinomial}. For these two functions, the lengths of the leading
axes of the output may be specified. The length of the last axis is determined
by the length of some other parameter.
\begin{verbatim}
>>> multivariate_normal([1,2], [[1,2],[2,1]], [2,3])
array([[[ 0.14157988, 1.46232224],
[-1.11820295, -0.82796288],
[ 1.35251635, -0.2575901 ]],
[[-0.61142141, 1.0230465 ],
[-1.08280948, -0.55567217],
[ 2.49873002, 3.28136372]]])
>>> x = multivariate_normal([10,100], [[1,2],[2,1]], 10000)
>>> x_mean = sum(x)/10000
>>> print x_mean
[ 9.98599893 100.00032416]
>>> x_minus_mean = x - x_mean
>>> cov = matrixmultiply(transpose(x_minus_mean), x_minus_mean) / 9999.
>>> cov
array([[ 2.01737122, 1.00474408],
[ 1.00474408, 2.0009806 ]])
\end{verbatim}
The a priori probabilities for a multinomial distribution must sum to one. The
prior probability argument to \function{multinomial} doesn't give the prior
probability of the last event: it is computed to be one minus the sum of the
others.
\begin{verbatim}
>>> multinomial(16, [.1, .4, .2]) # prior probabilities [.1, .4, .2, .3]
array([2, 7, 1, 6])
>>> multinomial(16, [.1, .4, .2], [2,3]) # output shape [2,3,4]
array([[[ 1, 9, 1, 5],
[ 0, 10, 3, 3],
[ 4, 9, 3, 0]],
[[ 1, 6, 1, 8],
[ 3, 4, 5, 4],
[ 1, 5, 2, 8]]])
\end{verbatim}
Many of the functions accept arrays or sequences for the distribution
arguments. If no \var{shape} argument is given, then the shape of the output is
determined by the shape of the parameter argument. For instance:
\begin{verbatim}
>>> beta([5.0, 50.0], [10.0, 100.])
array([ 0.54379648,  0.35352072])
\end{verbatim}
Broadcasting rules apply if two or more arguments are arrays:
\begin{verbatim}
>>> beta([5.0, 50.0], [[10.0, 100.], [20.0, 200.0]])
array([[ 0.30204576,  0.32154009],
       [ 0.10851908,  0.19207685]])
\end{verbatim}
The \var{shape} argument can still be used to specify the output shape. Any
array argument will be broadcasted to have the given shape:
\begin{verbatim}
>>> beta(5.0, [10.0, 100.0], shape = (3, 2))
array([[ 0.49521708,  0.02218186],
       [ 0.21000148,  0.04366644],
       [ 0.43169656,  0.05285903]])
\end{verbatim}
%% Local Variables:
%% mode: LaTeX
%% mode: auto-fill
%% fill-column: 79
%% indent-tabs-mode: nil
%% ispell-dictionary: "american"
%% reftex-fref-is-default: nil
%% TeX-auto-save: t
%% TeX-command-default: "pdfeLaTeX"
%% TeX-master: "numarray"
%% TeX-parse-self: t
%% End:

\chapter{Multi-dimensional image processing}
\label{cha:ndimage}

%begin{latexonly}
\makeatletter
\py@reset
\makeatother
%end{latexonly}
\declaremodule[numarray.ndimage]{extension}{numarray.nd_image}
\moduleauthor{Peter Verveer}{verveer@users.sourceforge.net}
\modulesynopsis{Multidimensional image analysis functions}

\begin{quote}
  The \module{numarray.nd\_image} module provides functions for
  multidimensional image analysis.
\end{quote}

\section{Introduction}
Image processing and analysis are generally seen as operations on
two-dimensional arrays of values. There are however a number of fields
where images of higher dimensionality must be analyzed. Good examples
of these are medical imaging and biological imaging. \module{numarray}
is suited very well for this type of applications due its inherent
multi-dimensional nature. The \module{numarray.nd\_image} packages
provides a number of general image processing and analysis functions
that are designed to operate with arrays of arbitrary dimensionality.
The packages currently includes functions for linear and non-linear
filtering, binary morphology, B-spline interpolation, and object
measurements.

\section{Properties shared by all functions}
All functions share some common properties. Notably, all functions allow the 
specification of an output array with the \var{output} argument. With this 
argument you can specify an array that will be changed in-place with the 
result with the operation. In this case the result is not returned. Usually, 
using the \var{output} argument is more efficient, since an existing array 
is used to store the result.

The type of arrays returned is dependent on the type of operation, but it is  in most cases equal to the type of the input. If, however, the \var{output} argument is used, the type of the result is equal to the type of the specified output argument. If no output argument is given, it is still possible to specify what the result of the output should be. This is done by simply assigning the desired numarray type object to the output argument. For example:
\begin{verbatim}
>>> print correlate(arange(10), [1, 2.5])
[ 0  2  6  9 13 16 20 23 27 30]
>>> print correlate(arange(10), [1, 2.5], output = Float64)
[  0.    2.5   6.    9.5  13.   16.5  20.   23.5  27.   30.5]                   
\end{verbatim}
\note{In previous versions of \module{numarray.nd\_image}, some functions accepted the \var{output_type} argument to achieve the same effect. This argument is still supported, but its use will generate an deprecation warning. In a future version all instances of this argument will be removed. The preferred way to specify an output type, is by using the \var{output} argument, either by specifying an output array of the desired type, or by specifying the type of the output that is to be returned.}
%
\section{Filter functions}
\label{sec:ndimage:filter-functions}
The functions described in this section all perform some type of spatial
filtering of the the input array: the elements in the output are some 
function of the values in the neighborhood of the corresponding input 
element. We refer to this neighborhood of elements as the filter kernel, 
which is often rectangular in shape but may also have an arbitrary 
footprint. Many of the functions described below allow you to define the 
footprint of the kernel, by passing a mask through the \var{footprint} 
parameter. For example a cross shaped kernel can be defined as follows:
\begin{verbatim}
>>> footprint = array([[0,1,0],[1,1,1],[0,1,0]])
>>> print footprint
[[0 1 0]
 [1 1 1]
 [0 1 0]]
\end{verbatim}
Usually the origin of the kernel is at the center calculated by dividing 
the dimensions of the kernel shape by two.  For instance, the origin of a
one-dimensional kernel of length three is at the second element. Take for
example the correlation of a one-dimensional array with a filter of
length 3 consisting of ones:
\begin{verbatim}
>>> a = [0, 0, 0, 1, 0, 0, 0]
>>> correlate1d(a, [1, 1, 1])
[0 0 1 1 1 0 0]
\end{verbatim}
Sometimes it is convenient to choose a different origin for the kernel. For
this reason most functions support the \var{origin} parameter which gives 
the origin of the filter relative to its center. For example:
\begin{verbatim}
>>> a = [0, 0, 0, 1, 0, 0, 0]
>>> print correlate1d(a, [1, 1, 1], origin = -1)
[0 1 1 1 0 0 0]
\end{verbatim}
The effect is a shift of the result towards the left. This feature will not 
be needed very often, but it may be useful especially for filters that have 
an even size.  A good example is the calculation of backward and forward
differences:
\begin{verbatim}
>>> a = [0, 0, 1, 1, 1, 0, 0]
>>> print correlate1d(a, [-1, 1])              ## backward difference
[ 0  0  1  0  0 -1  0]
>>> print correlate1d(a, [-1, 1], origin = -1) ## forward difference
[ 0  1  0  0 -1  0  0]
\end{verbatim}
We could also have calculated the  forward difference as follows:
\begin{verbatim}
>>> print correlate1d(a, [0, -1, 1])
[ 0  1  0  0 -1  0  0]
\end{verbatim}
however, using the origin parameter instead of a larger kernel is more
efficient. For multi-dimensional kernels \var{origin} can be a number, in 
which case the origin is assumed to be equal along all axes, or a sequence  
giving the origin along each axis.

Since the output elements are a function of elements in the neighborhood of 
the input elements, the borders of the array need to be dealt with 
appropriately by providing the values outside the borders. This is done by 
assuming that the arrays are extended beyond their boundaries according 
certain boundary conditions. In the functions described below, the boundary 
conditions can be selected using the \var{mode} parameter which must be a 
string with the name of the boundary condition.  Following boundary 
conditions are currently supported:
\begin{tableiii}{l|l|l}{constant}{Boundary condition}{Description}{Example}
  \lineiii{"nearest"}{Use the value at the boundary}
  {\constant{[1 2 3]->[1 1 2 3 3]}}
  \lineiii{"wrap"}{Periodically replicate the array}
  {\constant{[1 2 3]->[3 1 2 3 1]}}
  \lineiii{"reflect"}{Reflect the array at the boundary}
  {\constant{[1 2 3]->[1 1 2 3 3]}}
  \lineiii{"constant"}{Use a constant value, default value is 0.0}
  {\constant{[1 2 3]->[0 1 2 3 0]}}
\end{tableiii}
The \constant{"constant"} mode is special since it needs an additional
parameter to specify the constant value that should be used.

\note{The easiest way to implement such boundary conditions would be to 
copy the data to a larger array and extend the data at the borders 
according to the boundary conditions. For large arrays and large filter 
kernels, this would be very memory consuming, and the functions described 
below therefore use a different approach that does not require allocating 
large temporary buffers.}

\subsection{Correlation and convolution}

\begin{funcdesc}{correlate1d}{input, weights, axis=-1, output=None, 
    mode='reflect', cval=0.0, origin=0, output_type=None} The
  \function{correlate1d} function calculates a one-dimensional correlation
  along the given axis. The lines of the array along the given axis are
  correlated with the given \var{weights}. The \var{weights} parameter must 
  be a one-dimensional sequences of numbers.
\end{funcdesc}

\begin{funcdesc}{correlate}{input, weights, output=None, mode='reflect', 
    cval=0.0, origin=0, output_type=None} The function \function{correlate}
  implements multi-dimensional correlation of the input array with a given
  kernel.
\end{funcdesc}

\begin{funcdesc}{convolve1d}{input, weights, axis=-1, output=None, 
    mode='reflect', cval=0.0, origin=0, output_type=None} The
  \function{convolve1d} function calculates a one-dimensional convolution 
  along the given axis. The lines of the array along the given axis are 
  convoluted with the given \var{weights}. The \var{weights} parameter must 
  be a one-dimensional sequences of numbers.
  
  \note{A convolution is essentially a correlation after mirroring the 
  kernel. As a result, the \var{origin} parameter behaves differently than 
  in the case of a correlation: the result is shifted in the opposite 
  directions.}
\end{funcdesc}

\begin{funcdesc}{convolve}{input, weights, output=None, mode='reflect', 
    cval=0.0, origin=0, output_type=None} The function \function{convolve}
  implements multi-dimensional convolution of the input array with a given
  kernel.
  
  \note{A convolution is essentially a correlation after mirroring the 
  kernel. As a result, the \var{origin} parameter behaves differently than 
  in the case of a correlation: the results is shifted in the opposite 
  direction.}
\end{funcdesc}

\subsection{Smoothing filters}
\label{sec:ndimage:filter-functions:smoothing}

\begin{funcdesc}{gaussian_filter1d}{input, sigma, axis=-1, order=0, 
    output=None, mode='reflect', cval=0.0, output_type=None} The
  \function{gaussian_filter1d} function implements a one-dimensional 
  Gaussian
  filter. The standard-deviation of the Gaussian filter is passed through 
  the parameter \var{sigma}. Setting \var{order}=0 corresponds to 
  convolution with a Gaussian kernel.  An order of 1, 2, or 3 corresponds 
  to convolution with the first, second or third derivatives of a Gaussian. 
  Higher order derivatives are not implemented.
\end{funcdesc}

\begin{funcdesc}{gaussian_filter}{input, sigma, order=0, output=None, 
  mode='reflect', cval=0.0, output_type=None} The 
  \function{gaussian_filter} function implements a multi-dimensional 
  Gaussian filter. The standard-deviations of the Gaussian filter along 
  each axis are passed through the parameter \var{sigma} as a sequence or 
  numbers.  If \var{sigma} is not a sequence but a single number, the 
  standard deviation of the filter is equal along all directions. The 
  order of the filter can be specified separately for each axis. An order 
  of 0 corresponds to convolution with a Gaussian kernel. An order of 1, 
  2, or 3 corresponds to convolution with the first, second or
  third derivatives of a Gaussian. Higher order derivatives are not
  implemented. The \var{order} parameter must be a number, to specify the 
  same order for all axes, or a sequence of numbers to specify a different 
  order for each axis.
  
  \note{The multi-dimensional filter is implemented as a sequence of
    one-dimensional Gaussian filters. The intermediate arrays are stored in 
    the same data type as the output.  Therefore, for output types with a 
    lower precision, the results may be imprecise because intermediate 
    results may be stored with insufficient precision. This can be 
    prevented by specifying a more precise output type.}
\end{funcdesc}

\begin{funcdesc}{uniform_filter1d}{input, size, axis=-1, output=None,
    mode='reflect', cval=0.0, origin=0, output_type=None} The
  \function{uniform_filter1d} function calculates a one-dimensional uniform
  filter of the given \var{size} along the given axis.
\end{funcdesc}

\begin{funcdesc}{uniform_filter}{input, size, output=None, mode='reflect', 
    cval=0.0, origin=0, output_type=None} The \function{uniform_filter}
  implements a multi-dimensional uniform filter.  The sizes of the uniform 
  filter are given for each axis as a sequence of integers by the 
  \var{size} parameter. If \var{size} is not a sequence, but a single 
  number, the sizes along all axis are assumed to be equal.
    
  \note{The multi-dimensional filter is implemented as a sequence of
    one-dimensional uniform filters. The intermediate arrays are stored in 
    the same data type as the output. Therefore, for output types with a 
    lower precision, the results may be imprecise because intermediate 
    results may be stored with insufficient precision. This can be 
    prevented by specifying a
    more precise output type.}
  \end{funcdesc}

\subsection{Filters based on order statistics}

\begin{funcdesc}{minimum_filter1d}{input, size, axis=-1, output=None, 
    mode='reflect', cval=0.0, origin=0} The \function{minimum_filter1d}
  function calculates a one-dimensional minimum filter of given \var{size}
  along the given axis.
\end{funcdesc}

\begin{funcdesc}{maximum_filter1d}{input, size, axis=-1, output=None, 
    mode='reflect', cval=0.0, origin=0} The \function{maximum_filter1d}
  function calculates a one-dimensional maximum filter of given \var{size}
  along the given axis.
\end{funcdesc}

\begin{funcdesc}{minimum_filter}{input,  size=None, footprint=None, 
    output=None, mode='reflect', cval=0.0, origin=0} The
  \function{minimum_filter} function calculates a multi-dimensional minimum
  filter. Either the sizes of a rectangular kernel or the footprint of the
  kernel must be provided. The \var{size} parameter, if provided, must be a
  sequence of sizes or a single number in which case the size of the filter 
  is assumed to be equal along each axis. The \var{footprint}, if provided, 
  must be an array that defines the shape of the kernel by its non-zero 
  elements.
\end{funcdesc}

\begin{funcdesc}{maximum_filter}{input,  size=None, footprint=None, 
    output=None, mode='reflect', cval=0.0, origin=0} The
  \function{maximum_filter} function calculates a multi-dimensional maximum
  filter. Either the sizes of a rectangular kernel or the footprint of the
  kernel must be provided. The \var{size} parameter, if provided, must be a
  sequence of sizes or a single number in which case the size of the filter 
  is assumed to be equal along each axis. The \var{footprint}, if provided, 
  must be an array that defines the shape of the kernel by its non-zero 
  elements.
\end{funcdesc}

\begin{funcdesc}{rank_filter}{input, rank, size=None, footprint=None,
  output=None, mode='reflect', cval=0.0, origin=0} The 
  \function{rank_filter}
  function calculates a multi-dimensional rank filter.  The \var{rank} may 
  be less then zero, i.e., \var{rank}=-1 indicates the largest element. 
  Either the sizes of a rectangular kernel or the footprint of the kernel 
  must be provided. The \var{size} parameter, if provided, must be a 
  sequence of sizes or a single number in which case the size of the filter 
  is assumed to be equal along each axis. The \var{footprint}, if provided, 
  must be an array that defines the shape of the kernel by its non-zero 
  elements.
\end{funcdesc}

\begin{funcdesc}{percentile_filter}{input, percentile, size=None, 
  footprint=None, output=None, mode='reflect', cval=0.0, origin=0} The
  \function{percentile_filter} function calculates a multi-dimensional
  percentile filter.  The \var{percentile} may be less then zero, i.e.,
  \var{percentile}=-20 equals \var{percentile}=80. Either the sizes of a 
  rectangular kernel or the footprint of the kernel must be provided. The 
  \var{size} parameter, if provided, must be a sequence of sizes or a 
  single number in which case the size of the filter is assumed to be equal 
  along each axis. The \var{footprint}, if provided, must be an array that 
  defines the shape of the kernel by its non-zero elements.
\end{funcdesc}

\begin{funcdesc}{median_filter}{input, size=None, footprint=None, 
  output=None, mode='reflect', cval=0.0, origin=0} The 
  \function{median_filter} function calculates a multi-dimensional median 
  filter. Either the sizes of a rectangular kernel or the footprint of the 
  kernel must be provided. The \var{size} parameter, if provided, must be a 
  sequence of sizes or a single number in which case the size of the filter 
  is assumed to be equal along each axis. The \var{footprint} if provided, 
  must be an array that defines the shape of the kernel by its non-zero 
  elements.
\end{funcdesc}

\subsection{Derivatives}

Derivative filters can be constructed in several ways. The function
\function{gaussian_filter1d} described in section
\ref{sec:ndimage:filter-functions:smoothing} can be used to calculate
derivatives along a given axis using the \var{order} parameter. Other
derivative filters are the Prewitt and Sobel filters:

\begin{funcdesc}{prewitt}{input, axis=-1, output=None, mode='reflect', 
  cval=0.0} The \function{prewitt} function calculates a derivative along 
  the given axis.
\end{funcdesc}

\begin{funcdesc}{sobel}{input, axis=-1, output=None, mode='reflect', 
  cval=0.0} The \function{sobel} function calculates a derivative along 
  the given axis.
\end{funcdesc}

The Laplace filter is calculated by the sum of the second derivatives along 
all axes. Thus, different Laplace filters can be constructed using 
different second derivative functions. Therefore we provide a general 
function that takes a function argument to calculate the second derivative 
along a given direction and to construct the Laplace filter:

\begin{funcdesc}{generic_laplace}{input, derivative2, output=None,
  mode='reflect', cval=0.0, output_type=None, extra_arguments = (), 
  extra_keywords = {}} The function 
  \function{generic_laplace} calculates a laplace filter using the
  function passed through \var{derivative2} to calculate second 
  derivatives. The function \function{derivative2} should have the 
  following signature:

  \function{derivative2(input, axis, output, mode, cval, *extra_arguments, **extra_keywords)}
  
  It should calculate the second derivative along the dimension \var{axis}. 
  If \var{output} is not \constant{None} it should use that for the output 
  and return \constant{None}, otherwise it should return the result. 
  \var{mode}, \var{cval} have the usual meaning.
  
  The \var{extra_arguments} and \var{extra_keywords} arguments can be used 
  to pass a tuple of extra arguments and a dictionary of named 
  arguments that are passed to \function{derivative2} at each call.

  For example:
\begin{verbatim}
>>> def d2(input, axis, output, mode, cval):
...     return correlate1d(input, [1, -2, 1], axis, output, mode, cval, 0)
... 
>>> a = zeros((5, 5))
>>> a[2, 2] = 1
>>> print generic_laplace(a, d2)
[[ 0  0  0  0  0]
 [ 0  0  1  0  0]
 [ 0  1 -4  1  0]
 [ 0  0  1  0  0]
 [ 0  0  0  0  0]]
\end{verbatim}
To demonstrate the use of the \var{extra_arguments} argument we could do:
\begin{verbatim}
>>> def d2(input, axis, output, mode, cval, weights):
...     return correlate1d(input, weights, axis, output, mode, cval, 0,)
... 
>>> a = zeros((5, 5))
>>> a[2, 2] = 1
>>> print generic_laplace(a, d2, extra_arguments = ([1, -2, 1],))
[[ 0  0  0  0  0]
 [ 0  0  1  0  0]
 [ 0  1 -4  1  0]
 [ 0  0  1  0  0]
 [ 0  0  0  0  0]]
\end{verbatim}
or:
\begin{verbatim}
>>> print generic_laplace(a, d2, extra_keywords = {'weights': [1, -2, 1]})
[[ 0  0  0  0  0]
 [ 0  0  1  0  0]
 [ 0  1 -4  1  0]
 [ 0  0  1  0  0]
 [ 0  0  0  0  0]]
\end{verbatim}
\end{funcdesc}

The following two functions are implemented using 
\function{generic_laplace} by providing appropriate functions for the 
second derivative function:

\begin{funcdesc}{laplace}{input, output=None, mode='reflect', 
  cval=0.0, output_type=None} 
  The function \function{laplace} calculates 
  the Laplace using discrete differentiation for the second derivative 
  (i.e. convolution with \constant{[1, -2, 1]}).
\end{funcdesc}

\begin{funcdesc}{gaussian_laplace}{input, sigma, output=None, 
  mode='reflect', cval=0.0, output_type=None} The function 
  \function{gaussian_laplace} calculates the Laplace using 
  \function{gaussian_filter} to calculate the
  second derivatives. The standard-deviations of the Gaussian filter along 
  each axis are passed through the parameter \var{sigma} as a sequence or 
  numbers.  If \var{sigma} is not a sequence but a single number, the 
  standard deviation of the filter is equal along all directions.
  \end{funcdesc}

The gradient magnitude is defined as the square root of the sum of the 
squares of the gradients in all directions. Similar to the generic Laplace 
function there is a \function{generic_gradient_magnitude} function that 
calculated the gradient magnitude of an array:

\begin{funcdesc}{generic_gradient_magnitude}{input, derivative,
  output=None, mode='reflect', cval=0.0, output_type=None, 
  extra_arguments = (), extra_keywords = {}} The 
  function \function{generic_gradient_magnitude} calculates a gradient 
  magnitude using the function passed through \var{derivative} to calculate 
  first derivatives. The function \function{derivative} should have the 
  following signature:

  \function{derivative(input, axis, output, mode, cval, *extra_arguments, **extra_keywords)}
  
  It should calculate the derivative along the dimension \var{axis}. If
  \var{output} is not \constant{None} it should use that for the output and
  return \constant{None}, otherwise it should return the result. 
  \var{mode}, \var{cval} have the usual meaning.
  
  The \var{extra_arguments} and \var{extra_keywords} arguments can be used 
  to pass a tuple of extra arguments and a dictionary of named 
  arguments that are passed to \function{derivative} at each call.

  For example, the \function{sobel} function fits the required signature:
\begin{verbatim}
>>> a = zeros((5, 5))
>>> a[2, 2] = 1
>>> print generic_gradient_magnitude(a, sobel)
[[0 0 0 0 0]
 [0 1 2 1 0]
 [0 2 0 2 0]
 [0 1 2 1 0]
 [0 0 0 0 0]]
\end{verbatim}
See the documentation of \function{generic_laplace} for examples of using the \var{extra_arguments} and \var{extra_keywords} arguments.
\end{funcdesc}

The \function{sobel} and \function{prewitt} functions fit the required
signature and can therefore directly be used with
\function{generic_gradient_magnitude}. The following function implements 
the gradient magnitude using Gaussian derivatives:

\begin{funcdesc}{gaussian_gradient_magnitude}{input, sigma, output=None, 
  mode='reflect', cval=0.0, output_type=None} The function
  \function{gaussian_gradient_magnitude} calculates the gradient magnitude
  using \function{gaussian_filter} to calculate the first derivatives. The
  standard-deviations of the Gaussian filter along each axis are passed 
  through the parameter \var{sigma} as a sequence or numbers.  If 
  \var{sigma} is not a sequence but a single number, the standard deviation 
  of the filter is equal along all directions.
\end{funcdesc}

\subsection{Generic filter functions}
\label{sec:ndimage:genericfilters}
To implement filter functions, generic functions can be used that accept a 
callable object that implements the filtering operation. The iteration over 
the input and output arrays is handled by these generic functions, along 
with such details as the implementation of the boundary conditions. Only a 
callable object implementing a callback function that does the actual 
filtering work must be provided. The callback function can also be written 
in C and passed using a CObject (see \ref{sec:ndimage:ccallbacks} for more 
information).

\begin{funcdesc}{generic_filter1d}{input, function, filter_size, axis=-1,
  output=None, mode="reflect", cval=0.0, origin=0, output_type=None,
  extra_arguments = (), extra_keywords = {}}
  The \function{generic_filter1d} function implements a generic 
  one-dimensional filter function, where the actual filtering operation 
  must be supplied as a python function (or other callable object). The 
  \function{generic_filter1d} function iterates over the lines of an array 
  and calls \var{function} at each line. The arguments that are passed to 
  \var{function} are one-dimensional arrays of the \constant{tFloat64} 
  type. The first contains the values of the current line. It is extended 
  at the beginning end the end, according to the \var{filter_size} and 
  \var{origin} arguments. The second array should be modified in-place to 
  provide the output values of the line. For example 
  consider a correlation along one dimension:

\begin{verbatim}
>>> a = arange(12, shape = (3,4))
>>> print correlate1d(a, [1, 2, 3])
[[ 3  8 14 17]
 [27 32 38 41]
 [51 56 62 65]]
\end{verbatim}
The same operation can be implemented using \function{generic_filter1d} as 
follows:
\begin{verbatim} 
>>> def fnc(iline, oline):
...     oline[...] = iline[:-2] + 2 * iline[1:-1] + 3 * iline[2:]
... 
>>> print generic_filter1d(a, fnc, 3)
[[ 3  8 14 17]
 [27 32 38 41]
 [51 56 62 65]]
\end{verbatim}
  Here the origin of the kernel was (by default) assumed to be in the 
  middle of the filter of length 3. Therefore, each input line was
  extended by one value at the beginning and at the end, before the 
  function was called.
  
  Optionally extra arguments can be defined and passed to the filter 
  function. The \var{extra_arguments} and \var{extra_keywords} arguments 
  can be used to pass a tuple of extra arguments and/or a dictionary of 
  named arguments that are passed to derivative at each call. For example, 
  we can pass the parameters of our filter as an argument:
\begin{verbatim} 
>>> def fnc(iline, oline, a, b):
...     oline[...] = iline[:-2] + a * iline[1:-1] + b * iline[2:]
... 
>>> print generic_filter1d(a, fnc, 3, extra_arguments = (2, 3))
[[ 3  8 14 17]
 [27 32 38 41]
 [51 56 62 65]]
\end{verbatim}
or
\begin{verbatim} 
>>> print generic_filter1d(a, fnc, 3, extra_keywords = {'a':2, 'b':3})
[[ 3  8 14 17]
 [27 32 38 41]
 [51 56 62 65]]
\end{verbatim}
\end{funcdesc}

\begin{funcdesc}{generic_filter}{input, function, size=None,
  footprint=None, output=None, mode='reflect', cval=0.0, origin=0, 
  output_type=None, extra_arguments = (), extra_keywords = {}}
  The \function{generic_filter} function implements a generic filter  
  function,  where the actual filtering operation must be supplied as a 
  python function (or other callable object). The \function{generic_filter} 
  function iterates over the array and calls \var{function} at each 
  element. The argument of \var{function} is a one-dimensional array of the 
  \constant{tFloat64} type, that contains the values around the current
  element that are within the footprint of the filter. The function should 
  return a single value that can be converted to a double precision 
  number. For example consider a correlation:

\begin{verbatim}
>>> a = arange(12, shape = (3,4))
>>> print correlate(a, [[1, 0], [0, 3]])
[[ 0  3  7 11]
 [12 15 19 23]
 [28 31 35 39]]
\end{verbatim}
The same operation can be implemented using \function{generic_filter} as 
follows:
\begin{verbatim} 
>>> def fnc(buffer): 
...     return (buffer * array([1, 3])).sum()
... 
>>> print generic_filter(a, fnc, footprint = [[1, 0], [0, 1]])
[[ 0  3  7 11]
 [12 15 19 23]
 [28 31 35 39]]
\end{verbatim}
  Here a kernel footprint was specified that contains only two elements.
  Therefore the filter function receives a buffer of length equal to two,
  which was multiplied with the proper weights and the result summed.

  When calling \function{generic_filter}, either the sizes of a rectangular 
  kernel or the footprint of the kernel must be provided. The \var{size} 
  parameter, if provided, must be a sequence of sizes or a single number in 
  which case the size of the filter is assumed to be equal along each axis. 
  The \var{footprint}, if provided, must be an array that defines the shape 
  of the kernel by its non-zero elements.

  Optionally extra arguments can be defined and passed to the filter 
  function. The \var{extra_arguments} and \var{extra_keywords} arguments 
  can be used to pass a tuple of extra arguments and/or a dictionary of 
  named arguments that are passed to derivative at each call. For example, 
  we can pass the parameters of our filter as an argument:
\begin{verbatim} 
>>> def fnc(buffer, weights): 
...     weights = asarray(weights)
...     return (buffer * weights).sum()
... 
>>> print generic_filter(a, fnc, footprint = [[1, 0], [0, 1]], extra_arguments = ([1, 3],))
[[ 0  3  7 11]
 [12 15 19 23]
 [28 31 35 39]]
\end{verbatim}
or
\begin{verbatim} 
>>> print generic_filter(a, fnc, footprint = [[1, 0], [0, 1]], extra_keywords= {'weights': [1, 3]})
[[ 0  3  7 11]
 [12 15 19 23]
 [28 31 35 39]]
\end{verbatim}
\end{funcdesc}

These functions iterate over the lines or elements starting at the 
last axis, i.e. the last index changest the fastest. This order of iteration 
is garantueed for the case that it is important to adapt the filter 
dependening on spatial location. Here is an example of using a class that 
implements the filter and keeps track of the current coordinates while 
iterating. It performs the same filter operation as described above for 
\function{generic_filter}, but additionally prints the current coordinates:
\begin{verbatim}
>>> a = arange(12, shape = (3,4))
>>> 
>>> class fnc_class:
...     def __init__(self, shape):
...         # store the shape:
...         self.shape = shape
...         # initialize the coordinates:
...         self.coordinates = [0] * len(shape)
...         
...     def filter(self, buffer):
...         result = (buffer * array([1, 3])).sum()
...         print self.coordinates
...         # calculate the next coordinates:
...         axes = range(len(self.shape))
...         axes.reverse()
...         for jj in axes:
...             if self.coordinates[jj] < self.shape[jj] - 1:
...                 self.coordinates[jj] += 1
...                 break
...             else:
...                 self.coordinates[jj] = 0
...         return result
... 
>>> fnc = fnc_class(shape = (3,4))
>>> print generic_filter(a, fnc.filter, footprint = [[1, 0], [0, 1]]) 
[0, 0]
[0, 1]
[0, 2]
[0, 3]
[1, 0]
[1, 1]
[1, 2]
[1, 3]
[2, 0]
[2, 1]
[2, 2]
[2, 3]
[[ 0  3  7 11]
 [12 15 19 23]
 [28 31 35 39]]
\end{verbatim}

For the \function{generic_filter1d} function the same approach works, except that this function does not iterate over the axis that is being filtered. The example for \function{generic_filte1d} then becomes this:
\begin{verbatim}
>>> a = arange(12, shape = (3,4))
>>> 
>>> class fnc1d_class:
...     def __init__(self, shape, axis = -1):
...         # store the filter axis:
...         self.axis = axis
...         # store the shape:
...         self.shape = shape
...         # initialize the coordinates:
...         self.coordinates = [0] * len(shape)
...         
...     def filter(self, iline, oline):
...         oline[...] = iline[:-2] + 2 * iline[1:-1] + 3 * iline[2:]
...         print self.coordinates
...         # calculate the next coordinates:
...         axes = range(len(self.shape))
...         # skip the filter axis:
...         del axes[self.axis]
...         axes.reverse()
...         for jj in axes:
...             if self.coordinates[jj] < self.shape[jj] - 1:
...                 self.coordinates[jj] += 1
...                 break
...             else:
...                 self.coordinates[jj] = 0
... 
>>> fnc = fnc1d_class(shape = (3,4))
>>> print generic_filter1d(a, fnc.filter, 3)
[0, 0]
[1, 0]
[2, 0]
[[ 3  8 14 17]
 [27 32 38 41]
 [51 56 62 65]]
\end{verbatim}

\section{Fourier domain filters}
The functions described in this section perform filtering operations in the
Fourier domain. Thus, the input array of such a function should be 
compatible with an inverse Fourier transform function, such as the 
functions from the \module{numarray.fft} module. We therefore have to deal 
with arrays that may be the result of a real or a complex Fourier 
transform. In the case of a real Fourier transform only half of the of the 
symmetric complex transform is stored. Additionally, it needs to be known 
what the length of the axis was that was transformed by the real fft.  The 
functions described here provide a parameter \var{n} that in the case of a 
real transform must be equal to the length of the real transform axis 
before transformation. If this parameter is less than zero, it is assumed 
that the input array was the result of a complex Fourier transform. The 
parameter \var{axis} can be used to indicate along which axis the real 
transform was executed.

\begin{funcdesc}{fourier_shift}{input, shift, n=-1, axis=-1, output=None}
  The \function{fourier_shift} function multiplies the input array with the
  multi-dimensional Fourier transform of a shift operation for the given 
  shift. The \var{shift} parameter is a sequences of shifts for each 
  dimension, or a single value for all dimensions.
\end{funcdesc}

\begin{funcdesc}{fourier_gaussian}{input, sigma, n=-1, axis=-1, output=None}
  The \function{fourier_gaussian} function multiplies the input array with 
  the multi-dimensional Fourier transform of a Gaussian filter with given
  standard-deviations \var{sigma}. The \var{sigma} parameter is a sequences 
  of values for each dimension, or a single value for all dimensions.
\end{funcdesc}

\begin{funcdesc}{fourier_uniform}{input, size, n=-1, axis=-1, output=None}
  The \function{fourier_uniform} function multiplies the input array with 
  the multi-dimensional Fourier transform of a uniform filter with given
  sizes \var{size}. The \var{size} parameter is a sequences of
  values for each dimension, or a single value for all dimensions.
\end{funcdesc}

\begin{funcdesc}{fourier_ellipsoid}{input, size, n=-1, axis=-1, 
  output=None}
  The \function{fourier_ellipsoid} function multiplies the input array with 
  the multi-dimensional Fourier transform of a elliptically shaped filter 
  with given sizes \var{size}. The \var{size} parameter is a sequences of 
  values for each dimension, or a single value for all dimensions.  
  \note{This function is
    only implemented for dimensions 1, 2, and 3.}
\end{funcdesc}

\section{Interpolation functions}
This section describes various interpolation functions that are based on
B-spline theory. A good introduction to B-splines can be found in: M. 
Unser, "Splines: A Perfect Fit for Signal and Image Processing," IEEE 
Signal Processing Magazine, vol. 16, no. 6, pp. 22-38, November 1999.
\subsection{Spline pre-filters}
Interpolation using splines of an order larger than 1 requires a pre-
filtering step. The interpolation functions described in section
\ref{sec:ndimage:interpolation} apply pre-filtering by calling
\function{spline_filter}, but they can be instructed not to do this by 
setting the \var{prefilter} keyword equal to \constant{False}.  This is 
useful if more than one interpolation operation is done on the same array. 
In this case it is more efficient to do the pre-filtering only once and use 
a prefiltered array as the input of the interpolation functions. The 
following two functions implement the pre-filtering:

\begin{funcdesc}{spline_filter1d}{input, order=3, axis=-1, output=None,
    output_type=numarray.Float64} The \function{spline_filter1d} function
  calculates a one-dimensional spline filter along the given axis. An 
  output array can optionally be provided. The order of the spline must be 
  larger then 1 and less than 6.
\end{funcdesc}

\begin{funcdesc}{spline_filter}{input, order=3, output=None, 
    output_type=numarray.Float64} The \function{spline_filter} function
  calculates a multi-dimensional spline filter.
  
  \note{The multi-dimensional filter is implemented as a sequence of
    one-dimensional spline filters. The intermediate arrays are stored in 
    the same data type as the output. Therefore, if an output 
    with a limited precision is requested, the results may be imprecise 
    because intermediate results may be stored with insufficient precision. 
    This can be prevented by specifying a output type of high precision.}
\end{funcdesc}

\subsection{Interpolation functions}
\label{sec:ndimage:interpolation}
Following functions all employ spline interpolation to effect some type of
geometric transformation of the input array. This requires a mapping of the
output coordinates to the input coordinates, and therefore the possibility
arises that input values outside the boundaries are needed. This problem 
is solved in the same way as described in section
\ref{sec:ndimage:filter-functions} for the multi-dimensional filter 
functions. Therefore these functions all support a \var{mode} parameter 
that determines how the boundaries are handled, and a \var{cval} parameter 
that gives a constant value in case that the \constant{'constant'} mode is 
used.

\begin{funcdesc}{geometric_transform}{input, mapping, output_shape=None,
    output_type=None, output=None, order=3, mode='constant', cval=0.0,
    prefilter=True, extra_arguments = (), extra_keywords = {}} The \function{geometric_transform} function applies an
  arbitrary geometric transform to the input. The given \var{mapping} 
  function is called at each point in the output to find the corresponding 
  coordinates in the input.  \var{mapping} must be a callable object that 
  accepts a tuple of length equal to the output array rank and returns the 
  corresponding input coordinates as a tuple of length equal to the input 
  array rank. The output shape and output type can optionally be provided. 
  If not given they are equal to the input shape and type.
  
  For example:
\begin{verbatim}
>>> a = arange(12, shape=(4,3), type = Float64)
>>> def shift_func(output_coordinates):
...     return (output_coordinates[0] - 0.5, output_coordinates[1] - 0.5)
... 
>>> print geometric_transform(a, shift_func)
[[ 0.      0.      0.    ]
 [ 0.      1.3625  2.7375]
 [ 0.      4.8125  6.1875]
 [ 0.      8.2625  9.6375]]  
\end{verbatim}  

  Optionally extra arguments can be defined and passed to the filter 
  function. The \var{extra_arguments} and \var{extra_keywords} arguments 
  can be used to pass a tuple of extra arguments and/or a dictionary of 
  named arguments that are passed to derivative at each call. For example, 
  we can pass the shifts in our example as arguments:

\begin{verbatim}
>>> def shift_func(output_coordinates, s0, s1):
...     return (output_coordinates[0] - s0, output_coordinates[1] - s1)
... 
>>> print geometric_transform(a, shift_func, extra_arguments = (0.5, 0.5))
[[ 0.      0.      0.    ]
 [ 0.      1.3625  2.7375]
 [ 0.      4.8125  6.1875]
 [ 0.      8.2625  9.6375]]  
\end{verbatim}  
or
\begin{verbatim}
>>> print geometric_transform(a, shift_func, extra_keywords = {'s0': 0.5, 's1': 0.5})
[[ 0.      0.      0.    ]
 [ 0.      1.3625  2.7375]
 [ 0.      4.8125  6.1875]
 [ 0.      8.2625  9.6375]]  
\end{verbatim}  

\note{The mapping function can also be written in C and passed using a CObject. See \ref{sec:ndimage:ccallbacks} for more information.}
\end{funcdesc}

\begin{funcdesc}{map_coordinates}{input, coordinates, output_type=None, 
    output=None, order=3, mode='constant', cval=0.0, prefilter=True} 
  The function \function{map_coordinates} applies an arbitrary coordinate
  transformation using the given array of coordinates. The shape of the 
  output is derived from that of the coordinate array by dropping the first 
  axis. The parameter \var{coordinates} is used to find for each point in 
  the output the corresponding coordinates in the input. The values of 
  \var{coordinates} along the first axis are the coordinates in the input 
  array at which the output value is found. (See also the numarray 
  \function{coordinates} function.) Since the coordinates may be non-
  integer coordinates, the value of the input at these coordinates is 
  determined by spline interpolation of the requested order. Here is an 
  example that interpolates a 2D array at (0.5, 0.5) and (1, 2):
\begin{verbatim}
>>> a = arange(12, shape=(4,3), type = numarray.Float64)
>>> print a
[[  0.   1.   2.]
 [  3.   4.   5.]
 [  6.   7.   8.]
 [  9.  10.  11.]]
>>> print map_coordinates(a, [[0.5, 2], [0.5, 1]])
[ 1.3625  7.    ]
\end{verbatim}
\end{funcdesc}

\begin{funcdesc}{affine_transform}{input, matrix, offset=0.0, 
  output_shape=None, output_type=None, output=None, order=3, 
  mode='constant', cval=0.0, prefilter=True} The 
  \function{affine_transform} function applies an affine transformation to 
  the input array. The given transformation \var{matrix} and \var{offset} 
  are used to find for each point in the output the corresponding 
  coordinates in the input.  The value of the input at the
  calculated coordinates is determined by spline interpolation of the 
  requested order. The transformation \var{matrix} must be two-dimensional 
  or can also be given as a one-dimensional sequence or array.  In the 
  latter case, it is assumed that the matrix is diagonal. A more efficient 
  interpolation algorithm is then applied that exploits the separability of 
  the problem.  The output shape and output type can optionally be 
  provided. If not given they are equal to the input shape and type.
\end{funcdesc}

\begin{funcdesc}{shift}{input, shift, output_type=None, output=None, 
  order=3, mode='constant', cval=0.0, prefilter=True} The \function{shift} 
  function returns a shifted version of the input, using spline 
  interpolation of the requested \var{order}.
\end{funcdesc}

\begin{funcdesc}{zoom}{input, zoom, output_type=None, output=None, order=3, 
    mode='constant', cval=0.0, prefilter=True} The \function{zoom} function
  returns a rescaled version of the input, using spline interpolation of 
  the requested \var{order}.
\end{funcdesc}

\begin{funcdesc}{rotate}{input, angle, axes=(-1, -2), reshape=1,
    output_type=None, output=None, order=3, mode='constant', cval=0.0,
    prefilter=True} The \function{rotate} function returns the input array
  rotated in the plane defined by the two axes given by the parameter
  \var{axes}, using spline interpolation of the requested \var{order}. The
  angle must be given in degrees. If \var{reshape} is true, then the size 
  of the output array is adapted to contain the rotated input.
\end{funcdesc}

\section{Binary morphology}
\label{sec:ndimage:binary-morphology}

\begin{funcdesc}{generate_binary_structure}{rank, connectivity}
  The \function{generate_binary_structure} functions generates a binary
  structuring element for use in binary morphology operations. The 
  \var{rank} of the structure must be provided. The size of the structure 
  that is returned is equal to three in each direction. The value of each 
  element is equal to one if the square of the Euclidean distance from the 
  element to the center is less or equal to \var{connectivity}. For 
  instance, two dimensional 4-connected and 8-connected structures are 
  generated as follows:
\begin{verbatim}
>>> print generate_binary_structure(2, 1)
[[0 1 0]
 [1 1 1]
 [0 1 0]]
>>> print generate_binary_structure(2, 2)
[[1 1 1]
 [1 1 1]
 [1 1 1]]
\end{verbatim}
\end{funcdesc}

Most binary morphology functions can be expressed in terms of the basic
operations erosion and dilation:

\begin{funcdesc}{binary_erosion}{input, structure=None, iterations=1,
    mask=None, output=None, border_value=0, origin=0} The
  \function{binary_erosion} function implements binary erosion of arrays of
  arbitrary rank with the given structuring element. The origin parameter
  controls the placement of the structuring element as described in section
  \ref{sec:ndimage:filter-functions}. If no structuring element is 
  provided, an element with connectivity equal to one is generated using
  \function{generate_binary_structure}. The \var{border_value} parameter 
  gives the value of the array outside boundaries. The erosion is repeated
  \var{iterations} times. If \var{iterations} is less than one, the erosion 
  is repeated until the result does not change anymore. If a \var{mask} 
  array is given, only those elements with a true value at the 
  corresponding mask element are modified at each iteration.
\end{funcdesc}

\begin{funcdesc}{binary_dilation}{input, structure=None, iterations=1,
    mask=None, output=None, border_value=0, origin=0} The
  \function{binary_dilation} function implements binary dilation of arrays 
  of arbitrary rank with the given structuring element. The origin 
  parameter controls the placement of the structuring element as described 
  in section \ref{sec:ndimage:filter-functions}. If no structuring element 
  is provided, an element with connectivity equal to one is generated using
  \function{generate_binary_structure}. The \var{border_value} parameter 
  gives the value of the array outside boundaries. The dilation is repeated
  \var{iterations} times.  If \var{iterations} is less than one, the 
  dilation is repeated until the result does not change anymore. If a 
  \var{mask} array is given, only those elements with a true value at the 
  corresponding mask element are modified at each iteration.

  Here is an example of using \function{binary_dilation} to find all 
  elements that touch the border, by repeatedly dilating an empty array 
  from the border using the data array as the mask:
\begin{verbatim}
>>> struct = array([[0, 1, 0], [1, 1, 1], [0, 1, 0]])
>>> a = array([[1,0,0,0,0], [1,1,0,1,0], [0,0,1,1,0], [0,0,0,0,0]])
>>> print a
[[1 0 0 0 0]
 [1 1 0 1 0]
 [0 0 1 1 0]
 [0 0 0 0 0]]
>>> print binary_dilation(zeros(a.shape), struct, -1, a, border_value=1)
[[1 0 0 0 0]
 [1 1 0 0 0]
 [0 0 0 0 0]
 [0 0 0 0 0]]
\end{verbatim}
\end{funcdesc}

The \function{binary_erosion} and \function{binary_dilation} functions both
have an \var{iterations} parameter which allows the erosion or dilation to 
be repeated a number of times. Repeating an erosion or a dilation with a 
given structure \constant{n} times is equivalent to an erosion or a 
dilation with a structure that is \constant{n-1} times dilated with itself. 
A function is provided that allows the calculation of a structure that is 
dilated a number of times with itself:

\begin{funcdesc}{iterate_structure}{structure, iterations, origin=None} 
  The \function{iterate_structure} function returns a structure by dilation 
  of the input structure \var{iteration} - 1 times with itself. For 
  instance:
  \begin{verbatim}
>>> struct = generate_binary_structure(2, 1)
>>> print struct
[[0 1 0]
 [1 1 1]
 [0 1 0]]
>>> print iterate_structure(struct, 2)
[[0 0 1 0 0]
 [0 1 1 1 0]
 [1 1 1 1 1]
 [0 1 1 1 0]
 [0 0 1 0 0]]
\end{verbatim}
  If the origin of the original structure is equal to 0, then it is also 
  equal to 0 for the iterated structure. If not, the origin must also be 
  adapted if the equivalent of the \var{iterations} erosions or dilations 
  must be achieved with the iterated structure. The adapted origin is 
  simply obtained by multiplying with the number of iterations. For 
  convenience the
  \function{iterate_structure} also returns the adapted origin if the
  \var{origin} parameter is not \constant{None}:
\begin{verbatim}
>>> print iterate_structure(struct, 2, -1)
(array([[0, 0, 1, 0, 0],
       [0, 1, 1, 1, 0],
       [1, 1, 1, 1, 1],
       [0, 1, 1, 1, 0],
       [0, 0, 1, 0, 0]], type=Bool), [-2, -2])
\end{verbatim}
\end{funcdesc}

Other morphology operations can be defined in terms of erosion and d
dilation. Following functions provide a few of these operations for 
convenience:

\begin{funcdesc}{binary_opening}{input, structure=None, iterations=1,
  output=None, origin=0} The \function{binary_opening} function implements
  binary opening of arrays of arbitrary rank with the given structuring
  element. Binary opening is equivalent to a binary erosion followed by a
  binary dilation with the same structuring element. The origin parameter
  controls the placement of the structuring element as described in section
  \ref{sec:ndimage:filter-functions}. If no structuring element is 
  provided, an element with connectivity equal to one is generated using
  \function{generate_binary_structure}. The \var{iterations} parameter 
  gives the number of erosions that is performed followed by the same 
  number of dilations.
\end{funcdesc}

\begin{funcdesc}{binary_closing}{input, structure=None, iterations=1,
  output=None, origin=0} The \function{binary_closing} function implements
  binary closing of arrays of arbitrary rank with the given structuring
  element. Binary closing is equivalent to a binary dilation followed by a
  binary erosion with the same structuring element. The origin parameter
  controls the placement of the structuring element as described in section
  \ref{sec:ndimage:filter-functions}. If no structuring element is 
  provided, an element with connectivity equal to one is generated using
  \function{generate_binary_structure}. The \var{iterations} parameter   
  gives the number of dilations that is performed followed by the same 
  number of erosions.
\end{funcdesc}

\begin{funcdesc}{binary_fill_holes}{input, structure = None, output = None, 
origin = 0} The \function{binary_fill_holes} function is used to close 
holes in objects in a binary image, where the structure defines the 
connectivity of the holes. The origin parameter controls the placement of 
the structuring element as described in section \ref{sec:ndimage:filter-
functions}. If no structuring element is provided, an element with 
connectivity equal to one is generated using 
\function{generate_binary_structure}. 
\end{funcdesc}

\begin{funcdesc}{binary_hit_or_miss}{input, structure1=None, 
  structure2=None, output=None, origin1=0, origin2=None} The 
  \function{binary_hit_or_miss}
  function implements a binary hit-or-miss transform of arrays of arbitrary
  rank with the given structuring elements.  The hit-or-miss transform is
  calculated by erosion of the input with the first structure, erosion of   
  the logical \emph{not} of the input with the second structure, followed 
  by the logical \emph{and} of these two erosions.  The origin parameters 
  control the placement of the structuring elements as described in section
  \ref{sec:ndimage:filter-functions}. If \var{origin2} equals 
  \constant{None} it is set equal to the \var{origin1} parameter. If the 
  first structuring element is not provided, a structuring element with 
  connectivity equal to one is generated using 
  \function{generate_binary_structure}, if \var{structure2} is not 
  provided, it is set equal to the logical \emph{not} of \var{structure1}.
\end{funcdesc}

\section{Grey-scale morphology}
\label{sec:ndimage:grey-morphology}

Grey-scale morphology operations are the equivalents of binary morphology
operations that operate on arrays with arbitrary values. Below we describe 
the grey-scale equivalents of erosion, dilation, opening and closing. These
operations are implemented in a similar fashion as the filters described in
section \ref{sec:ndimage:filter-functions}, and we refer to this section 
for the description of filter kernels and footprints, and the handling of 
array borders. The grey-scale morphology operations optionally take a 
\var{structure} parameter that gives the values of the structuring element. 
If this parameter is not given the structuring element is assumed to be 
flat with a value equal to zero. The shape of the structure can optionally 
be defined by the \var{footprint} parameter. If this parameter is not 
given, the structure is assumed to be rectangular, with sizes equal to the 
dimensions of the \var{structure} array, or by the \var{size} parameter if 
\var{structure} is not given. The \var{size} parameter is only used if both 
\var{structure} and \var{footprint} are not given, in which case the 
structuring element is assumed to be rectangular and flat with the 
dimensions given by \var{size}. The \var{size} parameter, if provided, must 
be a sequence of sizes or a single number in which case the size of the 
filter is assumed to be equal along each axis. The \var{footprint} 
parameter, if provided, must be an array that defines the shape of the 
kernel by its non-zero elements.

Similar to binary erosion and dilation there are operations for grey-scale
erosion and dilation:

\begin{funcdesc}{grey_erosion}{input, size=None, footprint=None, 
    structure=None, output=None, mode='reflect', cval=0.0, origin=0} The
  \function{grey_erosion} function calculates a multi-dimensional grey-
  scale erosion.
\end{funcdesc}

\begin{funcdesc}{grey_dilation}{input, size=None, footprint=None, 
    structure=None, output=None, mode='reflect', cval=0.0, origin=0} The
  \function{grey_dilation} function calculates a multi-dimensional grey-
  scale dilation.
\end{funcdesc}

Grey-scale opening and closing operations can be defined similar to their
binary counterparts:

\begin{funcdesc}{grey_opening}{input, size=None, footprint=None, 
    structure=None, output=None, mode='reflect', cval=0.0, origin=0} The
  \function{grey_opening} function implements grey-scale opening of arrays 
  of arbitrary rank. Grey-scale opening is equivalent to a grey-scale 
  erosion followed by a grey-scale dilation.
\end{funcdesc}

\begin{funcdesc}{grey_closing}{input, size=None, footprint=None, 
    structure=None, output=None, mode='reflect', cval=0.0, origin=0} The
  \function{grey_closing} function implements grey-scale closing of arrays 
  of arbitrary rank. Grey-scale opening is equivalent to a grey-scale 
  dilation followed by a grey-scale erosion.
\end{funcdesc}

\begin{funcdesc}{morphological_gradient}{input, size=None, footprint=None, 
    structure=None, output=None, mode='reflect', cval=0.0, origin=0} The
  \function{morphological_gradient} function implements a grey-scale
  morphological gradient of arrays of arbitrary rank. The grey-scale
  morphological gradient is equal to the difference of a grey-scale 
  dilation and a grey-scale erosion.
\end{funcdesc}

\begin{funcdesc}{morphological_laplace}{input, size=None, footprint=None, 
    structure=None, output=None, mode='reflect', cval=0.0, origin=0} The
  \function{morphological_laplace} function implements a grey-scale
  morphological laplace of arrays of arbitrary rank. The grey-scale
  morphological laplace is equal to the sum of a grey-scale dilation and a
  grey-scale erosion minus twice the input.
\end{funcdesc}

\begin{funcdesc}{white_tophat}{input, size=None, footprint=None, 
    structure=None, output=None, mode='reflect', cval=0.0, origin=0} The
  \function{white_tophat} function implements a white top-hat filter of 
  arrays of arbitrary rank. The white top-hat is equal to the difference of 
  the input and a grey-scale opening.
\end{funcdesc}

\begin{funcdesc}{black_tophat}{input, size=None, footprint=None, 
    structure=None, output=None, mode='reflect', cval=0.0, origin=0} The
  \function{black_tophat} function implements a black top-hat filter of 
  arrays of arbitrary rank. The black top-hat is equal to the difference of 
  the a grey-scale closing and the input.
\end{funcdesc}

\section{Distance transforms}
\label{sec:ndimage:grey-morphology}
Distance transforms are used to calculate the minimum distance from each
element of an object to the background. The following functions implement
distance transforms for three different distance metrics: Euclidean, City
Block, and Chessboard distances.

\begin{funcdesc}{distance_transform_cdt}{input, structure="chessboard",
  return_distances=True, return_indices=False, distances=None, 
  indices=None} The function \function{distance_transform_cdt} uses a 
  chamfer type algorithm to calculate the distance transform of the input, 
  by replacing each object element (defined by values larger than zero) 
  with the shortest distance to the background (all non-object elements). 
  The structure determines the type of chamfering that is done. If the 
  structure is equal to 'cityblock' a structure is generated using 
  \function{generate_binary_structure} with a squared distance equal to 1. 
  If the structure is equal to 'chessboard', a structure is generated using 
  \function{generate_binary_structure} with a squared distance equal to the 
  rank of the array. These choices correspond to the common interpretations 
  of the cityblock and the chessboard distancemetrics in two dimensions.
  
  In addition to the distance transform, the feature transform can be
  calculated. In this case the index of the closest background element is
  returned along the first axis of the result.  The \var{return_distances}, 
  and \var{return_indices} flags can be used to indicate if the distance 
  transform, the feature transform, or both must be returned.
  
  The \var{distances} and \var{indices} arguments can be used to give 
  optional output arrays that must be of the correct size and type (both
  \constant{Int32}).

  The basics of the algorithm used to implement this function is described
  in: G. Borgefors, "Distance transformations in arbitrary dimensions.",
  Computer Vision, Graphics, and Image Processing, 27:321--345, 1984.
\end{funcdesc}

\begin{funcdesc}{distance_transform_edt}{input, sampling=None,
  return_distances=True, return_indices=False, distances=None, 
  indices=None} The function \function{distance_transform_edt} calculates 
  the exact euclidean distance transform of the input, by replacing each 
  object element (defined by values larger than zero) with the shortest 
  euclidean distance to the background (all non-object elements).
  
  In addition to the distance transform, the feature transform can be
  calculated. In this case the index of the closest background element is
  returned along the first axis of the result.  The \var{return_distances}, 
  and \var{return_indices} flags can be used to indicate if the distance 
  transform, the feature transform, or both must be returned.
  
  Optionally the sampling along each axis can be given by the 
  \var{sampling} parameter which should be a sequence of length equal to 
  the input rank, or a single number in which the sampling is assumed to be 
  equal along all axes.

  The \var{distances} and \var{indices} arguments can be used to give 
  optional output arrays that must be of the correct size and type 
  (\constant{Float64} and \constant{Int32}).
  
  The algorithm used to implement this function is described in: C. R. 
  Maurer, Jr., R. Qi, and V. Raghavan, "A linear time algorithm for 
  computing exact euclidean distance transforms of binary images in 
  arbitrary dimensions. IEEE Trans. PAMI 25, 265-270, 2003.
\end{funcdesc}

\begin{funcdesc}{distance_transform_bf}{input, metric="euclidean",
  sampling=None, return_distances=True, return_indices=False, 
  distances=None, indices=None} The function  
  \function{distance_transform_bf} uses a brute-force algorithm to 
  calculate the distance transform of the input, by replacing each object 
  element (defined by values larger than zero) with the shortest distance 
  to the background (all non-object elements).  The metric must be one of 
  \constant{"euclidean"}, \constant{"cityblock"}, or 
  \constant{"chessboard"}.
  
  In addition to the distance transform, the feature transform can be
  calculated. In this case the index of the closest background element is
  returned along the first axis of the result.  The \var{return_distances}, 
  and \var{return_indices} flags can be used to indicate if the distance 
  transform, the feature transform, or both must be returned.
  
  Optionally the sampling along each axis can be given by the 
  \var{sampling} parameter which should be a sequence of length equal to 
  the input rank, or a single number in which the sampling is assumed to be 
  equal along all axes. This parameter is only used in the case of the 
  euclidean distance transform.

  The \var{distances} and \var{indices} arguments can be used to give 
  optional output arrays that must be of the correct size and type 
  (\constant{Float64} and \constant{Int32}).

  \note{This function uses a slow brute-force algorithm, the function
    \function{distance_transform_cdt} can be used to more efficiently 
    calculate cityblock and chessboard distance transforms. The function
    \function{distance_transform_edt} can be used to more efficiently 
    calculate the exact euclidean distance transform.}
\end{funcdesc}

\section{Segmentation and labeling}
Segmentation is the process of separating objects of interest from the
background. The most simple approach is probably intensity thresholding, 
which is easily done with \module{numarray} functions:
\begin{verbatim}
>>> a = array([[1,2,2,1,1,0],
...            [0,2,3,1,2,0],
...            [1,1,1,3,3,2],
...            [1,1,1,1,2,1]])
>>> print where(a > 1, 1, 0)
[[0 1 1 0 0 0]
 [0 1 1 0 1 0]
 [0 0 0 1 1 1]
 [0 0 0 0 1 0]]
\end{verbatim}

The result is a binary image, in which the individual objects still need to 
be identified and labeled.  The function \function{label} generates an 
array where each object is assigned a unique number:

\begin{funcdesc}{label}{input, structure=None, output=None}
  The \function{label} function generates an array where the objects in the
  input are labeled with an integer index. It returns a tuple consisting of 
  the array of object labels and the number of objects found, unless the
  \var{output} parameter is given, in which case only the number of objects 
  is returned. The connectivity of the objects is defined by a structuring
  element. For instance, in two dimensions using a four-connected 
  structuring element gives:
\begin{verbatim}
>>> a = array([[0,1,1,0,0,0],[0,1,1,0,1,0],[0,0,0,1,1,1],[0,0,0,0,1,0]])
>>> s = [[0, 1, 0], [1,1,1], [0,1,0]]
>>> print label(a, s)
(array([[0, 1, 1, 0, 0, 0],
       [0, 1, 1, 0, 2, 0],
       [0, 0, 0, 2, 2, 2],
       [0, 0, 0, 0, 2, 0]]), 2)
\end{verbatim}
These two objects are not connected because there is no way in which we can
place the structuring element such that it overlaps with both objects. 
However, an 8-connected structuring element results in only a single 
object:
\begin{verbatim}
>>> a = array([[0,1,1,0,0,0],[0,1,1,0,1,0],[0,0,0,1,1,1],[0,0,0,0,1,0]])
>>> s = [[1,1,1], [1,1,1], [1,1,1]]
>>> print label(a, s)[0]
[[0 1 1 0 0 0]
 [0 1 1 0 1 0]
 [0 0 0 1 1 1]
 [0 0 0 0 1 0]]
\end{verbatim}
If no structuring element is provided, one is generated by calling
\function{generate_binary_structure} (see section \ref{sec:ndimage:
morphology}) using a connectivity of one (which in 2D is the 4-connected 
structure of the first example).  The input can be of any type, any value 
not equal to zero is taken to be part of an object. This is useful if you 
need to 're-label' an array of object indices, for instance after removing 
unwanted objects. Just apply the label function again to the index array. 
For instance:
\begin{verbatim}
>>> l, n = label([1, 0, 1, 0, 1])
>>> print l
[1 0 2 0 3]
>>> l = where(l != 2, l, 0)
>>> print l
[1 0 0 0 3]
>>> print label(l)[0]
[1 0 0 0 2]
\end{verbatim}

\note{The structuring element used by \function{label} is assumed to be
  symmetric.}
\end{funcdesc}

There is a large number of other approaches for segmentation, for instance 
from an estimation of the borders of the objects that can be obtained for 
instance by derivative filters. One such an approach is watershed 
segmentation.  The function \function{watershed_ift} generates an array 
where each object is assigned a unique label, from an array that localizes 
the object borders, generated for instance by a gradient magnitude filter. 
It uses an array containing initial markers for the objects:
\begin{funcdesc}{watershed_ift}{input, markers, structure=None, 
  output=None} The \function{watershed_ift} function applies a watershed 
  from markers algorithm, using an Iterative Forest Transform, as described 
  in: P. Felkel, R.  Wegenkittl, and M. Bruckschwaiger, "Implementation and 
  Complexity of the Watershed-from-Markers Algorithm Computed as a Minimal 
  Cost Forest.", Eurographics 2001, pp. C:26-35.
  
  The inputs of this function are the array to which the transform is 
  applied, and an array of markers that designate the objects by a unique 
  label, where any non-zero value is a marker. For instance:
\begin{verbatim}
>>> input = array([[0, 0, 0, 0, 0, 0, 0],
...                [0, 1, 1, 1, 1, 1, 0],
...                [0, 1, 0, 0, 0, 1, 0],
...                [0, 1, 0, 0, 0, 1, 0],
...                [0, 1, 0, 0, 0, 1, 0],
...                [0, 1, 1, 1, 1, 1, 0],
...                [0, 0, 0, 0, 0, 0, 0]], numarray.UInt8)
>>> markers = array([[1, 0, 0, 0, 0, 0, 0],
...                  [0, 0, 0, 0, 0, 0, 0],
...                  [0, 0, 0, 0, 0, 0, 0],
...                  [0, 0, 0, 2, 0, 0, 0],
...                  [0, 0, 0, 0, 0, 0, 0],
...                  [0, 0, 0, 0, 0, 0, 0],
...                  [0, 0, 0, 0, 0, 0, 0]], numarray.Int8)
>>> print watershed_ift(input, markers)
[[1 1 1 1 1 1 1]
 [1 1 2 2 2 1 1]
 [1 2 2 2 2 2 1]
 [1 2 2 2 2 2 1]
 [1 2 2 2 2 2 1]
 [1 1 2 2 2 1 1]
 [1 1 1 1 1 1 1]]
\end{verbatim}
  
  Here two markers were used to designate an object (marker=2) and the
  background (marker=1).  The order in which these are processed is 
  arbitrary: moving the marker for the background to the lower right corner 
  of the array yields a different result:
\begin{verbatim}
>>> markers = array([[0, 0, 0, 0, 0, 0, 0],
...                  [0, 0, 0, 0, 0, 0, 0],
...                  [0, 0, 0, 0, 0, 0, 0],
...                  [0, 0, 0, 2, 0, 0, 0],
...                  [0, 0, 0, 0, 0, 0, 0],
...                  [0, 0, 0, 0, 0, 0, 0],
...                  [0, 0, 0, 0, 0, 0, 1]], numarray.Int8)
>>> print watershed_ift(input, markers)
[[1 1 1 1 1 1 1]
 [1 1 1 1 1 1 1]
 [1 1 2 2 2 1 1]
 [1 1 2 2 2 1 1]
 [1 1 2 2 2 1 1]
 [1 1 1 1 1 1 1]
 [1 1 1 1 1 1 1]]
\end{verbatim}
  The result is that the object (marker=2) is smaller because the second 
  marker was processed earlier. This may not be the desired effect if the 
  first marker was supposed to designate a background object. Therefore
  \function{watershed_ift} treats markers with a negative value explicitly 
  as background markers and processes them after the normal markers. For 
  instance, replacing the first marker by a negative marker gives a result 
  similar to the first example:
\begin{verbatim}
>>> markers = array([[0, 0, 0, 0, 0, 0, 0],
...                  [0, 0, 0, 0, 0, 0, 0],
...                  [0, 0, 0, 0, 0, 0, 0],
...                  [0, 0, 0, 2, 0, 0, 0],
...                  [0, 0, 0, 0, 0, 0, 0],
...                  [0, 0, 0, 0, 0, 0, 0],
...                  [0, 0, 0, 0, 0, 0, -1]], numarray.Int8)
>>> print watershed_ift(input, markers)
[[-1 -1 -1 -1 -1 -1 -1]
 [-1 -1  2  2  2 -1 -1]
 [-1  2  2  2  2  2 -1]
 [-1  2  2  2  2  2 -1]
 [-1  2  2  2  2  2 -1]
 [-1 -1  2  2  2 -1 -1]
 [-1 -1 -1 -1 -1 -1 -1]]
\end{verbatim}
  
  The connectivity of the objects is defined by a structuring element. If 
  no structuring element is provided, one is generated by calling
  \function{generate_binary_structure} (see section
  \ref{sec:ndimage:morphology}) using a connectivity of one (which in 2D is 
  a 4-connected structure.) For example, using an 8-connected structure 
  with the last example yields a different object:
\begin{verbatim}
>>> print watershed_ift(input, markers,
...                     structure = [[1,1,1], [1,1,1], [1,1,1]])
[[-1 -1 -1 -1 -1 -1 -1]
 [-1  2  2  2  2  2 -1]
 [-1  2  2  2  2  2 -1]
 [-1  2  2  2  2  2 -1]
 [-1  2  2  2  2  2 -1]
 [-1  2  2  2  2  2 -1]
 [-1 -1 -1 -1 -1 -1 -1]]
\end{verbatim}

\note{The implementation of \function{watershed_ift} limits the data types 
of the input to \constant{UInt8} and \constant{UInt16}.}
\end{funcdesc}

\section{Object measurements}
Given an array of labeled objects, the properties of the individual objects 
can be measured. The \function{find_objects} function can be used to 
generate a list of slices that for each object, give the smallest sub-array 
that fully contains the object:

\begin{funcdesc}{find_objects}{input, max_label=0}
  The \function{find_objects} finds all objects in a labeled array and 
  returns a list of slices that correspond to the smallest regions in the 
  array that contains the object. For instance:
\begin{verbatim}
>>> a = array([[0,1,1,0,0,0],[0,1,1,0,1,0],[0,0,0,1,1,1],[0,0,0,0,1,0]])
>>> l, n = label(a)
>>> f = find_objects(l)
>>> print a[f[0]]
[[1 1]
 [1 1]]
>>> print a[f[1]]
[[0 1 0]
 [1 1 1]
 [0 1 0]]
\end{verbatim}
\function{find_objects} returns slices for all objects, unless the
\var{max_label} parameter is larger then zero, in which case only the first
\var{max_label} objects are returned. If an index is missing in the 
\var{label} array, \constant{None} is return instead of a slice. For 
example:
\begin{verbatim}
>>> print find_objects([1, 0, 3, 4], max_label = 3)
[(slice(0, 1, None),), None, (slice(2, 3, None),)]
\end{verbatim}
\end{funcdesc}

The list of slices generated by \function{find_objects} is useful to find 
the position and dimensions of the objects in the array, but can also be 
used to perform measurements on the individual objects. Say we want to find 
the sum of the intensities of an object in image:
\begin{verbatim}
>>> image = arange(4*6,shape=(4,6))
>>> mask = array([[0,1,1,0,0,0],[0,1,1,0,1,0],[0,0,0,1,1,1],[0,0,0,0,1,0]])
>>> labels = label(mask)[0]
>>> slices = find_objects(labels)
\end{verbatim}
Then we can calculate the sum of the elements in the second object:
\begin{verbatim}
>>> print where(labels[slices[1]] == 2, image[slices[1]], 0).sum()
80
\end{verbatim}
That is however not particularly efficient, and may also be more 
complicated for other types of measurements. Therefore a few measurements 
functions are defined that accept the array of object labels and the index 
of the object to be measured. For instance calculating the sum of the 
intensities can be done by:
\begin{verbatim}
>>> print sum(image, labels, 2)
80.0
\end{verbatim}
For large arrays and small objects it is more efficient to call the 
measurement functions after slicing the array:
\begin{verbatim}
>>> print sum(image[slices[1]], labels[slices[1]], 2)
80.0
\end{verbatim}
Alternatively, we can do the measurements for a number of labels with a 
single function call, returning a list of results. For instance, to measure 
the sum of the values of the background and the second object in our 
example we give a list of labels:
\begin{verbatim}
>>> print sum(image, labels, [0, 2])
[178.0, 80.0]
\end{verbatim}

The measurement functions described below all support the \var{index} 
parameter to indicate which object(s) should be measured. The default value 
of \var{index} is \constant{None}. This indicates that all elements where 
the label is larger than zero should be treated as a single object and 
measured. Thus, in this case the \var{labels} array is treated as a mask 
defined by the elements that are larger than zero. If \var{index} is a 
number or a sequence of numbers it gives the labels of the objects that are 
measured. If \var{index} is a sequence, a list of the results is returned. 
Functions that return more than one result, return their result as a tuple 
if \var{index} is a single number, or as a tuple of lists, if \var{index} 
is a sequence.

\begin{funcdesc}{sum}{input, labels=None, index=None}
  The \function{sum} function calculates the sum of the elements of the 
  object with label(s) given by \var{index}, using the \var{labels} array 
  for the object labels. If \var{index} is \constant{None}, all elements 
  with a non-zero label value are treated as a single object. If 
  \var{label} is \constant{None}, all elements of \var{input} are used in 
  the calculation.
\end{funcdesc}

\begin{funcdesc}{mean}{input, labels=None, index=None}
  The \function{mean} function calculates the mean of the elements of the
  object with label(s) given by \var{index}, using the \var{labels} array 
  for the object labels. If \var{index} is \constant{None}, all elements 
  with a non-zero label value are treated as a single object. If 
  \var{label} is \constant{None}, all elements of \var{input} are used in 
  the calculation.
\end{funcdesc}

\begin{funcdesc}{variance}{input, labels=None, index=None}
  The \function{variance} function calculates the variance of the elements 
  of the object with label(s) given by \var{index}, using the \var{labels} 
  array for the object labels. If \var{index} is \constant{None}, all 
  elements with a non-zero label value are treated as a single object. If 
  \var{label} is \constant{None}, all elements of \var{input} are used in 
  the calculation.
\end{funcdesc}

\begin{funcdesc}{standard_deviation}{input, labels=None, index=None}
  The \function{standard_deviation} function calculates the standard 
  deviation of the elements of the object with label(s) given by 
  \var{index}, using the \var{labels} array for the object labels. If 
  \var{index} is \constant{None}, all elements with a non-zero label value 
  are treated as a single object. If \var{label} is \constant{None}, all 
  elements of \var{input} are used in the calculation.
\end{funcdesc}

\begin{funcdesc}{minimum}{input, labels=None, index=None}
  The \function{minimum} function calculates the minimum of the elements of 
  the object with label(s) given by \var{index}, using the \var{labels} 
  array for the object labels. If \var{index} is \constant{None}, all 
  elements with a non-zero label value are treated as a single object. If 
  \var{label} is \constant{None}, all elements of \var{input} are used in 
  the calculation.
\end{funcdesc}

\begin{funcdesc}{maximum}{input, labels=None, index=None}
  The \function{maximum} function calculates the maximum of the elements of 
  the object with label(s) given by \var{index}, using the \var{labels} 
  array for the object labels. If \var{index} is \constant{None}, all 
  elements with a non-zero label value are treated as a single object. If 
  \var{label} is \constant{None}, all elements of \var{input} are used in 
  the calculation.
\end{funcdesc}

\begin{funcdesc}{minimum_position}{input, labels=None, index=None}
  The \function{minimum_position} function calculates the position of the
  minimum of the elements of the object with label(s) given by \var{index},
  using the \var{labels} array for the object labels. If \var{index} is
  \constant{None}, all elements with a non-zero label value are treated as 
  a single object. If \var{label} is \constant{None}, all elements of 
  \var{input} are used in the calculation.
\end{funcdesc}

\begin{funcdesc}{maximum_position}{input, labels=None, index=None}
  The \function{maximum_position} function calculates the position of the
  maximum of the elements of the object with label(s) given by \var{index},
  using the \var{labels} array for the object labels. If \var{index} is
  \constant{None}, all elements with a non-zero label value are treated as 
  a single object. If \var{label} is \constant{None}, all elements of 
  \var{input} are used in the calculation.
\end{funcdesc}

\begin{funcdesc}{extrema}{input, labels=None, index=None}
  The \function{extrema} function calculates the minimum, the maximum, and 
  their positions, of the elements of the object with label(s) given by 
  \var{index}, using the \var{labels} array for the object labels. If 
  \var{index} is \constant{None}, all elements with a non-zero label value 
  are treated as a single object. If \var{label} is \constant{None}, all 
  elements of \var{input} are used in the calculation. The result is a 
  tuple giving the minimum, the maximum, the position of the mininum and 
  the postition of the maximum. The result is the same as a tuple formed by 
  the results of the functions \function{minimum}, \function{maximum}, 
  \function{minimum_position}, and \function{maximum_position} that are 
  described above.
\end{funcdesc}

\begin{funcdesc}{center_of_mass}{input, labels=None, index=None}
  The \function{center_of_mass} function calculates the center of mass of 
  the of the object with label(s) given by \var{index}, using the 
  \var{labels} array for the object labels. If \var{index} is 
  \constant{None}, all elements with a non-zero label value are treated as 
  a single object. If \var{label} is \constant{None}, all elements of 
  \var{input} are used in the calculation.
\end{funcdesc}

\begin{funcdesc}{histogram}{input, min, max, bins, labels=None, index=None}
  The \function{histogram} function calculates a histogram of 
  the of the object with label(s) given by \var{index}, using the 
  \var{labels} array for the object labels. If \var{index} is 
  \constant{None}, all elements with a non-zero label value are treated as 
  a single object. If \var{label} is \constant{None}, all elements of 
  \var{input} are used in the calculation. Histograms are defined by their 
  minimum (\var{min}), maximum (\var{max}) and the number of bins 
  (\var{bins}). They are returned as one-dimensional arrays of type Int32. 
\end{funcdesc}

\section{Extending \module{nd\_image} in C}
\label{sec:ndimage:ccallbacks}
\subsection{C callback functions}
A few functions in the \module{numarray.nd\_image} take a call-back 
argument. This can be a python function, but also a CObject containing a 
pointer to a C function. To use this feature, you must write your own C 
extension that defines the function, and define a python function that 
returns a CObject containing a pointer to this function.

An example of a function that supports this is 
\function{geometric_transform} (see section \ref{sec:ndimage:
interpolation}). You can pass it a python callable object that defines a 
mapping from all output coordinates to corresponding coordinates in the 
input array. This mapping function can also be a C function, which 
generally will be much more efficient, since the overhead of calling a
python function at each element is avoided.

For example to implement a simple shift function we define the following 
function:
\begin{verbatim}
static int 
_shift_function(int *output_coordinates, double* input_coordinates,
                int output_rank, int input_rank, void *callback_data)
{
  int ii;
  /* get the shift from the callback data pointer: */
  double shift = *(double*)callback_data;
  /* calculate the coordinates: */
  for(ii = 0; ii < irank; ii++)
    icoor[ii] = ocoor[ii] - shift;
  /* return OK status: */
  return 1;
}
\end{verbatim}
This function is called at every element of the output array, passing the 
current coordinates in the \var{output_coordinates} array. On return, the 
\var{input_coordinates} array must contain the coordinates at which the 
input is interpolated. The ranks of the input and output array are passed 
through \var{output_rank} and \var{input_rank}. The value of the shift is 
passed through the \var{callback_data} argument, which is a pointer to 
void. The function returns an error status, in this case always 1, since no 
error can occur.

A pointer to this function and a pointer to the shift value must be passed 
to \function{geometric_transform}. Both are passed by a single CObject 
which is created by the following python extension function:
\begin{verbatim}
static PyObject *
py_shift_function(PyObject *obj, PyObject *args)
{
  double shift = 0.0;
  if (!PyArg_ParseTuple(args, "d", &shift)) {
    PyErr_SetString(PyExc_RuntimeError, "invalid parameters");
    return NULL;
  } else {
    /* assign the shift to a dynamically allocated location: */
    double *cdata = (double*)malloc(sizeof(double));
    *cdata = shift;
    /* wrap function and callback_data in a CObject: */
    return PyCObject_FromVoidPtrAndDesc(_shift_function, cdata,
                                        _destructor);
  }
}
\end{verbatim}
The value of the shift is obtained and then assigned to a dynamically 
allocated memory location. Both this data pointer and the function pointer 
are then wrapped in a CObject, which is returned. Additionally, a pointer 
to a destructor function is given, that will free the memory we allocated 
for the shift value when the CObject is destroyed. This destructor is very 
simple:
\begin{verbatim}
static void
_destructor(void* cobject, void *cdata)
{
  if (cdata)
    free(cdata);
}
\end{verbatim}
To use these functions, an extension module is build:
\begin{verbatim}
static PyMethodDef methods[] = {
  {"shift_function", (PyCFunction)py_shift_function, METH_VARARGS, ""},
  {NULL, NULL, 0, NULL}
};

void
initexample(void)
{
  Py_InitModule("example", methods);
}
\end{verbatim}
This extension can then be used in Python, for example:
\begin{verbatim}
>>> import example
>>> array = arange(12, shape=(4,3), type = Float64)
>>> fnc = example.shift_function(0.5)
>>> print geometric_transform(array, fnc)
[[ 0.      0.      0.    ]
 [ 0.      1.3625  2.7375]
 [ 0.      4.8125  6.1875]
 [ 0.      8.2625  9.6375]]
\end{verbatim}

C Callback functions for use with \module{nd\_image} functions must all be 
written according to this scheme. The next section lists the 
\module{nd\_image} functions that acccept a C callback function and gives 
the prototype of the callback function.

\subsection{Functions that support C callback functions}
The \module{nd\_image} functions that support C callback functions are 
described here. Obviously, the prototype of the function that is provided 
to these functions must match exactly that what they expect. Therefore we 
give here the prototypes of the callback functions. All these callback 
functions accept a void \var{callback_data} pointer that must be wrapped in 
a CObject using the Python \cfunction{PyCObject_FromVoidPtrAndDesc} 
function, which can also accept a pointer to a destructor function to free 
any memory allocated for \var{callback_data}. If \var{callback_data} is not 
needed, \cfunction{PyCObject_FromVoidPtr} may be used instead. The callback 
functions must return an integer error status that is equal to zero if 
something went wrong, or 1 otherwise. If an error occurs, you should 
normally set the python error status with an informative message before 
returning, otherwise, a default error message is set by the calling 
function.

The function \function{generic_filter} (see section 
\ref{sec:ndimage:genericfilters}) accepts a callback function with the 
following prototype:
\begin{cfuncdesc}{int}{FilterFunction}{double *buffer, int filter_size,
double *return_value, void *callback_data} The calling function iterates 
over the elements of the input and output arrays, calling the callback 
function at each element. The elements within the footprint of the filter 
at the current element are passed through the \var{buffer} parameter, and 
the number of elements within the footprint through \var{filter_size}. The 
calculated valued should be returned in the \var{return_value} argument.
\end{cfuncdesc}

The function \function{generic_filter1d} (see section 
\ref{sec:ndimage:genericfilters}) accepts a callback function with the 
following prototype: 
\begin{cfuncdesc}{int}{FilterFunction1D}{double *input_line, int 
input_length, double *output_line, int output_length, void *callback_data} 
The calling function iterates over the lines of the input and output 
arrays, calling the callback function at each line. The current line is 
extended according to the border conditions set by the calling function, 
and the result is copied into the array that is passed through the 
\var{input_line} array. The length of the input line (after extension) is 
passed through \var{input_length}. The callback function should apply the 
1D filter and store the result in the array passed through 
\var{output_line}. The length of the output line is passed through 
\var{output_length}.
\end{cfuncdesc}

The function \function{geometric_transform} (see section 
\ref{sec:ndimage:interpolation}) expects a function with the following 
prototype: 
\begin{cfuncdesc}{int}{MapCoordinates}{int *output_coordinates, 
double* input_coordinates, int output_rank, int input_rank, 
void *callback_data} The calling function iterates over the elements of the 
output array, calling the callback function at each element. The 
coordinates of the current output element are passed through 
\var{output_coordinates}. The callback function must return the coordinates 
at which the input must be interpolated in \var{input_coordinates}. The 
rank of the input and output arrays are given by \var{input_rank} and 
\var{output_rank} respectively.
\end{cfuncdesc}


\chapter{Memory Mapping}
\label{cha:memmap}
\declaremodule{extension}{numarray.memmap}
\index{character array}
\index{string array}

\section{Introduction}
\label{sec:memmap-intro}

\code{numarray} provides support for the creation of arrays which are
mapped directly onto files with the \code{numarray.memmap} module.
Much of \code{numarray}'s design, the ability to handle misaligned and
byteswapped arrays for instance, was motivated by the desire to create
arrays from portable files which contain binary array data.  One
advantage of memory mapping is efficient random access to small
regions of a large file: only the region of the mapped file which is
actually used in array operations needs to be paged into system
memory; the rest of the file remains unread and unwritten.

\code{numarray.memmap} is pure Python and is layered on top of
Python's \code{mmap} module.  The basic idea behind \code{numarray}'s
memory mapping is to create a ``buffer'' referring to a region in a
mapped file and to use it as the data store for an array.  The
\code{numarray.memmap} module contains two classes, one which
corresponds to an entire mapped file (\class{Memmap}) and one which
corresponds to a contiguous region within a file
(\class{MemmapSlice}).  \class{MemmapSlice} objects have these
properties:

\begin{itemize}
\item MemmapSlices can be used as NumArray buffers.
\item MemmapSlices are non-overlapping.
\item MemmapSlices are resizable.
\item Changing the size of a MemmapSlice changes the parent Memmap.
\end{itemize}

\section{Opening a Memmap}
\label{sec:memmap-open}

You can create a \class{Memmap} object by calling the \function{open} function,
as in:

\begin{verbatim}
>>> m = open("memmap.tst","w+",len=48)
>>> m
<Memmap on file 'memmap.tst' with mode='w+', length=48, 0 slices>
\end{verbatim}

Here, the file ``memmap.tst'' is created/truncated to a length of 48
bytes and used to construct a Memmap object in write mode whose
contents are considered undefined.  

\section{Slicing a Memmap}
\label{sec:memmap-slicing}

Once opened, a \class{Memmap} object can be sliced into regions.

\begin{verbatim}
# Slice m into the buffers "n" and "p" which will correspond to numarray:

>>> n = m[0:16]
>>> n
<MemmapSlice of length:16 writable>

>>> p = m[24:48]
>>> p
<MemmapSlice of length:24 writable>
\end{verbatim}

NOTE: You cannot make \emph{overlapping} slices of a Memmap:

\begin{verbatim}
>>> q = m[20:28]
Traceback (most recent call last):
...
IndexError: Slice overlaps prior slice of same file.
\end{verbatim}

Deletion of a slice is possible once all other references to it are
forgotten, e.g. all arrays that used it have themselves been deleted.
Deletion of a slice of a Memmap "un-registers" the slice, making that
region of the Memmap available for reallocation.  Delete directly from
the Memmap without referring to the MemmapSlice:

\begin{verbatim}
>>> m = Memmap("memmap.tst",mode="w+",len=100)
>>> m1 = m[0:50]
>>> del m[0:50]      # note: delete from m, not m1
>>> m2 = m[0:70]
\end{verbatim}

Note that since the region of m1 was deleted, there is no overlap when
m2 is created.  However, deleting the region of m1 has invalidated it:

\begin{verbatim}
>>> m1
Traceback (most recent call last):
...
RuntimeError: A deleted MemmapSlice has been used.
\end{verbatim}

Don't mix operations on a Memmap which modify its data or file
structure with slice deletions.  In this case, the status of the
modifications is undefined; the underlying map may or may not reflect
the modifications after the deletion.

\section{Creating an array from a MemmapSlice}
\label{sec:memmap-array-construction}

Arrays are created from \class{MemmapSlice}s simply by specifying the
slice as the \var{buffer} parameter of the array.  Since the slice is
essentially just a byte string, it's necessary to specify the
\var{type} of the binary data as well.

\begin{verbatim}
>>> a = num.NumArray(buffer=n, shape=(len(n)/4,), type=num.Int32)
>>> a[:] = 0  # Since the initial contents of 'n' are undefined.
>>> a += 1
array([1, 1, 1, 1], type=Int32)
\end{verbatim}

\section{Resizing a MemmapSlice}
\label{sec:memmap-slice}

Arrays based on \class{MemmapSlice} objects are resizable.  As soon as
they're resized, slices become un-mapped or ``free floating''.
Resizing a slice affects the parent \class{Memmap}.

\begin{verbatim}
>>> a.resize(6)
array([1, 1, 1, 1, 1, 1], type=Int32)
\end{verbatim}

\section{Forcing file updates and closing the Memmap}
\label{sec:memmap-flushing-closing}

After doing slice resizes or inserting new slices, call
\function{flush} to synchronize the underlying map file with any free
floating slices.  This explicit step is required to avoid implicitly
shuffling huge amounts of file space for every \function{resize} or
\function{insert}.  After calling \function{flush}, all slices are
once again memory mapped rather than free floating.

\begin{verbatim}
>>> m.flush()
\end{verbatim}

A related concept is ``syncing'' which applies even to arrays which
have not been resized.  Since memory maps don't guarantee when the
underlying file will be updated with the values you have written to
the map, call \function{sync} when you want to be sure your changes
are on disk.  This is similar to syncing a UNIX file system.  Note
that \function{sync} does not consolidate the mapfile with any free
floating slices (newly inserted or resized), it merely ensures that
mapped slices whose contents have been altered are written to disk.

\begin{verbatim}
>>> m.sync()
\end{verbatim}

Now "a" and "b" are both memory mapped on "memmap.tst" again.

When you're done with the memory map and numarray, call
\function{close}. \function{close} calls \function{flush} which will
consolidate resized or inserted slices as necessary.

\begin{verbatim}
>>> m.close()
\end{verbatim}

It is an error to use "m" (or slices of m) any further after closing
it.

\section{numarray.memmap functions}
\label{sec:memmap-functions}

\begin{funcdesc}{open}{filename, mode='r+', len=None}
\label{func:memmap-open}
\function{open} opens a \class{Memmap} object on the file
\var{filename} with the specified \var{mode}.  Available \var{mode}
values include 'readonly' ('r'), 'copyonwrite' ('c'), 'readwrite'
('r+'), and 'write' ('w+'), all but the last of which have contents
defined by the file.

Neither mode 'r' nor mode 'c' can affect the underlying map file.
Readonly maps impose no requirement on system swap space and raise
exceptions when their contents are modified.  Copy-on-write maps
require system swap space corresponding to their size, but have
modifiable pages which become reassociated with system swap as they
are changed leaving the original map file unaltered.  Insufficient
swap space can prevent the creation of a copy-on-write memory map.
Modifications to readwrite memory maps are eventually reflected onto
the map file;  see flushing and syncing.
\end{funcdesc}
   
\begin{funcdesc}{close}{map}
\label{func:memmap-close}
\function{close} closes the \class{Memmap} object specified by
\var{map}.
\end{funcdesc}
   
\section{Memmap methods}
\label{sec:memmap-methods}
A Memmap object represents an entire mapped file and is sliced to
create objects which can be used as array buffers.  It has these
public methods:

\begin{methoddesc}[Memmap]{close}{}
  \function{close} unites the \class{Memmap} and any RAM based slices with
  its underlying file and removes the mapping and all references to
  its slices.  Once a \class{Memmap} has been closed, all of its
  slices become unusable.
\end{methoddesc}

\begin{methoddesc}[Memmap]{find}{string, offset=0}
  find(string, offset=0) returns the first index at which string
  is found, or -1 on failure.
  \begin{verbatim}
    >>> _open("memmap.tst","w+").write("this is a test")
    >>> Memmap("memmap.tst",len=14).find("is")
    2
    >>> Memmap("memmap.tst",len=14).find("is", 3)
    5
    >>> _open("memmap.tst","w+").write("x")
    >>> Memmap("memmap.tst",len=1).find("is")
    -1
  \end{verbatim}
\end{methoddesc}

\begin{methoddesc}[Memmap]{insert}{offset, size=None, buffer=None}
  \function{insert} places a new slice at the specified \var{offset} of
  the \class{Memmap}.  \var{size} indicates the length in bytes of the
  inserted slice when \var{buffer} is not specified.  If \function{buffer}
  is specified, it should refer to an existing memory object created
  using \code{numarray.memory.new_memory} and \function{size} should not
  be specified.
\end{methoddesc}

\begin{methoddesc}[Memmap]{flush}{}
  \function{flush} writes a \class{Memmap} out to its associated file,
  reconciling any inserted or resized slices by backing them directly
  on the map file rather than a system swap file.  \function{flush}
  only makes sense for write and readwrite memory maps.
\end{methoddesc}

\begin{methoddesc}[Memmap]{sync}{}
  \function{sync} forces slices which are backed on the map file to be
  immediately written to disk.  Resized or newly inserted slices are
  not affected.  \function{sync} only makes sense for write and
  readwrite memory maps.
\end{methoddesc}

\section{MemmapSlice methods}
\label{sec:memmap-methods}
A \class{MemmapSlice} object represents a subregion of a
\class{Memmap} and has these public methods:

\begin{methoddesc}[MemmapSlice]{__buffer__}{}
  Returns an object which supports the Python buffer protocol and
  represents this slice.  The Python buffer protocol enables a C
  function to obtain the pointer and size corresponding to the data
  region of the slice.
\end{methoddesc}

\begin{methoddesc}[MemmapSlice]{resize}{newsize}
  \function{resize} expands or contracts this slice to the specified
  \var{newsize}.
\end{methoddesc}

%% mode: LaTeX
%% mode: auto-fill
%% fill-column: 79
%% indent-tabs-mode: nil
%% ispell-dictionary: "american"
%% reftex-fref-is-default: nil
%% TeX-auto-save: t
%% TeX-command-default: "pdfeLaTeX"
%% TeX-master: "numarray"
%% TeX-parse-self: t
%% End:


\appendix
\part*{Appendix}
%begin{latexonly}
\makeatletter
\py@reset
\makeatother
%end{latexonly}

\chapter{Glossary}
\label{cha:glossary}

\begin{quote} 
   This chapter provides a glossary of terms.\footnote{Please let us know of
      any additions to this list which you feel would be helpful.}
\end{quote}

\begin{description}
\item[array] An array refers to the Python object type defined by the NumPy
   extensions to store and manipulate numbers efficiently.
\item[byteswapped]
\item[discontiguous]  
\item[misaligned] 
\item[misbehaved array] A \class{\numarray} which is byteswapped, misaligned,
   or discontiguous.
\item[rank] The rank of an array is the number of dimensions it has, or the
   number of integers in its shape tuple.
\item[shape] Array objects have an attribute called shape which is necessarily
   a tuple. An array with an empty tuple shape is treated like a scalar (it
   holds one element).
\item[ufunc] A callable object which performs operations on all of the elements
   of its arguments, which can be lists, tuples, or arrays. Many ufuncs are
   defined in the umath module.
\item[universal function] See ufunc.
\end{description}
 


%% Local Variables:
%% mode: LaTeX
%% mode: auto-fill
%% fill-column: 79
%% indent-tabs-mode: nil
%% ispell-dictionary: "american"
%% reftex-fref-is-default: nil
%% TeX-auto-save: t
%% TeX-command-default: "pdfeLaTeX"
%% TeX-master: "numarray"
%% TeX-parse-self: t
%% End:

% Complete documentation on the extended LaTeX markup used for Python
% documentation is available in ``Documenting Python'', which is part
% of the standard documentation for Python.  It may be found online
% at:
%
%     http://www.python.org/doc/current/doc/doc.html

\documentclass[hyperref]{manual}
\pagestyle{plain}

% latex2html doesn't know [T1]{fontenc}, so we cannot use that:(

\usepackage{amsmath}
\usepackage[latin1]{inputenc}
\usepackage{textcomp}


% The commands of this document do not reset module names at section level
% (nor at chapter level).
% --> You have to do that manually when a new module starts!
%     (use \py@reset)
%begin{latexonly}
\makeatletter
\renewcommand{\section}{\@startsection{section}{1}{\z@}%
   {-3.5ex \@plus -1ex \@minus -.2ex}%
   {2.3ex \@plus.2ex}%
   {\reset@font\Large\py@HeaderFamily}}
\makeatother
%end{latexonly}


% additional mathematical functions
\DeclareMathOperator{\abs}{abs}

% provide a cross-linking command for the index
%begin{latexonly}
\newcommand*\see[2]{\protect\seename #1}
\newcommand*{\seename}{$\to$}
%end{latexonly}


% some convenience declarations
\newcommand{\numarray}{numarray}
\newcommand{\Numarray}{Numarray}  % Only beginning of sentence, otherwise use \numarray
\newcommand{\NUMARRAY}{NumArray}
\newcommand{\numpy}{Numeric}
\newcommand{\NUMPY}{Numerical Python}
\newcommand{\python}{Python}


% mark internal comments
% for any published version switch to the second (empty) definition of the macro!
% \newcommand{\remark}[1]{(\textbf{Note to authors: #1})}
\newcommand{\remark}[1]{}


\title{numarray\\User's Manual}

\author{Perry Greenfield \\
   Todd Miller \\
   Rick White \\
   J.C. Hsu \\
   Paul Barrett \\
   Jochen K�pper \\
   Peter J. Verveer \\[1ex]
   Previously authored by: \\
   David Ascher \\
   Paul F. Dubois \\
   Konrad Hinsen \\
   Jim Hugunin \\
   Travis Oliphant \\[1ex]
   with contributions from the Numerical Python community}

\authoraddress{Space Telescope Science Institute, 3700 San Martin Dr,
   Baltimore, MD 21218 \\ UCRL-MA-128569}

% I use date to indicate the manual-updates,
% release below gives the matching software version.
\date{November 2, 2005}        % update before release!
                                % Use an explicit date so that reformatting
                                % doesn't cause a new date to be used.  Setting
                                % the date to \today can be used during draft
                                % stages to make it easier to handle versions.

\release{1.5}                 % (software) release version;
\setshortversion{1.5}         % this is used to define the \version macro

\makeindex                      % tell \index to actually write the .idx file



\begin{document}

\maketitle

% This makes the contents more accessible from the front page of the HTML.
\ifhtml
\part*{General}
\chapter*{Front Matter}
\label{front}
\fi

\section*{Legal Notice}
\label{sec:legal-notice}

Please see file LICENSE.txt in the source distribution.  

This open source project has been contributed to by many people, including
personnel of the Lawrence Livermore National Laboratory, Livermore, CA, USA.
The following notice covers those contributions, including contributions to
this this manual.

Copyright (c) 1999, 2000, 2001.  The Regents of the University of California.
All rights reserved.

Permission to use, copy, modify, and distribute this software for any purpose
without fee is hereby granted, provided that this entire notice is included in
all copies of any software which is or includes a copy or modification of this
software and in all copies of the supporting documentation for such software.

This work was produced at the University of California, Lawrence Livermore
National Laboratory under contract no. W-7405-ENG-48 between the U.S.
Department of Energy and The Regents of the University of California for the
operation of UC LLNL.



\subsection*{Special license for package numarray.ma}
\label{sec:license-numarray.ma}


The package \module{numarray.ma} was written by Paul Dubois, Lawrence Livermore
National Laboratory, Livermore, CA, USA.

Copyright (c) 1999, 2000. The Regents of the University of California. All
rights reserved.

Permission to use, copy, modify, and distribute this software for any purpose
without fee is hereby granted, provided that this entire notice is included in
all copies of any software which is or includes a copy or modification of this
software and in all copies of the supporting documentation for such software.

This work was produced at the University of California, Lawrence Livermore
National Laboratory under contract no. W-7405-ENG-48 between the U.S.
Department of Energy and The Regents of the University of California for the
operation of UC LLNL.



\subsection*{Disclaimer}

This software was prepared as an account of work sponsored by an agency of the
United States Government. Neither the United States Government nor the
University of California nor any of their employees, makes any warranty,
express or implied, or assumes any liability or responsibility for the
accuracy, completeness, or usefulness of any information, apparatus, product,
or process disclosed, or represents that its use would not infringe
privately-owned rights. Reference herein to any specific commercial products,
process, or service by trade name, trademark, manufacturer, or otherwise, does
not necessarily constitute or imply its endorsement, recommendation, or
favoring by the United States Government or the University of California. The
views and opinions of authors expressed herein do not necessarily state or
reflect those of the United States Government or the University of California,
and shall not be used for advertising or product endorsement purposes.




%% Local Variables:
%% mode: LaTeX
%% mode: auto-fill
%% fill-column: 79
%% indent-tabs-mode: nil
%% ispell-dictionary: "american"
%% reftex-fref-is-default: nil
%% TeX-auto-save: t
%% TeX-command-default: "pdfeLaTeX"
%% TeX-master: "numarray"
%% TeX-parse-self: t
%% End:
  \cleardoublepage


\tableofcontents


\part{Numerical Python}

\NUMARRAY{} (``\numarray{}'') adds a fast multidimensional array facility to
Python.  This part contains all you need to know about ``\numarray{}'' arrays
and the functions that operate upon them.

\label{part:numerical-python}

\declaremodule{extension}{numarray}
\moduleauthor{The numarray team}{numpy-discussion@lists.sourceforge.net}
\modulesynopsis{Numerics}

\chapter{Introduction}
\label{cha:introduction}

\begin{quote}
   This chapter introduces the numarray Python extension and outlines the rest
   of the document.
\end{quote}

Numarray is a set of extensions to the Python programming language which allows
Python programmers to efficiently manipulate large sets of objects organized in
grid-like fashion. These sets of objects are called arrays, and they can have
any number of dimensions. One-dimensional arrays are similar to standard Python
sequences, and two-dimensional arrays are similar to matrices from linear
algebra. Note that one-dimensional arrays are also different from any other
Python sequence, and that two-dimensional matrices are also different from the
matrices of linear algebra. One significant difference is that numarray objects
must contain elements of homogeneous type, while standard Python sequences can
contain elements of mixed type. Two-dimensional arrays differ from matrices
primarily in the way multiplication is performed; 2-D arrays are multiplied
element-by-element.

This is a reimplementation of the earlier Numeric module (aka numpy). For the
most part, the syntax of numarray is identical to that of Numeric, although
there are significant differences. The differences are primarily in new
features. For Python 2.2 and later, the syntax is completely backwards
compatible. See the High-Level Overview (chapter \ref{cha:high-level-overview})
for incompatibilities for earlier versions of Python. The reasons for rewriting
Numeric and a comparison between Numeric and numarray are also described in
chapter \ref{cha:high-level-overview}. Portions of the present document are
almost word-for-word identical to the Numeric manual. It has been updated to
reflect the syntax and behavior of numarray, and there is a new section
(~\ref{sec:diff-numarray-numpy}) on differences between Numeric and numarray.

Why are these extensions needed? The core reason is a very prosaic one:
manipulating a set of a million numbers in Python with the
standard data structures such as lists, tuples or classes is much too slow and
uses too much space. A more subtle
reason for these extensions, however, is that the kinds of operations that
programmers typically want to do on arrays, while sometimes very complex, can
often be decomposed into a set of fairly standard operations. This
decomposition has been similarly developed in many array languages. In some
ways, numarray is simply the application of this experience to the Python
language.  Thus many of the operations described in numarray work the way they
do because experience has shown that way to be a good one, in a variety of
contexts. The languages which were used to guide the development of numarray
include the infamous APL family of languages, Basis, MATLAB, FORTRAN, S and S+,
and others.  This heritage will be obvious to users of numarray who already
have experience with these other languages.  This manual, however, does not
assume any such background, and all that is expected of the reader is a
reasonable working knowledge of the standard Python language.

This document is the ``official'' documentation for numarray. It is both a
tutorial and the most authoritative source of information about numarray with
the exception of the source code. The tutorial material will walk you through a
set of manipulations of simple, small arrays of numbers. This choice was made
because:
\begin{itemize}
\item A concrete data set makes explaining the behavior of some functions much
   easier to motivate than simply talking about abstract operations on abstract
   data sets.
\item Every reader will have at least an intuition as to the meaning of the
   data and organization of image files.  \remark{These ``image files'' are not
      mentioned anywhere before, and not really used later...?}
\end{itemize}
All users of numarray, whether interested in image processing or not, are
encouraged to follow the tutorial with a working numarray installed,
testing the examples, and more importantly, transferring the
understanding gained by working on arrays to their specific domain. The best
way to learn is by doing --- the aim of this tutorial is to guide you along
this "doing."

This manual contains:
\begin{description}
\item[Installing numarray] Chapter \ref{cha:installation} provides information
   on testing Python, numarray, and compiling and installing numarray if
   necessary.
\item[High-Level Overview] Chapter \ref{cha:high-level-overview} gives a
   high-level overview of the components of the numarray system as a whole.
\item[Array Basics] Chapter \ref{cha:array-basics} provides a detailed
   step-by-step introduction to the most important aspect of numarray, the
   multidimensional array objects.
\item[Ufuncs] Chapter \ref{cha:ufuncs} provides information on universal
   functions, the mathematical functions which operate on arrays and other
   sequences elementwise.
\item[Pseudo Indices] Chapter \ref{cha:pseudo-indices} covers syntax for some
   special indexing operators.
\item[Array Functions] Chapter \ref{cha:array-functions} is a catalog of each
   of the utility functions which allow easy algorithmic processing of arrays.
\item[Array Methods] Chapter \ref{cha:array-methods} discusses the methods of
   array objects.
\item[Array Attributes] Chapter \ref{cha:array-attributes} presents the
   attributes of array objects.
\item[Character Array] Chapter \ref{cha:character-array} describes the
  \code{numarray.strings} module that provides support for arrays of fixed
  length strings.
\item[Record Array] Chapter \ref{cha:record-array} describes the
   \code{numarray.records} module that supports arrays of fixed length records
   of string or numerical data.
\item[Object Array] Chapter \ref{cha:object-array} describes the
   \code{numarray.objects} module that supports arrays of Python objects.
\item[C extension API] Chapter \ref{cha:C-API} describes the C-APIs provided
   for \module{numarray} based extension modules.
\item[Convolution] Chapter \ref{cha:convolve} describes the
   \module{numarray.convolve} module for computing one-D and two-D convolutions
   and correlations of \class{numarray} objects.
\item[Fast-Fourier-Transform] Chapter \ref{cha:fft} describes the
   \module{numarray.fft} module for computing Fast-Fourier-Transforms
   (FFT) and Inverse FFTs over \class{numarray} objects in one- or
   two-dimensional manner.  Ported from Numeric.
\item[Linear Algebra] Chapter \ref{cha:linear-algebra} describes the
   \module{numarray.linear_algebra} module which provides a simple
   interface to some commonly used linear algebra routines; 
   \program{LAPACK}.   Ported from Numeric.
\item[Masked Arrays] Chapter \ref{cha:masked-arrays} describes the
   \module{numarray.ma} module which supports Masked Arrays: arrays which
   potentially have missing or invalid elements.  Ported from Numeric.
\item[Random Numbers] Chapter \ref{cha:random-array} describes the
   \module{numarray.random_array} module which supports generation of arrays of
   random numbers.  Ported from Numeric.
\item[Multidimentional image analysis functions] Chapter \ref{cha:ndimage}
   describes the \module{numarray.ndimage} module which provides
   functions for multidimensional image analysis such as filtering,
   morphology or interpolation.
\item[Glossary] Appendix \ref{cha:glossary} gives a glossary of terms.
\end{description}


\section{Where to get information and code}

Numarray and its documentation are available at SourceForge
(\ulink{sourceforge.net}{http://sourceforge.net}; SourceForge addresses can
also be abbreviated as \ulink{sf.net}{http://sf.net}). The main web site is:
\url{http://numpy.sourceforge.net}. Downloads, bug reports, a patch facility,
and releases are at the main project page, reachable from the above site or
directly at: \url{http://sourceforge.net/projects/numpy} (see Numarray under
"Latest File Releases").  The Python web site is \url{http://www.python.org}.
For up-to-date status on compatible modules available for numarray, please 
check \url{http://www.stsci.edu/resources/software_hardware/numarray/}.

NOTE: because numarray shares the numpy Source Forge project with Numeric and
Numeric3, there are dedicated Source Forge ``Trackers'' for numarray, .e.g.
``Numarray Bugs'' rather than just ``Bugs''.  When submitting bug reports,
patches, or requests, please look for the numarray version of the tracker under
the top level menu item ``Tracker'', nominally here:
\url{http://sourceforge.net/tracker/?group_id=1369}.

\section{Acknowledgments}

Numerical Python was the outgrowth of a long collaborative design process
carried out by the Matrix SIG of the Python Software Activity (PSA). Jim
Hugunin, while a graduate student at MIT, wrote most of the code and initial
documentation. When Jim joined CNRI and began working on JPython, he didn't
have the time to maintain Numerical Python so Paul Dubois at LLNL agreed to
become the maintainer of Numerical Python. David Ascher, working as a
consultant to LLNL, wrote most of the Numerical Python version of this
document, incorporating contributions from Konrad Hinsen and Travis Oliphant,
both of whom are major contributors to Numerical Python.  The reimplementation
of Numeric as numarray was done primarily by Perry Greenfield, Todd Miller, and
Rick White, with some assistance from J.C. Hsu and Paul Barrett. Although
numarray is almost a completely new implementation, it owes a great deal to the
ideas, interface and behavior expressed in the Numeric implementation. It is
not an overstatement to say that the existence of Numeric made the
implementation of numarray far, far easier that it would otherwise have been.
Since the source for the original Numeric module was moved to SourceForge, the
numarray user community has become a significant part of the process.
Numeric/numarray illustrates the power of the open source software concept.
Please send comments and corrections to this manual to
\ulink{perry@stsci.edu}{mailto:perry@stsci.edu}, or to Perry Greenfield, 3700
San Martin Dr, Baltimore, MD 21218, U.S.A.




%% Local Variables:
%% mode: LaTeX
%% mode: auto-fill
%% fill-column: 79
%% indent-tabs-mode: nil
%% ispell-dictionary: "american"
%% reftex-fref-is-default: nil
%% TeX-auto-save: t
%% TeX-command-default: "pdfeLaTeX"
%% TeX-master: "numarray"
%% TeX-parse-self: t
%% End:

\chapter{Installing numarray}
\label{cha:installation}

\begin{quote}
   This chapter explains how to install and test numarray, from either the
   source distribution or from the binary distribution.
\end{quote}

Before we start with the actual tutorial, we will describe the steps needed for
you to be able to follow along the examples step by step. These steps include
installing and testing Python, the numarray extensions, and some tools and
sample files used in the examples of this tutorial.


\section{Testing the Python installation}

The first step is to install Python if you haven't already. Python is available
from the Python project page at \url{http://sourceforge.net/projects/python}.
Click on the link corresponding to your platform, and follow the instructions
described there. Unlike earlier versions of numarray, version 0.7 and later
require Python version 2.2.2 at a minimum.  When installed, starting Python by
typing python at the shell or double-clicking on the Python interpreter should
give a prompt such as:
\begin{verbatim}
Python 2.3 (#2, Aug 22 2003, 13:47:10) [C] on sunos5
Type "help", "copyright", "credits" or "license" for more information.
\end{verbatim}
If you have problems getting this part to work, consider contacting a local
support person or emailing
\ulink{python-help@python.org}{mailto:python-help@python.org} for help. If
neither solution works, consider posting on the
\ulink{comp.lang.python}{news:comp.lang.python} newsgroup (details on the
newsgroup/mailing list are available at
\url{http://www.python.org/psa/MailingLists.html\#clp}).


\section{Testing the Numarray Python Extension Installation}

The standard Python distribution does not come, as of this writing, with the
numarray Python extensions installed, but your system administrator may have
installed them already. To find out if your Python interpreter has numarray
installed, type \samp{import numarray} at the Python prompt. You'll see one of
two behaviors (throughout this document user input and python interpreter
output will be emphasized as shown in the block below):
\begin{verbatim}
>>> import numarray
Traceback (innermost last):
File "<stdin>", line 1, in ?
ImportError: No module named numarray
\end{verbatim}
indicating that you don't have numarray installed, or:
\begin{verbatim}
>>> import numarray
>>> numarray.__version__
'0.6'
\end{verbatim}
indicating that you do. If you do, you can skip the next section
and go ahead to section \ref{sec:at-sourceforge}.  If you don't, you have to
get and install the numarray extensions as described in section
\ref{sec:installing-numarray}.

\section{Installing numarray}
\label{sec:installing-numarray}

The release facility at SourceForge is accessed through the project page,
\url{http://sourceforge.net/projects/numpy}.  Click on the "Numarray" release
and you will be presented with a list of the available files. The files whose
names end in ".tar.gz" are source code releases. The other files are binaries
for a given platform (if any are available).

It is possible to get the latest sources directly from our CVS repository using
the facilities described at SourceForge. Note that while every effort is made
to ensure that the repository is always ``good'', direct use of the repository
is subject to more errors than using a standard release.


\subsection{Installing on Unix, Linux, and Mac OSX}
\label{sec:installing-unix}

The source distribution should be uncompressed and unpacked as follows (for
example):
\begin{verbatim}
gunzip numarray-0.6.tar.gz
tar xf numarray-0.6.tar
\end{verbatim}
Follow the instructions in the top-level directory for compilation and
installation. Note that there are options you must consider before beginning.
Installation is usually as simple as:
\begin{verbatim}
python setup.py install
\end{verbatim}
or:
\begin{verbatim}
python setupall.py install
\end{verbatim}
if you want to install all additional packages, which include
\module{\mbox{numarray.convolve}}, \module{\mbox{numarray.fft}},
\module{\mbox{numarray.linear_algebra}}, and
\module{\mbox{numarray.random_array}}.

See numarray-X.XX/Doc/INSTALL.txt for the latest details (X.XX is the version 
number).

\paragraph*{Important Tip} \label{sec:tip:from-numarray-import} Just like all
Python modules and packages, the numarray module can be invoked using either
the \samp{import numarray} form, or the \samp{from numarray import ...} form.
Because most of the functions we'll talk about are in the numarray module, in
this document, all of the code samples will assume that they have been preceded
by a statement:
\begin{verbatim}
>>> from numarray import *
\end{verbatim}
Note the lowercase name in \module{\numarray} as opposed to \module{\numpy}.


\subsection{Installing on Windows}
\label{sec:installing-windows}

To install numarray, you need to be in an account with Administrator
privileges.  As a general rule, always remove (or hide) any old version of
numarray before installing the next version.

We have tested Numarrray on several Win-32 platforms including:
   
\begin{itemize}
\item Windows-XP-Pro-x86 ( MSVC-6.0) 
\item Windows-NT-x86 (MSVC-6.0) 
\item Windows-98-x86 (MSVC-6.0)
\end{itemize}

\subsubsection{Installation from source}

\begin{enumerate}
\item Unpack the distribution: (NOTE: You may have to download an "unzipping"
   utility)
\begin{verbatim}
C:\> unzip numarray.zip 
C:\> cd numarray
\end{verbatim}
\item Build it using the distutils defaults:
\begin{verbatim}
C:\numarray> python setup.py install
\end{verbatim}
   This installs numarray in \texttt{C:\textbackslash{}pythonXX} where XX is the version
   number of your python installation, e.g. 20, 21, etc.
\end{enumerate}


\subsubsection{Installation from self-installing executable}

\begin{enumerate}
\item Click on the executable's icon to run the installer.
\item Click "next" several times.  I have not experimented with customizing the
   installation directory and don't recommend changing any of the installation
   defaults.  If you do and have problems, let us know.
\item Assuming everything else goes smoothly, click "finish".
\end{enumerate}


\subsubsection{Testing your Installation}

Once you have installed numarray, test it with:
\begin{verbatim}
C:\numarray> python
Python 2.2.2 (#18, Dec 30 2002, 02:26:03) [MSC 32 bit (Intel)] on win32
Type "copyright", "credits" or "license" for more information.
>>> import numarray.testall as testall
>>> testall.test()
numeric:  (0, 1115)
records:  (0, 48)
strings:  (0, 166)
objects:  (0, 72)
memmap:  (0, 75)
\end{verbatim}
Each line in the above output indicates that 0 of X tests failed.  X grows
steadily with each release, so the numbers shown above may not be current.


\subsubsection{Installation on Cygwin}

For an installation of numarray for python running on Cygwin, see section
\ref{sec:installing-unix}.



\section{At the SourceForge...}
\label{sec:at-sourceforge}

The SourceForge project page for numarray is at
\url{http://sourceforge.net/projects/numpy}. On this project page you will find
links to:
\begin{description}
\item[The Numpy Discussion List] You can subscribe to a discussion list about
   numarray using the project page at SourceForge. The list is a good place to
   ask questions and get help. Send mail to
   numpy-discussion@lists.sourceforge.net.  Note that there is no
   numarray-discussion group, we share the list created by the numeric community.

\item[The Web Site] Click on "home page" to get to the Numarray Home Page,
   which has links to documentation and other resources, including tools for
   connecting numarray to Fortran.
\item[Bugs and Patches] Bug tracking and patch-management facilities is
   provided on the SourceForge project page.
\item[CVS Repository] You can get the latest and greatest (albeit less tested
   and trustworthy) version of numarray directly from our CVS repository.
\item[FTP Site] The FTP Site contains this documentation in several formats,
   plus maybe some other goodies we have lying around.
\end{description}



%% Local Variables:
%% mode: LaTeX
%% mode: auto-fill
%% fill-column: 79
%% indent-tabs-mode: nil
%% ispell-dictionary: "american"
%% reftex-fref-is-default: nil
%% TeX-auto-save: t
%% TeX-command-default: "pdfeLaTeX"
%% TeX-master: "numarray"
%% TeX-parse-self: t
%% End:

\chapter{High-Level Overview}
\label{cha:high-level-overview}

\begin{quote} 
   In this chapter, a high-level overview of the extensions is provided, giving
   the reader the definitions of the key components of the system. This section
   defines the concepts used by the remaining sections.
\end{quote}

Numarray makes available a set of universal functions (technically ufunc
objects), used in the same way they were used in Numeric. These are discussed
in some detail in chapter \ref{cha:ufuncs}.


\section{Numarray Objects}
\label{sec:numarray-objects}

The array objects are generally homogeneous collections of potentially large
numbers of numbers. All numbers in a numarray are the same kind (i.e. number
representation, such as double-precision floating point). Array objects must be
full (no empty cells are allowed), and their size is immutable. The specific
numbers within them can change throughout the life of the array, however.
There is a "mask array" package ("MA") for Numeric, which has been ported
to numarray as ``numarray.ma''.

Mathematical operations on arrays return new arrays containing the results of
these operations performed element-wise on the arguments of the operation.

The size of an array is the total number of elements therein (it can be 0 or
more). It does not change throughout the life of the array, unless the array
is explicitly resized using the resize function.

The shape of an array is the number of dimensions of the array and its extent
in each of these dimensions (it can be 0, 1 or more). It can change throughout
the life of the array. In Python terms, the shape of an array is a tuple of
integers, one integer for each dimension that represents the extent in that
dimension.  The rank of an array is the number of dimensions along which it is
defined. It can change throughout the life of the array. Thus, the rank is the
length of the shape (except for rank 0). \note{This is not the same meaning of
rank as in linear algebra.}

Use more familiar mathematicial examples: A vector is a rank-1 array
(it has only one dimension along which it can be indexed). A matrix as used in
linear algebra is a rank-2 array (it has two dimensions along which it can be
indexed). It is possible to create a rank-0 array which is just a scalar of 
one single value --- it has no dimension along which it can be indexed.

The type of an array is a description of the kind of element it contains. It
determines the itemsize of the array.  In contrast to Numeric, an array type in
numarray is an instance of a NumericType class, rather than a single character
code. However, it has been implemented in such a way that one may use aliases,
such as `\constant{u1}', `\constant{i1}', `\constant{i2}', `\constant{i4}',
`\constant{f4}', `\constant{f8}', etc., as well as the original character
codes, to set array types.  The itemsize of an array is the number of 8-bit
bytes used to store a single element in the array. The total memory used by an
array tends to be its size times its itemsize, when the size is large (there
is a fixed overhead per array, as well as a fixed overhead per dimension).

Here is an example of Python code using the array objects:
\begin{verbatim}
>>> vector1 = array([1,2,3,4,5])
>>> print vector1
[1 2 3 4 5]
>>> matrix1 = array([[0,1],[1,3]])
>>> print matrix1
[[0 1]
 [1 3]]
>>> print vector1.shape, matrix1.shape
(5,) (2,2)
>>> print vector1 + vector1
[ 2  4  6  8  10]
>>> print matrix1 * matrix1
[[0 1]                                  # note that this is not the matrix
 [1 9]]                                 # multiplication of linear algebra
\end{verbatim}
If this example complains of an unknown name "array", you forgot to begin
your session with
\begin{verbatim}
>>> from numarray import *
\end{verbatim}
See section \ref{sec:tip:from-numarray-import}.


\section{Universal Functions}
\label{sec:universal-functions}

Universal functions (ufuncs) are functions which operate on arrays and other
sequences. Most ufuncs perform mathematical operations on their arguments, also
elementwise.

Here is an example of Python code using the ufunc objects:
\begin{verbatim}
>>> print sin([pi/2., pi/4., pi/6.])
[ 1. 0.70710678 0.5       ]
>>> print greater([1,2,4,5], [5,4,3,2])
[0 0 1 1]
>>> print add([1,2,4,5], [5,4,3,2])
[6 6 7 7]
>>> print add.reduce([1,2,4,5])
12                                      # 1 + 2 + 4 + 5
\end{verbatim}
Ufuncs are covered in detail in "Ufuncs" on page~\pageref{cha:ufuncs}.


\section{Convenience Functions}
\label{sec:conv-funct}

The numarray module provides, in addition to the functions which are needed to
create the objects above, a set of powerful functions to manipulate arrays,
select subsets of arrays based on the contents of other arrays, and other
array-processing operations.
\begin{verbatim}
>>> data = arange(10)                   # analogous to builtin range()
>>> print data
[0 1 2 3 4 5 6 7 8 9]
>>> print where(greater(data, 5), -1, data)
[ 0  1  2  3  4  5 -1 -1 -1 -1]         # selection facility
>>> data = resize(array([0,1]), (9, 9)) # or just: data=resize([0,1], (9,9))
>>> print data
[[0 1 0 1 0 1 0 1 0]
 [1 0 1 0 1 0 1 0 1]
 [0 1 0 1 0 1 0 1 0]
 [1 0 1 0 1 0 1 0 1]
 [0 1 0 1 0 1 0 1 0]
 [1 0 1 0 1 0 1 0 1]
 [0 1 0 1 0 1 0 1 0]
 [1 0 1 0 1 0 1 0 1]
 [0 1 0 1 0 1 0 1 0]]
\end{verbatim}
All of the functions which operate on numarray arrays are described in chapter
\ref{cha:array-functions}.  See page \pageref{func:where} for more information
about \function{where} and page \pageref{func:resize} for
information on \function{resize}.

\section{Differences between numarray and Numeric.}
\label{sec:diff-numarray-numpy}

This new module numarray was developed for a number of reasons. To 
summarize, we regularly deal with large datasets and numarray gives us the
capabilities that we feel are necessary for working with such datasets. In
particular:
\begin{enumerate}
\item Avoid promotion of array types in expressions involving Python scalars
   (e.g., \code{2.*<Float32 array>} should not result in a \code{Float64}
   array).
\item Ability to use memory mapped files.
\item Ability to access fields in arrays of records as numeric arrays without
   copying the data to a new array.
\item Ability to reference byteswapped data or non-aligned data (as might be
   found in record arrays) without producing new temporary arrays.
\item Reuse temporary arrays in expressions when possible.
\item Provide more convenient use of index arrays (put and take).
\end{enumerate}
We decided to implement a new module since many of the existing Numeric
developers agree that the existing Numeric implementation is not suitable 
for massive changes and enhancements.

This version has nearly the full functionality of the basic Numeric.
\emph{Numarray is not fully compatible with Numeric}.
(But it is very similar in most respects).

The incompatibilities are listed below. 
\begin{enumerate}
\item Coercion rules are different. Expressions involving scalars may not
   produce the same type of arrays.  
\item Types are represented by Type Objects rather than character codes (though
   the old character codes may still be used as arguments to the functions).
\item For versions of Python prior to 2.2, arrays have no public attributes.
   Accessor functions must be used instead (e.g., to get shape for array x, one
   must use x.getshape() instead of x.shape). When using Python 2.2 or later,
   however, the attributes of Numarray are in fact available.
\end{enumerate}
A further comment on type is appropriate here. In numarray, types are
represented by type objects and not character codes. As with Numeric there is a
module variable Float32, but now it represents an instance of a FloatingType
class. For example, if x is a Float32 array, x.type() will return a
FloatingType instance associated with 32-bit floats (instead of using
x.typecode() as is done in Numeric). The following will still work in
numarray, to be backward compatible:
\begin{verbatim}
>>> if x.typecode() == 'f':
\end{verbatim}
or use:
\begin{verbatim}
>>> if x.type() == Float32:
\end{verbatim}
(All examples presume ``\code{from numarray import *}'' has been used instead
of ``\code{import numarray}'', see section \ref{sec:tip:from-numarray-import}.)
The advantage of the new scheme is that other kinds of tests become simpler.
The type classes are hierarchical so one can easily test to see if the array is
an integer array. For example:
\begin{verbatim}
>>> if isinstance(x.type(), IntegralType): 
\end{verbatim}
or:
\begin{verbatim}
>>> if isinstance(x.type(), UnsignedIntegralType):
\end{verbatim}



%% Local Variables:
%% mode: LaTeX
%% mode: auto-fill
%% fill-column: 79
%% indent-tabs-mode: nil
%% ispell-dictionary: "american"
%% reftex-fref-is-default: nil
%% TeX-auto-save: t
%% TeX-command-default: "pdfeLaTeX"
%% TeX-master: "numarray"
%% TeX-parse-self: t
%% End:

\chapter{Array Basics}
\label{cha:array-basics}

\begin{quote} 
   This chapter introduces some of the basic functions which will be used
   throughout the text.
\end{quote}

\section{Basics}
\label{sec:arraybasics:basics}

Before we explore the world of image manipulation as a case-study in array
manipulation, we should first define a few terms which we'll use over and
over again. Discussions of arrays and matrices and vectors can get confusing
due to differences in nomenclature. Here is a brief definition of the terms
used in this tutorial, and more or less consistently in the error messages of
\numarray{}.

The Python objects under discussion are formally called ``\NUMARRAY{}'' (or
even more correctly ``\numarray{}'') objects (N-dimensional arrays), but
informally we'll just call them ``array objects'' or just ``arrays''. These are
different from the array objects defined in the standard Python \module{array}
module (which is an older module designed for processing one-dimensional data
such as sound files).

These array objects hold their data in a fixed length, homogeneous (but not
necessarily contiguous) block of elements, i.e.\ their elements all have the
same C type (such as a 64-bit floating-point number). This is quite different
from most Python container objects, which are variable length heterogeneous
collections.

Any given array object has a \index{rank}rank, which is the number of
``dimensions'' or ``axes'' it has. For example, a point in 3D space \code{[1,
   2, 1]} is an array of rank 1 --- it has one dimension. That dimension has a
length of 3.  As another example, the array
\begin{verbatim}
1.0 0.0 0.0
0.0 1.0 2.0
\end{verbatim}
is an array of rank 2 (it is 2-dimensional). The first dimension has a length
of 2, the second dimension has a length of 3. Because the word ``dimension''
has many different meanings to different folks, in general the word ``axis''
will be used instead. Axes are numbered just like Python list indices: they
start at 0, and can also be counted from the end, so that \code{axis=-1} is the
last axis of an array, \code{axis=-2} is the penultimate axis, etc.  

There are there important and potentially unintuitive behaviors of
\module{numarray} arrays which take some getting used to. The first is that by
default, operations on arrays are performed elementwise.\footnote{This is
common to IDL behavior but contrary to Matlab behavior.}  This means that when
adding two arrays, the resulting array has as elements the pairwise sums of the
two operand arrays.  This is true for all operations, including multiplication.
Thus, array multiplication using the * operator will default to elementwise
multiplication, not matrix multiplication as used in linear algebra. Many
people will want to use arrays as linear algebra-type matrices (including their
\index{rank}rank-1 versions, vectors). For those users, the matrixmultiply
function will be useful.

The second behavior which will catch many users by surprise is that
certain operations, such as slicing, return arrays which are simply different
views of the same data; that is, they will in fact share their data. This will
be discussed at length in examples later.  Now that these definitions and 
warnings are laid out, let's see what we can do with these arrays.

The third behavior which may catch Matlab or Fortran users unaware is the use
of row-major data storage as is done in C.  So while a Fortran array might 
be indexed a[x,y],  numarray is indexed a[y,x].

\newpage
\section{Creating arrays from scratch}
\label{sec:creating-arrays-from}


\subsection{array() and types}
\label{sec:array-types}

\begin{funcdesc}{array}{sequence=None, typecode=None, copy=1, savespace=0,
    type=None, shape=None}
   There are many ways to create arrays. The most basic one is the use of the
   \function{array} function:
\begin{verbatim}
>>> a = array([1.2, 3.5, -1])
\end{verbatim}
   to make sure this worked, do:
\begin{verbatim}
>>> print a
[ 1.2  3.5 -1. ]
\end{verbatim}
   The \function{array} function takes several arguments --- the first one is
   the values, which can be a Python sequence object (such as a list or a
   tuple).  If the optional argument \code{type} is omitted, numarray tries to
   find the best data type which can represent all the elements. 
   
   Since the elements we gave our example were two floats and one integer, it
   chose \class{Float64} as the type of the resulting array. One can specify
   unequivocally the \code{type} of the elements --- this is especially 
   useful when, for example, one wants an array contains floats even
   though all of its input elements are integers:
\begin{verbatim}
>>> x,y,z = 1,2,3
>>> a = array([x,y,z])                  # integers are enough for 1, 2 and 3
>>> print a
[1 2 3]
>>> a = array([x,y,z], type=Float32)    # not the default type
>>> print a
[ 1.  2.  3.]
\end{verbatim}
    Another optional argument is the \code{shape} to use for the array.  When
    passed a \class{NumArray} instance, by default \function{array} will make
    an independent, aligned, contiguous, non-byteswapped copy.  If also passed
    a shape or different type, the resulting ``copy'' will be reshaped or
    cast as the new type.
\end{funcdesc}

\begin{funcdesc}{asarray}{seq, type=None, typecode=None}
   This function converts scalars, lists and tuples to a \class{numarray}, when
   possible. It passes \class{numarray}s through, making copies only to
   convert types.  In any other case a \class{TypeError} is raised.
\end{funcdesc}

\begin{funcdesc}{inputarray}{seq, type=None, typecode=None}
  This is an obosolete alias for \function{asarray}.
\end{funcdesc}


\paragraph*{Important Tip} \label{sec:important-tips} 
Pop Quiz: What will be the type of the array below:
\begin{verbatim}
>>> mystery = array([1, 2.0, -3j])
\end{verbatim}
Hint: -3j is an imaginary number. \\
Answer: Complex64
         
A very common mistake is to call \function{array} with a set of numbers as
arguments, as in \code{array(1, 2, 3, 4, 5)}. This doesn't produce the expected
result if at least two numbers are used, because the first argument to
\function{array} must be the entire data for the array --- thus, in most cases,
a sequence of numbers. The correct way to write the preceding invocation is
most likely \code{array([1, 2, 3, 4, 5])}.

Possible values for the type \index{type argument}argument to the
\function{array} creator function (and indeed to any function which accepts a
so-called type for arrays) are:
\begin{enumerate}
\item Elements that can have values true or false: \index{Bool}\class{Bool}.
\item Unsigned numeric types: \index{UInt8}\class{UInt8},
  \index{UInt16}\class{UInt16}, \index{UInt32}\class{UInt32}, and
  \index{UInt64}\class{UInt64}\footnote[1]{UInt64 is unsupported on Windows}.
\item Signed numeric types: 
   \begin{itemize}
   \item Signed integer choices: \index{Int8}\class{Int8},
      \index{Int16}\class{Int16}, \index{Int32}\class{Int32}, \index{Int64}\class{Int64}.
   \item Floating point choices: \index{Float32}\class{Float32},
      \index{Float64}\class{Float64}.
   \end{itemize}
\item Complex number types: \index{Complex32}\class{Complex32},
   \index{Complex64}\class{Complex64}.
\end{enumerate}

To specify a type, e.g. \class{UInt8}, etc, the easiest method is just to
specify it as a string:
\begin{verbatim}
a = array([10], type = 'UInt8')
\end{verbatim}

The various means for specifying types are defined in table
\ref{tab:type-specifiers}, with each item in a row being equivalent.  The
\emph{preferred} methods are in the first 3 columns: numarray type object, type
string, or type code.  The last two columns were added for backwards
compatabililty with Numeric and are not recommended for new code.  Numarray
type object and string names denote the size of the type in bits.  The numarray
type code names denote the size of the type in bytes.  The type objects must be
imported from or referenced via the numerictypes module.  All type strings and
type codes are specified using ordinary Python strings, and hence don't require
an import.  Complex type names denote the size of one component, real or
imaginary, in bits/bytes, but the letter code is the total size of the 
whole number ('c8' and 'c16').

\begin{table}[h]
  \centering
  \caption{Type specifiers}
  \label{tab:type-specifiers}
  \begin{tabular}{|l|l|l|l|l|}
    \hline
    Numarray Type&Numarray String&Numarray Code&Numeric String&Numeric Code\\
    \hline
    Int8&'Int8'&'i1'&'Byte'&'1'\\
    \hline
    UInt8&'UInt8'&'u1'&'UByte'& \\
    \hline
    Int16&'Int16'&'i2'&'Short'&'s'\\
    \hline
    UInt16&'UInt16'&'u2'&'UShort'& \\
    \hline
    Int32&'Int32'&'i4'&'Int'&'i'\\
    \hline
    UInt32&'UInt32'&'u4'&'UInt'&'u'\\
    \hline
    Int64&'Int64'&'i8'& & \\
    \hline
    UInt64\footnotemark[1]&'UInt64'&'u8'& & \\
    \hline
    Float32&'Float32'&'f4'&'Float'&'f'\\
    \hline
    Float64&'Float64'&'f8'&'Double'&'d'\\
    \hline
    Complex32&'Complex32'&'c8'& &'F'\\
    \hline
    Complex64&'Complex64'&'c16'&'Complex'&'D'\\
    \hline
    Bool&'Bool'& & & \\
    \hline
  \end{tabular}
\end{table}

\subsection{Multidimensional Arrays}
\label{sec:multi-dim-arrays}

The following example shows one way of creating \index{multidimensional
   arrays}multidimensional arrays:
\begin{verbatim}
>>> ma = array([[1,2,3],[4,5,6]])
>>> print ma
[[1 2 3]
 [4 5 6]]
\end{verbatim}
The first argument to \function{array} in the code above is a single
\class{list} containing two lists, each containing three elements. If we wanted
floats instead, we could specify, as discussed in the previous section, the
optional type we wished:
\begin{verbatim}
>>> ma_floats = array([[1,2,3],[4,5,6]], type=Float32)
>>> print ma_floats
[[ 1.  2.  3.]
 [ 4.  5.  6.]]
\end{verbatim}
This array allows us to introduce the notion of \index{shape}``shape''. The
shape of an array is the set of numbers which define its dimensions. The shape
of the array \var{ma} defined above is 2 by 3. More precisely, all arrays have
an attribute which is a tuple of integers giving the shape. The
\index{getshape}\method{getshape} method returns this tuple.  In general, one
can directly use the \member{shape} attribute (but only for Python 2.2 and
later) to get or set its value. Since it isn't supported for earlier versions
of Python, subsequent examples will use \method{getshape} and
\index{setshape}\method{setshape} only. So, in this case:
\begin{verbatim}
>>> print ma.shape                      # works only with Python 2.2 or later
>>> print ma.getshape()                 # works with all Python versions
(2, 3)
\end{verbatim}
Using the earlier definitions, this is a shape of \index{rank}rank 2, where the
first axis has length 2, and the second axis has length 3. The rank of an array
\code{A} is always equal to \code{len(A.getshape())}.  Note that shape is an
attribute and \method{getshape} is a method of array objects. They are the
first of several that we will see throughout this tutorial. If you're not used
to object-oriented programming, you can think of attributes as ``features'' or
``qualities'' of individual arrays, and methods are functions that operate on
individual arrays.  The relation between an array and its shape is similar to
the relation between a person and their hair color. In Python, it's called an
object/attribute relation.

\begin{funcdesc}{reshape}{a, shape}
   What if one wants to change the dimensions of an array? For now, let us
   consider changing the shape of an array without making it ``grow''. Say, for
   example, we want to make the 2x3 array defined above (\var{ma}) an array of
   rank 1:
\begin{verbatim}
>>> flattened_ma = reshape(ma, (6,))
>>> print flattened_ma
[1 2 3 4 5 6]
\end{verbatim}
   One can change the shape of arrays to any shape as long as the product of
   all the lengths of all the axes is kept constant (in other words, as long as
   the number of elements in the array doesn't change):
\begin{verbatim}
>>> a = array([1,2,3,4,5,6,7,8])
>>> print a
[1 2 3 4 5 6 7 8]
>>> b = reshape(a, (2,4))               # 2*4 == 8
>>> print b
[[1 2 3 4]
 [5 6 7 8]]
>>> c = reshape(b, (4,2))               # 4*2 == 8
>>> print c
[[1 2]
 [3 4]
 [5 6]
 [7 8]]
\end{verbatim}
   The function \function{reshape} expects an array/sequence as its 
   first argument, and a shape as its second argument.
   The shape has to be a sequence of integers (a \class{list} or a
   \class{tuple}).  There is also a \method{setshape}
   method, which changes the shape of an array in-place (see below).
   
   One nice feature of shape tuples is that one entry in the shape tuple is
   allowed to be -1. The -1 will be automatically replaced by whatever number
   is needed to build a shape which does not change the size of the array.
   Thus:
\begin{verbatim}
>>> a = reshape(array(range(25)), (5,-1))
>>> print a, a.getshape()
[[ 0  1  2  3  4]
 [ 5  6  7  8  9]
 [10 11 12 13 14]
 [15 16 17 18 19]
 [20 21 22 23 24]] (5, 5)
\end{verbatim}
   The \member{shape} of an array is a modifiable attribute of the array, but
   it is an internal attribute. You can change the shape of an array by calling
   the \method{setshape} method (or by assigning a \class{tuple} to the shape
   attribute, in Python 2.2 and later), which assigns a new shape to the array:
\begin{verbatim}
>>> a = array([1,2,3,4,5,6,7,8,9,10])
>>> a.getshape()
(10,)
>>> a.setshape((2,5))
>>> a.shape = (2,5)                     # for Python 2.2 and later
>>> print a
[[ 1  2  3  4  5]
 [ 6  7  8  9 10]]
>>> a.setshape((10,1))                  # second axis has length 1
>>> print a
[[ 1]
 [ 2]
 [ 3]
 [ 4]
 [ 5]
 [ 6]
 [ 7]
 [ 8]
 [ 9]
 [10]]
>>> a.setshape((5,-1))                  # note the -1 trick described above
>>> print a
[[ 1  2]
 [ 3  4]
 [ 5  6]
 [ 7  8]
 [ 9 10]]
\end{verbatim}
   As in the rest of Python, violating rules (such as the one about which
   shapes are allowed) results in exceptions:
\begin{verbatim}
>>> a.setshape((6,-1))
Traceback (innermost last):
  File "<stdin>", line 1, in ?
ValueError: New shape is not consistent with the old shape
\end{verbatim}
\end{funcdesc}


\paragraph*{For Advanced Users: Printing arrays}

\begin{quote}
   Sections denoted ``For Advanced Users'' indicates 
   function aspects which may not be needed for a first introduction of
   \numarray{}, but is mentioned for the sake of completeness.
\end{quote}

The default \index{printing arrays}printing routine provided by the
\module{\numarray} module prints arrays as follows:
\begin{enumerate}
\item The last axis is always printed left to right.
\item The next-to-last axis is printed top to bottom.
\end{enumerate}
The remaining axes are printed top to bottom with increasing numbers of
separators.

This explains why rank-1 arrays are printed from left to right, rank-2 arrays
have the first dimension going down the screen and the second dimension going
from left to right, etc.

If you want to change the shape of an array so that it has more elements than
it started with (i.e. grow it), then you have several options: One solution is
to use the \index{concatenate}\function{concatenate} function discussed later.
\begin{verbatim}
>>> print a
[0 1 2 3 4 5 6 6 7]
>>> print concatenate([[a],[a]])
>>> print b
[[0 1 2 3 4 5 6 7]
 [0 1 2 3 4 5 6 7]]
>>> print b.getshape()
(2, 8)
\end{verbatim}


\begin{funcdesc}{resize}{array, shape}
   A final possibility is the \function{resize} function, which takes a
   \var{base} array as its first argument and the desired \var{shape} as the
   second argument.  Unlike \function{reshape}, the shape argument to
   \function{resize} can be a smaller or larger shape than the input
   array. Smaller shapes will result in arrays with the data at the
   ``beginning'' of the input array, and larger shapes result in arrays with
   data containing as many replications of the input array as are needed to
   fill the shape. For example, starting with a simple array
\begin{verbatim}
>>> base = array([0,1])
\end{verbatim}
   one can quickly build a large array with replicated data:
\begin{verbatim}
>>> big = resize(base, (9,9))
>>> print big
[[0 1 0 1 0 1 0 1 0]
 [1 0 1 0 1 0 1 0 1]
 [0 1 0 1 0 1 0 1 0]
 [1 0 1 0 1 0 1 0 1]
 [0 1 0 1 0 1 0 1 0]
 [1 0 1 0 1 0 1 0 1]
 [0 1 0 1 0 1 0 1 0]
 [1 0 1 0 1 0 1 0 1]
 [0 1 0 1 0 1 0 1 0]]
\end{verbatim}
\end{funcdesc}

\newpage
\section{Creating arrays with values specified ``on-the-fly''}
\label{sec:creating-arrays-on-the-fly}

\begin{funcdesc}{zeros}{shape, type}
\end{funcdesc}
\begin{funcdesc}{ones}{shape, type}
   Often, one needs to manipulate arrays filled with numbers which aren't
   available beforehand. The \module{\numarray} module provides a few functions
   which create arrays from scratch: \function{zeros} and \function{ones}
   simply create arrays of a given \var{shape} filled with zeros and ones
   respectively:
\begin{verbatim}
>>> z = zeros((3,3))
>>> print z
[[0 0 0]
 [0 0 0]
 [0 0 0]]
>>> o = ones([2,3])
>>> print o
[[1 1 1]
 [1 1 1]]
\end{verbatim}
   Note that the first argument is a shape --- it needs to be a \class{tuple} or
   a \class{list} of integers. Also note that the default type for the
   returned arrays is \class{Int}, which you can override, e. g.: 
\begin{verbatim}
>>> o = ones((2,3), Float32)
>>> print o
[[ 1.  1.  1.]
 [ 1.  1.  1.]]
\end{verbatim}
\end{funcdesc}


\begin{funcdesc}{arrayrange}{a1, a2=None, stride=1, type=None, shape=None}
\end{funcdesc}
\begin{funcdesc}{arange}{a1, a2=None, stride=1, type=None, shape=None}
   The \function{arange} function is similar to the \function{range} function
   in Python, except that it returns an \class{array} as opposed to a
   \class{list}.
   \function{arange} and \function{arrayrange} are equivalent.
\begin{verbatim}
>>> r = arange(10)
>>> print r
[0 1 2 3 4 5 6 7 8 9]
\end{verbatim}
   Combining the \function{arange} with the \function{reshape} function, we can
   get:
\begin{verbatim}
>>> big = reshape(arange(100),(10,10))
>>> print big
[[ 0  1  2  3  4  5  6  7  8  9]
 [10 11 12 13 14 15 16 17 18 19]
 [20 21 22 23 24 25 26 27 28 29]
 [30 31 32 33 34 35 36 37 38 39]
 [40 41 42 43 44 45 46 47 48 49]
 [50 51 52 53 54 55 56 57 58 59]
 [60 61 62 63 64 65 66 67 68 69]
 [70 71 72 73 74 75 76 77 78 79]
 [80 81 82 83 84 85 86 87 88 89]
 [90 91 92 93 94 95 96 97 98 99]]
\end{verbatim}
   One can set the \code{a1}, \code{a2}, and \code{stride} arguments, which 
   allows for more varied ranges:
\begin{verbatim}
>>> print arange(10,-10,-2)
[10  8  6  4  2  0  -2  -4  -6  -8]
\end{verbatim}
   An important feature of arange is that it can be used with non-integer
   starting points and strides:
\begin{verbatim}
>>> print arange(5.0)
[ 0. 1. 2. 3. 4.]
>>> print arange(0, 1, .2)
[ 0.   0.2  0.4  0.6  0.8]
\end{verbatim}
   If you want to create an array with just one value, repeated over and over,
   you can use the * operator applied to lists
\begin{verbatim}
>>> a = array([[3]*5]*5)
>>> print a
[[3 3 3 3 3]
 [3 3 3 3 3]
 [3 3 3 3 3]
 [3 3 3 3 3]
 [3 3 3 3 3]]
\end{verbatim}
   but that is relatively slow, since the duplication is done on Python lists.
   A quicker way would be to start with 0's and add 3:
\begin{verbatim}
         >>> a = zeros([5,5]) + 3
         >>> print a
         [[3 3 3 3 3]
          [3 3 3 3 3]
          [3 3 3 3 3]
          [3 3 3 3 3]
          [3 3 3 3 3]]
\end{verbatim}
   The optional \code{type} argument forces the type of the resulting array,
   which is otherwise the ``highest'' of the \code{a1}, \code{a2}, and 
   \code{stride} arguments.  The \code{a1} argument defaults to 0 if not 
   specified. Note that if the specified \code{type} is
   is ``lower'' than what \function{arange} would
   normally use, the array is the result of a precision-losing cast (a
   round-down, as that used in the \method{astype} method for arrays.)
\end{funcdesc}


\subsection{Creating an array from a function}
\label{sec:creating-array-from-function}

\begin{funcdesc}{fromfunction}{object, shape} 
   Finally, one may want to create an array whose elements are the result
   of a function evaluation. This is done using the \function{fromfunction}
   function, which takes two arguments, a \var{shape} and a callable
   \var{object} (usually a function).  For example:
\begin{verbatim}
>>> def dist(x,y):
...   return (x-5)**2+(y-5)**2          # distance from (5,5) squared
...
>>> m = fromfunction(dist, (10,10))
>>> print m
[[50 41 34 29 26 25 26 29 34 41]
 [41 32 25 20 17 16 17 20 25 32]
 [34 25 18 13 10  9 10 13 18 25]
 [29 20 13  8  5  4  5  8 13 20]
 [26 17 10  5  2  1  2  5 10 17]
 [25 16  9  4  1  0  1  4  9 16]
 [26 17 10  5  2  1  2  5 10 17]
 [29 20 13  8  5  4  5  8 13 20]
 [34 25 18 13 10  9 10 13 18 25]
 [41 32 25 20 17 16 17 20 25 32]]
>>> m = fromfunction(lambda i,j,k: 100*(i+1)+10*(j+1)+(k+1), (4,2,3))
>>> print m
[[[111 112 113]
  [121 122 123]]
 [[211 212 213]
  [221 222 223]]
 [[311 312 313]
  [321 322 323]]
 [[411 412 413]
  [421 422 423]]]
\end{verbatim}
   These examples show that \function{fromfunction}
   creates an array of the shape specified by its second argument, and with the
   contents corresponding to the value of the function argument (the first
   argument) evaluated at the indices of the array. Thus the value of
   \code{m[3, 4]} in the first example above is the value of dist when
   \code{x=3} and \code{y=4}.  Similarly for the lambda function in the second
   example, but with a rank-3 array.  The implementation of
   \function{fromfunction} consists of:
\begin{verbatim}
def fromfunction(function, dimensions):
    return apply(function, tuple(indices(dimensions)))
\end{verbatim}
   which means that the function \function{function} is called with arguments
   given by the sequence \code{indices(dimensions)}. As described in the
   definition of indices, this consists of arrays of indices which will be of
   rank one less than that specified by dimensions. This means that the
   function argument must accept the same number of arguments as there are
   dimensions in \var{dimensions}, and that each argument will be an array of
   the same shape as that specified by dimensions.  Furthermore, the array
   which is passed as the first argument corresponds to the indices of each
   element in the resulting array along the first axis, that which is passed as
   the second argument corresponds to the indices of each element in the
   resulting array along the second axis, etc. A consequence of this is that
   the function which is used with \function{fromfunction} will work as
   expected only if it performs a separable computation on its arguments, and
   expects its arguments to be indices along each axis. Thus, no logical
   operation on the arguments can be performed, or any non-shape preserving
   operation. Thus, the following will not work as expected:
\begin{verbatim}
>>> def buggy(test):
...     if test > 4: return 1
...     else: return 0
...
>>> print fromfunction(buggy,(10,))
Traceback (most recent call last):
...
RuntimeError: An array doesn't make sense as a truth value.  Use any(a) or
all(a).
\end{verbatim}
The reason \function{buggy()} failed is that indices((10,)) results in an array
passed as \var{test}.  The result of comparing \var{test} with 4 is also an
array which has no unambiguous meaning as a truth value.

Here is how to do it properly. We add a print statement to the
   function for clarity:
\begin{verbatim}
>>> def notbuggy(test):                 # only works in Python 2.1 & later
...     print test
...     return where(test>4,1,0)
...
>>> fromfunction(notbuggy,(10,))
[0 1 2 3 4 5 6 7 8 9]
array([0, 0, 0, 0, 0, 1, 1, 1, 1, 1])
\end{verbatim}
   We leave it as an excercise for the reader to figure out why the ``buggy''
   example gave the result 1.
\end{funcdesc}


\begin{funcdesc}{identity}{size}
   The \function{identity} function takes a single integer argument and returns
   a square identity array (in the ``matrix'' sense) of that \var{size} of
   integers:
\begin{verbatim}
      >>> print identity(5)
      [[1 0 0 0 0]
       [0 1 0 0 0]
       [0 0 1 0 0]
       [0 0 0 1 0]
       [0 0 0 0 1]]
\end{verbatim}
\end{funcdesc}



\newpage
\section{Coercion and Casting}
\label{sec:coercion-casting}

We've mentioned the types of arrays, and how to create arrays with the right
type.  But what happens when arrays with different types interact?  For 
some operations, the behavior of \numarray{} is significantly different 
from Numeric.

\subsection{Automatic Coercions and Binary Operations}
\label{sec:automatic-coercion-binary-casting}

In \numarray{} (in contrast to Numeric), there is now a distinction between how
coercion is treated in two basic cases: array/scalar operations and array/array
operations. In the array/array case, the coercion rules are nearly identical to
those of Numeric, the only difference being combining signed and unsigned
integers of the same size.  The array/array result types are enumerated in
table \ref{tab:array-array-result-types}.
\begin{table}[h]
\footnotesize
\centering
\caption{Array/Array Result Types}
\label{tab:array-array-result-types}
\begin{tabular}{|l|l|l|l|l|l|l|l|l|l|l|l|l|l|}
\hline
 &Bool&Int8&UInt8&Int16&UInt16&Int32&UInt32&Int64&UInt64&Float32&Float64&Complex32&Complex64\\
\hline
Bool&Int8&Int8&UInt8&Int16&UInt16&Int32&UInt32&Int64&UInt64&Float32&Float64&Complex32&Complex64\\
\hline
Int8& &Int8&Int16&Int16&Int32&Int32&Int64&Int64&Int64&Float32&Float64&Complex32&Complex64\\
\hline
UInt8& & &UInt8&Int16&UInt16&Int32&UInt32&Int64&UInt64&Float32&Float64&Complex32&Complex64\\
\hline
Int16& & & &Int16&Int32&Int32&Int64&Int64&Int64&Float32&Float64&Complex32&Complex64\\
\hline
UInt16& & & & &UInt16&Int32&UInt32&Int64&UInt64&Float32&Float64&Complex32&Complex64\\
\hline
Int32& & & & & &Int32&Int64&Int64&Int64&Float32&Float64&Complex32&Complex64\\
\hline
UInt32& & & & & & &UInt32&Int64&UInt64&Float32&Float64&Complex32&Complex64\\
\hline
Int64& & & & & & & &Int64&Int64&Float64&Float64&Complex64&Complex64\\
\hline
UInt64& & & & & & & & &UInt64&Float64&Float64&Complex64&Complex64\\
\hline
Float32& & & & & & & & & &Float32&Float64&Complex32&Complex64\\
\hline
Float64& & & & & & & & & & &Float64&Complex64&Complex64\\
\hline
Complex32& & & & & & & & & & & &Complex32&Complex64\\
\hline
Complex64& & & & & & & & & & & & &Complex64\\
\hline
\end{tabular}
\end{table}

Scalars, however, are treated differently. If the scalar is of the same
``kind'' as the array (for example, the array and scalar are both integer
types) then the output is the type of the array, even if it is of a normally
``lower'' type than the scalar.  Adding an \class{Int16} array with an integer
scalar results in an \class{Int16} array, not an \class{Int32} array as is the
case in Numeric.  Likewise adding a \class{Float32} array to a float scalar
results in a \class{Float32} array rather than a \class{Float64} array as is
the case with Numeric.  Adding an \class{Int16} array and a float scalar will
result in a \class{Float64} array, however, since the scalar is of a higher
kind than the array.  Finally, when scalars and arrays are operated on
together, the scalar is converted to a rank-0 array first.  Thus, adding a
``small'' integer to a ``large'' floating point array is equivalent to first
casting the integer ``up'' to the type of the array.
\begin{verbatim}
>>> print (array ((1, 2, 3), type=Int16) * 2).type()
numarray type: Int16
>>> arange(0, 1.0, .1) + 12
array([ 12. , 12.1, 12.2, 12.3, 12.4, 12.5, 12.6, 12.7, 12.8, 12.9]
\end{verbatim}

The results of array/scalar operations are enumerated in table
\ref{tab:Array-Scalar-Result-Types}.  Entries marked with " are identical to
their neighbors on the same row.
\begin{table}[h]
\footnotesize
\centering
\caption{Array/Scalar Result Types}
\label{tab:Array-Scalar-Result-Types}
\begin{tabular}{|l|l|l|l|l|l|l|l|l|l|l|l|l|l|}
\hline
 &Bool&Int8&UInt8&Int16&UInt16&Int32&UInt32&Int64&UInt64&Float32&Float64&Complex32&Complex64\\
\hline
int&Int32&Int8&UInt8&Int16&UInt16&Int32&UInt32&Int64&UInt64&Float32&Float64&Complex32&Complex64\\
\hline
long&Int32&Int8&UInt8&Int16&UInt16&Int32&UInt32&Int64&UInt64&Float32&Float64&Complex32&Complex64\\
\hline
float&Float64&"&"&"&"&"&"&"&Float64&Float32&Float64&Complex32&Complex64\\
\hline
complex&Complex64&"&"&"&"&"&"&"&"&"&"&"&Complex64\\
\hline
\end{tabular}
\end{table}

\footnotetext[10]{Float64}
\footnotetext[20]{Complex64}

\subsection{The type value table}
\label{sec:type-value-table}

The type identifiers (\class{Float32}, etc.) are \class{NumericType} instances.
The mapping between type and the equivalent C variable is machine dependent.
The correspondences between types and C variables for 32-bit architectures are
shown in Table \ref{tab:type-identifiers}.

\begin{table}[h]
   \centering
   \caption{Type identifier table on a x86 computer.}
   \label{tab:type-identifiers}
   \begin{tabular}{ccl}
      \# of bytes & \# of bits      & Identifier \\
           1      &       8         &   Bool \\
           1      &       8         &   Int8 \\
           1      &       8         &   UInt8 \\
           2      &       16        &   Int16 \\
           2      &       16        &   UInt16 \\
           4      &       32        &   Int32 \\
           4      &       32        &   UInt32 \\
           8      &       64        &   Int64 \\
           8      &       64        &   UInt64 \\
           4      &       32        &   Float32 \\
           8      &       64        &   Float64 \\
           8      &       64        &   Complex32 \\
           16     &      128        &   Complex64 
   \end{tabular}
\end{table}

\subsection{Long: the platform relative type}
The type identifier \class{Long} is aliased to either \class{Int32} or
\class{Int64}, depending on the machine architecture where numarray is
installed.  On 32-bit platforms, \class{Long} is defined as \class{Int32}.  On
64-bit (LP64) platforms, \class{Long} is defined as \class{Int64}. \class{Long}
is used as the default integer type for arrays and for index values, such as
those returned by \function{nonzero}.  

\subsection{Deliberate casts (potentially down)}
\label{sec:deliberate-casts}

\begin{methoddesc}{astype}{type}
   You may also force \module{numarray} to cast any number array to another
   number array.  For example, to take an array of any numeric type
   (\class{IntX} or \class{FloatX} or \class{ComplexX} or \class{UIntX}) and
   convert it to a 64-bit float, one can do:
\begin{verbatim}
>>> floatarray = otherarray.astype(Float64)
\end{verbatim}
   The \var{type} can be any of the number types, ``larger'' or ``smaller''. If
   it is larger, this is a cast-up. If it is smaller, the standard casting
   rules of the underlying language (C) are used, which means that truncation
   or integer wrap-around can occur:
\begin{verbatim}
>>> print x
[   0.     0.4    0.8    1.2  300.6]
>>> print x.astype(Int32)
[  0   0   0   1 300]
>>> print x.astype(Int8)      # wrap-around
[ 0  0  0  1 44]
\end{verbatim}
   If the \var{type} used with \method{astype} is the original array's type,
   then a copy of the original array is returned.
\end{methoddesc}


\newpage
\section{Operating on Arrays}
\label{sec:operating-arrays}

\subsection{Simple operations}
\label{sec:simple-operations}

If you have a keen eye, you have noticed that some of the previous examples did
something new: they added a number to an array. Indeed, most Python operations
applicable to numbers are directly applicable to arrays:
\begin{verbatim}
>>> print a
[1 2 3]
>>> print a * 3
[3 6 9]
>>> print a + 3
[4 5 6]
\end{verbatim}
Note that the mathematical operators behave differently depending on the types
of their operands. When one of the operands is an array and the other a
number, the number is added to all the elements of the array, and the resulting
array is returned. This is called \index{broadcasting}\var{broadcasting}. 
This also occurs for unary mathematical operations such as sine and the 
negative sign:
\begin{verbatim}
>>> print sin(a)
[ 0.84147096 0.90929741 0.14112 ]
>>> print -a
[-1 -2 -3]
\end{verbatim}
When both elements are arrays of the same shape, then a new array is created,
where each element is the operation result of the corresponding elements in 
the original arrays:
\begin{verbatim}
>>> print a + a
[2 4 6]
\end{verbatim}
If the operands of operations such as addition, are arrays having the same
rank but different dimensions, then an exception is generated:
\begin{verbatim}
>>> a = array([1,2,3])
>>> b = array([4,5,6,7])                # note this has four elements
>>> print a + b
Traceback (innermost last):
  File "<stdin>", line 1, in ?
ValueError: Arrays have incompatible shapes
\end{verbatim}
This is because there is no reasonable way for numarray to interpret addition
of a \code{(3,)} shaped array and a \code{(4,)} shaped array.

Note what happens when adding arrays with different rank:
\begin{verbatim}
>>> print a
[1 2 3]
>>> print b
[[ 4  8 12]
 [ 5  9 13]
 [ 6 10 14]
 [ 7 11 15]]
>>> print a + b
[[ 5 10 15]
 [ 6 11 16]
 [ 7 12 17]
 [ 8 13 18]]
\end{verbatim}
This is another form of \index{broadcasting}broadcasting. To understand this,
one needs to look carefully at the shapes of \code{a} and \code{b}:
\begin{verbatim}
>>> a.getshape()
(3,)
>>> b.getshape()
(4,3)
\end{verbatim}
Note that the last axis of \code{a} is the same length as that of \code{b}
(i.e.\ compare the last elements in their shape tuples).  Because \code{a}'s
and \code{b}'s last dimensions both have length 3, those two dimensions were
``matched'', and a new dimension was created and automatically ``assumed'' for
array \code{a}. The data already in \code{a} were ``replicated'' as many 
times as needed (4, in this case) to make the shapes of the two operand 
arrays conform. This
replication (\index{broadcasting}broadcasting) occurs when arrays are operands
to binary operations and their shapes differ, based on the following algorithm:
\begin{itemize}
\item starting from the last axis, the axis lengths (dimensions) of the
   operands are compared,
\item if both arrays have axis lengths greater than 1, but the lengths differ,
   an exception is raised,
\item if one array has an axis length greater than 1, then the other array's
   axis is ``stretched'' to match the length of the first axis; if the other
   array's axis is not present (i.e., if the other array has smaller rank),
   then a new axis of the same length is created.
\end{itemize}

Operands with the following shapes will work:
\begin{verbatim}
(3, 2, 4) and (3, 2, 4)
(3, 2, 4) and (2, 4)
(3, 2, 4) and (4,)
(2, 1, 2) and (2, 2)
\end{verbatim}

But not these:
\begin{verbatim}
(3, 2, 4) and (2, 3, 4)
(3, 2, 4) and (3, 4)
(4,) and (0,)
(2, 1, 2) and (0, 2)
\end{verbatim}

This algorithm is complex to describe, but intuitive in practice.


\subsection{In-place operations}
\label{sec:inplace-operations}

Beginning with Python 2.0, Python supports the in-place operators
\index{+=}\code{+=}, \index{+=}\code{-=}, \index{*=}\code{*=}, and
\index{/=}\code{/=}. \Numarray{} supports these operations, but you need to be
careful. The right-hand side should be of the same type. Some violation of this
is possible, but in general contortions may be necessary for using the smaller
``kinds'' of types.
\begin{verbatim}
>>> x = array ([1, 2, 3], type=Int16)
>>> x += 3.5
>>> print x
[4 5 6]
\end{verbatim}


%% Local Variables:
%% mode: LaTeX
%% mode: auto-fill
%% fill-column: 79
%% indent-tabs-mode: nil
%% ispell-dictionary: "american"
%% reftex-fref-is-default: nil
%% TeX-auto-save: t
%% TeX-command-default: "pdfeLaTeX"
%% TeX-master: "numarray"
%% TeX-parse-self: t
%% End:

\chapter{Array Indexing}
\label{cha:array-indexing}

This chapter discusses the rich and varied ways of indexing numarray
objects to specify individual elements, sub-arrays, sub-samplings, and
even random collections of elements.

\section{Getting and Setting array values}
\label{sec:get-set-array-values}

Just like other Python sequences, array contents are manipulated with the
\code{[]} notation. For rank-1 arrays, there are no differences between list
and array notations:
\begin{verbatim}
>>> a = arange(10)
>>> print a[0]                          # get first element
0
>>> print a[1:5]                        # get second through fifth elements
[1 2 3 4]
>>> print a[-1]                         # get last element
9
>>> print a[:-1]                        # get all but last element
[0 1 2 3 4 5 6 7 8]
\end{verbatim}
If an array is multidimensional (of rank > 1), then specifying a single 
integer index will return an array of
dimension one less than the original array.

\begin{verbatim}
>>> a = arange(9, shape=(3,3))
>>> print a
[[0 1 2]
 [3 4 5]
 [6 7 8]]
>>> print a[0]                          # get first row, not first element!
[0 1 2]
>>> print a[1]                          # get second row
[3 4 5]
\end{verbatim}
To get to individual elements in a rank-2 array, one specifies both indices
separated by commas:
\begin{verbatim}
>>> print a[0,0]                        # get element at first row, first column
0
>>> print a[0,1]                        # get element at first row, second column
1
>>> print a[1,0]                        # get element at second row, first column
3
>>> print a[2,-1]                       # get element at third row, last column
8
\end{verbatim}
Of course, the \code{[]} notation can be used to set values as well:
\begin{verbatim}
>>> a[0,0] = 123
>>> print a
[[123   1   2]
 [  3   4   5]
 [  6   7   8]]
\end{verbatim}
Note that when referring to rows, the right hand side of the equal sign needs
to be a sequence which ``fits'' in the referred array subset, as described 
by the broadcast rule (in the code sample below, a 3-element row):
\begin{verbatim}
>>> a[1] = [10,11,12] ; print a
[[123   1   2]
 [ 10  11  12]
 [  6   7   8]]
>>> a[2] = 99 ; print a
[[123   1   2]
 [ 10  11  12]
 [ 99  99  99]]
\end{verbatim}

Note also that when assigning floating point values to integer arrays that
the values are silently truncated:
\begin{verbatim}
>>> a[1] = 93.999432
[[123   1   2]
 [ 93  93  93]
 [ 99  99  99]]
\end{verbatim}

\newpage
\section{Slicing Arrays}
\label{sec:slicing-arrays}

The standard rules of Python slicing apply to arrays, on a per-dimension basis.
Assuming a 3x3 array:
\begin{verbatim}
>>> a = reshape(arange(9),(3,3))
>>> print a
[[0 1 2]
 [3 4 5]
 [6 7 8]]
\end{verbatim}
The plain \code{[:]} operator slices from beginning to end:
\begin{verbatim}
>>> print a[:,:]
[[0 1 2]
 [3 4 5]
 [6 7 8]]
\end{verbatim}
In other words, \code{[:]} with no arguments is the same as \code{[:]} for
lists --- it can be read ``all indices along this axis''.  (Actually, there is
an important distinction; see below.) So, to get the second row along the
second dimension:
\begin{verbatim}
>>> print a[:,1]
[1 4 7]
\end{verbatim}
Note that what was a ``column'' vector is now a ``row'' vector.  Any ``integer
slice'' (as in the 1 in the example above) results in a returned array with
rank one less than the input array.  

There is one important distinction between
slicing arrays and slicing standard Python sequence objects. A slice of a
\class{list} is a new copy of that subset of the \class{list}; a slice of an
array is just a view into the data of the first array.  To force a copy, you
can use the \function{copy} method. For example:
\begin{verbatim}
>>> a = arange (20)
>>> b = a[3:8]
>>> c = a[3:8].copy()
>>> a[5] = -99
>>> print b
[  3   4 -99   6   7]
>>> print c
[3 4 5 6 7]
\end{verbatim}
If one does not specify as many slices as there are dimensions in an array,
then the remaining slices are assumed to be ``all''. If \var{A} is a rank-3
array, then
\begin{verbatim}
A[1] == A[1,:] == A[1,:,:]
\end{verbatim}
An additional slice notation for arrays which does not exist for Python
lists (before Python 2.3), i. e. the optional third argument, meaning 
the ``step size'', also called \index{stride}stride or increment. Its 
default value is 1, meaning return every element in the specified range.  
Alternate values allow one to skip some of the elements in the slice:
\begin{verbatim}
>>> a = arange(12)
>>> print a
[ 0  1  2  3  4  5  6  7  8  9 10 11]
>>> print a[::2]                        # return every *other* element
[ 0  2  4  6  8 10]
\end{verbatim}
\index{stride!Negative}Negative strides are allowed as long as the starting
index is greater than the stopping index:
\begin{verbatim}
>>> a = reshape(arange(9),(3,3))                                                                                          Array Basics
>>> print a
[[0 1 2]
 [3 4 5]
 [6 7 8]]
>>> print a[:, 0]
[0 3 6]
>>> print a[0:3, 0]
[0 3 6]
>>> print a[2::-1, 0]
[6 3 0]
\end{verbatim}
If a negative stride is specified and the starting or stopping indices are
omitted, they default to ``end of axis'' and ``beginning of axis''
respectively.  Thus, the following two statements are equivalent for the array
given:
\begin{verbatim}
>>> print a[2::-1, 0]
[6 3 0]
>>> print a[::-1, 0]
[6 3 0]
>>> print a[::-1]                       # this reverses only the first axis
[[6 7 8]
 [3 4 5]
 [0 1 2]]
>>> print a[::-1,::-1]                  # this reverses both axes
[[8 7 6]
 [5 4 3]
 [2 1 0]]
\end{verbatim}
One final way of slicing arrays is with the keyword \samp{...} This keyword is
somewhat complicated. It stands for ``however many `:' I need depending on the
rank of the object I'm indexing, so that the indices I \emph{do} specify are at
the end of the index list as opposed to the usual beginning''.

So, if one has a rank-3 array \var{A}, then \code{A[...,0]} is the same thing
as \code{A[:,:,0]}, but if \var{B} is rank-4, then \code{B[...,0]} is the same
thing as: \code{B[:,:,:,0]}. Only one \samp{...} is expanded in an index
expression, so if one has a rank-5 array \var{C}, then \code{C[...,0,...]} is
the same thing as \code{C[:,:,:,0,:]}.

When assigment source and destination locations overlap, i.e. when an array is
assigned onto itself at overlapping indices, it may produce something
"surprising":

\begin{verbatim}
>>> n=numarray.arange(36)
>>> n[11:18]=n[7:14]
>>> n
array([ 0,  1,  2,  3,  4,  5,  6,  7,  8,  9, 10,  7,  8,  9, 10,  7,
        8,  9, 18, 19, 20, 21, 22, 23, 24, 25, 26, 27, 28, 29, 30, 31,
       32, 33, 34, 35])
\end{verbatim}

If the slice on the right hand side (RHS) is AFTER that on the left hand side
(LHS) for 1-D array, then it works fine:

\begin{verbatim}
>>> n=numarray.arange(36)
>>> n[1:8]=n[7:14]       
>>> n
array([ 0,  7,  8,  9, 10, 11, 12, 13,  8,  9, 10, 11, 12, 13, 14, 15,
       16, 17, 18, 19, 20, 21, 22, 23, 24, 25, 26, 27, 28, 29, 30, 31,
       32, 33, 34, 35])
\end{verbatim}

Actually, this behavior can be undedrstood if we follow the pixel by pixel
copying logic.  Parts of the slice start to get the "updated" values when the
RHS is before the LHS.

An easy solution which is guaranteed to work is to use the copy() method on the
righ hand side:

\begin{verbatim}
>>> n=numarray.arange(36)
>>> n[11:18]=n[7:14].copy()
>>> n
array([ 0,  1,  2,  3,  4,  5,  6,  7,  8,  9, 10,  7,  8,  9, 10, 11,
       12, 13, 18, 19, 20, 21, 22, 23, 24, 25, 26, 27, 28, 29, 30, 31,
       32, 33, 34, 35])
\end{verbatim}

\newpage
\section{Pseudo Indices}
This section discusses pseudo-indices, which allow arrays to have their shapes
modified by adding axes, sometimes only for the duration of the evaluation of a
Python expression.

Consider multiplication of a rank-1 array by a scalar:
\begin{verbatim}
>>> a = array([1,2,3])
>>> print a * 2
[2 4 6]
\end{verbatim}
This should be trivial by now; we've just multiplied a rank-1 array by a
scalar . The scalar was converted to a rank-0 array which was then broadcast to
the next rank. This works for adding some two rank-1 arrays as well:
\begin{verbatim}
>>> print a
[1 2 3]
>>> a + array([4])
[5 6 7]
\end{verbatim}
but it won't work if either of the two rank-1 arrays have non-matching
dimensions which aren't 1.  In other words, broadcast only works for
dimensions which are either missing (e.g. a lower-rank array) or for dimensions
of 1.

With this in mind, consider a classic task, matrix multiplication. Suppose we
want to multiply the row vector [10,20] by the column vector [1,2,3].
\begin{verbatim}
>>> a = array([10,20])
>>> b = array([1,2,3])
>>> a * b
ValueError: Arrays have incompatible shapes
\end{verbatim}
% In "This makes sense - we're ..." the hyphen disappears in the PDF.
This makes sense: we're trying to multiply a rank-1 array of shape (2,) with a
rank-1 array of shape (3,). This violates the laws of broadcast. What we really
want to do is make the second vector a vector of shape (3,1), so that the first
vector can be broadcast across the second axis of the second vector. One way to
do this is to use the reshape function:
\begin{verbatim}
>>> a.getshape()
(2,)
>>> b.getshape()
(3,)
>>> b2 = reshape(b, (3,1))
>>> print b2
[[1]
 [2]
 [3]]
>>> b2.getshape()
(3, 1)
>>> print a * b2    # Note: b2 * a gives the same result
[[10 20]
 [20 40]
 [30 60]]
\end{verbatim}
This is such a common operation that a special feature was added (it turns out
to be useful in many other places as well) --� the NewAxis "pseudo-index",
originally developed in the Yorick language. NewAxis is an index, just like
integers, so it is used inside of the slice brackets []. It can be thought of
as meaning "add a new axis here," in much the same ways as adding a 1 to an
array's shape adds an axis. Again, examples help clarify the situation:
\begin{verbatim}
>>> print b
[1 2 3]
>>> b.getshape()
(3,)
>>> c = b[:, NewAxis]
>>> print c
[[1]
 [2]
 [3]]
>>> c.getshape()
(3,1)
\end{verbatim}
Why use such a pseudo-index over the reshape function or setshape assignments?
Often one doesn't really want a new array with a new axis, one just wants it
for an intermediate computation. Witness the array multiplication mentioned
above, without and with pseudo-indices:
\begin{verbatim}
>>> without = a * reshape(b, (3,1))
>>> with = a * b[:,NewAxis]
\end{verbatim}
The second is much more readable (once you understand how NewAxis works), and
it's much closer to the intended meaning. Also, it's independent of the
dimensions of the array b. You might counter that using something like
reshape(b, (-1,1)) is also dimension-independent, but 
it's less readable and impossible with rank-3 or higher arrays? The
NewAxis-based idiom also works nicely with higher rank arrays, and with the ...
"rubber index" mentioned earlier. Adding an axis before the last axis in an
array can be done simply with:
\begin{verbatim}
>>> a[...,NewAxis,:]
\end{verbatim}
Note that \code{NewAxis} is a \code{numarray} object, so if you used 
\code{import numarray} instead of \code{from numarray import *}, you'll 
need \code{numarray.NewAxis}.

\newpage
\section{Index Arrays}
\label{sec:index-arrays}

Arrays used as subscripts have special meanings which implicitly invoke the
functions \function{put} (page \pageref{func:put}), \function{take} (page
\pageref{func:take}), or \function{compress} (page \pageref{func:compress}). If
the array is of \class{Bool} type, then the indexing will be treated as the
equivalent of the compress function. If the array is of an integer type, then a
\function{take} or \function{put} operation is implied. We will generalize the
existing take and put as follows: If \var{ind1}, \var{ind2}, ...  \var{indN}
are index arrays (arrays of integers whose values indicate the index into
another array), then \code{x[ind1, ind2]} forms a new array with the same shape
as \var{ind1}, \var{ind2} (they all must be broadcastable to the same shape)
and values such: \samp{result[i,j,k] = x[ind1[i,j,k], ind2[i,j,k]]} In this
example, \var{ind1}, \var{ind2} are index arrays with 3 dimensions (but they
could have an arbitrary number of dimensions).  To illustrate with some
specific examples:
\begin{verbatim}
>>> x=2*arange(10)
>>> ind1=[0,4,3,7]
>>> x[ind1]
array([ 0,  8,  6, 14])
>>> ind1=[[0,4],[3,7]]
>>> x[ind1]
array([[ 0,  8],
       [ 6, 14]])
\end{verbatim}
This shows that the same elements in the same order are extracted from x by
both forms of ind1, but the result shares the shape of ind1 Something similar
happens in two dimensions:
\begin{verbatim}
>>> x=reshape(arange(12),(3,4))
>>> x
array([[ 0,  1,  2,  3],
       [ 4,  5,  6,  7],
       [ 8,  9, 10, 11]])
>>> ind1=array([2,1])
>>> ind2=array([0,3])
>>> x[ind1,ind2]
array([8, 7])
\end{verbatim}
Notice this pulls out x[2,0] and x[1,3] as a one-dimensional array.
\begin{verbatim}
>>> ind1=array([[2,2],[1,0]])
>>> ind2=array([[0,1],[3,2]])
>>> x[ind1,ind2]
array([[8, 9],
       [7, 2]])
\end{verbatim}
This pulls out x[2,0], x[2,1], x[1,3], and x[0,2], reading the ind1 and ind2
arrays left to right, and then reshapes the result to the same (2,2) shape as
ind1 and ind2 have.
\begin{verbatim}
>>> ind1.shape=(4,)
>>> ind2.shape=(4,)
>>> x[ind1,ind2]
array([8, 9, 7, 2])
\end{verbatim}

\newpage
Notice this is the same values in the same order, but now as a one-d array.
One index array does a broadcast:
\begin{verbatim}
>>> x[ind1]
array([[ 8,  9, 10, 11],
       [ 8,  9, 10, 11],
       [ 4,  5,  6,  7],
       [ 0,  1,  2,  3]])
>>> ind1.shape=(2,2)
>>> x[ind1]
array([[[ 8,  9, 10, 11],
        [ 8,  9, 10, 11]],

       [[ 4,  5,  6,  7],
        [ 0,  1,  2,  3]]])
\end{verbatim}

Again, note that the same 'elements', in this case rows of x, are returned in
both cases.  But in the second case, ind1 had two dimensions, and so using it
to index only one dimension of a two-d array results in a three-d output of
shape (2,2,4);  i.e., a 2 by 2 'array' of 4-element rows.

When using constants for some of the index positions, then the result uses that
constant for all values. Slices and strides (at least initially) will not be
permitted in the same subscript as index arrays. So
\begin{verbatim}
>>> x[ind1, 2]
array([[10, 10],
  [ 6,  2]])
\end{verbatim}
would be legal, but
\begin{verbatim}
>>> x[ind1, 1:3]
Traceback (most recent call last):
...
IndexError: Cannot mix arrays and slices as indices
\end{verbatim}
would not be.  Similarly for assignment:
\begin{verbatim}
x[ind1, ind2, ind3] = values
\end{verbatim}
will form a new array such that:
\begin{verbatim}
x[ind1[i,j,k], ind2[i,j,k], ind3[i,j,k]] = values[i,j,k]
\end{verbatim}

The index arrays and the value array must be broadcast consistently. (As an
example: \code{ind1.setshape((5,4))}, \code{ind2.setshape((5,))},
\code{ind3.setshape((1,4))}, and \code{values.setshape((1,))}.)
\begin{verbatim}
>>> x=zeros((10,10))
>>> x[[2,5,6],array([0,1,9,3])[:,NewAxis]]=array([1,2,3,4])[:,NewAxis]
>>> x
array([[0, 0, 0, 0, 0, 0, 0, 0, 0, 0],
       [0, 0, 0, 0, 0, 0, 0, 0, 0, 0],
       [1, 2, 0, 4, 0, 0, 0, 0, 0, 3],
       [0, 0, 0, 0, 0, 0, 0, 0, 0, 0],
       [0, 0, 0, 0, 0, 0, 0, 0, 0, 0],
       [1, 2, 0, 4, 0, 0, 0, 0, 0, 3],
       [1, 2, 0, 4, 0, 0, 0, 0, 0, 3],
       [0, 0, 0, 0, 0, 0, 0, 0, 0, 0],
       [0, 0, 0, 0, 0, 0, 0, 0, 0, 0],
       [0, 0, 0, 0, 0, 0, 0, 0, 0, 0]])
\end{verbatim}
If indices are repeated, the last value encountered will be stored.  When an
index is too large, Numarray raises an IndexError exception. When an index is
negative, Numarray will interpret it in the usual Python style, counting
backwards from the end.  Use of the equivalent \index{take}\function{take} and
\index{put}\function{put} functions will allow other interpretations of the
indices (clip out of bounds indices, allow negative indices to work backwards
as they do when used individually, or for indices to wrap around). The same
behavior applies for functions such as choose and where.

%% Local Variables:
%% mode: LaTeX
%% mode: auto-fill
%% fill-column: 79
%% indent-tabs-mode: nil
%% ispell-dictionary: "american"
%% reftex-fref-is-default: nil
%% TeX-auto-save: t
%% TeX-command-default: "pdfeLaTeX"
%% TeX-master: "numarray"
%% TeX-parse-self: t
%% End:

\chapter{Intermediate Topics}
\label{cha:intermediate-topics}

This chapter discusses a few of the more esoteric features of numarray which
are certainly useful but probably not a top priority for new users.

\section{Rank-0 Arrays}
\label{sec:rank-0-arrays}
numarray provides limited support for dimensionless arrays which represent
single values, also known as rank-0 arrays.  Rank-0 arrays are the array
representation of a scalar value.  They have the advantage over scalars that
they include array specific type information, e.g. \var{Int16}.  Rank-0 arrays
can be created as follows:
\begin{verbatim}
>>> a = array(1); a
array(1)
\end{verbatim}
A rank-0 array has a 0-length or empty shape:
\begin{verbatim}
>>> a.shape
()
\end{verbatim}
numarray's rank-0 array handling differs from that of Numeric in two ways.
First, numarray's rank-0 arrays cannot be indexed by 0:
\begin{verbatim}
>>> array(1)[0]
Traceback (most recent call last):
...
IndexError: Too many indices
\end{verbatim}
Second, numarray's rank-0 arrays do not have a length.
\begin{verbatim}
>>> len(array(1))
Traceback (most recent call last):
...
ValueError: Rank-0 array has no length.
\end{verbatim}
Finally, numarray's rank-0 arrays can be converted to a Python scalar by
subscripting with an empty tuple as follows:
\begin{verbatim}
>>> a = array(1)
>>> a[()]
1
\end{verbatim}

\newpage
\section{Exception Handling}
\label{sec:exception-handling}

We desired better control over exception handling than currently exists in
Numeric. This has traditionally been a problem area (see the numerous posts in
\ulink{comp.lang.python}{news:comp.lang.python} regarding floating point
exceptions, especially those by Tim Peters). Numeric raises an exception for
integer computations that result in a divide by zero or multiplications that
result in overflows. The exception is raised after that operation has completed
on all the array elements. No exceptions are raised for floating point errors
(divide by zero, overflow, underflow, and invalid results), the compiler and
processor are left to their default behavior (which is usually to return Infs
and NaNs as values).

The approach for numarray is to provide customizable error handling behavior.
It should be possible to specify three different behaviors for each of the four
error types independently. These are:
\begin{itemize}
\item Ignore the error.
\item Print a warning.
\item Raise a Python exception.
\end{itemize}
The current implementation does that and has been tested successfully on
Windows, Solaris, Redhat and Tru64.  The implementation uses the floating point
processor ``sticky status flags'' to detect errors. One can set the error mode
by calling the error object's setMode method. For example:
\begin{verbatim}
>>> Error.setMode(all="warn") # the default mode
>>> Error.setMode(dividebyzero="raise", underflow="ignore", invalid="warn")
\end{verbatim}

The Error object can also be used in a stacking manner, by using the \function{pushMode}
and \function{popMode} methods rather than \function{setMode}.  For example:
\begin{verbatim}
>>> Error.getMode()
_NumErrorMode(overflow='warn', underflow='warn', dividebyzero='warn', invalid='warn')
>>> Error.pushMode(all="raise") # get really picky...
>>> Error.getMode()
_NumErrorMode(overflow='raise', underflow='raise', dividebyzero='raise', invalid='raise')
>>> Error.popMode()  # pop and return the ``new'' mode
_NumErrorMode(overflow='raise', underflow='raise', dividebyzero='raise', invalid='raise')
>>> Error.getMode()  # verify the original mode is back
_NumErrorMode(overflow='warn', underflow='warn', dividebyzero='warn', invalid='warn')
\end{verbatim}
Integer exception modes work the same way. Although integer computations do not
affect the floating point status flag directly, our code checks the denominator
of 0 in divisions (in much the same way Numeric does) and then performs a
floating point divide by zero to set the status flag (overflows are handled
similarly). So even integer exceptions use the floating point status flags
indirectly.

\newpage
\section{IEEE-754 Not a Number (NAN) and Infinity}
\label{sec:ieee-special-values}
\module{numarray.ieeespecial} has support for manipulating IEEE-754 floating
point special values NaN (Not a Number), Inf (infinity), etc.  The special
values are denoted using lower case as follows:
\begin{verbatim}
>>> import numarray.ieeespecial as ieee
>>> ieee.inf
inf
>>> ieee.plus_inf
inf
>>> ieee.minus_inf
-inf
>>> ieee.nan
nan
>>> ieee.plus_zero
0.0
>>> ieee.minus_zero
-0.0
\end{verbatim}
Note that the representation of IEEE special values is platform dependent so
your Python might for instance say \var{Infinity} rather than \var{inf}.
Below, \var{inf} is seen to arise as the result of floating point division by 0
and \var{nan} is seen to arise from 0 divided by 0:
\begin{verbatim}
>>> a = array([0.0, 1.0])
>>> b = a/0.0
Warning: Encountered invalid numeric result(s)  in divide
Warning: Encountered divide by zero(s)  in divide
>>> b
array([              nan,               inf])
\end{verbatim}
A curious property of \var{nan} is that it does not compare to \emph{itself} as
equal:
\begin{verbatim}
>>> b == ieee.nan
array([0, 0], type=Bool)
\end{verbatim}
The \function{isnan}, \function{isinf}, and \function{isfinite} functions
return boolean arrays which have the value True where the corresponding
predicate holds.  These functions detect bit ranges and are therefore more
robust than simple equality checks.
\begin{verbatim}
>>> ieee.isnan(b)
array([1, 0], type=Bool)
>>> ieee.isinf(b)
array([0, 1], type=Bool)
>>> ieee.isfinite(b)
array([0, 0], type=Bool)
\end{verbatim}
Array based indexing provides a convenient way to replace special values:
\begin{verbatim}
>>> b[ieee.isnan(b)] = 999
>>> b[ieee.isinf(b)] = 5
>>> b
array([ 999.,    5.])
\end{verbatim}

Here's an easy approach for compressing your data arrays to remove
NaNs:
\begin{verbatim}
>>> x, y = arange(10.), arange(10.)
>>> x[5] = ieee.nan
>>> y[6] = ieee.nan
>>> keep = ~ieee.isnan(x) & ~ieee.isnan(y)
>>> x[keep]
array([ 0.,  1.,  2.,  3.,  4.,  7.,  8.,  9.])
>>> y[keep]
array([ 0.,  1.,  2.,  3.,  4.,  7.,  8.,  9.])
\end{verbatim}

%% Local Variables:
%% mode: LaTeX
%% mode: auto-fill
%% fill-column: 79
%% indent-tabs-mode: nil
%% ispell-dictionary: "american"
%% reftex-fref-is-default: nil
%% TeX-auto-save: t
%% TeX-command-default: "pdfeLaTeX"
%% TeX-master: "numarray"
%% TeX-parse-self: t
%% End:

\chapter{Ufuncs}
\label{cha:ufuncs}

\section{What are Ufuncs?}
\label{sec:what-are-ufuncs}

The operations on arrays that were mentioned in the previous section
(element-wise addition, multiplication, etc.) all share some features --- they
all follow similar rules for broadcasting, coercion and ``element-wise
operation''. Just as standard addition is available in Python through the add
function in the operator module, array operations are available through
callable objects as well. Thus, the following objects are available in the
numarray module:

\begin{table}[htbp]
   \centering
   \caption{Universal Functions, or ufuncs. The operators which invoke them
   when applied to arrays are indicated in parentheses. The entries in slanted
   typeface refer to unary ufuncs, while the others refer to binary ufuncs.} 
   \label{tab:ufuncs}
   \begin{tabular}{llll}
      add ($+$)         & subtract ($-$)   & multiply (*)       & divide ($/$) \\
      remainder (\%)    & power (**)       & \textsl{arccos}    & \textsl{arccosh} \\
      \textsl{arcsin}   & \textsl{arcsinh} & \textsl{arctan}    & \textsl{arctanh} \\
      \textsl{cos}      & \textsl{cosh}    & \textsl{tan}       & \textsl{tanh} \\
      \textsl{log10}    & \textsl{sin}     & \textsl{sinh}      & \textsl{sqrt} \\
      \textsl{absolute (abs)} & \textsl{fabs}    & \textsl{floor}     & \textsl{ceil} \\
      fmod              & \textsl{exp}     & \textsl{log}       & \textsl{conjugate} \\
      maximum           & minimum \\
      greater ($>$)     & greater\_equal ($>=$) & equal ($==$)  \\
      less ($<$)        & less\_equal ($<=$)  & not\_equal ($!=$) \\
      logical\_or       & logical\_xor     & logical\_not  & logical\_and \\
      bitwise\_or ($|$) & bitwise\_xor (\^{}) 
                        & bitwise\_not (\textasciitilde)  & bitwise\_and (\&)
      \\
      rshift ($>>$)       & lshift ($<<$)
   \end{tabular}
\end{table}

\remark{Add a remark here on how numarray does (or will) handle 'true'
and 'floor' division, which can be activated in Python 2.2 with
\samp{from __future__ import division}?.
Note: with 'true' division, \samp{1/2 == 0.5} and not \samp{0}.}

All of these ufuncs can be used as functions. For example, to use
\function{add}, which is a binary ufunc (i.e.\ it takes two arguments), one can
do either of:
\begin{verbatim}
>>> a = arange(10)
>>> print add(a,a)
[ 0  2  4  6  8 10 12 14 16 18]
>>> print a + a
[ 0  2  4  6  8 10 12 14 16 18]
\end{verbatim}
In other words, the \code{+} operator on arrays performs exactly the same thing
as the \function{add} ufunc when operated on arrays.  For a unary ufunc such as
\function{sin}, one can do, e.g.:
\begin{verbatim}
>>> a = arange(10)
>>> print sin(a)
[ 0.          0.84147096  0.90929741  0.14112    -0.7568025
      -0.95892429 -0.27941549  0.65698659  0.98935825  0.41211849]
\end{verbatim}
A unary ufunc returns an array with the same shape as its argument array, but
with each element replaced by the application of the function to that element
(\code{sin(0)=0}, \code{sin(1)=0.84147098}, etc.).

There are three additional features of ufuncs which make them different from
standard Python functions. They can operate on any Python sequence in addition
to arrays; they can take an ``output'' argument; they have methods which are
themselves callable with arrays and sequences. Each of these will be described
in turn.

Ufuncs can operate on any Python sequence. Ufuncs have so far been described as
callable objects which take either one or two arrays as arguments (depending on
whether they are unary or binary). In fact, any Python sequence which can be
the input to the \function{array} constructor can be used.  The return value
from ufuncs is always an array.  Thus:
\begin{verbatim}
>>> add([1,2,3,4], (1,2,3,4))
array([2, 4, 6, 8])
\end{verbatim}


\subsection{Ufuncs can take output arguments}
\label{sec:ufuncs-can-take}

In many computations with large sets of numbers, arrays are often used only
once. For example, a computation on a large set of numbers could involve the
following step
\begin{verbatim}
dataset = dataset * 1.20 
\end{verbatim}
This can also be written as the following using the Ufunc form:
\begin{verbatim}
dataset = multiply(dataset, 1.20)
\end{verbatim}
In both cases, a temporary array is created to store the results of the
computation before it is finally copied into \var{dataset}. It is
more efficient, both in terms of memory and computation time, to do an
``in-place'' operation.  This can be done by specifying an existing array as
the place to store the result of the ufunc. In this example, one can 
write:\footnote[1]{for Python-2.2.2 or later: `dataset *= 1.20' also works}
\begin{verbatim}
multiply(dataset, 1.20, dataset) 
\end{verbatim}
This is not a step to take lightly, however. For example, the ``big and slow''
version (\code{dataset = dataset * 1.20}) and the ``small and fast'' version
above will yield different results in at least one case:
\begin{itemize}
\item If the type of the target array is not that which would normally be
   computed, the operation will not coerce the array to the expected data type.
   (The result is done in the expected data type, but coerced back to the
   original array type.)
\item Example:
\begin{verbatim}
\>>> a=arange(5,type=Int32)
>>> print a[::-1]*1.7
[ 6.8  5.1  3.4  1.7  0. ]
>>> multiply(a[::-1],1.7,a)
>>> print a
[6 5 3 1 0]
>>> a *= 1.7
>>> print a
[0 1 3 5 6]
\end{verbatim}
\end{itemize}

The output array does not need to be the same variable as the input array. In
numarray, in contrast to Numeric, the output array may have any type (automatic
conversion is performed on the output).

\subsection{Ufuncs have special methods}
\label{sec:ufuncs-have-special-methods}


\begin{methoddesc}{reduce}{a, axis=0}
   If you don't know about the \function{reduce} command in Python, review
   section 5.1.3 of the Python Tutorial
   (\url{http://www.python.org/doc/current/tut/}). Briefly,
   \function{reduce} is most often used with two arguments, a callable object
   (such as a function), and a sequence. It calls the callable object with the
   first two elements of the sequence, then with the result of that operation
   and the third element, and so on, returning at the end the successive
   ``reduction'' of the specified callable object over the sequence elements.
   Similarly, the \method{reduce} method of ufuncs is called with a sequence as
   an argument, and performs the reduction of that ufunc on the sequence. As an
   example, adding all of the elements in a rank-1 array can be done with:
\begin{verbatim}
>>> a = array([1,2,3,4])
>>> print add.reduce(a)   # with Python's reduce, same as reduce(add, a)
10
\end{verbatim}
   When applied to arrays which are of rank greater than one, the reduction
   proceeds by default along the first axis:
\begin{verbatim}
>>> b = array([[1,2,3,4],[6,7,8,9]])
>>> print b
[[1 2 3 4]
 [6 7 8 9]]
>>> print add.reduce(b)
[ 7  9 11 13]
\end{verbatim}
   A different axis of reduction can be specified with a second integer
   argument:
\begin{verbatim}
>>> print b
[[1 2 3 4]
 [6 7 8 9]]
>>> print add.reduce(b, axis=1)
[10 30]
\end{verbatim}
\end{methoddesc}


\begin{methoddesc}{accumulate}{a}
   The \method{accumulate} ufunc method is simular to \method{reduce}, except
   that it returns an array containing the intermediate results of the
   reduction:
\begin{verbatim}
>>> a = arange(10)
>>> print a
[0 1 2 3 4 5 6 7 8 9]
>>> print add.accumulate(a)
[ 0  1  3  6 10 15 21 28 36 45] # 0, 0+1, 0+1+2, 0+1+2+3, ... 0+...+9
>>> print add.reduce(a) # same as add.accumulate(a)[-1] w/o side effects on a
45                                      
\end{verbatim}
\end{methoddesc}


\begin{methoddesc}{outer}{a, b}
   The third ufunc method is \method{outer}, which takes two arrays as
   arguments and returns the ``outer ufunc'' of the two arguments. Thus the
   \method{outer} method of the \function{multiply} ufunc, results in the outer
   product. The \method{outer} method is only supported for binary methods.
\begin{verbatim}
>>> print a
[0 1 2 3 4]
>>> print b
[0 1 2 3]
>>> print add.outer(a,b)
[[0 1 2 3]
 [1 2 3 4]
 [2 3 4 5]
 [3 4 5 6]
 [4 5 6 7]]
>>> print multiply.outer(b,a)
[[ 0  0  0  0  0]
 [ 0  1  2  3  4]
 [ 0  2  4  6  8]
 [ 0  3  6  9 12]]
>>> print power.outer(a,b)
[[ 1  0  0  0]
 [ 1  1  1  1]
 [ 1  2  4  8]
 [ 1  3  9 27]
 [ 1  4 16 64]]
\end{verbatim}
\end{methoddesc}


\begin{methoddesc}{reduceat}{}
   The reduceat method of Numeric has not been implemented in numarray.
\end{methoddesc}

\subsection{Ufuncs always return new arrays}
\label{sec:ufuncs-always-return}

Except when the output argument is used as described above, ufuncs always
return new arrays which do not share any data with the input arrays.


\section{Which are the Ufuncs?}
\label{sec:which-are-ufuncs}

Table \ref{tab:ufuncs} lists all the ufuncs. We will first discuss the
mathematical ufuncs, which perform operations very similar to the functions in
the \module{math} and \module{cmath} modules, albeit elementwise, on arrays.
Originally,  numarray ufuncs came in two forms, unary and binary.  More
recently (1.3) numarray has added support for ufuncs with up to 16 total
input or output parameters.  These newer ufuncs are called N-ary ufuncs.

\subsection{Unary Mathematical Ufuncs}
\label{sec:unary-math-ufuncs}

Unary ufuncs take only one argument.  The following ufuncs apply the
predictable functions on their single array arguments, one element at a time:
\function{arccos}, \function{arccosh}, \function{arcsin}, \function{arcsinh},
\function{arctan}, \function{arctanh}, \function{cos}, \function{cosh},
\function{exp}, \function{log}, \function{log10}, \function{sin},
\function{sinh}, \function{sqrt}, \function{tan}, \function{tanh},
\function{abs}, \function{fabs}, \function{floor}, \function{ceil},
\function{conjugate}.  As an example:
\begin{verbatim}
>>> print x
[0 1 2 3 4]
>>> print cos(x)
[ 1.          0.54030231 -0.41614684 -0.9899925  -0.65364362]
>>> print arccos(cos(x))
[ 0.          1.          2.          3.          2.28318531]
# not a bug, but wraparound: 2*pi%4 is 2.28318531
\end{verbatim}


\subsection{Binary Mathematical Ufuncs}
\label{sec:binary-math-ufuncs}

These ufuncs take two arrays as arguments, and perform the specified
mathematical operation on them, one pair of elements at a time: \function{add},
\function{subtract}, \function{multiply}, \function{divide},
\function{remainder}, \function{power}, \function{fmod}.


\subsection{Logical and bitwise ufuncs}
\label{sec:logical-ufuncs}

The ``logical'' ufuncs also perform their operations on arrays or numbers 
in elementwise fashion, just like the "mathematical" ones.  Two are special
(\function{maximum} and \function{miminum}) in that they return arrays with
entries taken from their input arrays:
\begin{verbatim}
>>> print x
[0 1 2 3 4]
>>> print y
[ 2.   2.5  3.   3.5  4. ]
>>> print maximum(x, y)
[ 2.   2.5  3.   3.5  4. ]
>>> print minimum(x, y)
[ 0.  1.  2.  3.  4.]
\end{verbatim}
The others logical ufuncs return arrays of 0's or 1's and of type Bool:
\function{logical_and}, \function{logical_or}, \function{logical_xor},
\function{logical_not}, 
These are fairly
self-explanatory, especially with the associated symbols from the standard
Python version of the same operations in Table \ref{tab:ufuncs} above. 
The bitwise ufuncs,
\function{bitwise_and}, \function{bitwise_or},
\function{bitwise_xor}, \function{bitwise_not},  
\function{lshift}, \function{rshift},  
on the other hand, only work with integer arrays (of any word size), and
will return integer arrays of the larger bit size of the two input arrays:
\begin{verbatim}
>>> x
array([7, 7, 0], type=Int8)
>>> y
array([4, 5, 6])
>>> x & y          # bitwise_and(x,y)
array([4, 5, 0])
>>> x | y          # bitwise_or(x,y)
array([7, 7, 6])   
>>> x ^ y          # bitwise_xor(x,y)
array([3, 2, 6]) 
>>> ~ x            # bitwise_not(x)
array([-8, -8, -1], type=Int8)

\end{verbatim}
We discussed finding contents of arrays based on arrays' indices by using slice.
Often, especially when dealing with the result of computations or data
analysis, one needs to ``pick out'' parts of matrices based on the content of
those matrices. For example, it might be useful to find out which elements of
an array are negative, and which are positive. The comparison ufuncs are
designed for such operation. Assume an array with various positive
and negative numbers in it (for the sake of the example we'll generate it from
scratch):
\begin{verbatim}
>>> print a
[[ 0  1  2  3  4]
 [ 5  6  7  8  9]
 [10 11 12 13 14]
 [15 16 17 18 19]
 [20 21 22 23 24]]
>>> b = sin(a)
>>> print b
[[ 0.          0.84147098  0.90929743  0.14112001 -0.7568025 ]
 [-0.95892427 -0.2794155   0.6569866   0.98935825  0.41211849]
 [-0.54402111 -0.99999021 -0.53657292  0.42016704  0.99060736]
 [ 0.65028784 -0.28790332 -0.96139749 -0.75098725  0.14987721]
 [ 0.91294525  0.83665564 -0.00885131 -0.8462204  -0.90557836]]
>>> print greater(b, .3)
[[0 1 1 0 0]
 [0 0 1 1 1]
 [0 0 0 1 1]
 [1 0 0 0 0]
 [1 1 0 0 0]]
\end{verbatim}


\subsection{Comparisons}
\label{sec:comparisons}

The comparison functions \function{equal}, \function{not_equal},
\function{greater}, \function{greater_equal}, \function{less}, and
\function{less_equal} are invoked by the operators \code{==}, \code{!=},
\code{>}, \code{>=}, \code{<}, and \code{<=} respectively, but they can also be
called directly as functions. Continuing with the preceding example,
\begin{verbatim}
>>> print less_equal(b, 0)
[[1 0 0 0 1]
 [1 1 0 0 0]
 [1 1 1 0 0]
 [0 1 1 1 0]
 [0 0 1 1 1]]
\end{verbatim}
This last example has 1's where the corresponding elements are less than or
equal to 0, and 0's everywhere else.

The operators and the comparison functions are not exactly equivalent.  To
compare an array a with an object b, if b can be converted to an array, the
result of the comparison is returned. Otherwise, zero is returned. This means
that comparing a list and comparing an array can return quite different
answers. Since the functional forms such as equal will try to make arrays from
their arguments, using equal can result in a different result than using
\code{==}.
\begin{verbatim}
>>> a = array([1, 2, 3])
>>> b = [1, 2, 3]
>>> print a == 2
[0 1 0]
>>> print b == 2  
0          # (False since 2.3)
>>> print equal(a, 2)
[0 1 0]
>>> print equal(b, 2)
[0 1 0]
\end{verbatim}

\subsection{Ufunc shorthands}
\label{sec:ufunc-shorthands}

Numarray defines a few functions which correspond to popular ufunc methods:
for example, \function{add.reduce} is synonymous with the \function{sum}
utility function:
\begin{verbatim}
>>> a = arange(5)                       # [0 1 2 3 4]
>>> print sum(a)                        # 0 + 1 + 2 + 3 + 4
10
\end{verbatim}
Similarly, \function{cumsum} is equivalent to \function{add.accumulate} (for
``cumulative sum''), \function{product} to \function{multiply.reduce}, and
\function{cumproduct} to \function{multiply.accumulate}.  Additional useful
``utility'' functions are \function{all} and \function{any}:
\begin{verbatim}
>>> a = array([0,1,2,3,4])
>>> print greater(a,0)
[0 1 1 1 1]
>>> all(greater(a,0))
0
>>> any(greater(a,0))
1
\end{verbatim}

\section{Writing your own ufuncs!}

This section describes a new process for defining your own universal functions.
It explains a new interface that enables the description of N-ary ufuncs, those
that use semi-arbitrary numbers \((<= 16)\) of inputs and outputs.

\subsection{Runtime components of a ufunc}

A numarray universal function maps from a Python function name to a set of C
functions.  Ufuncs are polymorphic and figure out what to do in C when passed a
particular set of input parameter types.  C functions, on the other hand, can
only be run on parameters which match their type signatures.  The task of
defining a universal function is one of describing how different parameter
sequences are mapped from Python array types to C function signatures and back.

At runtime, there are three principle kinds of things used to define a
universal function.

\begin {enumerate}
\item Ufunc 

The universal function is itself a callable Python object.  Ufuncs organize a
collection of Cfuncs to be called based on the actual parameter types seen at
runtime.  The same Ufunc is typically associated with several Cfuncs each of
which handles a unique Ufunc type signature.  Because a Ufunc typically has
more than one C func, it can also be implemented using more than one library
function.

\item Library function

A pre-existing function written in C or Fortran which will ultimately be called
for each element of the ufunc parameter arrays.  

\item Cfunc

Cfuncs are binding objects that map C library functions safely into Python.
It's the job of a Cfunc to interpret typeless pointers corresponding to the
parameter arrays as particular C data types being passed down from the ufunc.
Further, the Cfunc casts array elements from the input type to the Libraray
function parameter type.  This process lets the ufunc implementer describe the
ufunc type signatures which will be processed most efficiently by the
underlying Library function by enabling per-call element-by-element type casts.
Ufunc calling signatures which are not represented directly by a Cfunc result
in blockwise coercion to the closest matching Cfunc, which is slower.

\end {enumerate}

\subsection{Source components of a ufunc}
There are 4 source components required to define numarray ufuncs, one of which
is hand written, two are generated, and the last is assumed to be pre-existing:

\begin {enumerate}
\item Code generation script

The primary source component for defining new universal functions is a Python
script used to generate the other components.  For a standalone set of
functions, putting the code generation directives in setup.py can be done as in
the example numarray/Examples/ufunc/setup_airy.py.  The code generation script
only executes at install time.

\item Extension module

A private extension module is generated which contains a collection of Cfuncs
for the package being created.  The extension module contains a dictionary
mapping from ufuncs/types to Cfuncs.

\item Ufunc init file 

A Python script used at ufunc import time is required to construct Ufunc
objects from Cfuncs.  This code is boilerplate created with the code generation
directive \function{make_stub()}.

\item Library functions

The C functions which are ultimately called by a Ufunc need to be defined
somewhere, typically in a third party C or Fortran library which is linked
to the Extension module.
\end{enumerate}

\subsection{Ufunc code generation}
There are several code generation directives provided by package
numarray.codegenerator which are called at installation time to generate the
Cfunc extension module and Ufunc init file.

\begin{funcdesc}{UfuncModule}{module_name}
The \class{UfuncModule} constructor creates a module object which collects code
which is later output to form the Cfunc extension module.  The name passed to
the constructor defines the name of the Python extension module, not the source
code.
\begin{verbatim}
m = UfuncModule("_na_special")
\end{verbatim}
\end{funcdesc}

\begin{methoddesc}{add_code}{code_string}
The \method{add_code()} method of a \class{UfuncModule} object is used to add
arbitrary code to the module at the point that \method{add_code()} is
called. Here it includes a header file used to define prototypes for the C
library functions which this extension will ultimately call.
\begin{verbatim}
m.add_code('#include "airy.h"')
\end{verbatim}
\end{methoddesc}

\begin{methoddesc}{add_nary_ufunc}{ufunc_name, c_name,
    ufunc_signatures, c_signature, forms=None} 
The \method{add_nary_ufunc()} method declares a Ufunc and relates it to one
library function and a collection of Cfunc bindings for it.  The
\var{signatures} parameter defines which ufunc type signatures receive Cfunc
bindings. Input types which don't match those signature are blockwise coerced
to the best matching signature.  \method{add_nary_ufunc()} can be called for
the same Ufunc name more than once and can thus be used to associate multiple
library functions with the same Ufunc.
\begin{verbatim}
m.add_nary_ufunc(ufunc_name = "airy",
                 c_function  = "airy",    
                 signatures  =["dxdddd",
                               "fxffff"],
                 c_signature = "dxdddd")
\end{verbatim}
\end{methoddesc}

\begin{methoddesc}{generate}{source_filename}
The \method{generate()} method asks the \class{UfuncModule} object to emit the
code for an extension module to the specified \var{source_filename}.
\begin{verbatim}
m.generate("Src/_na_specialmodule.c")
\end{verbatim}
\end{methoddesc}

\begin{funcdesc}{make_stub}{stub_filename, cfunc_extension, add_code=None}
The \function{make_stub()} function is used to generate the boilerplate Python
code which constructs universal functions from a Cfunc extension module at
import time.  \function{make_stub()} accepts a \var{add_code} parameter which
should be a string containing any additional Python code to be injected into
the stub module.  Here \function{make_stub()} creates the init file
``Lib/__init__.py'' associated with the Cfunc extension ``_na_special'' and
includes some extra Python code to define the \function{plot_airy()} function.
\begin{verbatim}
extra_stub_code = '''

import matplotlib.pylab as mpl

def plot_airy(start=-10,stop=10,step=0.1,which=1):
    a = mpl.arange(start, stop, step)
    mpl.plot(a, airy(a)[which])

    b = 1.j*a + a
    ba = airy(b)[which]

    h = mpl.figure(2)
    mpl.plot(b.real, ba.real)

    i = mpl.figure(3)
    mpl.plot(b.imag, ba.imag)
    
    mpl.show()
'''

make_stub("Lib/__init__", "_na_special", add_code=extra_stub_code)
\end{verbatim}

\end{funcdesc}

\subsection{Type signatures and signature ordering}

Type signatures are described using the single character typecodes from
Numeric.  Since the type signature and form of a Cfunc need to be encoded in
its name for later identification, it must be brief.  

\begin{verbatim}
typesignature ::= <inputtypes> + ``x'' + <outputtypes>
inputtypes ::= [<typecode>]+
outputtypes ::= [<typecode>]+
typecode ::= "B" | "1" | "b" | "s" | "w" | "i" | "u" |
             "N" | "U" | "f" | "d" | "F" | "D"
\end{verbatim}

For example,  the type signature corresponding to one Int32 input and one Int16
output is "ixs".

A type signature is a sequence of ordered types.  One signature can be compared
to another by comparing corresponding elements, in left to right order.
Individual elements are ranked using the order from the previous section.  A
ufunc maintains its associated Cfuncs as a sorted sequence and selects the
first Cfunc which is \(>=\) the input type signature;  this defines the notion
of ``best matching''.

\subsection{Forms}

The \method{add_nary_ufunc()} method has a parameter \var{forms} which enables
generation of code with some extra properties.  It specifies the list of
function forms for which dedicated code will be generated.  If you don't
specify \var{forms}, it defaults to a (list of a) single form which specifies
that all inputs and outputs corresponding to the type signature are vectors.
Input vectors are passed by value, output vectors are passed by reference.  The
default form implies that the library function return value, if there is one,
is ignored.  The following Python code shows the default form:

\begin{verbatim}
["v"*n_inputs + "x" + "v"*n_outputs] 
\end{verbatim}

Forms are denoted using a syntax very similar to, and typically symmetric with,
type signatures.

\begin{verbatim}

form ::=  <inputs> "x" <outputs>
inputs ::= ["v"|"s"]*
outputs ::= ["f"]?["v"]* | "A" | "R"

The form character values have different meanings than for type
signatures:

'v'  :   vector,  an array of input or output values
's'  :   scalar,  a non-array input value
'f'  :   function,  the c_function returns a value
'R'  :   reduce,    this binary ufunc needs a reduction method
'A'  :   accumulate this binary ufunc needs an accumulate method
'x'  :   separator  delineates inputs from outputs

\end{verbatim}

So, a form consists of some input codes followed by a lower case "x" followed
by some output codes.  

The form for a C function which takes 4 input values, the last of which is
assumed to be a scalar, returns one value, and fills in 2 additional output
values is:  "vvvsxfvv".

Using "s" to designate scalar parameters is a useful performance
optimization for cases where it is known that only a single value is
passed in from Python to be used in all calls to the c function.  This
prevents the blockwise expansion of the scalar value into a vector.

Use "f" to specify that the C function return value should be kept; it must
always be the first output and will therefore appear as the first element of
the result tuple.

For ufuncs of two input parameters (binary ufuncs), two additional form
characters are possible: A (accumulate) and R (reduce).  Each of these
characters constitutes the *entire* ufunc form, so the form is denoted "R" or
"A".  For these kinds of cfuncs, the type signature always reads \code{<t>x<t>}
where \code{<t>} is one of the type characters.  

One reason for all these codes is so that the many Cfuncs generated for Ufuncs
can be easily named.  The name for the Cfunc which implements \function{add()}
for two Int32 inputs and one Int32 output and where all parameters are arrays
is: "add_iixi_vvxv".  The cfunc name for \method{add.reduce()} with two integer
parameters would be written as "add_ixi_R" and for \method{add.accumulate()}
as "add_ixi_A".

The set of Cfuncs generated is based on the signatures \emph{crossed} with the
forms.  Multiple calls to \method{add_nary_ufunc()} can be used the reduce the
effects of signature/form crossing.

\newpage
\subsection{Ufunc Generation Example}

This section includes code from Examples/ufunc/setup_airy.py in the numarray
source distribution to illustrate how to create a package which defines your
own universal functions.  

This script eventually generates two files: _na_airymodule.c and
__init__.py.  The former defines an extension module which creates
numarray cfuncs, c helpers for the numarray airy() ufunc.  The latter
file includes Python code which automatically constructs numarray
universal functions (ufuncs) from the cfuncs in _na_airymodule.c.

\begin{verbatim}

import distutils, os, sys
from distutils.core import setup
from numarray.codegenerator import UfuncModule, make_stub
from numarray.numarrayext import NumarrayExtension

m = UfuncModule("_na_special")

m.add_code('#include "airy.h"')

m.add_nary_ufunc(ufunc_name = "airy",
                 c_function  = "airy",    
                 signatures  =["dxdddd",
                               "fxffff"],
                 c_signature = "dxdddd")

m.add_nary_ufunc(ufunc_name = "airy",
                 c_function  ="cairy_fake",
                 signatures  =["DxDDDD",
                               "FxFFFF"],
                 c_signature = "DxDDDD")

m.generate("Src/_na_specialmodule.c")

\end{verbatim}

\begin{verbatim}

extra_stub_code = '''
def plot_airy(start=-10,stop=10,step=0.1,which=1):
    import matplotlib.pylab as mpl;

    a = mpl.arange(start, stop, step);
    mpl.plot(a, airy(a)[which]);

    b = 1.j*a + a
    ba = airy(b)[which]

    h = mpl.figure(2)
    mpl.plot(b.real, ba.real)

    i = mpl.figure(3)
    mpl.plot(b.imag, ba.imag)
    
    mpl.show()
'''

make_stub("Lib/__init__", "_na_special", 
          add_code=extra_stub_code)

dist = setup(name = "na_special",
      version = "0.1",
      maintainer = "Todd Miller",
      maintainer_email = "jmiller@stsci.edu",
      description = "airy() universal function for numarray",
      url = "http://www.scipy.org/",
      packages = ["numarray.special"],
      package_dir = { "numarray.special":"Lib" },
      ext_modules = [ NumarrayExtension( 'numarray.special._na_special',
                                         ['Src/_na_specialmodule.c',
                                          'Src/airy.c',
                                          'Src/const.c',
                                          'Src/polevl.c']
                                        )
                     ]
     )

\end{verbatim}

Additional explanatory text is available in
numarray/Examples/ufunc/setup_airy.py.  Scripts used to extract
numarray ufunc specs from the existing Numeric ufunc definitions
in scipy.special are in numarray/Examples/ufunc/RipNumeric as an
example of how to convert existing Numeric code to numarray.



%% Local Variables:
%% mode: LaTeX
%% mode: auto-fill
%% fill-column: 79
%% indent-tabs-mode: nil
%% ispell-dictionary: "american"
%% reftex-fref-is-default: nil
%% TeX-auto-save: t
%% TeX-command-default: "pdfeLaTeX"
%% TeX-master: "numarray"
%% TeX-parse-self: t
%% End:

\chapter{Array Functions}
\label{cha:array-functions}

Most of the useful manipulations on arrays are done with functions. This might
be surprising given Python's object-oriented framework, and that many of these
functions could have been implemented using methods instead. Choosing functions
means that the same procedures can be applied to arbitrary python sequences,
not just to arrays. For example, while \code{transpose([[1,2],[3,4]])} works
just fine, \code{[[1,2],[3,4]].transpose()} does not. This approach also allows
uniformity in interface between functions defined in the numarray Python
system, whether implemented in C or in Python, and functions defined in
extension modules. We've already covered two functions which operate on arrays:
\code{reshape} and \code{resize}.

\begin{funcdesc}{take}{array, indices, axis=0, clipmode=CLIP}
   \label{sec:array-functions:take}
   \label{func:take}
   The function \code{take} is a generalized indexing/slicing of the array.  In 
   its simplest form, it is equivalent to indexing:
\begin{verbatim}
>>> a1 = array([10,20,30,40])
>>> print a1[[1,3]]
[20 40]
>>> print take(a1,[1,3])
[20 40]
\end{verbatim}
   See the description of index
   arrays in the Array Basics section for an alternative to \code{take} 
   and \code{put}. \code{take}
   selects the elements of the array (the first argument) based on the
   indices (the second argument). Unlike slicing, however, the array
   returned by \code{take} has the same rank as the input array.  
   Some 2-D examples:
\begin{verbatim}
>>> print a2
[[ 0  1  2  3  4]
 [ 5  6  7  8  9]
 [10 11 12 13 14]
 [15 16 17 18 19]]
>>> print take(a2, (0,))                 # first row
[[0 1 2 3 4]]
>>> print take(a2, (0,1))                # first and second row
[[0 1 2 3 4]
 [5 6 7 8 9]]
>>> print take (a2, (0, -1))             # index relative to dimension end
[[ 0 1 2 3 4]
[15 16 17 18 19]]
\end{verbatim}
   The optional third argument specifies the axis along which the selection
   occurs, and the default value (as in examples above) is 0, the first
   axis.  If you want a different axis, then you need to specify it:
\begin{verbatim}
>>> print take(a2, (0,), axis=1)         # first column
[[ 0]
 [ 5]
 [10]
 [15]]
>>> print take(a2, (0,1), axis=1)        # first and second column
[[ 0  1]
 [ 5  6]
 [10 11]
 [15 16]]
>>> print take(a2, (0,4), axis=1)        # first and last column
[[ 0  4]
 [ 5  9]
 [10 14]
 [15 19]]
\end{verbatim}
   
   This is considered to be a \var{structural} operation, because its 
   result does
   not depend on the content of the arrays or the result of a computation on
   those contents but uniquely on the structure of the array. Like all such
   structural operations, the default axis is 0 (the first rank). 
   Later in this tutorial, we will see other functions with a default axis 
   of -1.
   
   \function{take} is often used to create multidimensional arrays with the
   indices from a rank-1 array. As in the earlier examples, the shape of the
   array returned by \function{take} is a combination of the shape of its first
   argument and the shape of the array that elements are "taken" from �-- when
   that array is rank-1, the shape of the returned array has the same shape as
   the index sequence. This, as with many other facets of numarray, is best
   understood by experiment.
\begin{verbatim}
>>> x = arange(10) * 100
>>> print x
[  0 100 200 300 400 500 600 700 800 900]
>>> print take(x, [[2,4],[1,2]])
[[200 400]
 [100 200]]
\end{verbatim}
   A typical example of using \function{take} is to replace the grey values in
   an image according to a "translation table."  For example, suppose \code{m51}
   is a 2-D array of type \code{UInt8} containing a greyscale image. We can
   create a table mapping the integers 0--255 to integers 0--255 using a
   "compressive nonlinearity":
\begin{verbatim}
>>> table = (255 - arange(256)**2 / 256).astype(UInt8)
\end{verbatim}
   Then we can perform the take() operation
\begin{verbatim}
>>> m51b = take(table, m51)
\end{verbatim}
The numarray version of \function{take} can also take a sequence as its 
axis value:
\begin{verbatim}
>>> print take(a2, [0,1], [0,1])
1
>>> print take(a2, [0,1], [1,0])
5
\end{verbatim}
In this case, it functions like indexing: a2[0,1] and a2[1,0] respectively.
\end{funcdesc}


\begin{funcdesc}{put}{array, indices, values, axis=0, clipmode=CLIP}
  \label{func:put}
   \function{put} is the opposite of \function{take}. The values of \var{array}
   at the locations specified in \var{indices} are set to the corresponding
   \var{values}.  The \var{array} must be a contiguous array. The \var{indices}
   can be any integer sequence object with values suitable for indexing into
   the flat form of \var{array}.  The \var{values} must be any sequence of
   values that can be converted to the type of \var{a}.
\begin{verbatim}
>>> x = arange(6)
>>> put(x, [2,4], [20,40])
>>> print x
[ 0  1 20  3 40  5]
\end{verbatim}
   Note that the target \var{array} is not required to be one-dimensional.
   Since \var{array} is contiguous and stored in row-major order, the
   \var{indices} can be treated as indexing \var{array}'s elements in storage
   order.  The routine \function{put} is thus equivalent to the following
   (although the loop is in C for speed):
\begin{verbatim}
ind = array(indices, copy=0)
v = array(values, copy=0).astype(a.type())
for i in range(len(ind)): a.flat[i] = v[i]
\end{verbatim}
\end{funcdesc}


\begin{funcdesc}{putmask}{array, mask, values}
   \function{putmask} sets those elements of \var{array} for which 
   \var{mask} is true to the corresponding value in \var{values}. 
   The array \var{array} must be contiguous. The argument \var{mask} 
   must be an integer sequence of the same size (but not necessarily the 
   same shape) as \var{array}. The argument \var{values} will be 
   repeated as necessary; in particular it can be a
   scalar. The array values must be convertible to the type of \var{array}.
\begin{verbatim}
>>> x=arange(5)
>>> putmask(x, [1,0,1,0,1], [10,20,30,40,50])
>>> print x
[10  1 30  3 50]
>>> putmask(x, [1,0,1,0,1], [-1,-2])
>>> print x
[-1  1 -1  3 -1]
\end{verbatim}
   Note how in the last example, the third argument was treated as if it were
   \code{[-1, -2, -1, -2, -1]}.
\end{funcdesc}


\begin{funcdesc}{transpose}{array, axes=None}
   \function{transpose} takes an array \var{array} and returns a new 
   array which corresponds to \var{a} with the order of axes specified 
   by the second argument \var{axes} which is a sequence of dimension 
   indices.  The default is to reverse the order of all axes, i.e. 
   \code{axes=arange(a.rank)[::-1]}.
\begin{verbatim}
>>> a2=arange(6,shape=(2,3)); print a2
[[0 1 2]
 [3 4 5]]
>>> print transpose(a2)  # same as transpose(a2, axes=(1,0))
[[0 3]
 [1 4]
 [2 5]]
>>> a3=arange(24,shape=(2,3,4)); print a3
[[[ 0  1  2  3]
  [ 4  5  6  7]
  [ 8  9 10 11]]

 [[12 13 14 15]
  [16 17 18 19]
  [20 21 22 23]]]
>>> print transpose(a3)  # same as transpose(a3, axes=(2,1,0))
[[[ 0 12]
  [ 4 16]
  [ 8 20]]

 [[ 1 13]
  [ 5 17]
  [ 9 21]]

 [[ 2 14]
  [ 6 18]
  [10 22]]

 [[ 3 15]
  [ 7 19]
  [11 23]]]
>>> print transpose(a3, axes=(1,0,2))
[[[ 0  1  2  3]
  [12 13 14 15]]

 [[ 4  5  6  7]
  [16 17 18 19]]

 [[ 8  9 10 11]
  [20 21 22 23]]]
\end{verbatim}
\end{funcdesc}


\begin{funcdesc}{repeat}{array, repeats, axis=0}
   \function{repeat} takes an array \var{array} and returns an array 
   with each element in the input array repeated as often as indicated by the
   corresponding elements in the second array. It operates along the specified
   axis. So, to stretch an array evenly, one needs the repeats array to contain
   as many instances of the integer scaling factor as the size of the specified
   axis:
\begin{verbatim}
>>> print a
[[0 1 2]
 [3 4 5]]
>>> print repeat(a, 2*ones(a.shape[0]))   # i.e. repeat(a, (2,2)), broadcast 
                   # rules apply, so this is also equivalent to repeat(a, 2)
[[0 1 2]
 [0 1 2]
 [3 4 5]
 [3 4 5]]
>>> print repeat(a, 2*ones(a.shape[1]), 1)  # i.e. repeat(a, (2,2,2), 1), or
                                            # repeat(a, 2, 1)
[[0 0 1 1 2 2]
 [3 3 4 4 5 5]]
>>> print repeat(a, (1,2))
[[0 1 2]
 [3 4 5]
 [3 4 5]]
\end{verbatim}
\end{funcdesc}


\begin{funcdesc}{where}{condition, x, y, out=None}
  \label{func:where}
   The \function{where} function creates an array whose values are those of
   \var{x} at those indices where \var{condition} is true, and those of \var{y}
   otherwise.  The shape of the result is the shape of \var{condition}. The
   type of the result is determined by the types of \var{x} and \var{y}. Either
   \var{x} or \var{y} (or both) can be a scalar, which is then used for all
   appropriate elements of condition.  \var{out} can be used to specify an
   output array.
\begin{verbatim}
>>> where(arange(10) >= 5, 1, 2)
array([2, 2, 2, 2, 2, 1, 1, 1, 1, 1])
\end{verbatim}

   Starting with numarray-0.6, \function{where} supports a one parameter form
   that is equivalent to the \var{nonzero} function but reads better:

\begin{verbatim}
>>> where(arange(10) % 2)
(array([1, 3, 5, 7, 9]),)   # indices where expression is true 
\end{verbatim}
   Note that in this case, the output is a tuple.

   Like \function{nonzero}, the one parameter form of \function{where} can be
   used to do array indexing:

\begin{verbatim}
>>> a = arange(10,20)
>>> a[ where( a % 2 ) ]
array([11, 13, 15, 17, 19])
\end{verbatim}

   Note that for array indices which are boolean arrays, using \function{where}
   is not necessary but is still OK:

\begin{verbatim}
>>> a[(a % 2) != 0]
array([11, 13, 15, 17, 19])
>>> a[where((a%2) != 0)]
array([11, 13, 15, 17, 19])
\end{verbatim}
\end{funcdesc}

\begin{funcdesc}{choose}{selector, population, outarr=None, clipmode=RAISE}
   The function \function{choose} provides a more general mechanism for
   selecting members of a \var{population} based on a \var{selector} array.
   Unlike \function{where}, \function{choose} can select values from more than
   two arrays.  \var{selector} is an array of integers between \constant{0} and
   \constant{n}. The resulting array will have the same shape as
   \var{selector}, with element selected from \code{population=(b0, ..., bn)}
   as indicated by the value of the corresponding element in \var{selector}.
   Assume \var{selector} is an array that you want to "clip" so that no values
   are greater than \constant{100.0}.
\begin{verbatim}
>>> choose(greater(a, 100.0), (a, 100.0))
\end{verbatim}
   Everywhere that \code{greater(a, 100.0)} is false (i.e.\ \constant{0}) this
   will ``choose'' the corresponding value in \var{a}. Everywhere else 
   it will ``choose'' \constant{100.0}.  This works as well with arrays. 
   Try to figure out what the following does:
\begin{verbatim}
>>> ret = choose(greater(a,b), (c,d))
\end{verbatim}
\end{funcdesc}

\begin{funcdesc}{ravel}{array}
   Returns the argument \var{array} as a 1-D array. It is 
   equivalent to \code{reshape(a, (-1,))}. There is a \method{ravel} 
   method which reshapes the array in-place. Unlike \code{a.ravel()}, 
   however, the \function{ravel} function works with non-contiguous arrays.
\begin{verbatim}
>>> a=arange(25)
>>> a.setshape(5,5)
>>> a.transpose()
>>> a.iscontiguous()
0
>>> a
array([[ 0,  5, 10, 15, 20],
  [ 1,  6, 11, 16, 21],
  [ 2,  7, 12, 17, 22],
  [ 3,  8, 13, 18, 23],
  [ 4,  9, 14, 19, 24]])
>>> a.ravel()
Traceback (most recent call last):
...
TypeError: Can't reshape non-contiguous arrays
>>> ravel(a)
array([ 0,  5, 10, 15, 20,  1,  6, 11, 16, 21,  2,  7, 12, 17, 22,  3,
        8, 13, 18, 23,  4,  9, 14, 19, 24])
\end{verbatim}
\end{funcdesc}


\begin{funcdesc}{nonzero}{a}
   \function{nonzero} returns a tuple of arrays containing the indices of the
   elements in \var{a} that are nonzero.

\begin{verbatim}
>>> a = array([-1, 0, 1, 2])
>>> nonzero(a)
(array([0, 2, 3]),)
>>> print a2
[[-1  0  1  2]
 [ 9  0  4  0]]
>>> print nonzero(a2)
(array([0, 0, 0, 1, 1]), array([0, 2, 3, 0, 2]))
\end{verbatim}
\end{funcdesc}

\begin{funcdesc}{compress}{condition, a, axis=0}
  \label{func:compress}
   Returns those elements of a corresponding to those elements of condition
   that are nonzero. \var{condition} must be the same size as the given axis of
   \var{a}.  Alternately, \var{condition} may match \var{a} in shape; in this
   case the result is a 1D array and \var{axis} should not be specified.
\begin{verbatim}
>>> print x
[1 0 6 2 3 4]
>>> print greater(x, 2)
[0 0 1 0 1 1]
>>> print compress(greater(x, 2), x)
[6 3 4]
>>> print a2
[[-1  0  1  2]
 [ 9  0  4  0]]
>>> print compress(a2>1, a2)
[2 9 4]
>>> a = array([[1,2],[3,4]])
>>> print compress([1,0], a, axis = 1)
[[1]
[3]]
>>> print compress([[1,0],[0,1]], a)
[1, 4]
\end{verbatim}
\end{funcdesc}


\begin{funcdesc}{diagonal}{a, offset=0, axis1=0, axis2=1}
   Returns the entries along the diagonal of \var{a} specified by \var{offset}.
   The \var{offset} is relative to the \var{axis2} axis.  This is designed for
   2-D arrays. For arrays of higher ranks, it will return the diagonal of each
   2-D sub-array.  The 2-D array does not have to be square.

   Warning:  in Numeric (and numarray 0.7 or before), there is a bug in 
   the \function{diagonal} function which will give erronous result for 
   arrays of 3-D or higher.
\begin{verbatim}
>>> print x
[[ 0  1  2  3  4]
 [ 5  6  7  8  9]
 [10 11 12 13 14]
 [15 16 17 18 19]
 [20 21 22 23 24]]
>>> print diagonal(x)
[ 0  6 12 18 24]
>>> print diagonal(x, 1)
[ 1  7 13 19]
>>> print diagonal(x, -1)
[ 5 11 17 23]
\end{verbatim}
\end{funcdesc}


\begin{funcdesc}{trace}{a, offset=0, axis1=0, axis2=1}
   Returns the sum of the elements in a along the diagonal specified by offset.

   Warning:  in Numeric (and numarray 0.7 or before), there is a bug in 
   the \function{trace} function which will give erronous result for 
   arrays of 3-D or higher.
\begin{verbatim}
>>> print x
[[ 0  1  2  3  4]
 [ 5  6  7  8  9]
 [10 11 12 13 14]
 [15 16 17 18 19]
 [20 21 22 23 24]]
>>> print trace(x)                      # 0 + 6 + 12 + 18 + 24
60
>>> print trace(x, -1)                  # 5 + 11 + 17 + 23
56
>>> print trace(x, 1)                   # 1 + 7 + 13 + 19
40
\end{verbatim}
\end{funcdesc}


\begin{funcdesc}{searchsorted}{bin, values}
   Called with a rank-1 array sorted in ascending order,
   \function{searchsorted} will return the indices of the positions in 
   \var{bin} where the corresponding \var{values} would fit.
\begin{verbatim}
>>> print bin_boundaries
[ 0.   0.1  0.2  0.3  0.4  0.5  0.6  0.7  0.8  0.9  1. ]
>>> print data
[ 0.31  0.79  0.82  5.  -2.  -0.1 ]
>>> print searchsorted(bin_boundaries, data)
[4 8 9 11 0 0]
\end{verbatim}
   This can be used for example to write a simple histogramming function:
\begin{verbatim}
>>> def histogram(a, bins):
...         # Note that the argument names below are reverse of the 
...         # searchsorted argument names
...         n = searchsorted(sort(a), bins)
...         n = concatenate([n, [len(a)]])
...         return n[1:]-n[:-1]
...
>>> print histogram([0,0,0,0,0,0,0,.33,.33,.33], arange(0,1.0,.1))
[7 0 0 3 0 0 0 0 0 0]
>>> print histogram(sin(arange(0,10,.2)), arange(-1.2, 1.2, .1))
[0 0 4 2 2 2 0 2 1 2 1 3 1 3 1 3 2 3 2 3 4 9 0 0]
\end{verbatim}
\end{funcdesc}


\begin{funcdesc}{sort}{array, axis=-1}
   This function returns an array containing a copy of the data in 
   \var{array}, with the same shape as \var{array}, but with the 
   order of the elements along the specified \var{axis} sorted. The shape 
   of the returned array is the same as \var{array}'s.  Thus, 
   \code{sort(a, 3)} will be an array of the same shape as \var{array}, 
   where the elements of \var{array} have been sorted along the fourth
   axis.
\begin{verbatim}
>>> print data
[[5 0 1 9 8]
 [2 5 8 3 2]
 [8 0 3 7 0]
 [9 6 9 5 0]
 [9 0 9 7 7]]
>>> print sort(data)                    # Axis -1 by default
[[0 1 5 8 9]
 [2 2 3 5 8]
 [0 0 3 7 8]
 [0 5 6 9 9]
 [0 7 7 9 9]]
>>> print sort(data, 0)
[[2 0 1 3 0]
 [5 0 3 5 0]
 [8 0 8 7 2]
 [9 5 9 7 7]
 [9 6 9 9 8]]
\end{verbatim}
\end{funcdesc}


\begin{funcdesc}{argsort}{array, axis=-1}
   \function{argsort} will return the indices of the elements of the array
   needed to produce \code{sort(array)}. In other words, for a 1-D array,
   \code{take(a.flat, argsort(a))} is the same as \code{sort(a)}... but slower.
\begin{verbatim}
>>> print data
[5 0 1 9 8]
>>> print sort(data)
[0 1 5 8 9]
>>> print argsort(data)
[1 2 0 4 3]
>>> print take(data, argsort(data))
[0 1 5 8 9]
\end{verbatim}
\end{funcdesc}


\begin{funcdesc}{argmax}{array, axis=-1}
\end{funcdesc}
\begin{funcdesc}{argmin}{array, axis=-1}
   The \function{argmax} function returns an array (or scalar for a 1D array)
   with the index(es) of the maximum value(s) of its input \var{array} along
   the given \var{axis}. The returned array will have one less dimension than
   \var{array}. \function{argmin} is just like \function{argmax}, except that
   it returns the indices of the minima along the given axis.
\begin{verbatim}
>>> print data
[[9 6 1 3 0]
 [0 0 8 9 1]
 [7 4 5 4 0]
 [5 2 7 7 1]
 [9 9 7 9 7]]
>>> print argmax(data)
[0 3 0 3 1]
>>> print argmax(data, 0)
[4 4 1 4 4]
>>> print argmin(data)
[4 1 4 4 4]
>>> print argmin(data, 0)
[1 1 0 0 2]
\end{verbatim}
\end{funcdesc}

\begin{funcdesc}{fromstring}{datastring, type, shape=None}
   Will return the array formed by the binary data given in 
   \var{datastring}, of the specified \var{type}. This is mainly 
   used for reading binary data to and from files, it can also be used to 
   exchange binary data with other modules that use python strings as 
   storage (e.g.\ PIL). Note that this representation is dependent on the 
   byte order. To find out the byte ordering used, use the 
   \method{isbyteswapped} method described on page 
   \pageref{arraymethod:byteswap}.  If \var{shape} is not specified, the 
   created array will be one dimensional.
\end{funcdesc}

\begin{funcdesc}{fromfile}{file, type, shape=None}
  If \var{file} is a string then it is interpreted as the name of a 
  file which is opened and read.  Otherwise, \var{file} must be a 
  Python file object which is read as a source of binary array data.  
  \function{fromfile} reads directly into the newly created array buffer 
  with no intermediate string, but otherwise is similar to fromstring, 
  treating the contents of the specified file as a binary data string.
\end{funcdesc}

\begin{funcdesc}{dot}{a, b}
   The \function{dot} function returns the dot product of \var{a} and
   \var{b}. This is equivalent to matrix multiply for rank-2 arrays (without
   the transposition).  This function is identical to the
   \function{matrixmultiply} function.
\begin{verbatim}
>>> print a
[1 2]
>>> print b
[10 11]
# kind of like vector inner product with implicit transposition 
>>> print dot(a,b), dot(b,a) 
32 32
>>> print a
[[1 2]
 [5 7]]
>>> print b
[[  0   1]
 [ 10 100]]
>>> print dot(a,b)
[[ 20 201]
 [ 70 705]]
>>> print dot(b,a)
[[  5   7]
 [510 720]]
\end{verbatim}
\end{funcdesc}

\begin{funcdesc}{matrixmultiply}{a, b}
   This function multiplies matrices or matrices and vectors as matrices rather
   than elementwise. This function is identical to \function{dot}.  Compare:
\begin{verbatim}
>>> print a
[[0 1 2]
 [3 4 5]]
>>> print b
[1 2 3]
>>> print a*b
[[ 0  2  6]
 [ 3  8 15]]
>>> print matrixmultiply(a,b)
[ 8 26]
\end{verbatim}
\end{funcdesc}


\begin{funcdesc}{clip}{m, m_min, m_max}
   The clip function creates an array with the same shape and type as 
   \var{m}, but where every entry in \var{m} that is less than 
   \var{m_min} is replaced by \var{m_min}, and every entry greater 
   than \var{m_max} is replaced by \var{m_max}.  Entries within 
   the range \var{[m_min, m_max]} are left unchanged.
\begin{verbatim}
>>> a = arange(9, type=Float32)
>>> print clip(a, 1.5, 7.5)
[1.5 1.5 2. 3. 4. 5. 6. 7. 7.5]
\end{verbatim}
\end{funcdesc}


\begin{funcdesc}{indices}{shape, type=None}
   The indices function returns an array corresponding to the \var{shape} 
   given. The array returned is an array of a new shape which is based on 
   the specified \var{shape}, but has an added dimension of length 
   the number of dimensions in the specified shape.  For example, if 
   \code{shape=(3,4)}, then the shape of the array returned will be
   \code{(2,3,4)} since the length of \code{(3,4)} is \var{2} and if 
   \code{shape=(5,6,7)}, the returned array's shape will be \code{(3,5,6,7)}. 
   The contents of the returned arrays are such that the \var{i}th subarray 
   (along index 0, the first dimension) contains the indices for that axis 
   of the elements in the array.  An example makes things clearer:
\begin{verbatim}
>>> i = indices((4,3))
>>> i.getshape()
(2, 4, 3)
>>> print i[0]
[[0 0 0]
 [1 1 1]
 [2 2 2]
 [3 3 3]]
>>> print i[1]
[[0 1 2]
 [0 1 2]
 [0 1 2]
 [0 1 2]]
\end{verbatim}
   So, \code{i[0]} has an array of the specified shape, and each element in
   that array specifies the index of that position in the subarray for axis 0.
   Similarly, each element in the subarray in \code{i[1]} contains the index of
   that position in the subarray for axis 1.
\end{funcdesc}


\begin{funcdesc}{swapaxes}{array, axis1, axis2}
   Returns a new array which \var{shares} the data of \var{array}, but 
   has the two axes specified by \var{axis1} and \var{axis2} 
   swapped. If \var{array} is of rank 0 or 1, swapaxes simply returns a 
   new reference to \var{array}.
\begin{verbatim}
>>> x = arange(10)
>>> x.setshape((5,2,1))
>>> print x
[[[0]
  [1]]

 [[2]
  [3]]

 [[4]
  [5]]

 [[6]
  [7]]

 [[8]
  [9]]]
>>> y = swapaxes(x, 0, 2)
>>> y.getshape()
(1, 2, 5)
>>> print y
[[[0 2 4 6 8]
 [1 3 5 7 9]]]
\end{verbatim}
\end{funcdesc}


\begin{funcdesc}{concatenate}{arrs, axis=0}
   Returns a new array containing copies of the data contained in all arrays
   of \var{arrs= (a0, a1, ... an)}.  The arrays \var{ai} will be 
   concatenated along the specified \var{axis} (default=0). All 
   arrays \var{ai} must have the same shape along every axis except for 
   the one specified in \var{axis}. To concatenate arrays along a
   newly created axis, you can use \code{array((a0, ..., an))}, as long as all
   arrays have the same shape.
\begin{verbatim}
>>> print x
[[ 0  1  2  3]
 [ 5  6  7  8]
 [10 11 12 13]]
>>> print concatenate((x,x))
[[ 0  1  2  3]
 [ 5  6  7  8]
 [10 11 12 13]
 [ 0  1  2  3]
 [ 5  6  7  8]
 [10 11 12 13]]
>>> print concatenate((x,x), 1)
[[ 0  1  2  3  0  1  2  3]
 [ 5  6  7  8  5  6  7  8]
 [10 11 12 13 10 11 12 13]]
>>> print array((x,x))   # Note: one extra dimension
[[[ 0  1  2  3]
  [ 5  6  7  8]
  [10 11 12 13]]
 [[ 0  1  2  3]
  [ 5  6  7  8]
  [10 11 12 13]]]
>>> print a
[[1 2]]
>>> print b
[[3 4 5]]
>>> print concatenate((a,b),1)
[[1 2 3 4 5]]
>>> print concatenate((a,b),0)
ValueError: _concat array shapes must match except 1st dimension
\end{verbatim}
\end{funcdesc}


\begin{funcdesc}{innerproduct}{a, b}
   \function{innerproduct} produces the inner product of arrays \var{a} and
   \var{b}. It is equivalent to \code{matrixmultiply(a, transpose(b))}.
\end{funcdesc}


\begin{funcdesc}{outerproduct}{a,b}
   \function{outerproduct} produces the outer product of vectors \var{a} and
   \var{b}, that is \code{result[i, j] = a[i] * b[j]}.
\end{funcdesc}


\begin{funcdesc}{array_repr}{a, max_line_width=None, precision=None, supress_small=None}
   See section \ref{TBD} on Textual Representations of arrays.
\end{funcdesc}


\begin{funcdesc}{array_str}{a, max_line_width=None, precision=None, supress_small=None}
   See section \ref{TBD} Textual Representations of arrays.
\begin{verbatim}
>>> print a
[  1.00000000e+00   1.10000000e+00   1.11600000e+00   1.11380000e+00
   1.20000000e-02   1.34560000e-04]
>>> print array_str(a,precision=4,suppress_small=1)
[ 1.      1.1     1.116   1.1138  0.012   0.0001]
>>> print array_str(a,precision=3,suppress_small=1)
[ 1.     1.1    1.116  1.114  0.012  0.   ]
>>> print array_str(a,precision=3)
[  1.000e+00   1.100e+00   1.116e+00   1.114e+00   1.200e-02
   1.346e-04]
\end{verbatim}
\end{funcdesc}


\begin{funcdesc}{resize}{array, shape}
  \label{func:resize}
   The \function{resize} function takes an array and a shape, and returns a new
   array with the specified \var{shape}, and filled with the data in 
   the input \var{array}.  Unlike the \function{reshape} function, the 
   new shape does not have to yield the same size as the original array. 
   If the new size of is less than that of the input \var{array}, the 
   returned array contains the appropriate data from the "beginning" of the 
   old array. If the new size is greater than that of the input array, the 
   data in the input \var{array} is repeated as many times as needed
   to fill the new array.
\begin{verbatim}
>>> x = arange(10)
>>> y = resize(x, (4,2))                # note that 4*2 < 10
>>> print x
[0 1 2 3 4 5 6 7 8 9]
>>> print y
[[0 1]
 [2 3]
 [4 5]
 [6 7]]
>>> print resize(array((0,1)), (5,5))   # note that 5*5 > 2
[[0 1 0 1 0]
 [1 0 1 0 1]
 [0 1 0 1 0]
 [1 0 1 0 1]
 [0 1 0 1 0]]
\end{verbatim}
\end{funcdesc}


\begin{funcdesc}{identity}{n, type=None}
   The identity function returns an \var{n} by \var{n} array 
   where the diagonal elements are 1, and the off-diagonal elements are 0.
\begin{verbatim}
>>> print identity(5)
[[1 0 0 0 0]
 [0 1 0 0 0]
 [0 0 1 0 0]
 [0 0 0 1 0]
 [0 0 0 0 1]]
\end{verbatim}
\end{funcdesc}


\begin{funcdesc}{sum}{a, axis=0}
   The sum function is a synonym for the \method{reduce} method of the
   \function{add} ufunc. It returns the sum of all of the elements in the
   sequence given along the specified axis (first axis by default).
\begin{verbatim}
>>> print x
[[ 0  1  2  3]
 [ 4  5  6  7]
 [ 8  9 10 11]
 [12 13 14 15]
 [16 17 18 19]]
>>> print sum(x)
[40 45 50 55]                           # 0+4+8+12+16, 1+5+9+13+17,
2+6+10+14+18, ...
>>> print sum(x, 1)
[ 6 22 38 54 70]                        # 0+1+2+3, 4+5+6+7, 8+9+10+11, ...
\end{verbatim}
\end{funcdesc}


\begin{funcdesc}{cumsum}{a, axis=0}
   The cumsum function is a synonym for the \method{accumulate} method of the
   \function{add} ufunc.
\end{funcdesc}


\begin{funcdesc}{product}{a, axis=0}
   The product function is a synonym for the \method{reduce} method of the
   \function{multiply} ufunc.
\end{funcdesc}


\begin{funcdesc}{cumproduct}{a, axis=0}
   The cumproduct function is a synonym for the \method{accumulate} method of
   the \function{multiply} ufunc.
\end{funcdesc}


\begin{funcdesc}{alltrue}{a, axis=0}
   The alltrue function is a synonym for the \method{reduce} method of the
   \function{logical_and} ufunc.
\end{funcdesc}


\begin{funcdesc}{sometrue}{a, axis=0}
   The sometrue function is a synonym for the \method{reduce} method of the
   \function{logical_or} ufunc.
\end{funcdesc}


\begin{funcdesc}{all}{a}
   \function{all} is a synonym for the \method{reduce} method of the
   \function{logical_and} ufunc, preceded by a \function{ravel} which converts
   arrays with \(rank>1\) to \(rank=1\).  Thus, \function{all} tests that all
   the elements of a multidimensional array are nonzero.
\end{funcdesc}


\begin{funcdesc}{any}{a}
   The \function{any} function is a synonym for the \method{reduce} method of
   the \function{logical_and} ufunc, preceded by a \function{ravel} which
   converts arrays with \(rank>1\) to \(rank=1\).  Thus, \function{any} tests
   that at least one of the elements of a multidimensional array is nonzero.
\end{funcdesc}


\begin{funcdesc}{allclose}{a, b, rtol=1.e-5, atol=1.e-8}
   This function tests whether or not arrays \var{x} and \var{y} 
   of an integer or real type are equal subject to the given relative and 
   absolute tolerances: \code{rtol, atol}. The formula used is:
   \begin{equation}
      \left| x - y \right| < atol + rtol * \left| y \right|
   \end{equation}
   This means essentially that both elements are small compared to \var{atol}
   or their difference divided by \var{y}'s value is small compared to
   \var{rtol}.
\end{funcdesc}



\begin{seealso}
   \seemodule{numarray.convolve}{The \function{convolve} function is implemented in the
      optional \module{numarray.convolve} package.}%
   \seemodule{numarray.convolve}{The \function{correlation} function is implemented in
      the optional \module{numarray.convolve} package.}%
\end{seealso} 




%% Local Variables:
%% mode: LaTeX
%% mode: auto-fill
%% fill-column: 79
%% indent-tabs-mode: nil
%% ispell-dictionary: "american"
%% reftex-fref-is-default: nil
%% TeX-auto-save: t
%% TeX-command-default: "pdfeLaTeX"
%% TeX-master: "numarray"
%% TeX-parse-self: t
%% End:


\chapter{Array Methods}
\label{cha:array-methods}

As we discussed at the beginning of the last chapter, there are very few array
methods for good reasons, and these all depend on the implementation
details. They're worth knowing, though.

\begin{methoddesc}[numarray]{argmax}{axis=-1}
  \label{arraymethod:argmax}
  The \method{argmax} method returns the index of the largest element in a 1D
  array.  In the case of a multi-dimensional array, it returns and array of
  indices.
\begin{verbatim}
>>> array([1,2,4,3]).argmax()
2
>>> arange(100, shape=(10,10)).argmax()
array([9, 9, 9, 9, 9, 9, 9, 9, 9, 9])
\end{verbatim}
\end{methoddesc}


\begin{methoddesc}[numarray]{argmin}{axis=-1}
  \label{arraymethod:argmin}
  The \method{argmin} method returns the index of the smallest element in a 1D
  array.  In the case of a multi-dimensional array, it returns and array of
  indices.
\end{methoddesc}


\begin{methoddesc}[numarray]{argsort}{axis=-1}
  \label{arraymethod:argsort}
  The \method{argsort} method returns the array of indices which if taken from
  the array using \function{take} would return a sorted copy of the array.  For
  multi-dimensional arrays, \method{argsort} computes the indices for each 1D
  subarray independently and aggregates them all into a single array result;
  The \method{argsort} of a multi-dimensional array does not produce a sorted
  copy of the array when applied directly to it using \function{take}; instead,
  each 1D subarray must be passed to \function{take} independently.
\begin{verbatim}
  >>> array([1,2,4,3]).argsort()
  array([0, 1, 3, 2])
  >>> take([1,2,4,3], argsort([1,2,4,3]))
  array([1, 2, 3, 4])
\end{verbatim}
\end{methoddesc}


\begin{methoddesc}[numarray]{astype}{type}
  \label{arraymethod:astype}
  The \method{astype} method returns a copy of the array converted to the
  specified type.  As with any copy, the new array is aligned, contiguous, and
  in native machine byte order.  If the specified type is the same as current
  type, a copy is \emph{still} made.
\begin{verbatim}
  >>> arange(5).astype('Float64')
  array([ 0.,  1.,  2.,  3.,  4.])
\end{verbatim}
\end{methoddesc}


\begin{methoddesc}[numarray]{byteswap}{}
   \label{arraymethod:byteswap}
   The \method{byteswap} method performs a byte swapping operation on all the
   elements in the array, working inplace (i.e.\ it returns None).
   \method{byteswap} does not affect the array's byte order state variable.
   See \method{togglebyteorder} for changing the array's byte order state
   in addition to or rather than physically swapping bytes.
\begin{verbatim}
>>> print a
[1 2 3]
>>> a.byteswap()
>>> print a
[16777216 33554432 50331648]
\end{verbatim}
\end{methoddesc}


\begin{methoddesc}[numarray]{byteswapped}{}
  \label{arraymethod:byteswapped} 
  The \method{byteswapped} method returns a byteswapped copy of the array.
  \method{byteswapped} does not affect the array's own byte order state
  variable.  The result of \method{byteswapped} is logically in native byte
  order.
\begin{verbatim}
>>> array([1,2,3]).byteswapped()
array([16777216, 33554432, 50331648])
\end{verbatim}
\end{methoddesc}


\begin{methoddesc}[numarray]{conjugate}{}
  \label{arraymethod:conjugate}
   The \method{conjugate} method returns the complex conjugate of an array.
\begin{verbatim}
>>> (arange(3) + 1j).conjugate()
array([ 0.-1.j,  1.-1.j,  2.-1.j])
\end{verbatim}
\end{methoddesc}


\begin{methoddesc}[numarray]{copy}{}
  \label{arraymethod:copy}
   The \method{copy} method returns a copy of an array. When making an
   assignment or taking a slice, a new array object is created and has its own
   attributes, except that the data attribute just points to the data of the
   first array (a "view").  The \method{copy} method is used when it is
   important to obtain an independent copy.  \method{copy} returns arrays which
   are contiguous, aligned, and not byteswapped, i.e. well behaved.
\begin{verbatim}
>>> c = a[3:8:2].copy()
>>> print c.iscontiguous()
1
\end{verbatim}
\end{methoddesc}


\begin{methoddesc}[numarray]{diagonal}{}
  \label{arraymethod:diagonal}
   The \method{diagonal} method returns the diagonal elements of the array,
   those elements where the row and column indices are equal.
\begin{verbatim}
>>> arange(25,shape=(5,5)).diagonal()
array([ 0,  6, 12, 18, 24])
\end{verbatim}
\end{methoddesc}


\begin{methoddesc}[numarray]{info}{}
   \label{arraymethod:info} Calling an array's \method{info}
   method prints out information about the array which is useful for debugging.
\begin{verbatim}
>>> arange(10).info()
class: <class 'numarray.numarraycore.NumArray'>
shape: (10,)
strides: (4,)
byteoffset: 0
bytestride: 4
itemsize: 4
aligned: 1
contiguous: 1
data: <memory at 0x08931d18 with size:0x00000028 held by object 0x3ff91bd8 aliasing object 0x00000000>
byteorder: little
byteswap: 0
type: Int32
\end{verbatim}
\end{methoddesc}


\begin{methoddesc}[numarray]{isaligned}{}
  \label{arraymethod:isaligned} \method{isaligned} returns 1 IFF the buffer
  address for an array modulo the array itemsize is 0.  When the array
  itemsize exceeds 8 (sizeof(double)) aligment is done modulo 8.
\end{methoddesc}


\begin{methoddesc}[numarray]{isbyteswapped}{}
  \label{arraymethod:isbyteswapped} \method{isbyteswapped} returns 1 IFF the 
  array's binary data is not in native machine byte order, possibly because it
  originated on a machine with a different native order.
\end{methoddesc}


\begin{methoddesc}[numarray]{iscontiguous}{}
  \label{arraymethod:iscontiguous} \method{iscontiguous} returns 1 IFF
   an array is C-contiguous and 0 otherwise.  An array is C-contiguous if its
   smallest stride corresponds to the innermost dimension and its other strides
   strictly increase in size from the innermost dimension to the outermost,
   with each stride being the product of the previous inner stride and shape.
   A non-contiguous array can be converted to a contiguous array by the
   \method{copy} method.
\begin{verbatim}
>>> a=arange(25, shape=(5,5))
>>> a
array([[ 0,  1,  2,  3,  4],
       [ 5,  6,  7,  8,  9],
       [10, 11, 12, 13, 14],
       [15, 16, 17, 18, 19],
       [20, 21, 22, 23, 24]])
>>> a.iscontiguous()
1
\end{verbatim}
\end{methoddesc}


\begin{methoddesc}[numarray]{is_c_array}{}
   \label{arraymethod:is-c-array} 
   \method{is_c_array} returns 1 IFF an array is C-contiguous, aligned, and
   not byteswapped, and returns 0 otherwise.
\begin{verbatim}
>>> a=arange(25, shape=(5,5))
>>> a.is_c_array()
1
>>> a.is_f_array()
0
\end{verbatim}
\end{methoddesc}


\begin{methoddesc}[numarray]{is_fortran_contiguous}{}
   \label{arraymethod:is-fortran-contiguous} 
   \method{is_fortran_contiguous} returns 1 IFF an array is Fortran-contiguous
   and 0 otherwise.  An array is Fortran-contiguous if its smallest stride
   corresponds to its outermost dimension and each succesive stride is the
   product of the previous stride and shape element.
\begin{verbatim}
>>> a=arange(25, shape=(5,5))
>>> a.transpose()
>>> a
array([[ 0,  5, 10, 15, 20],
       [ 1,  6, 11, 16, 21],
       [ 2,  7, 12, 17, 22],
       [ 3,  8, 13, 18, 23],
       [ 4,  9, 14, 19, 24]])
>>> a.iscontiguous()
0
>>> a.is_fortran_contiguous()
1
\end{verbatim}
\end{methoddesc}


\begin{methoddesc}[numarray]{is_f_array}{}
   \label{arraymethod:is-f-array} \method{is_f_array} returns 1 IFF
   an array is Fortran-contiguous, aligned, and not byteswapped, and returns 0
   otherwise.
\begin{verbatim}
>>> a=arange(25, shape=(5,5))
>>> a.transpose()
>>> a.is_f_array()
1
>>> a.is_c_array()
0
\end{verbatim}
\end{methoddesc}


\begin{methoddesc}[numarray]{itemsize}{}
  \label{arraymethod:itemsize} The \method{itemsize} method 
  returns the number of bytes used by any one of its elements.
\begin{verbatim}
>>> a = arange(10)
>>> a.itemsize()
4
>>> a = array([1.0])
>>> a.itemsize()
8
>>> a = array([1], type=Complex64)
>>> a.itemsize()
16
\end{verbatim}
\end{methoddesc}


\begin{methoddesc}[numarray]{max}{}
  \label{arraymethod:max}
  The \method{max} method returns the largest element in an array.
\begin{verbatim}
>>> arange(100, shape=(10,10)).max()
99
\end{verbatim}
\end{methoddesc}
\begin{methoddesc}[numarray]{mean}{}
  \label{arraymethod:mean}
  The \method{mean} method returns the average of all elements in an array.
\begin{verbatim}
>>> arange(10).mean() 4.5
\end{verbatim}
\end{methoddesc}
\begin{methoddesc}[numarray]{min}{}
  \label{arraymethod:min}
  The \method{min} method returns the smallest element in an array.
\begin{verbatim}
>>> arange(10).min()
0
\end{verbatim}
\end{methoddesc}


\begin{methoddesc}[numarray]{nelements}{}
  \label{arraymethod:nelements}
  \method{nelements} returns the total number of elements in this array.
  Synonymous with \method{size}.
\begin{verbatim}
>>> arange(100).nelements()
100
\end{verbatim}
\end{methoddesc}


\begin{methoddesc}[numarray]{new}{type=None}
  \label{arraymethod:new}
   \method{new} returns a new array of the specified type with the same shape
   as this array.  The new array is uninitialized.
\end{methoddesc}


\begin{methoddesc}[numarray]{nonzero}{axis=-1}
  \label{arraymethod:nonzero}
   \method{nonzero} returns a tuple of arrays containing the indices of the
   elements that are nonzero.
\begin{verbatim}
>>> arange(5).nonzero()
(array([1, 2, 3, 4]),)
>>> b = arange(9, shape=(3,3)) % 2; b
array([[0, 1, 0],
       [1, 0, 1],
       [0, 1, 0]])
>>>b.nonzero()
(array([0, 1, 1, 2]), array([1, 0, 2, 1]))
\end{verbatim}
\end{methoddesc}


\begin{methoddesc}[numarray]{repeat}{r, axis=0}
  \label{arraymethod:repeat}
   The \method{repeat} method returns a new array with each element self[i]
   (along the specified axis) repeated r[i] times.
\begin{verbatim}
>>> a=arange(25, shape=(5,5))
>>> a
array([[ 0,  1,  2,  3,  4],
       [ 5,  6,  7,  8,  9],
       [10, 11, 12, 13, 14],
       [15, 16, 17, 18, 19],
       [20, 21, 22, 23, 24]])
>>> a.repeat(arange(5)%2*2)
array([[ 5,  6,  7,  8,  9],
       [ 5,  6,  7,  8,  9],
       [15, 16, 17, 18, 19],
       [15, 16, 17, 18, 19]])
\end{verbatim}
\end{methoddesc}


\begin{methoddesc}[numarray]{resize}{shape}
  \label{arraymethod:resize}
   \method{resize} shrinks/grows the array to new \var{shape}, possibly
    replacing the underlying buffer object.
\begin{verbatim}
>>> a = array([0, 1, 2, 3])
>>> a.resize(10)
array([0, 1, 2, 3, 0, 1, 2, 3, 0, 1])
\end{verbatim}
\end{methoddesc}


\begin{methoddesc}[numarray]{size}{}
  \label{arraymethod:size}
  \method{size} returns the total number of elements in this array.
  Synonymous with \method{nelements}.
\begin{verbatim}
>>> arange(100).size()
100
\end{verbatim}
\end{methoddesc}


\begin{methoddesc}[numarray]{type}{}
  \label{arraymethod:type}
   The \method{type} method returns the type of the array it is applied to.
   While we've been talking about them as Float32, Int16, etc., it is important
   to note that they are not character strings, they are instances of
   NumericType classes. 
\begin{verbatim}
>>> a = array([1,2,3])
>>> a.type()
Int32
>>> a = array([1], type=Complex64)
>>> a.type()
Complex64
\end{verbatim}
\end{methoddesc}


\begin{methoddesc}[numarray]{typecode}{}
  \label{arraymethod:typecode}
   The \method{typecode} method returns the typecode character of the array it
   is applied to.  \method{typecode} exists for backward compatibility with
   Numeric but the \method{type} method is preferred.
\begin{verbatim}
>>> a = array([1,2,3])
>>> a.typecode()
'l'
>>> a = array([1], type=Complex64)
>>> a.typecode()
'D'
\end{verbatim}
\end{methoddesc}


\begin{methoddesc}[numarray]{tofile}{file}
  \label{arraymethod:tofile}
  The \method{tofile} method writes the binary data of the array into
  \constant{file}.  If \constant{file} is a Python string, it is interpreted 
  as the name of a file to be created.  Otherwise, \constant{file} must be 
  Python file object to which the data will be written.  
\begin{verbatim}
>>> a = arange(65,100)
>>> a.tofile('test.dat')   # writes a's binary data to file 'test.dat'.
>>> f = open('test2.dat', 'w')
>>> a.tofile(f)            # writes a's binary data to file 'test2.dat'
\end{verbatim}
   Note that the binary representation of array data depends on the platform,
   with some platforms being little endian (sys.byteorder == 'little') and
   others being big endian.  The byte order of the array data is \emph{not}
   recorded in the file, nor are the array's shape and type.
\end{methoddesc}


\begin{methoddesc}[numarray]{tolist}{}
  \label{arraymethod:tolist}
   Calling an array's \method{tolist} method returns a hierarchical python list
   version of the same array:
\begin{verbatim}
>>> print a
[[65 66 67 68 69 70 71]
 [72 73 74 75 76 77 78]
 [79 80 81 82 83 84 85]
 [86 87 88 89 90 91 92]
 [93 94 95 96 97 98 99]]
>>> print a.tolist()
[[65, 66, 67, 68, 69, 70, 71], [72, 73, 74, 75, 76, 77, 78], [79, 80,
81, 82, 83, 84, 85], [86, 87, 88, 89, 90, 91, 92], [93, 94, 95, 96, 97,
98, 99]]
\end{verbatim}
\end{methoddesc}


\begin{methoddesc}[numarray]{tostring}{}
  \label{arraymethod:tostring}
   The \method{tostring} method returns a string representation of the 
   array data.
\begin{verbatim}
>>> a = arange(65,70)
>>> a.tostring()
'A\x00\x00\x00B\x00\x00\x00C\x00\x00\x00D\x00\x00\x00E\x00\x00\x00'
\end{verbatim}
Note that the arangement of the printable characters and interspersed NULL
characters is dependent on machine architecture.  The layout shown here is
for little endian platform.
\end{methoddesc}


\begin{methoddesc}[numarray]{transpose}{axis=-1}
  \label{arraymethod:transpose}
  \method{transpose} re-shapes the array by permuting it's dimensions
  as specified by 'axes'.  If 'axes' is none, \method{transpose}
  reverses the array's dimensions.  \method{transpose} operates
  in-place and returns None.
\begin{verbatim}
>>> a = arange(9, shape=(3,3))
>>> a.transpose()
>>> a
array([[0, 3, 6],
       [1, 4, 7],
       [2, 5, 8]])
\end{verbatim}
\end{methoddesc}


\begin{methoddesc}[numarray]{stddev}{}
  \label{arraymethod:stddev}
  The \method{stddev} method returns the standard deviation of all elements in
  an array.
\begin{verbatim}
>>> arange(10).stddev()
3.0276503540974917
\end{verbatim}
\end{methoddesc}


\begin{methoddesc}[numarray]{sum}{}
  \label{arraymethod:sum}
  The \method{sum} method returns the sum of all elements in an array.
\begin{verbatim}
>>> arange(10).sum()
45
\end{verbatim}
\end{methoddesc}


\begin{methoddesc}[numarray]{swapaxes}{axis1, axis2}
  \label{arraymethod:swapaxes}
  The \method{swapaxes} method adjusts the strides of an array so that
  the two specified axes appear to be swapped.  \method{swapaxes} operates
  in place and returns None.
\begin{verbatim}
>>> a = arange(25, shape=(5,5))
>>> a.swapaxes(0,1)
>>> a
array([[ 0,  5, 10, 15, 20],
       [ 1,  6, 11, 16, 21],
       [ 2,  7, 12, 17, 22],
       [ 3,  8, 13, 18, 23],
       [ 4,  9, 14, 19, 24]])
\end{verbatim}
\end{methoddesc}


\begin{methoddesc}[numarray]{togglebyteorder}{}
  \label{arraymethod:togglebyteorder}
  The \method{togglebyteorder} method adjusts the byte order state 
  variable for an array, with ``little'' being replaced by ``big'' and ``big''
  being replaced by ``little''.  \method{togglebyteorder} just reinterprets
  the existing data, it does not actually rearrange bytes.
\begin{verbatim}
>>> a = arange(4)
>>> a.togglebyteorder()
>>> a
array([       0, 16777216, 33554432, 50331648])
\end{verbatim}
\end{methoddesc}

\begin{methoddesc}[numarray]{trace}{}
  \label{arraymethod:togglebyteorder}
  The \method{trace} method returns the sum of the diagonal elements
  of an array.
\begin{verbatim}
>>> a = arange(25, shape=(5,5))
>>> a.trace()
60
\end{verbatim}
\end{methoddesc}


\begin{methoddesc}[numarray]{view}{}
  \label{arraymethod:view} The \method{view} method returns a new
  state object for an array but does not actually copy the array's
  data; views are used to reinterpret an existing data buffer by 
  changing the array's properties.
\begin{verbatim}
>>> a = arange(4)
>>> b = a.view()
>>> b.shape = (2,2)
>>> a
array([0, 1, 2, 3])
>>> b
array([[0, 1],
       [2, 3]])
>>> a is b
False
>>> a._data is b._data
True
\end{verbatim}
\end{methoddesc}


When using Python 2.2 or later, there are four public attributes which
correspond to those of Numeric type objects. These are \member{shape},
\member{flat}, \member{real}, and \member{imag} (or \member{imaginary}). The
following methods are used to implement and provide an alternative to using
these attributes.


\begin{methoddesc}[numarray]{getshape}{}
\end{methoddesc}
\begin{methoddesc}[numarray]{setshape}{}
   The \method{getshape} method returns the tuple that gives the shape of the
   array.  \method{setshape} assigns its argument (a tuple) to the internal
   attribute which defines the array shape. When using Python 2.2 or later, the
   \member{shape} attribute can be accessed or assigned to, which is equivalent
   to using these methods.
\begin{verbatim}
>>> a = arange(12)
>>> a.setshape((3,4))
>>> print a.getshape()
(3, 4)
>>> print a
[[ 0  1  2  3]
 [ 4  5  6  7]
 [ 8  9 10 11]]
\end{verbatim}
\end{methoddesc}


\begin{methoddesc}[numarray]{getflat}{}
   The \method{getflat} method is equivalent to using the \member{flat}
   attribute of Numeric. For compatibility with Numeric, there is no
   \method{setflat} method, although the attribute can in fact be set using
   \method{setshape}.
\begin{verbatim}
>>> print a
[[ 0  1  2  3]
 [ 4  5  6  7]
 [ 8  9 10 11]]
>>> print a.getflat()
[ 0  1  2  3  4  5  6  7  8  9 10 11]
\end{verbatim}
\end{methoddesc}


\begin{methoddesc}[numarray]{getreal}{}
\end{methoddesc}
\begin{methoddesc}[numarray]{setreal}{}
   The \method{getreal} and \method{setreal} methods can be used to access or
   assign to the real part of an array containing imaginary elements.
\end{methoddesc}


\begin{methoddesc}[numarray]{getimag}{}
\end{methoddesc}
\begin{methoddesc}[numarray]{getimaginary}{}
\end{methoddesc}
\begin{methoddesc}[numarray]{setimag}{}
\end{methoddesc}
\begin{methoddesc}[numarray]{setimaginary}{}
   The \method{getimag} and \method{setimag} methods can be used to access or
   assign to the imaginary part of an array containing imaginary elements.
   \method{getimaginary} is equivalent to \method{getimag}, and
   \method{setimaginary} is equivalent to \method{setimag}.
\end{methoddesc}

%% Local Variables:
%% mode: LaTeX
%% mode: auto-fill
%% fill-column: 79
%% indent-tabs-mode: nil
%% ispell-dictionary: "american"
%% reftex-fref-is-default: nil
%% TeX-auto-save: t
%% TeX-command-default: "pdfeLaTeX"
%% TeX-master: "numarray"
%% TeX-parse-self: t
%% End:

\chapter{Array Attributes}
\label{cha:array-attributes}

There are four public array attributes; however, they are only available 
in Python 2.2 or later. There are array methods that may be used instead. The
attributes are \code{shape, flat, real,} and \code{imaginary}.


\begin{memberdesc}[numarray]{shape}
   Accessing the \member{shape} attribute is equivalent to calling the
   \method{getshape} method; it returns the shape tuple.  Assigning a value to
   the shape attribute is equivalent to calling the \method{setshape} method.
\begin{verbatim}
>>> print a
[[0 1 2]
 [3 4 5]
 [6 7 8]]
>>> print a.shape
(3,3)
>>> a.shape = ((9,))
>>> print a.shape
(9,)
\end{verbatim}
\end{memberdesc}


\begin{memberdesc}[numarray]{flat}
   \label{mem:numarray:flat}
   Accessing the flat attribute of an array returns the flattened, or
   \method{ravel}ed version of that array, without having to do a function
   call.  This is equivalent to calling the \method{getflat} method. The
   returned array has the same number of elements as the input array, but it is
   of rank-1. One cannot set the flat attribute of an array, but one can use
   the indexing and slicing notations to modify the contents of the array:
\begin{verbatim}
>> print a
[[0 1 2]
 [3 4 5]
 [6 7 8]]
>> print a.flat
0 1 2 3 4 5 6 7 8]
>> a.flat[4] = 100
>> print a
[[  0   1   2]
 [  3 100   5]
 [  6   7   8]]
>> a.flat = arange(9,18)
>> print a
[[ 9 10 11]
 [12 13 14]
 [15 16 17]]
\end{verbatim}
\end{memberdesc}


\begin{memberdesc}[numarray]{real}
\end{memberdesc}
\begin{memberdesc}[numarray]{imag}
\end{memberdesc}
\begin{memberdesc}[numarray]{imaginary}
   These attributes exist only for complex arrays. They return respectively
   arrays filled with the real and imaginary parts of the elements. The
   equivalent methods for getting and setting these values are
   \method{getreal}, \method{setreal}, \method{getimag}, and \method{setimag}.
   \method{getimaginary} and \method{setimaginary} are synonyms for
   \method{getimag} and \method{setimag} respectively, and \method{.imag} is a
   synonym for \method{.imaginary}.  The arrays returned are not contiguous
   (except for arrays of length 1, which are always contiguous).
   The attributes \member{real}, \member{imag}, and \member{imaginary} are 
   modifiable:
\begin{verbatim}
>>> print x
[ 0.             +1.j               0.84147098+0.54030231j 0.90929743-0.41614684j]
>>> print x.real
[ 0.          0.84147098  0.90929743]
>>> print x.imag
[ 1.          0.54030231 -0.41614684]
>>> x.imag = arange(3)
>>> print x
[ 0.        +0.j  0.84147098+1.j  0.90929743+2.j]
>>> x = reshape(arange(10), (2,5)) + 0j # make complex array
>>> print x
[[ 0.+0.j  1.+0.j  2.+0.j  3.+0.j  4.+0.j]
 [ 5.+0.j  6.+0.j  7.+0.j  8.+0.j  9.+0.j]]
>>> print x.real
[[ 0.  1.  2.  3.  4.]
 [ 5.  6.  7.  8.  9.]]
>>> print x.type(), x.real.type()
Complex64 Float64
>>> print x.itemsize(), x.imag.itemsize()
16 8
\end{verbatim}
\end{memberdesc}


%% Local Variables:
%% mode: LaTeX
%% mode: auto-fill
%% fill-column: 79
%% indent-tabs-mode: nil
%% ispell-dictionary: "american"
%% reftex-fref-is-default: nil
%% TeX-auto-save: t
%% TeX-command-default: "pdfeLaTeX"
%% TeX-master: "numarray"
%% TeX-parse-self: t
%% End:

\chapter{Character Array}
\label{cha:character-array}
\declaremodule{extension}{numarray.strings}
\index{character array}
\index{string array}

\section{Introduction}
\label{sec:chararray-intro}

\code{numarray}, like \code{Numeric}, has support for arrays of character data
(provided by the \code{numarray.strings} module) in addition to arrays of
numbers.  The support for character arrays in \code{Numeric} is relatively
limited, restricted to arrays of single characters.  In contrast,
\code{numarray} supports arrays of fixed length strings.  As an additional
enhancement, the \code{numarray} design supports interleaving arrays of
characters with arrays of numbers, with both occupying the same memory buffer.
This provides basic infrastructure for building the arrays of heterogenous
records as provided by \code{numarray.records} (see chapter
\ref{cha:record-array}).  Currently, neither \code{Numeric} nor \code{numarray}
provides support for unicode.

Each character array is a \index{CharArray} \code{CharArray} object in the
\code{numarray.strings} module.  The easiest way to construct a character array
is to use the \code{numarray.strings.array()} function.  For example:

\begin{verbatim}
  >>> import numarray.strings as str
  >>> s = str.array(['Smith', 'Johnson', 'Williams', 'Miller'])
  >>> print s
  ['Smith', 'Johnson', 'Williams', 'Miller']
  >>> s.itemsize()
  8
\end{verbatim}
In this example, this string array has 4 elements.  The maximum string length
is automatically determined from the data.  In this case, the created array will
support fixed length strings of 8 characters (since the longest name is 8
characters long).

The character array is just like an array in numarray, except that now each
element is conceptually a Python string rather than a number.  We can do the
usual indexing and slicing:

\begin{verbatim}
  >>> print s[0]
  'Smith'
  >>> print s[:2]
  ['Smith', 'Johnson']
  >>> s[:2] = 'changed'
  >>> print s
  ['changed', 'changed', 'Williams', 'Miller']
\end{verbatim}

\section{Character array stripping, padding, and truncation}
\label{sec:chararray-clip-pad-truncate}
CharArrays are designed to store fixed length strings of visible ASCII text.
You may have noticed that although a \code{CharArray} stores fixed length
strings, it displays variable length strings.  This is a result of the
stripping and padding policies of the CharArray class.  

When an element of a \code{CharArray} is fetched trailing whitespace is
stripped off.  The sole exception to this rule is that a single whitespace is
never stripped down to the empty string.  \code{numarray.strings} defines
whitespace as an ASCII space, formfeed, newline, carriage return, tab, or
vertical tab.

When a string is assigned to a \code{CharArray}, the string is considered
terminated by the first of any NULL characters it contains and is padded with
spaces to the full length of the \code{CharArray} itemsize.  Thus, the memory
image of a \code{CharArray} element does not include anything at or after the
first NULL in an assigned string; instead, there are spaces, and no terminating
NULL character at all.

When a string which is longer than the \code{itemsize()} is assigned to a
\code{CharArray}, it is silently truncated.

The \code{RawCharArray} baseclass of \code{CharArray} implements transparent
\code{strip()} and \code{pad()} methods, enabling the storage and retrieval of
arbitrary ASCII values within array elements.  For \code{RawCharArray}, all
array elements are identical in percieved length.  Alternate
stripping and padding policies can be implemented by subclassing
\code{CharArray} or \code{RawCharArray}.

\section{Character array functions}
\label{sec:chararray-func}
\begin{funcdesc}{array}{buffer=None, itemsize=None, shape=None, byteoffset=0,
    bytestride=None, kind=CharArray}
\label{func:str.array}
   The function \code{array} is, for most practical purposes, all a user needs 
   to know to construct a character array.

   The first argument, \code{buffer}, may be any one of the following:

   (1) \code{None} (default).  The constructor will allocate a writeable memory
   buffer which will be uninitialized.  The user must assign valid data before
   trying to read the contents or before writing the character array to a disk
   file.
   
   (2) a Python string containing binary data.  For example:
\begin{verbatim}
     >>> print str.array('abcdefg'*10, itemsize=10)
     ['abcdefgabc', 'defgabcdef', 'gabcdefgab', 'cdefgabcde', 'fgabcdefga',
      'bcdefgabcd', 'efgabcdefg']
\end{verbatim}
   
   (3) a Python file object for an open file.  The data will be copied from 
   the file, starting at the current position of the read pointer.
   
   (4) a character array.  This results in a deep copy of the input character
   array; any other arguments to \code{array()} will be silently ignored.

\begin{verbatim}
     >>> print str.array(s)
     ['abcdefgabc', 'defgabcdef', 'gabcdefgab', 'cdefgabcde', 'fgabcdefga', 
      'bcdefgabcd', 'efgabcdefg']
\end{verbatim}
   
   (5) a nested sequence of strings.  The sequence nesting implies the
   shape of the string array unless shape is specified.

\begin{verbatim}
     >>> print str.array([['Smith', 'Johnson'], ['Williams', 'Miller']])
     [['Smith', 'Johnson'],
      ['Williams', 'Miller']]
\end{verbatim}

   \code{itemsize} can be used to increase or decrease the fixed size of an
   array element relative to the natural itemsize implied by any literal data
   specified by the \code{buffer} parameter.

\begin{verbatim}
     >>> print str.array([['Smith', 'Johnson'], ['Williams', 'Miller']], 
                         itemsize=2)
     [['Sm', 'Jo'],
      ['Wi', 'Mi']])
     >>> print str.array([['Smith', 'Johnson'], ['Williams', 'Miller']], 
                         itemsize=20)
     [['Smith', 'Johnson'],
      ['Williams', 'Miller']]
\end{verbatim}
   
   \code{shape} is the shape of the character array.  It can be an integer, in
   which case it is equivalent to the number of \var{rows} in a table.  It can
   also be a tuple implying the character array is an N-D array with fixed
   length strings as its elements. \code{shape} should be consistent with
   the number of elements implied by the data buffer and itemsize.

   \code{byteoffset} indicates an offset, specified in bytes, from the start
   of the array buffer to where the array data actually begins.  
   \code{byteoffset} enables the character array to be offset from the
   beginning of a table record.  This is mainly useful for implementing
   record arrays.

   \code{bytestride} indicates the separation, specified in bytes, between
   successive elements in the last dimension of the character array.
   \code{bytestride} is used in the implementation of record arrays to space
   character array elements with the size of the total record rather than the
   size of a single string.
   
   \code{kind} is used to specify the class of the created array, and should be
   \code{RawCharArray}, \code{CharArray}, or a subclass of either.
\end{funcdesc}
   
\begin{funcdesc}{num2char}{n, format, itemsize=32}
\label{func:str.num2char}
\code{num2char} formats the numarray \code{n} using the Python string format
\code{format} and stores the result in a character array with the specified
\code{itemsize}
\begin{verbatim}
     >>> num2char(num.arange(0.0,5), '%2.2f')
     CharArray(['0.00', '1.00', '2.00', '3.00', '4.00'])
\end{verbatim}
\end{funcdesc}

\section{Character array methods}
\label{sec:recarray-methods}
CharArray object has these public methods:

\begin{methoddesc}[RawCharArray]{tolist}{}
  \code{tolist()} returns a nested list of strings corresponding to all the
  elements in the array.
\end{methoddesc}
\begin{methoddesc}[RawCharArray]{copy}{}
  \code{copy()} returns a deep copy of the character array.
\end{methoddesc}
\begin{methoddesc}[RawCharArray]{raw}{}
  \code{raw()} returns the corresponding \code{RawCharArray} view.
\begin{verbatim}
     >>> c=str.array(["this","that","another"])
     >>> c.raw()
     RawCharArray(['this   ', 'that   ', 'another'])
\end{verbatim}
\end{methoddesc}
\begin{methoddesc}{resized}{n, fill=' '}
  \code{resized(n)} returns a copy of the array, resized so that each element
  is of length \code{n} characters.  Extra characters are filled with value
  \code{fill}.  Caution: do not confuse this method with \code{resize()} which
  changes the number of elements rather than the size of each element.
\begin{verbatim}
     >>> c = str.array(["this","that","another"])
     >>> c.itemsize()
     7
     >>> d = c.resized(20)
     >>> print d
     ['this', 'that', 'another']
     >>> d.itemsize()
     20
\end{verbatim}
\end{methoddesc}
\begin{methoddesc}[RawCharArray]{concatenate}{other}
  \code{concatenate(other)} returns a new array which corresponds to the
  element by element concatenation of \code{other} to \code{self}.  The
  addition operator is also overloaded to perform concatenation.
\begin{verbatim}
     >>> print map(str, range(3)) + array(["this","that","another one"])
     ['0this', '1that', '2another one']
     >>> print "prefix with trailing whitespace   " + array(["."])
     ['prefix with trailing whitespace   .']
\end{verbatim}
\end{methoddesc}
\begin{methoddesc}[RawCharArray]{sort}{}
  \code{sort} modifies the \code{CharArray} inplace so that its elements are in
  sorted order. \code{sort} only works for 1D character arrays.  Like the
  \code{sort()} for the Python list, \code{CharArray.sort()} returns nothing.
\begin{verbatim}
     >>> a=str.array(["other","this","that","another"])
     >>> a.sort()
     >>> print a
     ['another', 'other', 'that', 'this']
\end{verbatim}
\end{methoddesc}
\begin{methoddesc}[RawCharArray]{argsort}{}
   \code{argsort} returns a numarray corresponding to the permutation which will
   put the character array \code{self} into sorted order.  \code{argsort} only
   works for 1D character arrays.
\begin{verbatim}
     >>> a=str.array(["other","that","this","another"])
     >>> a.argsort()
     array([3, 0, 1, 2])
     >>> print a[ a.argsort ] 
     ['another', 'other', 'that', 'this']
\end{verbatim}
\end{methoddesc}
\begin{methoddesc}[RawCharArray]{amap}{f}
  \code{amap} applies the function \code{f} to every element of \code{self} and
  returns the nested list of the results.  The function \code{f} should operate
  on a single string and may return any Python value.
\end{methoddesc}
\begin{verbatim}
     >>> c = str.array(['this','that','another'])
     >>> print c.amap(lambda x: x[-2:])
     ['is', 'at', 'er']
\end{verbatim}
\begin{methoddesc}[RawCharArray]{match}{pattern, flags=0}
  \code{match} uses Python regular expression matching over all elements of a
  character array and returns a tuple of numarrays corresponding to the indices
  of \code{self} where the pattern matches. \code{flags} are passed directly to
  the Python pattern matcher defined in the \code{re} module of the standard
  library.
\begin{verbatim}
     >>> a=str.array([["wo","what"],["wen","erewh"]])
     >>> print a.match("wh[aebd]")
     (array([0]), array([1]))
     >>> print a[ a.match("wh[aebd]") ]
     ['what']
\end{verbatim}
\end{methoddesc}
\begin{methoddesc}[RawCharArray]{search}{pattern,flags=0}
  \code{search} uses Python regular expression searching over all elements of a
  character array and returns a tuple of numarrays corresponding to the indices
  of \code{self} where the pattern was found. \code{flags} are passed directly
  to the Python pattern \code{search} method defined in the \code{re} module of
  the standard library.  \code{flags} should be an or'ed combination (use the
  $\vert$ operator) of the following \code{re} variables: \code{IGNORECASE},
  \code{LOCALE}, \code{MULTILINE}, \code{DOTALL}, \code{VERBOSE}.  See the
  \code{re} module documentation for more details.
\end{methoddesc}
\begin{methoddesc}[RawCharArray]{sub}{pattern,replacement,flags=0,count=0}
  \code{sub} performs Python regular expression pattern substitution
  to all elements of a character array. \code{flags} and \code{count} work
  as they do for \code{re.sub()}.
\begin{verbatim}
     >>> a=str.array([["who","what"],["when","where"]])
     >>> print a.sub("wh", "ph")
     [['pho', 'phat'],
      ['phen', 'phere']])
\end{verbatim}
\end{methoddesc}
\begin{methoddesc}[RawCharArray]{grep}{pattern, flags=0}
  \code{grep} is intended to be used interactively to search a \code{CharArray}
  for the array of strings which match the given \code{pattern}.
  \code{pattern} should be a Python regular expression (see the \code{re}
  module in the Python standard library, which can be as simple as a string
  constant as shown below.
\begin{verbatim}
     >>> a=str.array([["who","what"],["when","where"]])
     >>> print a.grep("whe")
     ['when', 'where']
\end{verbatim}
\end{methoddesc}
\begin{methoddesc}[RawCharArray]{eval}{}
  \code{eval} executes the Python eval function on each element of a character
  array and returns the resulting numarray.  \code{eval} is intended for use
  converting character arrays to the corresponding numeric arrays.  An
  exception is raised if any string element fails to evaluate.
\begin{verbatim}
     >>> print str.array([["1","2"],["3","4."]]).eval()
     [[1., 2.],
      [3., 4.]]
\end{verbatim}
\end{methoddesc}
\begin{methoddesc}[RawCharArray]{maxLen}{}
  \code{maxLen} returns the minimum element length required to store the
  stripped elements of the array \code{self}.
\begin{verbatim}
     >>> print str.array(["this","there"], itemsize=20).maxLen()
     5
\end{verbatim}
\end{methoddesc}
\begin{methoddesc}[RawCharArray]{truncated}{}
  \code{truncated} returns an array corresponding to \code{self} resized
  so that it uses a minimum amount of storage.
\begin{verbatim}
     >>> a = str.array(["this  ","that"])
     >>> print a.itemsize()
     6
     >>> print a.truncated().itemsize()
     4
\end{verbatim}
\end{methoddesc}
\begin{methoddesc}[RawCharArray]{count}{s}
  \code{count} counts the occurences of string \code{s} in array \code{self}.
\begin{verbatim}
     >>> print array(["this","that","another","this"]).count("this")
     2
\end{verbatim}
\end{methoddesc}
\begin{methoddesc}[RawCharArray]{info}{}
   This will display key attributes of the character array.
\end{methoddesc}

%% Local Variables:
%% mode: LaTeX
%% mode: auto-fill
%% fill-column: 79
%% indent-tabs-mode: nil
%% ispell-dictionary: "american"
%% reftex-fref-is-default: nil
%% TeX-auto-save: t
%% TeX-command-default: "pdfeLaTeX"
%% TeX-master: "numarray"
%% TeX-parse-self: t
%% End:

\chapter{Record Array}
\label{cha:record-array}
\declaremodule{extension}{numarray.records}
\index{record array}

\section{Introduction}
\label{sec:recarray-intro}
One of the enhancements of \code{numarray} over \code{Numeric} is its 
support for record arrays, i.e. arrays with heterogeneous data types: 
for example, tabulated data where each field (or \var{column}) has the 
same data type but different fields may not.

Each record array is a \index{RecArray} \code{RecArray} object in the 
\code{numarray.records} module.  Most of 
the time, the easiest way to construct a record array is to use the 
\code{array()} function in the \code{numarray.records} module.  For example:
\begin{verbatim}
>>> import numarray.records as rec
>>> r = rec.array([('Smith', 1234),\
                   ('Johnson', 1001),\
                   ('Williams', 1357),\
                   ('Miller', 2468)], \
                   names='Last_name, phone_number')
\end{verbatim}
In this example, we \var{manually} construct a record array by longhand input of
the information.  This record array has 4 records (or rows) and two fields (or 
columns).  The names of the fields are specified in the \code{names} argument.  
When using this longhand input, the data types (formats) are 
automatically determined from the data.  In this case the first field is a 
string of 8 characters (since the longest name is 8 characters long) and 
the second field is an integer.

The record array is just like an array in numarray, except that now each 
element is a \code{Record}.  We can do the usual indexing and slicing:
\begin{verbatim}
>>> print r[0]
('Smith', 1234)
>>> print r[:2]
RecArray[ 
('Smith', 1234),
('Johnson', 1001)
]
\end{verbatim}
To access the record array's fields, use the \code{field()} method:
\begin{verbatim}
>>> print r.field(0)
['Smith', 'Johnson', 'Williams', 'Miller']
>>> print r.field('Last_name')
['Smith', 'Johnson', 'Williams', 'Miller']
\end{verbatim}
these examples show that the \code{field} method can accept either the 
numeric index or the field name.

Since each field is simply a numarray of numbers or strings, all 
functionalities of numarray are available to them.  The record array is one 
single object which allows the user to have either field-wise or row-wise 
access.  The following example:
\begin{verbatim}
>>> r.field('phone_number')[1]=9999
>>> print r[:2]
RecArray[ 
('Smith', 1234),
('Johnson', 9999)
]
\end{verbatim}
shows that a change using the field view will cause the corresponding change 
in the row-wise view without additional copying or computing.

\section{Record array functions}
\label{sec:recarray-func}
\begin{funcdesc}{array}{buffer=None, formats=None, shape=0, 
names=None, byteorder=sys.byteorder}
\label{func:rec.array}
   The function \code{array} is, for most practical purposes, all a user needs 
   to know to construct a record array.

   \code{formats} is a string containing the format information of all fields.  
   Each format can be the \var{letter code}, such as \code{f4} or \code{i2}, 
   or longer name like \code{Float32} or \code{Int16}.  For a list of letter 
   codes or the longer names, see Table \ref{tab:type-specifiers} or use 
   the \code{letterCode()} function.  A field of strings is specified by the 
   letter \code{a}, followed by an integer giving the maximum length; thus 
   \code{a5} is the format for a field of strings of (maximum) length of 5.  

   The formats are separated by commas, and each \var{cell} 
   (element in a field) can be a numarray itself, by attaching a number or a 
   tuple in front of the format specification.  So if 
   \code{formats='i4,Float64,a5,3i2,(2,3)f4,Complex64,b1'}, the record array 
   will have:
   \begin{verbatim}
   1st field: (4-byte) integers
   2nd field: double precision floating point numbers
   3rd field: strings of length 5
   4th field: short (2-byte) integers, each element is an array of shape=(3,)
   5th field: single precision floating point numbers, each element is an 
       array of shape=(2,3)
   6th field: double precision complex numbers
   7th field: (1-byte) Booleans
   \end{verbatim}
   \code{formats} specification takes precedence over the data.  For 
   example, if a field is specified as integers in \code{buffer}, but is 
   specified as floats in \code{formats}, it will be floats in the record 
   array.  If a field in the \code{buffer} is not convertible to the 
   corresponding data type in the \code{formats} specification, e.g. from 
   strings to numbers (integers, floats, Booleans) or vice versa, an 
   exception will be raised.
   
   \code{shape} is the shape of the record array.  It can be an integer, 
   in which case it is equivalent to the number of \var{rows} in a table.  
   It can also be a tuple where the record array is an N-D array with 
   \code{Records} as its elements. \code{shape} must be consistent with the 
   data in \code{buffer} for buffer types (5) and (6), explained below.
   
   \code{names} is a string containing the names of the fields, separated by 
   commas.  If there are more formats specified than names, then default 
   names will be used: If there are five fields specified in \code{formats} 
   but \code{names=None} (default), then the field names will be: 
   \code{c1, c2, c3, c4, c5}.  If \code{names="a,b"}, then the field 
   names will be: \code{a, b, c3, c4, c5}.
   
   If more names have been specified than there are formats, the extra names 
   will be discarded.  If duplicate names are specified, a \code{ValueError} 
   will be raised.  Field names are case sensitive, e.g. column \code{ABC} will 
   not be found if it is referred to as \code{abc} or \code{Abc} 
   (for example) when using the \code{field()} method.
   
   \code{byteorder} is a string of the value \code{big} or \code{little}, 
   referring to big endian or little endian.  This is useful when reading 
   (binary) data from a string or a file.  If not specified, it will use the 
   \code{sys.byteorder} value and the result will be platform dependent for 
   string or file input.
   
   The first argument, \code{buffer}, may be any one of the following:

   (1) \code{None} (default).  The data block in the record array will not be 
   initialized.  The user must assign valid data before trying to read the 
   contents or before writing the record array to a disk file.
   
   (2) a Python string containing binary data.  For example:
   \begin{verbatim}
   >>> r=rec.array('abcdefg'*100, formats='i2,a3,i4', shape=3, byteorder='big')
   >>> print r
   RecArray[ 
   (24930, 'cde', 1718051170),
   (25444, 'efg', 1633837924),
   (25958, 'gab', 1667523942)
   ]
   \end{verbatim}
   
   (3) a Python file object for an open file.  The data will be copied from 
   the file, starting at the current position of the read pointer, with 
   byte order as specified in \code{byteorder}.
   
   (4) a record array.  This results in a deep copy of the input record array; 
   any other arguments to \code{array()} will be silently ignored.
   
   (5) a list of numarrays.  There must be one such numarray for each field.  
   The \code{formats} and \code{shape} arguments to \code{array()} are not 
   required, but if they are specified, they need to be consistent with the 
   input arrays.  The shapes of all the input numarrays also need to be 
   consistent to one another.
   \begin{verbatim}
   # this will have 3 rows, each cell in the 2nd field is an array of 4 elements
   # note that the formats sepcification needs to reflect the data shape
   >>> arr1=numarray.arange(3)
   >>> arr2=numarray.arange(12,shape=(3,4))
   >>> r=rec.array([arr1, arr2],formats='i2,4f4')
   \end{verbatim}
   
   In this example, \code{arr2} is cast up to float.
   
   (6) a list of sequences.  Each sequence contains the 
   number(s)/string(s) of a record.  The example in the introduction 
   uses such input, sometimes called \var{longhand} input.  The data 
   types are automatically determined after comparing all input data.  
   Data of the same field will be cast to the highest type:
   \begin{verbatim}
   # the first field uses the highest data type: Float64
   >>> r=rec.array([[1,'abc'],(3.5, 'xx')]); print r
   RecArray[ 
   (1.0, 'abc'),
   (3.5, 'xx')
   ]
   \end{verbatim}
   unless overruled by the \code{formats} argument:
   \begin{verbatim}
   # overrule the first field to short integers, second field to shorter strings
   >>> r=rec.array([[1,'abc'],(3.5, 'xx')],formats='i2,a1'); print r
   RecArray[ 
   (1, 'a'),
   (3, 'x')
   ]
   \end{verbatim}
   Inconsistent data in the same field will cause a \code{ValueError}:
   \begin{verbatim}
   >>> r=rec.array([[1,'abc'],('a', 'xx')])
   ValueError: inconsistent data at row 1,field 0
   \end{verbatim}
   
   A record array with multi-dimensional numarray cells in a field can also 
   be constructed by using nested sequences:
   \begin{verbatim}
   >>> r=rec.array([[(11,12,13),'abc'],[(2,3,4), 'xx']]); print r
   RecArray[ 
   (array([11, 12, 13]), 'abc'),
   (array([2, 3, 4]), 'xx')
   ]
   \end{verbatim}
\end{funcdesc}
   
\begin{funcdesc}{letterCode}{}
   This function will list the letter codes acceptable by the \code{formats} 
   argument in \code{array()}.
\end{funcdesc}

\section{Record array methods}
\label{sec:recarray-methods}
RecArray object has these public methods:

\begin{methoddesc}[RecArray]{field}{fieldName}
   \code{fieldName} can be either an integer (field index) or string 
   (field name).
   \begin{verbatim}
   >>> r=rec.array([[11,'abc',1.],[12, 'xx', 2.]])
   >>> print r.field('c1')
   [11 12]
   >>> print r.field(0)  # same as field('c1')
   [11 12]
   \end{verbatim}
   To set values, simply use indexing or slicing, since each field is a 
   numarray:
   \begin{verbatim}
   >>> r.field(2)[1]=1000; r.field(1)[1]='xyz'
   >>> r.field(0)[:]=999
   >>> print r
   RecArray[ 
   (999, 'abc', 1.0),
   (999, 'xyz', 1000.0)
   ]
   \end{verbatim}
\end{methoddesc}
\begin{methoddesc}[RecArray]{info}{}
   This will display key attributes of the record array.
\end{methoddesc}

\section{Record object}
\label{sec:recarray-record}
\index{Record object}
Each single record (or \var{row}) in the record array is a 
\code{records.Record} object.  It has these methods:

\begin{methoddesc}[Record]{field}{fieldName}
\end{methoddesc}
\begin{methoddesc}[Record]{setfield}{fieldname, value}
   Like the \code{RecArray}, a \code{Record} object has the \code{field} 
   method to \var{get} the field value.  But since a \code{Record} object 
   is not an array, it does not take an index or slice, so one cannot 
   assign a value to it.  So a separate \var{set} method, \code{setfield()}, 
   is necessary:
   \begin{verbatim}
   >>> r[1].field(0)
   999
   >>> r[1].setfield(0, -1)
   >>> print r[1]
   (-1, 'xy', 1000.0)
   \end{verbatim}
   Like the \code{field()} method in \code{RecArray}, \code{fieldName} in 
   \code{Record}'s \code{field()} and \code{setfield()} methods can be 
   either an integer (index) or a string (field name).
\end{methoddesc}


%% Local Variables:
%% mode: LaTeX
%% mode: auto-fill
%% fill-column: 79
%% indent-tabs-mode: nil
%% ispell-dictionary: "american"
%% reftex-fref-is-default: nil
%% TeX-auto-save: t
%% TeX-command-default: "pdfeLaTeX"
%% TeX-master: "numarray"
%% TeX-parse-self: t
%% End:

\chapter{Object Array}
\label{cha:object-array}
\declaremodule{extension}{numarray.objects}
\index{object array}

\section{Introduction}
\label{sec:objectarray-intro}

\code{numarray}, like \code{Numeric}, has support for arrays of objects in
addition to arrays of numbers.  Arrays of objects are supported by the
\code{numarray.objects} module.  The \index{ObjectArray} \code{ObjectArray}
class is used to represent object arrays.  

The easiest way to construct an object array is to use the
\code{numarray.objects.array()} function.  For example:

\begin{verbatim}
  >>> import numarray.objects as obj
  >>> o = obj.array(['S', 'J', 1, 'M'])
  >>> print o
  ['S' 'J' 1 'M']
  >>> print o + o
  ['SS' 'JJ' 2 'MM']
\end{verbatim}

In this example, the array contains 3 Python strings and an integer, but the
array elements can be any Python object.  For each pair of elements, the
\function{add} operator is applied.  For strings, \function{add} is defined as
string concatenation.  For integers, \function{add} is defined as numerical
addition.  For a class object, the \function{__add__} and \function{__radd__}
methods would define the result.

\class{ObjectArray} is defined as a subclass of numarray's structural array
class, \class{NDArray}.  As a result, we can do the usual indexing and slicing:

\begin{verbatim}
  >>> import numarray.objects as obj
  >>> print s[0]
  'S'
  >>> print s[:2]
  ['S' 'J']
  >>> s[:2] = 'changed'
  >>> print s
  ['changed' 'changed' 1 'M']
  >>> a = obj.fromlist(numarray.arange(100), shape=(10,10))
  >>> a[2:5, 2:5]
  ObjectArray([[22, 23, 24],
               [32, 33, 34],
               [42, 43, 44]])
\end{verbatim}

\section{Object array functions}
\label{sec:objectarray-func}
\begin{funcdesc}{array}{sequence=None, shape=None, typecode='O'}
\label{func:obj.array}
   The function \function{array} is, for most practical purposes, all a user needs 
   to know to construct an object array.

   The first argument, \code{sequence}, can be an arbitrary sequence of Python
   objects, such as a list, tuple, or another object array.  

\begin{verbatim}
  >>> import numarray.objects as obj
  >>> class C:
  ...     pass
  >>> c = C()
  >>> a = obj.array([c, c, c])
  >>> a
  ObjectArray([c, c, c])
\end{verbatim}
   
   Like objects in Python lists, objects in object arrays are referred to, not
   copied, so changes to the objects are reflected in the originals because
   they are one and the same.

\begin{verbatim}
     >>> a[0].attribute  = 'this'
     >>> c.attribute
     'this'
\end{verbatim}
   
   The second argument, \code{shape}, optionally specifies the shape of the
   array.  If no \code{shape} is specified, the shape is implied by the
   sequence.

\begin{verbatim}
  >>> import numarray.objects as obj
  >>> class C:
  ...     pass
  >>> c = C()
  >>> a = obj.fromlist([c, c, c])
  >>> a
  ObjectArray([c, c, c])
\end{verbatim}
   
   The last argument, \code{typecode}, is there for backward compatibility with
   Numeric; it must be specified as 'O'.

\end{funcdesc}
   
\begin{funcdesc}{asarray}{obj}
  \label{func:obj.asarray}
  \code{asarray} converts sequences which are not object arrays into object
  arrays.  If \code{obj} is already an \class{ObjectArray}, it is returned
  unaltered.
\begin{verbatim}
  >>> import numarray.objects as obj
  >>> a = obj.asarray([1,''this'',''that''])
  >>> a
  ObjectArray([1 'this' 'that'])
  >>> b = obj.asarray(a)
  >>> b is a
  True
\end{verbatim}
\end{funcdesc}

\begin{funcdesc}{choose}{selector, population, output=None}
  \label{func:obj.choose}
  \code{choose} selects elements from \var{population} based on the values in
  \var{selector}, either returning the selected array or storing it in the
  optional \code{ObjectArray} specified by \var{output}.  \var{selector} should
  be an integer sequence where each element is within the range 0 to
  \function{len}{population}.  \var{population} should be a sequence of
  \class{ObjectArray}s. The shapes of \var{selector} and each element of
  \var{population} must be mutually broadcastable.
\begin{verbatim}
  >>> import numarray.objects as obj
  >>> s = num.arange(25, shape=(5,5)) % 3
  >>> p = obj.fromlist(["foo", 1, {"this":"that"}])
  >>> obj.choose(s, p)
  ObjectArray([['foo', 1, {'this': 'that'}, 'foo', 1],
    [{'this': 'that'}, 'foo', 1, {'this': 'that'}, 'foo'],
    [1, {'this': 'that'}, 'foo', 1, {'this': 'that'}],
    ['foo', 1, {'this': 'that'}, 'foo', 1],
    [{'this': 'that'}, 'foo', 1, {'this': 'that'}, 'foo']])
  
\end{verbatim}
\end{funcdesc}

\begin{funcdesc}{sort}{objects, axis=-1, output=None}
  \label{func:obj.sort}
  \code{sort} sorts the elements from \var{objects} along the specified
  \var{axis}.  If an output array is specified, the result is stored there
  and the return value is None,  otherwise the sort is returned.
\begin{verbatim}
    >>> import numarray.objects as obj
    >>> a = obj.ObjectArray(shape=(5,5))
    >>> a[:] = range(5,0,-1)
    >>> obj.sort(a)
    ObjectArray([[1, 2, 3, 4, 5],
                 [1, 2, 3, 4, 5],
                 [1, 2, 3, 4, 5],
                 [1, 2, 3, 4, 5],
                 [1, 2, 3, 4, 5]])
    >>> a[:] = range(5,0,-1)
    >>> a.transpose()
    >>> obj.sort(a, axis=0)
    ObjectArray([[1, 1, 1, 1, 1],
                 [2, 2, 2, 2, 2],
                 [3, 3, 3, 3, 3],
                 [4, 4, 4, 4, 4],
                 [5, 5, 5, 5, 5]])
\end{verbatim}
\end{funcdesc}

\begin{funcdesc}{argsort}{objects, axis=-1, output=None}
  \label{func:obj.argsort}
  \code{argsort} returns the sort order for the elements from \var{objects}
  along the specified \var{axis}.  If an output array is specified, the result
  is stored there and the return value is None, otherwise the sort order is
  returned.
\begin{verbatim}
  >>> import numarray.objects as obj
  >>> a = obj.ObjectArray(shape=(5,5))
  >>> a[:] = ['e','d','c','b','a']
  >>> obj.argsort(a)
  array([[4, 3, 2, 1, 0],
         [4, 3, 2, 1, 0],
         [4, 3, 2, 1, 0],
         [4, 3, 2, 1, 0],
         [4, 3, 2, 1, 0]])
\end{verbatim}
\end{funcdesc}

\begin{funcdesc}{take}{objects, indices, axis=0}
  \label{func:obj.take}
  \code{take} returns elements of \var{objects} specified by tuple of index
  arrays \var{indices} along the specified \var{axis}.
\begin{verbatim}
  >>> import numarray.objects as obj
  >>> o = obj.fromlist(range(10))
  >>> a = obj.arange(5)*2
  >>> obj.take(o, a)
  ObjectArray([0, 2, 4, 6, 8])
\end{verbatim}
\end{funcdesc}

\begin{funcdesc}{put}{objects, indices, values, axis=-1}
  \label{func:obj.put}
  \function{put} stores \var{values} at the locations of \var{objects}
  specified by tuple of index arrays \var{indices}.
\begin{verbatim}
  >>> import numarray.objects as obj
  >>> o = obj.fromlist(range(10))
  >>> a = obj.arange(5)*2
  >>> obj.put(o, a, 0); o
  ObjectArray([0, 1, 0, 3, 0, 5, 0, 7, 0, 9])
\end{verbatim}
\end{funcdesc}

\begin{funcdesc}{add}{objects1, objects2, out=None}
  \label{func:obj.add}
  \code{numarray.objects} defines universal functions which are named after and
  use the operators defined in the standard library module operator.py.  In
  addition, the operator hooks of the \class{ObjectArray} class are defined to
  call the operators.  \code{add} applies the \code{add} operator to
  corresponding elements of \var{objects1} and \var{objects2}.  Like the ufuncs
  in the numerical side of numarray, the object ufuncs support reduction and
  accumulation.  In addition to add, there are ufuncs defined for every unary
  and binary operator function in the standard library module operator.py.
  Some of these are given additional synonyms so that they use numarray naming
  conventions, e.g. \function{sub} has an alias named \function{subtract}.
\begin{verbatim}
  >>> import numarray.objects as obj
  >>> a = obj.fromlist(["t","u","w"])
  >>> a
  ObjectArray(['t', 'u', 'w'])
  >>> a+a
  ObjectArray(['tt', 'uu', 'ww'])
  >>> obj.add(a,a)
  ObjectArray(['tt', 'uu', 'ww'])
  >>> obj.add.reduce(a)
  'tuw' # not, as in the docs, an ObjectArray
  >>> obj.add.accumulate(a)
  ObjectArray(['t', 'tu', 'tuw']) # w, not v

  >>> a = obj.fromlist(["t","u","w"])
  >>> a
  ObjectArray(['t', 'u', 'w'])
  >>> a+a
  ObjectArray(['tt', 'uu', 'ww'])
  >>> obj.add(a,a)
  ObjectArray(['tt', 'uu', 'ww'])
  >>> obj.add.reduce(a)
  ObjectArray('tuv')
  >>> obj.add.accumulate(a)
  ObjectArray(['t', 'tu', 'tuv'])
\end{verbatim}
\end{funcdesc}

\section{Object array methods}
\label{sec:objectarray-methods}
\class{ObjectArray} maps each of its operator hooks (e.g. \code{__add__}) onto
the corresponding object ufunc (e.g. \code{numarray.objects.add}).  In addition
to its hook methods,  \class{ObjectArray} has these public methods:

\begin{methoddesc}[ObjectArray]{tolist}{}
  \code{tolist} returns a nested list of objects corresponding to all the
  elements in the array.
\end{methoddesc}

\begin{methoddesc}[ObjectArray]{copy}{}
  \code{copy} returns a shallow copy of the object array.
\end{methoddesc}

\begin{methoddesc}[ObjectArray]{astype}{type}
  \code{astype} returns either a copy of the \class{ObjectArray} or converts it
  into a numerical array of the specified \var{type}.
\end{methoddesc}

\begin{methoddesc}[ObjectArray]{info}{}
   This will display key attributes of the object array.
\end{methoddesc}

%% Local Variables:
%% mode: LaTeX
%% mode: auto-fill
%% fill-column: 79
%% indent-tabs-mode: nil
%% ispell-dictionary: "american"
%% reftex-fref-is-default: nil
%% TeX-auto-save: t
%% TeX-command-default: "pdfeLaTeX"
%% TeX-master: "numarray"
%% TeX-parse-self: t
%% End:

\chapter{C extension API}
\label{cha:C-API}
\declaremodule{extension}{C-API}
\index{C-API}

\begin{quote}
   This chapter describes the different available C-APIs for \module{numarray}
   based extension modules.
\end{quote}

While this chapter describes the \module{\numarray}-specifics for writing
extension modules, a basic understanding of \python extension modules is
expected. See \python's \ulink{Extending and
   Embedding}{http://www.python.org/doc/current/ext/ext.html} tutorial and the
\ulink{Python/C API}{http://www.python.org/doc/current/api/api.html}.

The numarray C-API has several different facets, and the first three facets
each make different tradeoffs between memory use, speed, and ease of use.  An
additional facet provides backwards compatability with legacy Numeric code.
The final facet consists of miscellaneous function calls used to implement
and utilize numarray, that were not part of Numeric.

In addition to most of the basic functionality provided by Numeric, these APIs
provide access to misaligned, byteswapped, and discontiguous \class{numarray}s.
Byteswapped arrays arise in the context of portable binary data formats where
the byteorder specified by the data format is not the same as the host
processor byte order.  Misaligned arrays arise in the context of tabular data:
files of records where arrays are superimposed on the column formed by a single
field in the record.  Discontiguous arrays arise from operations which permute
the shape and strides of an array, such as reshape.

\begin{description}
\item[Numeric compatability] This API provides a reasonable (if not complete)
   simulation of the Numeric C-API.  It is written in terms of the numarray
   high level API (discussed below) so that misbehaved numarrays are copied
   prior to processing with legacy Numeric code.  This API was actually written
   last because of the extra considerations in numarray not found in Numeric.
   Nevertheless, it is perhaps the most important because it enables writing
   extension modules which can be compiled for either numarray or Numeric.  It
   is also very useful for porting existing Numeric code.  See section
   \ref{sec:C-API:numeric-simulation}.
\item[High-level] This is the cleanest and eaisiest to use API.  It creates
   temporary arrays to handle difficult cases (discontiguous, byteswapped,
   misaligned) in C code.  Code using this API is written in terms of a pointer
   to a contiguous 1D array of C data.  See section
   \ref{sec:C-API:high-level-api}.
\item[Element-wise] This API handles misbehaved arrays without creating
     temporaries.  Code using this API is written to access single elements of
     an array via macros or functions.  \note{These macros are relatively slow
     compared to raw access to C data, and the functions even slower.} See
     section \ref{sec:C-API:element-wise-api}.
\item[One-dimensional] Code using this API get/sets consecutive elements of the
   inner dimension of an array, enabling the API to factor out tests for
   aligment and byteswapping to one test per array rather than one test per
   element.  Fewer tests means better performance, but at a cost of some
   temporary data and more difficult usage.  See section
   \ref{sec:C-API:One-dimensional-api}.
\item[New numarray functions] This last facet of the C-API consists of function
  calls which have been added to numarray which are orthogonal to each of the 3
  native access APIs and not part of the original Numeric. See section
  \ref{sec:C-API:new-numarray-functions}
\end{description}

\section{Numarray extension basics}
There's a couple things you need to do in order to access numarray's C-API in
your own C extension module:

\subsection{Include libnumarray.h}
  Near the top of your extension module add the lines:
\begin{verbatim}
  #include "Python.h"
  #include "libnumarray.h"
\end{verbatim} 
  This gives your C-code access to the numarray typedefs, macros, and function
  prototypes as well as the Python C-API.

  \subsection{Alternate include method}
  There's an alternate form of including libnumarray.h or arrayobject.h some
  people may prefer provided that they're willing to ignore the case where the
  numarray includes are not installed in the standard location.  The advantage
  of the following approach is that it automatically works with the default
  path to the Python include files which the distutils always provide.
  \begin{verbatim}
    #include "Python.h"
    #include "numarray/libnumarray.h"
  \end{verbatim}

\subsection{Import libnumarray}
  In your extension module's initialization function, add the line:
\begin{verbatim}
  import_libnumarray();
\end{verbatim} 
  
  import_libnumarray() is actually a macro which sets up a pointer to the
  numarray C-API function pointer table. If you forget to call
  import_libnumarray(), your extension module will crash as soon as you call a
  numarray API function, because your application will attempt to dereference a
  NULL pointer.

  Note that for the Numeric compatible API you should substitute arrayobject.h
  for libnumarray.h and import_array() for import_libnumarray() respectively.
  Unlike other versions of numarray prior to 1.0, arrayobject.h now includes
  only the Numeric simulation API.  To use the rest of the numarray API, you
  \emph{must} include libnumarray.h.  To use both, you must include both
  arrayobject.h and libnumarray.h, and you must both import_array() and
  import_libnumarray() in your module initialization function.

  \subsection{Writing a simple setup.py file for a numarray extension}
  One important practice for writing an extension module is the creation of a
  distutils setup.py file which automates both extension installation from
  source and the creation of binary distributions.  Here is a simple setup.py
  which builds a single extension module from a single C source file:
  \begin{verbatim}
    from distutils.core import setup, Extension
    from numarray.numarrayext import NumarrayExtension
    import sys
    
    if not hasattr(sys, 'version_info') or sys.version_info < (2,2,0,'alpha',0):        raise SystemExit, "Python 2.2 or later required to build this module."
    
    setup(name = "buildHistogram",
       version = "0.1",
       description = "",
       packages=[""],
       package_dir={"":""},
       ext_modules=[NumarrayExtension("buildHistogram",['buildHistogram.c'],\
         include_dirs=["./"],
         library_dirs=["./"],
         libraries=['m'])])
\end{verbatim}
\class{NumarrayExtension} is recommended rather than it's distutils baseclass
\class{Extension} because \class{NumarrayExtension} knows where to find the
numarray headers regardless of where the numarray installer or setup.py command
line options put them.  A disadvantage of using NumarrayExtension is that it
is numarray specific, so it does not work for compiling Numeric versions of the
extension.

See the Python manuals ``Installing Python Modules'' and ``Distributing Python
Modules'' for more information on how to use distutils.

\section{Fundamental data structures}
\label{C-API:fundamental-data-structures}

\subsection{Numarray Numerical Data Types}

Numarray hides the C implementation of its basic array elements behind a set of
C typedefs which specify the absolute size of the type in bits.  This approach
enables a programmer to specify data items of arrays and extension functions in
an explicit yet portable manner.  In contrast, basic C types are platform
relative, and so less useful for describing real physical data.  Here are the
names of the concrete Numarray element types:

\begin{itemize}
\item Bool                
\item Int8,      UInt8
\item Int16,     UInt16
\item Int32,     UInt32
\item Int64,     UInt64
\item Float32,   Float64
\item Complex32, Complex64
\end{itemize}

\subsection{NumarrayType}

The type of a numarray is communicated in C via one of the following
enumeration constants.  Type codes which are backwards compatible with Numeric
are defined in terms of these constants, but use these if you're not already
using the Numeric codes.  These constants communicate type requirements between
one function and another, since in C, you cannot pass a typedef as a value.
tAny is used to specify both ``no type requirement'' and ``no known type''
depending on context.   

\begin{verbatim}
typedef enum 
{
  tAny,

  tBool,
  tInt8,      tUInt8,
  tInt16,     tUInt16,
  tInt32,     tUInt32, 
  tInt64,     tUInt64,
  tFloat32,   tFloat64,
  tComplex32, tComplex64,

  tDefault = tFloat64,

#if LP64
  tLong = tInt64
#else
  tLong = tInt32
#endif

} NumarrayType;
\end{verbatim}

\subsection{PyArray_Descr}

\ctype{PyArray_Descr} is used to hold a few parameters related to the type of
an array and exists mostly for backwards compatability with Numeric.
\var{type_num} is a NumarrayType value.  \var{elsize} indicates the number of
bytes in one element of an array of that type.  \var{type} is a Numeric
compatible character code.

Numarray's \ctype{PyArray_Descr} is currently missing the type-casting,
\function{ones}, and \function{zeroes} functions.  Extensions which use these
missing Numeric features will not yet compile.  Arrays of type Object are not
yet supported.

\begin{verbatim}

typedef struct {
        int  type_num;  /* PyArray_TYPES */
        int  elsize;    /* bytes for 1 element */
        char type;      /* One of "cb1silfdFD "  Object arrays not supported. */
} PyArray_Descr;

\end{verbatim}

\subsection{PyArrayObject}

The fundamental data structure of numarray is the PyArrayObject, which is named
and layed out to provide source compatibility with Numeric.  It is compatible
with most but not all Numeric code.  The constant MAXDIM, the maximum number of
dimensions in an array, is defined as 40.  It should be noted that unlike
earlier versions of numarray, the present PyArrayObject structure is a first
class python object, with full support for the number protocols in C.
Well-behaved arrays have mutable fields which will reflect modifications back
into \python ``for free''.

\begin{verbatim}

typedef int maybelong;          /* towards 64-bit without breaking extensions. */

typedef struct {
        /* Numeric compatible stuff */

        PyObject_HEAD
        char *data;              /* points to the actual C data for the array */
        int nd;                  /* number of array shape elements */
        maybelong *dimensions;   /* values of shape elements */
        maybelong *strides;      /* values of stride elements */
        PyObject *base;          /* unused, but don't touch! */
        PyArray_Descr *descr;    /* pointer to descriptor for this array's type */
        int flags;               /* bitmask defining various array properties */

        /* numarray extras */

        maybelong _dimensions[MAXDIM];  /* values of shape elements */
        maybelong _strides[MAXDIM];     /* values of stride elements */
        PyObject *_data;       /* object must meet buffer API */
        PyObject *_shadows;    /* ill-behaved original array. */
        int      nstrides;     /* elements in strides array */
        long     byteoffset;   /* offset into buffer where array data begins */
        long     bytestride;   /* basic seperation of elements in bytes */
        long     itemsize;     /* length of 1 element in bytes */

        char      byteorder;   /* NUM_BIG_ENDIAN, NUM_LITTLE_ENDIAN */

        char      _unused0; 
        char      _unused1; 
        
        /* Don't expect the following vars to stay around.  Never use them.
        They're an implementation detail of the get/set macros. */

        Complex64      temp;   /* temporary for get/set macros */
        char *         wptr;   /* working pointer for get/set macros */
} PyArrayObject;

\end{verbatim}

\subsection{Flag Bits}

The following are the definitions for the bit values in the \var{flags} field
of each numarray.  Low order bits are Numeric compatible,  higher order bits
were added by numarray.

\begin{verbatim}
/* Array flags */
#define CONTIGUOUS        1       /* compatible, depends */
#define OWN_DIMENSIONS    2       /* always false */
#define OWN_STRIDES       4       /* always false */
#define OWN_DATA          8       /* always false */
#define SAVESPACE      0x10       /* not used */

#define ALIGNED       0x100       /* roughly: data % itemsize == 0 */
#define NOTSWAPPED    0x200       /* byteorder == sys.byteorder    */
#define WRITABLE      0x400       /* data buffer is writable       */

#define IS_CARRAY (CONTIGUOUS | ALIGNED | NOTSWAPPED)
\end{verbatim}

\section{Numeric simulation API}
\label{sec:C-API:numeric-simulation}

These notes describe the Numeric compatability functions which enable numarray
to utilize a subset of the extensions written for Numeric (NumPy).  Not all
Numeric C-API features and therefore not all Numeric extensions are currently
supported.  Users should be able to utilize suitable extensions written for
Numeric within the numarray environment by:

\begin{enumerate}
\item Writing a numarray setup.py file.
\item Scanning the extension C-code for all instances of array creation and
  return and making corrections as needed and specified below. 
\item Re-compiling the Numeric C-extension for numarray.
\end{enumerate}

Numarray's compatability with Numeric consists of 3 things:
\begin{enumerate}
\item A replacement header file, "arrayobject.h" which supplies simulation
   functions and macros for numarray just as the original arrayobject.h
   supplies the C-API for Numeric.
\item Layout and naming of the fundamental numarray C-type,
\ctype{PyArrayObject}, in a Numeric source compatible way.
\item A set of "simulation" functions.  These functions have the same names and
   parameters as the original Numeric functions, but operate on numarrays.  The
   simulation functions are also incomplete; features not currently supported
   should result in compile time warnings.
\end{enumerate}

\subsection{Simulation Functions}
\label{sec:C-API:compat:simulation-functions}

The basic use of numarrays by Numeric extensions is achieved in the extension
function's wrapper code by:
\begin{enumerate}
\item Ensuring creation of array objects by calls to simulation functions.
\item DECREFing each array or calling PyArray_Return.
\end{enumerate}

Unlike prior versions of numarray, this version *does* support access to array
objects straight out of PyArg_ParseTuple.  This is a consequence of a change to
the underlying object model, where a class instance has been replaced by
PyArrayObject.  Nevertheless, the ``right'' way to access arrays is either via
the high level interface or via emulated Numeric factory functions.  That way,
access to other python sequences is supported as well.  Using the ``right'' way
for numarray is also more important than for Numeric because numarray arrays
may be byteswapped or misaligned and hence unusable from simple C-code.  It
should be noted that the numarray and Numeric are not completely compatible,
and therefore this API does not provide support for string arrays or object
arrays.

The creation of array objects is illustrated by the following of wrapper code
for a 2D convolution function:

\begin{verbatim}
#include "python.h"
#include "arrayobject.h"

static PyObject *
Py_Convolve2d(PyObject *obj, PyObject *args)
{
        PyObject   *okernel, *odata, *oconvolved=Py_None;
        PyArrayObject *kernel, *data, *convolved;

        if (!PyArg_ParseTuple(args, "OO|O", &okernel, &odata, &oconvolved)) {
                return PyErr_Format(_Error, 
                                    "Convove2d: Invalid parameters.");  
                goto _fail;
        }
\end{verbatim}

The first step was simply to get object pointers to the numarray parameters to
the convolution function: okernel, odata, and oconvolved.  Oconvolved is an
optional output parameter, specified with a default value of Py_None which is
used when only 2 parameters are supplied at the python level.  Each of the
``o'' parameters should be thought of as an arbitrary sequence object, not
necessarily an array.

The next step is to call simulation functions which convert sequence objects
into PyArrayObjects.  In a Numeric extension, these calls map tuples and lists
onto Numeric arrays and assert their dimensionality as 2D.  The Numeric
simulation functions first map tuples, lists, and misbehaved numarrays onto
well-behaved numarrays.  Calls to these functions transparently use the
numarray high level interface and provide visibility only to aligned and
non-byteswapped array objects.

\begin{verbatim}
        kernel = (PyArrayObject *) PyArray_ContiguousFromObject(
                okernel, PyArray_DOUBLE, 2, 2);
        data = (PyArrayObject *) PyArray_ContiguousFromObject(
                odata, PyArray_DOUBLE, 2, 2);

        if (!kernel || !data) goto _fail;
\end{verbatim}

Extra processing is required to handle the output array \var{convolved},
cloning it from \var{data} if it was not specified.  Code should be supplied,
but is not, to verify that convolved and data have the same shape.  

\begin{verbatim}
        if (convolved == Py_None)
                convolved = (PyArrayObject *) PyArray_FromDims(
                        data->nd, data->dimensions, PyArray_DOUBLE);
        else
                convolved = (PyArrayObject *) PyArray_ContiguousFromObject(
                        oconvolved, PyArray_DOUBLE, 2, 2);
        if (!convolved) goto _fail;
\end{verbatim}

After converting all of the input paramters into \ctype{PyArrayObject}s, the
actual convolution is performed by a seperate function.  This could just as
well be done inline:

\begin{verbatim}
        Convolve2d(kernel, data, convolved);
\end{verbatim}

After processing the arrays, they should be DECREF'ed or returned using
\cfunction{PyArray_Return}.  It is generally not possible to directly return a
numarray object using \cfunction{Py_BuildValue} because the shadowing of
mis-behaved arrays needs to be undone.  Calling \cfunction{PyArray_Return}
destroys any temporary and passes the numarray back to \python.

\begin{verbatim}
        Py_DECREF(kernel);
        Py_DECREF(data);
        if (convolved != Py_None) {
                Py_DECREF(convolved);
                Py_INCREF(Py_None);
                return Py_None;
        } else
                return PyArray_Return(convolved);
_fail:
        Py_XDECREF(kernel);
        Py_XDECREF(data);
        Py_XDECREF(convolved);
        return NULL;
}
\end{verbatim}

Byteswapped or misaligned arrays are handled by a process of shadowing which
works like this:
\begin{enumerate}
\item When a "misbehaved" numarray is accessed via the Numeric simulation
  functions, first a well-behaved temporary copy (shadow) is created by
  NA_IoArray.
\item Operations performed by the extension function modifiy the data buffer
  belonging to the shadow.
\item On extension function exit, the shadow array is copied back onto the 
  original and the shadow is freed.
\end{enumerate}
All of this is transparent to the user; if the original array is well-behaved,
it works much like it always did; if not, what would have failed altogether
works at the cost of extra temporary storage.  Users which cannot afford the
cost of shadowing need to use numarray's native elementwise or 1D APIs.
\subsection{Numeric Compatible Functions}
\label{sec:C-API:compat:implemented-functions}

The following functions are currently implemented:
\begin{cfuncdesc}{PyObject*}{PyArray_FromDims}{int nd, int *dims, int type}
   This function will allocate a new numarray.

   An array created with PyArray_FromDims can be used as a temporary or
   returned using PyArray_Return.
   
   Used as a temporary, calling Py_DECREF deallocates it.   
\end{cfuncdesc}

\begin{cfuncdesc}{PyObject*}{PyArray_FromDimsAndData}{int nd, int *dims, int type, char *data}
   This function will allocate a numarray of the specified shape and type
   which will refer to the data buffer specified by \var{data}.  The contents
   of \var{data} will not be copied nor will \var{data} be deallocated upon
   the deletion of the array.
\end{cfuncdesc}

\begin{cfuncdesc}{PyObject*}{PyArray_ContiguousFromObject}{%
      PyObject *op, int type, int min_dim, int max_dim}% Returns an simulation
   object for a contiguous numarray of 'type' created from the sequence object
   'op'.  If 'op' is a contiguous, aligned, non-byteswapped numarray, then the
   simulation object refers to it directly.  Otherwise a well-behaved numarray
   will be created from 'op' and the simulation object will refer to it.
   min_dim and max_dim bound the expected rank as in Numeric.
   \code{min_dim==max_dim} specifies an exact rank.  \code{min_dim==max_dim==0}
   specifies \emph{any} rank.
\end{cfuncdesc}

\begin{cfuncdesc}{PyObject*}{PyArray_CopyFromObject}{%
      PyObject *op, int type, int min_dim, int max_dim}% Returns a contiguous
   array, similar to PyArray_FromContiguousObject, but always returning an
   simulation object referring to a new numarray copied from the original
   sequence.
\end{cfuncdesc}

\begin{cfuncdesc}{PyObject*}{PyArray_FromObject}{%
      PyObject  *op, int type, int min_dim, int max_dim}%
   Returns and simulation object based on 'op', possibly discontiguous.  The
   strides array must be used to access elements of the simulation object.
   
   If 'op' is a byteswapped or misaligned numarray, FromObject creates a
   temporary copy and the simulation object refers to it.
   
   If 'op' is a nonswapped, aligned numarray, the simulation object refers to
   it.
   
   If 'op' is some other sequence, it is converted to a numarray and the
   simulation object refers to that.
\end{cfuncdesc}

\begin{cfuncdesc}{PyObject*}{PyArray_Return}{PyArrayObject *apr}
   Returns simulation object 'apr' to python.  The simulation object itself is
   destructed.  The numarray it refers to (base) is returned as the result of
   the function.
   
   An additional check is (or eventually will be) performed to guarantee that
   rank-0 arrays are converted to appropriate python scalars.
   
   PyArray_Return has no net effect on the reference count of the underlying
   numarray.
\end{cfuncdesc}

\begin{cfuncdesc}{int}{PyArray_As1D}{PyObject **op, char **ptr, int *d1, int typecode}
   Copied from Numeric verbatim.
\end{cfuncdesc}

\begin{cfuncdesc}{int}{PyArray_As2D}{PyObject **op, char ***ptr, int *d1, int *d2, int typecode}
   Copied from Numeric verbatim.
\end{cfuncdesc}

\begin{cfuncdesc}{int}{PyArray_Free}{PyObject *op, char *ptr}
   Copied from Numeric verbatim. \note{This means including bugs and all!}
\end{cfuncdesc}

\begin{cfuncdesc}{int}{PyArray_Check}{PyObject *op}
   This function returns 1 if op is a PyArrayObject.  
\end{cfuncdesc}

\begin{cfuncdesc}{int}{PyArray_Size}{PyObject *op}
   This function returns the total element count of the array.
\end{cfuncdesc}

\begin{cfuncdesc}{int}{PyArray_NBYTES}{PyArrayObject *op}
   This function returns the total size in bytes of the array, and assumes that
bytestride == itemsize, so that the size is product(shape)*itemsize.
\end{cfuncdesc}

\begin{cfuncdesc}{PyObject*}{PyArray_Copy}{PyArrayObject *op}
   This function returns a copy of the array 'op'.  The copy returned is
   guaranteed to be well behaved, i.e. neither byteswapped nor misaligned.
\end{cfuncdesc}

\begin{cfuncdesc}{int}{PyArray_CanCastSafely}{PyArrayObject *op, int type}
  This function returns 1 IFF the array 'op' can be safely cast to 'type',
otherwise it returns 0.
\end{cfuncdesc}

\begin{cfuncdesc}{PyArrayObject*}{PyArray_Cast}{PyArrayObject *op, int type}
  This function casts the array 'op' into an equivalent array of type 'type'.
\end{cfuncdesc}

\begin{cfuncdesc}{PyArray_Descr*}{PyArray_DescrFromType}{int type}
This function returns a pointer to the array descriptor for 'type'.  The
numarray version of PyArray_Descr is incomplete and does not support casting,
getitem, setitem, one, or zero.
\end{cfuncdesc}

\begin{cfuncdesc}{int}{PyArray_isArray(PyObject *o)}
  This macro is designed to fail safe and return 0 when numarray is not
  installed at all.  When numarray is installed, it returns 1 iff object 'o' is
  a numarray, and 0 otherwise.  This macro facilitates the optional use of
  numarray within an extension.
\end{cfuncdesc}

\subsection{Unsupported Numeric Features}
\label{sec:C-API:compat:unsupported}

\begin{itemize}
\item PyArrayError 
\item PyArray_ObjectType() 
\item PyArray_Reshape()
\item PyArray_SetStringFunction() 
\item PyArray_SetNumericOps() 
\item PyArray_Take()
\item UFunc API
\end{itemize}

\section{High-level API}
\label{sec:C-API:high-level-api}

The high-level native API accepts an object (which may or may not be an array)
and transforms the object into an array which satisfies a set of ``behaved-ness
requirements''.  The idea behind the high-level API is to transparently convert
misbehaved numarrays, ordinary sequences, and python scalars into C-arrays.  A
``misbehaved array'' is one which is byteswapped, misaligned, or discontiguous.
This API is the simplest and fastest, provided that your arrays are small.  If
you find your program is exhausting all available memory, it may be time to
look at one of the other APIs.

\subsection{High-level functions}
\label{sec:C-API:high-level-functions}

The high-level support functions for interchanging \class{numarray}s between
\python{} and C are as follows:

\begin{cfuncdesc}{PyArrayObject*}{NA_InputArray}{%
      PyObject *seq, NumarrayType t, int requires}
The purpose of NA_InputArray is to transfer array data from \python to C.
\end{cfuncdesc}

\begin{cfuncdesc}{PyArrayObject*}{NA_OutputArray}{%
      PyObject *seq, NumarrayType t, int requires} The purpose of
NA_OutputArray is to transfer data from C to \python.  The output array must be
a PyArrayObject, i.e. it cannot be an arbitrary Python sequence.
\end{cfuncdesc}

\begin{cfuncdesc}{PyArrayObject*}{NA_IoArray}{%
      PyObject *seq, NumarrayType t, int requires} NA_IoArray has fully
bidirectional data transfer, creating the illusion of call-by-reference.
\end{cfuncdesc}

  For a well-behaved writable array, there is no difference between the three,
  as no temporary is created and the returned object is identical to the
  original object (with an additional reference).  For a mis-behaved input
  array, a well-behaved temporary will be created and the data copied from the
  original to the temporary.  Since it is an input, modifications to its
  contents are not guaranteed to be reflected back to \python, and in the case
  where a temporary was created, won't be.  For a mis-behaved output array, any
  data side-effects generated by the C code will be safely communicated back to
  \python, but the initial array contents are undefined.  For an I/O array, any
  required temporary will be initialized to the same contents as the original
  array, and any side-effects caused by C-code will be copied back to the
  original array.  The array factory routines of the Numeric compatability API
  are written in terms of NA_IoArray.
   
   The return value of each function (\cfunction{NA_InputArray},
   \cfunction{NA_OutputArray}, or \cfunction{NA_IoArray}) is either a reference
   to the original array object, or a reference to a temporary array.
   Following execution of the C-code in the extension function body this
   pointer should \emph{always} be DECREFed.  When a temporary is DECREFed, it
   is deallocated, possibly after copying itself onto the original array.  The
   one exception to this rule is that you should not DECREF an array returned
   via the NA_ReturnOutput function.
   
   The \var{seq} parameter specifies the original numeric sequence to be
   interfaced.  Nested lists and tuples of numbers can be converted by
   \cfunction{NA_InputArray} and \cfunction{NA_IoArray} into a temporary array.
   The temporary is lost on function exit.  Strictly speaking, allowing
   NA_IoArray to accept a list or tuple is a wart, since it will lose any side
   effects.  In principle, communication back to lists and tuples can be
   supported but is not currently.
   
   The \var{t} parameter is an enumeration value which defines the type the
   array data should be converted to.  Arrays of the same type are passed
   through unaltered, while mis-matched arrays are cast into temporaries of the
   specified type.  The value \constant{tAny} may be specified to indicate that
   the extension function can handle any type correctly so no temporary should
   is required.
   
   The \var{requires} integer indicates under what conditions, other than type
   mismatch, a temporary should be made.  The simple way to specify it is to
   use \constant{NUM_C_ARRAY}.  This will cause the API function to make a
   well-behaved temporary if the original is byteswapped, misaligned, or
   discontiguous.  

There is one other pair of high level function which serves to return output
arrays as the function value: NA_OptionalOutputArray and NA_ReturnOutput.

\begin{cfuncdesc}{PyArrayObject*}{NA_OptionalOutputArray}{%
      PyObject *seq, NumarrayType t, int requires, PyObject *master}%
   \cfunction{NA_OptionalOutputArray} is essentially
   \cfunction{NA_OutputArray}, but with one twist: if the original array
   \var{seq} has the value \constant{NULL} or \constant{Py_None}, a copy of
   \var{master} is returned.  This facilitates writing functions where the
   output array may or may-not be specified by the \python{} user.  
\end{cfuncdesc}

\begin{cfuncdesc}{PyObject*}{NA_ReturnOutput}{PyObject *seq, PyObject *shadow}
   \cfunction{NA_ReturnOutput} accepts as parameters both the original
   \var{seq} and the value returned from
   \cfunction{NA_OptionalOutputArray}, \var{shadow}.  If \var{seq} is
   \constant{Py_None} or \constant{NULL}, then \var{shadow} is returned.
   Otherwise, an output array was specified by the user, and \constant{Py_None}
   is returned.  This facilitates writing functions in the numarray style
   where the specification of an output array renders the function ``mute'',
   with all side-effects in the output array and None as the return value.
\end{cfuncdesc}

\subsection{Behaved-ness Requirements}

Calls to the high level API specify a set of requirements that incoming arrays
must satisfy.  The requirements set is specified by a bit mask which is or'ed
together from bits representing individual array requirements.  An ordinary C
array satisfies all 3 requirements: it is contiguous, aligned, and not
byteswapped.  It is possible to request arrays satisfying any or none of the
behavedness requirements.  Arrays which do not satisfy the specified
requirements are transparently ``shadowed'' by temporary arrays which do
satisfy them.  By specifying \constant{NUM_UNCONVERTED}, a caller is certifying
that his extension function can correctly and directly handle the special cases
possible for a \class{NumArray}, excluding type differences.

\begin{verbatim}
typedef enum
{
        NUM_CONTIGUOUS=1,
        NUM_NOTSWAPPED=2,
        NUM_ALIGNED=4,
        NUM_WRITABLE=8,
        NUM_COPY=16,

        NUM_C_ARRAY  = (NUM_CONTIGUOUS | NUM_ALIGNED | NUM_NOTSWAPPED),
        NUM_UNCONVERTED = 0
}
\end{verbatim}

\function{NA_InputArray} will return a guaranteed writable result if
\constant{NUM_WRITABLE} is specified. A writable temporary will be made for
arrays which have readonly buffers.  Any changes made to a writable input array
\emph{may} be lost at extension exit time depending on whether or not a
temporary was required.  \function{NA_InputArray} will also return a guaranteed
writable result by specifying \constant{NUM_COPY}; with \constant{NUM_COPY}, a
temporary is \emph{always} made and changes to it are \emph{always} lost at
extension exit time.

Omitting \constant{NUM_WRITABLE} and \constant{NUM_COPY} from the
\var{requires} of \function{NA_InputArray} asserts that you will not modify the
array buffer in your C code.  Readonly arrays (e.g. from a readonly memory map)
which you attempt to modify can result in a segfault if \constant{NUM_WRITABLE}
or \constant{NUM_COPY} was not specified.

Arrays passed to \function{NA_IoArray} and \function{NA_OutputArray} must be
writable or they will raise an exception; specifing \constant{NUM_WRITABLE} or
\constant{NUM_COPY} to these functions has no effect.

\subsection{Example}
\label{sec:C-API:high-level:example}

A C wrapper function using the high-level API would typically look like the
following.\footnote{This function is taken from the convolve example in the
source distribution.}

\begin{verbatim}
#include "Python.h"
#include "libnumarray.h"

static PyObject *
Py_Convolve1d(PyObject *obj, PyObject *args)
{
        PyObject   *okernel, *odata, *oconvolved=Py_None;
        PyArrayObject *kernel, *data, *convolved;

        if (!PyArg_ParseTuple(args, "OO|O", &okernel, &odata, &oconvolved)) {
                PyErr_Format(_convolveError, 
                             "Convolve1d: Invalid parameters.");
                goto _fail;
        }

\end{verbatim}

First, define local variables and parse parameters.  \cfunction{Py_Convolve1d}
expects two or three array parameters in \var{args}: the convolution kernel,
the data, and optionally the return array.  We define two variables for each
array parameter, one which represents an arbitrary sequence object, and one
which represents a PyArrayObject which contains a conversion of the sequence.
If the sequence object was already a well-behaved numarray, it is returned
without making a copy.

\begin{verbatim}
        /* Align, Byteswap, Contiguous, Typeconvert */
        kernel  = NA_InputArray(okernel, tFloat64, NUM_C_ARRAY);
        data    = NA_InputArray(odata, tFloat64, NUM_C_ARRAY);
        convolved = NA_OptionalOutputArray(oconvolved, tFloat64, NUM_C_ARRAY, data);

        if (!kernel || !data || !convolved) {
                PyErr_Format( _convolveError, 
                             "Convolve1d: error converting array inputs.");
                goto _fail;
        }
\end{verbatim}

These calls to NA_InputArray and OptionalOutputArray require that the arrays be
aligned, contiguous, and not byteswapped, and of type Float64, or a temporary
will be created.  If the user hasn't provided a output array we ask
\cfunction{NA_OptionalOutputArray} to create a copy of the input \var{data}.
We also check that the array screening and conversion process succeeded by
verifying that none of the array pointers is NULL.

\begin{verbatim}
        if ((kernel->nd != 1) || (data->nd != 1)) {
                PyErr_Format(_convolveError,
                      "Convolve1d: arrays must have 1 dimension.");
                goto _fail;
        }

        if (!NA_ShapeEqual(data, convolved)) {
                PyErr_Format(_convolveError,
                "Convolve1d: data and output arrays need identitcal shapes.");
                goto _fail;
        }
\end{verbatim}

Make sure we were passed one-dimensional arrays, and data and output have the
same size.

\begin{verbatim}
        Convolve1d(kernel->dimensions[0], NA_OFFSETDATA(kernel),
                   data->dimensions[0],   NA_OFFSETDATA(data),
                   NA_OFFSETDATA(convolved));
\end{verbatim}

Call the C function actually performing the work.  NA_OFFSETDATA returns the
pointer to the first element of the array,  adjusting for any byteoffset.

\begin{verbatim}
        Py_XDECREF(kernel);
        Py_XDECREF(data);
\end{verbatim}

Decrease the reference counters of the input arrays.  These were increased by
\cfunction{NA_InputArray}.  Py_XDECREF tolerates NULL.  DECREF'ing the
PyArrayObject is how temporaries are released and in the case of
IO and Output arrays, copied back onto the original.

\begin{verbatim}
        /* Align, Byteswap, Contiguous, Typeconvert */
        return NA_ReturnOutput(oconvolved, convolved);
_fail:
        Py_XDECREF(kernel);
        Py_XDECREF(data);
        Py_XDECREF(convolved);
        return NULL;
}
\end{verbatim}

Now return the results, which are either stored in the user-supplied array
\var{oconvolved} and \constant{Py_None} is returned, or if the user didn't
supply an output array the temporary \var{convolved} is returned.

If your C function creates the output array you can use the following sequence
to pass this array back to \python{}:

\begin{verbatim}
        double *result;
        int m, n;
        .
        .
        .
        result = func(...);
        if(NULL == result)
            return NULL;
        return NA_NewArray((void *)result, tFloat64, 2, m, n);
}
\end{verbatim}

The C function \cfunction{func} returns a newly allocated (m, n) array in
\var{result}.  After we check that everything is ok, we create a new numarray
using \cfunction{NA_NewArray} and pass it back to \python.  \cfunction{NA_NewArray}
creates a \class{numarray} with \constant{NUM_C_ARRAY} properties.  If you wish to
create an array that is byte-swapped, or misaligned, you can use
\cfunction{NA_NewAll}.

The C-code of the core convolution function is uninteresting.  The main point
of the example is that when using the high-level API, numarray specific code is
confined to the wrapper function.  The interface for the core function can be
written in terms of primitive numarray/C data items, not objects.  This is
possible because the high level API can be used to deliver C arrays.

\begin{verbatim}
static void Convolve1d(long ksizex, Float64 *kernel, 
     long dsizex, Float64*data, Float64 *convolved) 
{ 
  long xc; long halfk = ksizex/2;

  for(xc=0; xc<halfk; xc++)
      convolved[xc] = data[xc];
  
  for(xc=halfk; xc<dsizex-halfk; xc++) {
      long xk;
      double temp = 0;
      for (xk=0; xk<ksizex; xk++)
         temp += kernel[xk]*data[xc-halfk+xk];
      convolved[xc] = temp;
  }
  
  for(xc=dsizex-halfk; xc<dsizex; xc++)
     convolved[xc] = data[xc];
}
\end{verbatim}

\section{Element-wise API}
\label{sec:C-API:element-wise-api}

The element-wise in-place API is a family of macros and functions designed to
get and set elements of arrays which might be byteswapped, misaligned,
discontiguous, or of a different type.  You can obtain \class{PyArrayObject}s
for these misbehaved arrays from the high-level API by specifying fewer
requirements (perhaps just 0, rather than NUM_C_ARRAY).  In this way, you can
avoid the creation of temporaries at a cost of accessing your array with these
macros and functions and a significant performance penalty.  Make no mistake,
if you have the memory, the high level API is the fastest.  The whole point of
this API is to support cases where the creation of temporaries exhausts either
the physical or virtual address space.  Exhausting physical memory will result
in thrashing, while exhausting the virtual address space will result in program
exception and failure.  This API supports avoiding the creation of the
temporaries, and thus avoids exhausting physical and virual memory, possibly
improving net performance or even enabling program success where simpler
methods would just fail.

\subsection{Element-wise functions}
\label{sec:C-API:element-wise:functions}

The single element macros each access one element of an array at a time, and
specify the array type in two places: as part of the PyArrayObject type
descriptor, and as ``type''.  The former defines what the array is, and the
latter is required to produce correct code from the macro.  They should
\emph{match}.  When you pass ``type'' into one of these macros, you are
defining the kind of array the code can operate on.  It is an error to pass a
non-matching array to one of these macros.  One last piece of advice: call
these macros carefully, because the resulting expansions and error messages are
a *obscene*.  Note: the type parameter for a macro is one of the Numarray
Numeric Data Types, not a NumarrayType enumeration value.

\subsubsection{Pointer based single element macros}
\label{sec:C-API:pointer-based-single}

\begin{cfuncdesc}{}{NA_GETPa}{PyArrayObject*, type, char*}
   aligning
\end{cfuncdesc}
\begin{cfuncdesc}{}{NA_GETPb}{PyArrayObject*, type, char*}
   byteswapping
\end{cfuncdesc}
\begin{cfuncdesc}{}{NA_GETPf}{PyArrayObject*, type, char*}
   fast (well-behaved)
\end{cfuncdesc}
\begin{cfuncdesc}{}{NA_GETP}{PyArrayObject*,  type, char*}
   testing: any of above
\end{cfuncdesc}
\begin{cfuncdesc}{}{NA_SETPa}{PyArrayObject*, type, char*, v}
\end{cfuncdesc}
\begin{cfuncdesc}{}{NA_SETPb}{PyArrayObject*, type, char*, v}
\end{cfuncdesc}
\begin{cfuncdesc}{}{NA_SETPf}{PyArrayObject*, type, char*, v}
\end{cfuncdesc}
\begin{cfuncdesc}{}{NA_SETP}{PyArrayObject*,  type, char*, v}
\end{cfuncdesc}

\subsubsection{One index single element macros}
\begin{cfuncdesc}{}{NA_GET1a}{PyArrayObject*, type, i}
\end{cfuncdesc}
\begin{cfuncdesc}{}{NA_GET1b}{PyArrayObject*, type, i}
\end{cfuncdesc}
\begin{cfuncdesc}{}{NA_GET1f}{PyArrayObject*, type, i}
\end{cfuncdesc}
\begin{cfuncdesc}{}{NA_GET1}{PyArrayObject*,  type, i}
\end{cfuncdesc}
\begin{cfuncdesc}{}{NA_SET1a}{PyArrayObject*, type, i, v}
\end{cfuncdesc}
\begin{cfuncdesc}{}{NA_SET1b}{PyArrayObject*, type, i, v}
\end{cfuncdesc}
\begin{cfuncdesc}{}{NA_SET1f}{PyArrayObject*, type, i, v}
\end{cfuncdesc}
\begin{cfuncdesc}{}{NA_SET1}{PyArrayObject*,  type, i, v}
\end{cfuncdesc}

\subsubsection{Two index single element macros}
\begin{cfuncdesc}{}{NA_GET2a}{PyArrayObject*, type, i, j}
\end{cfuncdesc}
\begin{cfuncdesc}{}{NA_GET2b}{PyArrayObject*, type, i, j}
\end{cfuncdesc}
\begin{cfuncdesc}{}{NA_GET2f}{PyArrayObject*, type, i, j}
\end{cfuncdesc}
\begin{cfuncdesc}{}{NA_GET2}{PyArrayObject*,  type, i, j}
\end{cfuncdesc}
\begin{cfuncdesc}{}{NA_SET2a}{PyArrayObject*, type, i, j, v}
\end{cfuncdesc}
\begin{cfuncdesc}{}{NA_SET2b}{PyArrayObject*, type, i, j, v}
\end{cfuncdesc}
\begin{cfuncdesc}{}{NA_SET2f}{PyArrayObject*, type, i, j, v}
\end{cfuncdesc}
\begin{cfuncdesc}{}{NA_SET2}{PyArrayObject*,  type, i, j, v}
\end{cfuncdesc}

\subsubsection{One and Two Index, Offset, Float64/Complex64/Int64 functions}

The \class{Int64}/\class{Float64}/\class{Complex64} functions require a
function call to access a single element of an array, making them slower than
the single element macros.  They have two advantages:
\begin{enumerate}
\item They're function calls, so they're a little more robust. 
\item They can handle \emph{any} input array type and behavior properties.
\end{enumerate}


While these functions have no error return status, they *can* alter the Python
error state, so well written extensions should call
\cfunction{PyErr_Occurred()} to determine if an error occurred and report it.
It's reasonable to do this check once at the end of an extension function,
rather than on a per-element basis.


\begin{cfuncdesc}{}{void NA_get_offset}{PyArrayObject *, int N, ...}
  \cfunction{NA_get_offset} computes the offset into an array object given a
  variable number of indices.  It is not especially robust, and it is
  considered an error to pass it more indices than the array has, or indices
  which are negative or out of range.
\end{cfuncdesc}

\begin{cfuncdesc}{Float64}{NA_get_Float64}{PyArrayObject *, long offset}
\end{cfuncdesc}
\begin{cfuncdesc}{void}{NA_set_Float64}{PyArrayObject *, long offset, Float64 v}
\end{cfuncdesc}
\begin{cfuncdesc}{Float64}{NA_get1_Float64}{PyArrayObject *, int i}
\end{cfuncdesc}
\begin{cfuncdesc}{void}{NA_set1_Float64}{PyArrayObject *, int i, Float64 v}
\end{cfuncdesc}
\begin{cfuncdesc}{Float64}{NA_get2_Float64}{PyArrayObject *, int i, int j}
\end{cfuncdesc}
\begin{cfuncdesc}{void}{NA_set2_Float64}{PyArrayObject *, int i, int j, Float64 v}
\end{cfuncdesc}

\begin{cfuncdesc}{Int64}{NA_get_Int64}{PyArrayObject *, long offset}
\end{cfuncdesc}
\begin{cfuncdesc}{void}{NA_set_Int64}{PyArrayObject *, long offset, Int64 v}
\end{cfuncdesc}
\begin{cfuncdesc}{Int64}{NA_get1_Int64}{PyArrayObject *, int i}
\end{cfuncdesc}
\begin{cfuncdesc}{void}{NA_set1_Int64}{PyArrayObject *, int i, Int64 v}
\end{cfuncdesc}
\begin{cfuncdesc}{Int64}{NA_get2_Int64}{PyArrayObject *, int i, int j}
\end{cfuncdesc}
\begin{cfuncdesc}{void}{NA_set2_Int64}{PyArrayObject *, int i, int j, Int64 v}
\end{cfuncdesc}

\begin{cfuncdesc}{Complex64}{NA_get_Complex64}{PyArrayObject *, long offset}
\end{cfuncdesc}
\begin{cfuncdesc}{void}{NA_set_Complex64}{PyArrayObject *, long offset, Complex64 v}
\end{cfuncdesc}
\begin{cfuncdesc}{Complex64}{NA_get1_Complex64}{PyArrayObject *, int i}
\end{cfuncdesc}
\begin{cfuncdesc}{void}{NA_set1_Complex64}{PyArrayObject *, int i, Complex64 v}
\end{cfuncdesc}
\begin{cfuncdesc}{Complex64}{NA_get2_Complex64}{PyArrayObject *, int i, int j}
\end{cfuncdesc}
\begin{cfuncdesc}{void}{NA_set2_Complex64}{PyArrayObject *, int i, int j, Complex64 v}
\end{cfuncdesc}

\subsection{Example}
\label{sec:C-API:element-wise:example}

The \cfunction{convolve1D} wrapper function corresponding to section
\ref{sec:C-API:high-level:example} using the element-wise API could look
like:\footnote{This function is also available as an example in the source
   distribution.}

\begin{verbatim}
static PyObject *
Py_Convolve1d(PyObject *obj, PyObject *args)
{
        PyObject   *okernel, *odata, *oconvolved=Py_None;
        PyArrayObject *kernel, *data, *convolved;

        if (!PyArg_ParseTuple(args, "OO|O", &okernel, &odata, &oconvolved)) {
                PyErr_Format(_Error, "Convolve1d: Invalid parameters.");
                goto _fail;
        }

        kernel = NA_InputArray(okernel, tAny, 0);
        data   = NA_InputArray(odata, tAny, 0);
\end{verbatim}

For the kernel and data arrays, \class{numarray}s of any type are accepted
without conversion.  Thus there is no copy of the data made except for lists or
tuples.  All types, byteswapping, misalignment, and discontiguity must be
handled by Convolve1d.  This can be done easily using the get/set functions.
Macros, while faster than the functions, can only handle a single type.

\begin{verbatim}
        convolved = NA_OptionalOutputArray(oconvolved, tFloat64, 0, data);
\end{verbatim}

Also for the output array we accept any variety of type tFloat without
conversion.  No copy is made except for non-tFloat.  Non-numarray sequences are
not permitted as output arrays.  Byteswaping, misaligment, and discontiguity
must be handled by Convolve1d.  If the \python caller did not specify the
oconvolved array, it initially retains the value Py_None.  In that case,
\var{convolved} is cloned from the array \var{data} using the specified type.
It is important to clone from \var{data} and not \var{odata}, since the latter
may be an ordinary \python sequence which was converted into numarray
\var{data}.  

\begin{verbatim}
        if (!kernel || !data || !convolved)
                goto _fail;

        if ((kernel->nd != 1) || (data->nd != 1)) {
                PyErr_Format(_Error,
                     "Convolve1d: arrays must have exactly 1 dimension.");
                goto _fail;
        }

        if (!NA_ShapeEqual(data, convolved)) {
                PyErr_Format(_Error,
                    "Convolve1d: data and output arrays must have identical length.");
                goto _fail;
        }
        if (!NA_ShapeLessThan(kernel, data)) {
                PyErr_Format(_Error,
                    "Convolve1d: kernel must be smaller than data in both dimensions");
                goto _fail;
        }
        
        if (Convolve1d(kernel, data, convolved) < 0)  /* Error? */
            goto _fail;
        else {
           Py_XDECREF(kernel);
           Py_XDECREF(data);
           return NA_ReturnOutput(oconvolved, convolved);
        }
_fail:
        Py_XDECREF(kernel);
        Py_XDECREF(data);
        Py_XDECREF(convolved);
        return NULL;
}

\end{verbatim}

This function is very similar to the high-level API wrapper, the notable
difference is that we ask for the unconverted arrays \var{kernel} and
\var{data} and \var{convolved}.  This requires some attention in their usage.
The function that does the actual convolution in the example has to use
\cfunction{NA_get*} to read and \cfunction{NA_set*} to set an element of these
arrays, instead of using straight array notation.  These functions perform any
necessary type conversion, byteswapping, and alignment.

\begin{verbatim}
static int
Convolve1d(PyArrayObject *kernel, PyArrayObject *data, PyArrayObject *convolved)
{
        int xc, xk;
        int ksizex = kernel->dimensions[0];
        int halfk = ksizex / 2;
        int dsizex = data->dimensions[0];

        for(xc=0; xc<halfk; xc++)
                NA_set1_Float64(convolved, xc, NA_get1_Float64(data, xc));
                     
        for(xc=dsizex-halfk; xc<dsizex; xc++)
                NA_set1_Float64(convolved, xc, NA_get1_Float64(data, xc));

        for(xc=halfk; xc<dsizex-halfk; xc++) {
                Float64 temp = 0;
                for (xk=0; xk<ksizex; xk++) {
                        int i = xc - halfk + xk;
                        temp += NA_get1_Float64(kernel, xk) * 
                                NA_get1_Float64(data, i);
                }
                NA_set1_Float64(convolved, xc, temp);
        }
        if (PyErr_Occurred())
           return -1;
        else 
           return 0;
}
\end{verbatim}

\section{One-dimensional API}
\label{sec:C-API:One-dimensional-api}

The 1D in-place API is a set of functions for getting/setting elements from the
innermost dimension of an array.  These functions improve speed by moving type
switches, ``behavior tests'', and function calls out of the per-element loop.
The functions get/set a series of consequtive array elements to/from arrays of
\class{Int64}, \class{Float64}, or \class{Complex64}.  These functions are
(even) more intrusive than the single element functions, but have better
performance in many cases.  They can operate on arrays of any type, with the
exception of the Complex64 functions, which only handle Complex64.  The
functions return 0 on success and -1 on failure, with the Python error state
already set.  To be used profitably, the 1D API requires either a large single
dimension which can be processeed in blocks or a multi-dimensional array such
as an image.  In the latter case, the 1D API is suitable for processing one (or
more) scanlines at a time rather than the entire image at once.  See the source
distribution Examples/convolve/one_dimensionalmodule.c for an example of usage.

\begin{cfuncdesc}{long}{NA_get_offset}{PyArrayObject *, int N, ...}
   This function applies a (variable length) set of \var{N} indices to an array
   and returns a byte offset into the array.
\end{cfuncdesc}

\begin{cfuncdesc}{int}{NA_get1D_Int64}{%
      PyArrayObject *, long offset, int cnt, Int64 *out}%
\end{cfuncdesc}

\begin{cfuncdesc}{int}{NA_set1D_Int64}{%
      PyArrayObject *, long offset, int cnt, Int64 *in}%
\end{cfuncdesc}

\begin{cfuncdesc}{int}{NA_get1D_Float64}{%
      PyArrayObject *, long offset, int cnt, Float64 *out}%
\end{cfuncdesc}

\begin{cfuncdesc}{int}{NA_set1D_Float64}{%
      PyArrayObject *, long offset, int cnt, Float64 *in}%
\end{cfuncdesc}

\begin{cfuncdesc}{int}{NA_get1D_Complex64}{%
      PyArrayObject *, long offset, int cnt, Complex64 *out}%
\end{cfuncdesc}

\begin{cfuncdesc}{int}{NA_set1D_Complex64}{%
      PyArrayObject *, long offset, int cnt, Complex64 *in}%
\end{cfuncdesc}

\section{New numarray functions}
\label{sec:C-API:new-numarray-functions}

The following array creation functions share similar behavior.  All but one
create a new \class{numarray} using the data specified by \var{data}.  If
\var{data} is NULL, the routine allocates a buffer internally based on the
array shape and type; internally allocated buffers have undefined contents.
The data type of the created array is specified by \var{type}.

There are several functions to create \class{numarray}s at the C level:

\begin{cfuncdesc}{static PyArrayObject*}{NA_NewArray}{%
    void *data, NumarrayType type, int ndim, ...}% 

  \var{ndim} specifies the rank of the array (number of dimensions), and the
  length of each dimension must be given as the remaining (variable length)
  list of \emph{int} parameters.  The following example allocates a 100x100
  uninitialized array of Int32.
\begin{verbatim}
  if (!(array = NA_NewArray(NULL, tInt32, 2, 100, 100)))
      return NULL;
\end{verbatim}
\end{cfuncdesc}

\begin{cfuncdesc}{static PyObject*}{NA_vNewArray}{%
    void *data, NumarrayType type, int ndim, maybelong *shape}% 

  For \function{NA_vNewArray} the length of each dimension must be given in an
  array of \var{maybelong} pointed to by \var{shape}. The following code
  allocates a 2x2 array initialized to a copy of the specified \var{data}.
  \begin{verbatim}
    Int32 data[4] = { 1, 2, 3, 4 };
    maybelong shape[2] = { 2, 2 };
    if (!(array = NA_vNewArray(data, tInt32, 2, shape)))
       return NULL;
  \end{verbatim}
\end{cfuncdesc}

\begin{cfuncdesc}{static PyArrayObject*}{NA_NewAll}{%
    int ndim, maybelong *shape, NumarrayType type, void *data, maybelong
    byteoffset, maybelong bytestride, int byteorder, int aligned, int
    writable}%

    \function{NA_NewAll} is similar to \function{NA_vNewArray} except it
    provides for the specification of additional parameters. \var{byteoffset}
    specifies the byte offset from the base of the data array at which the
    \var{real} data begins.  \var{bytestride} specifies the miminum stride to
    use, the seperation in bytes between adjacent elements in the
    array. \var{byteorder} takes one of the values \constant{NUM_BIG_ENDIAN} or
    \constant{NUM_LITTLE_ENDIAN}.  \var{writable} defines whether the buffer
    object associated with the resulting array is readonly or writable.
\end{cfuncdesc}

\begin{cfuncdesc}{static PyArrayObject*}{NA_NewAllStrides}{%
    int ndim, maybelong *shape, maybelong *strides, NumarrayType type, void
    *data, maybelong byteoffset, maybelong byteorder, int aligned, int
    writable}% 

    \function{NA_NewAllStrides} is a variant of \function{NA_vNewAll} which
    also permits the specification of the array strides.  The strides are not
    checked for correctness.
\end{cfuncdesc}

\begin{cfuncdesc}{static PyArrayObject*}{NA_NewAllFromBuffer}{%
    int ndim, maybelong *shape, NumarrayType type, PyObject *buffer, maybelong
    byteoffset, maybelong bytestride, int byteorder, int aligned, int
    writable}% 

   \function{NA_NewAllFromBuffer} is similar to \function{NA_NewAll} except it
   accepts a buffer object rather than a pointer to C data.  The \var{buffer}
   object must support the buffer protocol.  If \var{buffer} is non-NULL, the
   returned array object stores a reference to \var{buffer} and locates its
   data there.  If \var{buffer} is specified as NULL, a buffer object and
   associated data space are allocated internally and the returned array object
   refers to that.  It is possible to create a Python buffer object from an
   array of C data and then construct a \class{numarray} using this function
   which refers to the C data without making a copy.
\end{cfuncdesc}

\begin{cfuncdesc}{int}{NA_ShapeEqual}{PyArrayObject*a,PyArrayObject*b}
This function compares the shapes of two arrays, and returns 1 if they
are the same, 0 otherwise.
\end{cfuncdesc}

\begin{cfuncdesc}{int}{NA_ShapeLessThan}{PyArrayObject*a,PyArrayObject*b}
This function compares the shapes of two arrays, and returns 1 if each
dimension of 'a' is less than the corresponding dimension of 'b', 0 otherwise.
\end{cfuncdesc}

\begin{cfuncdesc}{int}{NA_ByteOrder}{}
This function returns the system byte order, either NUM_LITTLE_ENDIAN or
NUM_BIG_ENDIAN.
\end{cfuncdesc}

\begin{cfuncdesc}{Bool}{NA_IeeeMask32}{Float32 value, Int32 mask}
This function returns 1 IFF Float32 \var{value} matches any of the IEEE special
value criteria specified by \var{mask}.  See ieeespecial.h for the mask bit
values which can be or'ed together to specify mask.
\function{NA_IeeeSpecial32} has been deprecated and will eventually be removed.
\end{cfuncdesc}

\begin{cfuncdesc}{Bool}{NA_IeeeMask64}{Float64 value,Int32 mask}
This function returns 1 IFF Float64 \var{value} matches any of the IEEE special
value criteria specified by \var{mask}.  See ieeespecial.h for the mask bit
values which can be or'ed together to specify mask.
\function{NA_IeeeSpecial64} has been deprecated and will eventually be removed.
\end{cfuncdesc}

\begin{cfuncdesc}{PyArrayObject *}{NA_updateDataPtr}{PyArrayObject *}
This function updates the values derived from the ``_data'' buffer, namely the
data pointer and buffer WRITABLE flag bit.  This needs to be called upon
entering or re-entering C-code from Python, since it is possible for buffer
objects to move their data buffers as a result of executing arbitrary Python
and hence arbitrary C-code.  The high level interface routines,
e.g. \function{NA_InputArray}, call this routine automatically.
\end{cfuncdesc}

\begin{cfuncdesc}{char*}{NA_typeNoToName}{int}
NA_typeNoToName translates a NumarrayType into a character string which can be
used to display it:  e.g.  tInt32 converts to the string ``Int32''
\end{cfuncdesc}

\begin{cfuncdesc}{PyObject*}{NA_typeNoToTypeObject}{int}
This function converts a NumarrayType C type code into the NumericType object
which implements and represents it more fully.  tInt32 converts to the type
object numarray.Int32.  
\end{cfuncdesc}

\begin{cfuncdesc}{int}{NA_typeObjectToTypeNo}{PyObject*}
This function converts a numarray type object (e.g. numarray.Int32) into the
corresponding NumarrayType (e.g. tInt32) C type code. 
\end{cfuncdesc}

\begin{cfuncdesc} {PyObject*}{NA_intTupleFromMaybeLongs}{int,maybelong*}
This function creates a tuple of Python ints from an array of C maybelong integers.
\end{cfuncdesc}

\begin{cfuncdesc}{long}{NA_maybeLongsFromIntTuple}{int,maybelong*,PyObject*}
This function fills an array of C long integers with the converted values from
a tuple of Python ints.  It returns the number of conversions, or -1 for error.
\end{cfuncdesc}

\begin{cfuncdesc}{long}{NA_isIntegerSequence}{PyObject*}
This function returns 1 iff the single parameter is a sequence of Python
integers, and 0 otherwise.
\end{cfuncdesc}

\begin{cfuncdesc}{PyObject*}{NA_setArrayFromSequence}{PyArrayObject*,PyObject*}
This function copies the elementwise from a sequence object to a numarray.
\end{cfuncdesc}

\begin{cfuncdesc}{int}{NA_maxType}{PyObject*}
This function returns an integer code corresponding to the highest kind of
Python numeric object in a sequence.  INT(0) LONG(1) FLOAT(2) COMPLEX(3).
On error -1 is returned.
\end{cfuncdesc}

\begin{cfuncdesc}{PyObject*}{NA_getPythonScalar}{PyArrayObject *a, long offset}
This function returns the Python object corresponding to the single element of 
the array 'a' at the given byte offset.
\end{cfuncdesc}

\begin{cfuncdesc}{int}{NA_setFromPythonScalar}{PyArrayObject *a, long offset, PyObject*value}
This function sets the single element of the array 'a' at the given byte
offset to 'value'.
\end{cfuncdesc}

\begin{cfuncdesc}{int}{NA_NDArrayCheck}{PyObject*o}
This function returns 1 iff the 'o' is an instance of NDArray or an instance of
a subclass of NDArray, and 0 otherwise.
\end{cfuncdesc}

\begin{cfuncdesc}{int}{NA_NumArrayCheck}{PyObject*}
This function returns 1 iff the 'o' is an instance of NumArray or an instance of
a subclass of NumArray, and 0 otherwise.
\end{cfuncdesc}

\begin{cfuncdesc}{int}{NA_ComplexArrayCheck}{PyObject*}
This function returns 1 iff the 'o' is an instance of ComplexArray or an instance of
a subclass of ComplexArray, and 0 otherwise.
\end{cfuncdesc}

\begin{cfuncdesc}{unsigned long}{NA_elements}{PyArrayObject*}
This function returns the total count of elements in an array,  essentially the
product of the elements of the array's shape.
\end{cfuncdesc}

\begin{cfuncdesc}{PyArrayObject *}{NA_copy}{PyArrayObject*}
This function returns a copy of the given array.  The array copy is guaranteed
to be well-behaved, i.e. neither byteswapped, misaligned, nor discontiguous.
\end{cfuncdesc}

\begin{cfuncdesc}{int}{NA_copyArray}{PyArrayObject*to, const PyArrayObject *from}
This function returns a copies one array onto another;  used in f2py.
\end{cfuncdesc}

\begin{cfuncdesc}{int}{NA_swapAxes}{PyArrayObject*a, int dim1, int dim2}
This function mutates the specified array \var{a} to exchange the shape and
strides values for the two dimensions, \var{dim1} and \var{dim2}.  Negative
dimensions count backwards from the innermost, with -1 being the innermost
dimension.  Returns 0 on success and -1 on error.
\end{cfuncdesc}

%% Local Variables:
%% mode: LaTeX
%% mode: auto-fill
%% fill-column: 79
%% indent-tabs-mode: nil
%% ispell-dictionary: "american"
%% reftex-fref-is-default: nil
%% TeX-auto-save: t
%% TeX-command-default: "pdfeLaTeX"
%% TeX-master: "numarray"
%% TeX-parse-self: t
%% End:




\part{Extension modules}

\label{part:numarray-extensions}

\chapter{Convolution}
\label{cha:convolve}

%begin{latexonly}
\makeatletter
\py@reset
\makeatother
%end{latexonly}
\declaremodule{extension}{numarray.convolve}
\moduleauthor{The numarray team}{numpy-discussion@lists.sourceforge.net}
\modulesynopsis{Convolution,correlation}

\begin{quote}
   This package (numarray.convolve) provides functions for one- and 
   two-dimensional convolutions and correlations of \class{numarray}s.
   Each of the following examples assumes that the following code has been 
   executed:
\begin{verbatim}
import numarray.convolve as conv
\end{verbatim}
\end{quote}


\section{Convolution functions}
\label{sec:CONV:convolution-functions}

\begin{funcdesc}{boxcar}{data, boxshape, output=None, mode='nearest', cval=0.0}
   \function{boxcar} computes a 1-D or 2-D boxcar filter on every 1-D or
   2-D subarray of \constant{data}. \constant{boxshape} is a tuple of integers
   specifying the dimensions of the filter, e.g. \code{(3,3)}.  If
   \constant{output} is specified, it should be the same shape as
   \constant{data} and the result will be stored in it.  In that case
   \class{None} will be returned.
        
   \constant{mode} can be any of the following values:
   \begin{description}
   \item[\var{nearest}]: Elements beyond boundary come from nearest edge pixel.
   \item[\var{wrap}]: Elements beyond boundary come from the opposite array
      edge.
   \item[\var{reflect}]: Elements beyond boundary come from reflection on same
      array edge.
   \item[\var{constant}]: Elements beyond boundary are set to what is specified
      in \constant{cval}, an optional numerical parameter; the default value is
      \code{0.0}.
   \end{description}        
\end{funcdesc}
\begin{verbatim}
>>> print a
[1 5 4 7 2 9 3 6]
>>> print conv.boxcar(a,(3,))
[ 2.33333333  3.33333333  5.33333333  4.33333333  6.          4.66666667
  6.          5.        ]
# for even number box size, it will take the extra point from the lower end
>>> print conv.boxcar(a,(2,))
[ 1.   3.   4.5  5.5  4.5  5.5  6.   4.5]
\end{verbatim}

\begin{funcdesc}{convolve}{data, kernel, mode=FULL}
   \label{func:CONV:convolve}
   Returns the discrete, linear convolution of 1-D sequences \constant{data} 
   and \constant{kernel}; \constant{mode} can be \class{VALID}, \class{SAME}, 
   or \code{FULL} to specify the size of the resulting sequence.  See section
   \ref{sec:CONV:global-constants}.
\end{funcdesc}

\begin{funcdesc}{convolve2d}{data, kernel, output=None, fft=0, mode='nearest', 
    cval=0.0} Return the 2-dimensional convolution of \constant{data} and
  \constant{kernel}.  If \constant{output} is not \class{None}, the result is
  stored in \constant{output} and \class{None} is returned.  \constant{fft} is
  used to switch between FFT-based convolution and the naive algorithm,
  defaulting to naive.  Using \constant{fft} mode becomes more beneficial as
  the size of the kernel grows; for small kernels, the naive algorithm is more
  efficient.  \constant{mode} has the same choices as those of
  \function{boxcar}.  A number of storage considerations come into play with
  large arrays: (1) boundary modes are implemented by making an oversized
  temporary copy of the \constant{data} array which has a shape equal to the
  sum of the \constant{data} and \constant{kernel} shapes.  (2) likewise, the
  \constant{kernel} is copied into an array with the same shape as the
  oversized \constant{data} array.  (3) In FFT mode, the fourier transforms of
  the \constant{data} and \constant{kernel} arrays are stored in double
  precision complex temporaries. The aggregate effect is that storage roughly
  equal to a factor of eight (x2 from 2 and x4 from 3) times the size of the
  \constant{data} is required to compute the convolution of a Float32
  \constant{data} array.
\end{funcdesc}

\begin{funcdesc}{correlate}{data, kernel, mode=FULL}
   Return the cross-correlation of \constant{data} and \constant{kernel};
   \constant{mode} can be \class{VALID}, \class{SAME}, or \code{FULL} to 
   specify the size of the resulting sequence.  \function{correlate} is
   very closely related to \function{convolve} in implementation.
   See section \ref{sec:CONV:global-constants}.
\end{funcdesc}

\begin{funcdesc}{correlate2d}{data, kernel, output=None, fft=0, mode='nearest', cval=0.0}
   \label{func:CONV:correlate2d}
  Return the 2-dimensional convolution of \constant{data} and
  \constant{kernel}.  If \constant{output} is not \class{None}, the result is
  stored in \constant{output} and \class{None} is returned.  \constant{fft} is
  used to switch between FFT-based convolution and the naive algorithm,
  defaulting to naive.  Using \constant{fft} mode becomes more beneficial as
  the size of the kernel grows; for small kernels, the naive algorithm is more
  efficient.  \constant{mode} has the same choices as those of
  \function{boxcar}.  See also \function{convolve2d} for notes regarding 
  storage consumption.
\end{funcdesc}

\note{\function{cross_correlate} is deprecated and should not be used.}



\section{Global constants}
\label{sec:CONV:global-constants}

These constants specify what part of the result the \function{convolve} and
\function{correlate} functions of this module return.  Each of the following
examples assumes that the following code has been executed:

\begin{verbatim}
arr = numarray.arange(8)
\end{verbatim}

\begin{datadesc}{FULL}
   Return the full convolution or correlation of two arrays.
\begin{verbatim}
>>> conv.correlate(arr, [1, 2, 3], mode=conv.FULL)
array([ 0,  3,  8, 14, 20, 26, 32, 38, 20,  7])
\end{verbatim}
\end{datadesc}

\begin{datadesc}{PASS}
   Correlate the arrays without padding the data.
\begin{verbatim}
>>> conv.correlate(arr, [1, 2, 3], mode=conv.PASS)
array([ 0,  8, 14, 20, 26, 32, 38,  7])
\end{verbatim}
\end{datadesc}

\begin{datadesc}{SAME}
   Return the part of the convolution or correlation of two arrays that
   corresponds to an array of the same shape as the input data.
\begin{verbatim}
>>> conv.correlate(arr, [1, 2, 3], mode=conv.SAME)
array([ 3,  8, 14, 20, 26, 32, 38, 20])
\end{verbatim}
\end{datadesc}

\begin{datadesc}{VALID}
   Return the valid part of the convolution or correlation of two arrays.
\begin{verbatim}
>>> conv.correlate(arr, [1, 2, 3], mode=conv.VALID)
array([ 8, 14, 20, 26, 32, 38])
\end{verbatim}
\end{datadesc}



%% Local Variables:
%% mode: LaTeX
%% mode: auto-fill
%% fill-column: 79
%% indent-tabs-mode: nil
%% ispell-dictionary: "american"
%% reftex-fref-is-default: nil
%% TeX-auto-save: t
%% TeX-command-default: "pdfeLaTeX"
%% TeX-master: "numarray"
%% TeX-parse-self: t
%% End:

\chapter{Fast-Fourier-Transform}
\label{cha:fft}

%begin{latexonly}
\makeatletter \py@reset \makeatother
%end{latexonly}
\declaremodule{extension}{numarray.fft}
\moduleauthor{The numarray team}{numpy-discussion@lists.sourceforge.net}
\modulesynopsis{Fast Fourier Transform}

\begin{quote}
   This package provides functions for one- and two-dimensional
   Fast-Fourier-Transforms (FFT) and inverse FFTs.
\end{quote}


The \module{numarray.fft} module provides a simple interface to the FFTPACK
Fortran library, which is a powerful standard library for doing fast Fourier
transforms of real and complex data sets, or the C fftpack library, which is
algorithmically based on FFTPACK and provides a compatible interface.


\section{Installation}
\label{sec:FFT:installation}

The default installation uses the provided \module{numarray.fft.fftpack} C
implementation of these routines and this works without any further
interaction.


\subsection{Installation using FFTPACK}
\label{sec:FFT:install-lapack}

On some platforms, precompiled optimized versions of the FFTPACK libraries are
preinstalled on the operating system, and the setup procedure needs to be
modified to force the \module{numarray.fft} module to be linked against those
rather than the builtin replacement functions.




\section{FFT Python Interface}
\label{sec:FFT:python-interface}

The Python user imports the \module{numarray.fft} module, which provides 
a set of utility functions of the most commonly used FFT routines, and
allows the specification of which axes (dimensions) of the input arrays are to
be used for the FFT's. These routines are:

\begin{funcdesc}{fft}{a, n=None, axis=-1} 
   Performs a \constant{n}-point discrete Fourier transform of the array 
   \constant{a}, \constant{n} defaults to the size of \constant{a}. It is 
   most efficient for \constant{n} a power of two. If \constant{n} is 
   larger than \code{len(a)}, then \constant{a} will be
   zero-padded to make up the difference. If \constant{n} is smaller than
   \code{len(a)}, then \constant{a} will be aliased to reduce its size. This
   also stores a cache of working memory for different sizes of 
   \module{fft}'s, so you could theoretically run into memory problems if 
   you call this too many times with too many different \constant{n}'s.
   
   The FFT is performed along the axis indicated by the \constant{axis} 
   argument, which defaults to be the last dimension of \constant{a}.
   
   The format of the returned array is a complex array of the same shape as
   \constant{a}, where the first element in the result array contains the DC
   (steady-state) value of the FFT.
   \remark{missing: ..., and where each successive ...}

   Some examples are:
\begin{verbatim}
>>> a = array([1., 0., 1., 0., 1., 0., 1., 0.]) + 10
>>> b = array([0., 1., 0., 1., 0., 1., 0., 1.]) + 10
>>> c = array([0., 1., 0., 0., 0., 1., 0., 0.]) + 10
>>> print numarray.fft.fft(a).real
[ 84.   0.   0.   0.   4.   0.   0.   0.]
>>> print numarray.fft.fft(b).real
[ 84.   0.   0.   0.  -4.   0.   0.   0.]
>>> print numarray.fft.fft(c).real
[ 82.   0.   0.   0.  -2.   0.   0.   0.]
\end{verbatim}
\end{funcdesc}
       
\begin{funcdesc}{inverse_fft}{a, n=None, axis=-1}
   Will return the \constant{n} point inverse discrete Fourier transform of
   \constant{a}; \constant{n} defaults to the length of \constant{a}. 
   It is most efficient for \constant{n} a power of two.  If \constant{n} 
   is larger than \constant{a}, then \constant{a} will be zero-padded to 
   make up the difference.  If \constant{n} is smaller than \constant{a}, 
   then \constant{a} will be aliased to reduce its size.
   This also stores a cache of working memory for different sizes of FFT's, so
   you could theoretically run into memory problems if you call this too many
   times with too many different \constant{n}'s.
\end{funcdesc}
       
\begin{funcdesc}{real_fft}{a, n=None, axis=-1}
   Will return the \constant{n} point discrete Fourier transform of the 
   real valued array \constant{a}; \constant{n} defaults to the length of 
   \constant{a}.  It is most efficient for \constant{n} a power of two.  
   The returned array will be one half of the symmetric complex transform of 
   the real array.

\begin{verbatim}
>>> x = cos(arange(30.0)/30.0*2*pi)
>>> print numarray.fft.real_fft(x)
[ -5.82867088e-16 +0.00000000e+00j   1.50000000e+01 -3.08862614e-15j
   7.13643755e-16 -1.04457106e-15j   1.13047653e-15 -3.23843935e-15j
  -1.52158521e-15 +1.14787259e-15j   3.60822483e-16 +3.60555504e-16j
   1.34237661e-15 +2.05127011e-15j   1.98981960e-16 -1.02472357e-15j
   1.55899290e-15 -9.94619821e-16j  -1.05417678e-15 -2.33364171e-17j
  -2.08166817e-16 +1.00955541e-15j  -1.34094426e-15 +8.88633386e-16j
   5.67513742e-16 -2.24823896e-15j   2.13735778e-15 -5.68448962e-16j
  -9.55398954e-16 +7.76890265e-16j  -1.05471187e-15 +0.00000000e+00j]
\end{verbatim}
\end{funcdesc}
       
\begin{funcdesc}{inverse_real_fft}{a, n=None, axis=-1}
   Will return the inverse FFT of the real valued array \constant{a}.
\end{funcdesc}
       
\begin{funcdesc}{fft2d}{a, s=None, axes=(-2,-1)}
   Will return the 2-dimensional FFT of the array \constant{a}.  This
   is really just \function{fft_nd()} with different default behavior.
\end{funcdesc}
       
\begin{funcdesc}{inverse_fft2d}{a, s=None, axes=(-2,-1)}
  The inverse of \function{fft2d()}. This is really just
  \function{inverse_fftnd()} with different default behavior.
\end{funcdesc}
       
\begin{funcdesc}{real_fft2d}{a, s=None, axes=(-2,-1)}
   Will return the 2-D FFT of the real valued array \constant{a}.
\end{funcdesc}
            
\begin{funcdesc}{inverse_real_fft2d}{a, s=None, axes=(-2,-1)}
  The inverse of \function{real_fft2d()}. This is really just
  \function{inverse_real_fftnd()} with different default behavior.
\end{funcdesc}

            
\section{fftpack Python Interface}
\label{sec:FFT:c-api}

%begin{latexonly}
\makeatletter \py@reset \makeatother
%end{latexonly}
\declaremodule{extension}{numarray.fft.fftpack}
\moduleauthor{The numarray team}{numpy-discussion@lists.sourceforge.net}
\modulesynopsis{Fast Fourier Transform}

The interface to the FFTPACK library is performed via the \module{fftpack}
module, which is responsible for making sure that the arrays sent to the
FFTPACK routines are in the right format (contiguous memory locations, right
numerical storage format, etc). It provides interfaces to the following FFTPACK
routines, which are also the names of the Python functions:
\begin{funcdesc}{cffti}{i}
\end{funcdesc}
\begin{funcdesc}{cfftf}{data, savearea}
\end{funcdesc}
\begin{funcdesc}{cfftb}{data, savearea}
\end{funcdesc}
\begin{funcdesc}{rffti}{i}
\end{funcdesc}
\begin{funcdesc}{rfftf}{data, savearea}
\end{funcdesc}
\begin{funcdesc}{rfftb}{data, savearea}
\end{funcdesc}
The routines which start with \texttt{c} expect arrays of complex numbers, the
routines which start with \texttt{r} expect real numbers only. The routines
which end with \texttt{i} are the initalization functions, those which end with
\texttt{f} perform the forward FFTs and those which end with \texttt{b} perform
the backwards FFTs.

The initialization functions require a single integer argument corresponding to
the size of the dataset, and returns a work array. The forward and backwards
FFTs require two array arguments -- the first is the data array, the second is
the work array returned by the initialization function. They return arrays
corresponding to the coefficients of the FFT, with the first element in the
returned array corresponding to the DC component, the second one to the first
fundamental, etc.The length of the returned array is 1 + half the length of the
input array in the case of real FFTs, and the same size as the input array in
the case of complex data.
\begin{verbatim}
>>> import numarray.fft.fftpack as fftpack
>>> x = cos(arange(30.0)/30.0*2*pi)
>>> w = fftpack.rffti(30)
>>> f = fftpack.rfftf(x, w)
>>> print f[0:5]
[ -5.68989300e-16 +0.00000000e+00j   1.50000000e+01 -3.08862614e-15j
        6.86516117e-16 -1.00588467e-15j   1.12688689e-15 -3.19983494e-15j
       -1.52158521e-15 +1.14787259e-15j]
\end{verbatim}



%% Local Variables:
%% mode: LaTeX
%% mode: auto-fill
%% fill-column: 79
%% indent-tabs-mode: nil
%% ispell-dictionary: "american"
%% reftex-fref-is-default: nil
%% TeX-auto-save: t
%% TeX-command-default: "pdfeLaTeX"
%% TeX-master: "numarray"
%% TeX-parse-self: t
%% End:

\chapter{Linear Algebra}
\label{cha:linear-algebra}

%begin{latexonly}
\makeatletter
\py@reset
\makeatother
%end{latexonly}
\declaremodule[numarray.linearalgebra]{extension}{numarray.linear_algebra}
\moduleauthor{The numarray team}{numpy-discussion@lists.sourceforge.net}
\modulesynopsis{Linear Algebra}

\begin{quote}
  The numarray.linear\_algebra module provides a simple interface to some
  commonly used linear algebra routines.
\end{quote}

The \module{numarray.linear_algebra} module provides a simple high-level
interface to some common linear algebra problems. It uses either the LAPACK
Fortran library or the compatible
\module{\mbox{numarray.linear_algebra.lapack_lite}} C library shipped with
\module{numarray}.

\section{Installation}
\label{sec:LA:installation}

The default installation uses the provided
\module{numarray.linear_algebra.lapack_lite} implementation of these routines
and this works without any further interaction.

Nevertheless if LAPACK is installed already or you are concerned about the
performance of these routines you should consider installing
\module{numarray.linear_algebra} to take advantage of the real LAPACK library.
See the next section for instructions.

\subsection{Installation using LAPACK}
\label{sec:LA:install-lapack}

On some platforms, precompiled optimized versions of the LAPACK and BLAS
libraries are preinstalled on the operating system, and the setup procedure
needs to be modified to force the \module{lapack_lite} module to be linked
against those rather than the builtin replacement functions.

Here's a recipe for building using LAPACK:

\begin{verbatim}
% setenv USE_LAPACK 1
% setenv LINALG_LIB <where your lapack, blas, atlas, etc are>
% setenv LINALG_INCLUDE <where your lapack, blas, atlas headers are>
% python setup.py install --selftest
\end{verbatim}

For your particular system and library installations, you may need to edit
\texttt{addons.py} and adjust the variables \texttt{sourcelist},
\texttt{lapack_dirs}, and \texttt{lapack_libs}.

\note{A frequent request is that somehow the maintainers of Numerical Python
   invent a procedure which will automatically find and use the \emph{best}
   available versions of these libraries.  We welcome any patches that provide
   the functionality in a simple, platform independent, and reliable way.  The
   \ulink{scipy}{http://www.scipy.org} project has done some work to provide
   such functionality, but is probably not mature enough for use by
   \module{numarray} yet.}


\section{Python Interface}
\label{sec:LA:python-interface}

All examples in this section assume that you performed a
\begin{verbatim}
from numarray import *
import numarray.linear_algebra as la
\end{verbatim}

\begin{funcdesc}{cholesky_decomposition}{a}
   This function returns a lower triangular matrix L which, when multiplied by
   its transpose yields the original matrix \code{a}; \code{a} must be 
   square, Hermitian, and positive definite. L is often referred to as the 
   Cholesky lower-triangular square-root of \code{a}.
\end{funcdesc}
 
\begin{funcdesc}{determinant}{a}
   This function returns the determinant of the square matrix \code{a}.
\begin{verbatim}
>>> print a
[[ 1  2]
 [ 3 15]]
>>> print la.determinant(a)
9.0
\end{verbatim}
\end{funcdesc}
 
\begin{funcdesc}{eigenvalues}{a}
   This function returns the eigenvalues of the square matrix \code{a}.
\begin{verbatim}
>>> print a
[[ 1.    0.    0.    0.  ]
 [ 0.    2.    0.    0.01]
 [ 0.    0.    5.    0.  ]
 [ 0.    0.01  0.    2.5 ]]
>>> print la.eigenvalues(a)
[ 2.50019992  1.99980008  1.          5.        ]
\end{verbatim}
\end{funcdesc}
 
\begin{funcdesc}{eigenvectors}{a}
   This function returns both the eigenvalues and the eigenvectors, the latter
   as a two-dimensional array (i.e. a sequence of vectors).
\begin{verbatim}
>>> print a
[[ 1.    0.    0.    0.  ]
 [ 0.    2.    0.    0.01]
 [ 0.    0.    5.    0.  ]
 [ 0.    0.01  0.    2.5 ]]
>>> eval, evec = la.eigenvectors(a)
>>> print eval  # same as eigenvalues()
[ 2.50019992  1.99980008  1.          5.        ]
>>> print transpose(evec)
[[ 0.          0.          1.          0.        ]
 [ 0.01998801  0.99980022  0.          0.        ]
 [ 0.          0.          0.          1.        ]
 [ 0.99980022 -0.01998801  0.          0.        ]]
\end{verbatim}
\end{funcdesc}
 
\begin{funcdesc}{generalized_inverse}{a, rcond=1e-10}
   This function returns the generalized inverse (also known as pseudo-inverse
   or Moore-Penrose-inverse) of the matrix \code{a}. It has numerous 
   applications related to linear equations and least-squares problems.
\begin{verbatim}
>>> ainv = la.generalized_inverse(a)
>>> print array_str(innerproduct(a,ainv),suppress_small=1,precision=8)
[[ 1.  0.  0.  0.]
 [ 0.  1.  0. -0.]
 [ 0.  0.  1.  0.]
 [ 0. -0.  0.  1.]]
\end{verbatim}
\end{funcdesc}
 
\begin{funcdesc}{Heigenvalues}{a}
   returns the (real positive) eigenvalues of the square, Hermitian positive
   definite matrix a.
\end{funcdesc}
 
\begin{funcdesc}{Heigenvectors}{a}
   returns both the (real positive) eigenvalues and the eigenvectors of a
   square, Hermitian positive definite matrix a. The eigenvectors are returned
   in an (orthornormal) two-dimensional matrix.
\end{funcdesc}

\begin{funcdesc}{inverse}{a}
   This function returns the inverse of the specified matrix a which must be
   square and non-singular. To within floating point precision, it should
   always be true that \code{matrixmultiply(a, inverse(a)) ==
      identity(len(a))}.  To test this claim, one can do e.g.:
\begin{verbatim}
>>> a = reshape(arange(25.0), (5,5)) + identity(5)
>>> print a
[[  1.   1.   2.   3.   4.]
 [  5.   7.   7.   8.   9.]
 [ 10.  11.  13.  13.  14.]
 [ 15.  16.  17.  19.  19.]
 [ 20.  21.  22.  23.  25.]]
>>> inv_a = la.inverse(a)
>>> print inv_a
[[ 0.20634921 -0.52380952 -0.25396825  0.01587302  0.28571429]
 [-0.5026455   0.63492063 -0.22751323 -0.08994709  0.04761905]
 [-0.21164021 -0.20634921  0.7989418  -0.1957672  -0.19047619]
 [ 0.07936508 -0.04761905 -0.17460317  0.6984127  -0.42857143]
 [ 0.37037037  0.11111111 -0.14814815 -0.40740741  0.33333333]]
\end{verbatim}
   Verify the inverse by printing the largest absolute element of
   $a\, a^{-1} - identity(5)$:
\begin{verbatim}
>>> print "Inversion error:", maximum.reduce(fabs(ravel(dot(a, inv_a)-identity(5))))
Inversion error: 8.18789480661e-16
\end{verbatim}
\end{funcdesc}
 
\begin{funcdesc}{linear_least_squares}{a, b, rcond=1e-10}
   This function returns the least-squares solution of an overdetermined system
   of linear equations. An optional third argument indicates the cutoff for the
   range of singular values (defaults to $10^{-10}$). There are four return
   values: the least-squares solution itself, the sum of the squared residuals
   (i.e.  the quantity minimized by the solution), the rank of the matrix a,
   and the singular values of a in descending order.
\begin{verbatim}

\end{verbatim}
\end{funcdesc}
 
\begin{funcdesc}{solve_linear_equations}{a, b}
   This function solves a system of linear equations with a square non-singular
   matrix a and a right-hand-side vector b. Several right-hand-side vectors can
   be treated simultaneously by making b a two-dimensional array (i.e. a
   sequence of vectors). The function inverse(a) calculates the inverse of the
   square non-singular matrix a by calling solve_linear_equations(a, b) with a
   suitable b.
\end{funcdesc}
 
\begin{funcdesc}{singular_value_decomposition}{a, full_matrices=0}
   This function returns three arrays V, S, and WT whose matrix product is the
   original matrix a. V and WT are unitary matrices (rank-2 arrays), whereas S
   is the vector (rank-1 array) of diagonal elements of the singular-value
   matrix. This function is mainly used to check whether (and in what way) a
   matrix is ill-conditioned.
\end{funcdesc}
 



%% Local Variables:
%% mode: LaTeX
%% mode: auto-fill
%% fill-column: 79
%% indent-tabs-mode: nil
%% ispell-dictionary: "american"
%% reftex-fref-is-default: nil
%% TeX-auto-save: t
%% TeX-command-default: "pdfeLaTeX"
%% TeX-master: "numarray"
%% TeX-parse-self: t
%% End:

\chapter{Masked Arrays}
\label{cha:masked-arrays}

%begin{latexonly}
\makeatletter
\py@reset
\makeatother
%end{latexonly}
\declaremodule{extension}{numarray.ma}
\moduleauthor{The numarray team}{numpy-discussion@lists.sourceforge.net}
\modulesynopsis{Masked Arrays}
\index{MaskedArray|see{numarray.ma}}
\index{observations, dealing with missing}

\begin{quote}
   Masked arrays are arrays that may have missing or invalid entries. Module
   \module{numarray.ma} provides a nearly work-alike replacement for numarray
   that supports data arrays with masks.
\end{quote}

\section{What is a masked array?}
\label{sec:numarray.ma:what-is-a-masked-array}

Masked arrays are arrays that may have missing or invalid entries. Module
\module{numarray.ma} provides a work-alike replacement for \module{\numarray}
that supports data arrays with masks. A mask is either None or an array of ones
and zeros, that determines for each element of the masked array whether or not
it contains an invalid entry.  The package assures that invalid entries are not
used in computations.  A particular element is said to be masked
(\index{numarray.ma!invalid}invalid) if the mask is not None and the
corresponding element of the mask is 1; otherwise it is unmasked
(\index{numarray.ma!valid}valid).

This package was written by \index{Dubois, Paul F.}Paul F.\ Dubois at Lawrence
Livermore National Laboratory. Please see the legal notice in the software and
section \ref{sec:legal-notice} ``License and disclaimer for packages
numarray.ma''. 

\section{Using numarray.ma}
\label{sec:numarray.ma:using}

Use numarray.ma as a replacement for numarray:
\begin{verbatim}
from numarray.ma import *
>>> x = array([1, 2, 3])
\end{verbatim}
To create an array with the second element invalid, we would do:
\begin{verbatim}
>>> y = array([1, 2, 3], mask = [0, 1, 0])
\end{verbatim}
To create a masked array where all values ``near'' 1.e20 are invalid, we can
do:
\begin{verbatim}
>>> z = masked_values([1.0, 1.e20, 3.0, 4.0], 1.e20)
\end{verbatim}
For a complete discussion of creation methods for masked arrays please see
section \ref{sec:numarray.ma:constructing-mask-arrays} ``Constructing masked
arrays''.

The \module{\numarray} module is an attribute in \module{numarray.ma}, so to
execute a method \method{foo} from numarray, you can reference it as
\method{numarray.foo}.

Usually people use both numarray.ma and numarray this way, but of course you can
always fully-qualify the names:
\begin{verbatim}
>>> import numarray.ma
>>> x = numarray.ma.array([1, 2, 3])
\end{verbatim}

The principal feature of module \module{numarray.ma} is class
\class{MaskedArray}, the class whose instances are returned by the array
constructors and most functions in module \module{numarray.ma}. We will discuss
this class first, and later cover the attributes and functions in module
\module{numarray.ma}. For now suffice it to say that among the attributes of
the module are the constants from module \module{\numarray} including those for
declaring typecodes, \constant{NewAxis}, and the mathematical constants such as
\constant{pi} and \constant{e}.  An additional typecode, \class{MaskType}, is
the typecode used for masks.


\section{Class MaskedArray}
\label{sec:numarray.ma:class-maskedarray}
\index{numarray.ma!MaskedArray@\class{MaskedArray}}

In Module \module{numarray.ma}, an array is an instance of class
\class{MaskedArray}, which is defined in the module \module{numarray.ma}. An
instance of class \class{MaskedArray} can be thought of as containing the
following parts:
\begin{itemize}
\item An array of data, of any shape;
\item A mask of ones and zeros of the same shape as the data where a one value
  (true) indicates that the element is masked and the corresponding data is
  invalid.
\item A ``fill value'' --- this is a value that may be used to replace the
   invalid entries in order to return a plain \module{\numarray} array. The
   chief method that does this is the method \method{filled} discussed below.
\end{itemize}
We will use the terms ``invalid value'' and ``invalid entry'' to refer to the
data value at a place corresponding to a mask value of 1. It should be
emphasized that the invalid values are \emph{never} used in any computation,
and that the fill value is not used for \emph{any} computational purpose. When
an instance \var{x} of class \class{MaskedArray} is converted to its string
representation, it is the result returned by \code{filled(x)} that is converted
to a string.


\subsection{Attributes of masked arrays}
\label{sec:numarray.ma:attr-mask-arrays}

\begin{memberdesc}[MaskedArray]{flat}
   (deprecated) \remark{why deprecated in numarray?}
   Returns the masked array as a one-dimensional one. This is
   provided for compatibility with \module{\numarray}. \method{ravel} is
   preferred.  \member{flat} can be assigned to: \samp{x.flat = value} will
   change the values of \var{x}.
\end{memberdesc}

\begin{memberdesc}[MaskedArray]{real}
   Returns the real part of the array if complex. It can be assigned to:
   \samp{x.real = value} will change the real parts of \var{x}.
\end{memberdesc}

\begin{memberdesc}[MaskedArray]{imaginary}
   Returns the imaginary part of the array if complex. It can be assigned to:
   \samp{x.imaginary = value} will change the imaginary parts of x.
\end{memberdesc}

\begin{memberdesc}[MaskedArray]{shape}
   The shape of a masked array can be accessed or changed by using the special
   attribute \member{shape}, as with \module{\numarray} arrays. It can be
   assigned to: \samp{x.shape = newshape} will change the shape of \var{x}. The
   new shape has to describe the same total number of elements.
   \remark{Correct?}
\end{memberdesc}

\begin{memberdesc}[MaskedArray]{shared_data}
   This read-only flag if true indicates that the masked array shared a
   reference with the original data used to construct it at the time of
   construction. Changes to the original array will affect the masked array.
   (This is not the default behavior; see ``Copying or not''.) This flag is
   informational only.
\end{memberdesc}

\begin{memberdesc}[MaskedArray]{shared_mask}
   This read-only flag if true indicates that the masked array \emph{currently}
   shares a reference to the mask used to create it. Unlike
   \member{shared_data}, this flag may change as the result of modifying the
   array contents, as the mask uses copy on write semantics if it is shared.
\end{memberdesc}



\subsection{Methods on masked arrays}
\label{sec:numarray.ma:meth-mask-arrays}

\begin{methoddesc}[MaskedArray]{__array__}
   A special method allows conversion to a \module{\numarray} array if no
   element is actually masked. If there is a masked element, an
   \exception{numarray.maError} exception is thrown. Many \module{\numarray}
   functions, such as \function{numarray.sqrt}, will attempt this conversion on
   their arguments. See also module function \function{filled} in section
   \ref{sec:numarray.ma:meth-mask-arrays}.
\begin{verbatim}
yn = numarray.array(x)
\end{verbatim}
\end{methoddesc}

\begin{methoddesc}[MaskedArray]{astype}{type}
   Return \var{self} as array of given \var{type}. 
\begin{verbatim}
y = x.astype(Float32)
\end{verbatim}
\end{methoddesc}

\begin{methoddesc}[MaskedArray]{byte_swapped}{}
   Returns the raw data \class{\numarray} byte-swapped; included for
   consistency with \module{\numarray} but probably meaningless. 
\begin{verbatim}
y = x.byte_swapped()
\end{verbatim}
\end{methoddesc}

\begin{methoddesc}[MaskedArray]{compressed}{}
   Return an array of the valid elements. Result is one-dimensional.  
\begin{verbatim}
y = x.compressed()
\end{verbatim}
\end{methoddesc}

\begin{methoddesc}[MaskedArray]{count}{axis=None}
   If \var{axis} is \constant{None} return the count of non-masked elements in
   the whole array.  Otherwise return an array of such counts along the axis
   given.
\begin{verbatim}
n = x.count()
y = x.count(0)
\end{verbatim}
\end{methoddesc}

\begin{methoddesc}[MaskedArray]{fill_value}{}
   Get the current fill value. 
\begin{verbatim}
v = x.fill_value()
\end{verbatim}
\end{methoddesc}

\begin{methoddesc}[MaskedArray]{filled}{fill_value=None}
   Returns a \module{\numarray} array with the masked values replaced by the
   fill value.  See also the description of module function filled in section
   \ref{sec:numarray.ma:meth-mask-arrays}.
\begin{verbatim}
yn = x.filled()
\end{verbatim}
\end{methoddesc}

\begin{methoddesc}[MaskedArray]{ids}{}
   Return the ids of the data and mask areas. 
\begin{verbatim}
id1, id2 = x.ids()
\end{verbatim}
\end{methoddesc}

\begin{methoddesc}[MaskedArray]{iscontiguous}{}
   Is the data area contiguous? See \method{numarray.scontiguous} in section
   \ref{arraymethod:iscontiguous}.
\begin{verbatim}
if x.iscontiguous():
\end{verbatim}
\end{methoddesc}

\begin{methoddesc}[MaskedArray]{itemsize}{}
   Size of individual data items in bytes. \samp{n = x.itemsize()}
\end{methoddesc}

\begin{methoddesc}[MaskedArray]{mask}{}
   Return the data mask, or \constant{None}. 
\begin{verbatim}
m = x.mask()
\end{verbatim}
\end{methoddesc}

\begin{methoddesc}[MaskedArray]{put}{values}
   Set the value at each non-masked entry to the corresponding entry in
   \var{values}. The mask is unchanged. See also module function
   \function{put}. 
\begin{verbatim}
x.put(values)
\end{verbatim}
\end{methoddesc}

\begin{methoddesc}[MaskedArray]{putmask}{values}
   Eliminate any masked values by setting the value at each masked entry to the
   corresponding entry in \var{values}. Set the mask to \constant{None}.
\begin{verbatim}
x.putmask(values)
assert getmask(x) is None
\end{verbatim}
\end{methoddesc}

\begin{methoddesc}[MaskedArray]{raw_data}{}
   A reference to the non-filled data; portions may be meaningless. Expert use
   only. 
\begin{verbatim}
d = x.raw_data ()
\end{verbatim}
\end{methoddesc}

\begin{methoddesc}[MaskedArray]{savespace}{v}
   Set the spacesaver attribute to \var{v}. 
\begin{verbatim}
x.savespace (1)
\end{verbatim}
\end{methoddesc}

\begin{methoddesc}[MaskedArray]{set_fill_value}{v}
   Set the fill value to \var{v}. Omit v to restore default.
   \samp{x.set_fill_value(1.e21)} \remark{Give correct default value for v.}
\end{methoddesc}

\begin{methoddesc}[MaskedArray]{set_shape}{args...}
   Set the shape. 
\begin{verbatim}
x.set_shape (3, 12)
\end{verbatim}
\end{methoddesc}

\begin{methoddesc}[MaskedArray]{size}{axis}
   Number of elements in array, or along a particular \var{axis}. 
\begin{verbatim}
totalsize = x.size ()
col_len = x.size (1)
\end{verbatim}
\end{methoddesc}

\begin{methoddesc}[MaskedArray]{spacesaver}{}
   Query the spacesave flag.
\begin{verbatim}
flag = x.spacesaver()
\end{verbatim}
\end{methoddesc}

\begin{methoddesc}[MaskedArray]{tolist}{fill_value=None}
   Return the Python \class{list} \code{self.filled(fill_value).tolist()}; note
   that masked values are filled. 
\begin{verbatim}
alist=x.tolist()
\end{verbatim}
\end{methoddesc}

\begin{methoddesc}[MaskedArray]{tostring}{fill_value=None}
   Return the string \code{self.filled(fill_value).tostring()s = x.tostring()}
\end{methoddesc}

\begin{methoddesc}[MaskedArray]{typecode}{}
   Return the type of the data. See module \module{Precision}, section \ref{TBD}.
\begin{verbatim}
z = x.typecode()
\end{verbatim}
\end{methoddesc}

\begin{methoddesc}[MaskedArray]{unmask}{}
   Replaces the mask by \constant{None} if possible. Subsequent operations may
   be faster if the array previously had an all-zero mask.
\begin{verbatim}
x.unmask()
\end{verbatim}
\end{methoddesc}

\begin{methoddesc}[MaskedArray]{unshare_mask}{}
   If shared_mask is currently true, replaces the reference to it with a
   copy. 
\begin{verbatim}
x.unshare_mask()
\end{verbatim}
\end{methoddesc}


\subsection{Constructing masked arrays}
\label{sec:numarray.ma:constructing-mask-arrays}

\index{numarray.ma!constructor}
\begin{methoddesc}[MaskedArray]{array}
   {data, type=None, copy=1, savespace=0, mask=None, fill_value=None}
   Creates a masked array with the given \var{data} and
   \var{mask}.  The name \class{array} is simply an alias for the class name,
   \class{MaskedArray}.  The fill value is set to \var{fill_value}, and the
   \var{savespace} flag is applied. If \var{data} is a \class{MaskedArray}, its
   \constant{mask}, \constant{typecode}, \constant{spacesaver} flag, and
   \constant{fill_value} will be used unless specifically overridden by one of
   the remaining arguments. In particular, if \var{d} is a masked array,
   \code{array(d, copy=0)} is \var{d}.
\end{methoddesc}

\index{numarray.ma!constructor}
\begin{methoddesc}[MaskedArray]{masked_array}{data, mask=None, fill_value=None}
   This is an easier-to-use version of \method{array},
   for the common case of \code{typecode = None}, \code{copy = 0}. When
   \var{data} is newly-created this function can be used to make it a masked
   array without copying the data if \var{data} is already a \module{\numarray}
   array.
\end{methoddesc}

\index{numarray.ma!constructor}
\begin{methoddesc}[MaskedArray]{masked_values}{data, value, rtol=1.e-5, atol=1.e-8, type=None, copy=1, savespace=0)}
   Constructs a masked array whose mask is set at those places where 
   \begin{equation}
      \abs(\var{data} - \var{value}) < \var{atol} + \var{rtol} * \abs(\var{data})
   \end{equation}
   That is a careful way of saying that those elements of the \var{data} that
   have a value of \var{value} (to within a tolerance) are to be treated as
   invalid.  If data is not of a floating point type, calls
   \method{masked_object} instead.
\end{methoddesc}

\index{numarray.ma!constructor}
\begin{methoddesc}[MaskedArray]{masked_object}{data, value, copy=1, savespace=0}
   Creates a masked array with those entries marked invalid that are equal to
   \var{value}. Again, \var{copy} and \var{/savespace} are passed on to the
   \module{\numarray} array constructor.
\end{methoddesc}

\index{numarray.ma!constructor}
\begin{methoddesc}[MaskedArray]{asarray}{data, type=None}
   This is the same as \code{array(data, typecode, copy=0)}. It is a short way
   of ensuring that something is an instance of \class{MaskedArray} of a given
   \var{type} before proceeding, as in \samp{data = asarray(data)}.
   
   If \var{data} already is a masked array and \var{type} is \constant{None}
   then the return value is \var{data}; nothing is copied in that case.
\end{methoddesc}

\index{numarray.ma!constructor}
\begin{methoddesc}[MaskedArray]{masked_where}{condition, data, copy=1)}
   Creates a masked array whose shape is that of \var{condition}, whose values
   are those of \var{data}, and which is masked where elements of
   \var{condition} are true.
\end{methoddesc}

\index{numarray.ma!constructor}
\begin{datadesc}{masked}
   This is a module constant that represents a scalar masked value. For
   example, if \var{x} is a masked array and a particular location such as
   \code{x[1]} is masked, the quantity \code{x[1]} will be this special
   constant. This special element is discussed more fully in section
   \ref{sec:numarray.ma:constant-masked} ``The constant \constant{masked}''.
\end{datadesc}


The following additional constructors are provided for convenience.

\index{numarray.ma!constructor}
\begin{methoddesc}[MaskedArray]{masked_equal}{data, value, copy=1}
\end{methoddesc} \index{numarray.ma!constructor}
\begin{methoddesc}[MaskedArray]{masked_greater}{data, value, copy=1}
\end{methoddesc} \index{numarray.ma!constructor}
\begin{methoddesc}[MaskedArray]{masked_greater_equal}{data, value, copy=1}
\end{methoddesc} \index{numarray.ma!constructor}
\begin{methoddesc}[MaskedArray]{masked_less}{data, value, copy=1}
\end{methoddesc} \index{numarray.ma!constructor}
\begin{methoddesc}[MaskedArray]{masked_less_equal}{data, value, copy=1}
\end{methoddesc} \index{numarray.ma!constructor}
\begin{methoddesc}[MaskedArray]{masked_not_equal}{data, value, copy=1}
   \method{masked_greater} is equivalent to \code{masked_where(greater(data,
      value), data))}.  Similarly, \method{masked_greater_equal},
   \method{masked_equal}, \method{masked_not_equal}, \method{masked_less},
   \method{masked_less_equal} are called in the same way with the obvious
   meanings.  Note that for floating point data, \method{masked_values} is
   preferable to \method{masked_equal} in most cases.  \remark{because...}
\end{methoddesc}

\index{numarray.ma!constructor}
\begin{methoddesc}[MaskedArray]{masked_inside}{data, v1, v2, copy=1}
   Creates an array with values in the closed interval \code{[v1, v2]} masked.
   \var{v1} and \var{v2} may be in either order.
\end{methoddesc}

\index{numarray.ma!constructor}
\begin{methoddesc}[MaskedArray]{masked_outside}{data, v1, v2, copy=1}
   Creates an array with values outside the closed interval \code{[v1, v2]}
   masked.  \var{v1} and \var{v2} may be in either order.
\end{methoddesc}

On entry to any of these constructors, \var{data} must be any object which the
\module{\numarray} package can accept to create an array (with the desired
\var{type}, if specified). The \var{mask}, if given, must be \constant{None} or
any object that can be turned into a \module{\numarray} array of integer type
(it will be converted to type \class{MaskType}, if necessary), have the same
shape as \var{data}, and contain only values of 0 or 1.

If the \var{mask} is not \constant{None} but its shape does not match that of
\var{data}, an exception will be thrown, unless one of the two is of length 1,
in which case the scalar will be resized (using \method{numarray.resize}) to
match the other.

See section \ref{sec:numarray.ma:copying-or-not} ``Copying or not'' for a
discussion of whether or not the resulting array shares its data or its mask
with the arguments given to these constructors.


\paragraph*{Important Tip} \method{filled} is very important. It converts its
argument to a plain \module{\numarray} array.

\begin{funcdesc}{filled}{x, value=None}
   Returns \var{x} with any invalid locations replaced by a fill \var{value}.
   \function{filled} is guaranteed to return a plain \module{\numarray} array.
   The argument \var{x} does not have to be a masked array or even an array,
   just something that \module{\numarray}/\module{numarray.ma} can turn into
   one.
   \begin{itemize}
   \item If \var{x} is not a masked array, and not a \module{\numarray} array,
      \code{numarray.array(x)} is returned.
   \item If \var{x} is a contiguous \module{\numarray} array then \var{x} is
      returned. (A \module{\numarray} array is contiguous if its data storage
      region is layed out in column-major order; \module{\numarray} allows
      non-contiguous arrays to exist but they are not allowed in certain
      operations).
   \item If \var{x} is a masked array, but the mask is \constant{None}, and
      \var{x}'s data array is contiguous, then it is returned. If the data
      array is not contiguous, a (contiguous) copy of it is returned.
   \item If \var{x} is a masked array with an actual mask, then an array formed
      by replacing the invalid entries with \var{value}, or
      \code{fill_value(x)} if \var{value} is \constant{None}, is returned. If
      the fill value used is of a different type or precision than \var{x}, the
      result may be of a different type or precision than \var{x}.
\end{itemize}
Note that a new array is created only if necessary to create a correctly
filled, contiguous, \module{\numarray} array.

The function \method{filled} plays a central role in our design. It is the
``exit'' back to \module{\numarray}, and is used whenever the invalid values
must be replaced before an operation. For example, adding two masked arrays
\var{a} and \var{b} is roughly:
\begin{verbatim}
masked_array(filled(a, 0) + filled(b, 0), mask_or(getmask(a), getmask(b))
\end{verbatim}
That is, fill the invalid entries of \var{a} and \var{b} with zeros, add them
up, and declare any entry of the result invalid if either \var{a} or \var{b}
was invalid at that spot. The functions \function{getmask} and
\function{mask_or} are discussed later.

\function{filled} also can be used to simply be certain that some expression is
a contiguous \module{\numarray} array at little cost. If its argument is a
\module{\numarray} array already, it is returned without copying.

If you are certain that a masked array \var{x} contains a mask that is None or
is all zeros, you can convert it to a numarray array with the
\method{numarray.array(x)} constructor. If you turn out to be wrong, an
\exception{MAError} exception is raised.
\end{funcdesc}

\begin{funcdesc}{fill_value}{x}
\end{funcdesc}
\begin{methoddesc}[MaskedArray]{fill_value}{}
   \code{fill_value(x)} and the method \code{x.fill_value()} on masked arrays,
   return a value suitable for filling \var{x} based on its type.  If \var{x}
   is a masked array, then \var{x.fill_value()} results. The returned value for
   a given type can be changed by assigning to the following names in module
   \module{numarray.ma}. They should be set to scalars or one element arrays.
   \index{numarray.ma!default_real_fill_value@\constant{default_real_fill_value}}
   \index{numarray.ma!default_complex_fill_value@\constant{default_complex_fill_value}}
   \index{numarray.ma!default_character_fill_value@\constant{default_character_fill_value}}
   \index{numarray.ma!default_integer_fill_value@\constant{default_integer_fill_value}}
   \index{numarray.ma!default_object_fill_value@\constant{default_object_fill_value}}
\begin{verbatim}
default_real_fill_value = numarray.array([1.0e20], Float32)
default_complex_fill_value = numarray.array([1.0e20 + 0.0j], Complex32)
default_character_fill_value = masked
default_integer_fill_value = numarray.array([0]).astype(UnsignedInt8)
default_object_fill_value = masked
\end{verbatim}
   The variable \var{masked} is a module variable of \module{numarray.ma} and
   is discussed in section \ref{sec:numarray.ma:constant-masked}. Calling
   \function{filled} with a \var{fill_value} of \constant{masked} sometimes
   produces a useful printed representation of a masked array.  The function
   \function{fill_value} works on any kind of object.
\end{methoddesc}

\index{numarray.ma!set_fill_value@\method{set_fill_value}}\code{set_fill_value(a,
   fill_value)} is the same as \code{a.set_fill_value (fill_value)} if \var{a}
   is a masked array; otherwise it does nothing. Please note that the fill
   value is mostly cosmetic; it is used when it is needed to convert the masked
   array to a plain \module{\numarray} array but not involved in most
   operations. In particular, setting the \member{fill_value} to
   \constant{1.e20} will \emph{not}, repeat not, cause elements of the array
   whose values are currently 1.e20 to be masked. For that sort of behavior use
   the \method{masked_value} constructor.



\subsection{What are masks?}
\label{sec:numarray.ma:what-are-masks}
\index{masks, description of}
\index{masks, savespace attribute}

Masks are either \constant{None} or 1-byte \module{\numarray} arrays of 1's and
0's. To avoid excessive performance penalties, mask arrays are never checked to
be sure that the values are 1's and 0's, and supplying a \var{mask} argument to
a constructor with an illegal mask will have undefined consequences later.

\emph{Masks have the savespace attribute set.}  This attribute, discussed in
part \ref{part:numerical-python}, may have surprising consequences if you
attempt to do any operations on them other than those supplied by this package.
In particular, do not add or multiply a quantity involving a mask. For example,
if \var{m} is a mask consisting of 1080 1 values, \code{sum(m)} is 56, not
1080. Oops.


\subsection{Working with masks}

\begin{funcdesc}{is_mask}{m}
   Returns true if \var{m} is of a type and precision that would be allowed as
   the mask field of a masked array (that is, it is an array of integers with
   \module{\numarray}'s typecode \class{MaskType}, or it is \constant{None}).
   To be a legal mask, \var{m} should contain only zeros or ones, but this is
   not checked.
\end{funcdesc}

\begin{funcdesc}{make_mask}{m, copy=0, flag=0}
   Returns an object whose entries are equal to \var{m} and for which
   \function{is_mask} would return true. If \var{m} is already a mask or
   \constant{None}, it returns \var{m} or a copy of it. Otherwise it will
   attempt to make a mask, so it will accept any sequence of integers for
   \var{m}. If \var{flag} is true, \method{make_mask} returns \constant{None}
   if its return value otherwise would contain no true elements. To make a
   legal mask, \var{m} should contain only zeros or ones, but this is not
   checked.
\end{funcdesc}

\begin{funcdesc}{make_mask_none}{s}
   Returns a mask of all zeros of shape \var{s} (deprecated name:
   \index{numarray.ma!create_mask@\method{create_mask}
      (deprecated)|see{\method{make_mask_none}}}create_mask).
\end{funcdesc}

\begin{funcdesc}{getmask}{x}
   Returns \index{numarray.ma!mask@\method{mask}}\code{x.mask()}, the mask of
   \var{x}, if \var{x} is a masked array, and \constant{None} otherwise.
   \note{\function{getmask} may return \constant{None} if \var{x} is a masked
   array but has a mask of \constant{None}.  (Please see caution above about
   operating on the result).}
\end{funcdesc}

\begin{funcdesc}{getmaskarray}{x}
   Returns \code{x.mask()} if \var{x} is a masked array and has a mask that is
   not \constant{None}; otherwise it returns a zero mask array of the same
   shape as \var{x}.  Unlike \method{getmask}, \method{getmaskarray} always
   returns an \module{\numarray} array of typecode \class{MaskType}. (Please
   see caution above about operating on the result).
\end{funcdesc}

\begin{funcdesc}{mask_or}{m1, m2}
   Returns an object which when used as a mask behaves like the element-wise
   ``logical or'' of \var{m1} and \var{m2}, where \var{m1} and \var{/m2} are
   either masks or \constant{None} (e.g., they are the results of calling
   \method{getmask}). A \constant{None} is treated as everywhere false. If both
   \var{m1} and \var{m2} are \constant{None}, it returns \constant{None}. If
   just one of them is \constant{None}, it returns the other. If \var{m1} and
   \var{m2} refer to the same object, a reference to that object is returned.
\end{funcdesc}


\subsection{Operations}
\label{sec:numarray.ma:operations}

Masked arrays support the operators $+$, $*$, $/$, $-$, $**$, and unary plus
and minus.  The other operand can be another masked array, a scalar, a
\module{\numarray} array, or something \method{numarray.array} can convert to a
\module{\numarray} array. The results are masked arrays.

In addition masked arrays support the in-place operators $+=$, $-=$, $*=$, and
$/=$.  Implementation of in-place operators differs from \module{\numarray}
semantics in being more generous about converting the right-hand side to the
required type: any kind or lesser type accepted via an \method{astype}
conversion.  In-place operators truly operate in-place when the target is not
masked.



\subsection{Copying or not?}
\label{sec:numarray.ma:copying-or-not}

Depending on the arguments results of constructors may or may not contain a
separate copy of the data or mask arguments. The easiest way to think about
this is as follows: the given field, be it data or a mask, is required to be a
\module{\numarray} array, possibly with a given typecode, and a mask's shape
must match that of the data. If the copy argument is zero, and the candidate
array otherwise qualifies, a reference will be made instead of a copy. If for
any reason the data is unsuitable as is, an attempt will be made to make a copy
that is suitable. Should that fail, an exception will be thrown. Thus, a
\code{copy=0} argument is more of a hope than a command.

If the basic array \index{numarray.ma!constructor}constructor is given a masked
array as the first argument, its mask, typecode, spacesaver flag, and fill
value will be used unless specifically specified by one of the remaining
arguments. In particular, if \var{d} is a masked array, \code{array(d, copy=0)}
is \var{d}.

Since the default behavior for masks is to use a reference if possible, rather
than a copy, which produces a sizeable time and space savings, it is especially
important not to modify something you used as a mask argument to a masked array
creation routine, if it was a \module{\numarray} array of typecode
\class{MaskType}.





\subsection{Behaviors}
\label{sec:numarray.ma:behaviors}
\begin{funcdesc}{float}{a}
\end{funcdesc}
\begin{funcdesc}{int}{a}
  The conversion operators \function{float}, and \function{int} are defined
  to operate on masked arrays consisting of a single unmasked element.
  Masked values and multi-element arrays are not convertible.  
\end{funcdesc}
\begin{funcdesc}{repr}{a}
\end{funcdesc}
\begin{funcdesc}{str}{a}
  A masked array defines the conversion operators \function{str} and
  \function{repr} by applying the corresponding operator to the
  \module{\numarray} array \code{filled(a)}.  
\end{funcdesc}


\subsection{Indexing and Slicing}
\label{sec:numarray.ma:indexing-slicing}

Indexing and slicing differ from Numeric: while generally the same, they return
a copy, not a reference, when used in an expression that produces a non-scalar
result. Consider this example:
\begin{verbatim}
from Numeric import *
x = array([1.,2.,3.])
y = x[1:]
y[0] = 9.
print x
\end{verbatim}
This will print \code{[1., 9., 3.]} since \code{x[1:]} returns a reference to a
portion of \var{x}.  Doing the same operation using \module{numarray.ma},
\begin{verbatim}
from numarray.ma import *
x = array([1.,2.,3.])
y = x[1:]
y[0] = 9.
print x
\end{verbatim}
will print \code{[1., 2., 3.]}, while \var{y} will be a separate array whose
present value would be \code{[9., 3.]}. While sentiment on the correct
semantics here is divided amongst the Numeric Python community as a whole, it
is not divided amongst the author's community, on whose behalf this package is
written.


\subsection{Indexing in assignments}
\label{sec:numarray.ma:indexing-assignments}

Using multiple sets of square brackets on the left side of an assignment
statement will not produce the desired result:
\begin{verbatim}
x = array([[1,2],[3,4]])
x[1][1] = 20.                           # Error, does not change x
x[1,1] = 20.                            # Correct, changes x
\end{verbatim}
The reason is that \code{x[1]} is a copy, so changing it changes that copy, not
\var{x}.  Always use just one single square bracket for assignments.


\subsection{Operations that produce a scalar result}
\label{sec:numarray.ma:operations-producing-scalars}

If indexing or another operation on a masked array produces a scalar result,
then a scalar value is returned rather than a one-element masked array. This
raises the issue of what to return if that result is masked. The answer is that
the module constant
\index{numarray.ma!masked@\constant{masked}}\constant{masked} is returned. This
constant is discussed in section \ref{sec:numarray.ma:constant-masked}.  While
this most frequently occurs from indexing, you can also get such a result from
other functions. For example, averaging a 1-D array, all of whom's values are
invalid, would result in \constant{masked}.


\subsection{Assignment to elements and slices}
\label{sec:numarray.ma:assignments-elements-slices}

Assignment of a normal value to a single element or slice of a masked array has
the effect of clearing the mask in those locations. In this way previously
\index{numarray.ma!invalid}invalid elements become
\index{numarray.ma!valid}valid. The value being assigned is filled first, so
that you are guaranteed that all the elements on the left-hand side are now
valid.  \remark{???}

Assignment of \constant{None} to a single element or slice of a masked array
has the effect of setting the mask in those locations, and the locations become
invalid.

Since these operations change the mask, the result afterwards will no longer
share a mask, since masks have copy-on-write semantics.



\section{MaskedArray Attributes}
\label{sec:numarray.ma:attributes}

\begin{datadesc}{e}
\end{datadesc}
\begin{datadesc}{pi}
\end{datadesc}
\begin{datadesc}{NewAxis}
   Constants \constant{e}, \constant{pi}, \constant{NewAxis} from
   \module{\numarray}, and the constants from module \module{Precision} that
   define nice names for the typecodes.
\end{datadesc}

The special variables \index{numarray.ma!masked@\constant{masked}}\constant{masked} and
\index{numarray.ma!masked@\constant{masked}}masked_print_option are discussed in section
\ref{sec:numarray.ma:constant-masked}.

The module \module{\numarray} is an element of \module{numarray.ma}, so after \samp{from
   numarray.ma import *}, you can refer to the functions in \module{\numarray} such as
\constant{numarray.ones}; see part \ref{part:numerical-python} for the
constants available in \module{\numarray}.




\section{MaskedArray Functions}
\label{sec:numarray.ma:functions}

Each of the operations discussed below returns an instance of \module{numarray.ma} class
\index{numarray.ma!MaskedArray@\class{MaskedArray}}\class{MaskedArray}, having performed
the desired operation element-wise.  In most cases the array arguments can be
masked arrays or \module{\numarray} arrays or something that \module{\numarray}
can turn into a \module{\numarray} array, such as a list of real numbers.

In most cases, if \module{\numarray} has a function of the same name, the
behavior of the one in \module{numarray.ma} is the same, except that it ``respects'' the
mask.


\subsection{Unary functions}
\label{sec:numarray.ma:unary-functions}

The result of a unary operation will be masked wherever the original operand
was masked. It may also be masked if the argument is not in the domain of the
function.  The following functions have their standard meaning:
\begin{quote}
   \index{absolute@\function{absolute} (in module numarray.ma)}\function{absolute}, 
   \index{arccos@\function{arccos} (in module numarray.ma)}\function{arccos}, 
   \index{arcsin@\function{arcsin} (in module numarray.ma)}\function{arcsin}, 
   \index{arctan@\function{arctan} (in module numarray.ma)}\function{arctan}, 
   \index{around@\function{around} (in module numarray.ma)}\function{around}, 
   \index{conjugate@\function{conjugate} (in module numarray.ma)}\function{conjugate}, 
   \index{cos@\function{cos} (in module numarray.ma)}\function{cos}, 
   \index{cosh@\function{cosh} (in module numarray.ma)}\function{cosh}, 
   \index{exp@\function{exp} (in module numarray.ma)}\function{exp},
   \index{floor@\function{floor} (in module numarray.ma)}\function{floor},
   \index{log@\function{log} (in module numarray.ma)}\function{log}, 
   \index{log10@\function{log10} (in module numarray.ma)}\function{log10}, 
   \index{negative@\function{negative} (in module numarray.ma)}\function{negative}
   (also as operator \index{- (in module numarray.ma)}\index{numarray.ma!-}-),
   \index{nonzero@\function{nonzero} (in module numarray.ma)}\function{nonzero}, 
   \index{sin@\function{sin} (in module numarray.ma)}\function{sin}, 
   \index{sinh@\function{sinh} (in module numarray.ma)}\function{sinh}, 
   \index{sqrt@\function{sqrt} (in module numarray.ma)}\function{sqrt}, 
   \index{tan@\function{tan} (in module numarray.ma)}\function{tan}, 
   \index{tanh@\function{tanh} (in module numarray.ma)}\function{tanh}.
\end{quote}

\begin{funcdesc}{fabs}{x}
   The absolute value of \var{x} as a \constant{Float32} array.
   \remark{What happens when you pass \constant{Float64} ?}
\end{funcdesc}


\subsection{Binary functions}
\label{sec:numarray.ma:binary-functions}

Binary functions return a result that is masked wherever either of the operands
were masked; it may also be masked where the arguments are not in the domain of
the function.

\begin{quote}
   \index{add@\function{add} (in module numarray.ma)}\function{add}
   (also as operator \index{+}\index{numarray.ma!+}+),
   \index{subtract@\function{subtract} (in module numarray.ma)}\function{subtract}
   \index{- (in module numarray.ma)}\index{numarray.ma!-}(also as operator -),
   \index{multiply@\function{multiply} (in module numarray.ma)}\function{multiply}
   \index{* (in module numarray.ma)}\index{numarray.ma!*}(also as operator *), 
   \index{divide@\function{divide} (in module numarray.ma)}\function{divide}
   \index{/ (in module numarray.ma)}\index{numarray.ma!/}(also as operator / ), 
   \index{power@\function{power} (in module numarray.ma)}\function{power}
   \index{** (in module numarray.ma)}\index{numarray.ma!**}(also as operator **), 
   \index{remainder@\function{remainder} (in module numarray.ma)}\function{remainder},
   \index{fmod@\function{fmod} (in module numarray.ma)}\function{fmod},
   \index{hypot@\function{hypot} (in module numarray.ma)}\function{hypot},
   \index{arctan2@\function{arctan2} (in module numarray.ma)}\function{arctan2},
   \index{bitwise_and@\function{bitwise_and} (in module numarray.ma)}\function{bitwise_and},
   \index{bitwise_or@\function{bitwise_or} (in module numarray.ma)}\function{bitwise_or},
   \index{bitwise_xor@\function{bitwise_xor} (in module numarray.ma)}\function{bitwise_xor}.
\end{quote}



\subsection{Comparison operators}

To compare arrays, use the following binary functions. Each of them returns a
masked array of 1's and 0's.

\begin{quote}
   \index{equal@\function{equal} (in module numarray.ma)}\function{equal},
   \index{greater@\function{greater} (in module numarray.ma)}\function{greater},
   \index{greater_equal@\function{greater_equal} (in module numarray.ma)}\function{greater_equal},
   \index{less@\function{less} (in module numarray.ma)}\function{less},
   \index{less_equal@\function{less_equal} (in module numarray.ma)}\function{less_equal},
   \index{not_equal@\function{not_equal} (in module numarray.ma)}\function{not_equal}.
\end{quote}

Note that as in \module{\numarray}, you can use a scalar for one argument and
an array for the other. \note{See section \ref{TBD} why operators and comparison
   functions are not excatly equivalent.}



\subsection{Logical operators}

Arrays of logical values can be manipulated with:

\begin{quote}
   \index{logical_and@\function{logical_and} (in module numarray.ma)}\function{logical_and},
   \index{logical_not@\function{logical_not} (in module numarray.ma)}\function{logical_not (unary)},
   \index{logical_or@\function{logical_or} (in module numarray.ma)}\function{logical_or},
   \index{logical_xor@\function{logical_xor} (in module numarray.ma)}\function{logical_xor}.
\end{quote}

\begin{funcdesc}{alltrue}{x}
   Returns 1 if all elements of \var{x} are true. Masked elements are treated
   as true.
\end{funcdesc}

\begin{funcdesc}{sometrue}{x}
   Returns 1 if any element of \var{x} is true. Masked elements are treated as
   false.
\end{funcdesc}



\subsection{Special array operators}

\begin{funcdesc}{isarray}{x}
   Return true \var{x} is a masked array.
   \remark{What is about \numarray's?}
\end{funcdesc}

\begin{funcdesc}{rank}{x} 
   The number of dimensions in \var{x}.
\end{funcdesc}

\begin{funcdesc}{shape}{x}
   Returns the shape of \var{x}, a tuple of array extents.
\end{funcdesc}

\begin{funcdesc}{resize}{x, shape}
   Returns a new array with specified \var{shape}.
\end{funcdesc}

\begin{funcdesc}{reshape}{x, shape}
   Returns a copy of \var{x} with the given new \var{shape}.
\end{funcdesc}

\begin{funcdesc}{ravel}{x}
   Returns \var{x} as one-dimensional \class{MaskedArray}.
\end{funcdesc}

\begin{funcdesc}{concatenate}{(a0, ... an), axis=0}
   Concatenates the arrays \code{a0, ... an} along the specified \var{axis}.
\end{funcdesc}

\begin{funcdesc}{repeat}{a, repeats, axis=0}
   Repeat elements \var{i} of \var{a} \code{repeats[i]} times along \var{axis}.
   \var{repeats} is a sequence of length \code{a.shape[axis]} telling how many
   times to repeat each element.
\end{funcdesc}

\begin{funcdesc}{identity}{n}
   Returns the identity matrix of shape \var{n} by \var{n}.
\end{funcdesc}

\begin{funcdesc}{indices}{dimensions, type=None}
   Returns an array representing a grid of indices with row-only and
   column-only variation.
\end{funcdesc}

\begin{funcdesc}{len}{x}
   This is defined to be the length of the first dimension of \var{x}. This
   definition, peculiar from the array point of view, is required by the way
   Python implements slicing. Use \function{size} for the total length of
   \var{x}.
\end{funcdesc}

\begin{funcdesc}{size}{x, axis=None}
   This is the total size of \var{x}, or the length of a particular dimension
   \var{axis} whose index is given. When axis is given the dimension of the
   result is one less than the dimension of \var{x}.
\end{funcdesc}

\begin{funcdesc}{count}{x, axis=None}
   Count the number of (non-masked) elements in the array, or in the array
   along a certain \var{axis}.  When \var{axis} is given the dimension of the
   result is one less than the dimension of \var{x}.
\end{funcdesc}

\begin{funcdesc}{arange}{}
\end{funcdesc}
\begin{funcdesc}{arrayrange}{}
\end{funcdesc}
\begin{funcdesc}{diagonal}{}
\end{funcdesc}
\begin{funcdesc}{fromfunction}{}
\end{funcdesc}
\begin{funcdesc}{ones}{}
\end{funcdesc}
\begin{funcdesc}{zeros}{}
   are the same as in numarray, but return masked arrays.
\end{funcdesc}

\begin{funcdesc}{sum}{}
\end{funcdesc}
\begin{funcdesc}{product}{}
   are called the same way as count; the difference is that the result is the
   sum or product of the unmasked element.
\end{funcdesc}

\begin{funcdesc}{average}{x, axis=0, weights=None, returned=0}
   Compute the average value of the non-masked elements of \var{x} along the
   selected \var{axis}. If \var{weights} is given, it must match the size and
   shape of \var{x}, and the value returned is:
   \begin{equation}
      \text{average} = \frac{\sum{}weights_i\cdot{}x_i}{\sum{}weights_i}
   \end{equation}
   In computing these sums, elements that correspond to those that are masked
   in \var{x} or \var{weights} are ignored. If successful a 2-tuple consisting
   of the average and the sum of the weights is returned.
\end{funcdesc}

\begin{funcdesc}{allclose}{x, y, fill_value=1, rtol=1.e-5, atol=1.e-8}
   Test whether or not arrays \var{x} and \var{y} are equal subject to the
   given relative and absolute tolerances. If \var{fill_value} is 1, masked
   values are considered equal, otherwise they are considered different. The
   formula used for elements where both \var{x} and \var{y} have a valid value
   is:
   \begin{equation}
      |x-y| < \var{atol} + \var{rtol} \cdot{} |y|
   \end{equation}
   This means essentially that both elements are small compared to \var{atol}
   or their difference divided by their value is small compared to \var{rtol}.
\end{funcdesc}

\begin{funcdesc}{allequal}{x, y, fill_value=1}
   This function is similar to \function{allclose}, except that exact equality
   is demanded. \note{Consider the problems of floating-point representations
      when using this function on non-integer numbers arrays.}
\end{funcdesc}

\begin{funcdesc}{take}{a, indices, axis=0}
   Returns a selection of items from \var{a}. See the documentation of
   \function{numarray.take} in section \ref{sec:array-functions:take}.
\end{funcdesc}

\begin{funcdesc}{transpose}{a, axes=None}
   Performs a reordering of the axes depending on the tuple of indices
   \var{axes}; the default is to reverse the order of the axes.
\end{funcdesc}

\begin{funcdesc}{put}{a, indices, values}
   The opposite of \function{take}. The values of the array \var{a} at the
   locations specified in \var{indices} are set to the corresponding value of
   \var{values}.  The array \var{a} must be a contiguous array. The argument
   \var{indices} can be any integer sequence object with values suitable for
   indexing into the flat form of \var{a}.  The argument \var{values} must be
   any sequence of values that can be converted to the typecode of \var{a}.
\begin{verbatim}
>>> x = arange(6)
>>> put(x, [2,4], [20,40])
>>> print x
[ 0  1 20  3 40  5 ]
\end{verbatim}
   Note that the target array \var{a} is not required to be one-dimensional.
   Since it is contiguous and stored in row-major order, the array indices can
   be treated as indexing \var{a}s elements in storage order.
   
   The wrinkle on this for masked arrays is that if the locations being set by
   \function{put} are masked, the mask is cleared in those locations.
\end{funcdesc}

\begin{funcdesc}{choose}{condition, t}
   This function has a result shaped like \var{condition}. \var{t} must be a
   tuple. Each element of the tuple can be an array, a scalar, or the constant
   element \constant{masked} (See section \ref{sec:numarray.ma:constant-masked}). Each
   element of the result is the corresponding element of \code{t[i]} where
   \var{condition} has the value \var{i}. The result is masked where
   \var{condition} is masked or where the selected element is masked or the
   selected element of \var{t} is the constant \constant{masked}.
\end{funcdesc}

\begin{funcdesc}{where}{condition, x, y}
   Returns an array that is \code{filled(x)} where \var{condition} is true,
   \code{filled(y)} where the condition is false. One of \var{x} or \var{y} can
   be the constant element \constant{masked} (See section
   \ref{sec:numarray.ma:constant-masked}). The result is masked where \var{condition} is
   masked, where the element selected from \var{x} or \var{y} is masked, or
   where \var{x} or \var{y} itself is the constant \constant{masked} and it is
   selected.
\end{funcdesc}

\begin{funcdesc}{innerproduct}{a, b}
\end{funcdesc}
\begin{funcdesc}{dot}{a, b}
   These functions work as in \module{\numarray}, but missing values don't
   contribute. The result is always a masked array, possibly of length one,
   because of the possibility that one or more entries in it may be invalid
   since all the data contributing to that entry was invalid.
\end{funcdesc}

\begin{funcdesc}{outerproduct}{a, b}
   Produces a masked array such that \code{result[i, j] = a[i] * b[j]}. The
   result will be masked where \code{a[i]} or \code{b[j]} is masked.
\end{funcdesc}

\begin{funcdesc}{compress}{condition, x, dimension=-1}
   Compresses out only those valid values where \var{condition} is true. Masked
   values in \var{condition} are considered false.
\end{funcdesc}

\begin{funcdesc}{maximum}{x, y=None}
\end{funcdesc}
\begin{funcdesc}{minimum}{x, y=None}
   Compute the maximum (minimum) valid values of \var{x} if \var{y} is
   \constant{None}; with two arguments, they return the element-wise larger or
   smaller of valid values, and mask the result where either \var{x} or \var{y}
   is masked.  If both arguments are scalars a scalar is returned.
\end{funcdesc}

\begin{funcdesc}{sort}{x, axis=-1, value=None}
   Returns the array \var{x} sorted along the given axis, with masked values
   treated as if they have a sort value of \var{value} but locations containing
   \var{value} are masked in the result if \var{x} had a mask to start with.
   \note{Thus if \var{x} contains \var{value} at a non-masked spot, but has
      other spots masked, the result may not be what you want.}
\end{funcdesc}

\begin{funcdesc}{argsort}{x, axis=-1, fill_value=None}
   This function is unusual in that it returns a \module{\numarray} array,
   equal to \code{numarray.argsort(filled(x, fill_value), axis)}; this is an
   array of indices for sorting along a given axis.
\end{funcdesc}



\subsection{Controlling the size of the string representations}
\label{sec:numarray.ma:contr-size-string}


\begin{funcdesc}{get_print_limit}{}
\end{funcdesc}
\begin{funcdesc}{set_print_limit}{n=0}
   These functions are used to limit printing of large arrays; query and set
   the limit for converting arrays using \function{str} or \function{repr}.
   
   If an array is printed that is larger than this, the values are not printed;
   rather you are informed of the type and size of the array. If \var{n} is
   zero, the standard \module{\numarray} conversion functions are used.
   
   When imported, \module{numarray.ma} sets this limit to 300, and the limit is also
   made to apply to standard \module{\numarray} arrays as well.
\end{funcdesc}



\section{Helper classes}
\label{sec:numarray.ma:helper-classes}

\begin{quote}
   This section discusses other classes defined in module numarray.ma.
\end{quote}

\begin{classdesc}{MAError}
   This class inherits from Exception, used to raise exceptions in the
   \module{numarray.ma} module. Other exceptions are possible, such as errors from the
   underlying \module{\numarray} module.
\end{classdesc}


\subsection{The constant masked}
\label{sec:numarray.ma:constant-masked}
\index{numarray.ma!masked@\constant{masked} (constant)}

A constant named \index{numarray.ma!masked@\constant{masked}}\constant{masked} in
\module{numarray.ma} serves several purposes.
\begin{enumerate}
\item When a indexing operation on an \class{MaskedArray} instance returns a
   scalar result, but the location indexed was masked, then \constant{masked}
   is returned. For example, given a one-dimensional array \var{x} such that
   \code{x.mask()[3]} is 1, then \code{x[3]} is \constant{masked}.
\item When \constant{masked} is assigned to elements of an array via indexing
   or slicing, those elements become masked. So after \code{x[3] = masked},
   \code{x[3]} is masked.
\item Some other operations that may return scalar values, such as
   \function{average}, may return \constant{masked} if given only invalid data.
\item To test whether or not a variable is this element, use the \function{is}
   or \function{is not} operator, not \code{==} or \code{!=}.
\item Operations involving the constant \constant{masked} may result in an
   exception.  In operations, \constant{masked} behaves as an integer array of
   shape \code{()} with one masked element. For example, using \code{+} for
   illustration,
   \begin{itemize}
   \item \constant{masked} + \constant{masked} is \constant{masked}.
   \item \constant{masked} + numeric scalar or numeric scalar +
      \constant{masked} is \constant{masked}.
   \item \constant{masked} + array or array + \constant{masked} is a masked
      array with all elements \constant{masked} if array is of a numeric type.
      The same is true if array is a \module{\numarray} array.
   \end{itemize}
\end{enumerate}



\subsection{The constant masked_print_option}
\index{numarray.ma!masked_print_option@\constant{masked_print_option} (constant)}


Another constant, \constant{masked_print_option} controls what happens when
masked arrays and the constant
\index{numarray.ma!masked@\constant{masked}}\constant{masked} are printed:

\begin{methoddesc}[masked_print_option]{display}{} 
   Returns a string that may be used to indicate those elements of an array
   that are masked when the array is converted to a string, as happens with the
   print statement.
\end{methoddesc}

\begin{methoddesc}[masked_print_option]{set_display}{string} 
   This functions can be used to set the string that is used to indicate those
   elements of an array that are masked when the array is converted to a
   string, as happens with the print statement.
\end{methoddesc}

\begin{methoddesc}[masked_print_option]{enable}{flag}
   can be used to enable (\var{flag} = 1, default) the use of the display
   string. If disabled (\var{flag} = 0), the conversion to string becomes
   equivalent to \code{str(self.filled())}.
\end{methoddesc}

\begin{methoddesc}[masked_print_option]{enabled}{}
   Returns the state of the display-enabling flag.
\end{methoddesc}


\paragraph*{Example of masked behavior}
\label{sec:numarray.ma:example-mask-behavior}
\begin{verbatim}
>>> from numarray.ma import *
>>> x=arange(5)
>>> x[3] = masked
>>> print x
[0 ,1 ,2 ,-- ,4 ,]
>>> print repr(x)
array(data = 
 [0,1,2,0,4,],
      mask = 
 [0,0,0,1,0,],
      fill_value=[0,])
>>> print x[3]
--
>>> print x[3] + 1.0
--
>>> print masked + x
[-- ,-- ,-- ,-- ,-- ,]
>>> masked_print_option.enable(0)
>>> print x
[0,1,2,0,4,]
>>> print x + masked
[0,0,0,0,0,]
>>> print filled(x+masked, -99)
[-99,-99,-99,-99,-99,]
\end{verbatim}


\begin{classdesc}{masked_unary_function}{f, fill=0, domain=None}
   Given a \index{unary}unary array function \function{f}, give a function
   which when applied to an argument \var{x} returns \function{f} applied to
   the array \code{filled(x, fill)}, with a mask equal to
   \code{mask_or(getmask(x), domain(x))}.
   
   The argument domain therefore should be a callable object that returns true
   where \var{x} is not in the domain of \function{f}. 
\end{classdesc}

The following domains are also supplied as members of module \module{numarray.ma}:
\begin{classdesc}{domain_check_interval}{a, b)(x}
   Returns true where \code{x < a or y > b}.
\end{classdesc}

\begin{classdesc}{domain_tan}{eps}{x}
   This is true where \code{abs(cos (x)) < eps}, that is, a domain suitable for
   the tangent function.
\end{classdesc}

\begin{classdesc}{domain_greater}{v)(x}
   True where \code{x <= v}.
\end{classdesc}

\begin{classdesc}{domain_greater_equal}{v)(x}
   True where x < v.
\end{classdesc}


\begin{classdesc}{masked_binary_function}{f, fillx=0, filly=0}
   Given a binary array function \function{f}, \code{masked_binary_function(f,
      fillx=0, filly=0)} defines a function whose value at \var{x} is
   \code{f(filled(x, fillx), filled (y, filly))} with a resulting mask of
   \code{mask_or(getmask (x), getmask(y))}. The values \var{fillx} and
   \var{filly} must be chosen so that \code{(fillx, filly)} is in the domain of
   \function{f}.
\end{classdesc}

In addition, an instance of
\index{numarray.ma!masked_binary_function@\class{masked_binary_function}}\class{masked_binary_function}
has two methods defined upon it:

\begin{methoddesc}[masked_binary_function]{reduce}{target, axis = 0}
\end{methoddesc}

\begin{methoddesc}[masked_binary_function]{accumulate}{target, axis = 0}
\end{methoddesc}

\begin{methoddesc}[masked_binary_function]{outer}{a, b}
   These methods perform reduction, accumulation, and applying the function in
   an outer-product-like manner, as discussed in the section
   \ref{sec:ufuncs-have-special-methods}.
\end{methoddesc}


\begin{classdesc}{domained_binary_function}{}
   This class exists to implement division-related operations. It is the same
   as \class{masked_binary_function}, except that a new second argument is a
   domain which is used to mask operations that would otherwise cause failure,
   such as dividing by zero. The functions that are created from this class are
   \function{divide}, \function{remainder} (\function{mod}), and
   \function{fmod}.
\end{classdesc}

The following domains are available for use as the domain argument:

\begin{classdesc}{domain_safe_divide}{)(x, y}
   True where \code{absolute(x)*divide_tolerance > absolute(y)}.  As the
   comments in the code say, \emph{better ideas welcome}. The constant
   \index{numarray.ma!divide_tolerance@\constant{divide_tolerance}}\constant{divide_tolerance}
   is set to \constant{1.e-35} in the source and can be changed by editing its
   value in \file{MA.py} and reinstalling. This domain is used for the divide
   operator.
\end{classdesc}


\section{Examples of Using numarray.ma}
\label{sec:numarray.ma:examples-using-ma}


\subsection{Data with a given value representing missing data}
\label{sec:numarray.ma:data-with-given-repr-miss-data}

Suppose we have read a one-dimensional list of elements named \var{x}. We also
know that if any of the values are \constant{1.e20}, they represent missing
data. We want to compute the average value of the data and the vector of
deviations from average.
\begin{verbatim}
>>> from numarray.ma import *
>>> x = array([0.,1.,2.,3.,4.])
>>> x[2] = 1.e20
>>> y = masked_values (x, 1.e20)
>>> print average(y)
2.0
>>> print y-average(y)
[ -2.00000000e+00, -1.00000000e+00,  --,  1.00000000e+00,
        2.00000000e+00,]
\end{verbatim}


\subsection{Filling in the missing data}
\label{sec:numarray.ma:filling-missing-data}

Suppose now that we wish to print that same data, but with the missing values
replaced by the average value.
\begin{verbatim}
>>> print filled (y, average(y))
\end{verbatim}


\subsection{Numerical operations}
\label{sec:numarray.ma:numerical-operations}

We can do numerical operations without worrying about missing values, dividing
by zero, square roots of negative numbers, etc.
\begin{verbatim}
>>> from numarray.ma import *
>>> x=array([1., -1., 3., 4., 5., 6.], mask=[0,0,0,0,1,0])
>>> y=array([1., 2., 0., 4., 5., 6.], mask=[0,0,0,0,0,1])
>>> print sqrt(x/y)
[  1.00000000e+00,  --,  --,  1.00000000e+00, --,  --,]
\end{verbatim}
Note that four values in the result are invalid: one from a negative square
root, one from a divide by zero, and two more where the two arrays \var{x} and
\var{y} had invalid data. Since the result was of a real type, the print
command printed \code{str(filled(sqrt (x/y)))}.



\subsection{Seeing the mask}
\label{sec:numarray.ma:seeing-mask}

There are various ways to see the mask. One is to print it directly, the other
is to convert to the \function{repr} representation, and a third is get the
mask itself.  Use of \function{getmask} is more robust than \code{x.mask()},
since it will work (returning \constant{None}) if \var{x} is a
\module{\numarray} array or list.
\begin{verbatim}
>>> x = arange(10)
>>> x[3:5] = masked
>>> print x
[0 ,1 ,2 ,-- ,-- ,5 ,6 ,7 ,8 ,9 ,]
>>> print repr(x)
*** Masked array, mask present ***
Data:
[0 ,1 ,2 ,-- ,-- ,5 ,6 ,7 ,8 ,9 ,]
Mask (fill value [0,])
[0,0,0,1,1,0,0,0,0,0,]
>>> print getmask(x)
[0,0,0,1,1,0,0,0,0,0,]
\end{verbatim}



\subsection{Filling it your way}
\label{sec:numarray.ma:filling-it-your-way}

If we want to print the data with \constant{-1}'s where the elements are
masked, we use \function{filled}.
\begin{verbatim}
>>> print filled(z, -1)
[ 1.,-1.,-1., 1.,-1.,-1.,]
\end{verbatim}



\subsection{Ignoring extreme values}
\label{sec:numarray.ma:ignore-extreme-values}

Suppose we have an array \var{d} and we wish to compute the average of the
values in \var{d} but ignore any data outside the range -100. to 100.
\begin{verbatim}
v = masked_outside(d, -100., 100.)
print average(v)
\end{verbatim}


\subsection{Averaging an entire multidimensional array}
\label{sec:numarray.ma:averaging-an-entire}

The problem with averaging over an entire array is that the average function
only reduces one dimension at a time. So to average the entire array,
\function{ravel} it first.
\begin{verbatim}
>>> x
*** Masked array, no mask ***
Data:
[[ 0, 1, 2,]
 [ 3, 4, 5,]
 [ 6, 7, 8,]
 [ 9,10,11,]]
>>> average(x)
*** Masked array, no mask ***
Data:
[ 4.5, 5.5, 6.5,]
>>> average(ravel(x))
5.5
\end{verbatim}




%% Local Variables:
%% mode: LaTeX
%% mode: auto-fill
%% fill-column: 79
%% indent-tabs-mode: nil
%% ispell-dictionary: "american"
%% reftex-fref-is-default: nil
%% TeX-auto-save: t
%% TeX-command-default: "pdfeLaTeX"
%% TeX-master: "numarray"
%% TeX-parse-self: t
%% End:

\chapter{Mlab}
\label{cha:mlab}

%begin{latexonly}
\makeatletter
\py@reset
\makeatother
%end{latexonly}
\declaremodule[numarray.mlab]{extension}{numarray.mlab}
\moduleauthor{The numarray team}{numpy-discussion@lists.sourceforge.net}
\modulesynopsis{mlab}

\section{Matlab(tm) compatible functions}
\label{sec:Matlab-compatible-functions}

\begin{quote}
  \module{numarray.mlab} provides a set of Matlab(tm) compatible functions.
\end{quote}

This will hopefully become a complete set of the basic functions available in
Matlab.  The syntax is kept as close to the Matlab syntax as possible.  One
fundamental change is that the first index in Matlab varies the fastest (as in
FORTRAN).  That means that it will usually perform reductions over columns,
whereas with this object the most natural reductions are over rows.  It's
perfectly possible to make this work the way it does in Matlab if that's
desired.

\begin{funcdesc}{mean}{m, axis=0}
   \label{cha:mlab:mean}
   \label{func:mean}
   returns the mean along the axis'th dimension of m.  Note: if m is an integer
   array, the result will be floating point. This was changed in release 10.1;
   previously, a meaningless integer divide was used.
\end{funcdesc}
\begin{funcdesc}{median}{m}
   \label{cha:mlab:median}
   \label{func:median}
   returns a mean of m along the first dimension of m.
\end{funcdesc}
\begin{funcdesc}{min}{m, axis=0}
   \label{cha:mlab:min}
   \label{func:min}
   returns the minimum along the axis'th dimension of m.
\end{funcdesc}
\begin{funcdesc}{msort}{m}
   \label{cha:mlab:msort}
   \label{func:msort}
   returns a sort along the first dimension of m as in MATLAB.
\end{funcdesc}
\begin{funcdesc}{prod}{m, axis=0}
   \label{cha:mlab:prod}
   \label{func:prod}
   returns the product of the elements along the axis'th dimension of m.
\end{funcdesc}
\begin{funcdesc}{ptp}{m, axis = 0}
   \label{cha:mlab:ptp}
   \label{func:ptp}
   returns the maximum - minimum along the axis'th dimension of m.
\end{funcdesc}
\begin{funcdesc}{rand}{d1, ..., dn}
   \label{cha:mlab:rand}
   \label{func:rand}
   returns a matrix of the given dimensions which is initialized to random numbers from a uniform distribution in
the range [0,1).
\end{funcdesc}
\begin{funcdesc}{rot90}{m,k=1}
   \label{cha:mlab:rot90}
   \label{func:rot90}
   returns the matrix found by rotating m by k*90 degrees in the counterclockwise direction.
\end{funcdesc}
\begin{funcdesc}{sinc}{x}
   \label{cha:mlab:sinc}
   \label{func:sinc}
   returns sin(pi*x)
\end{funcdesc}
\begin{funcdesc}{squeeze}{a}
   \label{cha:mlab:squeeze}
   \label{func:squeeze}
   removes any ones from the shape of a
\end{funcdesc}
\begin{funcdesc}{std}{m, axis = 0}
   \label{cha:mlab:std}
   \label{func:std}
   returns the unbiased estimate of the population standard deviation from a
   sample along the axis'th dimension of m. (That is, the denominator for the
   calculation is n-1, not n.) 
\end{funcdesc}
\newpage
\begin{funcdesc}{sum}{m, axis=0}
   \label{cha:mlab:sum}
   \label{func:sum}
   returns the sum of the elements along the axis'th dimension of m.
\end{funcdesc}
\begin{funcdesc}{svd}{m}
   \label{cha:mlab:svd}
   \label{func:svd}
   return the singular value decomposition of m [u,x,v]
\end{funcdesc}
\begin{funcdesc}{trapz}{y,x=None}
   \label{cha:mlab:trapz}
   \label{func:trapz}
   integrates y = f(x) using the trapezoidal rule
\end{funcdesc}
\begin{funcdesc}{tri}{N, M=N, k=0, typecode=None}
   \label{cha:mlab:tri}
   \label{func:tri}
   returns a N-by-M matrix where all the diagonals starting from lower left corner up to the k-th are all ones.
\end{funcdesc}
\begin{funcdesc}{tril}{m,k=0}
   \label{cha:mlab:tril}
   \label{func:tril}
   returns the elements on and below the k-th diagonal of m. k=0 is the main
   diagonal, \begin{math} k > 0\end{math} is above and \begin{math}k < 0
   \end{math} is below the main diagonal.
\end{funcdesc}

\begin{funcdesc}{triu}{m,k=0}
   \label{sec:mlab-functions:triu}
   \label{func:triu}
   returns the elements on and above the k-th diagonal of m. k=0 is the main
   diagonal, \begin{math}k > 0\end{math} is above and \begin{math}k <
   0\end{math} is below the main diagonal.
\end{funcdesc}

%% Local Variables:
%% mode: LaTeX
%% mode: auto-fill
%% fill-column: 79
%% indent-tabs-mode: nil
%% ispell-dictionary: "american"
%% reftex-fref-is-default: nil
%% TeX-auto-save: t
%% TeX-command-default: "pdfeLaTeX"
%% TeX-master: "numarray"
%% TeX-parse-self: t
%% End:

\chapter{Random Numbers}
\label{cha:random-array}

%begin{latexonly}
\makeatletter \py@reset \makeatother
%end{latexonly}
\declaremodule[numarray.randomarray]{extension}{numarray.random_array}
\moduleauthor{The numarray team}{numpy-discussion@lists.sourceforge.net}
\modulesynopsis{Random Numbers}

\begin{quote}
  The \module{numarray.random_array} module (in conjunction with the
  \module{numarray.random_array.ranlib} submodule) provides a high-level
  interface to ranlib, which provides a good quality C implementation of a
  random-number generator.
\end{quote}

\section{General functions}
\label{sec:RA:general-functions}

\begin{funcdesc}{seed}{x=0, y=0}
The \function{seed} function takes two integers and sets the two seeds of the
random number generator to those values. If the default values of 0 are used
for both \var{x} and \var{y}, then a seed is generated from the current time,
providing a pseudo-random seed.
\end{funcdesc}

\begin{funcdesc}{get_seed}{}
This function returns the two seeds used by the current random-number
generator. It is most often used to find out what seeds the \function{seed}
function chose at the last iteration.  \remark{Are there any thread-safety
issues?}
\end{funcdesc}

\begin{funcdesc}{random}{shape=[]}
   The \function{random} function takes a \var{shape}, and returns an array of
   \class{Float} numbers between 0.0 and 1.0.  Neither 0.0 nor 1.0 is ever
   returned by this function.  The array is filled from the generator following
   the canonical array organization.
   
   If no argument is specified, the function returns a single floating point
   number, not an array.
   
   \note{See discussion of the \member{flat} attribute in section
      \ref{mem:numarray:flat}.}
\end{funcdesc}

\begin{funcdesc}{uniform}{minimum, maximum, shape=[]}
   The \function{uniform} function returns an array of the specified
   \var{shape} and containing \class{Float} random numbers strictly between
   \var{minimum} and \var{maximum}.
   
   The \var{minimum} and \var{maximum} arguments can be arrays. If this is the
   case, and the output \var{shape} is specified, \var{minimum} and
   \var{maximum} are broadcasted if their dimensions are not equal to
   \var{shape}. If \var{shape} is not specified, the shape of the output is
   equal to the shape of \var{minimum} and \var{maximum} after broadcasting.
   
   If no \var{shape} is specified, and \var{minimum} and \var{maximum} are
   scalars, a single value is returned.
\end{funcdesc}

\begin{funcdesc}{randint}{minimum, maximum, shape=[]}
   The \function{randint} function returns an array of the specified
   \var{shape} and containing random (standard) integers greater than or equal
   to \var{minimum} and strictly less than \var{maximum}. 
   
   The \var{minimum} and \var{maximum} arguments can be arrays. If this is the
   case, and the output \var{shape} is specified, \var{minimum} and
   \var{maximum} are broadcasted if their dimensions are not equal to
   \var{shape}. If \var{shape} is not specified, the shape of the output is
   equal to the shape of \var{minimum} and \var{maximum} after broadcasting.
   
   If no \var{shape} is specified, and \var{minimum} and \var{maximum} are
   scalars, a single value is returned.
\end{funcdesc}

\begin{funcdesc}{permutation}{n}
   The \function{permutation} function returns an array of the integers between
   \code{0} and \code{\var{n}-1}, in an array of shape \code{(n,)} with its
   elements randomly permuted.
\end{funcdesc}


\section{Special random number distributions}
\label{sec:RA:special-distribution}



\subsection{Random floating point number distributions}
\label{sec:RA:float-distribution}

\begin{funcdesc}{beta}{a, b, shape=[]}
   The \function{beta} function returns an array of the specified shape that
   contains \class{Float} numbers $\beta$-distributed with $\alpha$-parameter
   \var{a} and $\beta$-parameter \var{b}. 
   
   The \var{a} and \var{b} arguments can be arrays. If this is the case, and
   the output \var{shape} is specified, \var{a} and \var{b} are broadcasted if
   their dimensions are not equal to \var{shape}. If \var{shape} is not
   specified, the shape of the output is equal to the shape of \var{a} and
   \var{b} after broadcasting.
   
   If no \var{shape} is specified, and \var{a} and \var{b} are
   scalars, a single value is returned.
\end{funcdesc}

\begin{funcdesc}{chi_square}{df, shape=[]}
   The \function{chi_square} function returns an array of the specified
   \var{shape} that contains \class{Float} numbers with the
   $\chi^2$-distribution with \var{df} degrees of freedom.
   
   The \var{df} argument can be an array. If this is the case, and the output
   \var{shape} is specified, \var{df} is broadcasted if its dimensions are not
   equal to \var{shape}. If \var{shape} is not specified, the shape of the
   output is equal to the shape of \var{df}.
   
   If no \var{shape} is specified, and \var{df} is a scalar, a single value is
   returned.
\end{funcdesc}

\begin{funcdesc}{exponential}{mean, shape=[]}
   The \function{exponential} function returns an array of the specified
   \var{shape} that contains \class{Float} numbers exponentially distributed
   with the specified \var{mean}. 
   
   The \var{mean} argument can be an array. If this is the case, and the output
   \var{shape} is specified, \var{mean} is broadcasted if its dimensions are
   not equal to \var{shape}. If \var{shape} is not specified, the shape of the
   output is equal to the shape of \var{mean}.
   
   If no \var{shape} is specified, and \var{mean} is a scalar, a single value
   is returned.
\end{funcdesc}

\begin{funcdesc}{F}{dfn, dfd, shape=[]}
  The \function{F} function returns an array of the specified \var{shape} that
  contains \class{Float} numbers with the F-distribution with \var{dfn} degrees
  of freedom in the numerator and \var{dfd} degrees of freedom in the
  denominator.
   
  The \var{dfn} and \var{dfd} arguments can be arrays. If this is the case, and
  the output \var{shape} is specified, \var{dfn} and \var{dfd} are broadcasted
  if their dimensions are not equal to \var{shape}. If \var{shape} is not
  specified, the shape of the output is equal to the shape of \var{dfn} and
  \var{dfd} after broadcasting.
   
  If no \var{shape} is specified, and \var{dfn} and \var{dfd} are scalars, a
  single value is returned.
\end{funcdesc}

\begin{funcdesc}{gamma}{a, r, shape=[]}
   The \function{gamma} function returns an array of the specified \var{shape}
   that contains \class{Float} numbers $\beta$-distributed with location
   parameter \var{a} and distribution shape parameter \var{r}.
   
   The \var{a} and \var{r} arguments can be arrays. If this is the case, and
   the output \var{shape} is specified, \var{a} and \var{r} are broadcasted if
   their dimensions are not equal to \var{shape}. If \var{shape} is not
   specified, the shape of the output is equal to the shape of \var{a} and
   \var{r} after broadcasting.
   
   If no \var{shape} is specified, and \var{a} and \var{r} are scalars, a
   single value is returned.
\end{funcdesc}

\begin{funcdesc}{multivariate_normal}{mean, cov, shape=[]}
   The multivariate_normal function takes a one dimensional array argument
   \var{mean} and a two dimensional array argument \var{cov}. Suppose
   the shape of \var{mean} is \code{(n,)}. Then the shape of \var{cov}
   must be \code{(n,n)}. The function returns an array of \class{Float}s.
   
   The effect of the \var{shape} parameter is:
   \begin{itemize}
   \item If no \var{shape} is specified, then an array with shape \code{(n,)}
      is returned containing a vector of numbers with a multivariate normal
      distribution with the specified mean and covariance.
   \item If \var{shape} is specified, then an array of such vectors is
      returned.  The shape of the output is \code{shape.append((n,))}. The
      leading indices into the output array select a multivariate normal from
      the array. The final index selects one number from within the
      multivariate normal.
 \end{itemize}
 In either case, the behavior of \function{multivariate_normal} is undefined if
 \var{cov} is not symmetric and positive definite.
\end{funcdesc}

\begin{funcdesc}{normal}{mean, std, shape=[]}
   The \function{normal} function returns an array of the specified \var{shape}
   that contains \class{Float} numbers normally distributed with the specified
   \var{mean} and standard deviation \var{std}. 
   
   The \var{mean} and \var{std} arguments can be arrays. If this is the
   case, and the output \var{shape} is specified, \var{mean} and \var{std}
   are broadcasted if their dimensions are not equal to \var{shape}. If
   \var{shape} is not specified, the shape of the output is equal to the shape
   of \var{mean} and \var{std} after broadcasting.
   
   If no \var{shape} is specified, and \var{mean} and \var{std} are scalars, a
   single value is returned.
\end{funcdesc}

\begin{funcdesc}{noncentral_chi_square}{df, nonc, shape=[]}
   The \function{noncentral_chi_square} function returns an array of the
   specified \var{shape} that contains \class{Float} numbers with
   the$\chi^2$-distribution with \var{df} degrees of freedom and noncentrality
   parameter \var{nconc}.
   
   The \var{df} and \var{nonc} arguments can be arrays. If this is the case,
   and the output \var{shape} is specified, \var{df} and \var{nonc} are
   broadcasted if their dimensions are not equal to \var{shape}. If \var{shape}
   is not specified, the shape of the output is equal to the shape of \var{df}
   and \var{nonc} after broadcasting.
   
   If no \var{shape} is specified, and \var{df} and \var{nonc} are scalars, a
   single value is returned.
\end{funcdesc}

\begin{funcdesc}{noncentral_F}{dfn, dfd, nconc, shape=[]}
   The \function{noncentral_F} function returns an array of the specified
   \var{shape} that contains \class{Float} numbers with the F-distribution with
   \var{dfn} degrees of freedom in the numerator, \var{dfd} degrees of freedom
   in the denominator, and noncentrality parameter \var{nconc}.
   
   The \var{dfn}, \var{dfd} and \var{nonc} arguments can be arrays. If this is
   the case, and the output \var{shape} is specified, \var{dfn}, \var{dfd} and
   \var{nonc} are broadcasted if their dimensions are not equal to \var{shape}.
   If \var{shape} is not specified, the shape of the output is equal to the
   shape of \var{dfn}, \var{dfd} and \var{nonc} after broadcasting.
   
   If no \var{shape} is specified, and \var{dfn}, \var{dfd} and \var{nonc} are
   scalars, a single value is returned.
\end{funcdesc}

\begin{funcdesc}{standard_normal}{shape=[]}
   The \function{standard_normal} function returns an array of the specified
   \var{shape} that contains \class{Float} numbers normally (Gaussian)
   distributed with mean zero and variance and standard deviation one. 
   
   If no \var{shape} is specified, a single number is returned.
\end{funcdesc}

\begin{funcdesc}{F}{dfn, dfd, shape=[]}
Returns array of F distributed random numbers with \var{dfn} degrees of freedom
in the numerator and \var{dfd} degrees of freedom in the denominator.
\end{funcdesc}

\begin{funcdesc}{noncentral_F}{dfn, dfd, nconc, shape=[]}
 Returns array of noncentral F distributed random numbers with dfn degrees of
 freedom in the numerator and dfd degrees of freedom in the denominator, and
 noncentrality parameter nconc.
\end{funcdesc}



\subsection{Random integer number distributions }
\label{sec:RA:int-distributions}

\begin{funcdesc}{binomial}{trials, p, shape=[]}
   The \function{binomial} function returns an array with the specified
   \var{shape} that contains \class{Integer} numbers with the binomial
   distribution with \var{trials} and event probability \var{p}. In other
   words, each value in the returned array is the number of times an event with
   probability \var{p} occurred within \var{trials} repeated trials.
   
   The \var{trials} and \var{p} arguments can be arrays. If this is the
   case, and the output \var{shape} is specified, \var{trials} and \var{p}
   are broadcasted if their dimensions are not equal to \var{shape}. If
   \var{shape} is not specified, the shape of the output is equal to the shape
   of \var{trials} and \var{p} after broadcasting.
   
   If no \var{shape} is specified, and \var{trials} and \var{p} are scalars,
   a single value is returned.
\end{funcdesc}

\begin{funcdesc}{negative_binomial}{trials, p, shape=[]}
  The \function{negative_binomial} function returns an array with the specified
  \var{shape} that contains \class{Integer} numbers with the negative binomial
  distribution with \var{trials} and event probability \var{p}.
   
  The \var{trials} and \var{p} arguments can be arrays. If this is the case,
  and the output \var{shape} is specified, \var{trials} and \var{p} are
  broadcasted if their dimensions are not equal to \var{shape}. If \var{shape}
  is not specified, the shape of the output is equal to the shape of
  \var{trials} and \var{p} after broadcasting.
   
   If no \var{shape} is specified, and \var{trials} and \var{p} are scalars,
   a single value is returned.
\end{funcdesc}

\begin{funcdesc}{multinomial}{trials, probs, shape=[]}
   The \function{multinomial} function returns an array with that contains
   integer numbers with the multinomial distribution with \var{trials} and
   event probabilities given in \var{probs}.  \var{probs} must be a one
   dimensional array.  There are \code{len(probs)+1} events. \code{probs[i]} is
   the probability of the i-th event for \code{0<=i<len(probs)}. The
   probability of event \code{len(probs)} is \code{1.-Numeric.sum(prob)}.
   
   The function returns an integer array of shape
   \code{shape~+~(len(probs)+1,)}.  If \var{shape} is not specified this is one
   multinomially distributed vector of shape \code{(len(prob)+1,)}.  Otherwise
   each \code{returnarray[i,j,...,:]} is an integer array of shape
   \code{(len(prob)+1,)} containing one multinomially distributed vector.
\end{funcdesc}

\begin{funcdesc}{poisson}{mean, shape=[]}
   The \function{poisson} function returns an array with the specified shape
   that contains \class{Integer} numbers with the Poisson distribution with the
   specified \var{mean}.
   
   The \var{mean} argument can be an array. If this is the case, and the output
   \var{shape} is specified, \var{mean} is broadcasted if its dimensions are
   not equal to \var{shape}. If \var{shape} is not specified, the shape of the
   output is equal to the shape of \var{mean}.
   
   If no \var{shape} is specified, and \var{mean} is a scalar, a single value
   is returned.
\end{funcdesc}



\section{Examples}
\label{sec:examples}

Some example uses of the \module{numarray.random_array} module. \note{Naturally the exact
   output of running these examples will be different each time!} \remark{Make
   sure these examples are correct!}
\begin{verbatim}
>>> from numarray.random_array import *
>>> seed() # Set seed based on current time
>>> print get_seed() # Find out what seeds were used
(897800491, 192000)
>>> print random()
0.0528018975065
>>> print random((5,2))
[[ 0.14833829 0.99031458]
[ 0.7526806 0.09601787]
[ 0.1895229 0.97674777]
[ 0.46134511 0.25420982]
[ 0.66132009 0.24864472]]
>>> print uniform(-1,1,(10,))
[ 0.72168852 -0.75374185 -0.73590945 0.50488248 -0.74462822 0.09293685
-0.65898308 0.9718067 -0.03252475 0.99611011]
>>> print randint(0,100, (12,))
[28 5 96 19 1 32 69 40 56 69 53 44]
>>> print permutation(10)
[4 2 8 9 1 7 3 6 5 0]
>>> seed(897800491, 192000) # resetting the same seeds
>>> print random() # yields the same numbers
0.0528018975065
\end{verbatim}
Most of the functions in this package take zero or more distribution specific
parameters plus an optional \var{shape} parameter. The \var{shape} parameter
gives the shape of the output array:
\begin{verbatim}
>>> from numarray.random_array import *
>>> print standard_normal()
-0.435568600893
>>> print standard_normal(5)
[-1.36134553 0.78617644 -0.45038718 0.18508556 0.05941355]
>>> print standard_normal((5,2))
[[ 1.33448863 -0.10125473]
[ 0.66838062 0.24691346]
[-0.95092064 0.94168913]
[-0.23919107 1.89288616]
[ 0.87651485 0.96400219]]
>>> print normal(7., 4., (5,2)) #mean=7, std. dev.=4
[[ 2.66997623 11.65832615]
[ 6.73916003 6.58162862]
[ 8.47180378 4.30354905]
[ 1.35531998 -2.80886841]
[ 7.07408469 11.39024973]]
>>> print exponential(10., 5) #mean=10
[ 18.03347754 7.11702306 9.8587961 32.49231603 28.55408891]
>>> print beta(3.1, 9.1, 5) # alpha=3.1, beta=9.1
[ 0.1175056 0.17504358 0.3517828 0.06965593 0.43898219]
>>> print chi_square(7, 5) # 7 degrees of freedom (dfs)
[ 11.99046516 3.00741053 4.72235727 6.17056274 8.50756836]
>>> print noncentral_chi_square(7, 3, 5) # 7 dfs, noncentrality 3
[ 18.28332138 4.07550335 16.0425396 9.51192093 9.80156231]
>>> F(5, 7, 5) # 5 and 7 dfs
array([ 0.24693671, 3.76726145, 0.66883826, 0.59169068, 1.90763224])
>>> noncentral_F(5, 7, 3., 5) # 5 and 7 dfs, noncentrality 3
array([ 1.17992553, 0.7500126 , 0.77389943, 9.26798989, 1.35719634])
>>> binomial(32, .5, 5) # 32 trials, prob of an event = .5
array([12, 20, 21, 19, 17])
>>> negative_binomial(32, .5, 5) # 32 trials: prob of an event = .5
array([21, 38, 29, 32, 36])
\end{verbatim}
Two functions that return generate multivariate random numbers (that is, random
vectors with some known relationship between the elements of each vector,
defined by the distribution). They are \function{multivariate_normal} and
\function{multinomial}. For these two functions, the lengths of the leading
axes of the output may be specified. The length of the last axis is determined
by the length of some other parameter.
\begin{verbatim}
>>> multivariate_normal([1,2], [[1,2],[2,1]], [2,3])
array([[[ 0.14157988, 1.46232224],
[-1.11820295, -0.82796288],
[ 1.35251635, -0.2575901 ]],
[[-0.61142141, 1.0230465 ],
[-1.08280948, -0.55567217],
[ 2.49873002, 3.28136372]]])
>>> x = multivariate_normal([10,100], [[1,2],[2,1]], 10000)
>>> x_mean = sum(x)/10000
>>> print x_mean
[ 9.98599893 100.00032416]
>>> x_minus_mean = x - x_mean
>>> cov = matrixmultiply(transpose(x_minus_mean), x_minus_mean) / 9999.
>>> cov
array([[ 2.01737122, 1.00474408],
[ 1.00474408, 2.0009806 ]])
\end{verbatim}
The a priori probabilities for a multinomial distribution must sum to one. The
prior probability argument to \function{multinomial} doesn't give the prior
probability of the last event: it is computed to be one minus the sum of the
others.
\begin{verbatim}
>>> multinomial(16, [.1, .4, .2]) # prior probabilities [.1, .4, .2, .3]
array([2, 7, 1, 6])
>>> multinomial(16, [.1, .4, .2], [2,3]) # output shape [2,3,4]
array([[[ 1, 9, 1, 5],
[ 0, 10, 3, 3],
[ 4, 9, 3, 0]],
[[ 1, 6, 1, 8],
[ 3, 4, 5, 4],
[ 1, 5, 2, 8]]])
\end{verbatim}
Many of the functions accept arrays or sequences for the distribution
arguments. If no \var{shape} argument is given, then the shape of the output is
determined by the shape of the parameter argument. For instance:
\begin{verbatim}
>>> beta([5.0, 50.0], [10.0, 100.])
array([ 0.54379648,  0.35352072])
\end{verbatim}
Broadcasting rules apply if two or more arguments are arrays:
\begin{verbatim}
>>> beta([5.0, 50.0], [[10.0, 100.], [20.0, 200.0]])
array([[ 0.30204576,  0.32154009],
       [ 0.10851908,  0.19207685]])
\end{verbatim}
The \var{shape} argument can still be used to specify the output shape. Any
array argument will be broadcasted to have the given shape:
\begin{verbatim}
>>> beta(5.0, [10.0, 100.0], shape = (3, 2))
array([[ 0.49521708,  0.02218186],
       [ 0.21000148,  0.04366644],
       [ 0.43169656,  0.05285903]])
\end{verbatim}
%% Local Variables:
%% mode: LaTeX
%% mode: auto-fill
%% fill-column: 79
%% indent-tabs-mode: nil
%% ispell-dictionary: "american"
%% reftex-fref-is-default: nil
%% TeX-auto-save: t
%% TeX-command-default: "pdfeLaTeX"
%% TeX-master: "numarray"
%% TeX-parse-self: t
%% End:

\chapter{Multi-dimensional image processing}
\label{cha:ndimage}

%begin{latexonly}
\makeatletter
\py@reset
\makeatother
%end{latexonly}
\declaremodule[numarray.ndimage]{extension}{numarray.nd_image}
\moduleauthor{Peter Verveer}{verveer@users.sourceforge.net}
\modulesynopsis{Multidimensional image analysis functions}

\begin{quote}
  The \module{numarray.nd\_image} module provides functions for
  multidimensional image analysis.
\end{quote}

\section{Introduction}
Image processing and analysis are generally seen as operations on
two-dimensional arrays of values. There are however a number of fields
where images of higher dimensionality must be analyzed. Good examples
of these are medical imaging and biological imaging. \module{numarray}
is suited very well for this type of applications due its inherent
multi-dimensional nature. The \module{numarray.nd\_image} packages
provides a number of general image processing and analysis functions
that are designed to operate with arrays of arbitrary dimensionality.
The packages currently includes functions for linear and non-linear
filtering, binary morphology, B-spline interpolation, and object
measurements.

\section{Properties shared by all functions}
All functions share some common properties. Notably, all functions allow the 
specification of an output array with the \var{output} argument. With this 
argument you can specify an array that will be changed in-place with the 
result with the operation. In this case the result is not returned. Usually, 
using the \var{output} argument is more efficient, since an existing array 
is used to store the result.

The type of arrays returned is dependent on the type of operation, but it is  in most cases equal to the type of the input. If, however, the \var{output} argument is used, the type of the result is equal to the type of the specified output argument. If no output argument is given, it is still possible to specify what the result of the output should be. This is done by simply assigning the desired numarray type object to the output argument. For example:
\begin{verbatim}
>>> print correlate(arange(10), [1, 2.5])
[ 0  2  6  9 13 16 20 23 27 30]
>>> print correlate(arange(10), [1, 2.5], output = Float64)
[  0.    2.5   6.    9.5  13.   16.5  20.   23.5  27.   30.5]                   
\end{verbatim}
\note{In previous versions of \module{numarray.nd\_image}, some functions accepted the \var{output_type} argument to achieve the same effect. This argument is still supported, but its use will generate an deprecation warning. In a future version all instances of this argument will be removed. The preferred way to specify an output type, is by using the \var{output} argument, either by specifying an output array of the desired type, or by specifying the type of the output that is to be returned.}
%
\section{Filter functions}
\label{sec:ndimage:filter-functions}
The functions described in this section all perform some type of spatial
filtering of the the input array: the elements in the output are some 
function of the values in the neighborhood of the corresponding input 
element. We refer to this neighborhood of elements as the filter kernel, 
which is often rectangular in shape but may also have an arbitrary 
footprint. Many of the functions described below allow you to define the 
footprint of the kernel, by passing a mask through the \var{footprint} 
parameter. For example a cross shaped kernel can be defined as follows:
\begin{verbatim}
>>> footprint = array([[0,1,0],[1,1,1],[0,1,0]])
>>> print footprint
[[0 1 0]
 [1 1 1]
 [0 1 0]]
\end{verbatim}
Usually the origin of the kernel is at the center calculated by dividing 
the dimensions of the kernel shape by two.  For instance, the origin of a
one-dimensional kernel of length three is at the second element. Take for
example the correlation of a one-dimensional array with a filter of
length 3 consisting of ones:
\begin{verbatim}
>>> a = [0, 0, 0, 1, 0, 0, 0]
>>> correlate1d(a, [1, 1, 1])
[0 0 1 1 1 0 0]
\end{verbatim}
Sometimes it is convenient to choose a different origin for the kernel. For
this reason most functions support the \var{origin} parameter which gives 
the origin of the filter relative to its center. For example:
\begin{verbatim}
>>> a = [0, 0, 0, 1, 0, 0, 0]
>>> print correlate1d(a, [1, 1, 1], origin = -1)
[0 1 1 1 0 0 0]
\end{verbatim}
The effect is a shift of the result towards the left. This feature will not 
be needed very often, but it may be useful especially for filters that have 
an even size.  A good example is the calculation of backward and forward
differences:
\begin{verbatim}
>>> a = [0, 0, 1, 1, 1, 0, 0]
>>> print correlate1d(a, [-1, 1])              ## backward difference
[ 0  0  1  0  0 -1  0]
>>> print correlate1d(a, [-1, 1], origin = -1) ## forward difference
[ 0  1  0  0 -1  0  0]
\end{verbatim}
We could also have calculated the  forward difference as follows:
\begin{verbatim}
>>> print correlate1d(a, [0, -1, 1])
[ 0  1  0  0 -1  0  0]
\end{verbatim}
however, using the origin parameter instead of a larger kernel is more
efficient. For multi-dimensional kernels \var{origin} can be a number, in 
which case the origin is assumed to be equal along all axes, or a sequence  
giving the origin along each axis.

Since the output elements are a function of elements in the neighborhood of 
the input elements, the borders of the array need to be dealt with 
appropriately by providing the values outside the borders. This is done by 
assuming that the arrays are extended beyond their boundaries according 
certain boundary conditions. In the functions described below, the boundary 
conditions can be selected using the \var{mode} parameter which must be a 
string with the name of the boundary condition.  Following boundary 
conditions are currently supported:
\begin{tableiii}{l|l|l}{constant}{Boundary condition}{Description}{Example}
  \lineiii{"nearest"}{Use the value at the boundary}
  {\constant{[1 2 3]->[1 1 2 3 3]}}
  \lineiii{"wrap"}{Periodically replicate the array}
  {\constant{[1 2 3]->[3 1 2 3 1]}}
  \lineiii{"reflect"}{Reflect the array at the boundary}
  {\constant{[1 2 3]->[1 1 2 3 3]}}
  \lineiii{"constant"}{Use a constant value, default value is 0.0}
  {\constant{[1 2 3]->[0 1 2 3 0]}}
\end{tableiii}
The \constant{"constant"} mode is special since it needs an additional
parameter to specify the constant value that should be used.

\note{The easiest way to implement such boundary conditions would be to 
copy the data to a larger array and extend the data at the borders 
according to the boundary conditions. For large arrays and large filter 
kernels, this would be very memory consuming, and the functions described 
below therefore use a different approach that does not require allocating 
large temporary buffers.}

\subsection{Correlation and convolution}

\begin{funcdesc}{correlate1d}{input, weights, axis=-1, output=None, 
    mode='reflect', cval=0.0, origin=0, output_type=None} The
  \function{correlate1d} function calculates a one-dimensional correlation
  along the given axis. The lines of the array along the given axis are
  correlated with the given \var{weights}. The \var{weights} parameter must 
  be a one-dimensional sequences of numbers.
\end{funcdesc}

\begin{funcdesc}{correlate}{input, weights, output=None, mode='reflect', 
    cval=0.0, origin=0, output_type=None} The function \function{correlate}
  implements multi-dimensional correlation of the input array with a given
  kernel.
\end{funcdesc}

\begin{funcdesc}{convolve1d}{input, weights, axis=-1, output=None, 
    mode='reflect', cval=0.0, origin=0, output_type=None} The
  \function{convolve1d} function calculates a one-dimensional convolution 
  along the given axis. The lines of the array along the given axis are 
  convoluted with the given \var{weights}. The \var{weights} parameter must 
  be a one-dimensional sequences of numbers.
  
  \note{A convolution is essentially a correlation after mirroring the 
  kernel. As a result, the \var{origin} parameter behaves differently than 
  in the case of a correlation: the result is shifted in the opposite 
  directions.}
\end{funcdesc}

\begin{funcdesc}{convolve}{input, weights, output=None, mode='reflect', 
    cval=0.0, origin=0, output_type=None} The function \function{convolve}
  implements multi-dimensional convolution of the input array with a given
  kernel.
  
  \note{A convolution is essentially a correlation after mirroring the 
  kernel. As a result, the \var{origin} parameter behaves differently than 
  in the case of a correlation: the results is shifted in the opposite 
  direction.}
\end{funcdesc}

\subsection{Smoothing filters}
\label{sec:ndimage:filter-functions:smoothing}

\begin{funcdesc}{gaussian_filter1d}{input, sigma, axis=-1, order=0, 
    output=None, mode='reflect', cval=0.0, output_type=None} The
  \function{gaussian_filter1d} function implements a one-dimensional 
  Gaussian
  filter. The standard-deviation of the Gaussian filter is passed through 
  the parameter \var{sigma}. Setting \var{order}=0 corresponds to 
  convolution with a Gaussian kernel.  An order of 1, 2, or 3 corresponds 
  to convolution with the first, second or third derivatives of a Gaussian. 
  Higher order derivatives are not implemented.
\end{funcdesc}

\begin{funcdesc}{gaussian_filter}{input, sigma, order=0, output=None, 
  mode='reflect', cval=0.0, output_type=None} The 
  \function{gaussian_filter} function implements a multi-dimensional 
  Gaussian filter. The standard-deviations of the Gaussian filter along 
  each axis are passed through the parameter \var{sigma} as a sequence or 
  numbers.  If \var{sigma} is not a sequence but a single number, the 
  standard deviation of the filter is equal along all directions. The 
  order of the filter can be specified separately for each axis. An order 
  of 0 corresponds to convolution with a Gaussian kernel. An order of 1, 
  2, or 3 corresponds to convolution with the first, second or
  third derivatives of a Gaussian. Higher order derivatives are not
  implemented. The \var{order} parameter must be a number, to specify the 
  same order for all axes, or a sequence of numbers to specify a different 
  order for each axis.
  
  \note{The multi-dimensional filter is implemented as a sequence of
    one-dimensional Gaussian filters. The intermediate arrays are stored in 
    the same data type as the output.  Therefore, for output types with a 
    lower precision, the results may be imprecise because intermediate 
    results may be stored with insufficient precision. This can be 
    prevented by specifying a more precise output type.}
\end{funcdesc}

\begin{funcdesc}{uniform_filter1d}{input, size, axis=-1, output=None,
    mode='reflect', cval=0.0, origin=0, output_type=None} The
  \function{uniform_filter1d} function calculates a one-dimensional uniform
  filter of the given \var{size} along the given axis.
\end{funcdesc}

\begin{funcdesc}{uniform_filter}{input, size, output=None, mode='reflect', 
    cval=0.0, origin=0, output_type=None} The \function{uniform_filter}
  implements a multi-dimensional uniform filter.  The sizes of the uniform 
  filter are given for each axis as a sequence of integers by the 
  \var{size} parameter. If \var{size} is not a sequence, but a single 
  number, the sizes along all axis are assumed to be equal.
    
  \note{The multi-dimensional filter is implemented as a sequence of
    one-dimensional uniform filters. The intermediate arrays are stored in 
    the same data type as the output. Therefore, for output types with a 
    lower precision, the results may be imprecise because intermediate 
    results may be stored with insufficient precision. This can be 
    prevented by specifying a
    more precise output type.}
  \end{funcdesc}

\subsection{Filters based on order statistics}

\begin{funcdesc}{minimum_filter1d}{input, size, axis=-1, output=None, 
    mode='reflect', cval=0.0, origin=0} The \function{minimum_filter1d}
  function calculates a one-dimensional minimum filter of given \var{size}
  along the given axis.
\end{funcdesc}

\begin{funcdesc}{maximum_filter1d}{input, size, axis=-1, output=None, 
    mode='reflect', cval=0.0, origin=0} The \function{maximum_filter1d}
  function calculates a one-dimensional maximum filter of given \var{size}
  along the given axis.
\end{funcdesc}

\begin{funcdesc}{minimum_filter}{input,  size=None, footprint=None, 
    output=None, mode='reflect', cval=0.0, origin=0} The
  \function{minimum_filter} function calculates a multi-dimensional minimum
  filter. Either the sizes of a rectangular kernel or the footprint of the
  kernel must be provided. The \var{size} parameter, if provided, must be a
  sequence of sizes or a single number in which case the size of the filter 
  is assumed to be equal along each axis. The \var{footprint}, if provided, 
  must be an array that defines the shape of the kernel by its non-zero 
  elements.
\end{funcdesc}

\begin{funcdesc}{maximum_filter}{input,  size=None, footprint=None, 
    output=None, mode='reflect', cval=0.0, origin=0} The
  \function{maximum_filter} function calculates a multi-dimensional maximum
  filter. Either the sizes of a rectangular kernel or the footprint of the
  kernel must be provided. The \var{size} parameter, if provided, must be a
  sequence of sizes or a single number in which case the size of the filter 
  is assumed to be equal along each axis. The \var{footprint}, if provided, 
  must be an array that defines the shape of the kernel by its non-zero 
  elements.
\end{funcdesc}

\begin{funcdesc}{rank_filter}{input, rank, size=None, footprint=None,
  output=None, mode='reflect', cval=0.0, origin=0} The 
  \function{rank_filter}
  function calculates a multi-dimensional rank filter.  The \var{rank} may 
  be less then zero, i.e., \var{rank}=-1 indicates the largest element. 
  Either the sizes of a rectangular kernel or the footprint of the kernel 
  must be provided. The \var{size} parameter, if provided, must be a 
  sequence of sizes or a single number in which case the size of the filter 
  is assumed to be equal along each axis. The \var{footprint}, if provided, 
  must be an array that defines the shape of the kernel by its non-zero 
  elements.
\end{funcdesc}

\begin{funcdesc}{percentile_filter}{input, percentile, size=None, 
  footprint=None, output=None, mode='reflect', cval=0.0, origin=0} The
  \function{percentile_filter} function calculates a multi-dimensional
  percentile filter.  The \var{percentile} may be less then zero, i.e.,
  \var{percentile}=-20 equals \var{percentile}=80. Either the sizes of a 
  rectangular kernel or the footprint of the kernel must be provided. The 
  \var{size} parameter, if provided, must be a sequence of sizes or a 
  single number in which case the size of the filter is assumed to be equal 
  along each axis. The \var{footprint}, if provided, must be an array that 
  defines the shape of the kernel by its non-zero elements.
\end{funcdesc}

\begin{funcdesc}{median_filter}{input, size=None, footprint=None, 
  output=None, mode='reflect', cval=0.0, origin=0} The 
  \function{median_filter} function calculates a multi-dimensional median 
  filter. Either the sizes of a rectangular kernel or the footprint of the 
  kernel must be provided. The \var{size} parameter, if provided, must be a 
  sequence of sizes or a single number in which case the size of the filter 
  is assumed to be equal along each axis. The \var{footprint} if provided, 
  must be an array that defines the shape of the kernel by its non-zero 
  elements.
\end{funcdesc}

\subsection{Derivatives}

Derivative filters can be constructed in several ways. The function
\function{gaussian_filter1d} described in section
\ref{sec:ndimage:filter-functions:smoothing} can be used to calculate
derivatives along a given axis using the \var{order} parameter. Other
derivative filters are the Prewitt and Sobel filters:

\begin{funcdesc}{prewitt}{input, axis=-1, output=None, mode='reflect', 
  cval=0.0} The \function{prewitt} function calculates a derivative along 
  the given axis.
\end{funcdesc}

\begin{funcdesc}{sobel}{input, axis=-1, output=None, mode='reflect', 
  cval=0.0} The \function{sobel} function calculates a derivative along 
  the given axis.
\end{funcdesc}

The Laplace filter is calculated by the sum of the second derivatives along 
all axes. Thus, different Laplace filters can be constructed using 
different second derivative functions. Therefore we provide a general 
function that takes a function argument to calculate the second derivative 
along a given direction and to construct the Laplace filter:

\begin{funcdesc}{generic_laplace}{input, derivative2, output=None,
  mode='reflect', cval=0.0, output_type=None, extra_arguments = (), 
  extra_keywords = {}} The function 
  \function{generic_laplace} calculates a laplace filter using the
  function passed through \var{derivative2} to calculate second 
  derivatives. The function \function{derivative2} should have the 
  following signature:

  \function{derivative2(input, axis, output, mode, cval, *extra_arguments, **extra_keywords)}
  
  It should calculate the second derivative along the dimension \var{axis}. 
  If \var{output} is not \constant{None} it should use that for the output 
  and return \constant{None}, otherwise it should return the result. 
  \var{mode}, \var{cval} have the usual meaning.
  
  The \var{extra_arguments} and \var{extra_keywords} arguments can be used 
  to pass a tuple of extra arguments and a dictionary of named 
  arguments that are passed to \function{derivative2} at each call.

  For example:
\begin{verbatim}
>>> def d2(input, axis, output, mode, cval):
...     return correlate1d(input, [1, -2, 1], axis, output, mode, cval, 0)
... 
>>> a = zeros((5, 5))
>>> a[2, 2] = 1
>>> print generic_laplace(a, d2)
[[ 0  0  0  0  0]
 [ 0  0  1  0  0]
 [ 0  1 -4  1  0]
 [ 0  0  1  0  0]
 [ 0  0  0  0  0]]
\end{verbatim}
To demonstrate the use of the \var{extra_arguments} argument we could do:
\begin{verbatim}
>>> def d2(input, axis, output, mode, cval, weights):
...     return correlate1d(input, weights, axis, output, mode, cval, 0,)
... 
>>> a = zeros((5, 5))
>>> a[2, 2] = 1
>>> print generic_laplace(a, d2, extra_arguments = ([1, -2, 1],))
[[ 0  0  0  0  0]
 [ 0  0  1  0  0]
 [ 0  1 -4  1  0]
 [ 0  0  1  0  0]
 [ 0  0  0  0  0]]
\end{verbatim}
or:
\begin{verbatim}
>>> print generic_laplace(a, d2, extra_keywords = {'weights': [1, -2, 1]})
[[ 0  0  0  0  0]
 [ 0  0  1  0  0]
 [ 0  1 -4  1  0]
 [ 0  0  1  0  0]
 [ 0  0  0  0  0]]
\end{verbatim}
\end{funcdesc}

The following two functions are implemented using 
\function{generic_laplace} by providing appropriate functions for the 
second derivative function:

\begin{funcdesc}{laplace}{input, output=None, mode='reflect', 
  cval=0.0, output_type=None} 
  The function \function{laplace} calculates 
  the Laplace using discrete differentiation for the second derivative 
  (i.e. convolution with \constant{[1, -2, 1]}).
\end{funcdesc}

\begin{funcdesc}{gaussian_laplace}{input, sigma, output=None, 
  mode='reflect', cval=0.0, output_type=None} The function 
  \function{gaussian_laplace} calculates the Laplace using 
  \function{gaussian_filter} to calculate the
  second derivatives. The standard-deviations of the Gaussian filter along 
  each axis are passed through the parameter \var{sigma} as a sequence or 
  numbers.  If \var{sigma} is not a sequence but a single number, the 
  standard deviation of the filter is equal along all directions.
  \end{funcdesc}

The gradient magnitude is defined as the square root of the sum of the 
squares of the gradients in all directions. Similar to the generic Laplace 
function there is a \function{generic_gradient_magnitude} function that 
calculated the gradient magnitude of an array:

\begin{funcdesc}{generic_gradient_magnitude}{input, derivative,
  output=None, mode='reflect', cval=0.0, output_type=None, 
  extra_arguments = (), extra_keywords = {}} The 
  function \function{generic_gradient_magnitude} calculates a gradient 
  magnitude using the function passed through \var{derivative} to calculate 
  first derivatives. The function \function{derivative} should have the 
  following signature:

  \function{derivative(input, axis, output, mode, cval, *extra_arguments, **extra_keywords)}
  
  It should calculate the derivative along the dimension \var{axis}. If
  \var{output} is not \constant{None} it should use that for the output and
  return \constant{None}, otherwise it should return the result. 
  \var{mode}, \var{cval} have the usual meaning.
  
  The \var{extra_arguments} and \var{extra_keywords} arguments can be used 
  to pass a tuple of extra arguments and a dictionary of named 
  arguments that are passed to \function{derivative} at each call.

  For example, the \function{sobel} function fits the required signature:
\begin{verbatim}
>>> a = zeros((5, 5))
>>> a[2, 2] = 1
>>> print generic_gradient_magnitude(a, sobel)
[[0 0 0 0 0]
 [0 1 2 1 0]
 [0 2 0 2 0]
 [0 1 2 1 0]
 [0 0 0 0 0]]
\end{verbatim}
See the documentation of \function{generic_laplace} for examples of using the \var{extra_arguments} and \var{extra_keywords} arguments.
\end{funcdesc}

The \function{sobel} and \function{prewitt} functions fit the required
signature and can therefore directly be used with
\function{generic_gradient_magnitude}. The following function implements 
the gradient magnitude using Gaussian derivatives:

\begin{funcdesc}{gaussian_gradient_magnitude}{input, sigma, output=None, 
  mode='reflect', cval=0.0, output_type=None} The function
  \function{gaussian_gradient_magnitude} calculates the gradient magnitude
  using \function{gaussian_filter} to calculate the first derivatives. The
  standard-deviations of the Gaussian filter along each axis are passed 
  through the parameter \var{sigma} as a sequence or numbers.  If 
  \var{sigma} is not a sequence but a single number, the standard deviation 
  of the filter is equal along all directions.
\end{funcdesc}

\subsection{Generic filter functions}
\label{sec:ndimage:genericfilters}
To implement filter functions, generic functions can be used that accept a 
callable object that implements the filtering operation. The iteration over 
the input and output arrays is handled by these generic functions, along 
with such details as the implementation of the boundary conditions. Only a 
callable object implementing a callback function that does the actual 
filtering work must be provided. The callback function can also be written 
in C and passed using a CObject (see \ref{sec:ndimage:ccallbacks} for more 
information).

\begin{funcdesc}{generic_filter1d}{input, function, filter_size, axis=-1,
  output=None, mode="reflect", cval=0.0, origin=0, output_type=None,
  extra_arguments = (), extra_keywords = {}}
  The \function{generic_filter1d} function implements a generic 
  one-dimensional filter function, where the actual filtering operation 
  must be supplied as a python function (or other callable object). The 
  \function{generic_filter1d} function iterates over the lines of an array 
  and calls \var{function} at each line. The arguments that are passed to 
  \var{function} are one-dimensional arrays of the \constant{tFloat64} 
  type. The first contains the values of the current line. It is extended 
  at the beginning end the end, according to the \var{filter_size} and 
  \var{origin} arguments. The second array should be modified in-place to 
  provide the output values of the line. For example 
  consider a correlation along one dimension:

\begin{verbatim}
>>> a = arange(12, shape = (3,4))
>>> print correlate1d(a, [1, 2, 3])
[[ 3  8 14 17]
 [27 32 38 41]
 [51 56 62 65]]
\end{verbatim}
The same operation can be implemented using \function{generic_filter1d} as 
follows:
\begin{verbatim} 
>>> def fnc(iline, oline):
...     oline[...] = iline[:-2] + 2 * iline[1:-1] + 3 * iline[2:]
... 
>>> print generic_filter1d(a, fnc, 3)
[[ 3  8 14 17]
 [27 32 38 41]
 [51 56 62 65]]
\end{verbatim}
  Here the origin of the kernel was (by default) assumed to be in the 
  middle of the filter of length 3. Therefore, each input line was
  extended by one value at the beginning and at the end, before the 
  function was called.
  
  Optionally extra arguments can be defined and passed to the filter 
  function. The \var{extra_arguments} and \var{extra_keywords} arguments 
  can be used to pass a tuple of extra arguments and/or a dictionary of 
  named arguments that are passed to derivative at each call. For example, 
  we can pass the parameters of our filter as an argument:
\begin{verbatim} 
>>> def fnc(iline, oline, a, b):
...     oline[...] = iline[:-2] + a * iline[1:-1] + b * iline[2:]
... 
>>> print generic_filter1d(a, fnc, 3, extra_arguments = (2, 3))
[[ 3  8 14 17]
 [27 32 38 41]
 [51 56 62 65]]
\end{verbatim}
or
\begin{verbatim} 
>>> print generic_filter1d(a, fnc, 3, extra_keywords = {'a':2, 'b':3})
[[ 3  8 14 17]
 [27 32 38 41]
 [51 56 62 65]]
\end{verbatim}
\end{funcdesc}

\begin{funcdesc}{generic_filter}{input, function, size=None,
  footprint=None, output=None, mode='reflect', cval=0.0, origin=0, 
  output_type=None, extra_arguments = (), extra_keywords = {}}
  The \function{generic_filter} function implements a generic filter  
  function,  where the actual filtering operation must be supplied as a 
  python function (or other callable object). The \function{generic_filter} 
  function iterates over the array and calls \var{function} at each 
  element. The argument of \var{function} is a one-dimensional array of the 
  \constant{tFloat64} type, that contains the values around the current
  element that are within the footprint of the filter. The function should 
  return a single value that can be converted to a double precision 
  number. For example consider a correlation:

\begin{verbatim}
>>> a = arange(12, shape = (3,4))
>>> print correlate(a, [[1, 0], [0, 3]])
[[ 0  3  7 11]
 [12 15 19 23]
 [28 31 35 39]]
\end{verbatim}
The same operation can be implemented using \function{generic_filter} as 
follows:
\begin{verbatim} 
>>> def fnc(buffer): 
...     return (buffer * array([1, 3])).sum()
... 
>>> print generic_filter(a, fnc, footprint = [[1, 0], [0, 1]])
[[ 0  3  7 11]
 [12 15 19 23]
 [28 31 35 39]]
\end{verbatim}
  Here a kernel footprint was specified that contains only two elements.
  Therefore the filter function receives a buffer of length equal to two,
  which was multiplied with the proper weights and the result summed.

  When calling \function{generic_filter}, either the sizes of a rectangular 
  kernel or the footprint of the kernel must be provided. The \var{size} 
  parameter, if provided, must be a sequence of sizes or a single number in 
  which case the size of the filter is assumed to be equal along each axis. 
  The \var{footprint}, if provided, must be an array that defines the shape 
  of the kernel by its non-zero elements.

  Optionally extra arguments can be defined and passed to the filter 
  function. The \var{extra_arguments} and \var{extra_keywords} arguments 
  can be used to pass a tuple of extra arguments and/or a dictionary of 
  named arguments that are passed to derivative at each call. For example, 
  we can pass the parameters of our filter as an argument:
\begin{verbatim} 
>>> def fnc(buffer, weights): 
...     weights = asarray(weights)
...     return (buffer * weights).sum()
... 
>>> print generic_filter(a, fnc, footprint = [[1, 0], [0, 1]], extra_arguments = ([1, 3],))
[[ 0  3  7 11]
 [12 15 19 23]
 [28 31 35 39]]
\end{verbatim}
or
\begin{verbatim} 
>>> print generic_filter(a, fnc, footprint = [[1, 0], [0, 1]], extra_keywords= {'weights': [1, 3]})
[[ 0  3  7 11]
 [12 15 19 23]
 [28 31 35 39]]
\end{verbatim}
\end{funcdesc}

These functions iterate over the lines or elements starting at the 
last axis, i.e. the last index changest the fastest. This order of iteration 
is garantueed for the case that it is important to adapt the filter 
dependening on spatial location. Here is an example of using a class that 
implements the filter and keeps track of the current coordinates while 
iterating. It performs the same filter operation as described above for 
\function{generic_filter}, but additionally prints the current coordinates:
\begin{verbatim}
>>> a = arange(12, shape = (3,4))
>>> 
>>> class fnc_class:
...     def __init__(self, shape):
...         # store the shape:
...         self.shape = shape
...         # initialize the coordinates:
...         self.coordinates = [0] * len(shape)
...         
...     def filter(self, buffer):
...         result = (buffer * array([1, 3])).sum()
...         print self.coordinates
...         # calculate the next coordinates:
...         axes = range(len(self.shape))
...         axes.reverse()
...         for jj in axes:
...             if self.coordinates[jj] < self.shape[jj] - 1:
...                 self.coordinates[jj] += 1
...                 break
...             else:
...                 self.coordinates[jj] = 0
...         return result
... 
>>> fnc = fnc_class(shape = (3,4))
>>> print generic_filter(a, fnc.filter, footprint = [[1, 0], [0, 1]]) 
[0, 0]
[0, 1]
[0, 2]
[0, 3]
[1, 0]
[1, 1]
[1, 2]
[1, 3]
[2, 0]
[2, 1]
[2, 2]
[2, 3]
[[ 0  3  7 11]
 [12 15 19 23]
 [28 31 35 39]]
\end{verbatim}

For the \function{generic_filter1d} function the same approach works, except that this function does not iterate over the axis that is being filtered. The example for \function{generic_filte1d} then becomes this:
\begin{verbatim}
>>> a = arange(12, shape = (3,4))
>>> 
>>> class fnc1d_class:
...     def __init__(self, shape, axis = -1):
...         # store the filter axis:
...         self.axis = axis
...         # store the shape:
...         self.shape = shape
...         # initialize the coordinates:
...         self.coordinates = [0] * len(shape)
...         
...     def filter(self, iline, oline):
...         oline[...] = iline[:-2] + 2 * iline[1:-1] + 3 * iline[2:]
...         print self.coordinates
...         # calculate the next coordinates:
...         axes = range(len(self.shape))
...         # skip the filter axis:
...         del axes[self.axis]
...         axes.reverse()
...         for jj in axes:
...             if self.coordinates[jj] < self.shape[jj] - 1:
...                 self.coordinates[jj] += 1
...                 break
...             else:
...                 self.coordinates[jj] = 0
... 
>>> fnc = fnc1d_class(shape = (3,4))
>>> print generic_filter1d(a, fnc.filter, 3)
[0, 0]
[1, 0]
[2, 0]
[[ 3  8 14 17]
 [27 32 38 41]
 [51 56 62 65]]
\end{verbatim}

\section{Fourier domain filters}
The functions described in this section perform filtering operations in the
Fourier domain. Thus, the input array of such a function should be 
compatible with an inverse Fourier transform function, such as the 
functions from the \module{numarray.fft} module. We therefore have to deal 
with arrays that may be the result of a real or a complex Fourier 
transform. In the case of a real Fourier transform only half of the of the 
symmetric complex transform is stored. Additionally, it needs to be known 
what the length of the axis was that was transformed by the real fft.  The 
functions described here provide a parameter \var{n} that in the case of a 
real transform must be equal to the length of the real transform axis 
before transformation. If this parameter is less than zero, it is assumed 
that the input array was the result of a complex Fourier transform. The 
parameter \var{axis} can be used to indicate along which axis the real 
transform was executed.

\begin{funcdesc}{fourier_shift}{input, shift, n=-1, axis=-1, output=None}
  The \function{fourier_shift} function multiplies the input array with the
  multi-dimensional Fourier transform of a shift operation for the given 
  shift. The \var{shift} parameter is a sequences of shifts for each 
  dimension, or a single value for all dimensions.
\end{funcdesc}

\begin{funcdesc}{fourier_gaussian}{input, sigma, n=-1, axis=-1, output=None}
  The \function{fourier_gaussian} function multiplies the input array with 
  the multi-dimensional Fourier transform of a Gaussian filter with given
  standard-deviations \var{sigma}. The \var{sigma} parameter is a sequences 
  of values for each dimension, or a single value for all dimensions.
\end{funcdesc}

\begin{funcdesc}{fourier_uniform}{input, size, n=-1, axis=-1, output=None}
  The \function{fourier_uniform} function multiplies the input array with 
  the multi-dimensional Fourier transform of a uniform filter with given
  sizes \var{size}. The \var{size} parameter is a sequences of
  values for each dimension, or a single value for all dimensions.
\end{funcdesc}

\begin{funcdesc}{fourier_ellipsoid}{input, size, n=-1, axis=-1, 
  output=None}
  The \function{fourier_ellipsoid} function multiplies the input array with 
  the multi-dimensional Fourier transform of a elliptically shaped filter 
  with given sizes \var{size}. The \var{size} parameter is a sequences of 
  values for each dimension, or a single value for all dimensions.  
  \note{This function is
    only implemented for dimensions 1, 2, and 3.}
\end{funcdesc}

\section{Interpolation functions}
This section describes various interpolation functions that are based on
B-spline theory. A good introduction to B-splines can be found in: M. 
Unser, "Splines: A Perfect Fit for Signal and Image Processing," IEEE 
Signal Processing Magazine, vol. 16, no. 6, pp. 22-38, November 1999.
\subsection{Spline pre-filters}
Interpolation using splines of an order larger than 1 requires a pre-
filtering step. The interpolation functions described in section
\ref{sec:ndimage:interpolation} apply pre-filtering by calling
\function{spline_filter}, but they can be instructed not to do this by 
setting the \var{prefilter} keyword equal to \constant{False}.  This is 
useful if more than one interpolation operation is done on the same array. 
In this case it is more efficient to do the pre-filtering only once and use 
a prefiltered array as the input of the interpolation functions. The 
following two functions implement the pre-filtering:

\begin{funcdesc}{spline_filter1d}{input, order=3, axis=-1, output=None,
    output_type=numarray.Float64} The \function{spline_filter1d} function
  calculates a one-dimensional spline filter along the given axis. An 
  output array can optionally be provided. The order of the spline must be 
  larger then 1 and less than 6.
\end{funcdesc}

\begin{funcdesc}{spline_filter}{input, order=3, output=None, 
    output_type=numarray.Float64} The \function{spline_filter} function
  calculates a multi-dimensional spline filter.
  
  \note{The multi-dimensional filter is implemented as a sequence of
    one-dimensional spline filters. The intermediate arrays are stored in 
    the same data type as the output. Therefore, if an output 
    with a limited precision is requested, the results may be imprecise 
    because intermediate results may be stored with insufficient precision. 
    This can be prevented by specifying a output type of high precision.}
\end{funcdesc}

\subsection{Interpolation functions}
\label{sec:ndimage:interpolation}
Following functions all employ spline interpolation to effect some type of
geometric transformation of the input array. This requires a mapping of the
output coordinates to the input coordinates, and therefore the possibility
arises that input values outside the boundaries are needed. This problem 
is solved in the same way as described in section
\ref{sec:ndimage:filter-functions} for the multi-dimensional filter 
functions. Therefore these functions all support a \var{mode} parameter 
that determines how the boundaries are handled, and a \var{cval} parameter 
that gives a constant value in case that the \constant{'constant'} mode is 
used.

\begin{funcdesc}{geometric_transform}{input, mapping, output_shape=None,
    output_type=None, output=None, order=3, mode='constant', cval=0.0,
    prefilter=True, extra_arguments = (), extra_keywords = {}} The \function{geometric_transform} function applies an
  arbitrary geometric transform to the input. The given \var{mapping} 
  function is called at each point in the output to find the corresponding 
  coordinates in the input.  \var{mapping} must be a callable object that 
  accepts a tuple of length equal to the output array rank and returns the 
  corresponding input coordinates as a tuple of length equal to the input 
  array rank. The output shape and output type can optionally be provided. 
  If not given they are equal to the input shape and type.
  
  For example:
\begin{verbatim}
>>> a = arange(12, shape=(4,3), type = Float64)
>>> def shift_func(output_coordinates):
...     return (output_coordinates[0] - 0.5, output_coordinates[1] - 0.5)
... 
>>> print geometric_transform(a, shift_func)
[[ 0.      0.      0.    ]
 [ 0.      1.3625  2.7375]
 [ 0.      4.8125  6.1875]
 [ 0.      8.2625  9.6375]]  
\end{verbatim}  

  Optionally extra arguments can be defined and passed to the filter 
  function. The \var{extra_arguments} and \var{extra_keywords} arguments 
  can be used to pass a tuple of extra arguments and/or a dictionary of 
  named arguments that are passed to derivative at each call. For example, 
  we can pass the shifts in our example as arguments:

\begin{verbatim}
>>> def shift_func(output_coordinates, s0, s1):
...     return (output_coordinates[0] - s0, output_coordinates[1] - s1)
... 
>>> print geometric_transform(a, shift_func, extra_arguments = (0.5, 0.5))
[[ 0.      0.      0.    ]
 [ 0.      1.3625  2.7375]
 [ 0.      4.8125  6.1875]
 [ 0.      8.2625  9.6375]]  
\end{verbatim}  
or
\begin{verbatim}
>>> print geometric_transform(a, shift_func, extra_keywords = {'s0': 0.5, 's1': 0.5})
[[ 0.      0.      0.    ]
 [ 0.      1.3625  2.7375]
 [ 0.      4.8125  6.1875]
 [ 0.      8.2625  9.6375]]  
\end{verbatim}  

\note{The mapping function can also be written in C and passed using a CObject. See \ref{sec:ndimage:ccallbacks} for more information.}
\end{funcdesc}

\begin{funcdesc}{map_coordinates}{input, coordinates, output_type=None, 
    output=None, order=3, mode='constant', cval=0.0, prefilter=True} 
  The function \function{map_coordinates} applies an arbitrary coordinate
  transformation using the given array of coordinates. The shape of the 
  output is derived from that of the coordinate array by dropping the first 
  axis. The parameter \var{coordinates} is used to find for each point in 
  the output the corresponding coordinates in the input. The values of 
  \var{coordinates} along the first axis are the coordinates in the input 
  array at which the output value is found. (See also the numarray 
  \function{coordinates} function.) Since the coordinates may be non-
  integer coordinates, the value of the input at these coordinates is 
  determined by spline interpolation of the requested order. Here is an 
  example that interpolates a 2D array at (0.5, 0.5) and (1, 2):
\begin{verbatim}
>>> a = arange(12, shape=(4,3), type = numarray.Float64)
>>> print a
[[  0.   1.   2.]
 [  3.   4.   5.]
 [  6.   7.   8.]
 [  9.  10.  11.]]
>>> print map_coordinates(a, [[0.5, 2], [0.5, 1]])
[ 1.3625  7.    ]
\end{verbatim}
\end{funcdesc}

\begin{funcdesc}{affine_transform}{input, matrix, offset=0.0, 
  output_shape=None, output_type=None, output=None, order=3, 
  mode='constant', cval=0.0, prefilter=True} The 
  \function{affine_transform} function applies an affine transformation to 
  the input array. The given transformation \var{matrix} and \var{offset} 
  are used to find for each point in the output the corresponding 
  coordinates in the input.  The value of the input at the
  calculated coordinates is determined by spline interpolation of the 
  requested order. The transformation \var{matrix} must be two-dimensional 
  or can also be given as a one-dimensional sequence or array.  In the 
  latter case, it is assumed that the matrix is diagonal. A more efficient 
  interpolation algorithm is then applied that exploits the separability of 
  the problem.  The output shape and output type can optionally be 
  provided. If not given they are equal to the input shape and type.
\end{funcdesc}

\begin{funcdesc}{shift}{input, shift, output_type=None, output=None, 
  order=3, mode='constant', cval=0.0, prefilter=True} The \function{shift} 
  function returns a shifted version of the input, using spline 
  interpolation of the requested \var{order}.
\end{funcdesc}

\begin{funcdesc}{zoom}{input, zoom, output_type=None, output=None, order=3, 
    mode='constant', cval=0.0, prefilter=True} The \function{zoom} function
  returns a rescaled version of the input, using spline interpolation of 
  the requested \var{order}.
\end{funcdesc}

\begin{funcdesc}{rotate}{input, angle, axes=(-1, -2), reshape=1,
    output_type=None, output=None, order=3, mode='constant', cval=0.0,
    prefilter=True} The \function{rotate} function returns the input array
  rotated in the plane defined by the two axes given by the parameter
  \var{axes}, using spline interpolation of the requested \var{order}. The
  angle must be given in degrees. If \var{reshape} is true, then the size 
  of the output array is adapted to contain the rotated input.
\end{funcdesc}

\section{Binary morphology}
\label{sec:ndimage:binary-morphology}

\begin{funcdesc}{generate_binary_structure}{rank, connectivity}
  The \function{generate_binary_structure} functions generates a binary
  structuring element for use in binary morphology operations. The 
  \var{rank} of the structure must be provided. The size of the structure 
  that is returned is equal to three in each direction. The value of each 
  element is equal to one if the square of the Euclidean distance from the 
  element to the center is less or equal to \var{connectivity}. For 
  instance, two dimensional 4-connected and 8-connected structures are 
  generated as follows:
\begin{verbatim}
>>> print generate_binary_structure(2, 1)
[[0 1 0]
 [1 1 1]
 [0 1 0]]
>>> print generate_binary_structure(2, 2)
[[1 1 1]
 [1 1 1]
 [1 1 1]]
\end{verbatim}
\end{funcdesc}

Most binary morphology functions can be expressed in terms of the basic
operations erosion and dilation:

\begin{funcdesc}{binary_erosion}{input, structure=None, iterations=1,
    mask=None, output=None, border_value=0, origin=0} The
  \function{binary_erosion} function implements binary erosion of arrays of
  arbitrary rank with the given structuring element. The origin parameter
  controls the placement of the structuring element as described in section
  \ref{sec:ndimage:filter-functions}. If no structuring element is 
  provided, an element with connectivity equal to one is generated using
  \function{generate_binary_structure}. The \var{border_value} parameter 
  gives the value of the array outside boundaries. The erosion is repeated
  \var{iterations} times. If \var{iterations} is less than one, the erosion 
  is repeated until the result does not change anymore. If a \var{mask} 
  array is given, only those elements with a true value at the 
  corresponding mask element are modified at each iteration.
\end{funcdesc}

\begin{funcdesc}{binary_dilation}{input, structure=None, iterations=1,
    mask=None, output=None, border_value=0, origin=0} The
  \function{binary_dilation} function implements binary dilation of arrays 
  of arbitrary rank with the given structuring element. The origin 
  parameter controls the placement of the structuring element as described 
  in section \ref{sec:ndimage:filter-functions}. If no structuring element 
  is provided, an element with connectivity equal to one is generated using
  \function{generate_binary_structure}. The \var{border_value} parameter 
  gives the value of the array outside boundaries. The dilation is repeated
  \var{iterations} times.  If \var{iterations} is less than one, the 
  dilation is repeated until the result does not change anymore. If a 
  \var{mask} array is given, only those elements with a true value at the 
  corresponding mask element are modified at each iteration.

  Here is an example of using \function{binary_dilation} to find all 
  elements that touch the border, by repeatedly dilating an empty array 
  from the border using the data array as the mask:
\begin{verbatim}
>>> struct = array([[0, 1, 0], [1, 1, 1], [0, 1, 0]])
>>> a = array([[1,0,0,0,0], [1,1,0,1,0], [0,0,1,1,0], [0,0,0,0,0]])
>>> print a
[[1 0 0 0 0]
 [1 1 0 1 0]
 [0 0 1 1 0]
 [0 0 0 0 0]]
>>> print binary_dilation(zeros(a.shape), struct, -1, a, border_value=1)
[[1 0 0 0 0]
 [1 1 0 0 0]
 [0 0 0 0 0]
 [0 0 0 0 0]]
\end{verbatim}
\end{funcdesc}

The \function{binary_erosion} and \function{binary_dilation} functions both
have an \var{iterations} parameter which allows the erosion or dilation to 
be repeated a number of times. Repeating an erosion or a dilation with a 
given structure \constant{n} times is equivalent to an erosion or a 
dilation with a structure that is \constant{n-1} times dilated with itself. 
A function is provided that allows the calculation of a structure that is 
dilated a number of times with itself:

\begin{funcdesc}{iterate_structure}{structure, iterations, origin=None} 
  The \function{iterate_structure} function returns a structure by dilation 
  of the input structure \var{iteration} - 1 times with itself. For 
  instance:
  \begin{verbatim}
>>> struct = generate_binary_structure(2, 1)
>>> print struct
[[0 1 0]
 [1 1 1]
 [0 1 0]]
>>> print iterate_structure(struct, 2)
[[0 0 1 0 0]
 [0 1 1 1 0]
 [1 1 1 1 1]
 [0 1 1 1 0]
 [0 0 1 0 0]]
\end{verbatim}
  If the origin of the original structure is equal to 0, then it is also 
  equal to 0 for the iterated structure. If not, the origin must also be 
  adapted if the equivalent of the \var{iterations} erosions or dilations 
  must be achieved with the iterated structure. The adapted origin is 
  simply obtained by multiplying with the number of iterations. For 
  convenience the
  \function{iterate_structure} also returns the adapted origin if the
  \var{origin} parameter is not \constant{None}:
\begin{verbatim}
>>> print iterate_structure(struct, 2, -1)
(array([[0, 0, 1, 0, 0],
       [0, 1, 1, 1, 0],
       [1, 1, 1, 1, 1],
       [0, 1, 1, 1, 0],
       [0, 0, 1, 0, 0]], type=Bool), [-2, -2])
\end{verbatim}
\end{funcdesc}

Other morphology operations can be defined in terms of erosion and d
dilation. Following functions provide a few of these operations for 
convenience:

\begin{funcdesc}{binary_opening}{input, structure=None, iterations=1,
  output=None, origin=0} The \function{binary_opening} function implements
  binary opening of arrays of arbitrary rank with the given structuring
  element. Binary opening is equivalent to a binary erosion followed by a
  binary dilation with the same structuring element. The origin parameter
  controls the placement of the structuring element as described in section
  \ref{sec:ndimage:filter-functions}. If no structuring element is 
  provided, an element with connectivity equal to one is generated using
  \function{generate_binary_structure}. The \var{iterations} parameter 
  gives the number of erosions that is performed followed by the same 
  number of dilations.
\end{funcdesc}

\begin{funcdesc}{binary_closing}{input, structure=None, iterations=1,
  output=None, origin=0} The \function{binary_closing} function implements
  binary closing of arrays of arbitrary rank with the given structuring
  element. Binary closing is equivalent to a binary dilation followed by a
  binary erosion with the same structuring element. The origin parameter
  controls the placement of the structuring element as described in section
  \ref{sec:ndimage:filter-functions}. If no structuring element is 
  provided, an element with connectivity equal to one is generated using
  \function{generate_binary_structure}. The \var{iterations} parameter   
  gives the number of dilations that is performed followed by the same 
  number of erosions.
\end{funcdesc}

\begin{funcdesc}{binary_fill_holes}{input, structure = None, output = None, 
origin = 0} The \function{binary_fill_holes} function is used to close 
holes in objects in a binary image, where the structure defines the 
connectivity of the holes. The origin parameter controls the placement of 
the structuring element as described in section \ref{sec:ndimage:filter-
functions}. If no structuring element is provided, an element with 
connectivity equal to one is generated using 
\function{generate_binary_structure}. 
\end{funcdesc}

\begin{funcdesc}{binary_hit_or_miss}{input, structure1=None, 
  structure2=None, output=None, origin1=0, origin2=None} The 
  \function{binary_hit_or_miss}
  function implements a binary hit-or-miss transform of arrays of arbitrary
  rank with the given structuring elements.  The hit-or-miss transform is
  calculated by erosion of the input with the first structure, erosion of   
  the logical \emph{not} of the input with the second structure, followed 
  by the logical \emph{and} of these two erosions.  The origin parameters 
  control the placement of the structuring elements as described in section
  \ref{sec:ndimage:filter-functions}. If \var{origin2} equals 
  \constant{None} it is set equal to the \var{origin1} parameter. If the 
  first structuring element is not provided, a structuring element with 
  connectivity equal to one is generated using 
  \function{generate_binary_structure}, if \var{structure2} is not 
  provided, it is set equal to the logical \emph{not} of \var{structure1}.
\end{funcdesc}

\section{Grey-scale morphology}
\label{sec:ndimage:grey-morphology}

Grey-scale morphology operations are the equivalents of binary morphology
operations that operate on arrays with arbitrary values. Below we describe 
the grey-scale equivalents of erosion, dilation, opening and closing. These
operations are implemented in a similar fashion as the filters described in
section \ref{sec:ndimage:filter-functions}, and we refer to this section 
for the description of filter kernels and footprints, and the handling of 
array borders. The grey-scale morphology operations optionally take a 
\var{structure} parameter that gives the values of the structuring element. 
If this parameter is not given the structuring element is assumed to be 
flat with a value equal to zero. The shape of the structure can optionally 
be defined by the \var{footprint} parameter. If this parameter is not 
given, the structure is assumed to be rectangular, with sizes equal to the 
dimensions of the \var{structure} array, or by the \var{size} parameter if 
\var{structure} is not given. The \var{size} parameter is only used if both 
\var{structure} and \var{footprint} are not given, in which case the 
structuring element is assumed to be rectangular and flat with the 
dimensions given by \var{size}. The \var{size} parameter, if provided, must 
be a sequence of sizes or a single number in which case the size of the 
filter is assumed to be equal along each axis. The \var{footprint} 
parameter, if provided, must be an array that defines the shape of the 
kernel by its non-zero elements.

Similar to binary erosion and dilation there are operations for grey-scale
erosion and dilation:

\begin{funcdesc}{grey_erosion}{input, size=None, footprint=None, 
    structure=None, output=None, mode='reflect', cval=0.0, origin=0} The
  \function{grey_erosion} function calculates a multi-dimensional grey-
  scale erosion.
\end{funcdesc}

\begin{funcdesc}{grey_dilation}{input, size=None, footprint=None, 
    structure=None, output=None, mode='reflect', cval=0.0, origin=0} The
  \function{grey_dilation} function calculates a multi-dimensional grey-
  scale dilation.
\end{funcdesc}

Grey-scale opening and closing operations can be defined similar to their
binary counterparts:

\begin{funcdesc}{grey_opening}{input, size=None, footprint=None, 
    structure=None, output=None, mode='reflect', cval=0.0, origin=0} The
  \function{grey_opening} function implements grey-scale opening of arrays 
  of arbitrary rank. Grey-scale opening is equivalent to a grey-scale 
  erosion followed by a grey-scale dilation.
\end{funcdesc}

\begin{funcdesc}{grey_closing}{input, size=None, footprint=None, 
    structure=None, output=None, mode='reflect', cval=0.0, origin=0} The
  \function{grey_closing} function implements grey-scale closing of arrays 
  of arbitrary rank. Grey-scale opening is equivalent to a grey-scale 
  dilation followed by a grey-scale erosion.
\end{funcdesc}

\begin{funcdesc}{morphological_gradient}{input, size=None, footprint=None, 
    structure=None, output=None, mode='reflect', cval=0.0, origin=0} The
  \function{morphological_gradient} function implements a grey-scale
  morphological gradient of arrays of arbitrary rank. The grey-scale
  morphological gradient is equal to the difference of a grey-scale 
  dilation and a grey-scale erosion.
\end{funcdesc}

\begin{funcdesc}{morphological_laplace}{input, size=None, footprint=None, 
    structure=None, output=None, mode='reflect', cval=0.0, origin=0} The
  \function{morphological_laplace} function implements a grey-scale
  morphological laplace of arrays of arbitrary rank. The grey-scale
  morphological laplace is equal to the sum of a grey-scale dilation and a
  grey-scale erosion minus twice the input.
\end{funcdesc}

\begin{funcdesc}{white_tophat}{input, size=None, footprint=None, 
    structure=None, output=None, mode='reflect', cval=0.0, origin=0} The
  \function{white_tophat} function implements a white top-hat filter of 
  arrays of arbitrary rank. The white top-hat is equal to the difference of 
  the input and a grey-scale opening.
\end{funcdesc}

\begin{funcdesc}{black_tophat}{input, size=None, footprint=None, 
    structure=None, output=None, mode='reflect', cval=0.0, origin=0} The
  \function{black_tophat} function implements a black top-hat filter of 
  arrays of arbitrary rank. The black top-hat is equal to the difference of 
  the a grey-scale closing and the input.
\end{funcdesc}

\section{Distance transforms}
\label{sec:ndimage:grey-morphology}
Distance transforms are used to calculate the minimum distance from each
element of an object to the background. The following functions implement
distance transforms for three different distance metrics: Euclidean, City
Block, and Chessboard distances.

\begin{funcdesc}{distance_transform_cdt}{input, structure="chessboard",
  return_distances=True, return_indices=False, distances=None, 
  indices=None} The function \function{distance_transform_cdt} uses a 
  chamfer type algorithm to calculate the distance transform of the input, 
  by replacing each object element (defined by values larger than zero) 
  with the shortest distance to the background (all non-object elements). 
  The structure determines the type of chamfering that is done. If the 
  structure is equal to 'cityblock' a structure is generated using 
  \function{generate_binary_structure} with a squared distance equal to 1. 
  If the structure is equal to 'chessboard', a structure is generated using 
  \function{generate_binary_structure} with a squared distance equal to the 
  rank of the array. These choices correspond to the common interpretations 
  of the cityblock and the chessboard distancemetrics in two dimensions.
  
  In addition to the distance transform, the feature transform can be
  calculated. In this case the index of the closest background element is
  returned along the first axis of the result.  The \var{return_distances}, 
  and \var{return_indices} flags can be used to indicate if the distance 
  transform, the feature transform, or both must be returned.
  
  The \var{distances} and \var{indices} arguments can be used to give 
  optional output arrays that must be of the correct size and type (both
  \constant{Int32}).

  The basics of the algorithm used to implement this function is described
  in: G. Borgefors, "Distance transformations in arbitrary dimensions.",
  Computer Vision, Graphics, and Image Processing, 27:321--345, 1984.
\end{funcdesc}

\begin{funcdesc}{distance_transform_edt}{input, sampling=None,
  return_distances=True, return_indices=False, distances=None, 
  indices=None} The function \function{distance_transform_edt} calculates 
  the exact euclidean distance transform of the input, by replacing each 
  object element (defined by values larger than zero) with the shortest 
  euclidean distance to the background (all non-object elements).
  
  In addition to the distance transform, the feature transform can be
  calculated. In this case the index of the closest background element is
  returned along the first axis of the result.  The \var{return_distances}, 
  and \var{return_indices} flags can be used to indicate if the distance 
  transform, the feature transform, or both must be returned.
  
  Optionally the sampling along each axis can be given by the 
  \var{sampling} parameter which should be a sequence of length equal to 
  the input rank, or a single number in which the sampling is assumed to be 
  equal along all axes.

  The \var{distances} and \var{indices} arguments can be used to give 
  optional output arrays that must be of the correct size and type 
  (\constant{Float64} and \constant{Int32}).
  
  The algorithm used to implement this function is described in: C. R. 
  Maurer, Jr., R. Qi, and V. Raghavan, "A linear time algorithm for 
  computing exact euclidean distance transforms of binary images in 
  arbitrary dimensions. IEEE Trans. PAMI 25, 265-270, 2003.
\end{funcdesc}

\begin{funcdesc}{distance_transform_bf}{input, metric="euclidean",
  sampling=None, return_distances=True, return_indices=False, 
  distances=None, indices=None} The function  
  \function{distance_transform_bf} uses a brute-force algorithm to 
  calculate the distance transform of the input, by replacing each object 
  element (defined by values larger than zero) with the shortest distance 
  to the background (all non-object elements).  The metric must be one of 
  \constant{"euclidean"}, \constant{"cityblock"}, or 
  \constant{"chessboard"}.
  
  In addition to the distance transform, the feature transform can be
  calculated. In this case the index of the closest background element is
  returned along the first axis of the result.  The \var{return_distances}, 
  and \var{return_indices} flags can be used to indicate if the distance 
  transform, the feature transform, or both must be returned.
  
  Optionally the sampling along each axis can be given by the 
  \var{sampling} parameter which should be a sequence of length equal to 
  the input rank, or a single number in which the sampling is assumed to be 
  equal along all axes. This parameter is only used in the case of the 
  euclidean distance transform.

  The \var{distances} and \var{indices} arguments can be used to give 
  optional output arrays that must be of the correct size and type 
  (\constant{Float64} and \constant{Int32}).

  \note{This function uses a slow brute-force algorithm, the function
    \function{distance_transform_cdt} can be used to more efficiently 
    calculate cityblock and chessboard distance transforms. The function
    \function{distance_transform_edt} can be used to more efficiently 
    calculate the exact euclidean distance transform.}
\end{funcdesc}

\section{Segmentation and labeling}
Segmentation is the process of separating objects of interest from the
background. The most simple approach is probably intensity thresholding, 
which is easily done with \module{numarray} functions:
\begin{verbatim}
>>> a = array([[1,2,2,1,1,0],
...            [0,2,3,1,2,0],
...            [1,1,1,3,3,2],
...            [1,1,1,1,2,1]])
>>> print where(a > 1, 1, 0)
[[0 1 1 0 0 0]
 [0 1 1 0 1 0]
 [0 0 0 1 1 1]
 [0 0 0 0 1 0]]
\end{verbatim}

The result is a binary image, in which the individual objects still need to 
be identified and labeled.  The function \function{label} generates an 
array where each object is assigned a unique number:

\begin{funcdesc}{label}{input, structure=None, output=None}
  The \function{label} function generates an array where the objects in the
  input are labeled with an integer index. It returns a tuple consisting of 
  the array of object labels and the number of objects found, unless the
  \var{output} parameter is given, in which case only the number of objects 
  is returned. The connectivity of the objects is defined by a structuring
  element. For instance, in two dimensions using a four-connected 
  structuring element gives:
\begin{verbatim}
>>> a = array([[0,1,1,0,0,0],[0,1,1,0,1,0],[0,0,0,1,1,1],[0,0,0,0,1,0]])
>>> s = [[0, 1, 0], [1,1,1], [0,1,0]]
>>> print label(a, s)
(array([[0, 1, 1, 0, 0, 0],
       [0, 1, 1, 0, 2, 0],
       [0, 0, 0, 2, 2, 2],
       [0, 0, 0, 0, 2, 0]]), 2)
\end{verbatim}
These two objects are not connected because there is no way in which we can
place the structuring element such that it overlaps with both objects. 
However, an 8-connected structuring element results in only a single 
object:
\begin{verbatim}
>>> a = array([[0,1,1,0,0,0],[0,1,1,0,1,0],[0,0,0,1,1,1],[0,0,0,0,1,0]])
>>> s = [[1,1,1], [1,1,1], [1,1,1]]
>>> print label(a, s)[0]
[[0 1 1 0 0 0]
 [0 1 1 0 1 0]
 [0 0 0 1 1 1]
 [0 0 0 0 1 0]]
\end{verbatim}
If no structuring element is provided, one is generated by calling
\function{generate_binary_structure} (see section \ref{sec:ndimage:
morphology}) using a connectivity of one (which in 2D is the 4-connected 
structure of the first example).  The input can be of any type, any value 
not equal to zero is taken to be part of an object. This is useful if you 
need to 're-label' an array of object indices, for instance after removing 
unwanted objects. Just apply the label function again to the index array. 
For instance:
\begin{verbatim}
>>> l, n = label([1, 0, 1, 0, 1])
>>> print l
[1 0 2 0 3]
>>> l = where(l != 2, l, 0)
>>> print l
[1 0 0 0 3]
>>> print label(l)[0]
[1 0 0 0 2]
\end{verbatim}

\note{The structuring element used by \function{label} is assumed to be
  symmetric.}
\end{funcdesc}

There is a large number of other approaches for segmentation, for instance 
from an estimation of the borders of the objects that can be obtained for 
instance by derivative filters. One such an approach is watershed 
segmentation.  The function \function{watershed_ift} generates an array 
where each object is assigned a unique label, from an array that localizes 
the object borders, generated for instance by a gradient magnitude filter. 
It uses an array containing initial markers for the objects:
\begin{funcdesc}{watershed_ift}{input, markers, structure=None, 
  output=None} The \function{watershed_ift} function applies a watershed 
  from markers algorithm, using an Iterative Forest Transform, as described 
  in: P. Felkel, R.  Wegenkittl, and M. Bruckschwaiger, "Implementation and 
  Complexity of the Watershed-from-Markers Algorithm Computed as a Minimal 
  Cost Forest.", Eurographics 2001, pp. C:26-35.
  
  The inputs of this function are the array to which the transform is 
  applied, and an array of markers that designate the objects by a unique 
  label, where any non-zero value is a marker. For instance:
\begin{verbatim}
>>> input = array([[0, 0, 0, 0, 0, 0, 0],
...                [0, 1, 1, 1, 1, 1, 0],
...                [0, 1, 0, 0, 0, 1, 0],
...                [0, 1, 0, 0, 0, 1, 0],
...                [0, 1, 0, 0, 0, 1, 0],
...                [0, 1, 1, 1, 1, 1, 0],
...                [0, 0, 0, 0, 0, 0, 0]], numarray.UInt8)
>>> markers = array([[1, 0, 0, 0, 0, 0, 0],
...                  [0, 0, 0, 0, 0, 0, 0],
...                  [0, 0, 0, 0, 0, 0, 0],
...                  [0, 0, 0, 2, 0, 0, 0],
...                  [0, 0, 0, 0, 0, 0, 0],
...                  [0, 0, 0, 0, 0, 0, 0],
...                  [0, 0, 0, 0, 0, 0, 0]], numarray.Int8)
>>> print watershed_ift(input, markers)
[[1 1 1 1 1 1 1]
 [1 1 2 2 2 1 1]
 [1 2 2 2 2 2 1]
 [1 2 2 2 2 2 1]
 [1 2 2 2 2 2 1]
 [1 1 2 2 2 1 1]
 [1 1 1 1 1 1 1]]
\end{verbatim}
  
  Here two markers were used to designate an object (marker=2) and the
  background (marker=1).  The order in which these are processed is 
  arbitrary: moving the marker for the background to the lower right corner 
  of the array yields a different result:
\begin{verbatim}
>>> markers = array([[0, 0, 0, 0, 0, 0, 0],
...                  [0, 0, 0, 0, 0, 0, 0],
...                  [0, 0, 0, 0, 0, 0, 0],
...                  [0, 0, 0, 2, 0, 0, 0],
...                  [0, 0, 0, 0, 0, 0, 0],
...                  [0, 0, 0, 0, 0, 0, 0],
...                  [0, 0, 0, 0, 0, 0, 1]], numarray.Int8)
>>> print watershed_ift(input, markers)
[[1 1 1 1 1 1 1]
 [1 1 1 1 1 1 1]
 [1 1 2 2 2 1 1]
 [1 1 2 2 2 1 1]
 [1 1 2 2 2 1 1]
 [1 1 1 1 1 1 1]
 [1 1 1 1 1 1 1]]
\end{verbatim}
  The result is that the object (marker=2) is smaller because the second 
  marker was processed earlier. This may not be the desired effect if the 
  first marker was supposed to designate a background object. Therefore
  \function{watershed_ift} treats markers with a negative value explicitly 
  as background markers and processes them after the normal markers. For 
  instance, replacing the first marker by a negative marker gives a result 
  similar to the first example:
\begin{verbatim}
>>> markers = array([[0, 0, 0, 0, 0, 0, 0],
...                  [0, 0, 0, 0, 0, 0, 0],
...                  [0, 0, 0, 0, 0, 0, 0],
...                  [0, 0, 0, 2, 0, 0, 0],
...                  [0, 0, 0, 0, 0, 0, 0],
...                  [0, 0, 0, 0, 0, 0, 0],
...                  [0, 0, 0, 0, 0, 0, -1]], numarray.Int8)
>>> print watershed_ift(input, markers)
[[-1 -1 -1 -1 -1 -1 -1]
 [-1 -1  2  2  2 -1 -1]
 [-1  2  2  2  2  2 -1]
 [-1  2  2  2  2  2 -1]
 [-1  2  2  2  2  2 -1]
 [-1 -1  2  2  2 -1 -1]
 [-1 -1 -1 -1 -1 -1 -1]]
\end{verbatim}
  
  The connectivity of the objects is defined by a structuring element. If 
  no structuring element is provided, one is generated by calling
  \function{generate_binary_structure} (see section
  \ref{sec:ndimage:morphology}) using a connectivity of one (which in 2D is 
  a 4-connected structure.) For example, using an 8-connected structure 
  with the last example yields a different object:
\begin{verbatim}
>>> print watershed_ift(input, markers,
...                     structure = [[1,1,1], [1,1,1], [1,1,1]])
[[-1 -1 -1 -1 -1 -1 -1]
 [-1  2  2  2  2  2 -1]
 [-1  2  2  2  2  2 -1]
 [-1  2  2  2  2  2 -1]
 [-1  2  2  2  2  2 -1]
 [-1  2  2  2  2  2 -1]
 [-1 -1 -1 -1 -1 -1 -1]]
\end{verbatim}

\note{The implementation of \function{watershed_ift} limits the data types 
of the input to \constant{UInt8} and \constant{UInt16}.}
\end{funcdesc}

\section{Object measurements}
Given an array of labeled objects, the properties of the individual objects 
can be measured. The \function{find_objects} function can be used to 
generate a list of slices that for each object, give the smallest sub-array 
that fully contains the object:

\begin{funcdesc}{find_objects}{input, max_label=0}
  The \function{find_objects} finds all objects in a labeled array and 
  returns a list of slices that correspond to the smallest regions in the 
  array that contains the object. For instance:
\begin{verbatim}
>>> a = array([[0,1,1,0,0,0],[0,1,1,0,1,0],[0,0,0,1,1,1],[0,0,0,0,1,0]])
>>> l, n = label(a)
>>> f = find_objects(l)
>>> print a[f[0]]
[[1 1]
 [1 1]]
>>> print a[f[1]]
[[0 1 0]
 [1 1 1]
 [0 1 0]]
\end{verbatim}
\function{find_objects} returns slices for all objects, unless the
\var{max_label} parameter is larger then zero, in which case only the first
\var{max_label} objects are returned. If an index is missing in the 
\var{label} array, \constant{None} is return instead of a slice. For 
example:
\begin{verbatim}
>>> print find_objects([1, 0, 3, 4], max_label = 3)
[(slice(0, 1, None),), None, (slice(2, 3, None),)]
\end{verbatim}
\end{funcdesc}

The list of slices generated by \function{find_objects} is useful to find 
the position and dimensions of the objects in the array, but can also be 
used to perform measurements on the individual objects. Say we want to find 
the sum of the intensities of an object in image:
\begin{verbatim}
>>> image = arange(4*6,shape=(4,6))
>>> mask = array([[0,1,1,0,0,0],[0,1,1,0,1,0],[0,0,0,1,1,1],[0,0,0,0,1,0]])
>>> labels = label(mask)[0]
>>> slices = find_objects(labels)
\end{verbatim}
Then we can calculate the sum of the elements in the second object:
\begin{verbatim}
>>> print where(labels[slices[1]] == 2, image[slices[1]], 0).sum()
80
\end{verbatim}
That is however not particularly efficient, and may also be more 
complicated for other types of measurements. Therefore a few measurements 
functions are defined that accept the array of object labels and the index 
of the object to be measured. For instance calculating the sum of the 
intensities can be done by:
\begin{verbatim}
>>> print sum(image, labels, 2)
80.0
\end{verbatim}
For large arrays and small objects it is more efficient to call the 
measurement functions after slicing the array:
\begin{verbatim}
>>> print sum(image[slices[1]], labels[slices[1]], 2)
80.0
\end{verbatim}
Alternatively, we can do the measurements for a number of labels with a 
single function call, returning a list of results. For instance, to measure 
the sum of the values of the background and the second object in our 
example we give a list of labels:
\begin{verbatim}
>>> print sum(image, labels, [0, 2])
[178.0, 80.0]
\end{verbatim}

The measurement functions described below all support the \var{index} 
parameter to indicate which object(s) should be measured. The default value 
of \var{index} is \constant{None}. This indicates that all elements where 
the label is larger than zero should be treated as a single object and 
measured. Thus, in this case the \var{labels} array is treated as a mask 
defined by the elements that are larger than zero. If \var{index} is a 
number or a sequence of numbers it gives the labels of the objects that are 
measured. If \var{index} is a sequence, a list of the results is returned. 
Functions that return more than one result, return their result as a tuple 
if \var{index} is a single number, or as a tuple of lists, if \var{index} 
is a sequence.

\begin{funcdesc}{sum}{input, labels=None, index=None}
  The \function{sum} function calculates the sum of the elements of the 
  object with label(s) given by \var{index}, using the \var{labels} array 
  for the object labels. If \var{index} is \constant{None}, all elements 
  with a non-zero label value are treated as a single object. If 
  \var{label} is \constant{None}, all elements of \var{input} are used in 
  the calculation.
\end{funcdesc}

\begin{funcdesc}{mean}{input, labels=None, index=None}
  The \function{mean} function calculates the mean of the elements of the
  object with label(s) given by \var{index}, using the \var{labels} array 
  for the object labels. If \var{index} is \constant{None}, all elements 
  with a non-zero label value are treated as a single object. If 
  \var{label} is \constant{None}, all elements of \var{input} are used in 
  the calculation.
\end{funcdesc}

\begin{funcdesc}{variance}{input, labels=None, index=None}
  The \function{variance} function calculates the variance of the elements 
  of the object with label(s) given by \var{index}, using the \var{labels} 
  array for the object labels. If \var{index} is \constant{None}, all 
  elements with a non-zero label value are treated as a single object. If 
  \var{label} is \constant{None}, all elements of \var{input} are used in 
  the calculation.
\end{funcdesc}

\begin{funcdesc}{standard_deviation}{input, labels=None, index=None}
  The \function{standard_deviation} function calculates the standard 
  deviation of the elements of the object with label(s) given by 
  \var{index}, using the \var{labels} array for the object labels. If 
  \var{index} is \constant{None}, all elements with a non-zero label value 
  are treated as a single object. If \var{label} is \constant{None}, all 
  elements of \var{input} are used in the calculation.
\end{funcdesc}

\begin{funcdesc}{minimum}{input, labels=None, index=None}
  The \function{minimum} function calculates the minimum of the elements of 
  the object with label(s) given by \var{index}, using the \var{labels} 
  array for the object labels. If \var{index} is \constant{None}, all 
  elements with a non-zero label value are treated as a single object. If 
  \var{label} is \constant{None}, all elements of \var{input} are used in 
  the calculation.
\end{funcdesc}

\begin{funcdesc}{maximum}{input, labels=None, index=None}
  The \function{maximum} function calculates the maximum of the elements of 
  the object with label(s) given by \var{index}, using the \var{labels} 
  array for the object labels. If \var{index} is \constant{None}, all 
  elements with a non-zero label value are treated as a single object. If 
  \var{label} is \constant{None}, all elements of \var{input} are used in 
  the calculation.
\end{funcdesc}

\begin{funcdesc}{minimum_position}{input, labels=None, index=None}
  The \function{minimum_position} function calculates the position of the
  minimum of the elements of the object with label(s) given by \var{index},
  using the \var{labels} array for the object labels. If \var{index} is
  \constant{None}, all elements with a non-zero label value are treated as 
  a single object. If \var{label} is \constant{None}, all elements of 
  \var{input} are used in the calculation.
\end{funcdesc}

\begin{funcdesc}{maximum_position}{input, labels=None, index=None}
  The \function{maximum_position} function calculates the position of the
  maximum of the elements of the object with label(s) given by \var{index},
  using the \var{labels} array for the object labels. If \var{index} is
  \constant{None}, all elements with a non-zero label value are treated as 
  a single object. If \var{label} is \constant{None}, all elements of 
  \var{input} are used in the calculation.
\end{funcdesc}

\begin{funcdesc}{extrema}{input, labels=None, index=None}
  The \function{extrema} function calculates the minimum, the maximum, and 
  their positions, of the elements of the object with label(s) given by 
  \var{index}, using the \var{labels} array for the object labels. If 
  \var{index} is \constant{None}, all elements with a non-zero label value 
  are treated as a single object. If \var{label} is \constant{None}, all 
  elements of \var{input} are used in the calculation. The result is a 
  tuple giving the minimum, the maximum, the position of the mininum and 
  the postition of the maximum. The result is the same as a tuple formed by 
  the results of the functions \function{minimum}, \function{maximum}, 
  \function{minimum_position}, and \function{maximum_position} that are 
  described above.
\end{funcdesc}

\begin{funcdesc}{center_of_mass}{input, labels=None, index=None}
  The \function{center_of_mass} function calculates the center of mass of 
  the of the object with label(s) given by \var{index}, using the 
  \var{labels} array for the object labels. If \var{index} is 
  \constant{None}, all elements with a non-zero label value are treated as 
  a single object. If \var{label} is \constant{None}, all elements of 
  \var{input} are used in the calculation.
\end{funcdesc}

\begin{funcdesc}{histogram}{input, min, max, bins, labels=None, index=None}
  The \function{histogram} function calculates a histogram of 
  the of the object with label(s) given by \var{index}, using the 
  \var{labels} array for the object labels. If \var{index} is 
  \constant{None}, all elements with a non-zero label value are treated as 
  a single object. If \var{label} is \constant{None}, all elements of 
  \var{input} are used in the calculation. Histograms are defined by their 
  minimum (\var{min}), maximum (\var{max}) and the number of bins 
  (\var{bins}). They are returned as one-dimensional arrays of type Int32. 
\end{funcdesc}

\section{Extending \module{nd\_image} in C}
\label{sec:ndimage:ccallbacks}
\subsection{C callback functions}
A few functions in the \module{numarray.nd\_image} take a call-back 
argument. This can be a python function, but also a CObject containing a 
pointer to a C function. To use this feature, you must write your own C 
extension that defines the function, and define a python function that 
returns a CObject containing a pointer to this function.

An example of a function that supports this is 
\function{geometric_transform} (see section \ref{sec:ndimage:
interpolation}). You can pass it a python callable object that defines a 
mapping from all output coordinates to corresponding coordinates in the 
input array. This mapping function can also be a C function, which 
generally will be much more efficient, since the overhead of calling a
python function at each element is avoided.

For example to implement a simple shift function we define the following 
function:
\begin{verbatim}
static int 
_shift_function(int *output_coordinates, double* input_coordinates,
                int output_rank, int input_rank, void *callback_data)
{
  int ii;
  /* get the shift from the callback data pointer: */
  double shift = *(double*)callback_data;
  /* calculate the coordinates: */
  for(ii = 0; ii < irank; ii++)
    icoor[ii] = ocoor[ii] - shift;
  /* return OK status: */
  return 1;
}
\end{verbatim}
This function is called at every element of the output array, passing the 
current coordinates in the \var{output_coordinates} array. On return, the 
\var{input_coordinates} array must contain the coordinates at which the 
input is interpolated. The ranks of the input and output array are passed 
through \var{output_rank} and \var{input_rank}. The value of the shift is 
passed through the \var{callback_data} argument, which is a pointer to 
void. The function returns an error status, in this case always 1, since no 
error can occur.

A pointer to this function and a pointer to the shift value must be passed 
to \function{geometric_transform}. Both are passed by a single CObject 
which is created by the following python extension function:
\begin{verbatim}
static PyObject *
py_shift_function(PyObject *obj, PyObject *args)
{
  double shift = 0.0;
  if (!PyArg_ParseTuple(args, "d", &shift)) {
    PyErr_SetString(PyExc_RuntimeError, "invalid parameters");
    return NULL;
  } else {
    /* assign the shift to a dynamically allocated location: */
    double *cdata = (double*)malloc(sizeof(double));
    *cdata = shift;
    /* wrap function and callback_data in a CObject: */
    return PyCObject_FromVoidPtrAndDesc(_shift_function, cdata,
                                        _destructor);
  }
}
\end{verbatim}
The value of the shift is obtained and then assigned to a dynamically 
allocated memory location. Both this data pointer and the function pointer 
are then wrapped in a CObject, which is returned. Additionally, a pointer 
to a destructor function is given, that will free the memory we allocated 
for the shift value when the CObject is destroyed. This destructor is very 
simple:
\begin{verbatim}
static void
_destructor(void* cobject, void *cdata)
{
  if (cdata)
    free(cdata);
}
\end{verbatim}
To use these functions, an extension module is build:
\begin{verbatim}
static PyMethodDef methods[] = {
  {"shift_function", (PyCFunction)py_shift_function, METH_VARARGS, ""},
  {NULL, NULL, 0, NULL}
};

void
initexample(void)
{
  Py_InitModule("example", methods);
}
\end{verbatim}
This extension can then be used in Python, for example:
\begin{verbatim}
>>> import example
>>> array = arange(12, shape=(4,3), type = Float64)
>>> fnc = example.shift_function(0.5)
>>> print geometric_transform(array, fnc)
[[ 0.      0.      0.    ]
 [ 0.      1.3625  2.7375]
 [ 0.      4.8125  6.1875]
 [ 0.      8.2625  9.6375]]
\end{verbatim}

C Callback functions for use with \module{nd\_image} functions must all be 
written according to this scheme. The next section lists the 
\module{nd\_image} functions that acccept a C callback function and gives 
the prototype of the callback function.

\subsection{Functions that support C callback functions}
The \module{nd\_image} functions that support C callback functions are 
described here. Obviously, the prototype of the function that is provided 
to these functions must match exactly that what they expect. Therefore we 
give here the prototypes of the callback functions. All these callback 
functions accept a void \var{callback_data} pointer that must be wrapped in 
a CObject using the Python \cfunction{PyCObject_FromVoidPtrAndDesc} 
function, which can also accept a pointer to a destructor function to free 
any memory allocated for \var{callback_data}. If \var{callback_data} is not 
needed, \cfunction{PyCObject_FromVoidPtr} may be used instead. The callback 
functions must return an integer error status that is equal to zero if 
something went wrong, or 1 otherwise. If an error occurs, you should 
normally set the python error status with an informative message before 
returning, otherwise, a default error message is set by the calling 
function.

The function \function{generic_filter} (see section 
\ref{sec:ndimage:genericfilters}) accepts a callback function with the 
following prototype:
\begin{cfuncdesc}{int}{FilterFunction}{double *buffer, int filter_size,
double *return_value, void *callback_data} The calling function iterates 
over the elements of the input and output arrays, calling the callback 
function at each element. The elements within the footprint of the filter 
at the current element are passed through the \var{buffer} parameter, and 
the number of elements within the footprint through \var{filter_size}. The 
calculated valued should be returned in the \var{return_value} argument.
\end{cfuncdesc}

The function \function{generic_filter1d} (see section 
\ref{sec:ndimage:genericfilters}) accepts a callback function with the 
following prototype: 
\begin{cfuncdesc}{int}{FilterFunction1D}{double *input_line, int 
input_length, double *output_line, int output_length, void *callback_data} 
The calling function iterates over the lines of the input and output 
arrays, calling the callback function at each line. The current line is 
extended according to the border conditions set by the calling function, 
and the result is copied into the array that is passed through the 
\var{input_line} array. The length of the input line (after extension) is 
passed through \var{input_length}. The callback function should apply the 
1D filter and store the result in the array passed through 
\var{output_line}. The length of the output line is passed through 
\var{output_length}.
\end{cfuncdesc}

The function \function{geometric_transform} (see section 
\ref{sec:ndimage:interpolation}) expects a function with the following 
prototype: 
\begin{cfuncdesc}{int}{MapCoordinates}{int *output_coordinates, 
double* input_coordinates, int output_rank, int input_rank, 
void *callback_data} The calling function iterates over the elements of the 
output array, calling the callback function at each element. The 
coordinates of the current output element are passed through 
\var{output_coordinates}. The callback function must return the coordinates 
at which the input must be interpolated in \var{input_coordinates}. The 
rank of the input and output arrays are given by \var{input_rank} and 
\var{output_rank} respectively.
\end{cfuncdesc}


\chapter{Memory Mapping}
\label{cha:memmap}
\declaremodule{extension}{numarray.memmap}
\index{character array}
\index{string array}

\section{Introduction}
\label{sec:memmap-intro}

\code{numarray} provides support for the creation of arrays which are
mapped directly onto files with the \code{numarray.memmap} module.
Much of \code{numarray}'s design, the ability to handle misaligned and
byteswapped arrays for instance, was motivated by the desire to create
arrays from portable files which contain binary array data.  One
advantage of memory mapping is efficient random access to small
regions of a large file: only the region of the mapped file which is
actually used in array operations needs to be paged into system
memory; the rest of the file remains unread and unwritten.

\code{numarray.memmap} is pure Python and is layered on top of
Python's \code{mmap} module.  The basic idea behind \code{numarray}'s
memory mapping is to create a ``buffer'' referring to a region in a
mapped file and to use it as the data store for an array.  The
\code{numarray.memmap} module contains two classes, one which
corresponds to an entire mapped file (\class{Memmap}) and one which
corresponds to a contiguous region within a file
(\class{MemmapSlice}).  \class{MemmapSlice} objects have these
properties:

\begin{itemize}
\item MemmapSlices can be used as NumArray buffers.
\item MemmapSlices are non-overlapping.
\item MemmapSlices are resizable.
\item Changing the size of a MemmapSlice changes the parent Memmap.
\end{itemize}

\section{Opening a Memmap}
\label{sec:memmap-open}

You can create a \class{Memmap} object by calling the \function{open} function,
as in:

\begin{verbatim}
>>> m = open("memmap.tst","w+",len=48)
>>> m
<Memmap on file 'memmap.tst' with mode='w+', length=48, 0 slices>
\end{verbatim}

Here, the file ``memmap.tst'' is created/truncated to a length of 48
bytes and used to construct a Memmap object in write mode whose
contents are considered undefined.  

\section{Slicing a Memmap}
\label{sec:memmap-slicing}

Once opened, a \class{Memmap} object can be sliced into regions.

\begin{verbatim}
# Slice m into the buffers "n" and "p" which will correspond to numarray:

>>> n = m[0:16]
>>> n
<MemmapSlice of length:16 writable>

>>> p = m[24:48]
>>> p
<MemmapSlice of length:24 writable>
\end{verbatim}

NOTE: You cannot make \emph{overlapping} slices of a Memmap:

\begin{verbatim}
>>> q = m[20:28]
Traceback (most recent call last):
...
IndexError: Slice overlaps prior slice of same file.
\end{verbatim}

Deletion of a slice is possible once all other references to it are
forgotten, e.g. all arrays that used it have themselves been deleted.
Deletion of a slice of a Memmap "un-registers" the slice, making that
region of the Memmap available for reallocation.  Delete directly from
the Memmap without referring to the MemmapSlice:

\begin{verbatim}
>>> m = Memmap("memmap.tst",mode="w+",len=100)
>>> m1 = m[0:50]
>>> del m[0:50]      # note: delete from m, not m1
>>> m2 = m[0:70]
\end{verbatim}

Note that since the region of m1 was deleted, there is no overlap when
m2 is created.  However, deleting the region of m1 has invalidated it:

\begin{verbatim}
>>> m1
Traceback (most recent call last):
...
RuntimeError: A deleted MemmapSlice has been used.
\end{verbatim}

Don't mix operations on a Memmap which modify its data or file
structure with slice deletions.  In this case, the status of the
modifications is undefined; the underlying map may or may not reflect
the modifications after the deletion.

\section{Creating an array from a MemmapSlice}
\label{sec:memmap-array-construction}

Arrays are created from \class{MemmapSlice}s simply by specifying the
slice as the \var{buffer} parameter of the array.  Since the slice is
essentially just a byte string, it's necessary to specify the
\var{type} of the binary data as well.

\begin{verbatim}
>>> a = num.NumArray(buffer=n, shape=(len(n)/4,), type=num.Int32)
>>> a[:] = 0  # Since the initial contents of 'n' are undefined.
>>> a += 1
array([1, 1, 1, 1], type=Int32)
\end{verbatim}

\section{Resizing a MemmapSlice}
\label{sec:memmap-slice}

Arrays based on \class{MemmapSlice} objects are resizable.  As soon as
they're resized, slices become un-mapped or ``free floating''.
Resizing a slice affects the parent \class{Memmap}.

\begin{verbatim}
>>> a.resize(6)
array([1, 1, 1, 1, 1, 1], type=Int32)
\end{verbatim}

\section{Forcing file updates and closing the Memmap}
\label{sec:memmap-flushing-closing}

After doing slice resizes or inserting new slices, call
\function{flush} to synchronize the underlying map file with any free
floating slices.  This explicit step is required to avoid implicitly
shuffling huge amounts of file space for every \function{resize} or
\function{insert}.  After calling \function{flush}, all slices are
once again memory mapped rather than free floating.

\begin{verbatim}
>>> m.flush()
\end{verbatim}

A related concept is ``syncing'' which applies even to arrays which
have not been resized.  Since memory maps don't guarantee when the
underlying file will be updated with the values you have written to
the map, call \function{sync} when you want to be sure your changes
are on disk.  This is similar to syncing a UNIX file system.  Note
that \function{sync} does not consolidate the mapfile with any free
floating slices (newly inserted or resized), it merely ensures that
mapped slices whose contents have been altered are written to disk.

\begin{verbatim}
>>> m.sync()
\end{verbatim}

Now "a" and "b" are both memory mapped on "memmap.tst" again.

When you're done with the memory map and numarray, call
\function{close}. \function{close} calls \function{flush} which will
consolidate resized or inserted slices as necessary.

\begin{verbatim}
>>> m.close()
\end{verbatim}

It is an error to use "m" (or slices of m) any further after closing
it.

\section{numarray.memmap functions}
\label{sec:memmap-functions}

\begin{funcdesc}{open}{filename, mode='r+', len=None}
\label{func:memmap-open}
\function{open} opens a \class{Memmap} object on the file
\var{filename} with the specified \var{mode}.  Available \var{mode}
values include 'readonly' ('r'), 'copyonwrite' ('c'), 'readwrite'
('r+'), and 'write' ('w+'), all but the last of which have contents
defined by the file.

Neither mode 'r' nor mode 'c' can affect the underlying map file.
Readonly maps impose no requirement on system swap space and raise
exceptions when their contents are modified.  Copy-on-write maps
require system swap space corresponding to their size, but have
modifiable pages which become reassociated with system swap as they
are changed leaving the original map file unaltered.  Insufficient
swap space can prevent the creation of a copy-on-write memory map.
Modifications to readwrite memory maps are eventually reflected onto
the map file;  see flushing and syncing.
\end{funcdesc}
   
\begin{funcdesc}{close}{map}
\label{func:memmap-close}
\function{close} closes the \class{Memmap} object specified by
\var{map}.
\end{funcdesc}
   
\section{Memmap methods}
\label{sec:memmap-methods}
A Memmap object represents an entire mapped file and is sliced to
create objects which can be used as array buffers.  It has these
public methods:

\begin{methoddesc}[Memmap]{close}{}
  \function{close} unites the \class{Memmap} and any RAM based slices with
  its underlying file and removes the mapping and all references to
  its slices.  Once a \class{Memmap} has been closed, all of its
  slices become unusable.
\end{methoddesc}

\begin{methoddesc}[Memmap]{find}{string, offset=0}
  find(string, offset=0) returns the first index at which string
  is found, or -1 on failure.
  \begin{verbatim}
    >>> _open("memmap.tst","w+").write("this is a test")
    >>> Memmap("memmap.tst",len=14).find("is")
    2
    >>> Memmap("memmap.tst",len=14).find("is", 3)
    5
    >>> _open("memmap.tst","w+").write("x")
    >>> Memmap("memmap.tst",len=1).find("is")
    -1
  \end{verbatim}
\end{methoddesc}

\begin{methoddesc}[Memmap]{insert}{offset, size=None, buffer=None}
  \function{insert} places a new slice at the specified \var{offset} of
  the \class{Memmap}.  \var{size} indicates the length in bytes of the
  inserted slice when \var{buffer} is not specified.  If \function{buffer}
  is specified, it should refer to an existing memory object created
  using \code{numarray.memory.new_memory} and \function{size} should not
  be specified.
\end{methoddesc}

\begin{methoddesc}[Memmap]{flush}{}
  \function{flush} writes a \class{Memmap} out to its associated file,
  reconciling any inserted or resized slices by backing them directly
  on the map file rather than a system swap file.  \function{flush}
  only makes sense for write and readwrite memory maps.
\end{methoddesc}

\begin{methoddesc}[Memmap]{sync}{}
  \function{sync} forces slices which are backed on the map file to be
  immediately written to disk.  Resized or newly inserted slices are
  not affected.  \function{sync} only makes sense for write and
  readwrite memory maps.
\end{methoddesc}

\section{MemmapSlice methods}
\label{sec:memmap-methods}
A \class{MemmapSlice} object represents a subregion of a
\class{Memmap} and has these public methods:

\begin{methoddesc}[MemmapSlice]{__buffer__}{}
  Returns an object which supports the Python buffer protocol and
  represents this slice.  The Python buffer protocol enables a C
  function to obtain the pointer and size corresponding to the data
  region of the slice.
\end{methoddesc}

\begin{methoddesc}[MemmapSlice]{resize}{newsize}
  \function{resize} expands or contracts this slice to the specified
  \var{newsize}.
\end{methoddesc}

%% mode: LaTeX
%% mode: auto-fill
%% fill-column: 79
%% indent-tabs-mode: nil
%% ispell-dictionary: "american"
%% reftex-fref-is-default: nil
%% TeX-auto-save: t
%% TeX-command-default: "pdfeLaTeX"
%% TeX-master: "numarray"
%% TeX-parse-self: t
%% End:


\appendix
\part*{Appendix}
%begin{latexonly}
\makeatletter
\py@reset
\makeatother
%end{latexonly}

\chapter{Glossary}
\label{cha:glossary}

\begin{quote} 
   This chapter provides a glossary of terms.\footnote{Please let us know of
      any additions to this list which you feel would be helpful.}
\end{quote}

\begin{description}
\item[array] An array refers to the Python object type defined by the NumPy
   extensions to store and manipulate numbers efficiently.
\item[byteswapped]
\item[discontiguous]  
\item[misaligned] 
\item[misbehaved array] A \class{\numarray} which is byteswapped, misaligned,
   or discontiguous.
\item[rank] The rank of an array is the number of dimensions it has, or the
   number of integers in its shape tuple.
\item[shape] Array objects have an attribute called shape which is necessarily
   a tuple. An array with an empty tuple shape is treated like a scalar (it
   holds one element).
\item[ufunc] A callable object which performs operations on all of the elements
   of its arguments, which can be lists, tuples, or arrays. Many ufuncs are
   defined in the umath module.
\item[universal function] See ufunc.
\end{description}
 


%% Local Variables:
%% mode: LaTeX
%% mode: auto-fill
%% fill-column: 79
%% indent-tabs-mode: nil
%% ispell-dictionary: "american"
%% reftex-fref-is-default: nil
%% TeX-auto-save: t
%% TeX-command-default: "pdfeLaTeX"
%% TeX-master: "numarray"
%% TeX-parse-self: t
%% End:

% Complete documentation on the extended LaTeX markup used for Python
% documentation is available in ``Documenting Python'', which is part
% of the standard documentation for Python.  It may be found online
% at:
%
%     http://www.python.org/doc/current/doc/doc.html

\documentclass[hyperref]{manual}
\pagestyle{plain}

% latex2html doesn't know [T1]{fontenc}, so we cannot use that:(

\usepackage{amsmath}
\usepackage[latin1]{inputenc}
\usepackage{textcomp}


% The commands of this document do not reset module names at section level
% (nor at chapter level).
% --> You have to do that manually when a new module starts!
%     (use \py@reset)
%begin{latexonly}
\makeatletter
\renewcommand{\section}{\@startsection{section}{1}{\z@}%
   {-3.5ex \@plus -1ex \@minus -.2ex}%
   {2.3ex \@plus.2ex}%
   {\reset@font\Large\py@HeaderFamily}}
\makeatother
%end{latexonly}


% additional mathematical functions
\DeclareMathOperator{\abs}{abs}

% provide a cross-linking command for the index
%begin{latexonly}
\newcommand*\see[2]{\protect\seename #1}
\newcommand*{\seename}{$\to$}
%end{latexonly}


% some convenience declarations
\newcommand{\numarray}{numarray}
\newcommand{\Numarray}{Numarray}  % Only beginning of sentence, otherwise use \numarray
\newcommand{\NUMARRAY}{NumArray}
\newcommand{\numpy}{Numeric}
\newcommand{\NUMPY}{Numerical Python}
\newcommand{\python}{Python}


% mark internal comments
% for any published version switch to the second (empty) definition of the macro!
% \newcommand{\remark}[1]{(\textbf{Note to authors: #1})}
\newcommand{\remark}[1]{}


\title{numarray\\User's Manual}

\author{Perry Greenfield \\
   Todd Miller \\
   Rick White \\
   J.C. Hsu \\
   Paul Barrett \\
   Jochen K�pper \\
   Peter J. Verveer \\[1ex]
   Previously authored by: \\
   David Ascher \\
   Paul F. Dubois \\
   Konrad Hinsen \\
   Jim Hugunin \\
   Travis Oliphant \\[1ex]
   with contributions from the Numerical Python community}

\authoraddress{Space Telescope Science Institute, 3700 San Martin Dr,
   Baltimore, MD 21218 \\ UCRL-MA-128569}

% I use date to indicate the manual-updates,
% release below gives the matching software version.
\date{November 2, 2005}        % update before release!
                                % Use an explicit date so that reformatting
                                % doesn't cause a new date to be used.  Setting
                                % the date to \today can be used during draft
                                % stages to make it easier to handle versions.

\release{1.5}                 % (software) release version;
\setshortversion{1.5}         % this is used to define the \version macro

\makeindex                      % tell \index to actually write the .idx file



\begin{document}

\maketitle

% This makes the contents more accessible from the front page of the HTML.
\ifhtml
\part*{General}
\chapter*{Front Matter}
\label{front}
\fi

\input{copyright}  \cleardoublepage


\tableofcontents


\part{Numerical Python}

\NUMARRAY{} (``\numarray{}'') adds a fast multidimensional array facility to
Python.  This part contains all you need to know about ``\numarray{}'' arrays
and the functions that operate upon them.

\label{part:numerical-python}

\declaremodule{extension}{numarray}
\moduleauthor{The numarray team}{numpy-discussion@lists.sourceforge.net}
\modulesynopsis{Numerics}

\input{introduction}
\input{installation}
\input{overview}
\input{arraybasics}
\input{arrayindexing}
\input{intermediate}
\input{ufuncs}
\input{arrayfunctions}
\input{arraymethods}
\input{arrayattributes}
\input{chararray}
\input{recordarray}
\input{objectarray}
\input{extending}



\part{Extension modules}

\label{part:numarray-extensions}

\input{convolve}
\input{fft}
\input{linearalgebra}
\input{ma}
\input{mlab}
\input{randomarray}
\input{ndimage}
\input{memmap}

\appendix
\part*{Appendix}
%begin{latexonly}
\makeatletter
\py@reset
\makeatother
%end{latexonly}

\input{glossary}
\input{numarray.ind}

\end{document}


%% Local Variables:
%% mode: LaTeX
%% mode: auto-fill
%% fill-column: 79
%% indent-tabs-mode: nil
%% ispell-dictionary: "american"
%% reftex-fref-is-default: nil
%% TeX-auto-save: t
%% TeX-parse-self: t
%% End:


\end{document}


%% Local Variables:
%% mode: LaTeX
%% mode: auto-fill
%% fill-column: 79
%% indent-tabs-mode: nil
%% ispell-dictionary: "american"
%% reftex-fref-is-default: nil
%% TeX-auto-save: t
%% TeX-parse-self: t
%% End:


\end{document}


%% Local Variables:
%% mode: LaTeX
%% mode: auto-fill
%% fill-column: 79
%% indent-tabs-mode: nil
%% ispell-dictionary: "american"
%% reftex-fref-is-default: nil
%% TeX-auto-save: t
%% TeX-parse-self: t
%% End:


\end{document}


%% Local Variables:
%% mode: LaTeX
%% mode: auto-fill
%% fill-column: 79
%% indent-tabs-mode: nil
%% ispell-dictionary: "american"
%% reftex-fref-is-default: nil
%% TeX-auto-save: t
%% TeX-parse-self: t
%% End:
